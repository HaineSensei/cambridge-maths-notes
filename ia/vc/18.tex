\subsection{Introduction}
Harmonic functions are solutions to Laplace's equation,
\[
	\laplacian{\phi} = 0
\]
\begin{proposition}
	If \(\phi\) is harmonic on \(\Omega \subset \mathbb R^3\), then
	\begin{equation}
		\phi(\vb a) = \frac{1}{4 \pi r^2} \int_{\abs{\vb x - \vb a} = r} \phi(\vb x) \dd{S}\tag{\(\ast\)}
	\end{equation}
	for \(\vb a \in \Omega\), and \(r\) sufficiently small such that all \(\vb x\) are in \(\Omega\).
\end{proposition}
\noindent This is known as the `mean value' property; it essentially shows that the value of \(\phi\) at any given point \(\vb a\) is the average of \(\phi\) on the surface of any ball around \(\vb a\).
\begin{proof}
	Let \(F(r)\) denote the right hand side of \((\ast)\), \(\frac{1}{4 \pi r^2} \int_{\abs{\vb x - \vb a} = r} \phi(\vb x) \dd{S}\).
	Then,
	\[
		F(r) = \frac{1}{4\pi r^2} \int_{\abs{\vb x} = r} \phi(\vb a + \vb x) \dd{S}
	\]
	We can parametrise this sphere using spherical polar coordinates, giving
	\begin{align}
		F(r) & = \frac{1}{4\pi r^2} \int_{\phi = 0}^{2 \pi} \left[ \int_{\theta = 0}^\pi \phi(\vb a + r\vb e_r) r^2 \sin \theta \dd{\theta} \right] \dd{\phi} \notag    \\
		     & = \frac{1}{4\pi} \int_{\phi = 0}^{2 \pi} \left[ \int_{\theta = 0}^\pi \phi(\vb a + r\vb e_r) \sin \theta \dd{\theta} \right] \dd{\phi} \tag{\(\dagger\)}
	\end{align}
	Differentiating with respect to \(r\), using \(\dv{r}\phi(\vb a + r \vb e_r) = \vb e_r \cdot \grad\phi(\vb a + r \vb e_r)\),
	\begin{align*}
		F'(r) & = \frac{1}{4\pi} \int_{\phi = 0}^{2 \pi} \int_{\theta = 0}^\pi \vb e_r \cdot \grad\phi(\vb a + r\vb e_r) \sin \theta \dd{\theta} \dd{\phi}         \\
		      & = \frac{1}{4\pi r^2} \int_{\phi = 0}^{2 \pi} \int_{\theta = 0}^\pi \vb e_r \cdot \grad\phi(\vb a + r\vb e_r) r^2 \sin \theta \dd{\theta} \dd{\phi} \\
		      & = \frac{1}{4\pi r^2} \int_{\abs{\vb x} = r} \vb e_r \cdot \grad\phi(\vb a + r\vb e_r) \dd{S}                                                       \\
		      & = \frac{1}{4\pi r^2} \int_{\abs{\vb x} = r} \grad\phi(\vb a + \vb x) \cdot \dd{\vb S}                                                              \\
		      & = \frac{1}{4\pi r^2} \int_{\abs{\vb x - \vb a} = r} \grad\phi \cdot \dd{\vb S}                                                                     \\
		      & = \frac{1}{4\pi r^2} \int_{\abs{\vb x - \vb a} < r} \div{\grad\phi} \dd{V}                                                                         \\
		      & = \frac{1}{4\pi r^2} \int_{\abs{\vb x - \vb a} < r} \laplacian{\phi} \dd{V}                                                                        \\
		      & = \frac{1}{4\pi r^2} \int_{\abs{\vb x - \vb a} < r} 0 \dd{V}                                                                                       \\
		      & = 0
	\end{align*}
	Now, note from \((\dagger)\) that if \(r \to 0\), then \(F(r) \to \phi(\vb a)\), and the result follows.
\end{proof}

\subsection{Intuitive Explanation of Laplacian}
We can use the central idea of the above proof to examine what the Laplacian operator is really doing.
\begin{proposition}
	For any smooth function \(\phi \colon \mathbb R^3 \to \mathbb R\),
	\[
		\laplacian\phi(\vb a) = \lim_{r \to 0} \frac{6}{r^2} \left[ \frac{1}{4\pi r^2} \int_{\abs{\vb x - \vb a} = r} \phi(\vb x) \dd{S} \right] - \phi(\vb a)
	\]
	In particular, if \(\phi\) satisfies the mean value property, then it is harmonic.
\end{proposition}
\noindent In some sense, the Laplacian is measuring how the value of \(\phi\) at a point differs from its average over a small sphere centred at this point.
\begin{proof}
	Consider a function \(G(r)\) defined by
	\[
		G(r) = \frac{1}{4\pi r^2} \int_{\abs{\vb x - \vb a} = r} \phi(\vb x) \dd{S} - \phi(\vb a)
	\]
	\(G\) measures the extent to which \(\phi\) differs from its average.
	From the previous proof,
	\[
		G(r) = F(a) - \phi(\vb a) \implies G'(r) = F'(r)
	\]
	So,
	\[
		G'(r) = \frac{1}{4\pi r^2} \int_{\abs{\vb x - \vb a} < r} \laplacian{\phi} \dd{V}
	\]
	Now, note that as \(r \to 0\),
	\begin{align*}
		\int_{\abs{\vb x - \vb a} < r} \laplacian \phi(\vb x) \dd{V} & = \laplacian\phi(\vb a) \int_{\abs{\vb x - \vb a} < r} \dd{V} + \int_{\abs{\vb x - \vb a} < r} \left( \laplacian\phi(\vb x) - \laplacian\phi(\vb a) \right) \dd{V} \\
		                                                             & = \frac{4\pi}{3r^3} \laplacian\phi(\vb a) + o(r^3)
	\end{align*}
	Now, as \(r \to 0\),
	\[
		G'(r) = \frac{1}{4\pi r^2} \left[ \frac{4\pi}{3r^3} \laplacian\phi(\vb a) + o(r^3) \right] = \frac{r}{3}\laplacian\phi(\vb a) + o(r)
	\]
	Comparing this to the Taylor expansion,
	\[
		G'(r) = G'(0) + rG''(0) + o(r)
	\]
	So certainly, \(G'(0) = 0\) since there is no constant term in \(G'(r)\).
	Further, \(G''(0) = \frac{1}{3}\laplacian\phi(\vb a)\).
	Now,
	\[
		G(r) = G(0) + rG'(0) + \frac{r^2}{2}G''(0) + o(r^2)
	\]
	We know that \(G(0) = F(0) - \phi(\vb a) = 0\), hence
	\[
		G(r) = \frac{1}{6}\laplacian\phi(\vb a) r^2 + o(r^2) \implies \laplacian\phi(\vb a) = \lim_{r \to 0} \frac{6}{r^2}G(r)
	\]
	which gives the result as required.
\end{proof}

\subsection{Non-Existence of Maximum Points}
\begin{proposition}
	If \(\phi\) is harmonic on some volume \(\Omega \subset \mathbb R^3\), then \(\phi\) cannot have a maximum point at any interior point on \(\Omega\), unless \(\phi\) is constant.
\end{proposition}
\begin{proof}
	Suppose that there exists a maximum point at \(\vb a \in \Omega\).
	Then \(\phi(\vb a) \geq \phi(\vb x)\) for all \(\vb x \in \Omega\).
	Then,
	\[
		\phi(\vb a) \geq \phi(\vb x) \text{ on } \abs{\vb x - \vb a} \leq \varepsilon
	\]
	for some \(\varepsilon\) small enough such that the ball is inside \(\Omega\).
	By the mean value property,
	\[
		\phi(\vb a) = \frac{1}{4\pi \varepsilon^2} \int_{\abs{\vb x - \vb a} = \varepsilon} \phi(\vb x) \dd{S}
	\]
	Hence,
	\[
		0 = \frac{1}{4\pi \varepsilon^2} \int_{\abs{\vb x - \vb a} = \varepsilon} \left( \phi(\vb a) - \phi(\vb x) \right) \dd{S}
	\]
	Note that the integrand is always non-negative, so in order for the integral to equal zero, the integrand must be zero everywhere on the ball.
	So \(\phi(\vb a) = \phi(\vb x)\).
	Since \(\varepsilon\) was arbitrary, we can shrink the ball to a smaller ball around the same point, so \(\phi(\vb a) = \phi(\vb x)\) for all \(\vb x\) such that \(\abs{\vb x - \vb a} \leq \varepsilon\).
	Hence, \(\phi\) is locally constant.
	
	Now, given any other point \(\vb y\), we can introduce a finite sequence of overlapping balls such that the centre of the \((n+1)\)th ball is contained inside the \(n\)th ball, and where the first ball is centred at \(\vb a\) and the last ball is centred at \(\vb y\).
	Inductively, the function is constant on each such ball.
	Hence \(\phi\) is actually constant everywhere, since \(\vb y\) was arbitrarily chosen.
\end{proof}
\begin{corollary}
	If \(\phi\) is harmonic on \(\Omega\), then for \(\vb x \in \Omega\),
	\[
		\phi(\vb x) \leq \max_{\vb y \in \partial \Omega} \phi(\vb y)
	\]
	This is called the maximum principle.
\end{corollary}
