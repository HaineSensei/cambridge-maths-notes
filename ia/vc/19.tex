Throughout this section on tensors, we deal exclusively with Cartesian coordinate systems. This lecture stands as an intuitive introduction to tensors, without any definitions or proofs.
\subsection{Vectors and Changes of Basis}
Consider a right-handed orthonormal basis \(\{ \vb e_i \}\) for \(\mathbb R^3\), with respect to some fixed Cartesian coordinate axes. We can write a vector using this basis as
\[ \vb x = x_i \vb e_i \]
Note that the vector \(\vb x\) and the components \(x_i\) are not the same; the components only give the vector when in combination with the given basis vectors \(\{ \vb e_i \}\). If we instead use \(\{ \vb e'_i \}\), then the same position vector \(\vb x\) would be written as a linear combination \(x'_i \vb e'_i\). Hence,
\begin{equation}
	x_j \vb e_j = x'_j \vb e'_j
	\tag{\(\ast\)}
\end{equation}
Since the \(\{ \vb e_j \}\) and \(\{ \vb e'_j \}\) are orthonormal,
\[ \vb e_i \cdot \vb e_j = \delta_{ij};\quad \vb e'_i \cdot \vb e'_j = \delta_{ij} \]
From \((\ast)\),
\[ x_i' = \delta_{ij} x_j' = (\vb e_i' \cdot \vb e_j') x_j' = \vb e_i' \cdot (\vb e_j' x_j') = \vb e_i' \cdot (\vb e_j x_j) = (\vb e_i' \cdot \vb e_j) x_j \]
So let
\[ R_{ij} = \vb e_i' \cdot \vb e_j \]
Then
\[ x_i' = R_{ij} x_j \]
Alternatively,
\[ x_i = \delta_{ij} x_j = (\vb e_i \cdot \vb e_j) x_j = \vb e_i \cdot (\vb e_j x_j) = \vb e_i \cdot (\vb e_j' x_j') = (\vb e_i \cdot \vb e_j') x_j' \]
And therefore, we get
\[ x_i = R_{ji} x_j' = R_{ki} x_k' \implies x_j = R_{kj} x_k' \]
Combining the two results, we have
\[ x_i' R_{ij} x_j = R_{ij} R_{kj} x_k' \]
Therefore,
\[ (\delta_{ik} - R_{ij}R_{kj}) x_k' = 0 \]
Since this is true for all vectors \(\vb x\), we get
\[ R_{ij}R_{kj} = \delta_{ik} \]
So if \(R\) is a matrix with entries \(R_{ij}\), then
\[ R R^\transpose = I \]
So the \(R_{ij}\) are the components of an orthogonal matrix. Further, since
\[ x_j \vb e_j = x_i' \vb e_i' = R_{ij} x_j \vb e_i' \]
holds for all \(x_j\), we also have
\[ \vb e_j = R_{ij} \vb e_i' \]
and since both \(\{ \vb e_i \}\) and \(\{ \vb e_i' \}\) are right handed, we have
\[ 1 = \vb e_1 \cdot (\vb e_2 \times \vb e_3) = R_{i1} R_{j2} R_{k3} \vb e_i' \cdot (\vb e_j' \times \vb e_k') = R_{i1} R_{j2} R_{k3} \varepsilon_{ijk} = \det R \]
Hence \(R\) is orthogonal, and has determinant 1. Hence \(R\) is a rotation matrix. If we transform from a right-handed orthonormal set of basis vectors \(\{ \vb e_i \}\) to another basis \(\{ \vb e_i' \}\), then the components of a vector \(\vb v\) transform according to \(v_i' = R_{ij} v_j\). We call objects whose components transform in this way `rank 1 tensors', or more commonly, `vectors'. The basis vectors themselves transform according to \(\vb e_j' = R_{ij} \vb e_i\).

\subsection{Scalars and Scalar Products}
Consider the dot product between two vectors, \(\sigma = \vb a \cdot \vb b\). This should ideally be independent of the set of basis vectors chosen to describe \(\vb a\) and \(\vb b\). So with a basis \(\{ \vb e_i \}\), we have
\[ \sigma = a_i b_j \delta_{ij} = a_i b_i \]
If instead we use a different set of basis vectors \(\{ \vb e_j \}\), we define
\[ \sigma' = a_i' b_i' \]
We can use \(a_i' = R_{ip} a_p\) and \(b_i' = R_{iq} b_q\) to give
\[ \sigma' = R_{ip} R_{iq} a_p b_q = \delta_{pq} a_p b_q = a_i b_i = \sigma \]
Since the sets of basis vectors are related by \(R\), \(\sigma\) is unchanged under changes of coordinates. We call objects which are invariant under transformations like this `rank 0 tensors', or `scalars'.

\subsection{Linear Maps}
Let \(\vb n \in \mathbb R^3\) be a fixed unit vector, and we define a linear map
\[ T \colon \vb x \to \vb y = T(\vb x) = \vb x - (\vb x \cdot \vb n) \vb n \]
This \(T\) is the orthogonal projection into the plane normal to \(\vb n\). Using a set of basis vectors \(\{ \vb e_i \}\), we get
\[ y_i \vb e_i = T(x_j \vb e_j) = x_j T(\vb e_j) = x_j (\vb e_j - n_i n_j \vb e_i) = (\delta_{ij} - n_i n_j) x_j \vb e_i \]
Hence,
\[ y_i = (\delta_{ij} - n_i n_j) x_j \]
So we will set
\[ T_{ij} = \delta_{ij} - n_i n_j \implies y_i = T_{ij} x_j \]
We call the \(T_{ij}\) the \textit{components} of the linear map \(T\) with respect to the basis vectors \(\vb e_i\). Consider a different set of basis vectors \(\{ \vb e_i' \}\).
\[ y_i' = (\delta_{ij} - n_i' n_j') x_j';\quad T_{ij}' = \delta_{ij} - n_i' n_j' \]
Using \(n_i' = R_{ij} n_j\), noting that \(R\) is orthogonal, we have
\[ T_{ij}' = \delta_{ij} - R_{ip} n_j R_{jq} n_q = R_{ip} R_{jq} (\delta_{pq} - n_p n_q) = R_{ip} R_{jq} T_{pq} \]
So the components of a linear map transform according to two multiplications:
\[ T_{ij}' = R_{ip} R_{jq} T_{pq} \]
We call such objects `rank 2 tensors'.
