\subsection{Definition}
\begin{proposition}
	If \(\vb F(\vb x)\) is a continuously differentiable vector field, and \(S\) is an orientable, piecewise regular surface with a piecewise smooth boundary \(\partial S\), then
	\[
		\int_S (\curl{\vb F}) \cdot \dd \vb S = \oint_{\partial S} \vb F\cdot \dd{\vb x}
	\]
\end{proposition}
\noindent This can be thought of as a generalisation to the fundamental theorem of calculus.
From the fundamental theorem, we know that the integral of a differentiated function over an interval \(I\) is just the original function evaluated at the boundary \(\partial I\).
Likewise, Stokes' theorem states that the integral of the curl of a function (just another differential operator) over a surface \(S\) is just the original function evaluated at the boundary of the surface \(\partial S\).
In the one-dimensional fundamental theorem of calculus, we say that the function `evaluated over the boundary' is simply the function applied to the final point, minus the function applied to the initial point; we are in some sense considering every point on the boundary \(\partial I\).
But in the case of \(\partial S\) being a curve, we must integrate around the curve boundary, since without an integral we can't consider infinitely many boundary points.

Note that for a surface to be `orientable', it simply means that it has two sides, an inside and an outside.
There must be a consistent choice of normal at each point.
For example, a sphere is orientable, but a M\"obius strip is not orientable.

\subsection{First example}
Let \(S\) be a cap of a sphere:
\[
	S = \left\{ \vb x(\theta, \phi) = \begin{pmatrix}
		\sin\theta\cos\phi \\ \sin\theta\sin\phi \\ \cos\theta
	\end{pmatrix} \equiv \vb e_r; 0 \leq \theta \leq \alpha; 0 \leq \phi < 2 \pi \right\}
\]
Now, let
\[
	\vb F(\vb x) = \begin{pmatrix}
		-x^2y \\ 0 \\ 0
	\end{pmatrix} \implies \curl{\vb F} = \begin{pmatrix}
		0 \\ 0 \\ x^2
	\end{pmatrix}
\]
On \(S\),
\[
	\dd {\vb S} = \dv{\vb x}{\theta} \times \dv{\vb x}{\phi} \dd{\theta}\dd{\phi} = \vb e_r \sin\theta \dd{\theta}\dd{\phi}
\]
Note that since
\[
	x^2 \vb e_z \cdot \vb e_r = (\sin\theta\cos\phi)^2 \cos\theta
\]
on \(S\), we can compute
\[
	\int_S (\curl{\vb F}) \cdot \dd{\vb S} = \int_{\phi = 0}^{2 \pi} \left( \int_{\theta = 0}^\alpha (\sin\theta\cos\phi)^2 \cos\theta \sin\theta \dd{\theta} \right) \dd{\phi} = \frac{\pi}{4}\sin^4 \alpha
\]
We can instead compute the integral over the boundary.
By Stokes' theorem, the results should match.
\(\partial S\) is described by
\[
	[0, 2\pi] \ni t \mapsto \begin{pmatrix}
		\sin\alpha\cos t \\ \sin\alpha\sin t \\ \cos\alpha
	\end{pmatrix}
\]
Then
\[
	\dd{\vb x} = \dv{\vb x}{t} \dd{t} = \sin \alpha\begin{pmatrix}
		-\sin t \\ \cos t \\ 0
	\end{pmatrix} \dd{t}
\]
We can show that
\[
	\oint_{\partial S} \vb F \cdot \dd{\vb S} = \sin^4 \alpha \int_0^{2\pi} \left( -\cos^2 t \sin t \right)\left( -\sin t \right) \dd{t} = \frac{\pi}{4}\sin^4 \alpha
\]

\subsection{Stokes' theorem on closed surfaces}
If \(S\) is an orientable, closed surface, and \(\vb F\) is continuously differentiable, then
\[
	\int_S (\curl{\vb F}) \cdot \dd{\vb S} = 0
\]
This is clear since \(\partial S = \varnothing\).

\subsection{Zero circulation and irrotationality}
\begin{proposition}
	If \(\vb F\) is continuously differentiable, and for every loop \(C\) we have that
	\[
		\oint_C \vb F \cdot \dd{\vb x} = 0
	\]
	then \(\curl{\vb F} = \vb 0\).
	In other words, \(\vb F\) is irrotational if and only if \(\vb F\) has zero circulation around all closed loops.
\end{proposition}
\noindent Note that the backward implication is trivial.
If \(\vb F\) has zero circulation around all loops, we can define that loop to be the boundary of some surface, and so the integral of the curl vanishes.
\begin{proof}
	Suppose that the result is false; there exists a unit vector \(\vu{k}\) such that \(\vu{k} \cdot \left( \curl{\vb F(\vb x_0)} \right) = \varepsilon > 0\) for some \(\vb x_0\).
	By continuity, for a sufficiently small \(\delta > 0\),
	\[
		\vu{k} \cdot \left( \curl{\vb F(\vb x_0)} \right) > \frac{1}{2}\varepsilon;\quad \text{for }\abs{\vb x - \vb x_0} < \delta
	\]
	Now, we can take a loop in this ball \(\{ \vb x \colon \abs{\vb x - \vb x_0} < \delta \}\) that lies entirely in a plane with normal \(\vu{k}\).
	Let this small loop's enclosed surface be \(S\), with boundary \(\partial S\).
	Then
	\[
		0 = \oint_{\partial S} \vb F \cdot \dd{\vb x} = \int_S \curl{\vb F} \cdot \vu k \dd{S} > \frac{1}{2}\varepsilon \int \dd{S} > 0
	\]
	which is a contradiction.
\end{proof}

\subsection{Intuition for curl as infinitesimal circulation}
Let \(S_\varepsilon\) denote a region contained inside a disk of radius \(\varepsilon > 0\), centred at \(\vb x_0\) with normal \(\vu{k}\).
\begin{align*}
	\int_{S_\varepsilon} \curl{\vb F} \cdot \dd{\vb S} & = \int_{S_\varepsilon} \left( \curl{\vb F(\vb x)} - \curl{\vb F(\vb x_0)} \right) \cdot \dd{\vb S} + \int_{S_\varepsilon} \curl{\vb F(\vb x_0)} \cdot \vu k \dd{S}                                                                                        \\
	                                                   & = \underbrace{\int_{S_\varepsilon} \left( \curl{\vb F(\vb x)} - \curl{\vb F(\vb x_0)} \right) \cdot \dd{\vb S}}_{o(\text{area}(S_\varepsilon))} + \curl{\vb F(\vb x_0)} \cdot \vu k \underbrace{\int_{S_\varepsilon} \dd{S}}_{\text{area}(S_\varepsilon)} \\
\end{align*}
As \(\varepsilon\) shrinks, the first integral tends to zero faster than the second term.
Hence,
\[
	\int_{S_\varepsilon} \curl{\vb F} \cdot \dd{\vb S} =  \curl{\vb F(\vb x_0)} \cdot \vu k \cdot \text{area}(S_\varepsilon) + o(\text{area}(S_\varepsilon))
\]
We can then see, by Stokes' theorem, that
\[
	\curl{\vb F(\vb x_0)} \cdot \vu k = \lim_{\varepsilon \to 0} \frac{1}{\text{area}(S_\varepsilon)} \oint_{\partial S_\varepsilon} \vb F \cdot \dd{\vb x}
\]
So the curl of \(\vb F\) at \(\vb x_0\) in the direction \(\vu k\) is the infinitesimal circulation around \(\vb x_0\), per unit area.
