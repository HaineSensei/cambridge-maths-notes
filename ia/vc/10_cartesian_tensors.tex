Throughout this section on tensors, we deal exclusively with Cartesian coordinate systems.
\subsection{Intuitive description of vectors and changes of basis}
Consider a right-handed orthonormal basis \(\{ \vb e_i \}\) for \(\mathbb R^3\), with respect to some fixed Cartesian coordinate axes.
We can write a vector using this basis as
\[
	\vb x = x_i \vb e_i
\]
Note that the vector \(\vb x\) and the components \(x_i\) are not the same; the components only give the vector when in combination with the given basis vectors \(\{ \vb e_i \}\).
If we instead use \(\{ \vb e'_i \}\), then the same position vector \(\vb x\) would be written as a linear combination \(x'_i \vb e'_i\).
Hence,
\begin{equation}
	x_j \vb e_j = x'_j \vb e'_j
	\tag{\(\ast\)}
\end{equation}
Since the \(\{ \vb e_j \}\) and \(\{ \vb e'_j \}\) are orthonormal,
\[
	\vb e_i \cdot \vb e_j = \delta_{ij};\quad \vb e'_i \cdot \vb e'_j = \delta_{ij}
\]
From \((\ast)\),
\[
	x_i' = \delta_{ij} x_j' = (\vb e_i' \cdot \vb e_j') x_j' = \vb e_i' \cdot (\vb e_j' x_j') = \vb e_i' \cdot (\vb e_j x_j) = (\vb e_i' \cdot \vb e_j) x_j
\]
So let
\[
	R_{ij} = \vb e_i' \cdot \vb e_j
\]
Then
\[
	x_i' = R_{ij} x_j
\]
Alternatively,
\[
	x_i = \delta_{ij} x_j = (\vb e_i \cdot \vb e_j) x_j = \vb e_i \cdot (\vb e_j x_j) = \vb e_i \cdot (\vb e_j' x_j') = (\vb e_i \cdot \vb e_j') x_j'
\]
And therefore, we get
\[
	x_i = R_{ji} x_j' = R_{ki} x_k' \implies x_j = R_{kj} x_k'
\]
Combining the two results, we have
\[
	x_i' R_{ij} x_j = R_{ij} R_{kj} x_k'
\]
Therefore,
\[
	(\delta_{ik} - R_{ij}R_{kj}) x_k' = 0
\]
Since this is true for all vectors \(\vb x\), we get
\[
	R_{ij}R_{kj} = \delta_{ik}
\]
So if \(R\) is a matrix with entries \(R_{ij}\), then
\[
	R R^\transpose = I
\]
So the \(R_{ij}\) are the components of an orthogonal matrix.
Further, since
\[
	x_j \vb e_j = x_i' \vb e_i' = R_{ij} x_j \vb e_i'
\]
holds for all \(x_j\), we also have
\[
	\vb e_j = R_{ij} \vb e_i'
\]
and since both \(\{ \vb e_i \}\) and \(\{ \vb e_i' \}\) are right handed, we have
\[
	1 = \vb e_1 \cdot (\vb e_2 \times \vb e_3) = R_{i1} R_{j2} R_{k3} \vb e_i' \cdot (\vb e_j' \times \vb e_k') = R_{i1} R_{j2} R_{k3} \varepsilon_{ijk} = \det R
\]
Hence \(R\) is orthogonal, and has determinant 1.
Hence \(R\) is a rotation matrix.
If we transform from a right-handed orthonormal set of basis vectors \(\{ \vb e_i \}\) to another basis \(\{ \vb e_i' \}\), then the components of a vector \(\vb v\) transform according to \(v_i' = R_{ij} v_j\).
We call objects whose components transform in this way `rank 1 tensors', or more commonly, `vectors'.
The basis vectors themselves transform according to \(\vb e_j' = R_{ij} \vb e_i\).

\subsection{Intuitive description of scalars and scalar products}
Consider the dot product between two vectors, \(\sigma = \vb a \cdot \vb b\).
This should ideally be independent of the set of basis vectors chosen to describe \(\vb a\) and \(\vb b\).
So with a basis \(\{ \vb e_i \}\), we have
\[
	\sigma = a_i b_j \delta_{ij} = a_i b_i
\]
If instead we use a different set of basis vectors \(\{ \vb e_j \}\), we define
\[
	\sigma' = a_i' b_i'
\]
We can use \(a_i' = R_{ip} a_p\) and \(b_i' = R_{iq} b_q\) to give
\[
	\sigma' = R_{ip} R_{iq} a_p b_q = \delta_{pq} a_p b_q = a_i b_i = \sigma
\]
Since the sets of basis vectors are related by \(R\), \(\sigma\) is unchanged under changes of coordinates.
We call objects which are invariant under transformations like this `rank 0 tensors', or `scalars'.

\subsection{Intuitive description of linear maps}
Let \(\vb n \in \mathbb R^3\) be a fixed unit vector, and we define a linear map
\[
	T \colon \vb x \to \vb y = T(\vb x) = \vb x - (\vb x \cdot \vb n) \vb n
\]
This \(T\) is the orthogonal projection into the plane normal to \(\vb n\).
Using a set of basis vectors \(\{ \vb e_i \}\), we get
\[
	y_i \vb e_i = T(x_j \vb e_j) = x_j T(\vb e_j) = x_j (\vb e_j - n_i n_j \vb e_i) = (\delta_{ij} - n_i n_j) x_j \vb e_i
\]
Hence,
\[
	y_i = (\delta_{ij} - n_i n_j) x_j
\]
So we will set
\[
	T_{ij} = \delta_{ij} - n_i n_j \implies y_i = T_{ij} x_j
\]
We call the \(T_{ij}\) the \textit{components} of the linear map \(T\) with respect to the basis vectors \(\vb e_i\).
Consider a different set of basis vectors \(\{ \vb e_i' \}\).
\[
	y_i' = (\delta_{ij} - n_i' n_j') x_j';\quad T_{ij}' = \delta_{ij} - n_i' n_j'
\]
Using \(n_i' = R_{ij} n_j\), noting that \(R\) is orthogonal, we have
\[
	T_{ij}' = \delta_{ij} - R_{ip} n_j R_{jq} n_q = R_{ip} R_{jq} (\delta_{pq} - n_p n_q) = R_{ip} R_{jq} T_{pq}
\]
So the components of a linear map transform according to two multiplications:
\[
	T_{ij}' = R_{ip} R_{jq} T_{pq}
\]
We call such objects `rank 2 tensors'.

\subsection{Definition}
\begin{definition}
	An object whose components \(T_{ij\dots k}\) transform according to
	\[
		T_{ij\dots k}' = R_{ip}R_{jq}\dots R_{kr} T_{pq\dots r}
	\]
	is called a (Cartesian) tensor of rank \(n\) if \(T\) has \(n\) indices, where \(R_{ij} = \vb e_i' \cdot \vb e_j\) are the components of an orthogonal matrix, so \(R_{ip} R_{jp} = \delta_{ij}\).
\end{definition}
\noindent For example, if \(u_i, v_j, w_k\) are the components of \(n\) vectors, then
\[
	T_{ij\dots k} = u_i v_j \dots w_k
\]
define the components of a tensor of rank \(n\).
\begin{proof}
	We can transform each vector individually.
	\[
		T_{ij\dots k}' = u_i' v_j' \dots w_k' = R_{ip}u_p R_{jq}v_q \dots R_{kr}w_r = R_{ip}R_{jq}R_{kr} T_{ij\dots k}
	\]
	as expected.
\end{proof}

\subsection{Kronecker delta and Levi-Civita epsilon}
As another example, consider the Kronecker \(\delta\).
It was previously defined without reference to any basis by
\[
	\delta_{ij} = \begin{cases}
		1 & i = j    \\
		0 & i \neq j
	\end{cases}
\]
So \(\delta_{ij}' = \delta_{ij}\) by definition.
Note that
\[
	R_{ip}R_{jq} \delta_{pq} = R_{iq}R_{jq} = \delta_{ij} = \delta_{ij}'
\]
hence \(\delta\) transforms like a rank 2 tensor, so it is indeed a rank 2 tensor.
Now, consider the Levi-Civita symbol \(\varepsilon\).
It is defined without reference to any basis as
\[
	\varepsilon_{ijk} = \begin{cases}
		+1 & (i\ j\ k) \text{ even} \\
		-1 & (i\ j\ k) \text{ odd}  \\
		0  & \text{otherwise}
	\end{cases}
\]
Note that \(\varepsilon_{ijk}' = \varepsilon_{ijk}\), and
\[
	R_{ip}R_{jq}R_{kr} \varepsilon_{pqr} = \det R \cdot \varepsilon_{ijk} = \varepsilon_{ijk}
\]
Hence \(\varepsilon\) is a rank 3 tensor.

\subsection{Electrical conductivity tensor}
Experiments suggest that there is a linear relationship between the current \(\vb J\) produced in a conductive medium and the electric field \(\vb E\) that it is exposed to.
Hence \(\vb J = \sigma \vb E\), or \(J_i = \sigma_{ij} E_j\).
\(\sigma_{ij}\) is called the `electrical conductivity tensor'.
It really is a rank 2 tensor, indeed
\begin{align*}
	J_i'                 & = \sigma'_{ij} E_j' \\
	R_{ip}J_p            & = \sigma'_{ij} E_j' \\
	R_{ip}\sigma_{pq}E_q & = \sigma'_{ij} E_j'
\end{align*}
Since \(R\) is orthogonal,
\[
	E_j' = R_{jq}E_q \iff E_q = R_{jq}E_j'
\]
Hence,
\[
	R_{ip}R_{jq}\sigma_{pq}E_j' = \sigma_{ij}'E_j'
\]
Since this is true for all choices of \(E_j\),
\[
	R_{ip}R_{jq}\sigma_{pq} = \sigma_{ij}'
\]
So it really is a rank 2 tensor.

\subsection{Indexed objects without tensor transformation properties}
It is possible to construct objects with indices that do not transform as tensors.
For example, given a Cartesian right handed basis \(\{ \vb e_i \}\), we can define an arbitrary array of numbers with components \(A_{ij}\), and set \(A_{ij}' = 0\) in all other bases \(\{ \vb e_i' \}\).
Clearly this array of numbers does not transform like a tensor.

\subsection{Operations on tensors}
Let \(A_{ij\dots k}, B_{ij\dots k}\) be rank \(n\) tensors, we define
\[
	(A + B)_{ij \dots k} = A_{ij\dots k} + B_{ij\dots k}
\]
\(A + B\) is also a rank \(n\) tensor, by linearity.
Further,
\[
	(\alpha A)_{ij\dots k} = \alpha A_{ij\dots k}
\]
\(\alpha A\) is also a rank \(n\) tensor.
We also define the \textit{tensor product} between a rank \(m\) tensor \(U_{ij\dots k}\) and a rank \(n\) tensor \(V_{pq \dots r}\) as
\[
	(U \otimes V)_{ij\dots kpq\dots r} = U_{ij\dots k}V_{pq \dots r}
\]
Now, \(U \otimes V\) is a rank \(m+n\) tensor.
Indeed,
\[
	(U \otimes V)_{i \dots j p\dots q} = U_{i\dots j}' V_{p\dots q}' = R_{ia} \dots R_{jb} U_{a \dots b} R_{pc} \dots R_{qd} V_{c \dots d} = R_{ia} \dots R_{jb} R_{pc} \dots R_{qd} (U \otimes V)_{a \dots b c\dots d}
\]
Further, given a rank \(n\geq 2\) tensor \(T_{ijk\dots \ell}\), we can define a tensor of rank \(n-2\) by \textit{contracting} on a pair of indices.
For instance, contracting on \(i\) and \(j\) is defined by
\[
	\delta_{ij} T_{ijk\dots \ell} = T_{iik\dots \ell}
\]
This is really a tensor of rank \(n-2\):
\[
	T'_{iik\dots \ell} = R_{ip}R_{iq}R_{kr}\dots R_{\ell s} T_{pqr\dots s} = \delta_{pq}R_{kr}\dots R_{\ell s} T_{pqr\dots s} = R_{kr}\dots R_{\ell s} T_{ppr\dots s}
\]

\subsection{Symmetric and antisymmetric tensors}
We say that \(T_{ij\dots k}\) is symmetric in \((i, j)\) if
\[
	T_{ij\dots k} = T_{ji\dots k}
\]
This really is a well-defined property of the \textit{tensor}, not its coordinates.
In a different coordinate frame,
\[
	T_{ij\dots k}' = R_{ip}R_{jq}\dots R_{kr} T_{pq\dots r} = R_{ip}R_{jq}\dots R_{kr} T_{qp\dots r} = T_{ji\dots k}'
\]
Similarly, we say that \(A_{ij\dots k}\) is antisymmetric in \((i, j)\) if
\[
	A_{ij\dots k} = -A_{ji\dots k}
\]
which similarly is invariant of the choice of basis.
We say that a tensor is \textit{totally} (anti-) symmetric if it is (anti-) symmetric in all pairs of indices.
For example, the \(\delta_{ij}\) rank 2 tensor and \(a_i a_j a_k\) rank 3 tensor (where \(\vb a\) is a vector) are totally symmetric tensors.
The Levi-Civita alternating tensor \(\varepsilon\) is totally antisymmetric.

In fact, in three dimensions, \(\varepsilon\) is the only totally antisymmetric tensor (up to scaling), and there are no nonzero higher-rank antisymmetric tensors.
Indeed, if \(T_{ij\dots k}\) is totally antisymmetric and has rank \(n\), then \(T_{ij\dots k} = 0\) if any two indices are the same.
But if we have more than three indices, by the pigeonhole principle we must have two matching indices (provided we are working in three dimensions).
If \(n=3\), then there are only \(3!
= 6\) choices of components that give a nonzero value of \(T_{ijk}\), and by antisymmetry, \(T_{123} = T_{231} = T_{312} = \lambda\) and by antisymmetry \(T_{213} = T_{132} = T_{321} = -\lambda\) which defines the \(\varepsilon\) symbol.
