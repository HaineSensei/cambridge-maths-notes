\subsection{Bilinear and Multilinear Maps as Tensors}
For a tensor $T_{ij}$, consider the bilinear map $t \colon \mathbb R^3 \times \mathbb R^3 \to \mathbb R$ defined by
\[ t(\vb a, \vb b) = T_{ij} a_i b_j \]
The left hand side really is well defined, since the right hand side does not depend on the choice of basis vectors. Conversely, suppose we have a bilinear map $t$. Then, for a given basis $\{\vb e_i\}$, this defines an array $T_{ij}$ by
\[ t(\vb a, \vb b) = t(a_i \vb e_i, b_j \vb e_j) = a_i b_j t(\vb e_i, \vb e_j) = a_i b_j T_{ij} \]
Changing basis with $\vb e_i' = R_{ip} \vb e_p$, we find
\[ T_{ij}' = t(\vb e_i', \vb e_j') = t(R_{ip} \vb e_p, R_{jq} \vb e_q) = R_{ip} R_{jq} t(\vb e_p, \vb e_q) \]
hence this $T_{ij}$ really is a rank 2 tensor. So there is a bijection between bilinear maps and rank 2 tensors. In particular, if the map
\[ (\vb a, \vb b) \mapsto T_{ij} a_i b_j \]
is a bilinear map, and independent of basis, then $T_{ij}$ \textit{must} be the components of a rank 2 tensor. The same proof applies for higher-rank tensors.

\subsection{Quotient Theorem}
Recall from earlier that the conductivity tensor $\sigma_{ij}$ satisfying $J_i = \sigma_{ij} E_j$ was really a tensor, by using the definitions. The quotient theorem allows us to deduce similar results more generally. The name originates from the apparent `quotient' of $J_i$ by $E_j$ to give $\sigma_{ij}$.
\begin{proposition}
    Let $T_{i\dots j p\dots q}$ be an array of numbers defined in each Cartesian coordinate system, such that
    \[ v_{i\dots j} = T_{i\dots j p \dots q} u_{p\dots q} \]
    and that $v_{i\dots j}$ is a tensor for all tensors $u_{p\dots q}$. Then $T_{i\dots j p\dots q}$ is a tensor.
\end{proposition}
\begin{proof}
    We will first consider the special case $u_{p\dots q} = c_p \dots d_q$ for vectors $\vb c, \dots, \vb d$. Then by assumption,
    \[ v_{i\dots j} = T_{i\dots j p \dots q} c_p \dots d_q \]
    is a tensor. In particular,
    \[ v_{i\dots j} a_i \dots b_j = T_{i\dots j p \dots q} a_i \dots b_j c_p \dots d_q \]
    is a scalar, since the left hand side is just a contraction over all indices. Since the right hand side is invariant under a change in basis, this leads us to define the multilinear map
    \[ t(\vb a, \dots, \vb b, \vb c, \dots, \vb d) = T_{i\dots j p \dots q} a_i \dots b_j c_p \dots d_q \]
    Hence $T_{i\dots j p \dots q}$ really is a tensor.
\end{proof}
\noindent As an example, consider the linear strain tensor
\[ e_{ij} = \frac{1}{2}\qty( \pdv{u_i}{x_j} + \pdv{u_j}{x_i} ) \]
where $\vb u(\vb x)$ measures the change in displacement at $\vb x$. Experiments suggest that the internal stress tensor $\sigma_{ij}$ experienced by a body under a deformation $\vb u(\vb x)$ depends linearly on the strain $e_{ij}$ at each point. Hence we might assume that there exists some array $c_{ijk\ell}$ such that
\[ \sigma_{ij} = c_{ijk\ell} e_{k\ell} \]
However, we can't actually apply the quotient theorem here, since $e_{k\ell}$ cannot be \textit{any} tensor, it can only be any \textit{symmetric} tensor. See Example Sheet 4 for the resolution of this apparent problem: if $c_{ijk\ell} = c_{ij\ell k}$, then we can apply the quotient theorem. We call $c_{ijk\ell}$ the stiffness tensor, which is a property of the material being subjected to the force. Suppose that the material is isotropic, then we might guess that $c_{ijk\ell}$ should be isotropic. Hence,
\[ c_{ijk\ell} = \alpha \delta_{ij}\delta_{k\ell} + \beta \delta_{ik}\delta_{j\ell} + \gamma \delta_{i\ell}\delta_{jk} \]
where $\alpha, \beta, \gamma$ are scalars. Putting this into the relationship between $\sigma$ and $e$, we find
\[ \sigma_{ij} = \alpha \delta_{ij}e_{kk} + \beta e_{ij} + \gamma e_{ji} = \lambda \delta_{ij} e_{kk} + 2\mu e_{ij} \]
which is a higher-dimensional analogue of Hooke's Law. We can in fact invert this. By contracting on $(i, j)$ we find
\[ \sigma_{ii} = 3\lambda e_{ii} + 2\mu e_{ii} \]
Hence,
\[ e_{kk} = \frac{\sigma_{kk}}{3\lambda + 2\mu} \]
We then have
\[ \sigma_{ij} = \lambda \delta_{ij} \frac{\sigma_{kk}}{3\lambda + 2\mu} + 2\mu e_{ij} \implies 2\mu e_{ij} = \sigma_{ij} - \sigma_{kk} \delta_{ij} \frac{\lambda}{3\lambda + 2\mu} \]
