\subsection{Differentials and First Order Changes}
Recall that for a function $f(u_1, \dots, u_n)$, we define the differential of $f$, written $\dd{f}$, by
\[ \dd{f} = \frac{\partial f}{\partial u_i} \dd{u}_i \]
noting that the summation convention applies. The $\dd{u}_i$ are called differential forms, which can be thought of as linearly independent objects (if the coordinates $u_1, \dots, u_n$ are independent), i.e. $\alpha_i \dd{u}_i = 0 \implies \alpha_i = 0$ for all $i$. Similarly, if we have a vector $\vb x(u_1, \dots, u_n)$, we define
\[ \dd \vb x = \frac{\partial \vb x}{\partial u_i} \dd{u}_i \]
As an example, let $f(u, v, w) = u^2 + w \sin(v)$. Then
\[ \dd{f} = 2u \dd{u} + w \cos(v) \dd{v} + \sin(v) \dd{w} \]
Similarly, given
\[ \vb x(u, v, w) = \begin{pmatrix}
		u^2 - v^2 \\ w \\ e^v
	\end{pmatrix} \]
we can compute
\[ \dd \vb x = \begin{pmatrix}
		2u \\ 0 \\ 0
	\end{pmatrix} \dd{u} + \begin{pmatrix}
		-2v \\ 0 \\ e^v
	\end{pmatrix} \dd{v} + \begin{pmatrix}
		0 \\ 1 \\ 0
	\end{pmatrix} \dd{w} \]
Differentials encode information about how a function (or vector field) changes when we change the coordinates by a small amount. By calculus,
\[ f(u + \delta u_1, \dots, u_n + \delta u_n) - f(u_1, \dots, u_n) = \frac{\partial f}{\partial u_i} \delta u_i + o(\delta \vb u) \]
So if $\delta f$ denotes the change in $f(u_1, \dots, u_n)$ under this small change in coordinates, we have, to first order,
\[ \delta f \approx \frac{\partial f}{\partial u_i}\delta u_i \]
The analogous result holds for vector vields:
\[ \delta \vb x \approx \frac{\partial \vb x}{\partial u_i}\delta u_i \]

\subsection{Coordinates and Line Elements in $\mathbb R^2$}
We can create multiple different consistent coordinate systems by defining a relationship between them. For example, polar coordinates $(r, \theta)$ and Cartesian coordinates $(x, y)$ can be related by
\[ x = r \cos \theta;\quad y = r \sin \theta \]
Even though this relationship is not bijective (there are multiple polar coordinates mapping to the origin), it's still a useful coordinate system because the vast majority of points work well. Even coordinate systems with a countable amount of badly-behaved points are still useful.

A general set of coordinates $(u, v)$ on $\mathbb R^2$ can be specified by their relationship to the standard Cartesian coordinates $(x, y)$. We must specify smooth, invertible functions $x(u, v)$, $y(u, v)$. We would also like to have a small change in one coordinate system to be equivalent to a small change in the other coordinate system (i.e. the inverse is also smooth). The same principle applies in $\mathbb R^3$ for three coordinates, for example.

Consider the standard Cartesian coordinates in $\mathbb R^2$.
\[ \vb x(x, y) = \begin{pmatrix}
		x \\ y
	\end{pmatrix} = x \vb e_x + y \vb e_y \]
Note that $\{\vb e_x, \vb e_y\}$ are orthonormal, and point in the same direction regardless of the value of $\vb x$: $\vb e_x$ points in the direction of changing $x$ with $y$ held constant, for example. Equivalently,
\[ \vb e_x = \frac{\frac{\partial}{\partial x} \vb x(x, y)}{\abs{\frac{\partial}{\partial x} \vb x(x, y)}};\quad \vb e_y = \frac{\frac{\partial}{\partial y} \vb x(x, y)}{\abs{\frac{\partial}{\partial y} \vb x(x, y)}} \]
Note that
\[ \dd \vb x = \frac{\partial \vb x}{\partial x}\dd{x} + \frac{\partial \vb x}{\partial y} \dd{y} = \dd{x} \,\vb e_x + \dd{y} \,\vb e_y \]
In other words, when applying the change in coordinate $x \mapsto x + \delta x$, the vector changes (to first order) to $\vb x \mapsto \vb x + \delta x \vb e_x$. In fact, in the case of Cartesian coordinates, this change is precisely correct for any size of $\delta$, since the coordinate basis vectors are the same everywhere. We call $\dd \vb x$ the line element; it tells us how small changes in coordinates produce changes in position vectors.

Now, let us consider polar coordinates in two-dimensional space. We can use the same idea as before, giving
\[ \vb e_r = \frac{\frac{\partial}{\partial r} \vb x(r, \theta)}{\abs{\frac{\partial}{\partial r} \vb x(r, \theta)}} = \begin{pmatrix}
		\cos\theta \\ \sin\theta
	\end{pmatrix};\quad \vb e_\theta = \frac{\frac{\partial}{\partial \theta} \vb x(r, \theta)}{\abs{\frac{\partial}{\partial \theta} \vb x(r, \theta)}} = \begin{pmatrix}
		-\sin\theta \\ \cos\theta
	\end{pmatrix} \]
Therefore, we have
\[ \vb x(r, \theta) = \begin{pmatrix}
		r \cos\theta \\ r \sin\theta
	\end{pmatrix} = r\vb e_r \]
Note that $\{\vb e_r, \vb e_\theta\}$ are also orthonormal at each $(r, \theta)$, but their exact values are not the same everywhere. Since the basis vectors are orthogonal, we can call $r$ and $\theta$ orthogonal curvilinear coordinates. Also, we can compute the line element $\dd \vb x$ as
\[ \dd \vb x = \frac{\partial \vb x}{\partial r} \dd{r} + \frac{\partial \vb x}{\partial \theta} \dd \theta = \begin{pmatrix}
		\cos \theta \\ \sin \theta
	\end{pmatrix} \dd{r} + \begin{pmatrix}
		-r \sin \theta \\ r \cos \theta
	\end{pmatrix} \dd \theta = \dd{r} \, \vb e_r + r\, \dd \theta \, \vb e_\theta \]
We see that a change in $\theta$ produces (up to first order) a change $\vb x \mapsto \vb x + r \,\delta \theta \,\vb e_\theta$, a change proportional to $r$. So a small change in $\theta$ could cause quite a large change in Cartesian coordinates.

\subsection{Orthogonal Curvilinear Coordinates}
We say that $(u, v, w)$ are a set of orthogonal curvilinear coordinates if the vectors
\[ \vb e_u = \frac{\frac{\partial \vb x}{\partial u}}{\abs{\frac{\partial \vb x}{\partial u}}};\quad \vb e_v = \frac{\frac{\partial \vb x}{\partial v}}{\abs{\frac{\partial \vb x}{\partial v}}};\quad \vb e_w = \frac{\frac{\partial \vb x}{\partial w}}{\abs{\frac{\partial \vb x}{\partial w}}} \]
form a right-handed, orthonormal basis for each $(u, v, w)$; but not necessarily the same basis over the entire vector field. It is standard to write
\[ h_u = \abs{\frac{\partial \vb x}{\partial u}};\quad h_v = \abs{\frac{\partial \vb x}{\partial v}};\quad h_w = \abs{\frac{\partial \vb x}{\partial w}} \]
We call $h_u, h_v, h_w$ the scale factors.  Note that the line element is
\begin{align*}
	\dd \vb x & = \frac{\partial \vb x}{\partial u}\dd{u} + \frac{\partial \vb x}{\partial v}\dd{v} + \frac{\partial \vb x}{\partial w} \dd{w} \\
	          & = h_u \vb e_u \dd{u} + h_v \vb e_v \dd{v} + h_w \vb e_w \dd{w}
\end{align*}
So the scale factors show how first-order changes in the coordinates are scaled into changes in $\vb x$.

\subsection{Cylindrical Polar Coordinates}
We define $(\rho, \phi, z)$ by
\[ \vb x(\rho, \phi, z) = \begin{pmatrix}
		\rho \cos \phi \\
		\rho \sin \phi \\
		z
	\end{pmatrix} \]
where $0 \leq \rho; 0 \leq \phi < 2 \pi; z \in \mathbb R$. So we can find
\[ \vb e_\rho = \begin{pmatrix}
		\cos \phi \\ \sin \phi \\ 0
	\end{pmatrix};\quad \vb e_\phi = \begin{pmatrix}
		-\sin \phi \\ \cos \phi \\ 0
	\end{pmatrix};\quad \vb e_z = \begin{pmatrix}
		0 \\ 0 \\ 1
	\end{pmatrix} \]
The scale factors are
\[ h_\rho = 1;\quad h_\phi = \rho;\quad h_z = 1 \]
The line element is
\[ \dd \vb x = \dd \rho \, \vb e_\rho + \rho \, \dd \phi \, \vb e_\phi + \dd{z} \, \vb e_z \]
Note that
\[ \vb x = \rho \begin{pmatrix}
		\cos \phi \\ \sin \phi \\ 0
	\end{pmatrix} + z \begin{pmatrix}
		0 \\ 0 \\ 1
	\end{pmatrix} = \rho \vb e_\rho + z \vb e_z \]

\subsection{Spherical Polar Coordinates}
We define $(r, \theta, \phi)$ by
\[ \vb x(r, \theta, \phi) = \begin{pmatrix}
		r \cos \phi \sin \theta \\
		r \sin \phi \sin \theta \\
		r \cos \theta
	\end{pmatrix} \]
where $0 \leq r; 0 \leq \theta < 2 \pi; 0 \leq \phi < 2 \pi$. So we can find
\[ \vb e_r = \begin{pmatrix}
		\cos \phi \sin \theta \\ \sin \phi \sin \theta \\ \cos \theta
	\end{pmatrix};\quad \vb e_\theta = \begin{pmatrix}
		\cos \phi \cos \theta \\ \sin \phi \cos \theta \\ -\sin \theta
	\end{pmatrix};\quad \vb e_\phi = \begin{pmatrix}
		-\sin \phi \\ \cos \phi \\ 0
	\end{pmatrix} \]
The scale factors are
\[ h_r = 1;\quad h_\theta = r;\quad h_\phi = r \sin \theta \]
The line element is
\[ \dd \vb x = \dd{r} \, \vb e_r + r \, \dd \theta \, \vb e_\theta + r \sin \theta \, \dd \phi \, \vb e_\phi \]
Note that
\[ \vb x = r \begin{pmatrix}
		\cos \phi \sin \theta \\ \sin \phi \sin \theta \\ \cos \theta
	\end{pmatrix} = r \vb e_r \]
