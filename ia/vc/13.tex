\subsection{Divergence Theorem}
\begin{proof}
	Suppose first that
	\[
		\vb F = F_z(x,y,z) \vb e_z
	\]
	The divergence theorem states that
	\begin{equation}
		\int_V \underbrace{\pdv{F_z}{z}}_{\div{\vb F}} \dd{V} = \int_{\partial V} F_z \vb e_z \cdot \dd{\vb S}
		\tag{\(\dagger\)}
	\end{equation}
	We would like to show that these two are really the same.
	First, let us simplify the problem to a convex volume \(V\), such that we can split the boundary into two halves, one with normals in the positive \(z\) direction (\(S_+\)) and one with normals in the negative \(z\) direction (\(S_-\)).
	Then \(\partial V = S_+ \cup S_-\).
	Project the volume into the \(x\)-\(y\) plane, and call this region \(A\).
	This planar region is then the shape of the `cut' between the \(S_+\) and \(S_-\) halves.
	We can write
	\[
		S_\pm = \left\{ \vb x(x, y) = \begin{pmatrix}
			x \\ y \\ g_\pm (x, y)
		\end{pmatrix} : (x, y) \in A \right\}
	\]
	We can then say
	\begin{align*}
		\int_V \pdv{F_z}{z} \dd{V} & = \iint_{A} \left[ \int_{z = g_- (x, y)}^{g_+ (x, y)} \pdv{F_z}{z} \dd{z} \right] \dd{x} \dd{y} \\
		                           & = \iint_A \left[ F_z(x, y, g_+ (x, y)) - F_z(x, y, g_- (x, y)) \right] \dd{x}\dd{y}
	\end{align*}
	To calculate right hand side of \((\dagger)\), we need \(\dd{\vb S}\):
	\begin{align*}
		\dd{\vb S} & = \pdv{\vb x}{x} \times \pdv{\vb x}{y} \dd{x}\dd{y} \\
		           & = \begin{pmatrix}
			-\pdv*{g_\pm}{x} \\
			-\pdv*{g_\pm}{y} \\
			1                \\
		\end{pmatrix} \dd{x}\dd{y}
	\end{align*}
	Since we want the normal to point `out' of \(V\), on \(S_\pm\) we have
	\[
		\eval{\dd{\vb S}}_{S_\pm} = \pm \begin{pmatrix}
			-\pdv*{g_\pm}{x} \\
			-\pdv*{g_\pm}{y} \\
			1                \\
		\end{pmatrix} \dd{x}\dd{y}
	\]
	Therefore,
	\begin{align*}
		\int_{\partial V} \vb F \cdot \dd{\vb S} & = \left[ \int_{S_+} + \int_{S_-} \right] F_z \vb e_z \cdot \dd{\vb S}                   \\
		                                         & = \iint_A F_z(x, y, g_+(x, y)) \dd{x}\dd{y} - \iint_A F_z(x, y, g_-(x, y)) \dd{x}\dd{y}
	\end{align*}
	which matches the expression we found for the left hand side of \((\dagger)\) above.
	In the same way, we can show that
	\begin{align*}
		\int_V \pdv{F_x}{x} \dd{V} & = \int_{\partial V} F_x \vb e_x \cdot \dd{\vb S} \\
		\int_V \pdv{F_y}{y} \dd{V} & = \int_{\partial V} F_y \vb e_y \cdot \dd{\vb S} \\
	\end{align*}
	and because the integrals are linear, we can compute their sum to find
	\[
		\int_V \div{\vb F} \dd{V} = \int_{\partial V} \vb F \cdot \dd{\vb S}
	\]
	which is exactly the divergence theorem.
\end{proof}

\subsection{Green's Theorem}
We can use the two-dimensional divergence theorem to prove Green's theorem.
\begin{proof}
	Let
	\[
		\vb F = \begin{pmatrix}
			Q(x, y) \\ -P(x, y)
		\end{pmatrix}
	\]
	Then
	\[
		\iint_A \left( \pdv{Q}{x} - \pdv{P}{y} \right) \dd{x}\dd{y} = \int_A \div{\vb F} \dd{A} = \oint_{\partial A} \vb F \cdot \vb n \dd{s}
	\]
	If \(\partial A\) is parametrised with respect to arc length, this means that the unit tangent vector is
	\[
		\vb t = \begin{pmatrix}
			x'(s) \\ y'(s)
		\end{pmatrix}
	\]
	then the normal vector is
	\[
		\vb n = \begin{pmatrix}
			y'(s) \\ -x'(s)
		\end{pmatrix}
	\]
	Therefore,
	\[
		\oint_{\partial A} \vb F \cdot \vb n \dd{s} = \oint_{\partial A} \begin{pmatrix}
			Q \\ -P
		\end{pmatrix} \cdot \begin{pmatrix}
			y'(s) \\ -x'(s)
		\end{pmatrix} \dd{s} = \oint_{\partial A} P \dv{x}{s} \dd{s} + Q \dv{y}{s} \dd{s} = \oint_{\partial A} P \dd{x} + Q \dd{y}
	\]
\end{proof}

\subsection{Stokes' Theorem}
We can now use Green's theorem to derive Stokes' theorem.
\begin{proof}
	Consider a regular surface
	\[
		S = \left\{ \vb x = \vb x(u, v) \colon (u, v) \in A \right\}
	\]
	Then the boundary is
	\[
		\partial S = \left\{ \vb x = \vb x(u, v) \colon (u, v) \in \partial A \right\}
	\]
	Green's theorem gives
	\begin{equation}
		\oint_{\partial A} P \dd{u} + Q \dd{v} = \iint_A \left( \pdv{Q}{u} - \pdv{P}{v} \right) \dd{u}\dd{v}
		\tag{\(\dagger\)}
	\end{equation}
	We will now set
	\[
		P(u, v) = \vb F(\vb x(u, v)) \cdot \pdv{\vb x}{u};\quad Q(u, v) = \vb F(\vb x(u, v)) \cdot \pdv{\vb x}{v}
	\]
	Then
	\[
		P \dd{u} + Q \dd{v} = \vb F(\vb x(u, v)) \cdot \left( \pdv{\vb x}{u} \dd{u} + \pdv{\vb v} \dd{v} \right) = \vb F(\vb x(u, v)) \cdot \dd{\vb x(u, v)}
	\]
	And so we can compute the left hand side of \((\dagger)\):
	\[
		\oint_{\partial A} P \dd{u} + Q \dd{v} = \oint_{\partial S} \vb F \cdot \dd{\vb x}
	\]
	For the right hand side, we must first compute some derivatives.
	\[
		Q = F_i(\vb x(u, v))\pdv{x_i}{v} \implies \pdv{Q}{u} = \pdv{x_j}{u}\pdv{F_i}{x_j}\pdv{x_i}{v} + F_i \pdv{x_i}{u}{v}
	\]
	\[
		P = F_i(\vb x(u, v))\pdv{x_i}{u} \implies \pdv{Q}{v} = \pdv{x_j}{v}\pdv{F_i}{x_j}\pdv{x_i}{u} + F_i \pdv{x_i}{v}{u}
	\]
	Hence
	\begin{align*}
		\pdv{Q}{u} - \pdv{P}{v} & = \left( \pdv{x_i}{v}\pdv{x_j}{u} - \pdv{x_i}{u}\pdv{x_j}{v} \right) \pdv{F_i}{x_j}                       \\
		                        & = \left( \delta_{ip} \delta_{jq} - \delta_{iq} \delta_{jp} \right) \pdv{F_i}{x_j}\pdv{x_p}{v}\pdv{x_q}{u} \\
		                        & = \varepsilon_{ijk} \varepsilon_{pqk} \pdv{F_i}{x_j}\pdv{x_p}{v}\pdv{x_q}{u}                              \\
		                        & = \left[ -\curl{\vb F} \right]_k \left( -\pdv{\vb x}{u} \times \pdv{\vb x}{v} \right)_k                   \\
		                        & = \left( \curl{\vb F} \right) \cdot \left( \pdv{\vb x}{u} \times \pdv{\vb x}{v} \right)                   \\
	\end{align*}
	Therefore,
	\[
		\iint_{A} \left( \pdv{Q}{u} - \pdv{P}{v} \right) \dd{u}\dd{v} = \iint_{A} \left( \curl{\vb F} \right) \cdot \left( \pdv{\vb x}{u} \times \pdv{\vb x}{v} \right) = \iint_{S} (\curl{\vb F}) \cdot \dd{\vb S}
	\]
	which gives Stokes' theorem as required.
\end{proof}
