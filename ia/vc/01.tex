\subsection{Notation}
Throughout this course, a column vector e.g.
\[
	\begin{pmatrix}
		a \\ b \\ c
	\end{pmatrix}
\]
it should be interpreted as the vector
\[
	\vb x = a \vb e_x + b \vb e_y + x \vb e_z
\]
where \(\{ \vb e_x, \vb e_y, \vb e_z \}\) are the basis vectors aligned with the fixed Cartesian \(x, y, z\) axes in \(\mathbb R^3\).
We will be dealing with various kinds of basis vectors through the course, so it is useful to define now that column vectors written as above always represent the standard basis.

\subsection{Parametrised Curves and Smoothness}
A parametrised curve \(C\) in \(R^3\) is the image of a continuous map \(\vb x \colon [a, b] \to \mathbb R^3\), in which \(t \mapsto \vb x(t)\).
In Cartesian coordinates,
\[
	\vb x(t) = \begin{pmatrix}
		x_1(t) \\ x_2(t) \\ x_3(t)
	\end{pmatrix} = \begin{pmatrix}
		x(t) \\ y(t) \\ z(t)
	\end{pmatrix}
\]
The resultant curve has a direction, from \(\vb x(a)\) to \(\vb x(b)\).
\begin{definition}
	We say that \(C\) is a differentiable curve if each of the components \(\{x_i(t)\}\) are differentiable functions.
	\(C\) is regular if it is differentiable and \(\abs{\vb x'(t)} \neq 0\).
	If \(C\) is differentiable and regular, we say that \(C\) is smooth.
\end{definition}
\begin{note}
	We need this regularity condition because it is quite easy to create `bad curves' with cusps and spikes using only differentiable functions, for example
	\[
		\vb x(t) = (t^2, t^3)
	\]
	The components are clearly differentiable, but \(\vb x(t)\) has a cusp at \(t = 0\).
	At this point, \(\abs{\vb x'(0)} = 0\).
\end{note}
\begin{definition}
	Recall that \(x_i(t)\) is called `differentiable' at \(t\) if
	\[
		x_i(t+h) = x_i(t) + x_i'(t)h + o(h)
	\]
	where \(o(h)\) represents a function that obeys
	\[
		\lim_{h \to 0} \frac{o(h)}{h} = 0
	\]
	In terms of vectors,
	\[
		\vb x(t+h) = \vb x(t) + \vb x'(t)h + o(h)
	\]
	where here \(o(h)\) is a vector for which
	\[
		\lim_{h \to 0} \frac{\abs{o(h)}}{h} = 0
	\]
\end{definition}

\subsection{Arc Length}
We can approximate the length of a curve \(C\) by splitting it into small straight lines and summing the lengths of such lines.
We will introduce a partition \(P\) of \([a, b]\) with \(t_0 = a, t_N = b\) and
\[
	t_0 < t_1 < t_2 < \dots < t_N
\]
Let us now set \(\Delta t_i = t_{i+1} - t_i\) and \(\Delta t = \max_i \Delta t_i\).
The length of the curve relative to \(P\) is defined as
\[
	\ell(C, P) = \sum_{i=0}^{N-1} \abs{\vb x(t_{i+1}) - \vb x(t_i)}
\]
As \(\Delta t\) gets smaller, we would expect \(\ell(C, P)\) to give a better approximation to the true length of \(C\), which we will call \(\ell(C)\).
Therefore we can define the length of \(C\) by
\[
	\ell(C) = \lim_{\Delta t \to 0} \sum_{i=0}^{N-1}\abs{\vb x(t_{i+1}) - \vb x(t_i)} = \lim_{\Delta t \to 0} \ell(C, P)
\]
If this limit doesn't exist, we say that the curve is \textit{non-rectifiable}.
Suppose \(C\) is differentiable.
Then
\begin{align*}
	\vb x(t_{i+1}) & = \vb x(t_i + t_{i+1} - t_i)                          \\
	               & = \vb x(t_i + \Delta t_i)                             \\
	               & = \vb x(t_i) + \vb x'(t_i) \Delta t_i + o(\Delta t_i)
\end{align*}
It follows then that
\[
	\abs{\vb x(t_{i+1}) - \vb x(t_i)} = \abs{\vb x'(t_i)}\Delta t_i + o(\Delta t_i)
\]
So if \(C\) is differentiable,
\[
	\ell(C, P) = \lim_{\Delta t \to 0} \sum_{i=0}^{N-1} \left( \abs{\vb x'(t_i)}\Delta t_i + o(\Delta t_i)  \right)
\]
Recall that this \(o(\Delta t_i)\) term represents a function for which \(o(\Delta t_i) / \Delta t_i \to 0\).
So for any \(\varepsilon > 0\), if \(\Delta t = \max_i \Delta t_i\) is sufficiently small, we have \(\abs{o(\Delta t_i)} < \frac{\varepsilon}{b-a}\Delta t_i\), for \(i = 0, \dots, N-1\).
So by the Triangle Inequality, choosing \(\Delta t\) sufficiently small,
\[
	\abs{\ell(C, P) - \sum_{i=0}^{N-1} \abs{\vb x'(t_i)}\Delta t_i} = \abs{\sum_{i=0}^{N-1} o(\Delta t_i)} < \frac{\varepsilon}{b-a}\underbrace{\sum_{i=0}^{N-1} \Delta t_i}_{b-a} = \varepsilon
\]
So the left hand side tends to zero as \(\Delta t \to 0\).
We then get
\begin{align*}
	\ell(C) & = \lim_{\Delta t \to 0} \ell(C, P)                                     \\
	        & = \lim_{\Delta t \to 0} \sum_{i=0}^{N - 1} \abs{\vb x'(t_i)}\Delta t_i \\
	        & = \int_a^b \abs{\vb x'(t)} \dd{t}
\end{align*}
according to Analysis I, and the definition of the Riemann Integral.
So in summary, if \(C \colon [a, b] \ni t \mapsto \vb x(t)\), then
\begin{align*}
	\ell(C) & = \int_a^b \abs{\vb x'(t)} \dd{t} \\
	        & = \int_C \dd{s}
\end{align*}
where \(\dd{s}\) is the `arc length element', i.e.\ \(\dd{s} = \abs{\vb x'(t)} \dd{t}\).
Similarly, we define
\[
	\int_C f(\vb x) \dd{s} = \int_a^b f(\vb x(t)) \,\abs{\vb x'(t)} \dd{t}
\]
If \(C\) is made up of \(M\) smooth curves \(C_1, \dots, C_M\), we say that \(C\) is `piecewise smooth'.
We write \(C = C_1 + \dots + C_M\) and define
\[
	\int_C f(\vb x) \dd{s} = \sum_{i=1}^M \int_{C_i}f(\vb x) \dd{s}
\]
Now note (informally) that
\[
	\dd{s} = \abs{\vb x'(t)}\dd{t} = \sqrt{\left( \frac{\dd{x}}{\dd{t}} \right)^2 + \left( \frac{\dd{y}}{\dd{t}} \right)^2 + \left( \frac{\dd{z}}{\dd{t}} \right)^2} \dd{t}
\]
i.e.\ (now very informally)
\[
	\dd{s}^2 = \dd{x}^2 + \dd{y}^2 + \dd{z}^2
\]
which is Pythagoras' Theorem.

\subsection{Example of Arc Length}
Let \(C\) be the circle of radius \(r>0\) in \(\mathbb R^3\)
\[
	\vb x(t) = \begin{pmatrix}
		r\cos t \\ r\sin t \\ 0
	\end{pmatrix};\quad t \in [0, 2\pi]
\]
So
\[
	\vb x'(t) = \begin{pmatrix}
		-r\sin t \\ r\cos t \\ 0
	\end{pmatrix}
\]
Therefore
\[
	\int_C \dd{s} = \int_0^{2\pi} \abs{\vb x'(t)} \dd{t} = \int_0^{2\pi} \sqrt{r^2 \sin^2 t + r^2 \cos^2 t} \dd{t} = \int_0^{2\pi} r \dd{t} = 2\pi r
\]
Also, for example,
\[
	\int_C x^2 y \dd{s} = \int_0^{2\pi} (r \cos t)^2 (r \sin t) \sqrt{r^2 \sin^2 t + r^2 \cos^2 t} \dd{t} = \int_0^{2\pi} r^3 \cos^2 t \sin t \dd{t} = 0
\]

\subsection{Choice of Parametrisation of Curves}
Does \(\ell(C)\) depend on the choice of parametrisation of \(\vb x(t)\)?
For example,
\[
	\vb x(t) = \begin{pmatrix}
		r\cos t \\ r \sin t \\ 0
	\end{pmatrix};\quad t \in [0, 2\pi]
\]
and
\[
	\widetilde{\vb x}(t) = \begin{pmatrix}
		r\cos 2t \\ r \sin 2t \\ 0
	\end{pmatrix};\quad t \in [0, \pi]
\]
both give rise to a circle, but have different forms.
Suppose that \(C\) has two different parametrisations,
\[
	\vb x = \vb x_1(t);\quad a \leq t \leq b
\]
\[
	\vb x = \vb x_2(\tau);\quad \alpha \leq \tau \leq \beta
\]
There must be some relationship \(\vb x_2(\tau) = \vb x_1(t(\tau))\) for some function \(t(\tau)\), since they represent the same curve.
We can assume \(\frac{\dd{t}}{\dd \tau} \neq 0\), so the map between \(t\) and \(\tau\) is invertible and differentiable (see IB Analysis and Topology).
Note that
\begin{align*}
	\vb x_2'(\tau) & = \frac{\dd}{\dd \tau}\vb x_2(\tau)        \\
	               & = \frac{\dd}{\dd \tau}\vb x_1(t(\tau))     \\
	\intertext{By the Chain Rule,}
	               & = \frac{\dd{t}}{\dd \tau}\vb x_1'(t(\tau))
\end{align*}
And now from the above definitions,
\[
	\int_C f(\vb x) \dd{s} = \int_a^b f(\vb x_1(t)) \, \abs{\vb x_1'(t)} \dd{t}
\]
Making the substitution \(t = t(\tau)\), and assuming \(\frac{\dd{t}}{\dd \tau} > 0\), the latter integral becomes
\[
	\int_\alpha^\beta f(\vb x_2(\tau)) \, \underbrace{\abs{\vb x_1'(t(\tau))} \,\frac{\dd{t}}{\dd \tau}\dd \tau}_{\abs{\vb x_2'(\tau)}\,\dd \tau} = \int_\alpha^\beta f(\vb x_2(\tau)) \, \abs{\vb x_2'(\tau)} \,\dd \tau
\]
which is precisely the same as \(\int_C f(\vb x) \dd{s}\) using the \(\vb x_2(\tau)\) parametrisation.
When \(\frac{\dd{t}}{\dd \tau} < 0\), you get the same result.
So the definition of \(\int_C f(\vb x) \dd{s}\) does \textit{not} depend on the choice of parametrisation of \(C\).
