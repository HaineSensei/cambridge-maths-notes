\subsection{Definition of Integral in \(\mathbb R^2\)}
We can integrate over a bounded region \(D \subset \mathbb R^2\).
To do this, we can cover \(D\) with small, disjoint sets \(A_{ij}\) each with area \(\delta A_{ij}\).
Each of these sets \(A_{ij}\) are contained in a disc of radius \(\varepsilon > 0\).
Let \((x_i, y_j)\) be points contained in each \(A_{ij}\).
We now define
\[
	\int_D f(\vb x) \dd{A} = \lim_{\varepsilon \to 0} \sum_{i, j} f(x_i, y_j) \,\delta A_{ij}
\]
The integral exists if it is independent of the choice of partitions \(A_{ij}\) and the points \((x_i, y_j)\).
The obvious choice of partitioning \(D\) is to use rectangles where the area of each rectangle is \(\delta A_{ij} = \delta x_i \delta y_j\).
We can create horizontal `strips' of height \(\delta y\) which we can integrate over.
The possible \(x\) coordinates for this strip are \(x_y = \{ x \colon (x, y) \in D \}\).
We can take the limit as \(\delta x \to 0\), giving
\[
	\delta y \int_{x_y} f(x, y) \dd{x}
\]
Summing over each such strip, taking the limit as \(\delta y \to 0\), we have
\[
	\int_D f(x, y) \dd{A} = \int_Y \left( \int_{x_y} f(x, y) \dd{x} \right) \dd{y}
\]
where \(Y\) is the set of all possible \(y\) coordinates, i.e.\ \(Y = \{ y \colon \exists x, (x, y) \in D \}\).
We can equivalently sum over all vertical strips, and get
\[
	\int_D f(x, y) \dd{A} = \int_X \left( \int_{y_x} f(x, y) \dd{y} \right) \dd{x}
\]
More concisely, we can write the following (Fubini's Theorem):
\[
	\dd{A} = \dd{x} \, \dd{y} = \dd{y} \, \dd{x}
\]
Let us consider an example; let \(D\) be the triangle with vertices \((0, 0), (1, 0), (0, 1)\).
If \(f(x, y) = xy^2\), then by integrating over horizontal strips, we have
\begin{align*}
	\int_D f(x, y) \dd{A} & = \int_0^1 \left( \int_0^{1-y} xy^2 \dd{x} \right) \dd{y}  \\
	                      & = \int_0^1 \left[ \frac{1}{2}x^2y^2 \right]_0^{1-y} \dd{y} \\
	                      & = \int_0^1 \frac{1}{2}(1-y)^2y^2 \dd{y}                    \\
	                      & = \frac{1}{60}
\end{align*}
Instead, integrating over vertical strips, we have
\begin{align*}
	\int_D f(x, y) \dd{A} & = \int_0^1 \left( \int_0^{1-x} xy^2 \dd{y} \right) \dd{x} \\
	                      & = \int_0^1 \left[ \frac{1}{3} xy^3 \right]_0^{1-x} \dd{x} \\
	                      & = \int_0^1 \frac{1}{3} x(1-x)^3 \dd{x}                    \\
	                      & = \frac{1}{60}
\end{align*}
Note that if \(f(x, y) = g(x) \cdot h(y)\), and \(D\) is a rectangle \(\{ (x, y) \colon a \leq x \leq b, c \leq y \leq d \}\), then
\[
	\int_A f(x, y) \dd{A} = \left( \int_a^b g(x) \dd{x} \right)\left( \int_c^d h(y) \dd{y} \right)
\]

\subsection{Change of Variables in \(\mathbb R^2\)}
It can be useful to introduce a change of variables in order to compute the one-dimensional integral.
For example, if \(x\) is represented as a function of \(u\),
\[
	\int_a^b f(x) \dd{x} = \int_{x^{-1}(a)}^{x^{-1}(b)} f(x(u)) \frac{\dd{x}}{\dd{u}}\dd{u}
\]
Note that if \(\frac{\dd{x}}{\dd{u}} > 0\), then the right hand side integral is taken over a limit from a smaller value to a larger one, but if \(\frac{\dd{x}}{\dd{u}} < 0\), then the integral is the `wrong way round'.
If \(I = [a,b]\) and \(I' = x^{-1} I\), we have
\[
	\int_I f(x) \dd{x} = \int_{I'} f(x(u)) \abs{\frac{\dd{x}}{\dd{u}}} \dd{u}
\]
where the absolute value is used since \(I'\) is defined as going from the lower limit to the upper limit.
There is a similar formula in 2D.
\begin{proposition}
	Let \(\vb x(u, v) = (x(u, v), y(u, v))\) be a smooth, invertible transformation with a smooth inverse that maps the region \(D'\) in the \((u, v)\) plane to the region \(D\) in the \((x, y)\) plane.
	(This map must be a bijection; every point must have a unique inverse.) Then
	\[
		\iint_D f(x, y) \dd{x} \dd{y} = \iint_{D'} f(x(u, v), y(u, v)) \abs{\frac{\partial (x, y)}{\partial (u, v)}}\dd{u} \dd{v}
	\]
	where
	\[
		\frac{\partial (x, y)}{\partial (u, v)} = J = \det \begin{pmatrix}
			\partial x / \partial u & \partial x / \partial v \\
			\partial y / \partial u & \partial y / \partial v
		\end{pmatrix} = \det \left( \frac{\partial \vb x}{\partial u} \,\middle|\, \frac{\partial \vb x}{\partial v} \right)
	\]
	is the Jacobian determinant.
	More concisely,
	\[
		\dd{x} \, \dd{y} = \abs{J} \, \dd{u} \, \dd{v}
	\]
	It doesn't matter if the Jacobian vanishes at a single point, since the area of a single point is zero and hence will have no contribution to the result.
	The Jacobian being zero means that something non-smooth is happening at this point, so it is important to consider why this point is special.
\end{proposition}
\begin{proof}
	We can form a partition of \(D\) by using the image of a rectangular partition of \(D'\).
	Let the rectangular partition be characterised by a horizontal step \(\delta x\) and a vertical step of \(\delta y\).
	Then each small rectangle in \(D'\) is mapped to some small (not necessarily rectangular) region in \(D'\), with vertices
	\[
		\vb x(u_i, v_j), \vb x(u_{i+1}, v_j), \vb x(u_{i+1}, v_{j+1}), \vb x(u_i, v_{j+1})
	\]
	To first order, the area of this region is the area of the parallelogram with the same vertices.
	Two of the sides of the parallelogram are
	\[
		\vb x(u_{i+1}, v_j) - \vb x(u_i, v_j) \approx \frac{\partial \vb x}{\partial u}(u_i, v_j) \delta u
	\]
	\[
		\vb x(u_i, v_{j+1}) - \vb x(u_i, v_j) \approx \frac{\partial \vb x}{\partial v}(u_i, v_j) \delta v
	\]
	So the area of the parallelogram is approximately
	\begin{align*}
		\abs{\frac{\partial \vb x}{\partial u}(u_i, v_j) \delta u \cdot \frac{\partial \vb x}{\partial v}(u_i, v_j) \delta v} & = \abs{\det \left( \frac{\partial \vb x}{\partial u}(u_i, v_j) \,\middle|\, \frac{\partial \vb x}{\partial v}(u_i, v_j) \right)} \\
		                                                                                                                      & = \abs{J(u_i, v_j)} \,\delta u \,\delta v                                                                                        \\
		                                                                                                                      & = \delta A_{ij}
	\end{align*}
	Hence,
	\begin{align*}
		\int_D f \, \dd{A} & = \lim_{\varepsilon \to 0} \sum_{ij} f(x_i, y_j)\,\delta A_{ij}                                           \\
		                   & = \lim_{\varepsilon \to 0} \sum_{ij} f(x(u_i, v_j), y(u_i, v_j)) \,\abs{J(u_i, v_j)} \,\delta u\,\delta v \\
		                   & = \iint_{D'} f(x(u, v), y(u, v)) \,\abs{J(u_i, v_j)} \dd{u} \dd{v}
	\end{align*}
\end{proof}
\noindent As an example, let us consider polar coordinates \((\rho, \phi)\), where
\[
	x(\rho, \phi) = \rho \cos \phi;\quad y(\rho, \phi) = \rho \sin \phi
\]
Hence,
\[
	\abs{J} = \abs{\det \begin{pmatrix}
			\cos \phi & -\rho \sin \phi \\
			\sin \phi & \rho \cos \phi
		\end{pmatrix}} = \abs{\rho} = \rho
\]
If \(D = \{ (x, y) \colon x > 0,\, y > 0,\, x^2 + y^2 < r^2 \}\), which is a quarter-circle of radius \(r\) in the first quadrant, then \(D' = \{ (\rho, \phi) \colon 0 < \rho < r,\, 0 < \phi < \frac{\pi}{2} \}\).
This is notably a rectangle in polar coordinates.
\[
	\iint_D f(x, y)\dd{x} \dd{y} = \iint_{D'} f(\rho \cos \phi, \rho \sin \phi) \,\rho \,\dd \rho \,\dd \phi
\]
So, for example, if we let \(r \to \infty\), then
\[
	\int_{x=0}^\infty \int_{y=0}^\infty f(x, y) \dd{y}\dd{x} = \int_{\phi = 0}^{\frac{\pi}{2}} \int_{\rho = 0}^\infty f(\rho \cos \phi, \rho \sin \phi) \,\rho\,\dd \rho\,\dd \phi
\]
Consider
\[
	I = \int_0^\infty e^{-x^2} \dd{x}
\]
Then,
\begin{align*}
	I^2        & = \int_0^\infty e^{-x^2} \dd{x} \cdot \int_0^\infty e^{-y^2} \dd{y}                                \\
	           & = \int_{x=0}^\infty \int_{y=0}^\infty e^{-x^2-y^2} \dd{y}\dd{x}                                    \\
	           & = \int_{\phi = 0}^{\frac{\pi}{2}} \int_{\rho = 0}^\infty e^{-\rho^2} \,\rho\,\dd \rho\,\dd \phi    \\
	           & = \frac{\pi}{2} \int_0^\infty \frac{\dd}{\dd \rho} \left( -\frac{1}{2}e^{-\rho^2} \right) \dd \rho \\
	           & = \frac{\pi}{4}                                                                                    \\
	\implies I & = \frac{\sqrt{\pi}}{2}
\end{align*}
