\subsection{Notation}
Throughout this course, a column vector e.g.
\[
	\begin{pmatrix}
		a \\ b \\ c
	\end{pmatrix}
\]
should be interpreted as the vector
\[
	\vb x = a \vb e_x + b \vb e_y + x \vb e_z
\]
where \(\{ \vb e_x, \vb e_y, \vb e_z \}\) are the basis vectors aligned with the fixed Cartesian \(x, y, z\) axes in \(\mathbb R^3\).
We will be dealing with various kinds of basis vectors through the course, so it is useful to define now that column vectors written as above always represent the standard basis.

\subsection{Parametrised curves and smoothness}
A parametrised curve \(C\) in \(R^3\) is the image of a continuous map \(\vb x \colon [a, b] \to \mathbb R^3\), in which \(t \mapsto \vb x(t)\).
In Cartesian coordinates,
\[
	\vb x(t) = \begin{pmatrix}
		x_1(t) \\ x_2(t) \\ x_3(t)
	\end{pmatrix} = \begin{pmatrix}
		x(t) \\ y(t) \\ z(t)
	\end{pmatrix}
\]
The resultant curve has a direction, from \(\vb x(a)\) to \(\vb x(b)\).
\begin{definition}
	We say that \(C\) is a differentiable curve if each of the components \(\{x_i(t)\}\) are differentiable functions.
	\(C\) is regular if it is differentiable and \(\abs{\vb x'(t)} \neq 0\).
	If \(C\) is differentiable and regular, we say that \(C\) is smooth.
\end{definition}
\begin{note}
	We need this regularity condition because it is quite easy to create `bad curves' with cusps and spikes using only differentiable functions, for example
	\[
		\vb x(t) = (t^2, t^3)
	\]
	The components are clearly differentiable, but \(\vb x(t)\) has a cusp at \(t = 0\).
	At this point, \(\abs{\vb x'(0)} = 0\).
\end{note}
\begin{definition}
	Recall that \(x_i(t)\) is called `differentiable' at \(t\) if
	\[
		x_i(t+h) = x_i(t) + x_i'(t)h + o(h)
	\]
	where \(o(h)\) represents a function that obeys
	\[
		\lim_{h \to 0} \frac{o(h)}{h} = 0
	\]
	In terms of vectors,
	\[
		\vb x(t+h) = \vb x(t) + \vb x'(t)h + o(h)
	\]
	where here \(o(h)\) is a vector for which
	\[
		\lim_{h \to 0} \frac{\abs{o(h)}}{h} = 0
	\]
\end{definition}

\subsection{Arc length}
We can approximate the length of a curve \(C\) by splitting it into small straight lines and summing the lengths of such lines.
We will introduce a partition \(P\) of \([a, b]\) with \(t_0 = a, t_N = b\) and
\[
	t_0 < t_1 < t_2 < \dots < t_N
\]
Let us now set \(\Delta t_i = t_{i+1} - t_i\) and \(\Delta t = \max_i \Delta t_i\).
The length of the curve relative to \(P\) is defined as
\[
	\ell(C, P) = \sum_{i=0}^{N-1} \abs{\vb x(t_{i+1}) - \vb x(t_i)}
\]
As \(\Delta t\) gets smaller, we would expect \(\ell(C, P)\) to give a better approximation to the true length of \(C\), which we will call \(\ell(C)\).
Therefore we can define the length of \(C\) by
\[
	\ell(C) = \lim_{\Delta t \to 0} \sum_{i=0}^{N-1}\abs{\vb x(t_{i+1}) - \vb x(t_i)} = \lim_{\Delta t \to 0} \ell(C, P)
\]
If this limit doesn't exist, we say that the curve is \textit{non-rectifiable}.
Suppose \(C\) is differentiable.
Then
\begin{align*}
	\vb x(t_{i+1}) & = \vb x(t_i + t_{i+1} - t_i)                          \\
	               & = \vb x(t_i + \Delta t_i)                             \\
	               & = \vb x(t_i) + \vb x'(t_i) \Delta t_i + o(\Delta t_i)
\end{align*}
It follows then that
\[
	\abs{\vb x(t_{i+1}) - \vb x(t_i)} = \abs{\vb x'(t_i)}\Delta t_i + o(\Delta t_i)
\]
So if \(C\) is differentiable,
\[
	\ell(C, P) = \lim_{\Delta t \to 0} \sum_{i=0}^{N-1} \left( \abs{\vb x'(t_i)}\Delta t_i + o(\Delta t_i)  \right)
\]
Recall that this \(o(\Delta t_i)\) term represents a function for which \(o(\Delta t_i) / \Delta t_i \to 0\).
So for any \(\varepsilon > 0\), if \(\Delta t = \max_i \Delta t_i\) is sufficiently small, we have \(\abs{o(\Delta t_i)} < \frac{\varepsilon}{b-a}\Delta t_i\), for \(i = 0, \dots, N-1\).
So by the Triangle Inequality, choosing \(\Delta t\) sufficiently small,
\[
	\abs{\ell(C, P) - \sum_{i=0}^{N-1} \abs{\vb x'(t_i)}\Delta t_i} = \abs{\sum_{i=0}^{N-1} o(\Delta t_i)} < \frac{\varepsilon}{b-a}\underbrace{\sum_{i=0}^{N-1} \Delta t_i}_{b-a} = \varepsilon
\]
So the left hand side tends to zero as \(\Delta t \to 0\).
We then get
\begin{align*}
	\ell(C) & = \lim_{\Delta t \to 0} \ell(C, P)                                     \\
	        & = \lim_{\Delta t \to 0} \sum_{i=0}^{N - 1} \abs{\vb x'(t_i)}\Delta t_i \\
	        & = \int_a^b \abs{\vb x'(t)} \dd{t}
\end{align*}
according to Analysis I, and the definition of the Riemann Integral.
So in summary, if \(C \colon [a, b] \ni t \mapsto \vb x(t)\), then
\begin{align*}
	\ell(C) & = \int_a^b \abs{\vb x'(t)} \dd{t} \\
	        & = \int_C \dd{s}
\end{align*}
where \(\dd{s}\) is the `arc length element', i.e.\ \(\dd{s} = \abs{\vb x'(t)} \dd{t}\).
Similarly, we define
\[
	\int_C f(\vb x) \dd{s} = \int_a^b f(\vb x(t)) \,\abs{\vb x'(t)} \dd{t}
\]
If \(C\) is made up of \(M\) smooth curves \(C_1, \dots, C_M\), we say that \(C\) is `piecewise smooth'.
We write \(C = C_1 + \dots + C_M\) and define
\[
	\int_C f(\vb x) \dd{s} = \sum_{i=1}^M \int_{C_i}f(\vb x) \dd{s}
\]
Now note (informally) that
\[
	\dd{s} = \abs{\vb x'(t)}\dd{t} = \sqrt{\left( \frac{\dd{x}}{\dd{t}} \right)^2 + \left( \frac{\dd{y}}{\dd{t}} \right)^2 + \left( \frac{\dd{z}}{\dd{t}} \right)^2} \dd{t}
\]
i.e.\ (now very informally)
\[
	\dd{s}^2 = \dd{x}^2 + \dd{y}^2 + \dd{z}^2
\]
which is Pythagoras' Theorem.

\subsection{Example of arc length}
Let \(C\) be the circle of radius \(r>0\) in \(\mathbb R^3\)
\[
	\vb x(t) = \begin{pmatrix}
		r\cos t \\ r\sin t \\ 0
	\end{pmatrix};\quad t \in [0, 2\pi]
\]
So
\[
	\vb x'(t) = \begin{pmatrix}
		-r\sin t \\ r\cos t \\ 0
	\end{pmatrix}
\]
Therefore
\[
	\int_C \dd{s} = \int_0^{2\pi} \abs{\vb x'(t)} \dd{t} = \int_0^{2\pi} \sqrt{r^2 \sin^2 t + r^2 \cos^2 t} \dd{t} = \int_0^{2\pi} r \dd{t} = 2\pi r
\]
Also, for example,
\[
	\int_C x^2 y \dd{s} = \int_0^{2\pi} (r \cos t)^2 (r \sin t) \sqrt{r^2 \sin^2 t + r^2 \cos^2 t} \dd{t} = \int_0^{2\pi} r^3 \cos^2 t \sin t \dd{t} = 0
\]

\subsection{Choice of parametrisation of curves}
Does \(\ell(C)\) depend on the choice of parametrisation of \(\vb x(t)\)?
For example,
\[
	\vb x(t) = \begin{pmatrix}
		r\cos t \\ r \sin t \\ 0
	\end{pmatrix};\quad t \in [0, 2\pi]
\]
and
\[
	\widetilde{\vb x}(t) = \begin{pmatrix}
		r\cos 2t \\ r \sin 2t \\ 0
	\end{pmatrix};\quad t \in [0, \pi]
\]
both give rise to a circle, but have different forms.
Suppose that \(C\) has two different parametrisations,
\[
	\vb x = \vb x_1(t);\quad a \leq t \leq b
\]
\[
	\vb x = \vb x_2(\tau);\quad \alpha \leq \tau \leq \beta
\]
There must be some relationship \(\vb x_2(\tau) = \vb x_1(t(\tau))\) for some function \(t(\tau)\), since they represent the same curve.
We can assume \(\frac{\dd{t}}{\dd \tau} \neq 0\), so the map between \(t\) and \(\tau\) is invertible and differentiable (see IB Analysis and Topology).
Note that
\begin{align*}
	\vb x_2'(\tau) & = \frac{\dd}{\dd \tau}\vb x_2(\tau)        \\
	               & = \frac{\dd}{\dd \tau}\vb x_1(t(\tau))     \\
	\intertext{By the Chain Rule,}
	               & = \frac{\dd{t}}{\dd \tau}\vb x_1'(t(\tau))
\end{align*}
And now from the above definitions,
\[
	\int_C f(\vb x) \dd{s} = \int_a^b f(\vb x_1(t)) \, \abs{\vb x_1'(t)} \dd{t}
\]
Making the substitution \(t = t(\tau)\), and assuming \(\frac{\dd{t}}{\dd \tau} > 0\), the latter integral becomes
\[
	\int_\alpha^\beta f(\vb x_2(\tau)) \, \underbrace{\abs{\vb x_1'(t(\tau))} \,\frac{\dd{t}}{\dd \tau}\dd \tau}_{\abs{\vb x_2'(\tau)}\,\dd \tau} = \int_\alpha^\beta f(\vb x_2(\tau)) \, \abs{\vb x_2'(\tau)} \,\dd \tau
\]
which is precisely the same as \(\int_C f(\vb x) \dd{s}\) using the \(\vb x_2(\tau)\) parametrisation.
When \(\frac{\dd{t}}{\dd \tau} < 0\), you get the same result.
So the definition of \(\int_C f(\vb x) \dd{s}\) does \textit{not} depend on the choice of parametrisation of \(C\).

\subsection{Parametrisation according to arc length}
We know that for any curve \(C\) there exist multiple unique parametrisations.
We will define the arc-length function for a curve \([a, b] \ni t \mapsto \vb x(t)\) by
\[
	s(t) = \int_a^t \abs{\vb x'(\tau)} \,\dd \tau
\]
So \(s(a) = 0, s(b) = \ell(C)\).
Using the Fundamental Theorem of Calculus, we have
\[
	s'(t) = \abs{\vb x'(t)} \geq 0
\]
For regular curves, we have that
\[
	s'(t) > 0
\]
So we can invert the relationship between \(s\) and \(t\); i.e.\ we can find \(t\) as a function of \(s\).
Hence, we can parametrise curves with respect to arc length.
If we write
\[
	\vb r(s) = \vb x(t(s))
\]
where \(0 \leq s \leq \ell(C)\), then by the chain rule we have
\[
	\frac{\dd{t}}{\dd{s}} = \frac{1}{\frac{\dd{s}}{\dd{t}}} = \frac{1}{\abs{\vb x'(t(s))}}
\]
So
\[
	\vb r'(s) = \frac{\dd}{\dd{s}} \vb x(t(s)) = \frac{\dd{t}}{\dd{s}} \vb x'(t(s)) = \frac{\vb x'(t(s))}{\abs{\vb x'(t(s))}}
\]
In other words, \(\vb r'(s)\) is a unit vector tangential to the curve.
This (consistently) gives
\[
	\ell(C) = \int_0^{\ell(C)} \abs{\vb r'(s)} \dd{s} = \int_0^{\ell(C)} \dd{s}
\]
as previously found above.

\subsection{Curvature}
Throughout this section, we will be talking about a generic regular curve \(C\), parametrised with respect to arc length, where a position vector on \(C\) is given by \(\vb r(s)\).
We will define the tangent vector
\[
	\vb t(s) = \vb r'(s)
\]
We already know that \(\abs{\vb t(s)} = 1\).
Therefore the only part of \(\vb t\) that changes with respect to \(s\) is its direction.
So \(\vb t'(s) = \vb r''(s)\) only measures the change in the direction of the tangent as we move along the curve.
So intuitively, if \(\abs{\vb r''(s)}\) is large then the curve is rapidly changing direction.
If \(\abs{\vb r''(s)}\) is small, the curve is approximately flat; there is little change in direction.
Using this intuition, we will define curvature as
\[
	\kappa(s) = \abs{\vb r''(s)} = \abs{\vb t'(s)}
\]
In other words \(\kappa\) is the magnitude of the acceleration a particle experiences while moving along the curve at unit speed.

\subsection{Torsion}
Since \(\vb t = \vb r'(s)\) is a unit vector, differentiating \(\vb t \cdot \vb t = 1\) gives \(\vb t \cdot \vb t' = 0\).
We will define the principal normal \(\vb n\) by the formula
\[
	\vb t' = \kappa \vb n
\]
Note that \(\vb n\) is everywhere normal to the curve \(C\), since it is always perpendicular to the tangent vector \(\vb t\), since \(\vb t \cdot \vb n = 0\).
We can extend the vectors \(\{ \vb t, \vb n \}\) into an orthonormal basis by computing the cross product:
\[
	\vb b = \vb t \times \vb n
\]
We call \(\vb b\) the binormal.
It is a unit vector, since it is the cross product of two orthogonal unit vectors in \(\mathbb R^3\).
We also have that \(\vb b \cdot \vb b' = 0\); also since \(\vb t \cdot \vb b = 0\) and \(\vb n \cdot \vb b = 0\), we must have
\[
	0 = (\vb t \cdot \vb b)' = \vb t' \cdot \vb b + \vb t \cdot \vb b' = \kappa \vb n \cdot \vb b + \vb t \cdot \vb b' = \vb t \cdot \vb b'
\]
So \(\vb b'\) is orthogonal to both \(\vb t\) and \(\vb b\), i.e.\ it is parallel to \(\vb n\).
We will define the torsion \(\tau\) of a curve by
\[
	\vb b' = -\tau \vb n
\]
A physical interpretation of torsion is a kind of `corkscrew' rotation in three dimensions.

\begin{proposition}[Fundamental Theorem of Differential Geometry of Curves]
	The curvature \(\kappa(s)\) and torsion \(\tau(s)\) uniquely define a curve in \(\mathbb R^3\), up to translation and orientation.
\end{proposition}
\begin{proof}
	Since \(\vb n = \vb b \times \vb t\), we have \(\vb t' = \kappa(\vb b \times \vb t)\) and \(\vb b' = -\tau(\vb b \times \vb t)\).
	This gives six equations (written in component form) for six unknowns.
	Given \(\kappa(s)\) and \(\tau(s)\), and given \(\vb t(0)\) and \(\vb b(0)\), we can construct the functions \(\vb t(s), \vb b(s), \vb n(s) = \vb b(s) \times \vb t(s)\).
\end{proof}

\subsection{Radius of curvature}
A generic curve \(s \mapsto \vb r(s)\) can be Taylor expanded around \(s=0\).
Writing \(\vb t = \vb t(0), \vb n = \vb n(0)\) and so on, we have
\begin{align*}
	\vb r(s) & = \vb r + s\vb r' + \frac{1}{2}s^2 \vb r'' + o(s^2)    \\
	         & = \vb r + s\vb t + \frac{1}{2}s^2 \kappa\vb n + o(s^2)
\end{align*}
What circle that touches the curve at \(s=0\) would be the best approximation for the curve at this point?
Since the circle touches the curve, we know the position vectors (of the curve and the circle) match, and their first derivatives match.
So we want to unify the second derivatives.
The equation of such a circle of radius \(R\) is
\[
	\vb x(\theta) = \vb r + R(1 - \cos \theta)\vb n + R(\sin \theta) \vb t
\]
Expanding this for small \(\theta\) gives
\[
	\vb x(\theta) = \vb r + R\theta\vb t + \frac{1}{2} R\theta^2\vb n + o(\theta^2)
\]
But the arc length on a circle is simply \(R\theta\).
So in terms of arc length,
\[
	\vb x(\theta) = \vb r + s\vb t + \frac{1}{2} s^2 \frac{1}{R}\vb n + o(s^2)
\]
Hence by comparing coefficients,
\[
	R = \frac{1}{\kappa}
\]
We name this \(R(s)\) the radius of curvature.

\subsection{Gaussian curvature (non-examinable)}
This subsection is non-examinable.
How can we find the curvature of a surface?
At any point \(\vb r\) on a surface, we have a normal vector \(\vb n\).
We can construct a plane containing this normal; such a plane will then intersect the surface near \(\vb r\).
This intersection is a curve \(C\), which has a curvature \(\kappa\).
The choice of plane is arbitrary, however.
To unify all of these different possible results for \(\kappa\), we can compute the Gaussian curvature \(\kappa_G\) by
\[
	\kappa_G = \kappa_{\text{min}} \kappa_{\text{max}}
\]
\begin{itemize}
	\item The Gaussian curvature of a flat plane is zero, since the minimum and maximum curvatures are both zero.
	\item On any point on a sphere of radius \(R\), the Gaussian curvature is \(\frac{1}{R^2}\), since any plane containing the normal produces a great circle of radius \(R\), i.e.\ of curvature \(\frac{1}{\kappa}\).
\end{itemize}

\begin{theorem}[Gauss's Remarkable Theorem]
	The Gaussian curvature of a surface \(S\) is invariant under local isometries; i.e.\ if you bend the surface without stretching it.
\end{theorem}
