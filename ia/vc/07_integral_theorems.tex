\subsection{Green's theorem}
\begin{proposition}
	If \(P = P(x, y)\) and \(Q = Q(x, y)\) are continuously differentiable on a planar domain \(A \cup \partial A\) (\(A\) and its boundary), and \(\partial A\) is piecewise smooth, then
	\[
		\oint_{\partial A} P \dd{x} + Q \dd{y} = \iint_{A} \left( \pdv{Q}{x} - \pdv{P}{y} \right) \dd{x}\dd{y}
	\]
	where the orientation of \(\partial A\) is such that \(A\) lies to the left while traversing \(\partial A\).
\end{proposition}
\noindent Note that it is easy to arrive at this result for a rectangle.
In this case,
\begin{align*}
	\iint_{A} \left( \pdv{Q}{x} - \pdv{P}{y} \right) \dd{x}\dd{y} & = \int_c^d \dd{y} \int_a^b \dd{x} \pdv{Q}{x} - \int_a^b \dd{x} \int_x^d \dd{y} \pdv{P}{y} \\
	                                                              & = \int_c^d [Q(b, y) - Q(a, y)] \dd{y} + \int_a^b [P(x, c) - P(x, d)] \dd{x}               \\
	                                                              & = \oint_{\partial A} P \dd{x} + Q \dd{y}
\end{align*}
It then intuitively follows that we can approximate a surface with a set of small rectangles, and then the theorem should hold.
As an example, let
\[
	P = -\frac{1}{2}y;\quad Q = \frac{1}{2}x
\]
Then the area of some region is given by
\begin{align*}
	\iint_A \dd{x}\dd{y} & = \iint_A \left( \frac{1}{2} + \frac{1}{2} \right) \dd{x}\dd{y} \\
	                     & = \iint_A \left( \pdv{Q}{x} - \pdv{P}{y} \right) \dd{x}\dd{y}   \\
	                     & = \frac{1}{2}\oint_{\partial A} x\dd{y} - y\dd{x}
\end{align*}
So letting \(A\) be the ellipse \(\frac{x^2}{a^2} + \frac{y^2}{b^2} \leq 1\), we can parametrise \(\partial A\) by
\[
	[0, 2 \pi] \ni t \mapsto \begin{pmatrix}
		a \cos t \\ b \sin t
	\end{pmatrix}
\]
Hence the area is
\[
	\frac{1}{2}\int_0^{2\pi} \left( ab\cos^2 t + ab\sin^2 t \right) \dd{t} = \pi ab
\]

\subsection{Stokes' theorem}
\begin{proposition}
	If \(\vb F(\vb x)\) is a continuously differentiable vector field, and \(S\) is an orientable, piecewise regular surface with a piecewise smooth boundary \(\partial S\), then
	\[
		\int_S (\curl{\vb F}) \cdot \dd \vb S = \oint_{\partial S} \vb F\cdot \dd{\vb x}
	\]
\end{proposition}
\noindent This can be thought of as a generalisation to the fundamental theorem of calculus.
From the fundamental theorem, we know that the integral of a differentiated function over an interval \(I\) is just the original function evaluated at the boundary \(\partial I\).
Likewise, Stokes' theorem states that the integral of the curl of a function (just another differential operator) over a surface \(S\) is just the original function evaluated at the boundary of the surface \(\partial S\).
In the one-dimensional fundamental theorem of calculus, we say that the function `evaluated over the boundary' is simply the function applied to the final point, minus the function applied to the initial point; we are in some sense considering every point on the boundary \(\partial I\).
But in the case of \(\partial S\) being a curve, we must integrate around the curve boundary, since without an integral we can't consider infinitely many boundary points.

Note that for a surface to be `orientable', it simply means that it has two sides, an inside and an outside.
There must be a consistent choice of normal at each point.
For example, a sphere is orientable, but a M\"obius strip is not orientable.

\begin{example}
	Let \(S\) be a cap of a sphere:
	\[
		S = \left\{ \vb x(\theta, \phi) = \begin{pmatrix}
			\sin\theta\cos\phi \\ \sin\theta\sin\phi \\ \cos\theta
		\end{pmatrix} \equiv \vb e_r; 0 \leq \theta \leq \alpha; 0 \leq \phi < 2 \pi \right\}
	\]
	Now, let
	\[
		\vb F(\vb x) = \begin{pmatrix}
			-x^2y \\ 0 \\ 0
		\end{pmatrix} \implies \curl{\vb F} = \begin{pmatrix}
			0 \\ 0 \\ x^2
		\end{pmatrix}
	\]
	On \(S\),
	\[
		\dd {\vb S} = \dv{\vb x}{\theta} \times \dv{\vb x}{\phi} \dd{\theta}\dd{\phi} = \vb e_r \sin\theta \dd{\theta}\dd{\phi}
	\]
	Note that since
	\[
		x^2 \vb e_z \cdot \vb e_r = (\sin\theta\cos\phi)^2 \cos\theta
	\]
	on \(S\), we can compute
	\[
		\int_S (\curl{\vb F}) \cdot \dd{\vb S} = \int_{\phi = 0}^{2 \pi} \left( \int_{\theta = 0}^\alpha (\sin\theta\cos\phi)^2 \cos\theta \sin\theta \dd{\theta} \right) \dd{\phi} = \frac{\pi}{4}\sin^4 \alpha
	\]
	We can instead compute the integral over the boundary.
	By Stokes' theorem, the results should match.
	\(\partial S\) is described by
	\[
		[0, 2\pi] \ni t \mapsto \begin{pmatrix}
			\sin\alpha\cos t \\ \sin\alpha\sin t \\ \cos\alpha
		\end{pmatrix}
	\]
	Then
	\[
		\dd{\vb x} = \dv{\vb x}{t} \dd{t} = \sin \alpha\begin{pmatrix}
			-\sin t \\ \cos t \\ 0
		\end{pmatrix} \dd{t}
	\]
	We can show that
	\[
		\oint_{\partial S} \vb F \cdot \dd{\vb S} = \sin^4 \alpha \int_0^{2\pi} \left( -\cos^2 t \sin t \right)\left( -\sin t \right) \dd{t} = \frac{\pi}{4}\sin^4 \alpha
	\]
\end{example}

\subsection{Stokes' theorem on closed surfaces}
If \(S\) is an orientable, closed surface, and \(\vb F\) is continuously differentiable, then
\[
	\int_S (\curl{\vb F}) \cdot \dd{\vb S} = 0
\]
This is clear since \(\partial S = \varnothing\).

\subsection{Zero circulation and irrotationality}
\begin{proposition}
	If \(\vb F\) is continuously differentiable, and for every loop \(C\) we have that
	\[
		\oint_C \vb F \cdot \dd{\vb x} = 0
	\]
	then \(\curl{\vb F} = \vb 0\).
	In other words, \(\vb F\) is irrotational if and only if \(\vb F\) has zero circulation around all closed loops.
\end{proposition}
\noindent Note that the backward implication is trivial.
If \(\vb F\) has zero circulation around all loops, we can define that loop to be the boundary of some surface, and so the integral of the curl vanishes.
\begin{proof}
	Suppose that the result is false; there exists a unit vector \(\vu{k}\) such that \(\vu{k} \cdot \left( \curl{\vb F(\vb x_0)} \right) = \varepsilon > 0\) for some \(\vb x_0\).
	By continuity, for a sufficiently small \(\delta > 0\),
	\[
		\vu{k} \cdot \left( \curl{\vb F(\vb x_0)} \right) > \frac{1}{2}\varepsilon;\quad \text{for }\abs{\vb x - \vb x_0} < \delta
	\]
	Now, we can take a loop in this ball \(\{ \vb x \colon \abs{\vb x - \vb x_0} < \delta \}\) that lies entirely in a plane with normal \(\vu{k}\).
	Let this small loop's enclosed surface be \(S\), with boundary \(\partial S\).
	Then
	\[
		0 = \oint_{\partial S} \vb F \cdot \dd{\vb x} = \int_S \curl{\vb F} \cdot \vu k \dd{S} > \frac{1}{2}\varepsilon \int \dd{S} > 0
	\]
	which is a contradiction.
\end{proof}

\subsection{Intuition for curl as infinitesimal circulation}
Let \(S_\varepsilon\) denote a region contained inside a disk of radius \(\varepsilon > 0\), centred at \(\vb x_0\) with normal \(\vu{k}\).
\begin{align*}
	\int_{S_\varepsilon} \curl{\vb F} \cdot \dd{\vb S} & = \int_{S_\varepsilon} \left( \curl{\vb F(\vb x)} - \curl{\vb F(\vb x_0)} \right) \cdot \dd{\vb S} + \int_{S_\varepsilon} \curl{\vb F(\vb x_0)} \cdot \vu k \dd{S}                                                                                        \\
	                                                   & = \underbrace{\int_{S_\varepsilon} \left( \curl{\vb F(\vb x)} - \curl{\vb F(\vb x_0)} \right) \cdot \dd{\vb S}}_{o(\mathrm{area}(S_\varepsilon))} + \curl{\vb F(\vb x_0)} \cdot \vu k \underbrace{\int_{S_\varepsilon} \dd{S}}_{\mathrm{area}(S_\varepsilon)} \\
\end{align*}
As \(\varepsilon\) shrinks, the first integral tends to zero faster than the second term.
Hence,
\[
	\int_{S_\varepsilon} \curl{\vb F} \cdot \dd{\vb S} =  \curl{\vb F(\vb x_0)} \cdot \vu k \cdot \mathrm{area}(S_\varepsilon) + o(\mathrm{area}(S_\varepsilon))
\]
We can then see, by Stokes' theorem, that
\[
	\curl{\vb F(\vb x_0)} \cdot \vu k = \lim_{\varepsilon \to 0} \frac{1}{\mathrm{area}(S_\varepsilon)} \oint_{\partial S_\varepsilon} \vb F \cdot \dd{\vb x}
\]
So the curl of \(\vb F\) at \(\vb x_0\) in the direction \(\vu k\) is the infinitesimal circulation around \(\vb x_0\), per unit area.

\subsection{Gauss' divergence theorem}
\begin{proposition}
	If \(\vb F(\vb x)\) is a continuously differentiable vector field, and \(V\) is a volume with a piecewise regular boundary \(\partial V\), then
	\[
		\int_V \div{\vb F} \dd{V} = \int_{\partial V} \vb F \cdot \dd{\vb S}
	\]
	where the normal of \(\partial V\) points out of \(V\).
\end{proposition}
\noindent There is also a two-dimensional version.
If \(D\) is a planar region with a piecewise smooth boundary \(\partial D\),
\[
	\int_D \div{\vb F} \dd{A} = \oint_{\partial D} \vb F \cdot \vb n \dd{s}
\]
where the \(\dd{s}\) represents arc length, and where \(\vb n\) points out of \(D\).

\begin{example}
	Let \(V\) be a cylinder, defined in cylindrical polar coordinates \((\rho, \phi, z)\) as
	\[
		V = \left\{ (\rho, \phi, z) \colon 0 \leq \rho \leq R, -h \leq z \leq h, 0 \leq \phi < 2 \pi \right\}
	\]
	Let us label the boundary on the top \(S_+\), the boundary on the bottom \(S_-\), and the rest of the boundary \(S_R\):
	\begin{align*}
		S_\pm & = \{ (\rho, \phi, z) \colon 0 \leq \rho \leq R, z = \pm h, 0 \leq \phi < 2 \pi \} \\
		S_R   & = \{ (\rho, \phi, z) \colon \rho = R, -h \leq z \leq h, 0 \leq \phi < 2 \pi \}    \\
	\end{align*}
	Consider \(\vb F(\vb x) = \vb x\), hence \(\div{\vb F} = 3\).
	\[
		\int_V \div{\vb F} \dd{V} = 3 \int_V \dd{V} = 6\pi R^2 h
	\]
	Using instead the divergence theorem,
	\[
		\int_V \div{\vb F} \dd{V} = \int_{\partial V} \vb F \cdot \dd{\vb S}
	\]
	On \(S_R\), \(\dd{\vb S} = \vb e_\rho R \dd{\phi}\dd{z}\), and \(\vb x \cdot \vb e_\rho = R\).
	So we have the flux integral
	\[
		\int_{S_R} \vb F \cdot \dd{\vb S} = \int_{z = -h}^h \int_{\phi = 0}^{2 \pi} R^2 \dd{\phi} \dd{z} = 4\pi R^2 h
	\]
	On \(S_\pm\), \(\dd{\vb S} = \pm \vb e_z \rho \dd{\rho}\dd{\phi}\), and \(\vb x \cdot \vb e_z = \pm h\).
	Hence
	\[
		\int_{S_\pm} \vb F \cdot \dd{\vb S} = \int_{\phi = 0}^{2 \pi} \int_{\rho = 0}^R h \rho\dd{\rho}\dd{\phi} = \pi R^2 h
	\]
	The total is \(6\pi R^2 h\) as expected.
\end{example}

\begin{proposition}
	If \(\vb F\) is continuously differentiable, and for every closed surface \(S\) we have
	\[
		\int_S \vb F \cdot \dd{\vb S} = 0
	\]
	then \(\div{\vb F} = 0\).
\end{proposition}
\begin{proof}
	Suppose that the result is false; \(\div{\vb F}(\vb x_0) = \varepsilon > 0\).
	By continuity, for some sufficiently small \(\delta > 0\) we have
	\[
		\div{\vb F}(\vb x) > \frac{1}{2} \varepsilon \text{ for } \abs{\vb x_0 - \vb x} < \delta
	\]
	Now, we can choose a volume \(V\) inside the ball \(\abs{\vb x_0 - \vb x} < \delta\), and then by assumption, applying the divergence theorem,
	\[
		0 = \int_{\partial V} \vb F \cdot \dd{\vb S} = \int_V \div{\vb F} \dd{V} > 0
	\]
	which is a contradiction.
	We then conclude that if the vector field has zero net flux through any closed surface, it is solenoidal.
\end{proof}

\subsection{Intuition for divergence as infinitesimal flux}
Let \(V_\varepsilon\) be a volume in \(\mathbb R^3\), contained inside a ball of radius \(\varepsilon > 0\), centred at a point \(\vb x_0\).
Then
\[
	\int_{V_\varepsilon} \div{\vb F} \dd{V} = \mathrm{vol}(V_\varepsilon) \div{\vb F}(\vb x_0) + \underbrace{\int_{V_\varepsilon}  \left[ \div{\vb F}(\vb x) - \div{\vb F}(\vb x_0) \right] \dd{V}}_{o(\mathrm{vol}(V_\varepsilon))}
\]
Dividing both sides by the volume of \(V_\varepsilon\), and taking \(\varepsilon \to 0\), we can apply the divergence theorem to get
\[
	\div{\vb F}(\vb x_0) = \lim_{\varepsilon \to 0} \frac{1}{\mathrm{vol}(V_\varepsilon)} \int_{\partial V_\varepsilon} \vb F \cdot \dd{\vb S}
\]
The divergence of \(\vb F\) measures the infinitesimal flux per unit volume.
If the flux is moving `outward' at this point, \(\div{\vb F} > 0\), and vice versa.

\subsection{Conservation laws}
Many equations in mathematical physics can be represented using density \(\rho(\vb x, t)\) and a vector field \(\vb J(\vb x, t)\), as follows.
\begin{equation}
	\pdv{\rho}{t} + \div{\vb J} = 0 \tag{\(\dagger\)}
\end{equation}
This kind of equation is called a `conservation law'.
Suppose both \(\rho\) and \(\abs{\vb J}\) decrease rapidly as \(\abs{\vb x} \to \infty\).
We will define the charge \(Q\) by
\[
	Q = \int_{\mathbb R^3} \rho(\vb x, t) \dd{V}
\]
We have conservation of charge;
\begin{align*}
	\dv{Q}{t} & = \int_{\mathbb R^3} \pdv{\rho}{t} \dd{V}                            \\
	          & = -\int_{\mathbb R^3} \div{\vb J} \dd{V}                             \\
	          & = -\lim_{R \to \infty} \int_{\abs{\vb x} \leq R} \div{\vb J} \dd{V}  \\
	          & = -\lim_{R \to \infty} \int_{\abs{\vb x} = R} \vb J \cdot \dd{\vb S} \\
	          & = 0                                                                  \\
\end{align*}
as \(\vb J\) decreases rapidly to zero as \(\abs{\vb x} \to \infty\).
So \((\dagger)\) gives conservation of charge.

\subsection{Proof of divergence theorem}
\begin{proof}
	Suppose first that
	\[
		\vb F = F_z(x,y,z) \vb e_z
	\]
	The divergence theorem states that
	\begin{equation}
		\int_V \underbrace{\pdv{F_z}{z}}_{\div{\vb F}} \dd{V} = \int_{\partial V} F_z \vb e_z \cdot \dd{\vb S}
		\tag{\(\dagger\)}
	\end{equation}
	We would like to show that these two are really the same.
	First, let us simplify the problem to a convex volume \(V\), such that we can split the boundary into two halves, one with normals in the positive \(z\) direction (\(S_+\)) and one with normals in the negative \(z\) direction (\(S_-\)).
	Then \(\partial V = S_+ \cup S_-\).
	Project the volume into the \(x\)-\(y\) plane, and call this region \(A\).
	This planar region is then the shape of the `cut' between the \(S_+\) and \(S_-\) halves.
	We can write
	\[
		S_\pm = \left\{ \vb x(x, y) = \begin{pmatrix}
			x \\ y \\ g_\pm (x, y)
		\end{pmatrix} : (x, y) \in A \right\}
	\]
	We can then say
	\begin{align*}
		\int_V \pdv{F_z}{z} \dd{V} & = \iint_{A} \left[ \int_{z = g_- (x, y)}^{g_+ (x, y)} \pdv{F_z}{z} \dd{z} \right] \dd{x} \dd{y} \\
		                           & = \iint_A \left[ F_z(x, y, g_+ (x, y)) - F_z(x, y, g_- (x, y)) \right] \dd{x}\dd{y}
	\end{align*}
	To calculate right hand side of \((\dagger)\), we need \(\dd{\vb S}\):
	\begin{align*}
		\dd{\vb S} & = \pdv{\vb x}{x} \times \pdv{\vb x}{y} \dd{x}\dd{y} \\
		           & = \begin{pmatrix}
			               -\pdv*{g_\pm}{x} \\
			               -\pdv*{g_\pm}{y} \\
			               1                \\
		               \end{pmatrix} \dd{x}\dd{y}
	\end{align*}
	Since we want the normal to point `out' of \(V\), on \(S_\pm\) we have
	\[
		\eval{\dd{\vb S}}_{S_\pm} = \pm \begin{pmatrix}
			-\pdv*{g_\pm}{x} \\
			-\pdv*{g_\pm}{y} \\
			1                \\
		\end{pmatrix} \dd{x}\dd{y}
	\]
	Therefore,
	\begin{align*}
		\int_{\partial V} \vb F \cdot \dd{\vb S} & = \left[ \int_{S_+} + \int_{S_-} \right] F_z \vb e_z \cdot \dd{\vb S}                   \\
		                                         & = \iint_A F_z(x, y, g_+(x, y)) \dd{x}\dd{y} - \iint_A F_z(x, y, g_-(x, y)) \dd{x}\dd{y}
	\end{align*}
	which matches the expression we found for the left hand side of \((\dagger)\) above.
	In the same way, we can show that
	\begin{align*}
		\int_V \pdv{F_x}{x} \dd{V} & = \int_{\partial V} F_x \vb e_x \cdot \dd{\vb S} \\
		\int_V \pdv{F_y}{y} \dd{V} & = \int_{\partial V} F_y \vb e_y \cdot \dd{\vb S} \\
	\end{align*}
	and because the integrals are linear, we can compute their sum to find
	\[
		\int_V \div{\vb F} \dd{V} = \int_{\partial V} \vb F \cdot \dd{\vb S}
	\]
	which is exactly the divergence theorem.
\end{proof}

\subsection{Proof of Green's theorem}
We can use the two-dimensional divergence theorem to prove Green's theorem.
\begin{proof}
	Let
	\[
		\vb F = \begin{pmatrix}
			Q(x, y) \\ -P(x, y)
		\end{pmatrix}
	\]
	Then
	\[
		\iint_A \left( \pdv{Q}{x} - \pdv{P}{y} \right) \dd{x}\dd{y} = \int_A \div{\vb F} \dd{A} = \oint_{\partial A} \vb F \cdot \vb n \dd{s}
	\]
	If \(\partial A\) is parametrised with respect to arc length, this means that the unit tangent vector is
	\[
		\vb t = \begin{pmatrix}
			x'(s) \\ y'(s)
		\end{pmatrix}
	\]
	then the normal vector is
	\[
		\vb n = \begin{pmatrix}
			y'(s) \\ -x'(s)
		\end{pmatrix}
	\]
	Therefore,
	\[
		\oint_{\partial A} \vb F \cdot \vb n \dd{s} = \oint_{\partial A} \begin{pmatrix}
			Q \\ -P
		\end{pmatrix} \cdot \begin{pmatrix}
			y'(s) \\ -x'(s)
		\end{pmatrix} \dd{s} = \oint_{\partial A} P \dv{x}{s} \dd{s} + Q \dv{y}{s} \dd{s} = \oint_{\partial A} P \dd{x} + Q \dd{y}
	\]
\end{proof}

\subsection{Proof of Stokes' theorem}
We can now use Green's theorem to derive Stokes' theorem.
\begin{proof}
	Consider a regular surface
	\[
		S = \left\{ \vb x = \vb x(u, v) \colon (u, v) \in A \right\}
	\]
	Then the boundary is
	\[
		\partial S = \left\{ \vb x = \vb x(u, v) \colon (u, v) \in \partial A \right\}
	\]
	Green's theorem gives
	\begin{equation}
		\oint_{\partial A} P \dd{u} + Q \dd{v} = \iint_A \left( \pdv{Q}{u} - \pdv{P}{v} \right) \dd{u}\dd{v}
		\tag{\(\dagger\)}
	\end{equation}
	We will now set
	\[
		P(u, v) = \vb F(\vb x(u, v)) \cdot \pdv{\vb x}{u};\quad Q(u, v) = \vb F(\vb x(u, v)) \cdot \pdv{\vb x}{v}
	\]
	Then
	\[
		P \dd{u} + Q \dd{v} = \vb F(\vb x(u, v)) \cdot \left( \pdv{\vb x}{u} \dd{u} + \pdv{\vb v} \dd{v} \right) = \vb F(\vb x(u, v)) \cdot \dd{\vb x(u, v)}
	\]
	And so we can compute the left hand side of \((\dagger)\):
	\[
		\oint_{\partial A} P \dd{u} + Q \dd{v} = \oint_{\partial S} \vb F \cdot \dd{\vb x}
	\]
	For the right hand side, we must first compute some derivatives.
	\[
		Q = F_i(\vb x(u, v))\pdv{x_i}{v} \implies \pdv{Q}{u} = \pdv{x_j}{u}\pdv{F_i}{x_j}\pdv{x_i}{v} + F_i \pdv{x_i}{u}{v}
	\]
	\[
		P = F_i(\vb x(u, v))\pdv{x_i}{u} \implies \pdv{Q}{v} = \pdv{x_j}{v}\pdv{F_i}{x_j}\pdv{x_i}{u} + F_i \pdv{x_i}{v}{u}
	\]
	Hence
	\begin{align*}
		\pdv{Q}{u} - \pdv{P}{v} & = \left( \pdv{x_i}{v}\pdv{x_j}{u} - \pdv{x_i}{u}\pdv{x_j}{v} \right) \pdv{F_i}{x_j}                       \\
		                        & = \left( \delta_{ip} \delta_{jq} - \delta_{iq} \delta_{jp} \right) \pdv{F_i}{x_j}\pdv{x_p}{v}\pdv{x_q}{u} \\
		                        & = \varepsilon_{ijk} \varepsilon_{pqk} \pdv{F_i}{x_j}\pdv{x_p}{v}\pdv{x_q}{u}                              \\
		                        & = \left[ -\curl{\vb F} \right]_k \left( -\pdv{\vb x}{u} \times \pdv{\vb x}{v} \right)_k                   \\
		                        & = \left( \curl{\vb F} \right) \cdot \left( \pdv{\vb x}{u} \times \pdv{\vb x}{v} \right)                   \\
	\end{align*}
	Therefore,
	\[
		\iint_{A} \left( \pdv{Q}{u} - \pdv{P}{v} \right) \dd{u}\dd{v} = \iint_{A} \left( \curl{\vb F} \right) \cdot \left( \pdv{\vb x}{u} \times \pdv{\vb x}{v} \right) = \iint_{S} (\curl{\vb F}) \cdot \dd{\vb S}
	\]
	which gives Stokes' theorem as required.
\end{proof}
