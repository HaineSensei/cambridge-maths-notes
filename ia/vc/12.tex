\begin{proposition}
	If \(\vb F(\vb x)\) is a continuously differentiable vector field, and \(V\) is a volume with a piecewise regular boundary \(\partial V\), then
	\[
		\int_V \div{\vb F} \dd{V} = \int_{\partial V} \vb F \cdot \dd{\vb S}
	\]
	where the normal of \(\partial V\) points out of \(V\).
\end{proposition}
\noindent There is also a two-dimensional version.
If \(D\) is a planar region with a piecewise smooth boundary \(\partial D\),
\[
	\int_D \div{\vb F} \dd{A} = \oint_{\partial D} \vb F \cdot \vb n \dd{s}
\]
where the \(\dd{s}\) represents arc length, and where \(\vb n\) points out of \(D\).

\subsection{First Example}
Let \(V\) be a cylinder, defined in cylindrical polar coordinates \((\rho, \phi, z)\) as
\[
	V = \left\{ (\rho, \phi, z) \colon 0 \leq \rho \leq R, -h \leq z \leq h, 0 \leq \phi < 2 \pi \right\}
\]
Let us label the boundary on the top \(S_+\), the boundary on the bottom \(S_-\), and the rest of the boundary \(S_R\):
\begin{align*}
	S_\pm & = \{ (\rho, \phi, z) \colon 0 \leq \rho \leq R, z = \pm h, 0 \leq \phi < 2 \pi \} \\
	S_R   & = \{ (\rho, \phi, z) \colon \rho = R, -h \leq z \leq h, 0 \leq \phi < 2 \pi \}    \\
\end{align*}
Consider \(\vb F(\vb x) = \vb x\), hence \(\div{\vb F} = 3\).
\[
	\int_V \div{\vb F} \dd{V} = 3 \int_V \dd{V} = 6\pi R^2 h
\]
Using instead the divergence theorem,
\[
	\int_V \div{\vb F} \dd{V} = \int_{\partial V} \vb F \cdot \dd{\vb S}
\]
On \(S_R\), \(\dd{\vb S} = \vb e_\rho R \dd{\phi}\dd{z}\), and \(\vb x \cdot \vb e_\rho = R\).
So we have the flux integral
\[
	\int_{S_R} \vb F \cdot \dd{\vb S} = \int_{z = -h}^h \int_{\phi = 0}^{2 \pi} R^2 \dd{\phi} \dd{z} = 4\pi R^2 h
\]
On \(S_\pm\), \(\dd{\vb S} = \pm \vb e_z \rho \dd{\rho}\dd{\phi}\), and \(\vb x \cdot \vb e_z = \pm h\).
Hence
\[
	\int_{S_\pm} \vb F \cdot \dd{\vb S} = \int_{\phi = 0}^{2 \pi} \int_{\rho = 0}^R h \rho\dd{\rho}\dd{\phi} = \pi R^2 h
\]
The total is \(6\pi R^2 h\) as expected.

\subsection{Basic Corollary}
\begin{proposition}
	If \(\vb F\) is continuously differentiable, and for every closed surface \(S\) we have
	\[
		\int_S \vb F \cdot \dd{\vb S} = 0
	\]
	then \(\div{\vb F} = 0\).
\end{proposition}
\begin{proof}
	Suppose that the result is false; \(\div{\vb F}(\vb x_0) = \varepsilon > 0\).
	By continuity, for some sufficiently small \(\delta > 0\) we have
	\[
		\div{\vb F}(\vb x) > \frac{1}{2} \varepsilon \text{ for } \abs{\vb x_0 - \vb x} < \delta
	\]
	Now, we can choose a volume \(V\) inside the ball \(\abs{\vb x_0 - \vb x} < \delta\), and then by assumption, applying the divergence theorem,
	\[
		0 = \int_{\partial V} \vb F \cdot \dd{\vb S} = \int_V \div{\vb F} \dd{V} > 0
	\]
	which is a contradiction.
	We then conclude that if the vector field has zero net flux through any closed surface, it is solenoidal.
\end{proof}

\subsection{Intuition for Divergence as Infinitesimal Flux}
Let \(V_\varepsilon\) be a volume in \(\mathbb R^3\), contained inside a ball of radius \(\varepsilon > 0\), centred at a point \(\vb x_0\).
Then
\[
	\int_{V_\varepsilon} \div{\vb F} \dd{V} = \mathrm{vol}(V_\varepsilon) \div{\vb F}(\vb x_0) + \underbrace{\int_{V_\varepsilon}  \left[ \div{\vb F}(\vb x) - \div{\vb F}(\vb x_0) \right] \dd{V}}_{o(\mathrm{vol}(V_\varepsilon))}
\]
Dividing both sides by the volume of \(V_\varepsilon\), and taking \(\varepsilon \to 0\), we can apply the divergence theorem to get
\[
	\div{\vb F}(\vb x_0) = \lim_{\varepsilon \to 0} \frac{1}{\mathrm{vol}(V_\varepsilon)} \int_{\partial V_\varepsilon} \vb F \cdot \dd{\vb S}
\]
The divergence of \(\vb F\) measures the infinitesimal flux per unit volume.
If the flux is moving `outward' at this point, \(\div{\vb F} > 0\), and vice versa.

\subsection{Example of Conservation Laws}
Many equations in mathematical physics can be represented using density \(\rho(\vb x, t)\) and a vector field \(\vb J(\vb x, t)\), as follows.
\begin{equation}
	\pdv{\rho}{t} + \div{\vb J} = 0 \tag{\(\dagger\)}
\end{equation}
This kind of equation is called a `conservation law'.
Suppose both \(\rho\) and \(\abs{\vb J}\) decrease rapidly as \(\abs{\vb x} \to \infty\).
We will define the charge \(Q\) by
\[
	Q = \int_{\mathbb R^3} \rho(\vb x, t) \dd{V}
\]
We have conservation of charge;
\begin{align*}
	\dv{Q}{t} & = \int_{\mathbb R^3} \pdv{\rho}{t} \dd{V}                            \\
	          & = -\int_{\mathbb R^3} \div{\vb J} \dd{V}                             \\
	          & = -\lim_{R \to \infty} \int_{\abs{\vb x} \leq R} \div{\vb J} \dd{V}  \\
	          & = -\lim_{R \to \infty} \int_{\abs{\vb x} = R} \vb J \cdot \dd{\vb S} \\
	          & = 0                                                                  \\
\end{align*}
as \(\vb J\) decreases rapidly to zero as \(\abs{\vb x} \to \infty\).
So \((\dagger)\) gives conservation of charge.
