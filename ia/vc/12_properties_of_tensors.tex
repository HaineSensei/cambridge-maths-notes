\subsection{Symmetry and antisymmetry}
Observe for a rank 2 tensor that
\[
	T_{ij} = \frac{1}{2}\left( T_{ij} + T_{ji} \right) + \frac{1}{2}\left( T_{ij} - T{ji} \right) \equiv S_{ij} + A_{ij}
\]
where the \(S_{ij}\) are the symmetric components, and the \(A_{ij}\) are the antisymmetric components of the tensor.
Note that the symmetric part \(S_{ij}\) has six independent components (the main diagonal and everything above it), and the antisymmetric part \(A_{ij}\) has three independent components (everything above the main diagonal) since the main diagonal is zero.
So the number of independent components of the symmetric part and the antisymmetric part add up to the number of independent components of a general rank 2 tensor in \(\mathbb R^3\) (nine).
Intuitively, we might think that the information contained in \(A_{ij}\) could be represented as some vector, since it has the same amount of independent components.
\begin{proposition}
	Every rank 2 tensor \(T_{ij}\) can be decomposed uniquely into
	\[
		T_{ij} = S_{ij} + \varepsilon_{ijk} \omega_k
	\]
	where
	\[
		\omega_i = \frac{1}{2}\varepsilon_{ijk} T_{jk}
	\]
	and \(S_{ij}\) is symmetric.
\end{proposition}
\begin{proof}
	From above, we can find \(S_{ij} = \frac{1}{2}\left( T_{ij} + T_{ji} \right)\).
	We now just need to show that
	\[
		\varepsilon_{ijk} \omega_k = \frac{1}{2}\left( T_{ij} - T{ji} \right)
	\]
	We can see that
	\begin{align*}
		\varepsilon_{ijk} \omega_k & = \frac{1}{2}\varepsilon_{ijk} \varepsilon_{k\ell m} T_{\ell m}                      \\
		                           & = \frac{1}{2}\qty(\delta_{i\ell} \delta_{jm} - \delta_{im}\delta_{j\ell}) T_{\ell m} \\
		                           & = \frac{1}{2}\qty(T_{ij} - T_{ji})
	\end{align*}
	To show uniqueness, we now suppose that
	\[
		T_{ij} = S_{ij} + A_{ij} = \widetilde{S}_{ij} + \widetilde{A}_{ij} = \widetilde{T}_{ij}
	\]
	If we take the symmetric part of both sides (i.e.\ \(T_{ij} + T_{ji} = \widetilde{T}_{ij} + \widetilde{T}_{ji}\)), we get \(S_{ij} = \widetilde{S}_{ij}\).
	Likewise, we have \(A_{ij} = \widetilde{A}_{ij}\) by eliminating the equal symmetric parts.
\end{proof}
\noindent As an example, consider an elastic body.
Each point \(\vb x\) in such a body will undergo a small displacement \(\vb u(\vb x)\) when applied to some force.
Consider nearby points \(\vb x + \delta\vb x\) and \(\vb x\) that were initially separated by \(\delta x\).
They will become separated by
\[
	(\vb x + \delta \vb x + \vb u(\vb x + \delta\vb x)) - (\vb x + \vb u(\vb x)) = \delta \vb x + \vb u(\vb x + \delta\vb x) - \vb u(\vb x)
\]
So the change in displacement is
\[
	\vb u(\vb x + \delta\vb x) - \vb u(\vb x)
\]
This value gives us an idea of how much deformation the body is subjected to.
Assuming this is a smooth deformation, we have
\[
	u_i(\vb x + \delta\vb x) - u_i(\vb x) = \pdv{u_i}{x_j}\delta_{x_j} + o(\delta \vb x)
\]
We then decompose \(\pdv{u_i}{x_j}\) as follows.
\[
	\pdv{u_i}{x_j} = e_{ij} + \varepsilon_{ijk}\omega_k
\]
where the \(e_{ij}\) is the symmetric part, and the \(\varepsilon_{ijk}\omega_k\) is the antisymmetric part.
In particular,
\[
	e_{ij} = \frac{1}{2}\qty(\pdv{u_i}{x_j} + \pdv{u_j}{x_i})
\]
is called the linear strain tensor.
Considering the other tensor,
\[
	\omega_k = \frac{1}{2}\varepsilon_{ijk}\pdv{u_j}{x_k} = \frac{-1}{2}\qty(\curl{\vb u})_i
\]
Then,
\[
	u_i(\vb x + \delta \vb x) - u_i(\vb x) = e_{ij} \delta x_j + [\delta \vb x \times \bm \omega]_i + o(\delta \vb x)
\]
So the antisymmetric part corresponds to a rotation, and is irrelevant for describing the deformation of the internals of the body.
So by separating the symmetric and antisymmetric parts, we can in fact remove the antisymmetric part from the equation in order to study just the linear strain.

\begin{example}
	As another example, let us consider the inertia tensor, which is a common rank 2 tensor.
	Suppose a body with density \(\rho(\vb x)\) occupies a volume \(V \subset \mathbb R^3\), where each point in the body is rotating with constant angular velocity \(\bm\omega\) about an axis through the origin.
	The velocity of a point \(\vb x \in V\) is given by \(\vb v = \bm \omega \times \vb x\).
	Hence, the total angular momentum is
	\begin{align*}
		\vb L & = \int_V \rho(\vb x) (\vb x \times \vb v) \dd{V}                                         \\
		      & = \int_V \rho(\vb x) (\vb x \times (\bm \omega \times \vb x)) \dd{V}                     \\
		L_i   & = \int_{\mathcal V} \rho(\vb x) (x_k x_k \omega_i - x_i x_j \omega_j) \dd{V}             \\
		      & = \int_{\mathcal V} \rho(\vb x) (x_k x_k \delta_{ij} \omega_j - x_i x_j \omega_j) \dd{V} \\
		      & = I_{ij}\omega_j
	\end{align*}
	where \(I_{ij}\) is the inertia tensor defined by
	\[
		I_{ij} = \int_{\mathcal V} \rho(\vb x) (x_k x_k \delta_{ij} - x_i x_j) \dd{V}
	\]
	and where
	\[
		\mathcal V = \qty{ x_i \colon x_i \vb e_i \in V }
	\]
	If we had used a different basis, we would have found
	\begin{align*}
		I_{ij}' & = \int_{\mathcal V'} \rho(\vb x) (x_k' x_k' \delta_{ij} - x_i' x_j') \dd{V}          \\
		        & = R_{ip} R_{jq} \int_{\mathcal V} \rho(\vb x) (x_k x_k \delta_{pq} - x_p x_q) \dd{V} \\
		        & = R_{ip} R_{jq} I_{pq}
	\end{align*}
	So it really is a rank 2 tensor.
	As an example, consider the ellipsoid
	\[
		V = \qty{ \vb x \colon \frac{x_1^2}{a^2} + \frac{x_2^2}{b^2} + \frac{x_3^2}{c^2} \leq 1 }
	\]
	with uniform density \(\rho_0\).
	Then the mass is given by
	\[
		M = \frac{4}{3}\pi \rho_0 abc
	\]
	Then the inertia tensor with respect to this set of basis vectors is given by
	\[
		I_{ij} = \int_{\mathcal V} \rho(\vb x) (x_k x_k \delta_{ij} - x_i x_j) \dd{V}
	\]
	To help with these integrals, we make the following parametrisation into scaled spherical coordinates:
	\[
		\left\{ \begin{array}{l}
			x_1 = ar\cos\phi\sin\theta \\
			x_2 = br\sin\phi\sin\theta \\
			x_3 = cr\cos\theta
		\end{array} \right.\quad \phi\in[0, 2\pi),\theta\in[0, \pi], r \in[0, 1]
	\]
	Note that if \(i \neq j\), then by symmetry we have
	\[
		\int_V \rho_0 x_i x_j \dd{V} = 0
	\]
	Further,
	\begin{align*}
		I_{11} & = \rho_0 \int_V x_2^2 + x_3^2 \dd{V}                                                                                                                                  \\
		       & = \rho_0 abc \int_{\phi = 0}^{2\pi} \dd{\phi} \int_{\theta = 0}^\pi \dd{\theta} \int_{r = 0}^1 \dd{r} r^2 (b^2\sin^2\phi\sin^2\theta + c^2\cos^2\theta) r^2\sin\theta \\
		       & = \rho_0 \frac{abc}{5} \int_0^\pi (\pi b^2 \sin^2 \theta + 2\pi c^2\cos^2\theta) \sin\theta \dd{\theta}                                                               \\
		       & = \frac{3M}{20} \int_0^\pi (b^2 \sin^2 \theta + (2c^2 - b^2)\cos^2\theta\sin\theta) \dd{\theta}                                                                       \\
		       & = \frac{3M}{20} \qty(2b^2 + \frac{2}{3}\qty(2c^2 - b^2))                                                                                                              \\
		       & = \frac{M}{5} (b^2 + c^2)
	\end{align*}
	So by symmetry,
	\[
		I_{22} = \frac{M}{5} (a^2 + c^2);\quad I_{33} = \frac{M}{5} (a^2 + b^2)
	\]
	Hence,
	\[
		I_{ij} = \frac{M}{5} \begin{pmatrix}
			b^2 + c^2 & 0         & 0         \\
			0         & a^2 + c^2 & 0         \\
			0         & 0         & a^2 + b^2
		\end{pmatrix}
	\]
	In particular, if \(a=b=c\),
	\[
		I_{ij} = \frac{2M}{5}\delta_{ij}
	\]
\end{example}
\begin{proposition}
	If \(T_{ij}\) is symmetric, then there exists a basis \(\{ \vb e_i \}\) for which \(T_{ij}\) only has nonzero entries on the diagonal.
	The coordinate axes of this basis are called the principal axes of the tensor.
\end{proposition}
\begin{proof}
	Recall that for a real symmetric matrix \(M\), we can diagonalise it using an orthogonal transformation with determinant 1.
	The change of basis formula for a matrix is exactly that for a rank 2 tensor, so we can always choose such a change of basis to give a diagonal matrix.
\end{proof}

\subsection{Isotropic tensors}
\begin{definition}
	A tensor is isotropic if it is invariant under changes with respect to the choice of Cartesian coordinate axes.
	\[
		T_{ij\dots k}' = R_{ip} R_{jq} \dots R_{kr} T_{pq\dots r} = T_{ij\dots k}
	\]
	for any choice of rotation \(R\).
\end{definition}
\noindent Note that by definition, every scalar is isotropic.
The Kronecker and Levi-Civita tensors are also isotropic, as we saw above.

\subsection{Classifying isotropic tensors in three dimensions}
\begin{proposition}
	The isotropic tensors on \(\mathbb R^3\), ordered by rank, are exactly (up to the multiplication of a multiplicative scalar)
	\begin{enumerate}[label={Rank \arabic*:}, align=left, leftmargin=*]
		\setcounter{enumi}{-1}
		\item all tensors
		\item no nonzero tensors
		\item the Kronecker \(\delta\)
		\item the Levi-Civita \(\varepsilon\)
		\item \(\alpha \delta_{ij} \delta_{k\ell} + \beta \delta_{ik} \delta_{j\ell} + \gamma \delta_{i\ell} \delta_{jk}\) where \(\alpha\), \(\beta\), \(\gamma\) are scalars
	\end{enumerate}
	and for ranks higher than 4, they are a linear combination of products of \(\delta\) and \(\varepsilon\) terms, for instance \(\delta_{ij}\varepsilon_{k\ell m}\).
\end{proposition}
\begin{proof}
	This is a non-rigorous sketch proof.
	\begin{enumerate}[label={Rank \arabic*:}, align=left, leftmargin=*]
		\setcounter{enumi}{-1}
		\item By definition, such tensors do not transform components under a change of basis.
		\item Let \(v_i\) be the components of an isotropic vector of rank 1.
		      Then, for any \(R\), we must have
		      \[
			      v_i = R_{ij} v_j
		      \]
		      Let \(R\) be a rotation by \(\pi\) about the \(z\) axis, so
		      \[
			      R = \begin{pmatrix}
				      -1 & 0  & 0 \\
				      0  & -1 & 0 \\
				      0  & 0  & 1
			      \end{pmatrix}
		      \]
		      Hence,
		      \[
			      v_1 = -v_1;\quad v_2 = -v_2;\quad v_3 = v_3
		      \]
		      Hence, \(v_1 = 0, v_2 = 0\).
		      Alternatively, let
		      \[
			      R = \begin{pmatrix}
				      1 & 0  & 0  \\
				      0 & -1 & 0  \\
				      0 & 0  & -1
			      \end{pmatrix}
		      \]
		      Then clearly \(v_3 = -v_3 = 0\).
		      Hence the only tensor with this property is the zero tensor.
		\item If \(T_{ij}\) are the components of an isotropic tensor of rank 2, then for all choices of \(R\), we have
		      \[
			      T_{ij} = R_{ip} R_{jq} T_{pq}
		      \]
		      Let \(R\) be a rotation by \(\frac{\pi}{2}\) about each axis, so for example in the \(z\) direction,
		      \[
			      R = \begin{pmatrix}
				      0  & 1 & 0 \\
				      -1 & 0 & 0 \\
				      0  & 0 & 1
			      \end{pmatrix}
		      \]
		      So \(T_{13} = R_{1p} R_{3q} T_{pq} = R_{12} R_{33} T_{23} = T_{23}\).
		      Analogously we find, \(T_{23} = -T_{13}\).
		      Hence, \(T_{13} = T_{23} = 0\).
		      Further, \(T_{11} = R_{1p} R_{1q} T_{pq} = R_{12} R_{12} T_{22} = T_{22}\).
		      So by symmetry,
		      \[
			      T_{11} = T_{22} = T_{33};\quad T_{13} = T_{23} = T_{12} = T_{31} = T_{32} = T_{21} = 0
		      \]
		      which is exactly the \(\delta\) tensor, up to a scale factor.
		\item For rank 3 tensors, we can use the same idea, but with more indices.
	\end{enumerate}
\end{proof}

\subsection{Integrals with isotropic tensors}
Consider an integral of the form
\[
	T_{ij\dots k} = \int_{\abs{\vb x} < R} f(r) x_i x_j \dots x_k \dd{V}
\]
where \(x_k x_k = r^2\), and \(\dd{V(x)} = \dd{x_1} \dd{x_2} \dd{x_3}\).
Note that \(f(r)\) and \(\{ \vb x \colon \abs{\vb x} < R \}\) are invariant under rotation.
Since \(\abs{J}\) under a rotation is 1, we have
\begin{align*}
	T_{ij\dots k}' & = \int_{\abs{\vb x} < R} f(r) x_i' x_j' \dots x_k' \dd{x_1'} \dd{x_2'} \dd{x_3'}                \\
	               & = \int_{\abs{\vb x} < R} f(r) R_{ip} x_p R_{jq} x_q \dots R_{kr} x_r \dd{x_1} \dd{x_2} \dd{x_3}
\end{align*}
We will now make the substitution
\[
	y_i = R_{ij} x_j;\quad \dd{V} = \dd{y_1} \dd{y_2} \dd{y_3}
\]
Hence,
\begin{align*}
	T_{ij\dots k}' & = \int_{\abs{\vb x} < R} f(r) y_i y_j \dots y_k \dd{V(\vb y)} \\
	               & = \int_{\abs{\vb x} < R} f(r) x_i x_j \dots x_k \dd{V(\vb x)} \\
	               & = T_{ij\dots k}
\end{align*}
Hence such an integral always yields an isotropic tensor.
If we take \(R \to \infty\), this corresponds to an integral over \(\mathbb R^3\).
As an example, consider
\[
	T_{ij} = \int_{\mathbb R^3} e^{-r^5}x_i x_j \dd{V}
\]
Then \(T_{ij}\) is isotropic, hence \(T_{ij} = \alpha \delta_{ij}\).
Contracting on \((i, j)\) to find \(\alpha\), we get
\begin{align*}
	\alpha \delta_{ii} & = 3\alpha                                      \\
	                   & = \int_{\mathbb R^3} e^{-r^5} r^2 \dd{V}       \\
	                   & = 4\pi \int_{0}^\infty e^{-r^5} r^2 r^2 \dd{r} \\
	                   & = 4\pi \int_{0}^\infty e^{-r^5} r^4 \dd{r}     \\
	                   & = \frac{4}{5}\pi
\end{align*}
Hence,
\[
	T_{ij} = \frac{4}{15}\pi\delta_{ij}
\]
As another example, consider the inertia tensor \(I_{ij}\) of a ball of radius \(R\), uniform density \(\rho_0\), and mass \(M = \frac{4\pi}{3}R^3 \rho_0\).
Recall that
\[
	I_{ij} = \int_{\abs{\vb x} < R} \rho_0 (x_k x_k \delta_{ij} - x_i x_j) \dd{V}
\]
Both terms give an isotropic result, so the sum \(I_{ij}\) is isotropic.
Contracting on \((i, j)\), we have
\begin{align*}
	\alpha \delta_{ii} & = 3\alpha                                                                  \\
	                   & = \int_{\abs{\vb x} < R} \rho_0 (r^2 \delta_{ii} - x_i x_i) \dd{V}         \\
	                   & = \int_{\abs{\vb x} < R} \rho_0 (3r^2 - r^2) \dd{V}                        \\
	                   & = \int_{\abs{\vb x} < R} \rho_0 2r^2 \dd{V}                                \\
	                   & = 4\pi \int_0^R \rho_0 2r^4 \dd{r}                                         \\
	                   & = \frac{4\pi}{3}\rho_0 R^3 \qty(\frac{3}{R^3} \cdot 2 \cdot \frac{R^5}{5}) \\
	                   & = \frac{6MR^2}{5}
\end{align*}
Hence,
\[
	I_{ij} = \frac{2MR^2}{5}\delta_{ij}
\]

\subsection{Bilinear and multilinear maps as tensors}
For a tensor \(T_{ij}\), consider the bilinear map \(t \colon \mathbb R^3 \times \mathbb R^3 \to \mathbb R\) defined by
\[
	t(\vb a, \vb b) = T_{ij} a_i b_j
\]
The left hand side really is well defined, since the right hand side does not depend on the choice of basis vectors.
Conversely, suppose we have a bilinear map \(t\).
Then, for a given basis \(\{\vb e_i\}\), this defines an array \(T_{ij}\) by
\[
	t(\vb a, \vb b) = t(a_i \vb e_i, b_j \vb e_j) = a_i b_j t(\vb e_i, \vb e_j) = a_i b_j T_{ij}
\]
Changing basis with \(\vb e_i' = R_{ip} \vb e_p\), we find
\[
	T_{ij}' = t(\vb e_i', \vb e_j') = t(R_{ip} \vb e_p, R_{jq} \vb e_q) = R_{ip} R_{jq} t(\vb e_p, \vb e_q)
\]
hence this \(T_{ij}\) really is a rank 2 tensor.
So there is a bijection between bilinear maps and rank 2 tensors.
In particular, if the map
\[
	(\vb a, \vb b) \mapsto T_{ij} a_i b_j
\]
is a bilinear map, and independent of basis, then \(T_{ij}\) \textit{must} be the components of a rank 2 tensor.
The same proof applies for higher-rank tensors.

\subsection{Quotient theorem}
Recall from earlier that the conductivity tensor \(\sigma_{ij}\) satisfying \(J_i = \sigma_{ij} E_j\) was really a tensor, by using the definitions.
The quotient theorem allows us to deduce similar results more generally.
The name originates from the apparent `quotient' of \(J_i\) by \(E_j\) to give \(\sigma_{ij}\).
\begin{proposition}
	Let \(T_{i\dots j p\dots q}\) be an array of numbers defined in each Cartesian coordinate system, such that
	\[
		v_{i\dots j} = T_{i\dots j p \dots q} u_{p\dots q}
	\]
	and that \(v_{i\dots j}\) is a tensor for all tensors \(u_{p\dots q}\).
	Then \(T_{i\dots j p\dots q}\) is a tensor.
\end{proposition}
\begin{proof}
	We will first consider the special case \(u_{p\dots q} = c_p \dots d_q\) for vectors \(\vb c, \dots, \vb d\).
	Then by assumption,
	\[
		v_{i\dots j} = T_{i\dots j p \dots q} c_p \dots d_q
	\]
	is a tensor.
	In particular,
	\[
		v_{i\dots j} a_i \dots b_j = T_{i\dots j p \dots q} a_i \dots b_j c_p \dots d_q
	\]
	is a scalar, since the left hand side is just a contraction over all indices.
	Since the right hand side is invariant under a change in basis, this leads us to define the multilinear map
	\[
		t(\vb a, \dots, \vb b, \vb c, \dots, \vb d) = T_{i\dots j p \dots q} a_i \dots b_j c_p \dots d_q
	\]
	Hence \(T_{i\dots j p \dots q}\) really is a tensor.
\end{proof}
\noindent As an example, consider the linear strain tensor
\[
	e_{ij} = \frac{1}{2}\qty( \pdv{u_i}{x_j} + \pdv{u_j}{x_i} )
\]
where \(\vb u(\vb x)\) measures the change in displacement at \(\vb x\).
Experiments suggest that the internal stress tensor \(\sigma_{ij}\) experienced by a body under a deformation \(\vb u(\vb x)\) depends linearly on the strain \(e_{ij}\) at each point.
Hence we might assume that there exists some array \(c_{ijk\ell}\) such that
\[
	\sigma_{ij} = c_{ijk\ell} e_{k\ell}
\]
However, we can't actually apply the quotient theorem here, since \(e_{k\ell}\) cannot be \textit{any} tensor, it can only be any \textit{symmetric} tensor.
See Example Sheet 4 for the resolution of this apparent problem: if \(c_{ijk\ell} = c_{ij\ell k}\), then we can apply the quotient theorem.
We call \(c_{ijk\ell}\) the stiffness tensor, which is a property of the material being subjected to the force.
Suppose that the material is isotropic, then we might guess that \(c_{ijk\ell}\) should be isotropic.
Hence,
\[
	c_{ijk\ell} = \alpha \delta_{ij}\delta_{k\ell} + \beta \delta_{ik}\delta_{j\ell} + \gamma \delta_{i\ell}\delta_{jk}
\]
where \(\alpha, \beta, \gamma\) are scalars.
Putting this into the relationship between \(\sigma\) and \(e\), we find
\[
	\sigma_{ij} = \alpha \delta_{ij}e_{kk} + \beta e_{ij} + \gamma e_{ji} = \lambda \delta_{ij} e_{kk} + 2\mu e_{ij}
\]
which is a higher-dimensional analogue of Hooke's Law.
We can in fact invert this.
By contracting on \((i, j)\) we find
\[
	\sigma_{ii} = 3\lambda e_{ii} + 2\mu e_{ii}
\]
Hence,
\[
	e_{kk} = \frac{\sigma_{kk}}{3\lambda + 2\mu}
\]
We then have
\[
	\sigma_{ij} = \lambda \delta_{ij} \frac{\sigma_{kk}}{3\lambda + 2\mu} + 2\mu e_{ij} \implies 2\mu e_{ij} = \sigma_{ij} - \sigma_{kk} \delta_{ij} \frac{\lambda}{3\lambda + 2\mu}
\]
