\subsection{Definition}
\begin{definition}
	An object whose components \(T_{ij\dots k}\) transform according to
	\[
		T_{ij\dots k}' = R_{ip}R_{jq}\dots R_{kr} T_{pq\dots r}
	\]
	is called a (Cartesian) tensor of rank \(n\) if \(T\) has \(n\) indices, where \(R_{ij} = \vb e_i' \cdot \vb e_j\) are the components of an orthogonal matrix, so \(R_{ip} R_{jp} = \delta_{ij}\).
\end{definition}
\noindent For example, if \(u_i, v_j, w_k\) are the components of \(n\) vectors, then
\[
	T_{ij\dots k} = u_i v_j \dots w_k
\]
define the components of a tensor of rank \(n\).
\begin{proof}
	We can transform each vector individually.
	\[
		T_{ij\dots k}' = u_i' v_j' \dots w_k' = R_{ip}u_p R_{jq}v_q \dots R_{kr}w_r = R_{ip}R_{jq}R_{kr} T_{ij\dots k}
	\]
	as expected.
\end{proof}

\subsection{Kronecker \(\delta\) and Levi-Civita \(\varepsilon\)}
As another example, consider the Kronecker \(\delta\).
It was previously defined without reference to any basis by
\[
	\delta_{ij} = \begin{cases}
		1 & i = j    \\
		0 & i \neq j
	\end{cases}
\]
So \(\delta_{ij}' = \delta_{ij}\) by definition.
Note that
\[
	R_{ip}R_{jq} \delta_{pq} = R_{iq}R_{jq} = \delta_{ij} = \delta_{ij}'
\]
hence \(\delta\) transforms like a rank 2 tensor, so it is indeed a rank 2 tensor.
Now, consider the Levi-Civita symbol \(\varepsilon\).
It is defined without reference to any basis as
\[
	\varepsilon_{ijk} = \begin{cases}
		+1 & (i\ j\ k) \text{ even} \\
		-1 & (i\ j\ k) \text{ odd}  \\
		0  & \text{otherwise}
	\end{cases}
\]
Note that \(\varepsilon_{ijk}' = \varepsilon_{ijk}\), and
\[
	R_{ip}R_{jq}R_{kr} \varepsilon_{pqr} = \det R \cdot \varepsilon_{ijk} = \varepsilon_{ijk}
\]
Hence \(\varepsilon\) is a rank 3 tensor.

\subsection{Electrical Conductivity Tensor}
Experiments suggest that there is a linear relationship between the current \(\vb J\) produced in a conductive medium and the electric field \(\vb E\) that it is exposed to.
Hence \(\vb J = \sigma \vb E\), or \(J_i = \sigma_{ij} E_j\).
\(\sigma_{ij}\) is called the `electrical conductivity tensor'.
It really is a rank 2 tensor, indeed
\begin{align*}
	J_i'                 & = \sigma'_{ij} E_j' \\
	R_{ip}J_p            & = \sigma'_{ij} E_j' \\
	R_{ip}\sigma_{pq}E_q & = \sigma'_{ij} E_j'
\end{align*}
Since \(R\) is orthogonal,
\[
	E_j' = R_{jq}E_q \iff E_q = R_{jq}E_j'
\]
Hence,
\[
	R_{ip}R_{jq}\sigma_{pq}E_j' = \sigma_{ij}'E_j'
\]
Since this is true for all choices of \(E_j\),
\[
	R_{ip}R_{jq}\sigma_{pq} = \sigma_{ij}'
\]
So it really is a rank 2 tensor.

\subsection{Indexed Objects without Tensor Transformation Properties}
It is possible to construct objects with indices that do not transform as tensors.
For example, given a Cartesian right handed basis \(\{ \vb e_i \}\), we can define an arbitrary array of numbers with components \(A_{ij}\), and set \(A_{ij}' = 0\) in all other bases \(\{ \vb e_i' \}\).
Clearly this array of numbers does not transform like a tensor.

\subsection{Operations on Tensors}
Let \(A_{ij\dots k}, B_{ij\dots k}\) be rank \(n\) tensors, we define
\[
	(A + B)_{ij \dots k} = A_{ij\dots k} + B_{ij\dots k}
\]
\(A + B\) is also a rank \(n\) tensor, by linearity.
Further,
\[
	(\alpha A)_{ij\dots k} = \alpha A_{ij\dots k}
\]
\(\alpha A\) is also a rank \(n\) tensor.
We also define the \textit{tensor product} between a rank \(m\) tensor \(U_{ij\dots k}\) and a rank \(n\) tensor \(V_{pq \dots r}\) as
\[
	(U \otimes V)_{ij\dots kpq\dots r} = U_{ij\dots k}V_{pq \dots r}
\]
Now, \(U \otimes V\) is a rank \(m+n\) tensor.
Indeed,
\[
	(U \otimes V)_{i \dots j p\dots q} = U_{i\dots j}' V_{p\dots q}' = R_{ia} \dots R_{jb} U_{a \dots b} R_{pc} \dots R_{qd} V_{c \dots d} = R_{ia} \dots R_{jb} R_{pc} \dots R_{qd} (U \otimes V)_{a \dots b c\dots d}
\]
Further, given a rank \(n\geq 2\) tensor \(T_{ijk\dots \ell}\), we can define a tensor of rank \(n-2\) by \textit{contracting} on a pair of indices.
For instance, contracting on \(i\) and \(j\) is defined by
\[
	\delta_{ij} T_{ijk\dots \ell} = T_{iik\dots \ell}
\]
This is really a tensor of rank \(n-2\):
\[
	T'_{iik\dots \ell} = R_{ip}R_{iq}R_{kr}\dots R_{\ell s} T_{pqr\dots s} = \delta_{pq}R_{kr}\dots R_{\ell s} T_{pqr\dots s} = R_{kr}\dots R_{\ell s} T_{ppr\dots s}
\]
