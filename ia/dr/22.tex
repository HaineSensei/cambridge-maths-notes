\subsection{Proper Time}
A particle moves along a trajectory $\vb x(t)$. The velocity of this particle is $\dv{\vb x}{t} = \vb u(t)$. The path in spacetime is parametrised by $t$. Both $\vb x$ and $t$ vary under a Lorentz transformation. Now, consider a particle at rest in $S'$ with $\vb x' = 0$. The invariant interval on the world line is
\[ \Delta s^2 = c^2 \Delta t^2 \]
We define the proper time $\tau$ as
\[ \Delta \tau = \frac{1}{c}\Delta s \]
In particular, in $S'$, $\Delta\tau = \Delta t$, so the proper time is the time experienced in the rest frame of the particle. However, the equation $\Delta \tau = \frac{1}{c}\Delta s$ holds in all frames, since $\Delta s$ is Lorentz invariant. Note further that $\Delta s$ is real since this always represents a timelike interval, as it represents a particle travelling through spacetime. We can therefore instead parametrise this particle's world line by its proper time, rather than by considering the time in any particular frame. So $\vb x$ and $t$ are both functions of $\tau$ in any given reference frame. Further, infinitesimal changes are related by
\begin{align*}
	\dd{\tau}               & = \frac{\dd{s}}{c}                                       \\
	                        & = \frac{1}{c}\sqrt{c^2\dd{t}^2 - \abs{\dd{\vb x}}^2}     \\
	                        & = \frac{1}{c}\sqrt{c^2\dd{t}^2 - \abs{\vb u}^2 \dd{t}^2} \\
	                        & = \qty(1 - \frac{\vb u^2}{c^2})^{\frac{1}{2}}\dd{t}      \\
	\therefore \dv{t}{\tau} & = \gamma_{\vb u}
\end{align*}
where $\gamma_{\vb u} = \qty(1 - \frac{\vb u^2}{c^2})^{\frac{1}{2}}$. Now, the total time observed by a particle moving along its world line is
\[ T = \int \dd{\tau} = \int \frac{\dd{t}}{\gamma_{\vb u}} \]

\subsection{4-Velocity}
We can parametrise the position 4-vector of a particle using $\tau$, written
\[ X(\tau) = \begin{pmatrix}
		ct(\tau) \\ \vb x(\tau)
	\end{pmatrix} \]
We define the 4-velocity as
\[ U = \dv{\tau}X = \begin{pmatrix}
		c\dv*{t}{\tau} \\ \dv*{\vb x}{\tau}
	\end{pmatrix} = \dv{t}{\tau} \begin{pmatrix}
		c \\ \vb u
	\end{pmatrix} = \gamma_{\vb u} \begin{pmatrix}
		c \\ \vb u
	\end{pmatrix} \]
Since $X' = \Lambda X$, we also have that
\[ U' = \Lambda U \]
because $\tau$ is invariant. Note that any quantity whose components transform according to this rule is called a 4-vector, and in particular, the derivative of a 4-vector with respect to an invariant is also a 4-vector. Also, the scalar product $U \cdot U$ is invariant under Lorentz transforms. Indeed, in the rest frame of a particle moving with 4-velocity $U$, in this frame we have $U = c^2$. In other frames,
\[ U \cdot U = \gamma^2 (c^2 - \vb u^2) = c^2 \]
as expected.

\subsection{Transformation of Velocities}
We have found that in special relativity, we cannot simply add velocities together. Consider a transformation $\Lambda$ from $S$ to $S'$, where $S'$ is moving (relative to $S$) at a speed $v$ in the $x$ direction. Consider a particle moving in $S$ at speed $u$ at an angle $\theta$ to the $x$ axis (with no component in the $z$ axis). In $S'$, it moves with speed $u'$ at an angle $\theta'$. We can write the 4-velocities as
\[ U = \begin{pmatrix}
		\gamma_{\vb u}c           \\
		\gamma_{\vb u}u\cos\theta \\
		\gamma_{\vb u}u\sin\theta \\
		0
	\end{pmatrix};\quad U' = \begin{pmatrix}
		\gamma_{\vb u'}c             \\
		\gamma_{\vb u'}u'\cos\theta' \\
		\gamma_{\vb u'}u'\sin\theta' \\
		0
	\end{pmatrix} \]
and further,
\[ U' = \Lambda U \]
where
\[ \Lambda = \begin{pmatrix}
		\gamma_v              & -\gamma_v \frac{v}{c} & 0 & 0 \\
		-\gamma_v \frac{v}{c} & \gamma_v              & 0 & 0 \\
		0                     & 0                     & 1 & 0 \\
		0                     & 0                     & 0 & 1
	\end{pmatrix} \]
Carrying out the matrix multiplication, we find
\[ \begin{pmatrix}
		\gamma_{\vb u'}c             \\
		\gamma_{\vb u'}u'\cos\theta' \\
		\gamma_{\vb u'}u'\sin\theta' \\
		0
	\end{pmatrix} = \begin{pmatrix}
		\gamma_v              & -\gamma_v \frac{v}{c} & 0 & 0 \\
		-\gamma_v \frac{v}{c} & \gamma_v              & 0 & 0 \\
		0                     & 0                     & 1 & 0 \\
		0                     & 0                     & 0 & 1
	\end{pmatrix} \begin{pmatrix}
		\gamma_{\vb u}c           \\
		\gamma_{\vb u}u\cos\theta \\
		\gamma_{\vb u}u\sin\theta \\
		0
	\end{pmatrix} \implies \left\{ \begin{array}{l}
		\displaystyle
		u'\cos\theta' = \frac{u\cos\theta - v}{1 - uv\cos\theta/c^2}      \\
		\displaystyle
		\tan\theta' = \frac{u\sin\theta}{\gamma_{\vb u}(u\cos\theta - v)} \\
	\end{array} \right. \]
The first equation corresponds to the normal transformation law for Lorentz transforms. The second equation, corresponding to a change in angle due to the motion of the observer, is called aberration. In particular, when $u = c$, we can see that light rays appear to change direction due to the relative motion of the emitter and the observer.
