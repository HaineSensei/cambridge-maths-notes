\subsection{Energy-Momentum 4-Vector}
We define the 4-momentum of a particle of mass \(m\) and 4-velocity \(U\) to be
\[
	P = mU = m\gamma_{\vb u} \begin{pmatrix}
		c \\ \vb u
	\end{pmatrix}
\]
Since \(U\) is a 4-vector, we must have that \(m\) is invariant under a Lorentz transformation.
We will call this \(m\) the `rest mass' of the object, defined as the mass as measured in the rest frame of the particle.
The 4-momentum of a system of particles is defined as the sum of the 4-momenta of its individual particles.
The spatial components of \(P\), given by \(\mu = 1, 2, 3\), can be referred to as the relativistic 3-momentum, given by \(\vb p = \gamma_{\vb u} m \vb u\).
This matches with the definition as seen in Newtonian physics, except that the mass \(m\) is replaced by \(\gamma_{\vb u} m\).
We call this quantity the `apparent mass' of the particle or system of particles, as it represents the mass of the particle as observed by a different reference frame.
Note that \(\abs{\vb p}\) and \(\gamma_{\vb u} m\) both tend to infinity as the particle approaches the speed of light.
Note that the first component of \(P\), \(P^0\), is
\[
	\gamma_{\vb u} mc = \frac{mc}{\sqrt{1 - \frac{\vb u^2}{c^2}}} = \frac{1}{c}\qty(mc^2 + \frac{1}{2}m\vb u^2 + \dots)
\]
We recognise the \(\frac{1}{2}m\vb u^2\) term as the kinetic energy of the particle.
We interpret \(P^0\) as an energy, divided by \(c\) (to conserve units).
\[
	P = \begin{pmatrix}
		\frac{1}{c} E \\ \vb p
	\end{pmatrix}
\]
where
\[
	E = \gamma_{\vb u} mc^2 = mc^2 + \frac{1}{2}m\vb u^2 + \dots
\]
Note that as \(\abs{\vb u} \to c\), \(E \to \infty\).
Since \(P\) contains an energy term as well as a momentum term, we also call \(P\) the energy-momentum 4-vector.
Note that for a stationary particle of rest mass \(m\), we have
\[
	E = mc^2
\]
This implies that mass is a form of energy.
The energy of a moving particle is
\[
	E = mc^2 + \underbrace{(\gamma_{\vb u} - 1)mc^2}_{\mathclap{\text{relativistic kinetic energy}}}
\]
Since \(P \cdot P = \frac{E^2}{c^2} - \abs{\vb p}^2\) is Lorentz invariant, we have
\[
	P \cdot P = m^2c^2
\]
Hence,
\[
	\frac{E^2}{c^2} = \abs{\vb p}^2 + m^2c^2
\]
In Newtonian physics, mass is conserved, and energy is also conserved.
In relativistic physics, mass is not conserved by itself, since it is a form of energy.
From this derivation, it is theoretically possible to convert between mass and kinetic energy.

\subsection{Massless Particles}
A massless particle has zero rest mass.
Such particles can have nonzero momentum and nonzero energy, because they are travelling at the speed of light, giving \(\gamma_{\vb u} = \infty\).
Since \(P \cdot P = m^2c^2\), there are no factors of \(\gamma\) in this expression giving
\[
	P \cdot P = 0
\]
So such a particle travels along a light-like trajectory.
Therefore there is no Lorentz transformation that brings a given reference frame into the rest frame of the particle, so we cannot define proper time for such a particle.
Since \(m^2c^2 = 0\), we must have
\[
	\frac{E^2}{c^2} = \abs{\vb p}^2 \implies E = \abs{\vb p}c
\]
Then,
\[
	P = \frac{E}{c} \begin{pmatrix}
		1 \\ \vu n
	\end{pmatrix}
\]
where \(\vu n\) is a unit 3-vector in the direction of travel of the particle.

\subsection{Newton's Second Law}
Now that we have defined \(P\) for all particles, we can rewrite Newton's second law in special relativity as
\[
	\dv{P}{\tau} = F
\]
where \(F\) is the 4-force.
If the 3-force is \(\vb F\), we have
\[
	F = \gamma_{\vb u} \begin{pmatrix}
		\vb F \cdot \vb u / c \\
		\vb F
	\end{pmatrix}
\]
Hence,
\[
	\dv{E}{\tau} = \gamma_{\vb u} \vb F \cdot \vb u;\quad \dv{\vb p}{\tau} = \gamma_{\vb u} \vb F
\]
giving
\[
	\dv{E}{t} = \vb F \cdot \vb u;\quad \dv{\vb p}{t} = \vb F
\]
which are the familiar Newtonian expressions for rate of work and rate of change of momentum.
We can now define 4-acceleration:
\[
	F = mA
\]
where \(m\) is the rest mass.
Hence,
\[
	\dv{U}{\tau} = A
\]
