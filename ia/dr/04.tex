\subsection{Forces}
Consider a particle of mass \(m\) at position \(x(t)\) in one spatial dimension.
Let us consider the action of a force \(F(x)\) on the particle, i.e.\ a force dependent entirely on the position and not the velocity or time.
We define the potential energy \(V(x)\) by
\[
	F(x) = -\frac{\dd{V}}{\dd{x}}
\]
Hence,
\[
	V(x) = - \int^x F(u) \dd{u}
\]
The lower limit is unspecified to give an arbitrary constant in \(V(x)\).
By Newton's Second Law,
\[
	m\ddot{x} = -\frac{\dd{V}}{\dd{x}}
\]
We define the kinetic energy \(T = \frac{1}{2}m\dot x^2\).
The total energy in the system \(E\) is defined as \(T + V = \frac{1}{2} m \dot x^2 + V(x)\).
We will show that total energy is conserved: \(\frac{\dd{E}}{\dd{t}} = 0\).
\begin{proof}
	\begin{align*}
		\frac{\dd{E}}{\dd{t}} & = \frac{\dd}{\dd{t}}\left( \frac{1}{2}m\dot x^2 + V(x) \right) \\
		                      & = m\dot x \ddot x + \frac{\dd{V}}{\dd{x}} \dot x               \\
		                      & = \dot x\left( m \ddot x + \frac{\dd{V}}{\dd{x}} \right)       \\
		                      & = \dot x ( 0 )                                                 \\
		                      & = 0
	\end{align*}
\end{proof}
\noindent In general, in order to conserve a total energy \(\frac{1}{2}m\dot x^2 + \Phi\), we require that
\[
	\dot x F = -\frac{\dd{\phi}}{\dd{t}}
\]
It is usually the case that there exists no such \(\Phi\) if \(F\) depends on \(\dot x\) or \(t\).

\subsection{Force in the Harmonic Oscillator}
Let us consider the example of the harmonic oscillator, i.e.
\[
	F(x) = -kx
\]
Then we can construct
\[
	V(x) = -\int^x -ku \dd{u} = \int^x ku \dd{u} = \frac{1}{2} kx^2
\]
where we have chosen the arbitrary constant conveniently.
Note that we can explicitly solve the second order ordinary differential equation to compute \(x\) as a function of \(t\):
\[
	x(t) = A\cos \omega t + B\sin \omega t;\quad \dot x(t) = -\omega A \sin \omega t + \omega B \cos \omega t
\]
where \(\omega = \sqrt{\frac{k}{m}}\).
We can check that energy \(E\) is conserved:
\begin{align*}
	E & = \frac{1}{2}m\dot x^2 + \frac{1}{2}kx^2                                                                                                         \\
	  & = \frac{1}{2}m \left( -\omega A \cos \omega t + \omega B \sin \omega t \right)^2 + \frac{1}{2}k \left( A\sin \omega t + B\cos \omega t \right)^2 \\
	  & = \frac{1}{2}k(A^2 + B^2)
\end{align*}

\subsection{More General Potentials}
Note that conservation of energy is a first integral of Newton's Second Law.
In one dimension, conservation of energy gives useful information about a particle's motion that can help in deriving \(x\) as a function of \(t\).
In the previous example, we verified that conservation of energy holds having already solved the differential equation, but it can often be more useful to consider energy while solving the equation.
\[
	E = \frac{1}{2}m\dot x^2 + V(x)
\]
Hence,
\[
	\dot x = \pm \sqrt{\frac{2}{m}(E - V(x))}
\]
Therefore,
\[
	\int_{x_0}^x \frac{\dd{u}}{\sqrt{\frac{2}{m}(E - V(u))}} = t - t_0
\]
where \(x(t_0) = x_0\).
This gives \(t\) as a function of \(x\); we can invert this function to give \(x\) as a function of \(x\).
Realistically, this integral is mostly useful to get structural insight rather than actually solving \(x\) as a function of time, since it is difficult to do this analytically.
As an example, let
\[
	V(x) = \lambda(x^3 - 3 \beta^2 x)
\]
where \(\lambda, \beta\) are positive constants.
What happens if we release the particle from rest at \(x=x_0\)? We can draw the graph of \(V(x)\) and imagine the height of the graph as the height of a `rail' that the particle sits on, acted on under gravity, i.e.\ the particle `falls' from higher \(V(x)\) to lower \(V(x)\), gaining kinetic energy as it falls.
Since we start at rest, \(E = V(x_0)\) at \(t=0\), and in the subsequent motion \(E \leq V(x_0)\).
We have a few cases:
\begin{enumerate}[{Case} 1:]
	\item (\(x_0 < -\beta\)) \(x_0 = -\beta\) is a maximum point on the graph.
	      The particle will move to the left with \(x(t) \to -\infty\) as \(t \to \infty\).
	\item (\(-\beta < x_0 < 2\beta\)) Note that \(V(-\beta) = V(2\beta)\); they are the same height on the graph.
	      Since there is no friction in this model, the particle's motion is confined to the region \(-\beta < x < 2\beta\) and will oscillate forever.
	\item (\(2\beta < x_0\)) The particle will move to the left, reaching \(x=-\beta\), and then will continue to the left, since it has kinetic energy at this point.
	      So \(x \to -\infty\) as \(t \to \infty\).
\end{enumerate}
We also have special cases on the turning points \(\pm\beta\), where the particle does not move.
There is another case at \(x_0 = 2\beta\): the particle will move to the left, accelerating until \(x=\beta\), then decelerating until \(x=-\beta\), where it will then stop moving at this maximum point.
How long does it take for the particle to move from \(x_0=2\beta\) to \(x=-\beta\), where it rests? We can use the integral above to compute this, letting \(t_0 = 0\) and \(x(0) = 2\beta\).
\begin{align*}
	\int_{x(t)}^{2\beta} \frac{\dd \widetilde x}{\sqrt{\frac{2\lambda}{m}(2\beta^3 - \widetilde x^3 + 3 \beta^2 \widetilde x)}} & = t \\
	\int_{x(t)}^{2\beta} \frac{\dd \widetilde x}{\sqrt{\frac{2\lambda}{m}(\widetilde x + \beta)^2(2\beta - \widetilde x)}}      & = t \\
	\int_{x(t)}^{2\beta} \frac{\dd \widetilde x}{(\widetilde x + \beta)\sqrt{\frac{2\lambda}{m}(2\beta - \widetilde x)}}        & = t \\
\end{align*}
This integral diverges as \(\widetilde x \to -\beta\), so it takes an infinite amount of time to come to rest at this maximum point; specifically it exhibits logarithmic behaviour.
