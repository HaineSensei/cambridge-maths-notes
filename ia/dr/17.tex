\subsection{Simple Pendulum}
Consider a uniform rod of length \(\ell\) and mass \(M\), fixed at one end to a pivot point \(O\).
The centre of mass is the midpoint of the rod, at a distance of \(\ell/2\) from the pivot.
The angle between the rod and the rest position (when the rod is pointing downwards from the pivot) is \(\theta\).
We can consider the angular momentum about the pivot point.
\[
	\omega = \dot\theta;\quad L = I \dot\theta = \frac{1}{3}M\ell^2 \dot\theta
\]
The torque produced by the gravitational force is
\[
	G = -Mg \frac{\ell}{2}\sin\theta
\]
The torque associated with the force at the pivot will be zero, since it acts on the line of the rod.
We have
\[
	\dot L = G \implies I\ddot\theta = -Mg\frac{\ell}{2}\sin\theta \implies \ddot\theta = \frac{-3g}{2\ell}\sin\theta
\]
which is equivalent to a simple pendulum of length \(\flatfrac{2\ell}{3}\), and small oscillations will have period \(2\pi\sqrt{2\ell/3g}\).
We could alternatively solve this using conservation of energy.
\[
	T + V = \frac{1}{2}I\dot\theta^2 - \frac{Mg\ell}{2}\cos\theta = E
\]
where \(E\) is constant.
Then
\[
	I\dot\theta\ddot\theta + \frac{Mg\ell}{2}\dot\theta\sin\theta = 0
\]
So either \(\dot\theta = 0\) everywhere, or
\[
	I\ddot\theta + \frac{Mg\ell}{2}\sin\theta = 0
\]
which gives the equation of motion we found earlier.
In general, when solving a problem, there are three methods:
\begin{enumerate}
	\item use Newton's second law for the centre of mass, and use the rate of change of angular momentum about the centre of mass;
	\item use the rate of change of angular momentum about a fixed point; and
	\item use conservation of energy (less useful in general, since it removes dimensions).
\end{enumerate}

\subsection{Comparison of Sliding and Rolling}
Consider a cylinder or a sphere with radius \(a\), moving along a stationary horizontal surface.
The general motion is some combination of the rotation of the centre of mass with angular velocity \(\omega\) and the translation of the centre of mass with velocity \(v\).
The point \(P\) is the instantaneous point of contact between the body and the surface.
The horizontal velocity of the point of contact is given by
\[
	v_\text{slip} = v - a\omega
\]
In general, the point of contact \(P\) slips, and there may be some kinetic frictional force associated with this slip.
We can categorise rolling and sliding as follows.
\begin{itemize}
	\item A `pure sliding' motion is given by \(\omega = 0\), and \(v = v_\text{slip} \neq 0\).
	      In this case, the body slides across the surface without rotation.
	\item A `pure rolling' motion is given by \(v_\text{slip} = 0\), but \(v \neq 0\) and \(\omega \neq 0\).
	      In this case, the point of contact \(P\) is stationary.
	      A rolling body can be described instantaneously as rotating about the point of contact with angular velocity \(\omega\).
\end{itemize}
As an example, consider a body rolling downhill, where the hill has a constant incline \(\alpha\) to the horizontal.
Let \(x\) be the displacement of the centre of mass from its initial position, so \(v = \dot x\).
Let \(Mg\) be the gravitational force, \(N\) be the normal force, and \(F\) be the frictional force.
Now, we know that the rolling condition is that \(v - a\omega = 0\).
We will analyse the motion of this body, under the assumption that it is rolling, by consideration of energy.
\[
	T = \frac{1}{2}Mv^2 + \frac{1}{2}I\omega^2 = \frac{1}{2}\qty(M + \frac{I}{a^2})v^2;\quad V = -Mgx\sin\alpha
\]
The normal force does not do any work, since it is perpendicular to the direction of motion, and the frictional force does not do work because the point of contact is instantaneously stationary.
Hence, energy is conserved, giving
\[
	\frac{1}{2}\qty(M + \frac{I}{a^2})v^2 - Mgx\sin\alpha = E
\]
Hence,
\[
	\qty(M + \frac{I}{a^2})\dot x \ddot x - Mg \dot x \sin\alpha = 0
\]
We have therefore deduced that
\[
	\qty(M + \frac{I}{a^2})\ddot x = Mg\sin\alpha
\]
which is a second order differential equation with constant coefficients, which we can solve.
Note that due to the \(\frac{I}{a^2}\) term, the total acceleration is less than it would be for a frictionless particle (since such a particle would not rotate).
For example, a cylinder would have \(I = \frac{1}{2}Ma^2\) hence \(\ddot x = \frac{2}{3}Mg\sin\alpha\).
Alternatively, we could analyse the forces and torques.
We can use Newton's second law to deduce
\[
	M\dot v = Mg\sin\alpha - F
\]
Further, the rate of change of angular momentum about the centre of mass is
\[
	I \dot\omega = aF
\]
The rolling condition implies that \(\dot v = a \dot\omega\), hence
\[
	\frac{I\dot v}{a} = aF \implies M\dot v = Mg\sin\alpha - \frac{I\dot v}{a^2} \implies \left( M + \frac{I}{a^2} \right)\dot v = Mg\sin\alpha
\]
We could also alternatively look at the torque about the point \(P\).
In this case, using the parallel axes theorem,
\[
	I_P = I + Ma^2
\]
Then,
\[
	I_P \dot\omega = Mga\sin\alpha \implies (M a^2 + I)\frac{\dot v}{a} = Mga\sin\alpha
\]
and the substitution \(v = a\omega\) gives the equation we found before.

\subsection{Transition from Sliding to Rolling}
Consider a sphere with radius \(a\) that begins by sliding across a horizontal surface, for instance a snooker ball being hit parallel to the table, through the centre of mass, by a cue.
Eventually, the ball will transition from sliding to rolling across the table.
Initially, \(v = v_0\) and \(\omega = 0\).
The kinetic frictional force \(F\) is given by
\[
	F = \mu_k N = \mu_k Mg
\]
Considering linear motion, we have
\[
	M \dot v = -F
\]
Considering angular motion,
\[
	I\dot\omega = aF \implies \frac{2}{5}Ma \dot\omega = F
\]
Hence,
\[
	M\dot v +\frac{2}{5}Ma \dot\omega = 0 \implies v = v_0 - \mu_k g t; \omega = \frac{5}{2a}\mu_k g t
\]
We can now compute the slip velocity.
\[
	v_\text{slip} = v - a\omega = v_0 - \frac{7}{2}\mu_k gt
\]
There is slipping when \(v_\text{slip} > 0\), which occurs for
\[
	0 \leq t < \frac{2v_0}{7\mu_k g}
\]
Rolling begins when \(t = \frac{2v_0}{7\mu_k g} = t_\text{roll}\).
Note that at this time,
\[
	T = \frac{1}{2}Mv^2 + \frac{1}{2}I\omega^2 = \frac{1}{2}M\qty(1 + \frac{2}{5})v_\text{roll}^2 = \frac{5}{7}\qty( \frac{1}{2} Mv_0^2 )
\]
So during the sliding phase, we have lost \(\frac{2}{7}\) of the initial kinetic energy.
We can check the loss of kinetic energy due to friction, giving
\[
	\int_0^{t_\text{roll}} F v_\text{slip} \dd{t} = \int_0^{t_\text{roll}} F \qty(v_0 - \frac{7}{2}\mu_k gt) \dd{t} = \frac{2}{7}\qty(\frac{1}{2}Mv_0^2)
\]
as expected.
