\subsection{Invariant interval}
Consider two events \(P\) and \(Q\) with space-time coordinates \((ct_1, x_1)\) and \((ct_2, x_2)\), where the time coordinate is given first.
The time separation is \(\Delta t = t_1 - t_2\), and the space separation \(\Delta x = x_1 - x_2\).
These two separations are dependent on the choice of inertial frame.
The invariant interval between \(P\) and \(Q\) is defined as
\[
	\Delta s^2 = c^2 \Delta t^2 - \Delta x^2
\]
This is invariant under a Lorentz transformation.
%TODO show this
In three spatial dimensions, we simply replace this \(\Delta x^2\) with \(\Delta x^2 + \Delta y^2 + \Delta z^2\), so
\[
	\Delta s^2 = c^2 \Delta t^2 - \Delta x^2 - \Delta y^2 - \Delta z^2
\]
If the separation between \(P\) and \(Q\) is very small, we can define the infinitesimal invariant interval as
\[
	\dd{s}^2 = c^2\dd{t}^2 - \dd{x}^2 - \dd{y}^2 - \dd{z}^2
\]
Note that spacetime with three spatial dimensions (Minkowski spacetime) is topologically equivalent to \(\mathbb R^4\), where the distance measure is \(\dd{s}^2\) as defined above.
Note that this distance quantity, although squared, can be either positive or negative.
Sometimes this arrangement of one temporal and three spatial dimensions is denoted by the abbreviation `\(1+3\) dimensions'.

\subsection{Signs of the invariant interval}
As noted before, \(\Delta s^2\) can have either a positive or negative sign.
\begin{itemize}
	\item Events with \(\Delta s^2 > 0\) are `time-like separated'.
	      In this case, there exists a frame of reference in which the events occur in the same spatial position, but at different times.
	      In particular, the two events appear in each other's light cones.
	      The time ordering of the two events is unambiguous.
	\item Events with \(\Delta s^2 < 0\) are `space-like separated'.
	      Here, there exists a frame of reference in which the events occur at the same time, but in different places.
	      The two events are outside of each other's light cones, and the ordering of the two events can change depending on the choice of frame of reference.
	\item If \(\Delta s^2 = 0\), the events are `light-like separated'.
	      The events lie exactly on each other's light cones, and this does \textit{not} imply that the two events are the same (unlike in Euclidean space, where a distance measure of zero implies that two points are equal).
\end{itemize}

\subsection{The Lorentz group}
The coordinates of an event \(P\) in some frame \(S\) can be written as a 4-vector \(X\).
\[
	X^\mu = \begin{pmatrix}
		ct \\ x \\ y \\ z
	\end{pmatrix}
\]
where the \(ct\) coordinate is given by \(\mu = 0\) and the spatial dimensions are given by \(\mu = 1, 2, 3\) as usual.
The invariant interval betwen \(P\) and the origin is written as the inner product of \(X\) with itself:
\[
	X \cdot X := X^\transpose \eta X
\]
or alternatively,
\[
	X \cdot X = \eta_{\mu\nu} X^\mu X^\nu
\]
where \(\eta\) is the Minkowski metric given by
\[
	\eta = \begin{pmatrix}
		1 & 0  & 0  & 0  \\
		0 & -1 & 0  & 0  \\
		0 & 0  & -1 & 0  \\
		0 & 0  & 0  & -1
	\end{pmatrix}
\]
Then
\[
	X \cdot X = c^2t^2 - x^2 - y^2 - z^2
\]
We can classify 4-vectors as `space-like', `time-like' and `light-like' as before, by considering the sign of \(\eta_{\mu\nu}X^\mu X^\nu\).
The Lorentz transformation is a linear transformation that converts the components of \(X\) into the components of \(X\) in \(S'\).
Therefore, any Lorentz transform can be represented as a \(4\times 4\) matrix \(\Lambda\).
We now define Lorentz transforms as such linear transformations that preserve the Minkowski metric.
So considering a sets of coordinates \(X\) and \(X'\) in \(S\) and \(S'\), we have \(X' = \Lambda X\), and \(X' \cdot X' = X \cdot X\).
This then implies that
\begin{equation}
	\Lambda^\transpose \eta \Lambda = \eta \tag{\(\ast\)}
\end{equation}
Two classes of possible \(\Lambda\) are
\[
	\Lambda = \begin{pmatrix}
		1 & 0 & 0 & 0 \\
		0 & a & b & c \\
		0 & d & e & f \\
		0 & g & h & i
	\end{pmatrix};\quad R = \begin{pmatrix}
		a & b & c \\
		d & e & f \\
		g & h & i
	\end{pmatrix}
\]
where \(R^\transpose R = I\), giving that \(R\) may be a spatial rotation or a reflection.
We could also have
\[
	\Lambda = \begin{pmatrix}
		\gamma       & -\gamma\beta & 0 & 0 \\
		-\gamma\beta & \gamma       & 0 & 0 \\
		0            & 0            & 1 & 0 \\
		0            & 0            & 0 & 1
	\end{pmatrix}
\]
where \(\beta = \frac{v}{c}\).
This expresses a Lorentz transformation where the two frames are moving at a constant velocity \(v\) relative to each other, as discussed before in \(1+1\) spacetime.
We denote the Lorentz group as \(O(1, 3)\), defined by the set of \(\Lambda\) that satisfy \((\ast)\).
Note that this includes the group generated by the above two transformations (notably including spatial reflections), as well as time reflections.
We define the \textit{proper} Lorentz group as \(SO(1, 3)\), which is the kernel of the determinant homomorphism on the Lorentz group.
Note that this includes the \textit{composition} of both temporal and spatial reflection.
The subgroup that forbids any kind of reflection is called the \textit{restricted} Lorentz group, denoted \(SO^+(1, 3)\), generated by compositions of rotations and boosts, as shown in the above two examples (excluding the case when \(R\) is a reflection).

\subsection{Rapidity}
While a \(4\times 4\) matrix can be useful for computation, it is sometimes more convenient to label a Lorentz transformation using a concept of `rapidity'.
In \(1+1\) spacetime, we write
\[
	\Lambda[\beta] = \begin{pmatrix}
		\gamma       & -\gamma\beta \\
		-\gamma\beta & \gamma
	\end{pmatrix};\quad \gamma = (1 - \beta^2)^{-\frac{1}{2}}
\]
This represents a boost in the \(x\) direction.
Combining two boosts, we get
\[
	\Lambda[\beta_1]\Lambda[\beta_2] = \begin{pmatrix}
		\gamma_1         & -\gamma_1\beta_1 \\
		-\gamma_1\beta_1 & \gamma_1
	\end{pmatrix}\begin{pmatrix}
		\gamma_2         & -\gamma_2\beta_2 \\
		-\gamma_2\beta_2 & \gamma_2
	\end{pmatrix} = \begin{pmatrix}
		\gamma_1\gamma_2(1 + \beta_1 \beta_2) & -\gamma_1\gamma_2(\beta_1 + \beta_2)  \\
		-\gamma_1\gamma_2(\beta_1 + \beta_2)  & \gamma_1\gamma_2(1 + \beta_1 \beta_2)
	\end{pmatrix} = \Lambda\qty[\frac{\beta_1 + \beta_2}{1 + \beta_1\beta_2}]
\]
Note the relation to the velocity transformation law.
Recall that with spatial rotations, we can characterise a rotation \(R\) by some parameter \(\theta\), where \(R(\theta_1) R(\theta_2) = R(\theta_1 + \theta_2)\).
This is the same kind of composition law.
For Lorentz boosts, we can define \(\phi\) such that \(\beta = \tanh\phi\), and then we can redefine \(\Lambda\) to be in terms of \(\phi\), giving this new composition law
\[
	\Lambda[\phi_1]\Lambda[\phi_2] = \Lambda[\phi_1 + \phi_2]
\]
Note that \(\gamma = \cosh \phi\), and \(\gamma\beta = \sinh \phi\).
This suggests that Lorentz boosts can be thought of as hyperbolic rotations in spacetime.
