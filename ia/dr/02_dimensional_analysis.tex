\subsection{Choice of units}
Many problems in dynamics involve three basic dimensional quantities: length, mass and time.
These are commonly referred to using the symbols \(L\), \(M\) and \(T\), to be generic over the choice of unit system.
The dimensions of some quantity \(x\) can therefore be expressed in terms of powers of \(L\), \(M\), \(T\).
So the dimension of density is \(M \cdot L^{-3}\).
The dimension of force is \(M \cdot L \cdot T^{-2}\).

Only `power law' functions of these quantities are allowed; we are not allowed to exponentiate a dimensional quantity, for example.
This is because \(e^L = 1 + L + \frac{1}{2}L^2 + \dots\) would be comparing a dimensionless constant 1 with some length, and some area, and so forth.
This comparison does not make any sense.

We can choose a unit system that is convenient, for example SI units.
It defines the metre for \(L\), the kilogram for \(M\) and the second for \(T\).
So many other physical quantities can be formed from these.
For example, the SI unit for the gravitational constant is \si{\metre\cubed\per\kilogram\per\second\squared}.
In this unit system, we can say \(G = \SI{6.67e-11}{\metre\cubed\per\kilogram\per\second\squared}\).

As a general principle, dynamical and physical equations must work for any consistent choice of units.
If, however, we used SI units for length, mass and time, but the imperial unit pound-force as the unit for force, the equations would be inconsistent.

\subsection{Scaling and dimensional independence}
Suppose that a dimensional quantity \(Y\) depends on a set of dimensional quantities \(X_1, \dots, X_n\), so the dimension of \(Y\) is \(L^a M^b T^c\) and the dimension of the \(X_i\) are \(L^{a_i} M^{b_i} T^{c_i}\).

If \(n \leq 3\), then \(Y = C \cdot X_1^{p_1}X_2^{p_2}X_3^{p_3}\), and \(p_1, p_2, p_3\) can be found by balancing the dimensions.
Hence \(a = a_1p_1 + a_2p_2 + a_3p_3\) and so forth for \(b\) and \(c\).
This yields a unique solution for \(p_1, p_2, p_3\) if these three equations are linearly independent, i.e.\ if the dimensions of \(X_1, X_2, X_3\) are independent.

If \(n > 3\), then this property of dimensional independence does not hold; it is always possible to express one of the four (or more) dimensions in terms of the other three.
So let us choose \(X_1, X_2, X_3\) to be dimensionally independent, and then we can incorporate \(X_4, X_5\) and so on as dimensionless quantities:
\[
	\lambda_1 = \frac{X_4}{X_1^{q_{11}}X_2^{q_{12}}X_3^{q_{13}}};\quad \lambda_2 = \frac{X_5}{X_1^{q_{21}}X_2^{q_{22}}X_3^{q_{23}}} \cdots
\]
where the powers \(q_{ij}\) have been chosen such that the \(\lambda\) are dimensionless.
Then
\[
	Y = X_1^{p_1}X_2^{p_2}X_3^{p_3} \cdot C(\lambda_1, \lambda_2, \dots, \lambda_{n-3})
\]
This is known as Bridgman's Theorem.

\begin{example}
	As an example, let us consider a simple pendulum with a string of length \(\ell\), released from rest, when the horizontal distance from the end of the pendulum to the rest position is \(d\).
	How does the period \(P\) of the pendulum depend on the four dimensional quantities \(m, \ell, d, g\)?

	We know that the dimension of the period is \(T\), time.
	The dimension of \(m\) is \(M\), the dimension of \(g\) is \(L \cdot T^{-2}\), and the dimensions of \(\ell\) and \(d\) are both \(L\).
	We will form one dimensionless group, since \(n=4\) in this case.
	A simple way of doing so is letting \(\lambda = d/\ell\).
	So \(P = m^{p_1} \ell^{p_2} g^{p_3} \cdot f(d/\ell)\).
	Comparing units, we have \(T = M^{p_1} L^{p_2} (L \cdot T^{-2})^{p_3}\).
	Solving, we get \(p_1 = 0, p_2 = \frac{1}{2}, p_3 = \frac{-1}{2}\).
	Applying Bridgman's Theorem, we have \(P = \sqrt{\ell / g} \cdot f(d/\ell)\).
	This does not completely specify the formula, but it does provide useful insights.
	For example, doubling both \(d\) and \(\ell\), \(P \mapsto \sqrt{2} P\), since \(d/\ell\) does not change.
\end{example}

\begin{example}
	Taylor used publicly available data on the fireball's growth over time in order to estimate the energy released in the first atomic explosion.
	Let \(R(t)\) be the radius of the fireball as a function of time, which has dimension \(L\).
	The time \(t\) has dimension \(T\).
	The density of air \(\rho\) has dimension \(M \cdot L^{-3}\).
	The energy of the explosion is \(E\) which has dimension \(M \cdot L^2 \cdot T^{-2}\).
	Then, \(R = C \cdot t^\alpha \rho^\beta E^\gamma\).
	By balancing dimensions, we have \(\alpha = \frac{2}{5}, \beta = \frac{-1}{5}, \gamma = \frac{1}{5}\).
	Then, \(R(t) = C \cdot t^{\frac{2}{5}} \rho^{\frac{-1}{5}} \gamma^{\frac{1}{5}}\).

	Taylor then verified this \(\frac{2}{5}\) power law, and estimated the value of \(E\) as \(\frac{\rho R^5}{C^5 t^2}\).
	It was observed that \(\frac{R^5}{t^2} \sim \SI{6.7e13}{\metre\tothe{5}\per\second}\), and \(\rho\sim\SI{1.25}{\kilogram\per\metre\cubed}\).
	Then if \(C \sim 1\) then \(E \sim \SI{1e14}{\joule}\), which is approximately \SI{2.4e4}{\tonne} of TNT.\@
\end{example}
