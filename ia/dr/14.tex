\subsection{Two Body Problem}
Consider two bodies of mass \(m_1\), \(m_2\) experiencing gravitational attraction to the other, with no external forces.
Let \(m_1\) be at position \(\vb r_1\), and \(m_2\) at \(\vb r_2\), with the centre of mass at \(\vb R\), and total mass \(M = m_1 + m_2\).
Then certainly,
\[
	\vb R = \frac{1}{M}(m_1 \vb r_1 + m_2 \vb r_2)
\]
We will define the separation vector \(\vb r = \vb r_1 - \vb r_2\).
We can then further say that
\[
	\vb r_1 = \vb R + \frac{m_2}{M}\vb r;\quad \vb r_2 = \vb R - \frac{m_1}{M}\vb r
\]
Since \(\vb F^\text{ext} = \vb 0\), the centre of mass \(\vb R\) does not accelerate; it moves with constant velocity.
Now, let us consider \(\vb r\).
\[
	\rddot = \rddot_1 + \rddot_2 = \frac{\vb F_{12}}{m_1} - \frac{\vb F_{21}}{m_2} = \vb F_{12} \left( \frac{1}{m_1} + \frac{1}{m_2} \right)
\]
Equivalently, we can write
\[
	\mu\rddot = \vb F_{12};\quad \mu = \frac{m_1 m_2}{m_1 + m_2}
\]
Notice that \(\mu\) has the dimension of mass; we call it the `reduced' mass since it is less than \(m_1\) and \(m_2\).
This can be seen as the equation of motion of a particle of mass \(\mu\) under the effect of force \(\vb F_{12}\).
In the case of a gravitational force, we have
\[
	\mu \rddot = \frac{-Gm_1m_2}{\abs{\vb r}^3}\vb r
\]
Hence,
\[
	\rddot = \frac{-G(m_1 + m_2)}{\abs{\vb r}^3}\vb r
\]
This is the motion of a particle under the effect of a gravitational force due to a mass \(m_1 + m_2\) fixed at the origin.
The total kinetic energy \(T\) is
\[
	T = \frac{1}{2}M\dot {\vb R}^2 + \frac{1}{2}\mu \rdot^2
\]
The total angular momentum \(\vb L\) is
\[
	\vb L = M \vb R \times \dot{\vb R} + \mu \vb r \times \dot{\vb r}
\]

\subsection{Example of Two Body Problem}
Let us consider the orbit of the earth and the sun.
Both particles move around the centre of mass, and both orbits have the same shape.
However, the sizes of the orbits are very different.
The ratio of masses is around \num{3e-4}, and the radius of orbit is approximately \SI{1.5e7}{\kilo\metre}.
Hence the displacement of the sun is around \SI{450}{\kilo\metre}.

\subsection{Variable Mass Problems and the Rocket Problem}
Consider a rocket which ejects mass (exhaust gases) at a high speed in order to propel itself forward.
We cannot apply Newton's second law to the rocket alone, since in this system mass is not conserved.
Consider the rocket moving in one dimension, with speed \(v(t)\) and mass \(m(t)\).
The mass is being expelled at velocity \(u\) relative to the rocket.
At time \(t\), the rocket has momentum \(v(t) m(t)\).
At time \(t + \delta t\), the momentum is \(v(t + \delta t) m(t + \delta t)\).
The exhaust gases emitted during \(\delta t\) have velocity \(v(t) - u + O(\delta t)\) and mass \(m(t) - m(t + \delta t)\).
The total momentum at \(t + \delta t\) is
\[
	v(t + \delta t) m(t + \delta t) + (v(t) - u + O(\delta t))(m(t) - m(t + \delta t))
\]
So the change in momentum is
\begin{align*}
	\delta p & = v(t + \delta t) m(t + \delta t) + (v(t) - u + O(\delta t))(m(t) - m(t + \delta t)) - v(t) m(t) \\
	         & = \left( \dv{m}{t} u + m \dv{v}{t} \right) \delta t + O(\delta t^2)                              \\
\end{align*}
But since momentum is conserved,
\[
	\dv{m}{t} u + m \dv{v}{t} = 0
\]
This is called the rocket equation.
We can generalise this to \(\dv{m}{t} u + m \dv{v}{t} = \vb F^\text{ext}\) in the presence of external forces.
In the absence of such external forces,
\begin{align*}
	\dv{m}{t} u   & = - m \dv{v}{t}                                  \\
	\implies v(t) & = v(0) + u \log \left( \frac{m(0)}{m(t)} \right)
\end{align*}
