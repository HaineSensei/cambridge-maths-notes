\subsection{Angular Momentum}
We define the angular momentum for a particle with position vector \(\vb r(t)\), of mass \(m\), moving under the influence of a force \(\vb F\) as
\[
	\vb L = \vb r \times \vb p = \vb r \times m \dot{\vb r}
\]
Then
\[
	\dot{\vb L} = m\dot{\vb r} \times \dot{\vb r} + m\vb r \times \ddot{\vb r} = \vb r \times \vb F = \vb G
\]
This term \(\vb r \times \vb F = \vb G\) is sometimes called the torque or the moment of the force.
The values of \(\vb L\) and \(\vb G\) depend on the choice of origin, so we typically refer to the angular momentum about a particular point.
If \(\vb r \times \vb F = \vb 0\), then the angular momentum is conserved.
The angular momentum around some suitably chosen point may be constant; this may help with calculations since we are free to choose the origin.

\subsection{Orbits}
We will begin the topic of orbits by considering the problem of gravitational orbits.
Let
\[
	m\ddot{\vb r} = -\grad V(r)
\]
This represents a particle moving in a conservative force that is a function only of the radius from the origin.
For this problem, we are assuming that the `central' mass remains fixed at the origin.
This is a good approximation if the central mass is significantly larger than \(m\).

\subsection{Central Forces}
We define a central force as a conservative force with the potential \(V(r)\) being a function only of the radius from the origin.
Consequently,
\[
	\vb F = -\grad V(r) = -\grad V(\abs{\vb r}) = -\frac{\dd{V}}{\dd{r}}\vu{r}
\]
Consider the angular momentum \(\vb L\) about the origin, given by
\[
	\dot{\vb L} = \vb r \times \vb F = \vb r \times \left(-\frac{\dd{V}}{\dd{r}}\vu{r}\right) = 0
\]
So angular momentum about the origin is conserved for any central force.
Further, from the definition of \(\vb L\),
\[
	\vb L \cdot \vb r = 0
\]
Hence, the motion of the particle is confined to the plane through the origin, perpendicular to \(\vb L\).
This reduces a three-dimensional problem into a two-dimensional problem.

\subsection{Polar Coordinates in the Plane}
A convenient choice of coordinates to use is the set of two-dimensional polar coordinates, by choosing the \(z\) axis such that the orbit lies in the plane \(z=0\).
Then
\[
	x = r\cos\theta;\quad y = r\sin\theta
\]
Then, relative to the Cartesian axes,
\[
	\vb e_r = \vu{r} = \begin{pmatrix}
		\cos\theta \\ \sin\theta
	\end{pmatrix};\quad \vb e_\theta = \begin{pmatrix}
		-\sin\theta \\ \cos\theta
	\end{pmatrix}
\]
At any point, \(\vb e_r\), \(\vb e_\theta\) form an orthonormal basis, but the basis can point in different directions for different values of \(\theta\).
In other words, they form a set of orthonormal curvilinear coordinates.
We have
\[
	\frac{\dd}{\dd{\theta}}\vb e_r = \vb e_\theta;\quad \frac{\dd}{\dd{\theta}}\vb e_\theta = -\vb e_r
\]
Note that for a moving particle, \(r\) and \(\theta\) are functions of position, and hence functions of time.
So we can use the following results:
\[
	\frac{\dd \vb e_r}{\dd{t}} = \frac{\dd{\theta}}{\dd{t}} \frac{\dd \vb e_r}{\dd{\theta}} = \vb e_\theta \frac{\dd{\theta}}{\dd{t}};\quad \frac{\dd \vb e_\theta}{\dd{t}} = \frac{\dd{\theta}}{\dd{t}} \frac{\dd \vb e_\theta}{\dd{\theta}} = -\vb e_r \frac{\dd{\theta}}{\dd{t}}
\]
We can compute expressions for velocity and acceleration in terms of these new coordinates.
\begin{align*}
	\vb r                  & = r \vb e_r                                    \\
	\therefore \dot{\vb r} & = \dot r \vb e_r + r \frac{\dd}{\dd{t}}\vb e_r \\
	                       & = \dot r \vb e_r + r \dot\theta \vb e_\theta
\end{align*}
So \(\dot r\) is the radial component of the velocity, and \(r\dot\theta\) is the angular component of the velocity.
Note that \(\dot\theta\) is the angular velocity.
Further:
\begin{align*}
	\ddot{\vb r} & = \frac{\dd}{\dd{t}}\left( \dot r \vb e_r + r \dot \theta \vb e_\theta \right)                                                                                        \\
	             & = \ddot r \vb e_r + \dot r \dot{\vb e}_r + \dot r \dot \theta \vb e_\theta + r \ddot \theta \vb e_\theta + r \dot \theta \dot{\vb e}_\theta                           \\
	             & = \ddot r \vb e_r + \dot r \dot \theta \vb e_\theta + \dot r \dot \theta \vb e_\theta + r \ddot \theta \vb e_\theta + r \dot \theta \left(-\dot \theta \vb e_r\right) \\
	             & = \left(\ddot r - r \dot \theta^2\right) \vb e_r + \left(2\dot r\dot \theta + r\ddot \theta\right) \vb e_\theta
\end{align*}
Again we can read off the radial and angular components of the acceleration.

\subsection{Circular Motion}
Let us consider the example of circular motion with constant angular velocity.
Then we can set \(r = a\), \(\dot\theta = \omega\), and let \(\dot r = \ddot r = \ddot \theta = 0\).
We can find that
\[
	\dot {\vb r} = a \omega \vb e_\theta ;\quad \ddot{\vb r} = -a\omega^2 \vb e_r
\]
The acceleration is in the inward radial direction, which constrains the particle to follow a circlar path instead of flying off tangentially towards infinity.
Therefore, by Newton's second law, there is a constant force in this direction.
