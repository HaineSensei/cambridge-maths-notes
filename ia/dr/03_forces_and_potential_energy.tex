\subsection{Forces}
Consider a particle of mass \(m\) at position \(x(t)\) in one spatial dimension.
Let us consider the action of a force \(F(x)\) on the particle, i.e.\ a force dependent entirely on the position and not the velocity or time.
We define the potential energy \(V(x)\) by
\[
	F(x) = -\frac{\dd{V}}{\dd{x}}
\]
Hence,
\[
	V(x) = - \int^x F(u) \dd{u}
\]
The lower limit is unspecified to give an arbitrary constant in \(V(x)\).
If possible, the constant is usually chosen such that as \(\abs{x} \to \infty\), we have \(V \to 0\).
By Newton's Second Law,
\[
	m\ddot{x} = -\frac{\dd{V}}{\dd{x}}
\]
We define the kinetic energy \(T = \frac{1}{2}m\dot x^2\).
The total energy in the system \(E\) is defined as \(T + V = \frac{1}{2} m \dot x^2 + V(x)\).
We will show that total energy is conserved: \(\frac{\dd{E}}{\dd{t}} = 0\).
\begin{proof}
	\begin{align*}
		\frac{\dd{E}}{\dd{t}} & = \frac{\dd}{\dd{t}}\left( \frac{1}{2}m\dot x^2 + V(x) \right) \\
		                      & = m\dot x \ddot x + \frac{\dd{V}}{\dd{x}} \dot x               \\
		                      & = \dot x\left( m \ddot x + \frac{\dd{V}}{\dd{x}} \right)       \\
		                      & = \dot x ( 0 )                                                 \\
		                      & = 0
	\end{align*}
\end{proof}
In general, in order to conserve a total energy \(\frac{1}{2}m\dot x^2 + \Phi\), we require that
\[
	\dot x F = -\frac{\dd{\phi}}{\dd{t}}
\]
It is usually the case that there exists no such \(\Phi\) if \(F\) depends on \(\dot x\) or \(t\).

\subsection{Force in the harmonic oscillator}
Let us consider the example of the harmonic oscillator, i.e.
\[
	F(x) = -kx
\]
Then we can construct
\[
	V(x) = -\int^x -ku \dd{u} = \int^x ku \dd{u} = \frac{1}{2} kx^2
\]
where we have chosen the arbitrary constant conveniently.
Note that we can explicitly solve the second order ordinary differential equation to compute \(x\) as a function of \(t\):
\[
	x(t) = A\cos \omega t + B\sin \omega t;\quad \dot x(t) = -\omega A \sin \omega t + \omega B \cos \omega t
\]
where \(\omega = \sqrt{\frac{k}{m}}\).
We can check that energy \(E\) is conserved:
\begin{align*}
	E & = \frac{1}{2}m\dot x^2 + \frac{1}{2}kx^2                                                                                                         \\
	  & = \frac{1}{2}m \left( -\omega A \cos \omega t + \omega B \sin \omega t \right)^2 + \frac{1}{2}k \left( A\sin \omega t + B\cos \omega t \right)^2 \\
	  & = \frac{1}{2}k(A^2 + B^2)
\end{align*}

\subsection{More general potentials}
Note that conservation of energy is a first integral of Newton's Second Law.
In one dimension, conservation of energy gives useful information about a particle's motion that can help in deriving \(x\) as a function of \(t\).
In the previous example, we verified that conservation of energy holds having already solved the differential equation, but it can often be more useful to consider energy while solving the equation.
\[
	E = \frac{1}{2}m\dot x^2 + V(x)
\]
Hence,
\[
	\dot x = \pm \sqrt{\frac{2}{m}(E - V(x))}
\]
Therefore,
\[
	\int_{x_0}^x \frac{\dd{u}}{\sqrt{\frac{2}{m}(E - V(u))}} = t - t_0
\]
where \(x(t_0) = x_0\).
This gives \(t\) as a function of \(x\); we can invert this function to give \(x\) as a function of \(x\).
Realistically, this integral is mostly useful to get structural insight rather than actually solving \(x\) as a function of time, since it is difficult to do this analytically.
As an example, let
\[
	V(x) = \lambda(x^3 - 3 \beta^2 x)
\]
where \(\lambda, \beta\) are positive constants.
What happens if we release the particle from rest at \(x=x_0\)?
We can draw the graph of \(V(x)\) and imagine the height of the graph as the height of a `rail' that the particle sits on, acted on under gravity, i.e.\ the particle `falls' from higher \(V(x)\) to lower \(V(x)\), gaining kinetic energy as it falls.
Since we start at rest, \(E = V(x_0)\) at \(t=0\), and in the subsequent motion \(E \leq V(x_0)\).
We have a few cases:
\begin{enumerate}[{Case} 1:]
	\item (\(x_0 < -\beta\)) \(x_0 = -\beta\) is a maximum point on the graph.
	      The particle will move to the left with \(x(t) \to -\infty\) as \(t \to \infty\).
	\item (\(-\beta < x_0 < 2\beta\)) Note that \(V(-\beta) = V(2\beta)\); they are the same height on the graph.
	      Since there is no friction in this model, the particle's motion is confined to the region \(-\beta < x < 2\beta\) and will oscillate forever.
	\item (\(2\beta < x_0\)) The particle will move to the left, reaching \(x=-\beta\), and then will continue to the left, since it has kinetic energy at this point.
	      So \(x \to -\infty\) as \(t \to \infty\).
\end{enumerate}
We also have special cases on the turning points \(\pm\beta\), where the particle does not move.
There is another case at \(x_0 = 2\beta\): the particle will move to the left, accelerating until \(x=\beta\), then decelerating until \(x=-\beta\), where it will then stop moving at this maximum point.
How long does it take for the particle to move from \(x_0=2\beta\) to \(x=-\beta\), where it rests?
We can use the integral above to compute this, letting \(t_0 = 0\) and \(x(0) = 2\beta\).
\begin{align*}
	\int_{x(t)}^{2\beta} \frac{\dd \widetilde x}{\sqrt{\frac{2\lambda}{m}(2\beta^3 - \widetilde x^3 + 3 \beta^2 \widetilde x)}} & = t \\
	\int_{x(t)}^{2\beta} \frac{\dd \widetilde x}{\sqrt{\frac{2\lambda}{m}(\widetilde x + \beta)^2(2\beta - \widetilde x)}}      & = t \\
	\int_{x(t)}^{2\beta} \frac{\dd \widetilde x}{(\widetilde x + \beta)\sqrt{\frac{2\lambda}{m}(2\beta - \widetilde x)}}        & = t \\
\end{align*}
This integral diverges as \(\widetilde x \to -\beta\), so it takes an infinite amount of time to come to rest at this maximum point; specifically it exhibits logarithmic behaviour.

\subsection{Equilibrium points}
An equilibrium point is defined as a point where the potential is stationary, in other words where the force on the particle is zero.
So the particle stays at rest for all time.
In the example in the previous lecture, \(x = \pm \beta\) were the equilibrium points.
We can analyse the motion close to the equilibrium point in order to work out whether the equilibrium point is stable or unstable.
Let \(x_0\) be an equilibrium point, so \(V'(x_0) = 0\).
We can expand \(V(x)\) as a series, assuming that \(x-x_0\) is small.
\[
	V(x) = V(x_0) + \frac{1}{2}(x-x_0)^2V''(x_0) + o((x-x_0)^2)
\]
In the neighbourhood of \(x_0\),
\[
	m\ddot x = -V'(x) \approx -(x-x_0)V''(x_0)
\]
\begin{itemize}
	\item If \(V''(x_0) > 0\), we have a local minimum of potential, which gives rise to a stable equilibrium point.
	      The equation of motion of a particle near \(x_0\) is a harmonic oscillator.
	      The angular frequency of oscillation is \(\omega = \sqrt{\frac{V''(x_0)}{m}}\).
	\item If \(V''(x_0) < 0\), we have a local maximum of potential, which gives rise to an unstable equilibrium point.
	      Any perturbation from this point will cause an increased deviation from the point.
	      The equation of motion near this point is exponential; almost always exponentially increasing rather than decreasing.
	      The growth rate is \(\gamma = \sqrt{\frac{-V''(x_0)}{m}}\).
	\item If \(V''(x) = 0\), we must use higher-order terms from the Taylor series in order to determine the behaviour.
\end{itemize}
Let us consider the example of a simple pendulum with a mass \(m\) held by a rigid beam of length \(\ell\).
Let the angle between the beam and the vertical direction be \(\theta\).
By Newton's second law,
\[
	F(x = \ell \theta) = m \ell \ddot \theta = -mg \sin \theta
\]
We can derive an energy equation by using \(F(x) = -V'(x)\).
\[
	V(x = \ell \theta) = -\int_0^{\ell\theta} F(u) \dd{u} = -mg \ell \cos \theta
\]
The kinetic energy \(T\) is given by
\[
	T = \frac{1}{2}m\ell^2\dot\theta^2
\]
We can check that \(\frac{\dd{E}}{\dd{t}} = 0\) at all \(t\).
The stationary points of \(V\) are at \(\theta = 0\) and \(\theta = \pi\) (assuming \(0 \leq \theta < 2\pi\)).
The \(\theta=0\) point is stable, since \(V''(\theta = 0) > 0\).
The \(\theta=\pi\) point is unstable.
If \(-mg\ell < E < mg\ell\), the pendulum will oscillate between two values since it cannot continue spinning in circles.
In particular, this oscillation occurs about a position of stable equilibrium.
However, if we add additional energy into this system, either \(\dot\theta > 0\) or \(\dot\theta < 0\) for all time.
It is impossible to have \(E < -mg\ell\) since this is the minimum value of the potential.

Now, let us consider the period \(P\) of the oscillation of \(\theta\) after releasing the particle from rest at some initial angle \(\theta_0\).
Note that the oscillation consists of \(\theta_0 \to 0 \to -\theta_0 \to 0 \to \theta_0\).
By symmetry, this period is four times the time it takes to go from \(\theta_0\) to 0.
From the energy equation, we can deduce
\begin{align*}
	P & = 4 \int_0^{\theta_0} \frac{\dd{\theta}}{\sqrt{\frac{2g\ell}{\ell^2}(\cos \theta - \cos \theta_0)}}  \\
	  & = 4\sqrt{\frac{\ell}{g}} \int_0^{\theta_0} \frac{\dd{\theta}}{\sqrt{2\cos \theta - 2 \cos \theta_0}} \\
	  & = 4\sqrt{\frac{\ell}{g}} F(\theta_0)
\end{align*}
where \(f\) is notably a function only of \(\theta_0\).
Recall from the dimensional analysis lecture that
\[
	P = \sqrt{\frac{l}{g}}H\left( \frac{d}{\ell} \right)
\]
noting that \(d/\ell\) and \(\theta\) both define the initial condition.
So we have deduced this unknown function \(H\).
This integral is difficult to compute exactly; however, we can compute an approximation when \(\theta_0\) (and hence \(\theta\)) is small.
\begin{align*}
	F(\theta_0) & = \int_0^{\theta_0} \frac{\dd{\theta}}{\sqrt{\theta_0^2 - \theta^2}} \\
	            & = \frac{\pi}{2}
\end{align*}
which is independent of \(\theta_0\).
Hence, for small angles,
\[
	P \approx 2\pi \sqrt{\frac{\ell}{g}}
\]

\subsection{Force and potential in three spatial dimensions}
Consider a particle moving in three spatial dimensions under a force \(\vb F\).
Then Newton's second law states
\[
	m \ddot {\vb r} = \vb F
\]
We define the kinetic energy by
\[
	T = \frac{1}{2}\abs{\dot {\vb r}}^2 = \frac{1}{2}\abs{\vb u}^2
\]
Then
\[
	\frac{\dd{T}}{\dd{t}} = m \dot {\vb r} \cdot \ddot {\vb r} = \vb F \cdot \dot{\vb r} = \vb F \cdot \vb u
\]
This is the rate of working of the force on the particle.
Let us consider the total work done by a force on a particle as it moves along a finite curve \(C\) from \(t_1\) to \(t_2\).
Then the total work done is the line integral
\[
	W = \int_{t_1}^{t_2} \vb F \cdot \vb u \dd{t}
	= \int_{t_1}^{t_2} \vb F \cdot \dot {\vb r} \dd{t}
	= \int_{\vb r(t_1)}^{\vb r(t_2)} \vb F \cdot \dd \vb r
\]
Note that we must specify that this integral acts along the curve \(C\), since any other curve could connect the points \(\vb r(t_1)\) and \(\vb r(t_2)\).
We can write this integral in terms of coordinates:
\[
	\int_{\vb r(t_1)}^{\vb r(t_2)} F_x \dd{x} + F_y \dd{y} + F_z \dd{z}
\]
Now, if force is only a function of the position \(\vb r\), then we say that \(\vb F(\vb r)\) defines a force field.
A \textit{conservative} force field is such that
\[
	\vb F(\vb r) = -\grad V(\vb r)
\]
for some function \(V(\vb r)\).
In component form, this is equivalent to
\[
	F_i = -\frac{\partial V}{\partial x_i}
\]
If the force is conservative, then the energy \(E = T + V(\vb r)\) is conserved.
\begin{proof}
	\[
		\frac{\dd{E}}{\dd{t}} = \frac{\dd{T}}{\dd{t}} + \frac{\dd}{\dd{t}}V(\vb r) = m \dot{\vb r} \cdot \ddot{\vb r} + \grad V \cdot \dot{\vb r} = m\dot{\vb r} \cdot \ddot{\vb r} - m\ddot{\vb r} \cdot \dot{\vb r} = 0
	\]
\end{proof}
Let us consider the total work done on the particle under a conservative force.
From the properties of the gradient vector,
\[
	W = \int_C \vb F \cdot \dd \vb r = -\int_C \grad V \cdot \dd \vb r = V(\vb r_1) - V(\vb r_2)
\]
Note that this is dependent only on the end points of the curve; it is irrelevant of the path taken.
Hence, if \(C\) is closed, then no net work is done by the force.
Note that in general, \(F(\vb r)\) is not conservative, so in general there is no \(V(\vb r)\) such that \(\vb F = -\grad V\).
In fact, \(\vb F(\vb r)\) is conservative if
\[
	\grad \times \vb F(\vb r) = \vb 0
\]
