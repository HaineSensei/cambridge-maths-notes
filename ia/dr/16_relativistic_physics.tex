\subsection{Proper time}
A particle moves along a trajectory \(\vb x(t)\).
The velocity of this particle is \(\dv{\vb x}{t} = \vb u(t)\).
The path in spacetime is parametrised by \(t\).
Both \(\vb x\) and \(t\) vary under a Lorentz transformation.
Now, consider a particle at rest in \(S'\) with \(\vb x' = 0\).
The invariant interval on the world line is
\[
	\Delta s^2 = c^2 \Delta t^2
\]
We define the proper time \(\tau\) as
\[
	\Delta \tau = \frac{1}{c}\Delta s
\]
In particular, in \(S'\), \(\Delta\tau = \Delta t\), so the proper time is the time experienced in the rest frame of the particle.
However, the equation \(\Delta \tau = \frac{1}{c}\Delta s\) holds in all frames, since \(\Delta s\) is Lorentz invariant.
Note further that \(\Delta s\) is real since this always represents a timelike interval, as it represents a particle travelling through spacetime.
We can therefore instead parametrise this particle's world line by its proper time, rather than by considering the time in any particular frame.
So \(\vb x\) and \(t\) are both functions of \(\tau\) in any given reference frame.
Further, infinitesimal changes are related by
\begin{align*}
	\dd{\tau}               & = \frac{\dd{s}}{c}                                       \\
	                        & = \frac{1}{c}\sqrt{c^2\dd{t}^2 - \abs{\dd{\vb x}}^2}     \\
	                        & = \frac{1}{c}\sqrt{c^2\dd{t}^2 - \abs{\vb u}^2 \dd{t}^2} \\
	                        & = \qty(1 - \frac{\vb u^2}{c^2})^{\frac{1}{2}}\dd{t}      \\
	\therefore\ \dv{t}{\tau} & = \gamma_{\vb u}
\end{align*}
where \(\gamma_{\vb u} = \qty(1 - \frac{\vb u^2}{c^2})^{\frac{1}{2}}\).
Now, the total time observed by a particle moving along its world line is
\[
	T = \int \dd{\tau} = \int \frac{\dd{t}}{\gamma_{\vb u}}
\]

\subsection{4-velocity}
We can parametrise the position 4-vector of a particle using \(\tau\), written
\[
	X(\tau) = \begin{pmatrix}
		ct(\tau) \\ \vb x(\tau)
	\end{pmatrix}
\]
We define the 4-velocity as
\[
	U = \dv{\tau}X = \begin{pmatrix}
		c\dv*{t}{\tau} \\ \dv*{\vb x}{\tau}
	\end{pmatrix} = \dv{t}{\tau} \begin{pmatrix}
		c \\ \vb u
	\end{pmatrix} = \gamma_{\vb u} \begin{pmatrix}
		c \\ \vb u
	\end{pmatrix}
\]
Since \(X' = \Lambda X\), we also have that
\[
	U' = \Lambda U
\]
because \(\tau\) is invariant.
Note that any quantity whose components transform according to this rule is called a 4-vector, and in particular, the derivative of a 4-vector with respect to an invariant is also a 4-vector.
Also, the scalar product \(U \cdot U\) is invariant under Lorentz transforms.
Indeed, in the rest frame of a particle moving with 4-velocity \(U\), in this frame we have \(U\cdot U = c^2\).
In other frames,
\[
	U \cdot U = \gamma^2 (c^2 - \vb u^2) = c^2
\]
as expected.

\subsection{Transformation of velocities}
We have found that in special relativity, we cannot simply add velocities together.
Consider a transformation \(\Lambda\) from \(S\) to \(S'\), where \(S'\) is moving (relative to \(S\)) at a speed \(v\) in the \(x\) direction.
Consider a particle moving in \(S\) at speed \(u\) at an angle \(\theta\) to the \(x\) axis (with no component in the \(z\) axis).
In \(S'\), it moves with speed \(u'\) at an angle \(\theta'\).
We can write the 4-velocities as
\[
	U = \begin{pmatrix}
		\gamma_{\vb u}c           \\
		\gamma_{\vb u}u\cos\theta \\
		\gamma_{\vb u}u\sin\theta \\
		0
	\end{pmatrix};\quad U' = \begin{pmatrix}
		\gamma_{\vb u'}c             \\
		\gamma_{\vb u'}u'\cos\theta' \\
		\gamma_{\vb u'}u'\sin\theta' \\
		0
	\end{pmatrix}
\]
and further,
\[
	U' = \Lambda U
\]
where
\[
	\Lambda = \begin{pmatrix}
		\gamma_v              & -\gamma_v \frac{v}{c} & 0 & 0 \\
		-\gamma_v \frac{v}{c} & \gamma_v              & 0 & 0 \\
		0                     & 0                     & 1 & 0 \\
		0                     & 0                     & 0 & 1
	\end{pmatrix}
\]
Carrying out the matrix multiplication, we find
\[
	\begin{pmatrix}
		\gamma_{\vb u'}c             \\
		\gamma_{\vb u'}u'\cos\theta' \\
		\gamma_{\vb u'}u'\sin\theta' \\
		0
	\end{pmatrix} = \begin{pmatrix}
		\gamma_v              & -\gamma_v \frac{v}{c} & 0 & 0 \\
		-\gamma_v \frac{v}{c} & \gamma_v              & 0 & 0 \\
		0                     & 0                     & 1 & 0 \\
		0                     & 0                     & 0 & 1
	\end{pmatrix} \begin{pmatrix}
		\gamma_{\vb u}c           \\
		\gamma_{\vb u}u\cos\theta \\
		\gamma_{\vb u}u\sin\theta \\
		0
	\end{pmatrix} \implies \left\{ \begin{array}{l}
		\displaystyle
		u'\cos\theta' = \frac{u\cos\theta - v}{1 - uv\cos\theta/c^2}      \\
		\displaystyle
		\tan\theta' = \frac{u\sin\theta}{\gamma_{\vb u}(u\cos\theta - v)} \\
	\end{array} \right.
\]
The first equation corresponds to the normal transformation law for Lorentz transforms.
The second equation, corresponding to a change in angle due to the motion of the observer, is called aberration.
In particular, when \(u = c\), we can see that light rays appear to change direction due to the relative motion of the emitter and the observer.

\subsection{Energy-momentum 4-vector}
We define the 4-momentum of a particle of mass \(m\) and 4-velocity \(U\) to be
\[
	P = mU = m\gamma_{\vb u} \begin{pmatrix}
		c \\ \vb u
	\end{pmatrix}
\]
Since \(U\) is a 4-vector, we must have that \(m\) is invariant under a Lorentz transformation.
We will call this \(m\) the `rest mass' of the object, defined as the mass as measured in the rest frame of the particle.
The 4-momentum of a system of particles is defined as the sum of the 4-momenta of its individual particles.
The spatial components of \(P\), given by \(\mu = 1, 2, 3\), can be referred to as the relativistic 3-momentum, given by \(\vb p = \gamma_{\vb u} m \vb u\).
This matches with the definition as seen in Newtonian physics, except that the mass \(m\) is replaced by \(\gamma_{\vb u} m\).
We call this quantity the `apparent mass' of the particle or system of particles, as it represents the mass of the particle as observed by a different reference frame.
Note that \(\abs{\vb p}\) and \(\gamma_{\vb u} m\) both tend to infinity as the particle approaches the speed of light.
Note that the first component of \(P\), \(P^0\), is
\[
	\gamma_{\vb u} mc = \frac{mc}{\sqrt{1 - \frac{\vb u^2}{c^2}}} = \frac{1}{c}\qty(mc^2 + \frac{1}{2}m\vb u^2 + \dots)
\]
We recognise the \(\frac{1}{2}m\vb u^2\) term as the kinetic energy of the particle.
We interpret \(P^0\) as an energy, divided by \(c\) (to conserve units).
\[
	P = \begin{pmatrix}
		\frac{1}{c} E \\ \vb p
	\end{pmatrix}
\]
where
\[
	E = \gamma_{\vb u} mc^2 = mc^2 + \frac{1}{2}m\vb u^2 + \dots
\]
Note that as \(\abs{\vb u} \to c\), \(E \to \infty\).
Since \(P\) contains an energy term as well as a momentum term, we also call \(P\) the energy-momentum 4-vector.
Note that for a stationary particle of rest mass \(m\), we have
\[
	E = mc^2
\]
This implies that mass is a form of energy.
The energy of a moving particle is
\[
	E = mc^2 + \underbrace{(\gamma_{\vb u} - 1)mc^2}_{\mathclap{\text{relativistic kinetic energy}}}
\]
Since \(P \cdot P = \frac{E^2}{c^2} - \abs{\vb p}^2\) is Lorentz invariant, we have
\[
	P \cdot P = m^2c^2
\]
Hence,
\[
	\frac{E^2}{c^2} = \abs{\vb p}^2 + m^2c^2
\]
In Newtonian physics, mass is conserved, and energy is also conserved.
In relativistic physics, mass is not conserved by itself, since it is a form of energy.
From this derivation, it is theoretically possible to convert between mass and kinetic energy.

\subsection{Massless particles}
A massless particle has zero rest mass.
Such particles can have nonzero momentum and nonzero energy, because they are travelling at the speed of light, giving \(\gamma_{\vb u} = \infty\).
Since \(P \cdot P = m^2c^2\), there are no factors of \(\gamma\) in this expression giving
\[
	P \cdot P = 0
\]
So such a particle travels along a light-like trajectory.
Therefore there is no Lorentz transformation that brings a given reference frame into the rest frame of the particle, so we cannot define proper time for such a particle.
Since \(m^2c^2 = 0\), we must have
\[
	\frac{E^2}{c^2} = \abs{\vb p}^2 \implies E = \abs{\vb p}c
\]
Then,
\[
	P = \frac{E}{c} \begin{pmatrix}
		1 \\ \vu n
	\end{pmatrix}
\]
where \(\vu n\) is a unit 3-vector in the direction of travel of the particle.

\subsection{Newton's second law}
Now that we have defined \(P\) for all particles, we can rewrite Newton's second law in special relativity as
\[
	\dv{P}{\tau} = F
\]
where \(F\) is the 4-force.
If the 3-force is \(\vb F\), we have
\[
	F = \gamma_{\vb u} \begin{pmatrix}
		\vb F \cdot \vb u / c \\
		\vb F
	\end{pmatrix}
\]
Hence,
\[
	\dv{E}{\tau} = \gamma_{\vb u} \vb F \cdot \vb u;\quad \dv{\vb p}{\tau} = \gamma_{\vb u} \vb F
\]
giving
\[
	\dv{E}{t} = \vb F \cdot \vb u;\quad \dv{\vb p}{t} = \vb F
\]
which are the familiar Newtonian expressions for rate of work and rate of change of momentum.
We can now define 4-acceleration:
\[
	F = mA
\]
where \(m\) is the rest mass.
Hence,
\[
	\dv{U}{\tau} = A
\]

\subsection{Special relativity with particle physics}
In Newtonian physics, when two particles collide, we must consider the conservation of 3-momentum.
In special relativity however, we must instead consider the conservation of 4-momentum:
\[
	P = \begin{pmatrix}
		\frac{E}{c} \\ \vb p
	\end{pmatrix}
\]
It is often convenient, when dealing with systems of particles, to let the origin of our frame of reference be the centre of momentum.
This is the frame such that the total 3-momentum of the system is zero.
However, this cannot be done when dealing with massless particles since there does not exist such a rest frame.

\subsection{Particle decay}
Consider a particle of mass \(m_1\) with 3-momentum \(\vb p_1\) which decays into two particles of mass \(m_2\) and \(m_3\) with 3-momenta \(\vb p_2, \vb p_3\).
Since 4-momentum is conserved, we get \(P_1 = P_2 + P_3\).
First, consider the 0 component (the timelike component) of \(P\).
\[
	E_1 = E_2 + E_3
\]
Now, consider the \(1, 2, 3\) components (the spacelike components) of the 4-momentum.
We have
\[
	\vb p_1 = \vb p_2 + \vb p_3
\]
Let us look at this in the centre of momentum frame, so \(\vb p_1 = 0\).
Hence
\[
	\vb p_2 = -\vb p_3
\]
Because we are in the centre of momentum frame, we have \(E_1 = m_1 c^2\) hence
\[
	\frac{E_1}{c} = m_1 c = \frac{E_2}{c} + \frac{E_3}{c}
\]
Further,
\[
	\frac{E_2}{c} = \sqrt{\vb p_2^2 + m_2^2 c^2};\quad \frac{E_3}{c} = \sqrt{\vb p_3^2 + m_3^2 c^2}
\]
Hence,
\[
	m_1 c = \sqrt{\vb p_2^2 + m_2^2 c^2} + \sqrt{\vb p_3^2 + m_3^2 c^2} \geq m_2 c + m_3 c
\]
Hence the rest mass of the initial particle must be \textit{at least} the sum of the rest masses of the particles that result from the decay.

\subsection{Higgs to photon decay}
Consider the decay of the Higgs particle \(h\) into two photons \(\gamma_1, \gamma_2\).
By conservation of 4-momentum,
\[
	P_h = P_{\gamma_1} + P_{\gamma_2}
\]
In the Higgs rest frame,
\[
	P_h = \begin{pmatrix}
		m_h c \\ \vb 0
	\end{pmatrix} =
	\begin{pmatrix}
		\frac{E_{\gamma_1}}{c} \\ \vb p_{\gamma_1}
	\end{pmatrix}
	+
	\begin{pmatrix}
		\frac{E_{\gamma_2}}{c} \\ \vb p_{\gamma_2}
	\end{pmatrix}
\]
Looking at the \(1, 2, 3\) components we find
\[
	\vb p_{\gamma_1} = -\vb p_{\gamma_2}
\]
Looking at the 0 component we find
\[
	m_h c = \frac{E_{\gamma_1}}{c} + \frac{E_{\gamma_2}}{c}
\]
Since \(\frac{E^2}{c^2} = \vb p^2 + m^2c^2\), because the photons have zero rest mass we have
\[
	\frac{E_{\gamma_1}}{c} = \abs{\vb p_{\gamma_1}} = \abs{\vb p_{\gamma_2}} = \frac{E_{\gamma_2}}{c}
\]
Hence,
\[
	E_{\gamma_1} = E_{\gamma_2} = \frac{1}{2}m_h c^2
\]
Note that mass has been lost, but kinetic energy has been gained.

\subsection{Particle scattering}
Consider two identical particles colliding, without decaying into new particles.
In frame \(S\), particle 1 is moving horizontally with 3-velocity \(\vb u\), and particle 2 starts at rest.
After the collision, particle 1 has 3-velocity \(\vb q\) and particle 2 has 3-velocity \(\vb r\), where \(\vb q\) has angle \(\theta\) to the horizontal and \(\vb r\) has angle \(\phi\) to the horizontal.
In the centre of momentum frame \(S'\), particles 1 and 2 move towards each other horizontally with 3-momenta \(\vb p_1\) and \(\vb p_2 = -\vb p_1\).
After the collision, particle 1 moves with 3-momentum \(\vb p_3\) and particle 2 moves with 3-momentum \(\vb p_4 = -\vb p_3\).
The angle of deflection is \(\theta'\).
By conservation of 4-momentum,
\[
	P_1 + P_2 = P_3 + P_4
\]
Since particles 1 and 2 have the same mass, their speeds (in \(S'\)) are equal both before and after the collision.
Let the speed before the collision be \(v\) and the speed after the collision be \(w\).
\[
	P_1' = \begin{pmatrix}
		m\gamma_v c \\
		m\gamma_v v \\
		0           \\
		0
	\end{pmatrix};\quad P_2' = \begin{pmatrix}
		m\gamma_v c  \\
		-m\gamma_v v \\
		0            \\
		0
	\end{pmatrix};\quad P_3' = \begin{pmatrix}
		m\gamma_w c             \\
		m\gamma_w w \cos\theta' \\
		m\gamma_w w \sin\theta' \\
		0
	\end{pmatrix};\quad P_4' = \begin{pmatrix}
		m\gamma_w c              \\
		-m\gamma_w w \cos\theta' \\
		-m\gamma_w w \sin\theta' \\
		0
	\end{pmatrix}
\]
Looking at the 0 component,
\[
	2 m\gamma_v c = 2m\gamma_w c
\]
Since \(m\) is the same on both sides,
\[
	v = w
\]
Now we will apply a Lorentz transformation to return to \(S\).
\[
	\Lambda = \begin{pmatrix}
		\gamma_v             & \gamma_v \frac{v}{c} & 0 & 0 \\
		\gamma_v \frac{v}{c} & \gamma_v             & 0 & 0 \\
		0                    & 0                    & 1 & 0 \\
		0                    & 0                    & 0 & 1
	\end{pmatrix}
\]
Now, since \(u\) is the initial velocity of particle 1 in \(S\),
\[
	P_1 = \Lambda P_1' = \begin{pmatrix}
		m\gamma_v^2 \qty(c + \frac{v^2}{c}) \\
		m\gamma_v^2 (v+v)                   \\
		0                                   \\
		0
	\end{pmatrix} = \begin{pmatrix}
		m\gamma_u c \\
		m\gamma_u u \\
		0           \\
		0
	\end{pmatrix}
\]
After the collision, as seen in \(S\), particle 1's 4-momentum is
\[
	P_3 = \Lambda P_3' = \begin{pmatrix}
		m\gamma_v^2 \qty(c + \frac{v^2}{c}\cos\theta') \\
		m\gamma_v^2 \qty(v + v\cos\theta')             \\
		m\gamma_v v\sin\theta'                         \\
		0
	\end{pmatrix} = \begin{pmatrix}
		m\gamma_q c           \\
		m\gamma_q q\cos\theta \\
		m\gamma_q q\sin\theta \\
		0
	\end{pmatrix}
\]
By dividing the 1 and 2 components on both sides, we deduce
\[
	\tan\theta = \frac{m\gamma_v v\sin\theta'}{m\gamma_v^2 v(1 + \cos\theta')} = \frac{1}{\gamma_v} \tan\frac{1}{2}\theta'
\]
For the second particle, we can do the same calculation to get
\[
	\tan\phi = \frac{m\gamma_v v\sin\theta'}{m\gamma_v^2 v(1 - \cos\theta')} = \frac{1}{\gamma_v} \cot\frac{1}{2}\theta'
\]
So given the knowledge of the exact setup of the particles, we can find the angles between the particles as viewed in a different reference frame.
In particular,
\[
	\tan\theta \cdot \tan\phi = \frac{1}{\gamma_v^2} = \frac{2}{1+\gamma_u} \leq 1
\]
This is a generalisation of the Newtonian result, where \(\gamma_u = 1\) giving
\[
	\tan\theta \cdot \tan\phi = 1
\]
So the angle between the trajectories in the Newtonian case is \(\frac{\pi}{2}\).

\subsection{Particle creation}
Consider equal particles 1 and 2 of mass \(m\) moving towards each other horizontally with speed \(v\) in \(S\), with 4-momenta \(P_1\) and \(P_2\).
After the collision, particles 1 and 2 have 4-momenta \(P_3\) and \(P_4\), and a new particle 3 with 4-momentum \(P_5\) is created with mass \(M\).
Note that \(S\) is the centre of momentum frame.
By conservation of 4-momentum, we have
\[
	P_1 + P_2 = P_3 + P_4 + P_5
\]
We have
\[
	P_2 + P_2 = \begin{pmatrix}
		2m\gamma_v c \\ \vb 0
	\end{pmatrix} = \begin{pmatrix}
		\frac{E_3}{c} + \frac{E_4}{c} + \frac{E_5}{c} \\
		\vb 0
	\end{pmatrix}
\]
Certainly we have
\[
	2m\gamma_v c^2 = E_3 + E_4 + E_5 \geq (m + m + M)c^2 = (2m + M)c^2
\]
Hence, for the particle's creation to be possible, we must have
\[
	\gamma_v \geq 1 + \frac{M}{2m}
\]
So the initial kinetic energy in \(S\) must satisfy
\[
	2m(\gamma_v - 1)c^2 \geq Mc^2
\]
Consider some other reference frame \(S'\) where one particle moves with speed \(u\) and the other is at rest.
Then
\[
	u = \frac{2v}{1 + \frac{v^2}{c^2}}
\]
Hence, by the result above in the particle scattering experiment,
\[
	\gamma_u = 2(\gamma_v^2 - 1) \geq 2\qty(1 + \frac{M}{2m})^2 - 1 = 1 + \frac{2M}{m} + \frac{M^2}{2m^2}
\]
Hence, in this frame, the kinetic energy \(mc^2(\gamma_u - 1)\) must satisfy
\[
	mc^2(\gamma_u - 1) \geq mc^2\qty(\frac{2M}{m} + \frac{M^2}{2m^2}) \geq 2Mc^2 + \frac{M^2c^2}{2m}
\]
This extra \(\frac{M^2c^2}{2m}\) term (compared to the \(Mc^2\) expression in \(S\)) is produced by the transformation between frames.
So in a frame where one particle is at rest, we require significantly more kinetic energy.
So a particle accelerator is most efficiently utilised by accelerating two particles into each other, rather than by accelerating one particle into a fixed target.
