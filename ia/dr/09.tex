\subsection{Motion in a Central Force Field}
By Newton's second law, the force in a central force field is given by
\[ m\ddot{\vb r} = \vb F = -\grad V = -\frac{\dd{V}}{\dd{r}} \vb e_r \]
The results from the previous lecture give
\begin{equation}
	m\left(\ddot r - r\dot\theta^2\right) \vb e_r + m\left(2\dot r \dot \theta + r \ddot \theta\right) \vb e_\theta = \frac{-\dd{V}}{\dd{r}} \vb e_r \tag{$\ast$}
\end{equation}
But the right hand side has no angular component, so $m\left(2\dot r \dot \theta + r \ddot \theta\right) = 0$. Then
\[ \frac{m}{r} \frac{\dd}{\dd{t}}\left(r^2 \dot\theta\right) = 0 \]
So the quantity $h = r^2 \dot\theta$ is constant. Note that the angular momentum $\vb L$ is given by
\[ \vb L = m\vb r \times \dot{\vb r} = mr\vb e_r \times \left( \dot r\vb e_r + r\dot\theta \vb e_\theta \right) = mr^2 \dot\theta \vb e_z \]
Hence the magnitude of the angular momentum is constant. Now, let us consider the radial component in $(\ast)$.
\begin{align*}
	m\ddot r - mr\dot\theta^2 & = -\frac{\dd{V}}{\dd{r}}                    \\
	m\ddot r                  & = -\frac{\dd{V}}{\dd{r}} + \frac{mh^2}{r^3} \\
	m\ddot r                  & = -\frac{\dd{V}_\text{eff}}{\dd{r}}
\end{align*}
where
\[ V_\text{eff}(r) = V(r) + \frac{mh^2}{2r^2} \]
where $V_\text{eff}$ is called the effective potential. In other words, the motion of the particle is equivalent to one-dimensional motion under the influence of the effective potential. The energy of the particle is given by
\[ T + V(r) = \frac{1}{2}m\left( \dot r^2 + r^2 \dot\theta^2 \right) + V(r) = \frac{1}{2}m\dot r^2 + \frac{mh^2}{2r^2} + V(r) = \frac{1}{2}m\dot r^2 + V_\text{eff}(r) \]
which is consistent with our description of the effective potential.

\subsection{Orbits under Gravity}
As an example, let us consider
\[ V(r) = \frac{-GMm}{r};\quad V_\text{eff}(r) = \frac{-GMm}{r} + \frac{mh^2}{2r^2} \]
The effective potential has a single minimum point at $r_\ast$, and a single root at $r_0$. In other words, $V'_\text{eff}(r_\ast) = 0$ and $V_\text{eff}(r_0) = 0$. We can compute that
\[ r_0 = \frac{h^2}{2GM};\quad r_\ast = \frac{h^2}{GM} \]
The minimum energy is therefore
\[ E_\text{min} = \frac{-m(GM)^2}{2h^2} \]
What is the possible motion of the particle? At $E = E_\text{min}$, we have $r(t) = r_\ast$, an equilibrium position. Further, $\dot\theta = \frac{h}{r_\ast^2}$ everywhere. At $E_\text{min} < E < 0$, then $r(t)$ oscillates between a minimum point (periapsis/perihelion/perigee) and a maximum point (apoapsis/aphelion/apogee), and $\dot\theta$ varies. If $E_\text{min} \geq 0$, the particle escapes to infinity. This is sometimes called an unbound orbit.

\subsection{Stability of Circular Orbits}
Consider a general potential $V(r)$. Does a circular orbit exist, and is it stable? We will assume that the angular momentum is given and non-zero. For a circular orbit, the radius is a constant value $r_\ast$, so $\ddot r = 0$ and hence $V'_\text{eff}(r_\ast) = 0$. We know that we have a stable equilibrium if $V_\text{eff}$ has a minimum at this point. Correspondingly, it is unstable if this is a maximum. So, for instance, it is stable if $V''_\text{eff}(r_\ast) > 0$. Now, let us rewrite these conditions in terms of $V(r)$.
\[ V'(r_\ast) - \frac{mh^2}{r_\ast^3} = 0;\quad V''(r_\ast) = V''(r_\ast) + \frac{3mh^2}{r_\ast^4} > 0 \]
We can combine these to give the condition for stability as
\[ V''(r_\ast) + \frac{3V'(r_\ast)}{r_\ast} > 0 \]
Now let us consider an example,
\[ V(r) = \frac{-km}{r^p} \]
where $p>0$, $k>0$. If $p=1$, this is an example of an inverse square law. We have a circular orbit if
\[ \frac{pkm}{r_\ast^{p+1}} - \frac{mh^2}{r_\ast^3} = 0 \]
Hence,
\[ r_\ast^{p-2} = \frac{pkm}{mh^2} \implies r_\ast = \left( \frac{pkm}{mh^2} \right)^{\frac{1}{p-2}} \]
So there exists a circular orbit for all $h$ provided $p \neq 2$. Is this a stable orbit?
\[ V''(r_\ast) + \frac{3V'(r_\ast)}{r_\ast} = \frac{-kmp(p+1)}{r_\ast^{p+2}} + \frac{3kmp}{r_\ast^{p+2}} = \frac{p(2-p)km}{r_\ast^{p+2}} \]
So this is greater than zero (stable) if $0 < p < 2$ and less than zero (unstable) if $p > 2$.

\subsection{The Orbit Equation}
What shape does a non-circular orbit trace out? We could in principle find $r(t)$ by the energy equation
\[ E = \frac{1}{2}m \dot r^2 + V_\text{eff}(r) = \text{constant} \]
Hence
\[ t = \pm \sqrt{\frac{m}{2}} \int^r \frac{\dd{u}}{\sqrt{E - V_\text{eff}(u)}} \]
Then we can use $r(t)^2\dot \theta = h$ to deduce $\theta(t)$. However in practice, this is not useful. An analytic solution is only possible for a small family of effective potential functions. It is somewhat more convenient to find $r$ in terms of $\theta$, not in terms of $t$. We can write
\[ \frac{\dd}{\dd{t}} = \frac{\dd \theta}{\dd{t}} \frac{\dd}{\dd \theta} = \frac{h}{r^2} \frac{\dd}{\dd \theta} \]
Applying this to Newton's second law, we have
\[ m\frac{h}{r^2} \frac{\dd}{\dd \theta}\left( \frac{h}{r^2} \frac{\dd}{\dd \theta} r \right) - \frac{mh^2}{r^3} = F(r) \]
The $\frac{h}{r^2} \frac{\dd}{\dd \theta} r$ term suggests using the substitution $u = \frac{1}{r}$. Then
\[ mhu^2 \frac{\dd}{\dd \theta}\left( -h\frac{\dd{u}}{\dd \theta} \right) - mh^2u^3 = F(u^{-1}) \]
\[ \frac{\dd^2 u}{\dd \theta^2} + u = \frac{-1}{mh^2u^2}F(u^{-1}) \]
This is known as the orbit equation. We can solve this for $u$ as a function of $\theta$.
