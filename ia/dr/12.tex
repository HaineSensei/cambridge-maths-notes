\subsection{Centrifugal force}
\begin{align*}
	-m\vb\omega \times (\vb\omega \times \vb r) & = -m((\vb\omega \cdot \vb r)\vb\omega - \vb\omega^2\vb r) \\
	                                            & = m\omega^2(\vb r - \vu\omega(\vu\omega \cdot \vb r))     \\
	                                            & = m\omega^2 \vb r_\perp
\end{align*}
where \(\vb r_\perp\) is the component of \(\vb r\) which is perpendicular to \(\vb\omega\).
Note that \(\abs{\vb r_\perp}\) is the perpendicular distance from the point \(\vb r\) to the axis of rotation, and \(\vb r_\perp\) is directed away from this axis.
Hence the centrifugal force is always directed away from the rotation axis, perpendicular to it, with its magnitude proportional to the particle's distance from the axis.
Note that
\[
	\vb r_\perp^2 = \vb r^2 - (\vb r \cdot \vu\omega)^2 = \abs{\vu\omega \times \vb r}^2
\]
And
\[
	\grad \vb r_\perp^2 = 2\vb r - 2\vu\omega(\vu\omega \cdot \vb r) = 2\vb r_\perp
\]
Hence,
\[
	m\omega^2\vb r_\perp = \grad(\frac{1}{2}m\omega^2\vb r_\perp^2)
\]
Therefore the centrifugal force is conservative.
On a rotating planet such as the earth, it is often convenient to combine the centrifugal force and the gravitational force into an `effective gravity'
\[
	\vb g_{\text{eff}} = \vb g + \omega^2 \vb r_\perp
\]
As an example, consider a spherical planet which rotates through an axis through a pole.
Point \(P\) is at latitude \(\lambda\), i.e.\ it is \(\lambda\) radians above the equator.
On this point, we have \(\vu z\) normal to the Earth's surface, \(\vu y\) in the north direction parallel to the surface and \(\vu x\) in the east direction parallel to the surface.
The earth has radius \(R\).
Now,
\[
	\vb r = R\vu z;\quad \vb\omega = \omega(\vu y \cos\lambda + \vu z \sin\lambda)
\]
Hence,
\begin{align*}
	\vb g_\text{eff} & = -g\vu z + \omega^2 \vb r_\perp                                              \\
	                 & = -g\vu z + \omega^2 R\cos\lambda (\vu z \cos\lambda - \vu y \sin\lambda)     \\
	                 & = \vu z(\omega^2 R\cos^2\lambda - g) - \vu y(\omega^2R\cos\lambda\sin\lambda)
\end{align*}
The angle \(\alpha\) between \(\vb g_\text{eff}\) and \(\vu z\) is
\[
	\alpha = \arctan \frac{\omega^2R\cos\lambda\sin\lambda}{g - \omega^2R\cos^2\lambda}
\]
For earth, \(\omega = \frac{2\pi}{86400} \approx \SI{7.3e-5}{\per\second}\) and \(R \approx \SI{6.4e6}{\metre}\), hence \(\frac{\omega^2R}{g} \approx \num{3.5e-3}\).
Neglecting the \(\omega^2\) term in the denominator, \(\alpha\) is very small for the earth.

\subsection{Coriolis force}
\[
	-2 m\vb\omega \times \left( \dv{\vb r}{t} \right)_{S'} = -2m\vb\omega \times \vb v
\]
where \(\vb v\) is as observed in the rotating frame.
The force is proportional to, and perpendicular to, the velocity.
Consequently, this force does not do any work.
Considering the previous example of the earth, let us consider a velocity tangential to the surface of the planet, specifically \(\vb v = v_x \vu x + v_y \vu y\).
The angular velocity has components \(\vb \omega = \omega(\vu y \cos\lambda + \vu z \sin\lambda)\).
Hence,
\[
	-2m\vb\omega \times \vb v = \underbrace{2m\omega\sin\lambda(v_y \vu x - v_x \vu y)}_{\text{horizontal}} + \underbrace{2m\omega\cos\lambda(v_x \vu z)}_{\text{vertical}}
\]
The horizontal component of the Coriolis force gives an acceleration to the right of the horizontal velocity in the Northern hemisphere, and the acceleration is to the left in the southern hemisphere.
This appears due to the \(\sin\lambda\) term, where the sign changes depending on the hemisphere.

This force can be balanced by another force, notably a pressure gradient, which can be useful for predicting weather patterns in meteorology.
Hence, in the northern hemisphere, an area of low pressure implies an anticlockwise flow of fluid around it; in the southern hemisphere this would imply a clockwise flow of fluid.
This is called a cyclone.
A high-pressure environment (in either hemisphere), would have the opposite direction of flow, and can be called an anticyclone.

\subsection{Dropping a particle in a rotating frame}
As an example, let us consider dropping a ball from the top of a tower.
Where does it land, if we are in a rotating frame?
\[
	\rddot = \vb g - 2\vb\omega \times \rdot - \vb\omega \times (\vb\omega \times \vb r)
\]
We will assume the rotation is slow, i.e.\ \(\omega^2 R / g\) is small (we can accurately say `small' in this case since \(\omega^2R/g\) is a dimensionless constant).
\begin{align*}
	\rddot & = \vb g - 2\omega \times \rdot + o(\omega^2)                                                                                          \\
	\rdot  & = \vb gt - 2\omega \times \vb r + o(\omega^2) + \underbrace{2\omega\times \vb r_0}_{\mathclap{\text{to match the initial condition}}}
\end{align*}
Hence, neglecting \(o(\omega^3)\),
\begin{align*}
	\rddot & = \vb g - 2\omega \times \vb gt + o(\omega^2)                                       \\
	\vb r  & = \frac{1}{2}\vb gt^2 - \frac{1}{3}\vb\omega\times \vb gt^3 + \vb r_0 + o(\omega^2)
\end{align*}
Now, consider \(\vb g = (0, 0, -g)\) and \(\vb\omega = (0, \omega, 0)\), corresponding to the equator.
Let \(\vb r_0 = (0, 0, R + h)\).
Hence,
\[
	\vb r(t) = \qty(0, 0, -\frac{1}{2}gt^2) + \qty(\frac{1}{3}\omega g t^3, 0, 0) + (0, 0, R + h)
\]
The particle hits the ground when \(h = \frac{1}{2}gt^2\), hence \(t = \sqrt{2h/g}\), and the corresponding horizontal displacement is therefore
\[
	\Delta x = \frac{1}{3}\omega g \left( \frac{2h}{g} \right)^{\frac{3}{2}}
\]
So the particle hits the ground to the east of the tower's base.
This is consistent with conservation of angular momentum.

\subsection{Foucault pendulum}
Consider a pendulum at the north pole.
It will swing in the plane fixed in an inertial frame; the earth rotates relative to this frame.
From the point of view of an observer on the earth, the plane in which the pendulum moves is rotating to the west.

If we're at the north pole, the plane of rotation is observed to rotate once per day.
This can be explained using a fictitious force from the perspective of the rotating frame of reference of the pendulum.
At a general latitude \(\lambda\), the plane of rotation completes a circuit in \(\cosecant\lambda\) days.
We can derive this result by considering the dynamics of the pendulum under the Coriolis force.
