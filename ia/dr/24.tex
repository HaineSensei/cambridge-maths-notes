\subsection{Special Relativity with Particle Physics}
In Newtonian physics, when two particles collide, we must consider the conservation of 3-momentum.
In special relativity however, we must instead consider the conservation of 4-momentum:
\[
	P = \begin{pmatrix}
		\frac{E}{c} \\ \vb p
	\end{pmatrix}
\]
It is often convenient, when dealing with systems of particles, to let the origin of our frame of reference be the centre of momentum.
This is the frame such that the total 3-momentum of the system is zero.
However, this cannot be done when dealing with massless particles since there does not exist such a rest frame.

\subsection{Particle Decay}
Consider a particle of mass \(m_1\) with 3-momentum \(\vb p_1\) which decays into two particles of mass \(m_2\) and \(m_3\) with 3-momenta \(\vb p_2, \vb p_3\).
Since 4-momentum is conserved, we get \(P_1 = P_2 + P_3\).
First, consider the 0 component (the timelike component) of \(P\).
\[
	E_1 = E_2 + E_3
\]
Now, consider the \(1, 2, 3\) components (the spacelike components) of the 4-momentum.
We have
\[
	\vb p_1 = \vb p_2 + \vb p_3
\]
Let us look at this in the centre of momentum frame, so \(\vb p_1 = 0\).
Hence
\[
	\vb p_2 = -\vb p_3
\]
Because we are in the centre of momentum frame, we have \(E_1 = m_1 c^2\) hence
\[
	\frac{E_1}{c} = m_1 c = \frac{E_2}{c} + \frac{E_3}{c}
\]
Further,
\[
	\frac{E_2}{c} = \sqrt{\vb p_2^2 + m_2^2 c^2};\quad \frac{E_3}{c} = \sqrt{\vb p_3^2 + m_3^2 c^2}
\]
Hence,
\[
	m_1 c = \sqrt{\vb p_2^2 + m_2^2 c^2} + \sqrt{\vb p_3^2 + m_3^2 c^2} \geq m_2 c + m_3 c
\]
Hence the rest mass of the initial particle must be \textit{at least} the sum of the rest masses of the particles that result from the decay.

\subsection{Higgs to Photon Decay}
Consider the decay of the Higgs particle \(h\) into two photons \(\gamma\).
By conservation of 4-momentum,
\[
	P_h = P_{\gamma_1} + P_{\gamma_2}
\]
In the Higgs rest frame,
\[
	P_h = \begin{pmatrix}
		m_h c \\ \vb 0
	\end{pmatrix} =
	\begin{pmatrix}
		\frac{E_{\gamma_1}}{c} \\ \vb p_{\gamma_1}
	\end{pmatrix}
	+
	\begin{pmatrix}
		\frac{E_{\gamma_2}}{c} \\ \vb p_{\gamma_2}
	\end{pmatrix}
\]
Looking at the \(1, 2, 3\) components we find
\[
	\vb p_{\gamma_1} = -\vb p_{\gamma_2}
\]
Looking at the 0 component we find
\[
	m_h c = \frac{E_{\gamma_1}}{c} + \frac{E_{\gamma_2}}{c}
\]
Since \(\frac{E^2}{c^2} = \vb p^2 + m^2c^2\), because the photons have zero rest mass we have
\[
	\frac{E_{\gamma_1}}{c} = \abs{\vb p_{\gamma_1}} = \abs{\vb p_{\gamma_2}} = \frac{E_{\gamma_2}}{c}
\]
Hence,
\[
	E_{\gamma_1} = E_{\gamma_2} = \frac{1}{2}m_h c^2
\]
Note that mass has been lost, but kinetic energy has been gained.

\subsection{Particle Scattering}
Consider two identical particles colliding, without decaying into new particles.
In frame \(S\), particle 1 is moving horizontally with 3-velocity \(\vb u\), and particle 2 starts at rest.
After the collision, particle 1 has 3-velocity \(\vb q\) and particle 2 has 3-velocity \(\vb r\), where \(\vb q\) has angle \(\theta\) to the horizontal and \(\vb r\) has angle \(\phi\) to the horizontal.
In the centre of momentum frame \(S'\), particles 1 and 2 move towards each other horizontally with 3-momenta \(\vb p_1\) and \(\vb p_2 = -\vb p_1\).
After the collision, particle 1 moves with 3-momentum \(\vb p_3\) and particle 2 moves with 3-momentum \(\vb p_4 = -\vb p_3\).
The angle of deflection is \(\theta'\).
By conservation of 4-momentum,
\[
	P_1 + P_2 = P_3 + P_4
\]
Since particles 1 and 2 have the same mass, their speeds (in \(S'\)) are equal both before and after the collision.
Let the speed before the collision be \(v\) and the speed after the collision be \(w\).
\[
	P_1' = \begin{pmatrix}
		m\gamma_v c \\
		m\gamma_v v \\
		0           \\
		0
	\end{pmatrix};\quad P_2' = \begin{pmatrix}
		m\gamma_v c  \\
		-m\gamma_v v \\
		0            \\
		0
	\end{pmatrix};\quad P_3' = \begin{pmatrix}
		m\gamma_w c             \\
		m\gamma_w w \cos\theta' \\
		m\gamma_w w \sin\theta' \\
		0
	\end{pmatrix};\quad P_4' = \begin{pmatrix}
		m\gamma_w c              \\
		-m\gamma_w w \cos\theta' \\
		-m\gamma_w w \sin\theta' \\
		0
	\end{pmatrix}
\]
Looking at the 0 component,
\[
	2 m\gamma_v c = 2m\gamma_w c
\]
Since \(m\) is the same on both sides,
\[
	v = w
\]
Now we will apply a Lorentz transformation to return to \(S\).
\[
	\Lambda = \begin{pmatrix}
		\gamma_v             & \gamma_v \frac{v}{c} & 0 & 0 \\
		\gamma_v \frac{v}{c} & \gamma_v             & 0 & 0 \\
		0                    & 0                    & 1 & 0 \\
		0                    & 0                    & 0 & 1
	\end{pmatrix}
\]
Now, since \(u\) is the initial velocity of particle 1 in \(S\),
\[
	P_1 = \Lambda P_1' = \begin{pmatrix}
		m\gamma_v^2 \qty(c + \frac{v^2}{c}) \\
		m\gamma_v^2 (v+v)                   \\
		0                                   \\
		0
	\end{pmatrix} = \begin{pmatrix}
		m\gamma_u c \\
		m\gamma_u u \\
		0           \\
		0
	\end{pmatrix}
\]
After the collision, as seen in \(S\), particle 1's 4-momentum is
\[
	P_3 = \Lambda P_3' = \begin{pmatrix}
		m\gamma_v^2 \qty(c + \frac{v^2}{c}\cos\theta') \\
		m\gamma_v^2 \qty(v + v\cos\theta')             \\
		m\gamma_v v\sin\theta'                         \\
		0
	\end{pmatrix} = \begin{pmatrix}
		m\gamma_q c           \\
		m\gamma_q q\cos\theta \\
		m\gamma_q q\sin\theta \\
		0
	\end{pmatrix}
\]
By dividing the 1 and 2 components on both sides, we deduce
\[
	\tan\theta = \frac{m\gamma_v v\sin\theta'}{m\gamma_v^2 v(1 + \cos\theta')} = \frac{1}{\gamma_v} \tan\frac{1}{2}\theta'
\]
For the second particle, we can do the same calculation to get
\[
	\tan\phi = \frac{m\gamma_v v\sin\theta'}{m\gamma_v^2 v(1 - \cos\theta')} = \frac{1}{\gamma_v} \cot\frac{1}{2}\theta'
\]
So given the knowledge of the exact setup of the particles, we can find the angles between the particles as viewed in a different reference frame.
In particular,
\[
	\tan\theta \cdot \tan\phi = \frac{1}{\gamma_v^2} = \frac{2}{1+\gamma_u} \leq 1
\]
This is a generalisation of the Newtonian result, where \(\gamma_u = 1\) giving
\[
	\tan\theta \cdot \tan\phi = 1
\]
So the angle between the trajectories in the Newtonian case is \(\frac{\pi}{2}\).

\subsection{Particle Creation}
Consider equal particles 1 and 2 of mass \(m\) moving towards each other horizontally with speed \(v\) in \(S\), with 4-momenta \(P_1\) and \(P_2\).
After the collision, particles 1 and 2 have 4-momenta \(P_3\) and \(P_4\), and a new particle 3 with 4-momentum \(P_5\) is created with mass \(M\).
Note that \(S\) is the centre of momentum frame.
By conservation of 4-momentum, we have
\[
	P_1 + P_2 = P_3 + P_4 + P_5
\]
We have
\[
	P_2 + P_2 = \begin{pmatrix}
		2m\gamma_v c \\ \vb 0
	\end{pmatrix} = \begin{pmatrix}
		\frac{E_3}{c} + \frac{E_4}{c} + \frac{E_5}{c} \\
		\vb 0
	\end{pmatrix}
\]
Certainly we have
\[
	2m\gamma_v c^2 = E_3 + E_4 + E_5 \geq (m + m + M)c^2 = (2m + M)c^2
\]
Hence, for the particle's creation to be possible, we must have
\[
	\gamma_v \geq 1 + \frac{M}{2m}
\]
So the initial kinetic energy in \(S\) must satisfy
\[
	2m(\gamma_v - 1)c^2 \geq Mc^2
\]
Consider some other reference frame \(S'\) where one particle moves with speed \(u\) and the other is at rest.
Then
\[
	u = \frac{2v}{1 + \frac{v^2}{c^2}}
\]
Hence, by the result above in the particle scattering experiment,
\[
	\gamma_u = 2(\gamma_v^2 - 1) \geq 2\qty(1 + \frac{M}{2m})^2 - 1 = 1 + \frac{2M}{m} + \frac{M^2}{2m^2}
\]
Hence, in this frame, the kinetic energy \(mc^2(\gamma_u - 1)\) must satisfy
\[
	mc^2(\gamma_u - 1) \geq mc^2\qty(\frac{2M}{m} + \frac{M^2}{2m^2}) \geq 2Mc^2 + \frac{M^2c^2}{2m}
\]
This extra \(\frac{M^2c^2}{2m}\) term (compared to the \(Mc^2\) expression in \(S\)) is produced by the transformation between frames.
So in a frame where one particle is at rest, we require significantly more kinetic energy.
So a particle accelerator is most efficiently utilised by accelerating two particles into each other, rather than by accelerating one particle into a fixed target.
