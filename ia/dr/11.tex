\subsection{Rutherford Scattering}
Consider a positive charge fired towards another, fixed, positive charge. The particle will be deflected by the electrostatic force between the two particles. What is the angle \(\beta\) by which the particle is deflected? This is motion under a repulsive inverse square law force.
\[ V(r) = \frac{mk}{r};\quad F(r) = \frac{mk}{r^2} \]
We have already solved this problem for an attractive inverse square law force; this was the orbit equation. We can replace \(k\) with \(-k\) to model a repulsive force.
\[ u = \frac{-k}{h^2} + A\cos(\theta - \theta_0);\quad \theta_0 = 0, A \geq 0 \]
We can rewrite this as
\[ r = \frac{\ell}{e\cos\theta - 1};\quad \ell = \frac{h^2}{k}, e = \frac{Ah^2}{k} \]
Since we want \(r > 0\), we need \(e > 1\) such that for some \(\theta\), \(r > 0\). Then, \(r \to \infty\) as \(\theta \to \pm \alpha\), with \(\arccos(e^{-1}) \in \left(0, \frac{\pi}{2}\right)\). This gives a hyperbolic orbit. We find
\[ \frac{(x - ea)^2}{a^2} - \frac{y^2}{b^2} = 1;\quad a = \frac{\ell}{e^2 - 1}, b = \frac{\ell}{\sqrt{e^2 - 1}} \]
\(h\) is given by \(\abs{\vb r \times \dot{\vb r}}\). \(b\), the impact parameter, is the asymptotic distance of the moving particle from impacting the fixed particle. On the incoming asymptote, \(\dot {\vb r} \approx v\vb e_\parallel\), and \(\vb r \approx b\vb e_\perp + z\vb e_\parallel\) for some \(z\). Hence, \(h=bv\). Since \(\tan \alpha = \sqrt{e^2 - 1}\), we have
\[ b = \frac{\ell}{\tan \alpha} = \frac{\ell}{\tan\left( \frac{\pi}{2} - \beta \right)} = \frac{h^2}{k}\tan\left( \frac{\beta}{2} \right) = \frac{v^2b^2}{k}\tan\left( \frac{\beta}{2} \right) \]
Hence
\[ \beta = 2\arctan\left( \frac{k}{bv^2} \right) \]

\subsection{Rotating Frames of Reference}
Newton's laws of motion are only valid in inertial frames of reference. Hence, the laws of dynamics are different from the perspective of a rotating, or non-inertial, frame of reference. Let \(S\) be an inertial frame, and let \(S'\) be a non-inertial frame, rotating around the \(z\)-axis in \(S\) with angular velocity \(\omega = \dot\theta\) where \(\theta\) is the angle between the \(x\) or \(y\) axis in \(S\) or \(S'\). We will denote the basis vectors \(\vb e_i = \{ \vu{x}, \vu{y}, \vu{z} \}\) for \(S\) and \(\vb e'_i = \{ \vu{x}', \vu{y}', \vu{z}' \}\) for \(S'\). Consider a particle at rest in \(S'\), viewed in \(S\), with position vector \(\vb r\).
\[ \left( \dv{\vb r}{t} \right)_S = \vb\omega \times \vb r;\quad \vb\omega = \omega\vu{z} \]
This angular velocity vector is aligned with the axis of rotation. The convention is that viewed from the direction of the vector, the rotation is anticlockwise. The same formula applies to any vector which is fixed in \(S'\), not just the position vector. In particular, this applies to the basis vectors:
\[ \left( \dv{\vb e_i'}{t} \right)_S = \vb\omega \times \vb e_i' \]
Here, for instance, \(\left( \dv{\vb e_3'}{t} \right)_S = 0\). Consider a general time-dependent vector \(\vb a\), defined by the components of the basis vectors in \(S'\):
\[ \vb a(t) = \sum_{i=1}^3 a_i'(t)\vb e'_i(t) \]
Then we can deduce the key identity:
\[ \left( \dv{t} \vb a(t) \right)_{S'} = \sum_{i=1}^3\left( \dv{t} a_i'(t)  \right) \vb e'_i(t) \]
\begin{align*}
	\left( \dv{t} \vb a(t) \right)_{S} & = \sum_{i=1}^3\left( \dv{t} a_i'(t) \right) \vb e'_i(t) + \sum_{i=1}^3 a'_i(t) \left( \dv{t} \vb e_i'(t) \right)                   \\
	                                   & = \sum_{i=1}^3\left( \dv{t} a_i'(t) \right) \vb e'_i(t) + \sum_{i=1}^3 a'_i(t) \left( \vb\omega \times \dv{t} \vb  e_i'(t) \right) \\
	                                   & = \left( \dv{t} \vb a(t) \right)_{S'} + \vb\omega \times \vb a
\end{align*}
We can apply this identity to the position vector \(\vb r\), and the velocity \(\rdot\).
\[ \left( \dv{\vb r}{t} \right)_S = \left( \dv{\vb r}{t} \right)_{S'} + \vb\omega \times \vb r \]
For the purposes of this derivation, we will allow \(\vb\omega\) to depend on time.
\begin{align*}
	\left( \dv[2]{\vb r}{t} \right)_S & = \left\{ \left( \dv{t} \right)_{S'} + \vb\omega \times \right\} \left\{ \left( \dv{t} \right)_{S'} + \vb\omega \times \right\} \vb r                                     \\
	                                  & = \left( \dv[2]{\vb r}{t} \right)_{S'} + 2 \vb\omega \times \left( \dv{\vb r}{t} \right)_{S'} + \dot{\vb\omega} \times \vb r + \vb\omega \times (\vb \omega \times \vb r)
\end{align*}
Now, let us write down Newton's equation of motion in a rotating frame.
\begin{align*}
	m\left( \dv[2]{\vb r}{t} \right)_S                                                                                                                                          & = \vb F \\
	m\left( \dv[2]{\vb r}{t} \right)_{S'} + 2 m\vb\omega \times \left( \dv{\vb r}{t} \right)_{S'} + m\dot{\vb\omega} \times \vb r + m\vb\omega \times (\vb \omega \times \vb r) & = \vb F \\
\end{align*}
Note that we do not need to distinguish between \(\left( \dv{\vb\omega}{t} \right)_S\) and \(\left( \dv{\vb\omega}{t} \right)_{S'}\), since any difference vanishes under the cross product with \(\vb\omega\). These extra terms apart from \(m\left( \dv[2]{\vb r}{t} \right)_{S'}\) can be referred to as `fictitious' forces, since they only appear to be there as perceived by an observer in a rotating (or more general non-inertial) frame. According to this rotating observer, these fictitious forces act in the negative direction:
\[ m\left( \dv[2]{\vb r}{t} \right)_{S'} = \vb F - 2 m\vb\omega \times \left( \dv{\vb r}{t} \right)_{S'} - m\dot{\vb\omega} \times \vb r - m\vb\omega \times (\vb \omega \times \vb r) \]
We can give each fictitious force a name:
\begin{itemize}
	\item \(-2 m\vb\omega \times \left( \dv{\vb r}{t} \right)_{S'}\) is the Coriolis force;
	\item \(-m\dot{\vb\omega} \times \vb r\) is the Euler force;
	\item \(-m\vb\omega \times (\vb \omega \times \vb r)\) is the centrifugal force.
\end{itemize}
In many applications, we take the Euler force to be zero, since this only is relevant when the angular velocity is changing.
