\subsection{Results on Moments of Inertia}
\begin{theorem}[Perpendicular Axes Theorem]
	For a two-dimensional body (a lamina),
	\[ I_z = I_x + I_y \]
	where \(I_z\) is the moment of inertia about the axis perpendicular to the lamina, and the \(I_x\) and \(I_y\) are the moments of inertia in perpendicular directions in the plane of the lamina.
\end{theorem}
\begin{proof}
	Let \(A\) be the lamina as shown. Then
	\[ I_x = \int_A \rho y^2 \dd{A};\quad I_y = \int_A \rho x^2 \dd{A} \]
	where \(x, y\) are the plane Cartesian components of the position vector of a point. Then
	\[ I_z = \int_A \rho (x^2 + y^2) \dd{A} = I_x + I_y \]
	as required.
\end{proof}
\noindent This theorem is useful when there is a level of symmetry in the problem where \(I_x = I_y\).
\begin{theorem}[Parallel Axes Theorem]
	Consider a rigid body of mass \(M\) with a moment of inertia \(I_c\) about some axis through the centre of mass. Then the moment of inertia about a parallel axis a distance \(d\) from the centre of mass has moment of inertia
	\[ I = I_c + Md^2 \]
\end{theorem}
\begin{proof}
	Let us consider Cartesian coordinates, with the origin at the centre of mass. The moment of inertia about an axis in the \(z\) direction through the origin is \(I_c\), and the moment about the axis passing through the point \((d, 0, 0)\) is \(I\). Let us denote the volume of the body as \(V\). Then
	\[ I_c = \int_V \rho (x^2 + y^2) \dd{V};\quad I = \int_V \rho ((x-d)^2 + y^2) \dd{V} \]
	Hence,
	\[ I = \int_V \rho (x^2 + y^2) \dd{V} - 2\underbrace{\int_V \rho xd \dd{V}}_{=0} + \int_V \rho d^2 \dd{V} \]
	The middle term is zero since the origin is the centre of mass, and we are integrating over the \(x\) coordinate multiplied by a constant multiple of density.
	\[ I = I_c + Md^2 \]
	as required.
\end{proof}

\subsection{General Motion of a Rigid Body}
In general, the motion of a rigid body can be described by a combination of
\begin{itemize}
	\item the translation of the centre of mass, following a trajectory \(\vb R(t)\), and
	\item the rotation about an axis through the centre of mass.
\end{itemize}
Like before, we define the position vector of a point \(i\) in the body as \(\vb r_i = \vb R + \vb s_i\) where the \(\vb s_i\) are relative to the centre of mass. Recall that \(\sum_{i=1}^N \vb s_i = \vb 0\). If a body is rotating about the centre of mass with angular velocity \(\vb\omega\), then
\[ \dot{\vb s}_i = \vb\omega \times \vb s_i;\quad \dot{\vb r}_i = \dot{\vb R} + \vb\omega \times \dot{\vb s}_i \]
Recall that the kinetic energy is
\[ T = \frac{1}{2}M \dot{\vb R}^2 + \frac{1}{2}\sum_{i=1}^N m_i \dot{\vb s}_i^2 = \frac{1}{2}M \dot{\vb R}^2 + \frac{1}{2}I_c \omega^2 \]
where \(I_c\) is the moment of inertia about the axis of rotation \(\vb n = \vb\omega / \omega\) through the centre of mass. We can therefore consider \(T\) as the sum of a `translational' kinetic energy and a `rotational' kinetic energy. Recall that in a general multiparticle system, linear momentum and angular momentum satisfy
\[ \dot{\vb p} = \vb F;\quad \dot{\vb L} = \vb G \]
where \(\vb F\) is the total external force and \(\vb G\) is the total external torque. For a rigid body, these two equations determine the translational and rotational components of motion entirely. Note that sometimes we can determine the motion in a simpler way by using energy conservation laws. Note further that \(\vb L\) and \(\vb G\) depend on the choice of origin, which could be defined as any point fixed in an inertial frame, or alternatively we could define them with respect to the centre of mass. In this case, the equation \(\dot{\vb L} = \vb G\) still holds. Indeed,
\begin{align*}
	\underbrace{\vb G}_{\text{external torque about origin}} & = \dv{t} \left( M \vb R \times \dot{\vb R} + \sum_{i=1}^N m_i \vb s_i \times \dot{\vb s}_i \right)                      \\
	                                                         & = M \dot{\vb R} \times \dot{\vb R} + M \vb R \times \ddot{\vb R} + \dv{t} \sum_{i=1}^N m_i \vb s_i \times \dot{\vb s}_i \\
	                                                         & = \vb R \times \vb F^\text{ext} + \dv{t} \sum_{i=1}^N m_i \vb s_i \times \dot{\vb s}_i
\end{align*}
Hence the rate of change of the angular momentum about the centre of mass \(\dv{t} \sum_{i=1}^N m_i \vb s_i \times \dot{\vb s}_i\) is exactly \(\vb G - \vb R \times \vb F^\text{ext}\). Therefore,
\begin{align*}
	\vb G_c & = \sum_{i=1}^N \vb r_i \times \vb F_i^\text{ext} - \vb R \times \vb F^\text{ext} \\
	        & = \sum_{i=1}^N (\vb r_i - \vb R) \times \vb F_i^\text{ext}
\end{align*}
Hence the rate of change of the angular momentum about the centre of mass is exactly the external torque about the centre of mass.

\subsection{Example of Rigid Body Motion}
Consider the motion of a rigid body in a uniform gravitional field with constant acceleration \(\vb g\). The total gravitational force and torque acting on the rigid body are the same as those that would act on a particle of the same mass located at the rigid body's centre of mass. In a gravitational field, the centre of mass is often referred to as the `centre of gravity'. Indeed,
\begin{align*}
	\vb F & = \sum_{i=1}^N \vb F_i^\text{ext} \\
	      & = \sum_{i=1}^N m_i \vb g          \\
	      & = M\vb g
\end{align*}
Correspondingly, the total torque is given by
\begin{align*}
	\vb G & = \sum_{i=1}^N \vb G_i^\text{ext}       \\
	      & = \sum_{i=1}^N \vb r_i \times m_i \vb g \\
	      & = \sum_{i=1}^N m_i \vb r_i \times \vb g \\
	      & = M \vb R \times \vb g
\end{align*}
Note that the gravitational torque about the centre of mass is exactly zero, since
\begin{align*}
	\vb G_c & = \sum_{i=1}^N \vb s_i \times m_i \vb g \\
	        & = \sum_{i=1}^N m_i \vb s_i \times \vb g \\
	        & = \vb 0
\end{align*}
Note further that the external potential \(V^\text{ext}\), which is exactly the gravitational potential, will be given by
\begin{align*}
	V^\text{ext} & = -\sum_{i=1}^N m_i \vb r_i \cdot \vb g \\
	             & = -M\vb R \cdot \vb g
\end{align*}
\noindent Consider a stick thrown into the air. The centre of mass will follow a parabola, and the angular acceleration about the centre of mass is zero.
