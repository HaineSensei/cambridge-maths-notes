\subsection{Systems of linear ODEs}
Consider two functions \(y_1(t), y_2(t)\) which satisfy
\begin{align*}
	\dot y_1 & = ay_1 + by_2 + f_1(t) \\
	\dot y_2 & = cy_1 + dy_2 + f_2(t)
\end{align*}
This is a set of coupled differential equations which we must solve simultaneously.
In vector form,
\[
	\dot{\vb Y} = M\vb Y + \vb F
\]
where
\[
	\vb Y = \begin{pmatrix}
		y_1 \\ y_2
	\end{pmatrix};\quad M = \begin{pmatrix}
		a & b \\ c & d
	\end{pmatrix};\quad \vb F = \begin{pmatrix}
		f_1 \\ f_2
	\end{pmatrix}
\]
Any \(n\)th order differential equation can be written as a system of \(n\) first order ODEs.
For example, the standard form for a second order linear ODE is
\[
	\ddot y + a\dot y + by = f
\]
Let \(y_1 = y, y_2 = \dot y\).
Then
\[
	\vb Y = \begin{pmatrix}
		y \\ \dot y
	\end{pmatrix}
\]
Hence our two equations are
\[
	\dot y_1 = y_2
\]
\[
	\dot y_2 + ay_2 + by_1 = f \implies \dot y_2 = -ay_2 - by_1 + f
\]
And in vector form,
\[
	\dot{\vb Y} = \begin{pmatrix}
		0  & 1  \\
		-b & -a
	\end{pmatrix} \vb Y + \begin{pmatrix}
		0 \\ f
	\end{pmatrix}
\]

\subsection{Matrix methods}
To solve a system of \(n\) first-order linear ODEs,
\begin{equation}\label{matrix1}
	\dot {\vb Y} = M\vb Y + \vb F
\end{equation}
we need the following steps.
\begin{enumerate}
	\item Write \(\vb Y = \vb Y_c + \vb Y_p\) where \(\vb Y_c\) satisfies the homogeneous version of \eqref{matrix1}:
	      \begin{equation}\label{matrix2}
		      \dot {\vb Y}_c = M\vb Y_c
	      \end{equation}
	\item Seek solutions of \(\vb Y_c\) in the form \(\vb v e^{\lambda t}\).
	      \[
		      \eqref{matrix2} \implies \lambda \vb v = M\vb v
	      \]
	      So the vectors \(\vb v\) are the eigenvectors, with eigenvalues \(\lambda\).
	\item Find \(\vb Y_p\) based on the form of \(\vb F\).
\end{enumerate}
As a quick example, let us consider
\begin{equation}\label{matrixexample}
	\dot {\vb Y} - \begin{pmatrix}
		-4 & 24 \\ 1 & -2
	\end{pmatrix}\vb Y = \begin{pmatrix}
		4 \\ 1
	\end{pmatrix}e^t
\end{equation}
We will try \(\vb Y_c = \vb v e^{\lambda t}\), and we can compute that the eigenvalues and eigenvectors are
\[
	\lambda_1 = 2, \vb v_1 = \begin{pmatrix}
		4 \\ 1
	\end{pmatrix};\quad \lambda_2 = -8, \vb v_2 = \begin{pmatrix}
		-6 \\ 1
	\end{pmatrix}
\]
Hence,
\[
	\vb Y_c = A\begin{pmatrix}
		4 \\ 1
	\end{pmatrix}e^{2t} + B\begin{pmatrix}
		-6 \\ 1
	\end{pmatrix}e^{-8t}
\]
Now, to solve the particular integral, we will try a \(\vb Y_p\) of the form
\[
	\vb Y_p = \begin{pmatrix}
		u_1 \\ u_2
	\end{pmatrix} e^t
\]
We have
\begin{align*}
	\eqref{matrixexample} \implies \begin{pmatrix}
		                               u_1 \\ u_2
	                               \end{pmatrix} - \begin{pmatrix}
		                                               -4 & 24 \\ 1 & -2
	                                               \end{pmatrix}\begin{pmatrix}
		                                                            u_1 \\ u_2
	                                                            \end{pmatrix} & = \begin{pmatrix}
		                                                                              4 \\ 1
	                                                                              \end{pmatrix} \\
	I\begin{pmatrix}
		 u_1 \\ u_2
	 \end{pmatrix} - \begin{pmatrix}
		                 -4 & 24 \\ 1 & -2
	                 \end{pmatrix}\begin{pmatrix}
		                              u_1 \\ u_2
	                              \end{pmatrix}                & = \begin{pmatrix}
		                                                               4 \\ 1
	                                                               \end{pmatrix}                \\
	\begin{pmatrix}
		5 & -24 \\ -1 & 3
	\end{pmatrix}\begin{pmatrix}
		             u_1 \\ u_2
	             \end{pmatrix}                   & = \begin{pmatrix}
		                                                 4 \\ 1
	                                                 \end{pmatrix}                              \\
	\begin{pmatrix}
		u_1 \\ u_2
	\end{pmatrix}                                & = \begin{pmatrix}
		                                                 -4 \\ -1
	                                                 \end{pmatrix}
\end{align*}
So the general solution is
\[
	\vb Y = A\begin{pmatrix}
		4 \\ 1
	\end{pmatrix}e^{2t} + B\begin{pmatrix}
		-6 \\ 1
	\end{pmatrix}e^{-8t} - \begin{pmatrix}
		4 \\ 1
	\end{pmatrix}e^t
\]
If the forcing term matches one of the complementary functions, we would try \(\vb Y_p = \vb u te^{\lambda t}\).

\subsection{Decoupling ODEs}
From a linear system of \(n\) first order ODEs, we can construct \(n\) uncoupled \(n\)th order ODEs.
For the above example \eqref{matrixexample},
\begin{align}
	\label{decouple1} \dot y_1 & = -4y_1 + 24y_2 + 4e^t \\
	\label{decouple2} \dot y_2 & = y_1 - 2y_2 + e^t
\end{align}
We will create a linear equation for \(\ddot y\).
First, we will take the derivative of \eqref{decouple1}.
\[
	\ddot y_1 = -4\dot y_1 + 24\dot y_2 + 4e^t
\]
Now we will substitute in \eqref{decouple2} for \(\dot y_2\).
\[
	\ddot y_1 = -4\dot y_1 + 24(y_1 - 2y_2 + e^t) + 4e^t
\]
Now we can substitute back in the original equation \eqref{decouple1} to remove the \(y_2\) term.
\[
	\ddot y_1 = -4\dot y_1 + 24\left(y_1 - \frac{1}{12}(\dot y_1 + 4y_1 - 4e^t) + e^t\right) + 4e^t
\]
\[
	\ddot y_1 = -4\dot y_1 + 24y_1 - 2\dot y_1 - 8y_1 + 8e^t + 28e^t
\]
\[
	\ddot y_1 + 6\dot y_1 - 16y_1 = 36e^t
\]
which we can solve as normal.
The general solution matches the first component of the general solution vector from above.
We can of course construct an analogous equation for \(y_2\), which would match the second component of the solution vector.

\subsection{Phase portraits}
For some complementary function \(\vb Y_c\) (or equivalently, a solution to a homogeneous system of linear first order ODEs) satisfying
\begin{equation}\label{phasematrix}
	\dot {\vb Y}_c = M\vb Y_c
\end{equation}
Therefore,
\[
	\vb Y_c = A\vb v_1 e^{\lambda_1 t} + B\vb v_2 e^{\lambda_2 t}
\]
Let us consider three cases.
\begin{enumerate}
	\item \(\lambda_1, \lambda_2\) real and opposite sign.
	      Without loss of generality, let \(\lambda_1 > 0\).
	      %TODO draw this phase portrait
	      The origin here is known as a `saddle node' as the solution curves for \(A=0\) and \(B=0\) cross at the origin.
	\item \(\lambda_1, \lambda_2\) real and same sign.
	      let us say that without loss of generality that \(\abs{\lambda_1} > \abs{\lambda_2}\).
	      If they are both negative, %TODO this portrait
	      Here the origin is a stable node as all solution curves tend towards it.
	      If \(\lambda_1, \lambda_2\) are both positive, %TODO this
	      The origin here is an unstable node as the curves tend away from it.
	\item \(\lambda_1, \lambda_2\) form a complex conjuate pair.
	      If the real parts are negative, %TODO graph
	      This is a stable spiral; the curves tend towards zero.
	      If the real parts are positive we have an unstable spiral:
	      %TODO graph
	      If the real part is zero, the solution curves are circles.
	      This is known as a centre.
	      %TODO graph
\end{enumerate}
Note that to find the direction of rotations in these phase portraits, we would need to evaluate the system of equations at a given point to find the signs of the derivatives \(\dot y_1, \dot y_2\).

\subsection{Nonlinear systems of ODEs}
Consider an autonomous system of two nonlinear first order ODEs:
\begin{align}
	\label{nonlinear1} \dot x & = f(x, y) \\
	\label{nonlinear2} \dot y & = g(x, y)
\end{align}
where `nonlinear' means that \(f\) and \(g\) are nonlinear functions of \(x\) and \(y\), and where `autonomous' means that the independent variable \(t\) does not explicitly show up in these equations.
We will consider the equilibrium points, or fixed points.
Let \((x_0, y_0)\) be a fixed point, i.e.
\[
	\dot x = \dot y = 0 \implies f(x_0, y_0) = g(x_0, y_0) = 0
\]
at this point.
We may want to understand the stability of such fixed points by perturbation analysis, like before.
Let us consider a small perturbation away from the fixed point.
\[
	(x, y) = (x_0 + \xi(t), y_0 + \eta(t))
\]
We have
\[
	\eqref{nonlinear1} \implies \dot \xi = f(x_0 + \xi, y_0 + \eta)
\]
We can expand this in a multivariate Taylor series, keeping the first three terms --- the constant term and the two linear terms.
\begin{align*}
	\dot \xi  & \approx f(x_0, y_0) + \xi f_x(x_0, y_0) + \eta f_y(x_0, y_0) \\
	          & = \xi f_x(x_0, y_0) + \eta f_y(x_0, y_0)                     \\
	\dot \eta & \approx g(x_0, y_0) + \xi g_x(x_0, y_0) + \eta g_y(x_0, y_0) \\
	          & = \xi g_x(x_0, y_0) + \eta g_y(x_0, y_0)
\end{align*}
Hence,
\[
	\begin{pmatrix}
		\dot \xi \\ \dot \eta
	\end{pmatrix} = \eval{\begin{pmatrix}
			f_x & f_y \\ g_x & g_y
		\end{pmatrix}}_{x_0, y_0} \begin{pmatrix}
		\xi \\ \eta
	\end{pmatrix}
\]
This is a homogeneous linear system of ODEs.
The eigenvalues of \(M\), which we will call \(\lambda_1, \lambda_2\), determine the stability and behaviour.
This is just the same as the phase portrait analysis above to determine whether the perturbed point tends to the origin or not, and exactly how this movement happens.

\subsection{Lotka-Volterra equations}
This is a worked example of a coupled set of differential equations, which model a predator-prey system.
Let the quantity of prey be represented by \(x\), and the quantity of the predator be \(y\).
Then
\begin{align*}
	\dot x & = \alpha x - \beta xy = f(x, y) \\
	\dot y & = \delta xy - \gamma y = g(x,y)
\end{align*}
where \(\alpha, \beta, \gamma, \delta\) are positive real constants.
We will start by analysing the fixed points, where \(\dot x = \dot y = 0\).
\[
	\dot x = 0 \implies x=0 \text{ or } y = \frac{\alpha}{\beta}
\]
\[
	\dot y = 0 \implies y=0 \text{ or } x = \frac{\gamma}{\delta}
\]
Therefore,
\[
	(x_0, y_0) = (0, 0), \left( \frac{\gamma}{\delta}, \frac{\alpha}{\beta} \right)
\]
Using matrix methods,
\[
	M = \begin{pmatrix}
		f_x & f_y \\
		g_x & g_y
	\end{pmatrix} = \begin{pmatrix}
		\alpha - \beta y & -\beta x          \\
		\delta y         & \delta x - \gamma
	\end{pmatrix}
\]
Now we can analyse the stability of these fixed points by perturbation analysis.
\begin{itemize}
	\item At the fixed point \((0, 0)\), we have
	      \[
		      \begin{pmatrix}
			      \dot \xi \\
			      \dot \eta
		      \end{pmatrix} = \begin{pmatrix}
			      \alpha & 0 \\ 0 & -\gamma
		      \end{pmatrix} \begin{pmatrix}
			      \xi \\ \eta
		      \end{pmatrix}
	      \]
	      We can read off the eigenvalues to be \(\alpha\) and \(-\gamma\).
	      This is a saddle node, since one direction will increase (\(x\)) and one will decrease (\(y\)).
	\item At the fixed point \(\left( \frac{\gamma}{\delta}, \frac{\alpha}{\beta} \right)\), we have
	      \[
		      \begin{pmatrix}
			      \dot \xi \\
			      \dot \eta
		      \end{pmatrix} = \begin{pmatrix}
			      0 & -\beta\frac{\gamma}{\delta} \\ \delta\frac{\alpha}{\beta} & 0
		      \end{pmatrix} \begin{pmatrix}
			      \xi \\ \eta
		      \end{pmatrix}
	      \]
	      The characteristic equation is \(\chi_M(\lambda) = \lambda^2 + \alpha\lambda = 0\), so \(\lambda = \pm \sqrt{-\alpha\gamma}\).
	      Since \(\alpha\gamma > 0\), it is more convenient to write \(\lambda = \pm i\sqrt{\alpha\gamma}\).
	      Since the real part is zero, this gives a centre node.
	      To work out the direction of rotation, let us consider the \(x\) direction,
	      \[
		      \dot \xi = -\beta\frac{\gamma}{\delta}\eta
	      \]
	      If \(\eta > 0\), then \(\dot \xi < 0\), so we have anticlockwise rotation.
\end{itemize}
Now we can sketch a graph taking into account both of these fixed points, visually interpolating the values between them.
% TODO do this graph

\subsection{First order wave equation and method of characteristics}
We will define a partial differential equation to be a differential equation with multiple independent variables.
Here, we will consider three examples, starting with the first order wave equation.
Consider a function \(y(x, t)\) where
\begin{equation}\label{wave1}
	\frac{\partial y}{\partial t} - c\frac{\partial y}{\partial x} = 0
\end{equation}
where \(c\) is a constant.
We will solve this equation with the method of characteristics.
%TODO random-looking graph
Imagine moving a `probe' along a path \(x(t)\).
Then \(y\) is a function \(y(x(t), t)\), where now the only independent variable is \(t\).
Using the multivariate chain rule,
\[
	\frac{\dd{y}}{\dd{t}} = \frac{\partial y}{\partial t} + \frac{\partial y}{\partial x}\frac{\dd{x}}{\dd{t}}
\]
Comparing this with \eqref{wave1}, we note that if \(\frac{\dd{x}}{\dd{t}} = -c\), then \(\frac{\dd{y}}{\dd{t}} = 0\).
So we have found a path along which the `probe' is at a constant height, i.e.\ along \(x(t) = x_0 - ct\), \(y\) is a constant.
We can update our graph now showing the `characteristics' we have just shown to exist.
%TODO graph
If \(y(x, t = 0) = f(x)\), then \(y = f(x_0)\) along the characteristics.
Hence, our general solution is
\[
	y = f(x+ct)
\]
Let us consider some examples of wave equations \(f\).
\begin{enumerate}
	\item (unforced wave equation) Let \(y(x,0) = x^2 - 3\) in \eqref{wave1}.
	      Then
	      \[
		      y(x, t) = (x+ct)^2 - 3
	      \]
	\item (forced wave equation) Let
	      \[
		      \frac{\partial y}{\partial t} + 5\frac{\partial y}{\partial x} = e^{-t}
	      \]
	      and
	      \[
		      y(x, 0) = e^{-x^2}
	      \]
	      Then along paths with \(\frac{\dd{x}}{\dd{t}} = 5\) or \(x=x_0 + 5t\),
	      \[
		      \frac{\dd{y}}{\dd{t}} = e^{-t}
	      \]
	      So by integration,
	      \[
		      y = A-e^{-t}
	      \]
	      along these paths.
	      Applying our initial condition at \(t=0\), our `probe' is at \(x_0\) and \(y(x, 0) = A - 1 = e^{-x_0^2}\).
	      Hence, \(A = 1 + e^{-x_0^2}\).
	      So
	      \[
		      y = 1 + e^{-x_0^2} - e^{-t}
	      \]
	      along the path given by \(x_0\).
	      Substituting back for a general \(x\), we can create a formula for the general solution of \(y\) (not necessarily on a given path):
	      \[
		      y = 1 + e^{-(x-5t)^2} - e^{-t}
	      \]
\end{enumerate}
