\subsection{Classifying Stationary Points}
Since \(\grad f = \vb 0\) defines a stationary point, the Taylor series expansion around a stationary point \(\vb x = \vb x_s\) is
\[ f(\vb x) \approx f(\vb x_s) + \frac{1}{2}\delta \vb x \cdot \eval{H}_{\vb x_s} \cdot \delta \vb x^\transpose \]
So the nature of the stationary point depends on the Hessian matrix \(H\). Consider a function in \(n\)-dimensional space
\[ f = f(x_1, x_2, \dots, x_n) \]
Then the \(n\)-dimensional Hessian matrix is given by
\[ H = \begin{pmatrix}
		f_{x_1x_1} & f_{x_1x_2} & \cdots & f_{x_1x_n} \\
		f_{x_2x_1} & f_{x_2x_2} & \cdots & f_{x_2x_n} \\
		\vdots     & \vdots     & \ddots & \vdots     \\
		f_{x_nx_1} & f_{x_nx_2} & \cdots & f_{x_nx_n}
	\end{pmatrix} \]
If all of these derivatives are defined, \(f_{x_1x_2} = f_{x_2x_1}\) etc, so \(H = H^\transpose\), i.e.\ \(H\) is symmetric, and therefore it can be diagonalised with respect to its principal axes.
\[ \delta \vb x \cdot H \cdot \delta \vb x^\transpose = \begin{pmatrix}
		\delta x_1 & \delta x_2 & \cdots & \delta x_n
	\end{pmatrix} \begin{pmatrix}
		\lambda_1 & 0         & \cdots & 0         \\
		0         & \lambda_2 & \cdots & 0         \\
		\vdots    & \vdots    & \ddots & \vdots    \\
		0         & 0         & \cdots & \lambda_n
	\end{pmatrix} \begin{pmatrix}
		\delta x_1 \\ \delta x_2 \\ \cdots \\ \delta x_n
	\end{pmatrix} \]
where the \(\lambda_i\) are eigenvalues of \(H\) and the \(\delta x_i\) is the displacement along the principal axis (eigenvector) \(i\). Therefore
\[ \delta \vb x \cdot H \cdot \delta \vb x^\transpose = \lambda_1 \delta x_1^2 + \lambda_2 \delta x_2^2 + \dots + \lambda_n \delta x_n^2 \]
\begin{enumerate}
	\item At a minimum point, \(\delta \vb x \cdot H \cdot \delta \vb x^\transpose > 0\) for any \(\delta \vb x\) (moving in any direction, we go `downhill'). So all the \(\lambda_i > 0\). So \(H\) is positive definite.
	\item At a maximum point, \(\delta \vb x \cdot H \cdot \delta \vb x^\transpose < 0\) for any \(\delta \vb x\). So all the \(\lambda_i < 0\). \(H\) is negative definite.
	\item At a saddle point, \(H\) is indefinite.
\end{enumerate}

\subsection{Signature of Hessian}
\begin{definition}
	The signature of \(H\) is the pattern of the signs of its subdeterminants.
\end{definition}
For a function \(f(x_1, x_2, \dots, x_n)\), we want the signs of
\[ \underbrace{\abs{f_{x_1x_1}}}_{\abs{H_1}}, \underbrace{\begin{vmatrix}
			f_{x_1x_1} & f_{x_1x_2} \\
			f_{x_2x_1} & f_{x_2x_2}
		\end{vmatrix}}_{\abs{H_2}},\dots, \underbrace{\begin{vmatrix}
			f_{x_1x_1} & f_{x_1x_2} & \cdots & f_{x_1x_n} \\
			f_{x_2x_1} & f_{x_2x_2} & \cdots & f_{x_2x_n} \\
			\vdots     & \vdots     & \ddots & \vdots     \\
			f_{x_nx_1} & f_{x_nx_2} & \cdots & f_{x_nx_n}
		\end{vmatrix}}_{\abs{H_1}} \]
We know from Vectors and Matrices that if a symmetric matrix \(H\) is positive (or negative) definite, then \(H_1, H_2, \dots, H_{n-1}\) are positive (or negative) definite. This is known as Sylvester's Criterion. In other words, a minimum (or maximum) point in \(n\)-dimensional space is also a minimum (or maximum) in any subspace containing this point. Now let us list the signs of subdeterminants to see the types of signatures.
\begin{enumerate}
	\item At a minimum point (\(\lambda_i > 0\)), the signature is \(+, +, +, +, \dots\)
	\item At a maximum point (\(\lambda_i > 0\)), the signature is \(-, +, -, +, \dots\)
\end{enumerate}
If \(\abs{H} = 0\), we need higher order terms in the Taylor series.

\subsection{Contours Near Stationary Points}
Consider a coordinate system aligned with the principal axes of the Hessian \(H\) in two-dimensional space, so
\[ H = \begin{pmatrix}
		\lambda_1 & 0 \\ 0 & \lambda_2
	\end{pmatrix} \]
Let \(\delta \vb x = (\vb x - \vb x_s) = (\xi, \eta)\) where \(\vb x_s\) is the stationary point we're considering. In a small region near \(\vb x_s\), the contours of \(f\) satisfy
\[ f = \text{constant (since \(f\) is a contour)} \approx f(\vb x_s) = \frac{1}{2}\delta \vb x \cdot H \cdot \delta \vb x^\transpose \]
\begin{equation}\label{contourhessian}
	\therefore \lambda_1 \xi^2 + \lambda_2 \eta^2 \approx \text{constant}
\end{equation}
Near a minimum or maximum point, \(\lambda_1\) and \(\lambda_2\) have the same sign. \eqref{contourhessian} implies that the contours of \(f\) are elliptical. Near a saddle point, \(\lambda_1\) and \(\lambda_2\) have opposite sign so \eqref{contourhessian} shows that the contours of \(f\) are hyperbolic. As an example, let us consider
\[ f(x,y) = 4x^3 - 12xy + y^2 + 10y + 6 \]
Let us first identify the stationary points.
\[ f_x = f_y = 0 \]
After solving this, we get
\[ (x,y) = (1, 1), (5, 25) \]
To get the Hessian matrix:
\begin{align*}
	f_{xx}          & = 24x \\
	f_{xy} = f_{yx} & = -12 \\
	f_{yy}          & = 2
\end{align*}
Now considering the stationary points separately:
\begin{itemize}
	\item \((1, 1)\):
	      \[ H = \begin{pmatrix}
			      24 & -12 \\ -12 & 2
		      \end{pmatrix} \implies \abs{H_1} = 24;\quad \abs{H} = 48-144 \]
	      The signature is \(+, -\), so this is a saddle point.
	\item \((5, 25)\):
	      \[ H = \begin{pmatrix}
			      120 & -12 \\ -12 & 2
		      \end{pmatrix} \implies \abs{H_1} = 120;\quad \abs{H} = 240-144 \]
	      The signature is \(+, +\), so this is a minimum point.
\end{itemize}

\subsection{Systems of Linear ODEs}
Consider two functions \(y_1(t), y_2(t)\) which satisfy
\begin{align*}
	\dot y_1 & = ay_1 + by_2 + f_1(t) \\
	\dot y_2 & = cy_1 + dy_2 + f_2(t)
\end{align*}
This is a set of coupled differential equations which we must solve simultaneously. In vector form,
\[ \dot{\vb Y} = M\vb Y + \vb F \]
where
\[ \vb Y = \begin{pmatrix}
		y_1 \\ y_2
	\end{pmatrix};\quad M = \begin{pmatrix}
		a & b \\ c & d
	\end{pmatrix};\quad \vb F = \begin{pmatrix}
		f_1 \\ f_2
	\end{pmatrix} \]
Any \(n\)th order differential equation can be written as a system of \(n\) first order ODEs. For example, the standard form for a second order linear ODE is
\[ \ddot y + a\dot y + by = f \]
Let \(y_1 = y, y_2 = \dot y\). Then
\[ \vb Y = \begin{pmatrix}
		y \\ \dot y
	\end{pmatrix} \]
Hence our two equations are
\[ \dot y_1 = y_2 \]
\[ \dot y_2 + ay_2 + by_1 = f \implies \dot y_2 = -ay_2 - by_1 + f \]
And in vector form,
\[ \dot{\vb Y} = \begin{pmatrix}
		0  & 1  \\
		-b & -a
	\end{pmatrix} \vb Y + \begin{pmatrix}
		0 \\ f
	\end{pmatrix} \]
