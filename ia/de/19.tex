\subsection{Special Cases of Indicial Equation}
Before looking at some examples of the method of Frobenius, we will first look at special cases of the indicial equation provided by Fuch's theorem.
Consider an expansion about the point \(x=x_0\).
Let \(\sigma_1, \sigma_2\) be the roots of this equation.
There are two cases:
\begin{itemize}
	\item (\(\sigma_1 - \sigma_2 \notin \mathbb Z\)) We have two linearly independent solutions.
	      So our solution is of the form
	      \[
		      y = (x-x_0)^{\sigma_1}\sum_{n=0}^\infty a_n(x-x_0)^n + (x-x_0)^{\sigma_2}\sum_{n=0}^\infty b_n(x-x_0)^n
	      \]
	      Note that the limit as \(x \to x_0\), \(y \sim (x-x_0)^{\min(\sigma_1, \sigma_2)}\).
	\item (\(\sigma_1 - \sigma_2 \in \mathbb Z\)) There is one solution of the form
	      \[
		      y_1 = (x-x_0)^{\sigma_2}\sum_{n=0}^\infty a_n(x-x_0)^n
	      \]
	      The other solution is of the form
	      \[
		      y_2 = (x-x_0)^{\sigma_1}\sum_{n=0}^\infty b_n(x-x_0)^n + cy_1 \ln(x-x_0)
	      \]
	      where \(c\) may or may not equal zero.
	      If the two solutions are linearly independent without the \(c\) term, then \(c=0\).
	      Else, without loss of generality, we can let \(c=1\) since we're dealing with homogeneous equations.
	\item (\(\sigma_1 = \sigma_2 = \sigma\)) Here, \(c \neq 0\).
	      So our solutions are of the form
	      \[
		      y_1 = (x-x_0)^\sigma \sum_{n=0}^\infty a_n(x-x_0)^n
	      \]
	      \[
		      y_2 = (x-x_0)^\sigma \sum_{n=0}^\infty b_n(x-x_0)^n + y_1\ln(x-x_0)
	      \]
\end{itemize}

\subsection{Example 1}
Let us solve the equation
\begin{equation}\label{frobenius1}
	x^2y'' - xy = 0
\end{equation}
where we want series solutions about \(x=0\).
Note that this is a regular singular point.
We will try solutions of the form
\[
	y=\sum_{n=0}^\infty a_n x^{n+\sigma}
\]
Therefore, we have
\begin{equation}\label{frobenius1b}
	\sum_{n=0}^\infty a_n x^{n+\sigma}\left[ (n+\sigma)(n+\sigma - 1) - x \right] = 0
\end{equation}
Equating coefficients of \(x^{n+\sigma}\) for \(n \geq 1\):
\begin{equation}\label{frobenius1a}
	(n+\sigma)(n+\sigma-1) a_n = a_{n-1}
\end{equation}
By equating the coefficients of the lowest powers of \(x\) (here \(n=0\), so we equate coefficients of \(x^\sigma\)), we get an indicial equation for \(\sigma\):
\[
	\sigma(\sigma - 1)a_0 = 0
\]
So either \(\sigma = 0\) or \(\sigma = 1\), since \(a_0 \neq 0\).
So the values of \(\sigma\) differ by an integer.
\begin{itemize}
	\item (\(\sigma = 1\)) \eqref{frobenius1a} implies that
	      \[
		      a_n = \frac{a_{n-1}}{n(n-1)} = \frac{a_0}{(n+1)(n!)^2}
	      \]
	      So we have
	      \[
		      y_1 = a_0x\left(1 + \frac{x}{2} + \frac{x^2}{12} + \frac{x^3}{144} + \dots\right)
	      \]
	\item (\(\sigma = 0\)) \eqref{frobenius1a} now gives
	      \[
		      n(n-1)b_n = b_{n-1}
	      \]
	      Normally we could find \(b_1\) in terms of \(b_0\) using this relation, but this just reduces to \(0b_1 = 0\), so we can't deduce it here.
	      When \(n=1\), we can equate coefficients of \(x\) in \eqref{frobenius1b} (relabelling \(a\) to \(b\)) to get
	      \[
		      b_1 (1)(1-1) = 0
	      \]
	      So \(b_1\) is arbitrary.
	      Then of course we can find \(b_2\) and so on in terms of smaller \(b_i\) values.
	      It turns out that
	      \[
		      b_i = a_{i-1}
	      \]
	      And therefore \(y_2(x)\) is linearly dependent on the previous \(y_1(x)\).
	      So we now need to use that logarithmic term to achieve linear independence, so \(y\) here is of the form
	      \[
		      y_2 = y_1\ln x + \sum_{x=0}^\infty b_n x^n
	      \]
	      Why do we have specifically a logarithmic term? We can try the reduction of order method to find the other solution given the existence of \(y_1\).
	      Let \(y_2(x) = v(x)y_1(x)\) for some function \(v\).
	      Then we have
	      \[
		      x^2(v''y_1 + 2v'y_1') = 0
	      \]
	      Let \(u=v'\), then
	      \begin{align*}
		      u'y_1 + 2uy_1 & = 0                                                                                                  \\
		      \frac{u'}{u}  & = -2\frac{y_1'}{y_1}                                                                                 \\
		      \ln u         & = \ln(y_1^{-2}) + \ln B                                                                              \\
		      u = v'        & = \frac{B}{y_1^2}                                                                                    \\
		      v'            & = \frac{B}{a_0^2 x^2} \left( 1 + \frac{x}{2} + \frac{x^2}{12} + \frac{x^3}{144} + \dots \right)^{-2}
	      \end{align*}
	      Note that the constant of integration gives a constant multiple of \(y_1\), and since the equation is homogeneous the constant does not matter.
	      We will expand this now using the binomial theorem, continually redefining constants since they are arbitrary, to give
	      \[
		      v' = \frac{B}{a_0^2}\left( \frac{1}{x^2} - \frac{1}{x} + \sum_{n=0}^\infty B_n x^n \right)
	      \]
	      for some constants \(B_n\).
	      Then integrating with respect to \(x\),
	      \begin{align*}
		      v          & = \frac{-B}{a_0^2}\frac{1}{x} - \frac{B}{a_0^2}\ln x + \sum_{n=1}^\infty C_n x^n \\
		      y_2 = vy_1 & = \frac{-B}{a_0} - \frac{B}{2a_0}x + \sum_{n=2}^\infty D_n x^n + Cy_1\ln x       \\
		                 & = \sum_{n=0}^\infty b_n x^n + cy_1\ln x
	      \end{align*}
	      So the appearance of \(\ln x\) is natural here.
\end{itemize}

\subsection{Example 2}
Let us revisit \eqref{fuch1}.
\[
	(1-x^2)y'' - 2xy' + 2y = 0
\]
Instead of expanding around \(x=0\), let us now consider expanding around \(x=-1\), a singular point.
We will redefine the independent variable, let
\[
	z = 1 + x \implies z(2-z)y'' - 2(z-1)y' + 2y = 0
\]
Now we will expand around \(z=0\).
We know that \(z=0\) is a regular singular point, so we will try solutions of the form
\[
	y = \sum_{n=0}^\infty a_n z^{n+\sigma};\quad a_0 \neq 0
\]
We have
\[
	\sum_{n=0}^\infty a_n z^{n+\sigma-1}\left[ (n+\sigma)(n+\sigma-1)(2-z) - 2(n+\sigma)(z-1) + 2z \right] = 0
\]
As before, we will equate the coefficients of the lowest power of \(z\) (for \(n=0\), these are the coefficients of \(z^{\sigma - 1}\)) to get the indicial equation and recursion relation.
\[
	2\sigma(\sigma - 1)a_0 + 2\sigma a_0 = 0 \implies \sigma^2 = 0
\]
So \(\sigma = 0\) is a repeated root.
Note that we need a term of the form \(y_1\ln (x-x_0)\) in this problem.
We will not complete this example here.
