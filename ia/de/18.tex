\subsection{Definition}
\begin{theorem}
	\begin{enumerate}
		\item If \(x_0\) is an ordinary point, then there are two linearly independent solutions of the form
		      \[
			      y = \sum_{n=0}^\infty a_n(x-x_0)^n
		      \]
		      This series is convergent in some region around \(x_0\).
		\item If \(x_0\) is a regular singular point, then there is at least one solution of the form
		      \[
			      y = \sum_{n=0}^\infty a_n(x-x_0)^{n + \sigma}
		      \]
		      where \(\sigma\) is real and \(a_0 \neq 0\).
	\end{enumerate}
\end{theorem}

\subsection{Application at Ordinary Points}
Here is an example of case 1.
\begin{equation}\label{fuch1}
	(1-x^2)y'' - 2xy' + 2y = 0
\end{equation}
We will try to find series solutions about \(x_0=0\), an ordinary point.
We will therefore try solutions of the form
\begin{align*}
	y   & = \sum_{n=0}^\infty a_n x^n           \\
	y'  & = \sum_{n=1}^\infty na_n x^{n-1}      \\
	y'' & = \sum_{n=2}^\infty n(n-1)a_n x^{n-2}
\end{align*}
Now, to make all powers of \(x\) at least \(n\), we will multiply \eqref{fuch1} by \(x^2\) for convenience.

\[
	(1-x^2)x^2y'' - 2x^3y' + 2x^2y = 0
\]
\[
	(1-x^2)x^2\sum_{n=2}^\infty n(n-1)a_n(x-x_0)^{n-2} - 2x^3\sum_{n=1}^\infty na_n(x-x_0)^{n-1} + 2x^2\sum_{n=0}^\infty a_n(x-x_0)^n = 0
\]
\[
	(1-x^2)\sum_{n=2}^\infty n(n-1)a_n x^n - 2x^2\sum_{n=1}^\infty na_n x^n + 2x^2\sum_{n=0}^\infty a_n x^n = 0
\]
\[
	\sum_{n=2}^\infty a_n[n(n-1)(1-x^2)]x^n - 2\sum_{n=1}^\infty a_n(nx^2)x^n + 2\sum_{n=0}^\infty a_n(x^2)x^n = 0
\]

Now, for \(n \geq 2\), equating the \(x^n\) coefficients we have
\[
	a_n[n(n-1)] - a_{n-2}[(n-2)(n-3)] - 2a_{n-2}(n-2) + 2a_{n-2} = 0
\]
This is a discrete equation.
Rewritten in a more standard form, we have
\[
	n(n-1)a_n = (n^2 - 3n)a_{n-2}
\]
or
\begin{equation}\label{fuch1recurrence}
	a_n = \frac{n-3}{n-1}a_{n-2}
\end{equation}
This is known as the recurrence relation.
The values of \(a_0\) and \(a_1\) are the unknown constants to be found via initial or boundary conditions.
Note that \(a_3 = 0\) from \eqref{fuch1recurrence}.
Therefore, any odd power of \(x\) of higher order than \(x^1\) is zero.
For even \(n\), we have
\begin{align*}
	a_n            & = \frac{n-3}{n-1}a_{n-2}                                         \\
	a_n            & = \frac{n-3}{n-1}\frac{n-5}{n-3}a_{n-4} = \frac{n-5}{n-1}a_{n-4} \\
	a_n            & = \frac{n-5}{n-1}\frac{n-7}{n-5}a_{n-6} = \frac{n-7}{n-1}a_{n-6} \\
	\therefore a_n & = \frac{-1}{n-1}a_0
\end{align*}
Therefore
\[
	y = a_1 x + a_0\left[ 1 - x^2 - \frac{x^4}{3} - \frac{x^6}{5} - \frac{x^8}{7} - \dots \right]
\]
Note that
\[
	\ln(1 \pm x) = \pm x - \frac{x^2}{2} \pm \frac{x^3}{3} - \dots
\]
Therefore
\[
	\ln(\frac{1+x}{1-x}) = \ln(1+x) - \ln(1-x) = 2x + 2\frac{x^3}{3} + 2\frac{x^5}{5} + \dots
\]
Hence,
\[
	y = a_1x + a_0\left[ 1-\frac{x}{2}\ln\left( \frac{1+x}{1-x} \right) \right]
\]
Note the behaviour of this function near the singular points of the original differential equation.

\subsection{Application at Regular Singular Points}
Consider the following differential equation:
\begin{equation}\label{fuch2}
	4xy'' + 2(1-x^2)y' - xy = 0
\end{equation}
We want to find series solutions about \(x=0\).
In this case, \(\frac{q}{p}\) is undefined at \(x=0\), so it is a singular point, but it is regular.
We will try solutions of the form
\begin{align*}
	y   & = \sum_{n=0}^\infty a_n x^{n + \sigma}                         \\
	y'  & = \sum_{n=0}^\infty (n+\sigma)a_n x^{n + \sigma-1}             \\
	y'' & = \sum_{n=0}^\infty (n+\sigma)(n+\sigma-1)a_n x^{n + \sigma-2} \\
\end{align*}
where \(a_0 \neq 0\).
For convenience we will multiply \eqref{fuch2} by \(x\):
\[
	4x^2y'' + 2(1-x^2)xy' - x^2y = 0
\]
\[
	4x^2\sum_{n=0}^\infty (n+\sigma)(n+\sigma-1)a_n x^{n + \sigma-2} + 2(1-x^2)x\sum_{n=0}^\infty (n+\sigma)a_n x^{n + \sigma-1} - x^2\sum_{n=0}^\infty a_n x^{n + \sigma}
\]
Hence,
\begin{equation}\label{fuch2expanded}
	\sum_{n=0}^\infty a_n x^{n + \sigma}\left[4(n+\sigma)(n+\sigma-1) + 2\left(1-x^2\right)(n+\sigma) - x^2\right] = 0
\end{equation}
We will equate coefficients of \(x^{n+\sigma}\) for \(n\geq 2\), since here all terms will make some contribution to the coefficient.
\[
	a_n\left[4(n+\sigma)(n+\sigma-1) + 2(n+\sigma)\right] + a_{n-2}\left[-2(n-2+\sigma) - 1\right] = 0
\]
Therefore,
\begin{equation}\label{fuch2recurrence}
	2(n+\sigma)(2n+2\sigma-1)a_n = (2n+2\sigma-3)a_{n-2}
\end{equation}
This is the recurrence relation, which we can use to compute the \(a_n\).
A general technique to find \(\sigma\) is to equate the coefficients of the lowest power of \(x\) in \eqref{fuch2expanded}.
By setting \(n=0\), we can equate coefficients of \(x^\sigma\), giving
\[
	a_0(4\sigma(\sigma - 1)) + a_0 2\sigma = 0
\]
But since \(a_0 \neq 0\) in Fuch's Theorem, we have
\[
	4\sigma(\sigma - 1) + 2\sigma = 0
\]
So either \(\sigma = 0\) or \(\sigma = \frac{1}{2}\).
We must consider these two cases individually.
\begin{itemize}
	\item (\(\sigma = 0\)) Equate coefficients of the lowest powers of \(x\) in \eqref{fuch2expanded}.
	      \begin{itemize}
		      \item (\(n=0\)) The coefficient of \(x^0\) gives
		            \[
			            a_0[4(0)(-1)] + a_0[2(0)] = 0
		            \]
		            which is true for all \(a_0\).
		            So \(a_0\) is an arbitrary constant.
		      \item (\(n=1\)) The coefficient of \(x^1\) gives
		            \[
			            a_1[4(1)(0)] + a_1[2(1)] = 0
		            \]
		            so \(a_1 = 0\).
	      \end{itemize}
	      From the recurrence relation \eqref{fuch2recurrence} which is valid for \(n \geq 2\), plugging in \(\sigma = 0\) gives
	      \begin{equation}\label{fuch2recurrencezero}
		      2n(2n - 1)a_n = (2n - 3)a_{n-2}
	      \end{equation}
	      Since \(a_1 = 0\), clearly all \(a_k = 0\) for odd \(k\).
	      Therefore, using the recurrence relation \eqref{fuch2recurrencezero} we have
	      \[
		      y = a_0 \left( 1 + \frac{x^2}{4 \cdot 3} + \frac{5x^4}{8\cdot 7 \cdot 4 \cdot 3} + \dots \right)
	      \]
	      
	\item (\(\sigma = \frac{1}{2}\)) This time we will start with the recurrence relation \eqref{fuch2recurrence} with \(\sigma = \frac{1}{2}\), relabelling \(a\) to \(b\) to avoid confusion.
	      \begin{equation}\label{fuch2recurrencehalf}
		      (2n+1)(2n)b_n = (2n-2)b_{n-2}
	      \end{equation}
	      Now let us analyse the coefficients of the lowest powers of \(x\), substituting into \eqref{fuch2expanded}.
	      \begin{itemize}
		      \item (\(n=0\)) The coefficient of \(x^{\frac{1}{2}}\) gives
		            \[
			            b_0\left[4\left(\frac{1}{2}\right)\left(\frac{-1}{2}\right)\right] + b_0\left[2\left(\frac{1}{2}\right)\right] = 0
		            \]
		            which is true for all \(b_0\).
		            So \(b_0\) is an arbitrary constant.
		      \item (\(n=1\)) The coefficient of \(x^{\frac{3}{2}}\) gives
		            \[
			            b_1\left[4\left(\frac{3}{2}\right)\left(\frac{1}{2}\right)\right] + b_1\left[2\left(\frac{3}{2}\right)\right] = 0
		            \]
		            so \(b_1 = 0\).
	      \end{itemize}
	      As before, all \(b_k = 0\) where \(k\) is odd.
	      Therefore, using the recurrence relation \eqref{fuch2recurrencehalf}, we have
	      \[
		      y = b_0x^{\frac{1}{2}} \left[ 1 + \frac{x^2}{2 \cdot 5} + \frac{3x^4}{2 \cdot 5 \cdot 4 \cdot 9} + \dots \right]
	      \]
\end{itemize}
So we have found two linearly independent solutions to the differential equation, given by boundary conditions \(a_0\) and \(b_0\).
Note that Fuch's Theorem only specifies that there will be at least one, but we have found two in this case.
