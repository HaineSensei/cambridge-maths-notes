\subsection{Higher order discrete equations}
The general form of an \(m\)th order linear discrete equation with constant coefficients is
\begin{equation}\label{mlindiscrete}
	a_m y_{n+m} + a_{m-1}y_{n+m-1} + \dots + a_1y_{n+1} + a_0y_n = f_n
\end{equation}
To solve such an equation, we will exploit some principles used to solve higher order differential equations.

To apply eigenfunction properties, we will define a difference operator \(D[y_n] = y_{n+1}\).
Then, \(D\) has eigenfunction \(y_n = k^n\) for \(k\) constant, since \(D[k^n] = k^{n+1} = k \cdot k^n = ky_n\).

To apply linearity, notice that \eqref{mlindiscrete} is linear in \(y\), so the general solution \(y_n = y_n^{(c)} + y_n^{(p)}\) where \(y^{(c)}\) is the complementary function and \(y^{(p)}\) is the particular integral.

As an example, let us consider a second order difference equation
\[
	a_2 y_{n+2} + a_1 y_{n+1} + a_0 y_n = f_n
\]
We will first try to solve the homogeneous equation, letting \(f=0\).
\[
	a_2 y_{n+2} + a_1 y_{n+1} + a_0 y_n = 0
\]
We will look for solutions of the form of the eigenfunction: \(y_n = k^n\).
\[
	a_2 k^2 + a_1 k + a_0 = 0
\]
This quadratic may be solved to give \(k_1\) and \(k_2\).
Then our complementary function is
\[
	y_n^{(c)} = \begin{cases}
		A k_1^n + B k_2^n & k_1 \neq k_2  \\
		A k^n + Bnk^n     & k_1 = k_2 = k
	\end{cases}
\]
To solve the particular integral, let us consult this table:

\begin{tabular}{cc}
	Form of \(f_n\)      & Form of \(y_n^{(p)}\)                \\\midrule
	\(k^n\)              & \(Ak^n\) if \(k \neq k_1, k_2\)      \\
	\(k_1^n\), \(k_2^n\) & \(Ank_1^n + Bnk_2^n\)                \\
	\(n^p\)              & \(An^p + Bn^{p-1} + \dots + Cn + D\)
\end{tabular}

\subsection{Fibonacci sequence}
The Fibonacci sequence is given by
\[
	y_n = y_{n-1} + y_{n-2}
\]
with initial conditions \(y_0 = y_1 = 1\).
In standard form, we have
\[
	y_{n+2} - y_{n+1} - y_n = 0
\]
We will look for solutions of the form \(y=k^n\).
Then
\[
	k^2 - k - 1 = 0
\]
So we have
\[
	k_1 = \phi = \frac{1 + \sqrt 5}{2};\quad k_2 = -\phi^{-1} = \frac{1 - \sqrt 5}{2}
\]
Solving for the initial conditions gives
\[
	y_n = \frac{1}{\sqrt 5}\phi + \frac{1}{\sqrt 5}\phi^{-1} = \frac{\phi^{n+1} - (-\phi^{-1})^{n+1}}{\sqrt{5}}
\]
So we can deduce that
\[
	\lim_{n \to \infty} \frac{y_{n+1}}{y_n} = \lim_{n \to \infty} \frac{\phi^{n+2} - (-\phi^{-1})^{n+2}}{\phi^{n+1} - (-\phi^{-1})^{n+1}} = \phi
\]

\subsection{Method of Frobenius}
The Method of Frobenius is a way of computing series solutions to linear homogeneous second order ODEs.
The general form is
\[
	p(x)y'' + q(x)y' + r(x)y = 0
\]
We will seek a power series expansion about some point \(x = x_0\).
First, we must classify the point \(x_0\):
\begin{itemize}
	\item (ordinary point) \(x=x_0\) is an ordinary point if the Taylor series of \(q/p\) and \(r/p\) converge in some region around \(x_0\); i.e.\ \(q/p\) and \(r/p\) are analytic.
	\item (singular point) If \(x_0\) is not ordinary, it is singular.
	      There are two types of singular points:
	      \begin{itemize}
		      \item (regular singular point) If the original ODE can be written as
		            \[
			            P(x)(x-x_0)^2 y'' + Q(x)(x-x_0)y' + R(x)y = 0
		            \]
		            and \(\frac{Q}{P}\) and \(\frac{R}{P}\) are analytic, then \(x=x_0\) is a regular singular point.
		            Note that \(\frac{Q}{P} = (x-x_0)\frac{q}{p}; \frac{R}{P}(x-x_0)^2\frac{r}{p}\).
		      \item (irregular singular point) Otherwise, \(x=x_0\) is an irregular singular point.
	      \end{itemize}
\end{itemize}
Here are some examples.
\begin{enumerate}
	\item \((1-x^2)y'' - 2xy' + 2y = 0\).
	      We have \(q/p = \frac{-2x}{1-x^2}\), so \(x = \pm 1\) are singular points.
	      But \(Q/P = \frac{2x}{1+x}\) which is regular at \(x=1\); a similar argument holds for \(-1\).
	\item \(y''\sin x + y'\cos x + 2y = 0\).
	      We have \(q/p = \cot x\), \(r/p = 2\csc x\).
	      So where \(x = n\pi\) where \(n \in \mathbb Z\), we have regular singular points.
	\item \((1+\sqrt{x})y'' - 2xy' + 2y = 0\).
	      We have \(q/p = \frac{-2x}{1+\sqrt{x}}\).
	      Around \(x=0\), the second derivative is undefined, so this is an irregular singular point.
\end{enumerate}
