\subsection{Eigenfunctions}
\begin{definition}
	The eigenfunction of an operator is a function that is unchanged by the action of the operator (except for a multiplicative scaling).
\end{definition}
From this definition, we can see that $e^{\lambda x}$ is the eigenfunction of the differential operator. The eigenvalue of this function is $\lambda$, as this is the scaling factor.

\subsection{Rules for Linear ODEs}
\begin{enumerate}
	\item Any linear homogeneous ODE with constant coefficients has solutions in the form $e^{\lambda x}$. For example, in the equation $5y' - 3y = 0$ we can try a solution of the form $y = Ae^{\lambda x}$, and we get $5 \lambda - 3 = 0$. This equation is known as the characteristic equation.
	\item Any solution to a linear homogeneous ODE can be scaled to create more solutions. In particular, $y=0$ is a solution.
	\item An $n$th order linear ODE has $n$ linearly independent solutions. In the case of constant coefficient equations, this follows from the Fundamental Theorem of Algebra. However, the proof of this is outside the scope of this course. This implies that the above example has only one solution: $y = Ae^{3x/5}$.
	\item An $n$th order ODE requires $n$ initial/boundary conditions to create a particular solution.
\end{enumerate}
