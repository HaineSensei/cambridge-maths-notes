\subsection{Taylor series}
Suppose that we want to approximate a function \(f(x)\) using a polynomial of order \(n\).
\[
	f(x) \approx \underbrace{a_0 + a_1 x + a_2 x^2 + \cdots + a_n x^n}_{\mathclap{\equiv p_n(x)}}
\]
By assuming that the equality holds, we may set \(x=0\) to get the value of \(a_0\).
By differentiating the left and right hand sides \(k\) times, we can evaluate both sides at \(x=0\) to get the value of \(a_k\).
Therefore, term \(a_k\) is equivalent to \(f^{(k)}(0)/k!
\)
\[
	f(x) \approx p_n(x) = f(0) + xf'(0) + \frac{x^2}{2}f''(0) + \cdots + \frac{x^n}{n!}f^{(n)}(0)
\]
Alternatively, repeating the process at \(x_0\), we get the formula for the Taylor polynomial of degree \(n\) of \(f(x)\):
\[
	f(x) \approx p_n(x) = f(x_0) + (x-x_0)f'(x_0) + \frac{(x-x_0)^2}{2}f''(x_0) + \cdots + \frac{(x-x_0)^n}{n!}f^{(n)}(x_0)
\]
We can write
\begin{equation}\label{taylorerror}
	f(x) = p_n(x) + E_n
\end{equation}
where \(E_n\) is the error at term \(n\).
Recall that \(f(x+h) = f(x) + hf'(x) + o(h)\) as \(h \to 0\).
We can generalise this, provided that the first \(n\) derivatives of \(f(x)\) exist.
\begin{equation}\label{xplush}
	f(x+h) = f(x) + hf'(x) + \frac{h^2}{2}f''(x) + \cdots + \frac{h^n}{n!}f^{(n)}(x) + o(h^n)
\end{equation}
Comparing Equations \eqref{taylorerror} and \eqref{xplush}, we see that:
\[
	E_n = o(h^n)
\]
\begin{theorem}[Taylor's Theorem]
	\(E_n = O(h^{n+1})\) as \(h \to 0\) provided that \(f^{(n+1)}(x)\) exists.
\end{theorem}
Note that the big \(O\) notation in Taylor's Theorem is a stronger statement than the little \(o\) notation above.
For example, \(h^{n+a}=o(h^n)\) as \(h \to 0\ \forall a \in (0, 1)\) since \(\lim_{h\to 0} \frac{h^n+a}{h^n} = \lim_{h\to 0} h^a = 0\).
However, \(h^{n+a} \neq O(h^{n+1})\) as \(h \to 0\) for \(a \in (0, 1)\) because we can't bound \(h^{n+a}\) using \(h^{n+1}\) everywhere in the vicinity of \(0\).

\subsection{L'H\^opital's rule}
Let \(f(x)\) and \(g(x)\) be differentiable functions at \(x=x_0\), and that \(\lim_{x\to x_0} f(x) = f(x_0) = 0\) and similarly for \(g(x)\).
L'H\^opital's Rule states that
\[
	\lim_{x\to x_0} \frac{f(x)}{g(x)} = \lim_{x\to x_0} \frac{f'(x)}{g'(x)} \text{ if } g'(x_0) \neq 0
\]
\begin{proof}
	As \(x \to x_0\):
	\begin{align*}
		f(x)              & = f(x_0) + (x-x_0) f'(x_0) + o(x-x_0)                                       \\
		g(x)              & = g(x_0) + (x-x_0) g'(x_0) + o(x-x_0)                                       \\
		\intertext{But we know that \(f(x_0)=g(x_0)=0\) therefore}
		\frac{f(x)}{g(x)} & = \frac{f'(x_0) + \frac{o(x-x_0)}{x-x_0}}{g'(x_0) + \frac{o(x-x_0)}{x-x_0}} \\
		\intertext{By the definition of little \(o\), \(o(h)/h\) tends to zero, so}
		\frac{f(x)}{g(x)} & = \frac{f'(x)}{g'(x)}
	\end{align*}
\end{proof}
Note that L'H\^opital's rule can be applied recursively, using higher-order derivatives.
For example, consider \(f(x) = 3\sin x - \sin 3x\); \(g(x) = 2x - \sin 2x\).
The limit approaches 3 as \(x \to 0\).
