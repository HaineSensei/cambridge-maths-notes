\subsection{Linear 2nd Order ODEs with Constant Coefficients}
The general form of an equation of this type is
\[ ay'' + by' + cy = f(x) \]
To solve equations like this, we are going to exploit two facts: the linearity of the differential operator together with the principle of superposition. From the definition of the derivative, we have
\[ \frac{\dd}{\dd{x}}(y_1 + y_2) = y_1' + y_2' \]
And similarly,
\[ \frac{\dd^2}{\dd{x}^2}(y_1 + y_2) = y_1'' + y_2'' \]
For a linear differential operator \(D\) built from a linear combination of derivatives, for example
\[ D = \left[ a \frac{\dd^2}{\dd{x}^2} + b\frac{\dd}{\dd{x}} + c \right] \]
it then follows that
\[ D(y_1 + y_2) = D(y_1) + D(y_2) \]
We will then solve the above general equation in three steps.
\begin{enumerate}
	\item Find the complementary functions \(y_1\) and \(y_2\) which satisfy the equivalent homogeneous equation \(ay'' + by' + cy = 0\).
	\item Find a particular integral \(y_p\) which solves the original equation.
	\item If \(y_1\) and \(y_2\) are linearly independent, then \(y_1 + y_p\) and \(y_2 + y_p\) are each linearly independent solutions, which follows from the fact that \(D(y_1) = D(y_2) = 0\) and \(D(y_p) = f(x)\).
\end{enumerate}

\subsection{Eigenfunctions for 2nd Order ODEs}
\(e^{\lambda x}\) is the eigenfunction of \(\frac{\dd}{\dd{x}}\), and it is also the eigenfunction of \(\frac{\dd^2}{\dd{x}^2}\), but with eigenvalue \(\lambda^2\). More generally, it is the eigenfunction of \(\frac{\dd^n}{\dd{x}^n}\) with eigenfunction \(\lambda^n\). In fact, \(e^{\lambda x}\) is the eigenfunction of any linear differential operator \(D\). The equation \( ay'' + by' + cy = 0 \) can be written
\[ \underbrace{\left[ a \frac{\dd^2}{\dd{x}^2} + b\frac{\dd}{\dd{x}} + c \right]}_{\equiv\ D} y = 0 \]
Therefore, solutions to this take the form
\[ y_c = Ae^{\lambda x} \]
and by substituting, we have
\[ a \lambda^2 + b\lambda + c = 0 \]
This is known as the characteristic (or auxiliary) equation. From the fundamental theorem of algebra, this must have two real or complex solutions. Now, let \(\lambda_1, \lambda_2\) be these roots.

In the case that \(\lambda_1 \neq \lambda_2\), \(y_1 = Ae^{\lambda_1 x}; y_2 = Be^{\lambda_2 x}\). In this case, the two are linearly independent and complete; they form a basis of solution space. Therefore any other solution to this differential equation can be written as a linear combination of \(y_1\) and \(y_2\). In general, \(y_c = Ae^{\lambda_1 x} + Be^{\lambda_2 x}\).

\subsection{Detuning}
In the case that \(\lambda_1 = \lambda_2\), this is known as a degenerate case as we have repeated eigenvalues; \(y_1\) and \(y_2\) are linearly dependent and not complete. Let us take as an example the differential equation \(y'' - 4y' + 4y = 0\). We try \(y_c = e^{2x}\) as \(\lambda = 2\) in this case. We will consider a slightly modified (`detuned') equation to rectify the degeneracy.
\[ y'' - 4y' + (4-\varepsilon^2)y = 0 \text{ where } \varepsilon \ll 1 \]
Again we will try \(y_c = e^{\lambda x}\), giving
\[ \lambda^2 - 4 \lambda + (4 - \varepsilon^2) = 0 \]
So we have \(\lambda = 2 \pm \varepsilon\). The complementary function therefore is \(y_c = Ae^{(2+\varepsilon)x} + Be^{(2-\varepsilon)x} = e^{2x}\left( Ae^{\varepsilon x} + Be^{-\varepsilon x} \right)\). We will expand this in a Taylor series for small \(\varepsilon\), giving
\[ y_c = e^{2x}\left[ (A + B) + \varepsilon x(A - B) + O(\varepsilon^2) \right] \]
and by taking the limit, we have
\[ \lim_{\varepsilon \to 0} y_c \approx e^{2x} \left[ (A + B) + \varepsilon x(A - B) \right] \]
Now consider applying initial conditions to \(y_c\) at \(x = 0\).
\[ \eval{y_c}_{x=0} = C\quad \eval{y_c'}_{x=0} = D \]
and therefore
\[ C = A + B;\quad D = 2C + \varepsilon(A - B) \]
hence
\[ A + B = O(1);\quad A - B = O\left(\frac{1}{\varepsilon}\right) \]
in order that \(D\) is a constant. Now, let \(\alpha = A + B; \beta = \varepsilon(A - B)\), so that we can get constants of \(O(1)\) magnitude. Hence,
\[ \lim_{\varepsilon \to 0} y_c = e^{2x}\left[ \alpha + \beta x \right] \]
In general, if \(y_1(x)\) is a degenerate complementary function for linear ODEs with constant coefficients, then \(y_2 = xy_1\) is a linearly independent complementary function.
