\subsection{Reduction of Order}
Consider a homogeneous second-order linear ODE with non-constant coefficients. The general form of such an equation is
\begin{equation}\label{hom_2_lin_ode_const}
	y'' + p(x) y' + q(x) y = 0
\end{equation}
Our objective is to use one solution to this equation (here denoted $y_1$) to find the other solution $y_2$. The general idea is to look for a solution of the form
\begin{equation}\label{y2_y1_reduction}
	y_2(x) = v(x)y_1(x)
\end{equation}
First, note that
\begin{align*}
	y_2'  & = v'y_1 + vy_1'             \\
	y_2'' & = v''y_1 + 2v'y_1' + vy_1''
\end{align*}
If $y_2$ is a solution to \eqref{hom_2_lin_ode_const}, then
\[ y_2'' + p(x) y_2' + q(x) y_2 = 0 \]
We can use \eqref{y2_y1_reduction} and collect terms, to get
\[ v\cdot\underbrace{(y_1'' + py_1' + qy_1)}_{\mathclap{\text{0 since $y_1$ is a solution to \eqref{hom_2_lin_ode_const}}}} + v'\cdot(2y_1' + py_1) + v''\cdot y_1 = 0 \]
Hence
\[ v'\cdot (2y_1' + py_1) + v'' \cdot y_1 = 0 \]
This is a first order differential equation for $v'(x)$. Let $u=v'$. Then
\[ u'y_1 + u (2y_1' + py_1) = 0 \]
This is a separable first order ODE for $u(x)$. So we can solve for $u(x)$ and deduce $v(x)$ by integration.

\subsection{Solution Space}
An $n$th order linear ODE written
\[ p(x)y^{(n)} + q(x) y^{(n-1)} + \cdots + r(x)y = f(x) \]
can be used to write $y^{(n)}(x)$ in terms of lower derivatives of $y$. For example, the oscillations of a mass on a spring in a damped system can be modelled as
\[ m\ddot y = -k y - L\dot y \]
Therefore the state of the system can be described by an $n$-dimensional solution vector
\begin{equation}
	\vb Y(x) \equiv \begin{pmatrix}
		y(x) \\ y'(x) \\ \vdots \\ y^{(n-1)}(x)
	\end{pmatrix}
\end{equation}
For example, an undamped oscillator modelled by $y'' + 4y = 0$ has solutions
\[ y_1 = \cos 2x;\quad y_2 = \sin 2x \]
and has derivatives
\[ y_1' = -2\sin 2x;\quad y_2' = 2\cos 2x \]
and therefore two solution vectors are
\[
	\vb Y_1(x) = \begin{pmatrix}
		y_1 \\ y_1'
	\end{pmatrix} = \begin{pmatrix}
		\cos 2x \\ -2 \sin 2x
	\end{pmatrix}
\]
and
\[
	\vb Y_2(x) = \begin{pmatrix}
		y_2 \\ y_2'
	\end{pmatrix} = \begin{pmatrix}
		\sin 2x \\ 2 \cos 2x
	\end{pmatrix}
\]

\begin{wrapfigure}{l}{0.3\textwidth}
	\begin{tikzpicture}
		\begin{axis}[
				%axis lines = left,
				xlabel = $y$,
				ylabel = $\dot y$,
				width=5cm,
				height=5cm,
				xmin=-2.4,
				xmax=2.4,
				ymin=-2.4,
				ymax=2.4,
				xticklabel=\empty,
				yticklabel=\empty
			]

			\addplot [
				domain=-1:1,
				samples=200,
				color=red,
			]
			{2*sqrt(1-x^2)};
			\addplot [
				domain=-1:1,
				samples=200,
				color=red,
			]
			{-2*sqrt(1-x^2)};

			\draw[red] (axis cs: 1,0.05) -- (axis cs: 1,-0.05);
		\end{axis}
	\end{tikzpicture}
\end{wrapfigure}

We can plot the paths of these two solutions using a two-dimensional phase portrait. In this case, both solutions follow an elliptical path. Since $\vb Y_1$ and $\vb Y_2$ are linearly independent for all $x$, any point in solution space $(y, y')$ can be written as a linear combination of these solutions.

Solutions $y_1, y_2, \cdots, y_n$ are linearly independent for any ODE if their solution vectors $\vb Y_1, \vb Y_2, \cdots, \vb Y_n$ are linearly independent. A set of $n$ linearly independent solution vectors forms a basis for the solution space of an $n$th order ODE.

\subsection{Initial Conditions}
Consider initial conditions for a second order homogeneous ODE.
\[ y(0) = a,\quad y'(0) = b \]
If the general solution is
\[ y(x) = Ay_1(x) + By_2(x) \]
then we have the following linear system of equations
\begin{align*}
	Ay_1(0) + By_2(0)   & = a \\
	Ay_1'(0) + By_2'(0) & = b
\end{align*}
which is a system of two equations for two unknowns. Or alternatively,
\[
	\underbrace{\begin{pmatrix}
			y_1(0)  & y_2(0)  \\
			y_1'(0) & y_2'(0)
		\end{pmatrix}}_{\equiv M}
	\begin{pmatrix}
		A \\ B
	\end{pmatrix}
	=
	\begin{pmatrix}
		a \\b
	\end{pmatrix}
\]
Unique solutions for $A$ and $B$ exist if $\det M \neq 0$.

\subsection{Wronskian}
The fundamental matrix is a matrix formed by placing solution vector $\vb Y_i$ in the $i$th column. The Wronskian, denoted $W(x)$, is the determinant of the fundamental matrix.
\[
	W(x) \equiv \begin{vmatrix}
		\vdots  & \vdots  &        & \vdots  \\
		\vb Y_1 & \vb Y_2 & \cdots & \vb Y_n \\
		\vdots  & \vdots  &        & \vdots
	\end{vmatrix} = \begin{vmatrix}
		y_1         & y_2         & \cdots & y_n         \\
		y_1'        & y_2'        & \cdots & y_n'        \\
		\vdots      & \vdots      & \ddots & \vdots      \\
		y_1^{(n-1)} & y_2^{(n-1)} & \cdots & y_n^{(n-1)}
	\end{vmatrix}
\]
For a second order ODE:
\begin{equation}\label{wronskian2}
	W(x) = \begin{vmatrix}
		y_1 & y_2 \\ y_1' & y_2'
	\end{vmatrix}
	= y_1y_2' - y_2y_1'
\end{equation}
The solution vectors are linearly independent if $W(x) \neq 0$. This is a convenient test for the linear independence of two solution vectors. In our example above, we had
\[
	W(x) = \begin{vmatrix}
		\cos 2x   & \sin 2x  \\
		-2\sin 2x & 2\cos 2x
	\end{vmatrix}
	= 2\cos^2 2x + 2\sin^2 2x = 2 \neq 0
\]
So the solution vectors are linearly independent for all $x$.

\subsection{Reverse Implication of Wronskian}
If $\vb Y_1$ and $\vb Y_2$ are linearly dependent, then $W(x) = 0$. Suppose that a third solution $y(x)$ is a linear combination of $y_1(x)$ and $y_2(x)$. Then the solution vector $\vb Y, \vb Y_1, \vb Y_2$ are a linearly dependent set. Hence
\[
	\begin{vmatrix}
		y   & y_1   & y_2   \\
		y'  & y_1'  & y_2'  \\
		y'' & y_1'' & y_2''
	\end{vmatrix}
	= 0
\]
For $y_1 = \cos 2x$ and $y_2 = \sin 2x$, we can deduce the original differential equation that produced these solutions by solving for $y$.
\begin{align*}
	\begin{vmatrix}
		y   & \cos 2x   & \sin 2x    \\
		y'  & -2\sin 2x & 2 \cos 2x  \\
		y'' & -4\cos 2x & -4 \sin 2x
	\end{vmatrix}          & = 0 \\
	\implies y(8\sin^2 2x + 8\cos^2 2x) &     \\ - y'(-4 \cos 2x \sin 2x + 4 \cos 2x \sin 2x) &\\ + y''(2\cos^2 2x + 2\sin^2 2x) &= 0 \\
	\implies y'' + 4y                   & = 0
\end{align*}
Note that if $W(x) = 0$, this does not necessarily imply linear dependence.
