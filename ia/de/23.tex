\subsection{Lotka-Volterra equations}
This is a worked example of a coupled set of differential equations, which model a predator-prey system.
Let the quantity of prey be represented by \(x\), and the quantity of the predator be \(y\).
Then
\begin{align*}
	\dot x & = \alpha x - \beta xy = f(x, y) \\
	\dot y & = \delta xy - \gamma y = g(x,y)
\end{align*}
where \(\alpha, \beta, \gamma, \delta\) are positive real constants.
We will start by analysing the fixed points, where \(\dot x = \dot y = 0\).
\[
	\dot x = 0 \implies x=0 \text{ or } y = \frac{\alpha}{\beta}
\]
\[
	\dot y = 0 \implies y=0 \text{ or } x = \frac{\gamma}{\delta}
\]
Therefore,
\[
	(x_0, y_0) = (0, 0), \left( \frac{\gamma}{\delta}, \frac{\alpha}{\beta} \right)
\]
Using matrix methods,
\[
	M = \begin{pmatrix}
		f_x & f_y \\
		g_x & g_y
	\end{pmatrix} = \begin{pmatrix}
		\alpha - \beta y & -\beta x          \\
		\delta y         & \delta x - \gamma
	\end{pmatrix}
\]
Now we can analyse the stability of these fixed points by perturbation analysis.
\begin{itemize}
	\item At the fixed point \((0, 0)\), we have
	      \[
		      \begin{pmatrix}
			      \dot \xi \\
			      \dot \eta
		      \end{pmatrix} = \begin{pmatrix}
			      \alpha & 0 \\ 0 & -\gamma
		      \end{pmatrix} \begin{pmatrix}
			      \xi \\ \eta
		      \end{pmatrix}
	      \]
	      We can read off the eigenvalues to be \(\alpha\) and \(-\gamma\).
	      This is a saddle node, since one direction will increase (\(x\)) and one will decrease (\(y\)).
	\item At the fixed point \(\left( \frac{\gamma}{\delta}, \frac{\alpha}{\beta} \right)\), we have
	      \[
		      \begin{pmatrix}
			      \dot \xi \\
			      \dot \eta
		      \end{pmatrix} = \begin{pmatrix}
			      0 & -\beta\frac{\gamma}{\delta} \\ \delta\frac{\alpha}{\beta} & 0
		      \end{pmatrix} \begin{pmatrix}
			      \xi \\ \eta
		      \end{pmatrix}
	      \]
	      The characteristic equation is \(\chi_M(\lambda) = \lambda^2 + \alpha\lambda = 0\), so \(\lambda = \pm \sqrt{-\alpha\gamma}\).
	      Since \(\alpha\gamma > 0\), it is more convenient to write \(\lambda = \pm i\sqrt{\alpha\gamma}\).
	      Since the real part is zero, this gives a centre node.
	      To work out the direction of rotation, let us consider the \(x\) direction,
	      \[
		      \dot \xi = -\beta\frac{\gamma}{\delta}\eta
	      \]
	      If \(\eta > 0\), then \(\dot \xi < 0\), so we have anticlockwise rotation.
\end{itemize}
Now we can sketch a graph taking into account both of these fixed points, visually interpolating the values between them.
% TODO do this graph

\subsection{First order wave equation and method of characteristics}
We will define a partial differential equation to be a differential equation with multiple independent variables.
Here, we will consider three examples, starting with the first order wave equation.
Consider a function \(y(x, t)\) where
\begin{equation}\label{wave1}
	\frac{\partial y}{\partial t} - c\frac{\partial y}{\partial x} = 0
\end{equation}
where \(c\) is a constant.
We will solve this equation with the method of characteristics.
%TODO random-looking graph
Imagine moving a `probe' along a path \(x(t)\).
Then \(y\) is a function \(y(x(t), t)\), where now the only independent variable is \(t\).
Using the multivariate chain rule,
\[
	\frac{\dd{y}}{\dd{t}} = \frac{\partial y}{\partial t} + \frac{\partial y}{\partial x}\frac{\dd{x}}{\dd{t}}
\]
Comparing this with \eqref{wave1}, we note that if \(\frac{\dd{x}}{\dd{t}} = -c\), then \(\frac{\dd{y}}{\dd{t}} = 0\).
So we have found a path along which the `probe' is at a constant height, i.e.\ along \(x(t) = x_0 - ct\), \(y\) is a constant.
We can update our graph now showing the `characteristics' we have just shown to exist.
%TODO graph
If \(y(x, t = 0) = f(x)\), then \(y = f(x_0)\) along the characteristics.
Hence, our general solution is
\[
	y = f(x+ct)
\]
Let us consider some examples of wave equations \(f\).
\begin{enumerate}
	\item (unforced wave equation) Let \(y(x,0) = x^2 - 3\) in \eqref{wave1}.
	      Then
	      \[
		      y(x, t) = (x+ct)^2 - 3
	      \]
	\item (forced wave equation) Let
	      \[
		      \frac{\partial y}{\partial t} + 5\frac{\partial y}{\partial x} = e^{-t}
	      \]
	      and
	      \[
		      y(x, 0) = e^{-x^2}
	      \]
	      Then along paths with \(\frac{\dd{x}}{\dd{t}} = 5\) or \(x=x_0 + 5t\),
	      \[
		      \frac{\dd{y}}{\dd{t}} = e^{-t}
	      \]
	      So by integration,
	      \[
		      y = A-e^{-t}
	      \]
	      along these paths.
	      Applying our initial condition at \(t=0\), our `probe' is at \(x_0\) and \(y(x, 0) = A - 1 = e^{-x_0^2}\).
	      Hence, \(A = 1 + e^{-x_0^2}\).
	      So
	      \[
		      y = 1 + e^{-x_0^2} - e^{-t}
	      \]
	      along the path given by \(x_0\).
	      Substituting back for a general \(x\), we can create a formula for the general solution of \(y\) (not necessarily on a given path):
	      \[
		      y = 1 + e^{-(x-5t)^2} - e^{-t}
	      \]
\end{enumerate}
