\subsection{General rules}
To solve a differential equation, we can use the following technique to break it apart into two smaller functions:
\[
	y = y_p + y_c
\]
The function \(y_p\) is called the particular integral; it is simply any solution
the original equation.
Normally this does not have any arbitrary constants in it.
The other function \(y_c\) is the complementary function.
This is a solution to the equivalent homogeneous equation, which is formed by setting the right hand side (the side without the dependent variable) to zero.
This is generally easier to solve using the exponential function.

By adding the two together, we get the general solution.
Alternatively, once we have computed the particular integral, we can simply substitute the equation \(y = y_p + y_c\) into the original differential equation to get a new equation in terms of \(y_c\).

Note that we refer to terms that do not depend on the dependent variable as `forcing functions'.

\subsection{Constant forcing}
If the equation is linear, has constant coefficients and a constant on the right hand side, we can set \(y_p' = 0\).
For example, in the equation \(5y' - 3y = 10\), we can set \(y' = 0\) to get \(y_p = -10/3\).

Now we can insert this general solution into the differential equation.
Note that all terms with \(y_p\), along with the right hand side, cancel out because it is a solution.
This leaves \(5y_c' - 3y = 0\).
We can solve this normally (using methods such as trying \(Ae^{\lambda x}\) or just directly solving the characteristic equation) to give \(y_c = Ae^{-3x/5}\).

Combining the results, we get \(y = Ae^{3x/5} - 10/3\).

\subsection{Eigenfunction forcing}
If the equation has a \(e^{kt}\) term as the only forcing function (where the independent variable here is \(t\)), we can solve it in a similar way.
Here is an example question involving this concept.

\begin{quote}
	In a sample of rock, isotope A decays into isotope B at a rate proportional to \(a\), the number of nuclei of A, while B decays into isotope C at a rate proportional to \(b\), the number of nuclei of B.
	Find \(b(t)\).
\end{quote}

\noindent We can formulate an equation as follows:
\begin{align*}
	\dot a                    & = -k_a a \implies a = a_0 e^{-k_a t} \\
	\dot b                    & = k_a a - k_b b                      \\
	\therefore \dot b + k_b b & = k_a a_0 e^{-k_a t}
\end{align*}
\noindent So we have a linear first order ODE with an eigenfunction as the forcing function.
We can guess that the particular integral is of the form \(b_p = \lambda e^{-k_a t}\).
\begin{align*}
	-k_a\lambda e^{-k_a t} + k_b \lambda e^{-k_a t} & = k_a a_0 e^{-k_a t}        \\
	\lambda(k_b-k_a)                                & = k_a a_0                   \\
	\therefore \lambda                              & = \frac{k_a}{k_b - k_a} a_0
\end{align*}
We can form the complementary function by solving:
\begin{align*}
	\dot{b_c} + k_b b_c & = 0           \\
	\therefore b_c      & = Ae^{-k_b t}
\end{align*}
So combining everything, we have
\[
	b = \frac{k_a}{k_b - k_a} a_0 e^{-k_a t} + Ae^{-k_b t}
\]
In this instance, there is a special property that if \(b=0\) at \(t=0\), then we can divide \(b(t)/a(t)\) and completely eliminate \(a_0\), thus letting us calculate the age of a rock without knowing the original amount of isotope A at all.

\subsection{Non-constant coefficients}
If we have a differential equation in standard form, i.e.
\[
	y' + p(x)y = f(x)
\]
we can multiply the equation by an integrating factor \(\mu\) to solve it.
Ideally, we want the derivative of \(\mu\) to be \(\mu p(x)\) so that the equation becomes
\[
	\mu y' + \mu p(x) y = \mu y' + \mu' y = (\mu y)' = \mu f(x)
\]
So therefore \(\mu = e^{\int p(x)\ \dd{x}}\).
