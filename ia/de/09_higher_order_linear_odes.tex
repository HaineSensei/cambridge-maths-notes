\subsection{Linear 2nd order ODEs with constant coefficients}
The general form of an equation of this type is
\[
	ay'' + by' + cy = f(x)
\]
To solve equations like this, we are going to exploit two facts: the linearity of the differential operator together with the principle of superposition.
From the definition of the derivative, we have
\[
	\frac{\dd}{\dd{x}}(y_1 + y_2) = y_1' + y_2'
\]
And similarly,
\[
	\frac{\dd^2}{\dd{x}^2}(y_1 + y_2) = y_1'' + y_2''
\]
For a linear differential operator \(D\) built from a linear combination of derivatives, for example
\[
	D = \left[ a \frac{\dd^2}{\dd{x}^2} + b\frac{\dd}{\dd{x}} + c \right]
\]
it then follows that
\[
	D(y_1 + y_2) = D(y_1) + D(y_2)
\]
We will then solve the above general equation in three steps.
\begin{enumerate}
	\item Find the complementary functions \(y_1\) and \(y_2\) which satisfy the equivalent homogeneous equation \(ay'' + by' + cy = 0\).
	\item Find a particular integral \(y_p\) which solves the original equation.
	\item If \(y_1\) and \(y_2\) are linearly independent, then \(y_1 + y_p\) and \(y_2 + y_p\) are each linearly independent solutions, which follows from the fact that \(D(y_1) = D(y_2) = 0\) and \(D(y_p) = f(x)\).
\end{enumerate}

\subsection{Eigenfunctions for 2nd order ODEs}
\(e^{\lambda x}\) is the eigenfunction of \(\frac{\dd}{\dd{x}}\), and it is also the eigenfunction of \(\frac{\dd^2}{\dd{x}^2}\), but with eigenvalue \(\lambda^2\).
More generally, it is the eigenfunction of \(\frac{\dd^n}{\dd{x}^n}\) with eigenvalue \(\lambda^n\).
In fact, \(e^{\lambda x}\) is the eigenfunction of any linear differential operator \(D\).
The equation \( ay'' + by' + cy = 0 \) can be written
\[
	\underbrace{\left[ a \frac{\dd^2}{\dd{x}^2} + b\frac{\dd}{\dd{x}} + c \right]}_{\equiv\ D} y = 0
\]
Therefore, solutions to this take the form
\[
	y_c = Ae^{\lambda x}
\]
and by substituting, we have
\[
	a \lambda^2 + b\lambda + c = 0
\]
This is known as the characteristic (or auxiliary) equation.
From the fundamental theorem of algebra, this must have two real or complex solutions.
Now, let \(\lambda_1, \lambda_2\) be these roots.

In the case that \(\lambda_1 \neq \lambda_2\), \(y_1 = Ae^{\lambda_1 x}; y_2 = Be^{\lambda_2 x}\).
In this case, the two are linearly independent and complete; they form a basis of solution space.
Therefore any other solution to this differential equation can be written as a linear combination of \(y_1\) and \(y_2\).
In general, \(y_c = Ae^{\lambda_1 x} + Be^{\lambda_2 x}\).

\subsection{Detuning}
In the case that \(\lambda_1 = \lambda_2\), this is known as a degenerate case as we have repeated eigenvalues; \(y_1\) and \(y_2\) are linearly dependent and not complete.
Let us take as an example the differential equation \(y'' - 4y' + 4y = 0\).
We try \(y_c = e^{2x}\) as \(\lambda = 2\) in this case.
We will consider a slightly modified (`detuned') equation to rectify the degeneracy.
\[
	y'' - 4y' + (4-\varepsilon^2)y = 0 \text{ where } \varepsilon \ll 1
\]
Again we will try \(y_c = e^{\lambda x}\), giving
\[
	\lambda^2 - 4 \lambda + (4 - \varepsilon^2) = 0
\]
So we have \(\lambda = 2 \pm \varepsilon\).
The complementary function therefore is \(y_c = Ae^{(2+\varepsilon)x} + Be^{(2-\varepsilon)x} = e^{2x}\left( Ae^{\varepsilon x} + Be^{-\varepsilon x} \right)\).
We will expand this in a Taylor series for small \(\varepsilon\), giving
\[
	y_c = e^{2x}\left[ (A + B) + \varepsilon x(A - B) + O(\varepsilon^2) \right]
\]
and by taking the limit, we have
\[
	\lim_{\varepsilon \to 0} y_c \approx e^{2x} \left[ (A + B) + \varepsilon x(A - B) \right]
\]
Now consider applying initial conditions to \(y_c\) at \(x = 0\).
\[
	\eval{y_c}_{x=0} = C\quad \eval{y_c'}_{x=0} = D
\]
and therefore
\[
	C = A + B;\quad D = 2C + \varepsilon(A - B)
\]
hence
\[
	A + B = O(1);\quad A - B = O\left(\frac{1}{\varepsilon}\right)
\]
in order that \(D\) is a constant.
Now, let \(\alpha = A + B; \beta = \varepsilon(A - B)\), so that we can get constants of \(O(1)\) magnitude.
Hence,
\[
	\lim_{\varepsilon \to 0} y_c = e^{2x}\left[ \alpha + \beta x \right]
\]
In general, if \(y_1(x)\) is a degenerate complementary function for linear ODEs with constant coefficients, then \(y_2 = xy_1\) is a linearly independent complementary function.

\subsection{Reduction of order}
Consider a homogeneous second-order linear ODE with non-constant coefficients.
The general form of such an equation is
\begin{equation}\label{hom_2_lin_ode_const}
	y'' + p(x) y' + q(x) y = 0
\end{equation}
Our objective is to use one solution to this equation (here denoted \(y_1\)) to find the other solution \(y_2\).
The general idea is to look for a solution of the form
\begin{equation}\label{y2_y1_reduction}
	y_2(x) = v(x)y_1(x)
\end{equation}
First, note that
\begin{align*}
	y_2'  & = v'y_1 + vy_1'             \\
	y_2'' & = v''y_1 + 2v'y_1' + vy_1''
\end{align*}
If \(y_2\) is a solution to \eqref{hom_2_lin_ode_const}, then
\[
	y_2'' + p(x) y_2' + q(x) y_2 = 0
\]
We can use \eqref{y2_y1_reduction} and collect terms, to get
\[
	v\cdot\underbrace{(y_1'' + py_1' + qy_1)}_{\mathclap{\text{0 since \(y_1\) is a solution to \eqref{hom_2_lin_ode_const}}}} + v'\cdot(2y_1' + py_1) + v''\cdot y_1 = 0
\]
Hence
\[
	v'\cdot (2y_1' + py_1) + v'' \cdot y_1 = 0
\]
This is a first order differential equation for \(v'(x)\).
Let \(u=v'\).
Then
\[
	u'y_1 + u (2y_1' + py_1) = 0
\]
This is a separable first order ODE for \(u(x)\).
So we can solve for \(u(x)\) and deduce \(v(x)\) by integration.

\subsection{Solution space}
An \(n\)th order linear ODE written
\[
	p(x)y^{(n)} + q(x) y^{(n-1)} + \cdots + r(x)y = f(x)
\]
can be used to write \(y^{(n)}(x)\) in terms of lower derivatives of \(y\).
For example, the oscillations of a mass on a spring in a damped system can be modelled as
\[
	m\ddot y = -k y - L\dot y
\]
Therefore the state of the system can be described by an \(n\)-dimensional solution vector
\begin{equation}
	\vb Y(x) \equiv \begin{pmatrix}
		y(x) \\ y'(x) \\ \vdots \\ y^{(n-1)}(x)
	\end{pmatrix}
\end{equation}
For example, an undamped oscillator modelled by \(y'' + 4y = 0\) has solutions
\[
	y_1 = \cos 2x;\quad y_2 = \sin 2x
\]
and has derivatives
\[
	y_1' = -2\sin 2x;\quad y_2' = 2\cos 2x
\]
and therefore two solution vectors are
\[
	\vb Y_1(x) = \begin{pmatrix}
		y_1 \\ y_1'
	\end{pmatrix} = \begin{pmatrix}
		\cos 2x \\ -2 \sin 2x
	\end{pmatrix}
\]
and
\[
	\vb Y_2(x) = \begin{pmatrix}
		y_2 \\ y_2'
	\end{pmatrix} = \begin{pmatrix}
		\sin 2x \\ 2 \cos 2x
	\end{pmatrix}
\]

\begin{wrapfigure}{l}{0.3\textwidth}
	\begin{tikzpicture}
		\begin{axis}[
				%axis lines = left,
				xlabel = \(y\),
				ylabel = \(\dot y\),
				width=5cm,
				height=5cm,
				xmin=-2.4,
				xmax=2.4,
				ymin=-2.4,
				ymax=2.4,
				xticklabel=\empty,
				yticklabel=\empty
			]

			\addplot [
				domain=-1:1,
				samples=200,
				color=red,
			]
			{2*sqrt(1-x^2)};
			\addplot [
				domain=-1:1,
				samples=200,
				color=red,
			]
			{-2*sqrt(1-x^2)};

			\draw[red] (axis cs: 1,0.05) -- (axis cs: 1,-0.05);
		\end{axis}
	\end{tikzpicture}
\end{wrapfigure}

We can plot the paths of these two solutions using a two-dimensional phase portrait.
In this case, both solutions follow an elliptical path.
Since \(\vb Y_1\) and \(\vb Y_2\) are linearly independent for all \(x\), any point in solution space \((y, y')\) can be written as a linear combination of these solutions.

Solutions \(y_1, y_2, \cdots, y_n\) are linearly independent for any ODE if their solution vectors \(\vb Y_1, \vb Y_2, \cdots, \vb Y_n\) are linearly independent.
A set of \(n\) linearly independent solution vectors forms a basis for the solution space of an \(n\)th order ODE.\@

\subsection{Initial conditions}
Consider initial conditions for a second order homogeneous ODE.\@
\[
	y(0) = a,\quad y'(0) = b
\]
If the general solution is
\[
	y(x) = Ay_1(x) + By_2(x)
\]
then we have the following linear system of equations
\begin{align*}
	Ay_1(0) + By_2(0)   & = a \\
	Ay_1'(0) + By_2'(0) & = b
\end{align*}
which is a system of two equations for two unknowns.
Or alternatively,
\[
	\underbrace{\begin{pmatrix}
			y_1(0)  & y_2(0)  \\
			y_1'(0) & y_2'(0)
		\end{pmatrix}}_{\equiv M}
	\begin{pmatrix}
		A \\ B
	\end{pmatrix}
	=
	\begin{pmatrix}
		a \\b
	\end{pmatrix}
\]
Unique solutions for \(A\) and \(B\) exist if \(\det M \neq 0\).

\subsection{The fundamental matrix and the Wro\'nskian}
The fundamental matrix is a matrix formed by placing solution vector \(\vb Y_i\) in the \(i\)th column.
The Wro\'nskian, denoted \(W(x)\), is the determinant of the fundamental matrix.
\[
	W(x) \equiv \begin{vmatrix}
		\vdots  & \vdots  &        & \vdots  \\
		\vb Y_1 & \vb Y_2 & \cdots & \vb Y_n \\
		\vdots  & \vdots  &        & \vdots
	\end{vmatrix} = \begin{vmatrix}
		y_1         & y_2         & \cdots & y_n         \\
		y_1'        & y_2'        & \cdots & y_n'        \\
		\vdots      & \vdots      & \ddots & \vdots      \\
		y_1^{(n-1)} & y_2^{(n-1)} & \cdots & y_n^{(n-1)}
	\end{vmatrix}
\]
For a second order ODE:\@
\begin{equation}\label{wronskian2}
	W(x) = \begin{vmatrix}
		y_1 & y_2 \\ y_1' & y_2'
	\end{vmatrix}
	= y_1y_2' - y_2y_1'
\end{equation}
The solution vectors are linearly independent if \(W(x) \neq 0\).
This is a convenient test for the linear independence of two solution vectors.
In our example above, we had
\[
	W(x) = \begin{vmatrix}
		\cos 2x   & \sin 2x  \\
		-2\sin 2x & 2\cos 2x
	\end{vmatrix}
	= 2\cos^2 2x + 2\sin^2 2x = 2 \neq 0
\]
So the solution vectors are linearly independent for all \(x\).

\subsection{Using the Wro\'nskian to find the original ODE}
If \(\vb Y_1\) and \(\vb Y_2\) are linearly dependent, then \(W(x) = 0\).
Suppose that a third solution \(y(x)\) is a linear combination of \(y_1(x)\) and \(y_2(x)\).
Then the solution vectors \(\vb Y, \vb Y_1, \vb Y_2\) are a linearly dependent set.
Hence
\[
	\begin{vmatrix}
		y   & y_1   & y_2   \\
		y'  & y_1'  & y_2'  \\
		y'' & y_1'' & y_2''
	\end{vmatrix}
	= 0
\]
For \(y_1 = \cos 2x\) and \(y_2 = \sin 2x\), we can deduce the original differential equation that produced these solutions by solving for \(y\).
\begin{align*}
	\begin{vmatrix}
		y   & \cos 2x   & \sin 2x    \\
		y'  & -2\sin 2x & 2 \cos 2x  \\
		y'' & -4\cos 2x & -4 \sin 2x
	\end{vmatrix}
	                                             & = 0 \\
	\implies y(8\sin^2 2x + 8\cos^2 2x)          &     \\
	- y'(-4 \cos 2x \sin 2x + 4 \cos 2x \sin 2x) &     \\
	+ y''(2\cos^2 2x + 2\sin^2 2x)               & = 0 \\
	\implies y'' + 4y                            & = 0
\end{align*}
Note that if \(W(x) = 0\), this does not necessarily imply linear dependence.

\subsection{Abel's theorem}
Consider a second order homogeneous ODE:\@
\[
	y'' + p(x)y' + q(x)y = 0
\]
\begin{theorem}[Abel's Theorem]
	If \(p(x)\) and \(q(x)\) are continuous on an interval \(I\), then the Wro\'nskian \(W(x)\) is either zero or nonzero for all \(x \in I\).
\end{theorem}
\begin{proof}
	Let \(y_1, y_2\) be solutions to the equation.
	Then
	\begin{align}
		\label{abelproof1} y_2(y_1'' + p(x)y_1' + q(x)y_1) & = 0 \\
		\label{abelproof2} y_1(y_2'' + p(x)y_2' + q(x)y_2) & = 0
	\end{align}
	Now, calculating \eqref{abelproof2} \(-\) \eqref{abelproof1}, we get
	\begin{equation}\label{abelproof3}
		(y_1y_2'' - y_2y_1'') + p(x)(y_1y_2' - y_2y_1') = 0
	\end{equation}
	As we are solving a second order equation, \(W(x) = y_1y_2' - y_2y_1'\) and therefore
	\[
		\frac{\dd{W}}{\dd{x}} = y_1y_2'' + y_1'y_2' - y_2'y_1' - y_2y_1'' = y_1y_2'' - y_2y_1''
	\]
	Note that these are the coefficients in \eqref{abelproof3}.
	We have therefore
	\begin{equation}\label{abelproof4}
		W' + pW = 0
	\end{equation}
	Then by separating variables:
	\begin{align*}
		\frac{\dd{W}}{W}              & = -p(x)\dd{x}                         \\
		\int_{x_0}^x \frac{\dd{W}}{W} & = -\int_{x_0}^x p(u)\dd{u}            \\
		W(x)                          & = W(x_0)e^{-\int_{x_0}^x p(u) \dd{u}}
	\end{align*}
	This last equation is known as Abel's Identity, and is very important.
	Since \(p(x)\) is continuous on \(I\) with \(x \in I\), it is bounded and therefore integrable.
	Therefore \(e^{-\int_{x_0}^x p(u) \dd{u}} \neq 0\).
	It follows that if \(W(x_0) = 0\) then \(W(x) = 0\) for all \(x\).
	Likewise, if \(W(x_0) \neq 0\), then \(W(x) \neq 0\) for all \(x\) (on the interval).
\end{proof}
\begin{corollary}
	If \(p(x) = 0\), then \(W = W_0\) which is a constant.
\end{corollary}
Note that we can use this to find \(W(x)\) without actually solving the differential equation itself.
For example, Bessel's Equation
\[
	x^2y'' + xy' + (x^2 - n^2)y = 0
\]
has no closed form solutions, but the Wro\'nskian can be calculated be rewriting it as
\[
	y'' + \frac{1}{x}y' + \frac{x^2-n^2}{x^2}y = 0
\]
and by Abel's Identity,
\begin{align*}
	W(x) & = W_0 e^{-\int_{x_0}^x \frac{1}{u} \dd{u}} \\
	     & = W_0 e^{-\ln x}                           \\
	     & = \frac{W_0}{x}
\end{align*}

\subsection{Using Abel's identity}
We can find a second solution \(y_2\) given a solution \(y_1\) using a reduction of order method, but we can also use Abel's Identity.
\[
	y_1y_2' - y_2y_1' = W_0 e^{-\int_{x_0}^x p(u) \dd{u}}
\]
This is a first order ODE for \(y_2\) which we can now solve:
\[
	\frac{y_1y_2' - y_2y_1'}{y_1^2} =  \frac{W_0}{y_1^2} e^{-\int_{x_0}^x p(u) \dd{u}}
\]
The left hand side is exactly the quotient rule, giving
\[
	\dv{x}\frac{y_2}{y_1} = \frac{W_0}{y_1^2} e^{-\int_{x_0}^x p(u) \dd{u}}
\]
which can be solved to give \(y_2\) as a function of \(y_1\) and \(W\).

\subsection{Abel's theorem in higher dimensions}
Any linear \(n\)th order ODE can be written
\[
	\vb Y' + A(x) \vb Y = 0
\]
where \(A\) is a matrix; this converts an \(n\)th order ODE into a system of \(n\) first order ODEs.
This will be discussed later in the course.
It can be shown that this generalisation of Abel's Identity
\[
	W' + \tr(A)W = 0
\]
holds, and hence
\[
	W' = W_0e^{-\int_{x_0}^x \tr(A) \dd{u}}
\]
and Abel's theorem holds.
This is shown on example sheet 3, Question 7.

\subsection{Equidimensional equations}
An ODE is equidimensional if the differential operator is unaffected by a multiplicative scaling.
For example, rescaling
\[
	x \mapsto X = \alpha x
\]
where \(\alpha \in \mathbb R\).
The general form for a second order equidimensional equation is
\begin{equation}\label{equidimensional1}
	ax^2 y'' + bxy' + cy = f(x)
\end{equation}
where \(a, b, c\) are constant.
Note, \(\frac{\dd}{\dd{X}} = \frac{1}{\alpha}\frac{\dd}{\dd{x}}\), and \(\frac{\dd^2}{\dd{X}^2} = \frac{1}{\alpha^2}\frac{\dd}{\dd{x}^2}\), so plugging this into \eqref{equidimensional1} gives
\[
	aX^2\frac{\dd^2 y}{\dd{X}^2} + bX\frac{\dd{y}}{\dd{X}} + cy = f\left(\frac{X}{\alpha}\right)
\]
The left hand side was unaffected by this rescaling, so the equation is equidimensional.

There are two main methods for solving equidimensional equations.
\begin{enumerate}
	\item Note that \(y = x^k\) is an eigenfunction of the differential operator \(x\frac{\dd}{\dd{x}}\).
	      Inspired by this, to solve \eqref{equidimensional1} we will look for solutions of the form \(y=x^k\), so we have
	      \[
		      ak(k-1) + bk + c = 0
	      \]
	      We can simply solve this quadratic for two roots \(k_1\) and \(k_2\).
	      If \(k_1 \neq k_2\), then the complementary function is
	      \[
		      y_c = Ax^{k_1} + Bx^{k_2}
	      \]
	\item If \(k_1 = k_2\), then the substitution \(z = \ln x\) turns \eqref{equidimensional1} into an equation with constant coefficients.
	      \[
		      a \frac{\dd^2 y}{\dd{z}^2} + (b-a)\frac{\dd{y}}{\dd{z}} + cy = f(e^z)
	      \]
	      (TODO verify this).
	      Because this has constant coefficients, our complementary functions will be of the form \(y = e^{\lambda z}\), which can be solved as usual.
	      \[
		      y_c = Ae^{\lambda_1 z} + Be^{\lambda_2 z} = Ax^{\lambda_1} + Bx^{\lambda_2}
	      \]
	      which is the same form as above.
	      In this form, it is easier to see that if the two solutions \(\lambda_1\), \(\lambda_2\) are the same, then
	      \[
		      y_c = Ae^{\lambda_1 z} + Bze^{\lambda_1 z} = Ax^{k_1} + Bx^{k_1}\ln x
	      \]
\end{enumerate}
