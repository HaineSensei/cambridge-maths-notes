\subsection{Definition of integration}
We use a Riemann sum to approximate the area under a sufficiently well-behaved function \(f(x)\) on the real numbers.
\begin{equation}\label{riemannsum}
	\sum_{n=0}^{N-1} f(x_n) \Delta x
\end{equation}
where \(\Delta x = (b-a)/N\), and \(x_n = a + n\Delta x\).
How close is \eqref{riemannsum} to the area under \(f(x)\) for large \(N\)?
Consider a specific rectangle in the Riemann sum by fixing \(n\).
The area under the curve in the \(n\)th rectangle and the area of the rectangle itself differ by a value we denote here as \(\epsilon\).
By computing \(\epsilon\)'s order of magnitude, we can show how much the total error deviates by.

\begin{theorem}[Mean Value Theorem]
	For a continuous function \(f(x)\),
	\begin{equation}\label{meanvaluetheorem}
		\int_{x_n}^{x_{n+1}} f(x) \dd{x} = f(x_c) \cdot (x_{n+1} - x_n)
	\end{equation}
	for some \(x_c\in (x_n, n_{n+1})\).
\end{theorem}
\noindent We use the Taylor Series of \(f(x)\) at \(x_n\) to compute a value for \(x_c\).
\[
	f(x_c) = f(x_n) + O(x_c - x_n)
\]
as \(x_c - x_n \to 0\).
Since \(x_n < x_c < x_{n+1}\), which implies \(\abs{x_{n+1} - x_n} > \abs{x_c - x_n}\), we can make the statement that
\[
	f(x_c) = f(x_n) + O(x_{n+1} - x_n)
\]
as \(x_{n+1} - x_n \to 0\).
Thus, by \eqref{meanvaluetheorem}
\[
	\int_{x_n}^{x_{n+1}} f(x) \dd{x} = \left[ f(x_n) + O(x_{n+1} - x_n) \right] (x_{n+1} - x_n)
\]
By defining \(\Delta x = x_{n+1} - x_n\), we have
\begin{equation}
	\int_{x_n}^{x_{n+1}} f(x) \dd{x} = \Delta x f(x_n) + O(\Delta x ^ 2)
\end{equation}
By rearranging, we can compute \(\epsilon\):
\[
	\epsilon = \abs{\int_{x_n}^{x_{n+1}} f(x) \dd{x} - \Delta x f(x_n)} = O(\Delta x ^ 2)
\]
Therefore it follows that
\[
	\int_{a}^{b} f(x) \dd{x} = \lim_{\Delta x \to 0} \left[ \left( \sum_{n=0}^{N-1} f(x_n) \Delta x \right) + O(N\Delta x^2) \right]
\]
Note that \(O(N\Delta x^2) = O((\frac{b-a}{N})^2 \cdot N) = O(1/N)\), so
\[
	\int_{a}^{b} f(x) \dd{x} = \lim_{N \to \infty} \left[ \left( \sum_{n=0}^{N-1} f(x_n) \Delta x \right) + O(1/N) \right]
\]
Which gives our final result of
\begin{equation}\label{definiteintegral}
	\int_{a}^{b} f(x) \dd{x} = \lim_{N \to \infty} \sum_{n=0}^{N-1} f(x_n) \Delta x`
\end{equation}

\subsection{Fundamental theorem of calculus}
Let \(F(x) = \int_{a}^{x} f(t) \dd{t}\).
From the definition of the derivative, we have
\begin{align*}
	\frac{\dd{F}}{\dd{x}} & = \lim_{h \to 0} \frac{1}{h} \left[ F(x+h) - F(x) \right]                                         \\
	                      & = \lim_{h \to 0} \frac{1}{h} \left[ \int_{a}^{x+h} f(t) \dd{t} - \int_{a}^{x} f(t) \dd{t} \right] \\
	                      & = \lim_{h \to 0} \frac{1}{h} \int_{x}^{x+h} f(t) \dd{t}                                           \\
	\intertext{Using \eqref{definiteintegral}:}
	                      & = \lim_{h \to 0} \frac{1}{h} \left[ hf(x) + O(h^2) \right]                                        \\
	                      & = \lim_{h \to 0} \left[ f(x) + O(h) \right]                                                       \\
	                      & = f(x)
\end{align*}
Therefore:
\begin{equation}\label{ftc}
	\frac{\ddempty}{\dd{x}} \left[ \int_{a}^{x} f(t) \dd{t} \right] = f(x)
\end{equation}

\subsection{Integration techniques}
Three particularly important methods of integration are:
\begin{itemize}
	\item \(u\)-substitution,
	\item trigonometric substitutions, and
	\item integration by parts.
\end{itemize}
Of particular note is the trigonometric substitution method, since it can be difficult to work out exactly which substitution will yield the result.
A table is provided below.

\medskip\noindent\begin{tabular}{c c c}
	Identity                              & Term in Integrand  & Substitution        \\\midrule
	\(\cos^2 \theta + \sin^2 \theta = 1\) & \(\sqrt{1 - x^2}\) & \(x = \sin \theta\) \\
	\(1 + \tan^2 \theta = \sec^2 \theta\) & \(1 + x^2\)        & \(x = \tan \theta\) \\
	\(\cosh^2 u - \sinh^2 u = 1\)         & \(\sqrt{x^2 - 1}\) & \(x = \cosh u\)     \\
	\(\cosh^2 u - \sinh^2 u = 1\)         & \(\sqrt{x^2 + 1}\) & \(x = \sinh u\)     \\
	\(1 - \tanh^2 u = \sech^2 u\)         & \(1 - x^2\)        & \(x = \tanh u\)
\end{tabular}
