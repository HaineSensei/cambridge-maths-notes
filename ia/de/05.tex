\subsection{Change of Variables}
We can transform derivatives into different coordinate systems to make problems easier to solve.
For example, let \(f(x, y)\) be some function with a Cartesian coordinate input.
We can rewrite it in terms of polar coordinates \((r, \theta)\).
First, rewrite \(f\) as:
\[
	f(x(r, \theta), y(r, \theta))
\]
then we can write the derivatives.
\[
	\frac{\partial f}{\partial r} = \frac{\partial f}{\partial x}\frac{\partial x}{\partial r} + \frac{\partial f}{\partial y}\frac{\partial y}{\partial r}
\]
We can do similar evaluations for \(\frac{\partial f}{\partial \theta}\), for example.

\subsection{Implicit Differentiation}
Consider some surface defined by \(f(x, y, z) = c\).
Then \(f\) implicitly defines functions such as \(z(x, y)\) (provided the function is well-behaved).
We can find, for example, \(\eval{\frac{\partial z}{\partial x}}_y\) by using the multivariate chain rule in three dimensions.

\[
	\eval{\frac{\partial f}{\partial x}}_y =
	\eval{\frac{\partial f}{\partial x}}_{yz} \underbrace{\eval{\frac{\partial x}{\partial x}}_{y}}_{\mathclap{=1}} +
	\eval{\frac{\partial f}{\partial y}}_{xz} \underbrace{\eval{\frac{\partial y}{\partial x}}_{y}}_{\mathclap{=0}} +
	\eval{\frac{\partial f}{\partial z}}_{xy} \eval{\frac{\partial z}{\partial x}}_{y}
\]
Note that the \(\frac{\partial y}{\partial x}\) term is zero because we hold \(y\) to be fixed.
Simplifying, we get
\[
	\eval{\frac{\partial f}{\partial x}}_y =
	\eval{\frac{\partial f}{\partial x}}_{yz} +
	\eval{\frac{\partial f}{\partial z}}_{xy} \eval{\frac{\partial z}{\partial x}}_{y}
\]
The left hand side is zero because on the surface \(z(x, y)\), \(f\) is always equivalent to \(c\) so there is never any \(\delta f\).
The \(\eval{\frac{\partial f}{\partial x}}_{yz}\) term, however, is not zero in general because we are not going across the \(z(x, y)\) surface --- just parallel to the \(x\) axis, because we fixed both \(y\) and \(z\).
Hence,
\[
	\eval{\frac{\partial z}{\partial x}}_y = \frac{-\eval{\frac{\partial f}{\partial x}}_{yz}}{\eval{\frac{\partial f}{\partial z}}_{xy}}
\]

\subsection{Reciprocal Rule}
The reciprocal rule for derivatives applies also to partial derivatives so long as the same variables are held fixed.
For example, given the function \(f(x(r, \theta), y(r, \theta))\), we have
\[
	\eval{\frac{\partial r}{\partial x}}_y = \frac{1}{\eval{\frac{\partial x}{\partial r}}_y}
\]
But
\[
	\frac{\partial r}{\partial x} \neq \frac{1}{\frac{\partial x}{\partial r}}
\]
because the left hand side holds \(y\) constant and the right hand side holds \(\theta\) constant.

\subsection{Differentiating an Integral with Respect to a Parameter}
Consider a family of function \(f(x; \alpha)\) where \(\alpha\) is some parameter.
We can say that \(\alpha\) parametrises \(f\).
An example of a parametrised function is the logarithm; \(f(x; \alpha) = \log_\alpha x\).
We define
\[
	I(\alpha) = \int_{a(\alpha)}^{b(\alpha)} f(x; \alpha) \ \dd{x}
\]
So, what is \(\frac{\dd{I}}{\dd \alpha}\)?
By definition, we have
\begin{align*}
	\frac{\dd{I}}{\dd \alpha} & = \lim_{\delta \alpha \to 0} \frac{I(\alpha + \delta \alpha) - I(\alpha)}{\delta \alpha}                                                                                                                                                                                                                                                  \\
	                          & = \lim_{\delta \alpha \to 0} \frac{1}{\delta\alpha} \left[ \int_{a(\alpha + \delta\alpha)}^{b(\alpha + \delta\alpha)} f(x; \alpha + \delta\alpha)\ \dd{x} - \int_{a(\alpha)}^{b(\alpha)} f(x; \alpha)\ \dd{x} \right]                                                                                                                     \\
	                          & = \lim_{\delta \alpha \to 0} \frac{1}{\delta\alpha} \left[ \int_{a(\alpha)}^{b(\alpha)} f(x; \alpha + \delta\alpha) - f(x; \alpha)\ \dd{x} - \int_{a(\alpha)}^{a(\alpha + \delta)} f(x; \alpha + \delta \alpha)\ \dd{x} + \int_{b(\alpha)}^{b(\alpha + \delta)} f(x; \alpha + \delta \alpha)\ \dd{x} \right]                              \\
	                          & = \int_{a(\alpha)}^{b(\alpha)} \lim_{\delta \alpha \to 0} \frac{f(x; \alpha + \delta\alpha) - f(x; \alpha)}{\delta\alpha}\ \dd{x} - f(a; \alpha) \lim_{\delta \alpha \to 0} \frac{a(\alpha + \delta\alpha) - a(\alpha)}{\delta\alpha} + f(b; \alpha) \lim_{\delta \alpha \to 0} \frac{b(\alpha + \delta\alpha) - b(\alpha)}{\delta\alpha} \\
\end{align*}
Therefore:
\[
	\frac{\dd{I}}{\dd \alpha} = \frac{\dd}{\dd \alpha} \int_{a(\alpha)}^{b(\alpha)} f(x; \alpha) \ \dd{x} = \int_{a(\alpha)}^{b(\alpha)} \frac{\partial f}{\partial \alpha} \ \dd{x} + f(b; \alpha) \frac{\dd{b}}{\dd \alpha} - f(a; \alpha) \frac{\dd{a}}{\dd \alpha}
\]
