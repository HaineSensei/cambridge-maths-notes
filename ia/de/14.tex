We want to find methods for finding the particular integral of forced second order ODEs.
\subsection{Guesswork}
Here, \(f(x)\) is the forcing term.

\begin{tabular}{c c}
	Form of \(f(x)\)                        & Form of \(y_p(x)\)                          \\\midrule
	\(e^{mx}\)                              & \(Ae^{mx}\)                                 \\
	\(\sin(kx)\) or \(\cos(kx)\)            & \(A\sin(kx) + B\cos(kx)\)                   \\
	\(x^n\) or an \(n\)th degree polynomial & \(a_n x^n + a_{n-1}x^{n-1} + \cdots + a_0\)
\end{tabular}

In general:
\begin{enumerate}
	\item Insert our guess into the equation;
	\item Equate coefficients of functions;
	\item Solve for unknown coefficients.
\end{enumerate}

\subsection{Variation of Parameters}
This is a method for finding the particular integral \(y_p\) given complementary functions \(y_1\), \(y_2\), which are assumed to be linearly independent, with solution vectors
\[
	\vb Y_1 = \begin{pmatrix}
		y_1 \\ y_1'
	\end{pmatrix};\quad \vb Y_2 = \begin{pmatrix}
		y_2 \\ y_2'
	\end{pmatrix}
\]
Suppose that the solution vector for \(y_p\) satisfies
\begin{equation}\label{varparam1}
	\vb Y_p = \begin{pmatrix}
		y_p \\ y_p'
	\end{pmatrix} = u(x)\vb Y_1 + v(x)\vb Y_2
\end{equation}
This is not a linear combination of \(\vb Y_1\) and \(\vb Y_2\), but we can treat it as a linear combination at a fixed \(x\) point (just as a way to visualise it).
We want to find equations for \(u(x)\) and \(v(x)\).
By comparing components of the vectors on the left and right, we have
\begin{align*}
	y_p  & = uy_1 + vy_2 \tag{a}   \\
	y_p' & = uy_1' + vy_2' \tag{b} \\
\end{align*}
So therefore,
\[
	\frac{\dd}{\dd{x}}(\mathrm a) \implies y_p' = u'y_1 + uy_1' + v'y_2 + vy_2' \tag{c}
\]
\[
	(\mathrm c) - (\mathrm b) \implies u'y_1 + v'y_2 = 0 \tag{d}
\]
Now:
\[
	\frac{\dd}{\dd{x}}(\mathrm b) \implies y_p'' = uy_1'' + u'y_1' + v'y_2' + vy_2'' \tag{e}
\]
If \(y_p'' + p(x)y_p' + q(x)y_p = f(x)\), then
\[
	(\mathrm e) + p(\mathrm b) + q(\mathrm a) = f(x)
\]
But also, \(y_1\) and \(y_2\) satisfy the differential equation
\[
	y_c'' + py_c' + qy_c = 0\quad y_c \in \{ y_1, y_2 \}
\]
After we substitute in and cancel a lot of terms, we have
\[
	u'y_1' + v'y_2' = f(x) \tag{f}
\]
Combining (d) and (f), we can deduce \(u\) and \(v\).
\[
	\underbrace{\begin{pmatrix}
			y_1 & y_2 \\ y_1' & y_2'
		\end{pmatrix}}_{\mathclap{\text{fundamental matrix}}} \begin{pmatrix}
		u' \\ v'
	\end{pmatrix} = \begin{pmatrix}
		0 \\ f
	\end{pmatrix}
\]
So as long as \(y_1\) and \(y_2\) are linearly independent, i.e.\ the Wronskian is non-zero, we can write an equation for \(u'\) and \(v'\).
\[
	\begin{pmatrix}
		u' \\ v'
	\end{pmatrix} = \frac{1}{W(x)}\begin{pmatrix}
		y_2' & -y_2 \\ y_1' & y_1
	\end{pmatrix} \begin{pmatrix}
		0 \\ f
	\end{pmatrix}
\]
or explicitly,
\begin{align*}
	u' & = \frac{-y_2 f}{W} \\
	v' & = \frac{y_1 f}{W}
\end{align*}
and therefore
\[
	y_p = y_2 \int^x \frac{y_1(t) f(t)}{W(t)}\dd{t} - y_1 \int^x \frac{y_2(t) f(t)}{W(t)}\dd{t}
\]

\subsection{Forced Oscillating Systems: Transients and Damping}
Many physical systems have a restoring force and damping (e.g.
friction).
For example, the suspension of a car could be modelled with \(y(t)\), where \(y\) is the height of the wheel, given by
\[
	M\ddot y = F(t) - \underbrace{ky}_{\text{spring}} - \underbrace{L\dot y}_{\text{damper}}
\]
In standard form, we have
\[
	\ddot y + \frac{L}{M} \dot y + \frac{k}{m} y = \frac{F(t)}{M}
\]
Let \(\tau = \sqrt{\frac{k}{M}} t\).
Then we can rewrite this equation using a single parameter:
\[
	y'' + 2Ky' + y = f(\tau)
\]
where \(y' = \frac{\dd{y}}{\dd \tau}\), \(K = \frac{L}{2\sqrt{km}}\), \(f = \frac{F}{k}\).
Our unforced system is described here by one parameter \(K\).

In the case of the unforced response (also known as the free or natural response), we have \(f=0\), so
\[
	y'' + 2Ky' + y = 0
\]
Solving by the characteristic equation, we see
\[
	\lambda = -K \pm \sqrt{K^2 - 1}
\]
There are a number of cases here.
\begin{enumerate}
	\item (\(K < 1\)) This produces a decaying oscillation, known as `underdamped'.
	      \(\lambda_1, \lambda_2\) are both complex, and therefore
	      \[
		      y = e^{-K\tau}\left[A\sin(\sqrt{1-K^2}\tau) + B\cos(\sqrt{1-K^2}\tau)\right]
	      \]
	      The period is \(\frac{2\pi}{\sqrt{1-K^2}}\).
	      As \(K\) tends to 1, the period tends to \(\infty\).
	\item (\(K = 1\)) This is the degenerate case, \(\lambda_1 = \lambda_2 = -K\).
	      We can use detuning to deduce
	      \[
		      y = e^{-K\tau} (A + B\tau)
	      \]
	\item (\(K > 1\)) Here, we have two negative real roots \(\lambda_1, \lambda_2\); this situation is known as `overdamped'.
	      \[
		      y = Ae^{\lambda_1 \tau} + Be^{\lambda_2 \tau}
	      \]
\end{enumerate}
Note that the unforced response decays in all cases.
