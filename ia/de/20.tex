\subsection{Gradient Vector}
Consider a function \(f(x, y)\), and some small displacement \(\dd \vb s\). We want to find the rate of change of \(f\) in this direction. Recall that the multivariate chain rule tells us that a change in \(f\), given a change in \(x\) and \(y\), is given by
\begin{align*}
	\dd{f} & = \frac{\partial f}{\partial x} \dd{x} + \frac{\partial f}{\partial y}\dd{y}                         \\
	       & = (\dd{x}, \dd{y}) \cdot \left( \frac{\partial f}{\partial x}, \frac{\partial f}{\partial y} \right) \\
	       & = \dd \vb s \cdot \grad f
\end{align*}
where \(\dd \vb s = (\dd{x}, \dd{y})\); \(\grad f = \left( \frac{\partial f}{\partial x}, \frac{\partial f}{\partial y} \right)\). We call \(\grad f\) the `gradient vector', in this case in Cartesian coordinates. If we let \(\dd \vb s = \dd{s} \hat{\vb s}\) where \(\abs{\hat{\vb s}} = 1\), then we can write
\[ \dd{f} = \dd{s} (\hat{\vb s}\cdot \grad f) \]

\subsection{Directional Derivative}
We define the directional derivative by
\[ \frac{\dd{f}}{\dd{s}} = \hat{\vb s} \cdot \grad f \]
This is the rate of change of \(f\) in the direction given by \(\hat{\vb s}\).

\subsection{Properties of the Gradient Vector}
\begin{enumerate}
	\item The magnitude of the gradient vector \(\grad f\) is the maximum rate of change of \(f(x, y)\).
	      \[ \abs{\grad f} = \max\limits_{\forall \theta} \left( \frac{\dd{f}}{\dd{s}} \right) \]
	\item The direction of \(\grad f\) is the direction in which \(f\) increases most rapidly.
	      \[ \abs{\frac{\dd{f}}{\dd{s}}} = \abs{\grad f}\cos\theta \]
	      where \(\theta\) is the angle between \(\grad f\) and \(\hat{\vb s}\), which follows from the definition of the directional derivative.
	\item If \(\dd \vb s\) (and \(\hat{\vb s}\)) are parallel to contours of \(f\), then
	      \[ \frac{\dd{f}}{\dd{s}} = \hat{\vb s} \cdot \grad f = 0 \]
	      Hence the gradient vector is perpendicular to contours of \(f\), and \(\abs{\grad f}\) is the slope in the `uphill' direction.
\end{enumerate}

\subsection{Stationary Points}
In general, there is always at least one direction in which the directional derivative is zero, since we can just choose a direction perpendicular to the gradient vector, or equivalently parallel to contours of \(f\). At stationary points, \(\frac{\dd{f}}{\dd{s}} = 0\) for all directions, so \(\grad f = \vb 0\). Stationary points may have multiple types:
\begin{itemize}
	\item Minimum points, where the function is a minimum point in both directions;
	\item Maximum points, where the function is a maximum point in both directions; and
	\item Saddle points, where the function is a minimum point in one direction but a maximum point in another direction.
\end{itemize}
Note:
\begin{itemize}
	\item Near minima and maxima, the contours of \(f\) are elliptical.
	\item Near a saddle, the contours of \(f\) are hyperbolic.
	\item Contours of \(f\) can only cross at saddle points.
\end{itemize}

\subsection{Taylor Series for Multivariate Functions}
Let us expand a function \(f(x, y)\) around a point \(\vb s_0\), and evaluate it at some point \(\vb s_0 + \delta \vb s\), where \(\delta \vb s = \delta s \hat{\vb s}\). The Taylor series expansion in the direction of \(\hat{\vb s}\) is
\[ f(s_0 + \delta s) = f(s_0) + \delta s \eval{\frac{\dd{f}}{\dd{s}} }_{s_0} + \frac{1}{2} (\delta s)^2\eval{\frac{\dd^2 f}{\dd{s}^2}}_{s_0} + \dots \]
Further, by the definition of the directional derivative,
\[ \frac{\dd}{\dd{s}} = \hat{\vb s}\cdot \grad \]
Hence
\[ \delta s \frac{\dd}{\dd{s}} = \delta \vb s \cdot \grad \]
Now we can rewrite this Taylor series as follows:
\[ f(s_0 + \delta_s) = f(s_0) + (\delta s)(\hat{\vb s} \cdot \grad) \eval{f}_{s_0} + \frac{1}{2}(\delta s)^2(\hat{\vb s} \cdot \grad)(\hat{\vb s} \cdot \grad)\eval{f}_{s_0} + \dots \]
\[ f(s_0 + \delta_s) = f(s_0) + \underbrace{(\delta \vb s \cdot \grad) \eval{f}_{s_0}}_{(1)} + \underbrace{\frac{1}{2}(\delta \vb s \cdot \grad)(\delta \vb s \cdot \grad)\eval{f}_{s_0}}_{(2)} + \dots \]
Expressing this in Cartesian coordinates:
\[ \vb s_0 = (x_0, y_0);\quad \delta \vb s = (\delta x, \delta y);\quad x=x_0+\delta x;\quad y=y_0+\delta_y \]
Therefore,
\[ (1) = \delta x \frac{\partial f}{\partial x}(x_0, y_0) + \delta y \frac{\partial f}{\partial y}(x_0, y_0) \]
\begin{align*}
	(2) & = \eval{\frac{1}{2}\left( \delta x \frac{\partial}{\partial x} + \delta y + \frac{\partial}{\partial y} \right)\left( \delta x \frac{\partial}{\partial x} + \delta y + \frac{\partial}{\partial y} \right)f}_{x_0, y_0} \\
	    & = \eval{\frac{1}{2}\left( \delta x^2 f_{xx} + \delta x \delta y f_{yx} + \delta y \delta x f_{xy} + \delta y^2 f_{yy} \right)}_{x_0, y_0}                                                                                \\
	    & = \frac{1}{2}\begin{pmatrix}
		\delta x & \delta y
	\end{pmatrix} \eval{\begin{pmatrix}
			f_{xx} & f_{xy} \\
			f_{yx} & f_{yy}
		\end{pmatrix}}_{x_0, y_0} \begin{pmatrix}
		\delta x \\ \delta y
	\end{pmatrix}
\end{align*}

The matrix
\[ H = \begin{pmatrix}
		f_{xx} & f_{xy} \\
		f_{yx} & f_{yy}
	\end{pmatrix} = \grad(\grad f) \]
as used in the second derivative above, is called the Hessian matrix.

Putting this together, in 2D Cartesian Coordinates, we have
\begin{align*}
	f(x, y) & = f(x_0, y_0) + (x-x_0)\eval{f_x}_{x_0, y_0} + (y-y_0)\eval{f_y}_{x_0, y_0} \\&+ \frac{1}{2}\left[ (x-x_0)^2\eval{f_{xx}}_{x_0, y_0} + (y-y_0)^2\eval{f_{yy}}_{x_0, y_0} + 2(x-x_0)(y-y_0)\eval{f_{xy}}_{x_0, y_0} \right] + \dots
\end{align*}
And in the general coordinate-independent form:
\[ f(\vb x) = f(\vb x_0) + \delta \vb x \cdot \grad f(\vb x_0) + \frac{1}{2}\delta \vb x \cdot \eval{[\grad (\grad f)]}_{x_0} \cdot \delta \vb x^\transpose + \dots \]
