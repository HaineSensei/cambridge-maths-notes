\subsection{Definition of Limit}
\begin{definition}
	We say that the sequence \(a_n \to a\) as \(n \to \infty\) if given \(\varepsilon > 0\), \(\exists N\) such that \(\abs{a_n - a} < \varepsilon\) for all \(n \geq N\).
	Note that this \(N\) is actually a function of \(\varepsilon\); we may need to choose a very large \(N\) if the \(\varepsilon\) provided is very small, for instance.
\end{definition}
\begin{definition}
	An increasing sequence is a sequence for which \(a_n \leq a_{n+1}\), and a decreasing sequence is a sequence for which \(a_n \geq a_{n+1}\).
	Such increasing and decreasing sequences are called monotone.
	A strictly increasing sequence or a strictly decreasing sequence simply strengthens the inequalities to not include the equality case.
\end{definition}

\subsection{Fundamental Axiom of the Real Numbers}
If we have some increasing sequence \(a_n \in \mathbb R\), where \(\exists A \in \mathbb R\) such that \(\forall n \geq 1\), \(a_n \leq A\), then \(\exists a \in \mathbb R\) such that \(a_n \to a\) as \(n \to \infty\).
This is also known as the `least upper bound' axiom or property.
This axiom applies equivalently to decreasing sequences of real numbers bounded below.
We can also rephrase the axiom to state that every non-empty set of real numbers that is bounded above has a supremum.
\begin{definition}
	We say that the supremum \(\sup S\) of a non-empty, bounded above set \(S\) is \(K\) if
	\begin{enumerate}[(i)]
		\item \(x \leq K\) for all \(x \in S\)
		\item given \(\varepsilon > 0\), \(\exists x \in S\) such that \(x > K - \varepsilon\)
	\end{enumerate}
\end{definition}
Note that the supremum (and hence the infimum) is unique.

\subsection{Properties of Limits}
\begin{lemma}
	The following properties about real sequences hold.
	\begin{enumerate}[(i)]
		\item The limit is unique.
		      That is, if \(a_n \to a\) and \(a_n \to b\), then \(a = b\).
		\item If \(a_n \to a\) as \(n \to \infty\) and \(n_1 < n_2 < \dots\), then \(a_{n_j} \to a\) as \(j \to \infty\).
		      In other words, subsequences converge to the same limit.
		\item If \(a_n = c\) for all \(n\), then \(a_n \to c\) as \(n \to \infty\).
		\item If \(a_n \to a\) and \(b_n \to b\), then \(a_n + b_n \to a + b\).
		\item If \(a_n \to a\) and \(b_n \to b\), then \(a_nb_n \to ab\).
		\item If \(a_n \to a\), \(a_n \neq 0\) for all \(n\), and \(a \neq 0\), then \(\frac{1}{a_n} \to \frac{1}{a}\).
		\item If \(a_n \to a\), and \(a_n \leq A\) for all \(n\), then \(a \leq A\).
	\end{enumerate}
\end{lemma}
\begin{proof}
	We prove the some of these statements here.
	\begin{enumerate}[(i)]
		\item Given \(\varepsilon > 0\), \(\exists n_1\) such that \(\abs{a_n - a} < \varepsilon\) for all \(n \geq n_1\), and \(\exists n_2\) such that \(\abs{a_n - b} < \varepsilon\) for all \(n \geq n_2\).
		      So let \(N = \max(n_1, n_2)\), so both inequalities hold.
		      Then for all \(n \geq N\), using the triangle inequality, \(\abs{a - b} \leq \abs{a_n - a} + \abs{a_n - b} < 2\varepsilon\).
		      So \(a=b\).
		\item Given \(\varepsilon > 0\), \(\exists N\) such that \(\abs{a_n - a} < \varepsilon\) for all \(n \geq N\).
		      Since \(n_j \geq j\) (by induction), \(\abs{a_{n_j} - a} < \varepsilon\) for all \(j \geq N\).
		      \setcounter{enumi}{4}
		\item \(\abs{a_nb_n - ab} \leq \abs{a_nb_n - a_nb} + \abs{a_nb - ab} = \abs{a_n}\abs{b_n - b} + \abs{b}\abs{a_n - a}\).

		      If \(a_n \to a\), then given \(\varepsilon > 0\), \(\exists N_1\) such that \(\abs{a_n - a} < \varepsilon\) for all \(n \geq N_1\).
		      (\(\ast\))

		      If \(b_n \to b\), then given \(\varepsilon > 0\), \(\exists N_2\) such that \(\abs{b_n - b} < \varepsilon\) for all \(n \geq N_2\).

		      Using (\(\ast\)), if \(n \geq N_1(1)\) (i.e.\ \(\varepsilon = 1\)), \(\abs{a_n - a} < 1\), so \(\abs{a_n} \leq \abs{a} + 1\).

		      Therefore \(\abs{a_n b_n - ab} \leq \varepsilon(\abs{a} + 1 + \abs{b})\) for all \(n \geq N_3(\varepsilon) = \max\{ N_1(1), N_1(\varepsilon), N_2(\varepsilon) \}\).
	\end{enumerate}
\end{proof}

\subsection{Harmonic Series}
\begin{lemma}
	The sequence \(\frac{1}{n}\) tends to zero as \(n \to \infty\).
\end{lemma}
\begin{proof}
	We know that \(\frac{1}{n}\) is a decreasing sequence, and it is bounded below by zero.
	Hence it converges to a limit \(a\).
	We will prove now that \(a = 0\).
	\(\frac{1}{2n} = \frac{1}{2}\cdot \frac{1}{n}\), and by property (v) above, \(\frac{1}{2n}\) tends to \(\frac{1}{2}\cdot a\).
	But \(\frac{1}{2n}\) is a subsequence of \(\frac{1}{n}\), and so by property (ii) it converges to \(a\).
	So by property (i), \(\frac{1}{2} \cdot a = a\) hence \(a=0\).
\end{proof}

\subsection{Limits in the Complex Plane}
\begin{remark}
	The definition of the limit of a sequence makes perfect sense for \(a_n \in \mathbb C\).
\end{remark}
\begin{definition}
	\(a_n \to a\) if given \(\varepsilon > 0\), \(\exists N\) such that \(\forall n \geq N\), \(\abs{a_n - a} < \varepsilon\).
\end{definition}
From this definition, it is easy to check that properties (i)--(vi) hold for complex numbers.
However, property (vii) makes no sense in the world of the complex numbers since they do not have an ordering.
