\subsection{The Bolzano-Weierstrass Theorem}
\begin{theorem}
	If \(x_n\) is a sequence of real numbers, and there exists some \(k\) such that \(\abs{x_n} \leq k\) for all \(n\), then we can find \(n_1 < n_2 < n_3 < n_4 < \dots\) and \(x \in \mathbb R\) such that \(x_{n_j} \to x\) as \(j \to \infty\). In other words, any bounded sequence has a convergent subsequence.
\end{theorem}
\begin{remark}
	This theorem does not state anything about the uniqueness of such a subsequence; indeed, there could exist many subsequences that have possibly different limits. For example, \(x_n = (-1)^n\) gives \(x_{2n+1} \to -1\) and \(x_{2n} \to 1\).
\end{remark}
\begin{proof}
	Let \([a_1, b_1]\) be the range of the sequence, i.e.\ \([-k, k]\). Then let the midpoint \(c_1 = \frac{a_1 + b_1}{2}\). Consider the following alternatives:
	\begin{enumerate}
		\item \(x_n \in [a_1, c]\) for infinitely many values of \(n\).
		\item \(x_n \in [c, b_1]\) for infinitely many values of \(n\).
	\end{enumerate}
	Note that cases 1 and 2 could hold at the same time. If case 1 holds, we set \(a_2 = a_1\) and \(b_2 = c\). If case 1 fails, then case 2 must hold, so we can set \(a_2 = c\) and \(b_2 = b_1\). We have now constructed a subsequence whose range is half as large as the original sequence, and it contains infinitely many values of \(x_n\).

	We can proceed inductively to construct sequences \(a_n, b_n\) such that \(x_m \in [a_n, b_n]\) for infinitely many values of \(m\). This is known as a `bisection method'. By construction, \(a_{n-1} \leq a_n \leq b_n \leq b_{n-1}\). Since we are dividing by two each time,
	\[ b_n - a_n = \frac{1}{2}(b_{n-1} - a_{n-1}) \tag{\(\ast\)} \]
	Note that \(a_n\) is a bounded, increasing sequence; and \(b_n\) is a bounded, decreasing sequence. By the Fundamental Axiom of the Real Numbers, \(a_n\) and \(b_n\) converge to limits \(a \in [a_1, b_1]\) and \(b \in [a_1, b_1]\). Using \((\ast)\), \(b-a = \frac{b-a}{2} \implies b = a\).

	Since \(x_m \in [a_n, b_n]\) for infinitely many values of \(m\), having chosen \(n_j\) such that \(x_{n_j} \in [a_j, b_j]\), there is \(n_{j+1} > n_j\) such that \(x_{n_{j+1}} \in [a_{j+1}, b_{j+1}]\). Informally, this works because we have an unlimited supply of such \(x\) values. Hence
	\[ a_j \leq x_{n_j} \leq b_j \]
	So this \(x_{n_j} \to a\), so we have constructed a convergent subsequence.
\end{proof}

\subsection{Cauchy Sequences}
\begin{definition}
	A sequence \(a_n\) is called a Cauchy sequence if given \(\varepsilon > 0\) there exists \(N > 0\) such that \(\abs{a_n - a_m} < \varepsilon\) for all \(n, m \geq N\). Informally, the terms of the sequence grow ever closer together such that there are infinitely many consecutive terms within a small region.
\end{definition}
\begin{lemma}
	If a sequence converges, it is a Cauchy sequence.
\end{lemma}
\begin{proof}
	If \(a_n \to a\), given \(\varepsilon > 0\) then \(\exists N\) such that \(\forall n \geq N, \abs{a_n - a} < \varepsilon\). Then take \(m, n \geq N\), and we have
	\[ \abs{a_n - a_m} \leq \abs{a_n - a} + \abs{a_m - a} < 2\varepsilon \]
\end{proof}
\begin{theorem}
	Every Cauchy sequence converges.
\end{theorem}
\begin{proof}
	First, we note that if \(a_n\) is a Cauchy sequence then it is bounded. Let us take \(\varepsilon = 1\), so \(N = N(1)\) in the Cauchy property. Then
	\[ \abs{a_n - a_m} < 1 \]
	for all \(m, n \geq N(1)\). So by the triangle inequality,
	\[ \abs{a_m} \leq \abs{a_m - a_N} + \abs{a_N} < 1 + \abs{a_N} \]
	So the sequence after this point is bounded by \(1 + \abs{a_N}\). The remaining terms in the sequence are only finitely many, so we can compute the maximum of all of those terms along with \(1+\abs{a_N}\) to produce a bound \(k\) for all \(n\).

	By the Bolzano-Weierstrass Theorem, this sequence \(a_n\) has a convergent subsequence \(a_{n_j} \to a\). We want to prove that \(a_n \to a\). Given \(\varepsilon > 0\), there exists \(j_0\) such that \(\abs{a_{n_j} - a} < \varepsilon\) for all \(j \geq j_0\). Also, \(\exists N(\varepsilon)\) such that \(\abs{a_m - a_n} < \varepsilon\) for all \(m, n \geq N(\varepsilon)\). Combining these, we can take a \(j\) such that \(n_j \geq \max \{ N(\varepsilon), n_{j_0} \}\). Then, if \(n \geq N(\varepsilon)\), using the triangle inequality,
	\[ \abs{a_n - a} \leq \abs{a_n - a_{n_j}} + \abs{a_{n_j} - a} < 2\varepsilon \]
\end{proof}
\noindent Therefore, on \(\mathbb R\), a sequence is convergent if and only if it is a Cauchy sequence. This is sometimes referred to as the general principle of convergence, however this is a relatively old-fashioned name. This property is very useful, since we don't need to know what the limit actually is.

\subsection{Series}
Let \(a_n\) be a real or complex sequence. We say that \(\sum_{j=1}^\infty a_j\) converges to \(s\) if the sequence of partial sums \(s_N\) converges to \(s\) as \(N \to \infty\), i.e.
\[ s_N = \sum_{j=1}^N a_j \to s \]
If the sequence of partial sums does not converge, then we say that the series diverges. Note that any problem on series can be turned into a problem on sequences, by considering their partial sums.
\begin{lemma}
	\begin{enumerate}[(i)]
		\item If \(\sum_{j=1}^\infty a_j\) and \(\sum_{j=1}^\infty b_j\) converge, then so does \(\sum_{j=1}^\infty (\lambda a_j + \mu b_j)\), where \(\lambda, \mu \in \mathbb C\).
		\item Suppose \(\exists N\) such that \(a_j = b_j\) for all \(j \geq N\). Then either \(\sum_{j=1}^\infty a_j\) and \(\sum_{j=1}^\infty b_j\) both converge, or they both diverge. In other words, the initial terms do not matter for considering convergence (but the sum will change).
	\end{enumerate}
\end{lemma}
\begin{proof}
	\begin{enumerate}[(i)]
		\item We have
		      \begin{align*}
			      s_N            & = \sum_{j=1}^\infty (\lambda a_j + \mu b_j)                 \\
			                     & = \sum_{j=1}^\infty \lambda a_j + \sum_{j=1}^\infty \mu b_j \\
			                     & = \lambda c_N + \mu d_N                                     \\
			      \therefore s_N & \to \lambda c + \mu d
		      \end{align*}
		\item For any \(n \geq N\), we have
		      \begin{align*}
			      s_N & = \sum_{j=1}^n a_j = \sum_{j=1}^{N-1} a_j + \sum_{j=n}^N a_j \\
			      d_N & = \sum_{j=1}^n b_j = \sum_{j=1}^{N-1} b_j + \sum_{j=n}^N b_j \\
		      \end{align*}
		      Taking the difference, we get
		      \[ s_N - d_N = \sum_{j=1}^{N-1} a_j - \sum_{j=1}^{N-1} b_j \]
		      which is finite. So \(s_N\) converges if and only if \(d_N\) also converges.
	\end{enumerate}
\end{proof}
