\subsection{Bounds of a Continuous Function}
\begin{theorem}
	Let \(f \colon [a, b] \to \mathbb R\) be continuous.
	Then the function is bounded, i.e.\ there exists \(k \in \mathbb R\) such that \(\abs{f(x)} \leq k\) for every point \(x \in [a, b]\).
\end{theorem}
\begin{proof}
	Suppose that such a function \(f\) is not bounded.
	Then in particular, given any integer \(n \geq 1\), there exists \(x_n \in [a, b]\) such that \(\abs{f(x_n)} > n\).
	By the Bolzano-Weierstrass theorem, the sequence \(x_n\), which is bounded by \(a \leq x_n \leq b\), has a convergent subsequence \(x_{n_j} \to x\), such that \(x \in [a, b]\).
	Then by continuity of \(f\), \(f(x_{n_j}) \to f(x)\).
	But \(\abs{f(x_{n_j})} > n_j \to \infty\).
	This is a contradiction.
\end{proof}
\noindent We can actually improve this statement.
\begin{theorem}
	Suppose \(f \colon [a, b] \to \mathbb R\) is a continuous function.
	Then there exist \(x_1, x_2 \in [a, b]\) such that
	\[
		f(x_1) \leq f(x) \leq f(x_2)
	\]
	for all \(x \in [a, b]\).
	In other words, a continuous function on a closed bounded interval is bounded and attains its bounds.
\end{theorem}
\begin{proof}
	Let \(A = \{ f(x) \colon x \in [a, b] \}\) be the image of \([a, b]\) under \(f\).
	By the above theorem, \(A\) is bounded.
	It is also non-empty, hence it has a supremum \(M = \sup A\) (and analogously an infimum \(\inf A\), whose proof is almost identical).
	Then by the definition of the supremum, given an integer \(n \geq 1\) there exists \(x_n \in [a, b]\) such that \(M - \frac{1}{n} < f(x_n) \leq M\).
	By the Bolzano-Weierstrass theorem, there exists a convergent subsequence \(x_{n_j} \to x \in [a, b]\).
	Since \(f(x_{n_j}) \to M\), then by continuity, \(f(x) = M\).
\end{proof}
\noindent Here is an alternative proof of the same theorem.
\begin{proof}
	As before, let \(A\) be the image of \(f\), and \(M\) be the supremum of \(A\).
	Suppose there is no \(x_2 \in [a, b]\) such that \(f(x_2) = M\).
	Then let \(g(x) = \frac{1}{M - f(x)}\) for \(x \in [a, b]\).
	Since there exists no \(x\) such that \(M = f(x)\), \(g(x)\) is continuous since we are never dividing by zero.
	So \(g\) is bounded.
	So by the previous theorem, there is some \(k > 0\) such that \(g(x) \leq k\) for all \(x \in [a, b]\).
	This means that \(f(x) \leq M - \frac{1}{k}\) on \([a, b]\) for this \(k\), but this cannot happen since \(M\) is the supremum.
\end{proof}
\noindent Note that these theorems are certainly false if the interval is not closed: consider the counterexample \((0, 1]\) and the function \(x \mapsto x^{-1}\).

\subsection{Inverse Functions}
\begin{definition}
	\(f\) is increasing for \(x \in [a, b]\) if \(f(x_1) \leq f(x_2)\) for all \(x_1 \leq x_2 \in [a, b]\).
	If \(f(x_1) < f(x_2)\) then the function is strictly increasing.
	A function may be called decreasing or strictly decreasing analogously.
\end{definition}
\begin{definition}
	A function \(f\) is called monotone if it is either increasing or decreasing.
\end{definition}
\begin{theorem}
	Let \(f \colon [a, b] \to \mathbb R\) be continuous and strictly increasing for \(x \in [a, b]\).
	Let \(c = f(a)\), \(d = f(b)\).
	Then \(f \colon [a, b] \to [c, d]\) is bijective, and the inverse \(g := f^{-1} \colon [c, d] \to [a, b]\) is continuous and strictly increasing.
\end{theorem}
\noindent A similar theorem holds for strictly decreasing functions.
\begin{proof}
	Let \(c < k < d\).
	From the intermediate value theorem, there exists \(h\) such that \(f(h) = k\).
	This \(h\) must be unique since the function is strictly increasing.
	Then we can define \(g(k) = h\), giving us an inverse \(g \colon [c, d] \to [a, b]\) for \(f\).

	First, note that \(g\) is strictly increasing.
	Indeed, for \(y_1 < y_2\) then \(y_1 = f(x_1)\), \(y_2 = f(x_2)\).
	This means that if \(x_2 \geq x_1\), then since \(f\) is increasing \(y_2 \leq y_1\) which is a contradiction.

	Now, note that \(g\) is continuous.
	Indeed, given \(\varepsilon > 0\), we can let \(k_1 = f(h - \varepsilon)\) and \(k_2 = f(h + \varepsilon)\).
	If \(f\) is strictly increasing, then \(k_1 < k < k_2\).
	Then \(h - \varepsilon < g(y) < h + \varepsilon\).
	So let \(\delta = \min(k_2 - k, k - k_1)\) where \(k \in (c, d)\), establishing continuity as claimed.
\end{proof}
