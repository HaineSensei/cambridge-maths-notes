\subsection{Complex differentiation}
The complex derivative and the real derivative have the same core properties, for instance linearity, the product rule and the chain rule.
However, the complex derivative is significantly more restrictive than the real derivative, since we can approach a point in any number of directions.
If we can find a function that is complex differentiable with this restriction, we actually get a whole array of features for free.
As an example of this restriction, consider the function \(f(z) = \overline{z}\).
This function is actually nowhere differentiable.
If it were differentiable, then any sequence tending to \(z\) would yield the same limit when substituted into the definition of the derivative.
Consider first the sequence
\[
	z_n = z + \frac{1}{n} \to z
\]
Then
\[
	\frac{f(z_n) - f(z)}{z_n - z} = \frac{\overline{z} + \frac{1}{n} - \overline{z}}{z + \frac{1}{n} - z} = 1
\]
Now consider the sequence
\[
	z_n = z + \frac{i}{n} \to z
\]
Then
\[
	\frac{f(z_n) - f(z)}{z_n - z} = \frac{\overline{z} - \frac{i}{n} - \overline{z}}{z + \frac{i}{n} - z} = -1
\]
Hence \(f(z)\) is nowhere differentiable.
On the other hand, the real function \(f(x, y) = (x, -y)\) is clearly real differentiable, since it is linear; but in the complex world the function \(z \mapsto \overline{z}\) is not linear.

\subsection{Definition of power series}
A power series is a series of the form
\[
	\sum_{n=0}^\infty a_n z^n
\]
where \(z \in \mathbb C\), and the \(a_n\) is a given sequence of complex numbers.
We can also take a power series of the form
\[
	\sum_{n=0}^\infty a_n (z-z_0)^n
\]
but for simplicity we will take \(z_0 = 0\) in all of the analysis we will conduct on power series.

\subsection{Radius of convergence}
\begin{lemma}
	If the series
	\[
		\sum_{n=0}^\infty a_n z_1^n
	\]
	converges for some point \(z_1\), and \(\abs{z} < \abs{z_1}\), then the series
	\[
		\sum_{n=0}^\infty a_n z^n
	\]
	also converges absolutely.
\end{lemma}
\begin{proof}
	Since \(\sum_{n=0}^\infty a_n z_1^n\) converges, \(a_n z_1^n \to 0\).
	Thus the sequence \(a_n z_1^n\) is bounded by some \(k > 0\), i.e.\ for all \(n\), \(\abs{a_n z_1^n}<k\).
	Then
	\[
		\abs{a_n z^n} \leq k\abs{\frac{z}{z_1}}^n
	\]
	Since the geometric series \(\sum_0^\infty \abs{\frac{z}{z_1}}^n\) converges, the lemma follows by comparison.
\end{proof}
\noindent Using this lemma, we can find that there exists a radius inside which any given power series converges absolutely.
This radius might be zero, and it might be infinite.
\begin{theorem}
	Any power series either
	\begin{enumerate}
		\item converges absolutely for all \(z\), or
		\item converges absolutely for all \(z\) where \(\abs{z} < R\) and diverges for all \(z\) where \(\abs{z} > R\), or
		\item converges for \(z = 0\) only.
	\end{enumerate}
\end{theorem}
\noindent The circle \(\abs{z} = R\) is called the circle of convergence, and \(R\) is called the radius of convergence.
Note that this theorem does not make any claim about the behaviour \textit{on} the circle of convergence, just the behaviour inside it.
\begin{proof}
	Let
	\[
		S = \left\{ x \in \mathbb R \colon x \geq 0, \sum_0^\infty a_n x^n \text{ converges} \right\}
	\]
	Clearly, \(0 \in S\).
	By the above lemma, if \(x_1 \in S\), then \([0, x_1] \subseteq S\).
	If \(S = [0, \infty)\), then we have case (i) above due to the lemma.

	If \(S \neq [0, \infty)\), there exists a supremum \(0 \leq R = \sup S < \infty\).

	We must now just deal with case (ii), which is \(R > 0\).
	For all \(z_1\) with \(\abs{z_1} < R\) there exists \(R_0\) such that \(\abs{z_1} < R_0 < R\), and absolute convergence follows using the lemma.
	If \(\abs{z_1} > R\), there exists \(R_0\) such that \(\abs{z_1} > R_0 > R\).
	If the series with \(z_1\) converges, then by the lemma the same would be true for \(R_0\).
	But \(R_0\) does not converge, so this is a contradiction.
\end{proof}

\begin{lemma}
	If
	\[
		\abs{\frac{a_{n+1}}{a_n}} \to \ell
	\]
	as \(n \to \infty\), then \(R = \frac{1}{\ell}\).
\end{lemma}
\begin{proof}
	By the ratio test, we have absolute convergence if
	\[
		\abs{\frac{a_{n+1}}{a_n} \frac{z^{n+1}}{z^n}} < 1
	\]
	So we have absolute convergence if \(\abs{z} < \frac{1}{\ell}\) and divergence if \(\abs{z} > \frac{1}{\ell}\) as required.
\end{proof}
\begin{lemma}
	If
	\[
		\abs{a_n^{1/n}} \to \ell
	\]
	as \(n \to \infty\), then \(R = \frac{1}{\ell}\).
\end{lemma}
\noindent This can be shown similarly using the root test.

\subsection{Examples of radii of convergence}
\begin{enumerate}
	\item Consider the series \(\sum_0^\infty \frac{z^n}{n!}\).
	      Using the ratio test, the series converges absolutely everywhere.
	\item The geometric series \(\sum_0^\infty z^n\) gives \(R=1\) by the ratio test.
	      In this case, \(\abs{z} = 1\) gives divergence.
	\item The series \(\sum_0^\infty n!z^n\) has \(R=0\), which again can be seen using the ratio test.
	\item Consider \(\sum_1^\infty \frac{z^n}{n}\).
	      This also has \(R = 1\) by the ratio test.
	      Note that the series diverges for \(z=1\) since we get the harmonic series.
	      However, it converges when \(z = -1\) by the alternating series test.
	      To work out the behaviour at other points on the circle of convergence, we could consider the series \(\sum_1^\infty \frac{z^n}{n}(1-z)\), which converges exactly when the original series does.
	      The partial sums are
	      \begin{align*}
		      S_N & = \sum_1^N \left[ \frac{z_n - z^{n+1}}{n} \right]            \\
		          & = \sum_1^N \frac{z^n}{n} - \sum_1^N \frac{z^{n+1}}{n}        \\
		          & = \sum_1^N \frac{z^n}{n} - \sum_2^{N+1} \frac{z^n}{n-1}      \\
		          & = z - \frac{z^{N+1}}{N+1} + \sum_2^{N+1} \frac{-z^n}{n(n-1)} \\
	      \end{align*}
	      If \(\abs{z} = 1\), then the term \(\frac{z^{N+1}}{N+1}\) will vanish as \(N \to \infty\).
	      If \(z \neq 1\), the term \(\sum_2^{N+1} \frac{-z^n}{n(n-1)}\) converges as \(N \to \infty\).
	      So \(S_N\) does indeed converge for \(\abs{z} = 1\), \(z \neq 1\).
	\item Now, consider \(\sum_1^\infty \frac{z^n}{n^2}\).
	      This has \(R=1\) by the ratio test, but it converges for all \(z\) with \(\abs{z} = 1\).
	\item If we have \(\sum_0^\infty nz^n\), we have \(R=1\), but diverges for all \(\abs{z} = 1\).
\end{enumerate}
\noindent In conclusion, we cannot determine the behaviour at the boundary in the general case.
Inside the radius of convergence, power series will behave as if they were simply polynomials.

\subsection{Infinite differentiability}
\begin{theorem}
	Let \(f(z) = \sum_0^\infty a_n z^n\) have a radius of convergence \(R\).
	Then \(f\) is complex differentiable at all points with \(\abs{z} < R\), with
	\[
		f'(z) = \sum_1^\infty n a_n z^{n-1}
	\]
	with the same radius of convergence as the original series.
\end{theorem}
\noindent This proof comprises the entire subsection.
This whole subsection is non-examinable, but included for completeness.
First, we will state two lemmas.
\begin{lemma}
	If \(\sum_0^\infty a_n z^n\) has radius of convergence \(R\), then both series
	\[
		\sum_1^\infty n a_n z^{n-1}
	\]
	and
	\[
		\sum_2^\infty n(n-1)a_n z^{n-2}
	\]
	also have radius of convergence \(R\).
\end{lemma}
\begin{proof}
	Let \(R_0\) be such that \(0 < \abs{z} < R_0 < R\).
	Since \(a_0 R_0^n \to 0\), the sequence \(a_0 R_0^n\) is bounded.
	In other words there exists a \(k\) such that \(\abs{a_n R_0^n} \leq k\) for all \(n \geq 0\).
	Thus,
	\[
		\abs{a_n n z^{n-1}} = \frac{n}{\abs{z}}\abs{a_n R_0^n} \abs{\frac{z}{R_0}}^n \leq \frac{kn}{\abs{z}}\abs{\frac{z}{R_0}}^n
	\]
	But
	\[
		\sum n\abs{\frac{z}{R_0}}^n
	\]
	converges by the ratio test, since the ratio is
	\[
		\frac{n+1}{n}\abs{\frac{z}{R_0}}^{n+1} \abs{\frac{R_0}{z}}^n = \frac{n+1}{n}\abs{\frac{z}{R_0}} \to \abs{\frac{z}{R_0}} < 1
	\]
	Hence, the original series \(\sum_1^\infty n a_n z^{n-1}\) is absolutely bounded above by a convergent series, and therefore is absolutely convergent.
	So it is known that the radius of convergence of this derivative series is \textit{at least} \(R\).
	Now, if \(\abs{z} > R\), the series diverges since \(\abs{a_n z^n}\) is unbounded, and hence \(\abs{n a_n z^n}\) is also unbounded.
	The same proof applies to the series for the second derivative.
\end{proof}
\noindent We will need this `second derivative' condition in order to talk about the remainder term after the first derivative, which is related to the second derivative.
\begin{lemma}
	First, for all \(2 \leq r \leq n\).
	\[
		\binom{n}{r} \leq n(n-1)\binom{n-2}{r-2}
	\]
	Further, for all \(z \in \mathbb C\), \(h \in \mathbb C\),
	\[
		\abs{(z + h)^n - z^n - nhz^{n-1}} \leq n(n-1)(\abs{z} + \abs{h})^{n-2}\abs{h}^2
	\]
\end{lemma}
\begin{proof}
	For the first part, we can expand the definitions to get
	\[
		\frac{\binom{n}{r}}{\binom{n-2}{r-2}} = \frac{n(n-1)}{r(r-1)} \leq n(n-1)
	\]
	as required.
	For the second part, we can apply the binomial expansion to cancel the other two terms, and we get
	\begin{align*}
		(z + h)^n - z^n - nhz^{n-1}                  & = \left( \sum_{r=0}^n \binom{n}{r} z^{n-r} h^r \right)  - z^n - nhz^{n-1}                                                               \\
		                                             & = \sum_{r=2}^n \binom{n}{r} z^{n-r} h^r                                                                                                 \\
		\therefore \abs{(z + h)^n - z^n - nhz^{n-1}} & = \abs{\sum_{r=2}^n \binom{n}{r} z^{n-r} h^r}                                                                                           \\
		                                             & \leq \sum_{r=2}^n \abs{\binom{n}{r} z^{n-r} h^r}                                                                                        \\
		                                             & = \sum_{r=2}^n \binom{n}{r} \abs{z}^{n-r} \abs{h}^r                                                                                     \\
		                                             & \leq n(n-1) \underbrace{\left[ \sum_{r=2}^n \binom{n-2}{r-2} \abs{z}^{n-r} \abs{h}^{r-2} \right]}_{(\abs{z} + \abs{h})^{n-2}} \abs{h}^2 \\
		                                             & = n(n-1) (\abs{z} + \abs{h})^{n-2} \abs{h}^2                                                                                            \\
	\end{align*}
	as required.
\end{proof}
\noindent Now, we can prove the original theorem.
\begin{proof}
	By the first lemma, we may define \(f'(z)\) to be
	\[
		f'(z) = \sum_1^\infty n a_n z^{n-1}
	\]
	We now just need to prove that
	\[
		\lim_{h \to 0} I = 0;\quad I = \frac{f(z + h) - f(z) - h f'(z)}{h}
	\]
	We can substitute the expressions we have found for each power series:
	\begin{align*}
		I       & = \frac{\sum_0^\infty a_n (z+h)^n - \sum_0^\infty a_n z^n - h \sum_1^\infty n a_n z^{n-1}}{h}            \\
		        & = \frac{1}{h} \sum_0^\infty \left[ a_n (z+h)^n - a_n z^n - h n a_n z^{n-1} \right]                       \\
		        & = \frac{1}{h} \sum_0^\infty a_n \left[ (z+h)^n - z^n - h n z^{n-1} \right]                               \\
		\abs{I} & = \frac{1}{\abs{h}} \abs{\lim_{N \to \infty} \sum_0^N a_n \left[ (z+h)^n - z^n - h n z^{n-1} \right]}    \\
		\intertext{Since the modulus function is continuous,}
		\abs{I} & = \frac{1}{\abs{h}} \lim_{N \to \infty} \abs{\sum_0^N a_n \left[ (z+h)^n - z^n - h n z^{n-1} \right]}    \\
		        & \leq \frac{1}{\abs{h}} \lim_{N \to \infty} \sum_0^N \abs{a_n \left[ (z+h)^n - z^n - h n z^{n-1} \right]} \\
		        & = \frac{1}{\abs{h}} \sum_0^\infty \abs{a_n} \cdot \abs{(z+h)^n - z^n - h n z^{n-1}}                      \\
		\intertext{By the second part of the second lemma above,}
		\abs{I} & \leq \frac{1}{\abs{h}} \sum_0^\infty \abs{a_n} \cdot n(n-1)(\abs{z} + \abs{h})^{n-2}\abs{h}^2            \\
		        & = \abs{h} \sum_0^\infty \abs{a_n} \cdot n(n-1)(\abs{z} + \abs{h})^{n-2}                                  \\
	\end{align*}
	For \(\abs{h}\) small enough, \((\abs{z} + \abs{h}) < R\).
	Therefore, by the first lemma above,
	\[
		\sum_0^\infty \abs{a_n} \cdot n(n-1)(\abs{z} + \abs{h})^{n-2}
	\]
	converges to some \(A(h)\).
	But \(A(h)\) is monotonically decreasing, so
	\[
		\abs{I} \leq \abs{h} A(h) \leq \abs{h} A(r)
	\]
	for some \(r\) such that \(\abs{z} + r < R\).
	We can now let \(h \to 0\), giving
	\[
		\abs{I} \to 0
	\]
	as required.
\end{proof}

\subsection{Defining standard functions}
We can now use this differentiability property to cleanly define the standard exponential, logarithmic and trigonometric functions.
Let \(e \colon \mathbb C \to \mathbb C\) be defined by
\[
	e(z) = \sum_0^\infty \frac{z^n}{n!}
\]
We have already seen that it has infinite radius of convergence.
Straight from the above theorem, \(e\) is infinitely differentiable everywhere, and it is its own derivative.
Note that if a function \(F \colon \mathbb C \to \mathbb C\) has \(F'(z) = 0\) for all \(z \in \mathbb C\), then \(F\) is constant.
Indeed, consider \(g(t) = F(tz) = u(t) + iv(t)\) where \(t, u, v \in \mathbb R\).
Then by the chain rule, \(g'(t) = F'(tz)z = 0\) and hence \(u'(t) + iv'(t) = 0\), giving \(u'(t) = 0\) and \(v'(t) = 0\) everywhere.
We can now apply the real-valued case, showing that \(u\) and \(v\) (and hence \(F\)) are constant everywhere.
Now, let \(a, b \in \mathbb C\), and consider
\[
	F(z) = e(a + b - z)e(z)
\]
Then
\[
	F'(z) = -e(a+b-z)e(z) + e(a+b-z)e(z) = 0
\]
Hence \(e(a + b - z)e(z)\) is constant for all \(z\), hence
\[
	e(a + b - z)e(z) = e(a + b - 0)e(0) = e(a + b)
\]
Since \(z\) is arbitrary, we can set \(z=b\) to recover the familiar relation
\[
	e(a+b-b)e(b) = e(a+b) \implies e(a)e(b) = e(a+b)
\]

\subsection{Exponential and logarithmic functions}
Last lecture, we covered the power series form of the exponential function \(e \colon \mathbb C \to \mathbb C\).
Note that if we input a real number, the output is also real.
Hence, \(e \colon \mathbb R \to \mathbb R\).
This restricted definition of the function has the following properties.
\begin{theorem}
	\begin{enumerate}
		\item \(e \colon \mathbb R \to \mathbb R\) is everywhere differentiable, and \(e'(x) = e(x)\).
		\item \(e(x+y) = e(x)e(y)\).
		\item \(e(x) > 0\).
		\item \(e\) is strictly increasing.
		\item \(e(x) \to \infty\) as \(x \to \infty\), and \(e(x) \to 0\) as \(x \to -\infty\).
		\item \(e \colon \mathbb R \to (0, \infty)\) is a bijection.
	\end{enumerate}
\end{theorem}
\begin{proof}
	The first two properties follow from the last lecture.
	\begin{enumerate}
		\setcounter{enumi}{2}
		\item Clearly, \(e(x) > 0\) for all \(x \geq 0\) by considering the power series, which contains only positive terms for \(x>0\), and also \(e(0) = 1\).
		      Also, \(e(0) = e(x - x) = e(x)e(-x)\), hence for all negative \(x\), \(e(x) > 0\).
		\item Since \(e'(x) = e(x)\), \(e'(x) = e(x) > 0\) everywhere.
		\item By considering partial sums, if \(x>0\) we have \(e(x) > 1+x\), so if \(x \to \infty\), \(e(x) \to \infty\).
		      When \(x \to -\infty\), \(e(x) = \frac{1}{e(x)} \to 0\).
		\item Injectivity follows from being strictly increasing.
		      For surjectivity, we need to show that given any \(y \in (0, \infty)\) there exists some \(x\) such that \(e(x) = y\).
		      Due to property (v) above, we can certainly find real numbers \(a\) and \(b\) such that \(e(a) < y < e(b)\).
		      By the intermediate value theorem, there exists \(x \in \mathbb R\) such that \(e(x) = y\).
	\end{enumerate}
\end{proof}
\begin{remark}
	We have essentially proven that \(e \colon (\mathbb R, +) \to ((0, \infty), \times)\) is a group isomorphism.
	This is exactly the same as showing that it is a bijection.
	Since \(e\) is a function, there exists an inverse function \(\ell \colon ((0, \infty), \times) \to (\mathbb R, +)\).
\end{remark}
\begin{theorem}
	\begin{enumerate}
		\item \(\ell \colon (0, \infty) \to \mathbb R\) is a bijection, and \(\ell(e(x)) = x\) for all \(x \in \mathbb R\), and \(e(\ell(x)) = x\) for all \(x \in (0, \infty)\).
		\item \(\ell\) is differentiable and its derivative is \(\ell'(t) = \frac{1}{t}\).
		\item \(\ell(xy) = \ell(x) + \ell(y)\).
	\end{enumerate}
\end{theorem}
\begin{proof}
	\begin{enumerate}
		\item This first propety is obvious from the definition.
		\item By the inverse function theorem, \(\ell\) is differentiable everywhere and \(\ell'(t) = \frac{1}{t}\) as required.
		\item From IA Groups, if \(e\) is an isomorphism, so is its inverse.
	\end{enumerate}
\end{proof}

\subsection{Real numbered exponents}
We will now define for \(\alpha \in \mathbb R\) and \(x > 0\) the function
\[
	r_\alpha(x) = e(\alpha \ell(x))
\]
This can be taken as the definition of \(x\) raised to the power \(\alpha\).
\begin{theorem}
	Suppose \(x, y > 0\) and \(\alpha, \beta \in \mathbb R\).
	Then
	\begin{enumerate}
		\item \(r_\alpha(xy) = r_\alpha(x)r_\alpha(y)\)
		\item \(r_{\alpha + \beta}(x) = r_\alpha(x) r_\beta(x)\)
		\item \(r_\alpha(r_\beta(x)) = r_{\alpha\beta}(x)\)
		\item \(r_1(x) = x\), and \(r_0(x) = 1\)
	\end{enumerate}
\end{theorem}
\begin{proof}
	\begin{enumerate}
		\item \(r_\alpha(xy) = e(\alpha \ell(xy)) = e(\alpha \ell(x) + \alpha \ell(y)) = e(\alpha \ell(x))e(\alpha\ell(y)) = r_\alpha(x)r_\alpha(y)\)
		\item \(r_{\alpha + \beta}(x) = e((\alpha + \beta) \ell(x)) = e(\alpha\ell(x) + \beta\ell(x)) = e(\alpha\ell(x))e(\beta\ell(x)) = r_\alpha(x) r_\beta(x)\)
		\item \(r_\alpha(r_\beta(x)) = e(\alpha \ell[e(\beta \ell(x))]) = e(\alpha \beta \ell(x)) = r_{\alpha\beta}(x)\)
		\item \(r_1(x) = e(\ell(x)) = x\), and \(r_0(x) = e(0 \ell(x)) = e(0) = 1\)
	\end{enumerate}
\end{proof}
\noindent Suppose we want to compute \(r_n(x)\), where \(n \in\mathbb Z\).
Then \(r_n(x) = r_{1 + \dots + 1}(x) = x \cdots x\), so we have aggreement between \(r_n(x)\) and our previous definition of \(x^n\).
Similarly, since \(r_1(x) r_{-1}(x) = 1\), we have \(r_{-1}(x) = \frac{1}{x}\).
Further, \(r_{\frac{1}{q}}(x) = x^\frac{1}{q}\).
Therefore, \(r_{\frac{p}{q}}(x) = x^{\frac{p}{q}}\).
So this definition is simply a more general definition for exponentiation by a real number.

From now, we will let \(\exp(x) \equiv e(x)\), \(\log(x) \equiv \ell(x)\), and \(x^\alpha \equiv r_\alpha(x)\).
In fact, \(\exp(x) = e^x\) for a suitable number \(e\), since \(e(x) = e(x \log(e)) = r_x(e) = e^x\) where \(e := e(1) = \sum_0^\infty \frac{1}{n!}\).

Finally, we can compute the derivative of \(x^\alpha\) using the chain rule.
\[
	(x^\alpha)' = \left( e^{\alpha \log x} \right)' = e^{\alpha \log x} \alpha \frac{1}{x} = \alpha x^\alpha x^{-1} = \alpha x^{\alpha - 1}
\]
as expected.
Further, if \(f(x) = a^x\), we can find
\[
	f'(x) = \left( e^{x \log a} \right)' = e^{x \log a} \log a = a^x \log a
\]

\subsection{Trigonometric functions}
We define
\begin{align*}
	\cos z & = 1 - \frac{z^2}{2!} + \frac{z^4}{4!} - \frac{z^6}{6!} + \dots = \sum_0^\infty \frac{(-1)^k z^{2k}}{(2k)!}     \\
	\sin z & = z - \frac{z^3}{3!} + \frac{z^5}{5!} - \frac{z^7}{7!} + \dots = \sum_0^\infty \frac{(-1)^k z^{2k+1}}{(2k+1)!} \\
\end{align*}
Both power series have infinite radius of convergence, by the ratio test (the same proof from the exponential function can be used here).
Hence \(\cos z\) and \(\sin z\) are differentiable, and \(\dv{z}\cos z = -\sin z\) and \(\dv{z}\sin z = \cos z\) as expected, by termwise differentiation.
Further, we can deduce that
\[
	e^{iz} = \sum_0^\infty \frac{(iz)^n}{n!} = \sum_0^\infty \frac{(iz)^{2k}}{(2k)!} + \sum_0^\infty \frac{(iz)^{2k+1}}{(2k+1)!}
\]
Note that
\[
	(iz)^{2k} = (-1)^k z^{2k};\quad (iz)^{2k+1} = i (-1)^k z^{2k+1}
\]
Hence,
\[
	e^{iz} = \cos z + i \sin z
\]
Similarly,
\[
	e^{-iz} = \cos z - i \sin z
\]
We can then write
\[
	\cos z = \frac{1}{2}\qty( e^{iz} + e^{-iz} );\quad \sin z = \frac{1}{2i}\qty( e^{iz} - e^{-iz} )
\]
Many common trigonometric identities follow from this, such as the identity \(\cos^2 z + \sin^2 z \equiv 1\).
However, we have not deduced the period of the functions.
Now, restricted to the real case, \(\sin x, \cos x \in \mathbb R\), and the identity \(\cos^2 z + \sin^2 z \equiv 1\) gives that \(\abs{\sin x} \leq 1\) and \(\abs{\cos x} \leq 1\) for all real \(x\).

\subsection{Circle constants}
\begin{proposition}
	There is a smallest positive number \(\pi\) such that
	\[
		\cos \frac{\pi}{2} = 0
	\]
	and we have \(\sqrt{2} < \frac{\pi}{2} < \sqrt{3}\).
\end{proposition}
\begin{proof}
	If \(0 < x < 2\),
	\[
		\sin x = \qty(x - \frac{x^3}{3!}) + \qty(\frac{x^5}{5!} - \frac{x^7}{7!}) + \cdots
	\]
	For this range of values, each parenthesised block is positive, so \(\sin x > 0\).
	So in this range,
	\[
		\dv{x} \cos x < 0
	\]
	Hence, \(\cos x\) is a strictly decreasing function on this interval.
	Now,
	\[
		\cos \sqrt{2} = 1 - \frac{\sqrt{2}^2}{2!} + \qty(\frac{\sqrt{2}^4}{4!} - \frac{\sqrt{2}^6}{6!}) + \cdots > 0
	\]
	since each bracketed block is positive.
	\[
		\cos \sqrt{3} = 1 - \frac{\sqrt{3}^2}{2!} + \frac{\sqrt{3}^4}{4!} - \qty(\frac{\sqrt{3}^6}{6!} - \frac{\sqrt{3}^8}{8!}) + \cdots < 0
	\]
	since all the bracketed terms are positive, and being subtracted from a negative number.
	By the intermediate value theorem, the existence of such a \(\pi\) follows.
\end{proof}
\begin{corollary}
	We have that \(\sin \frac{\pi}{2} = 1\).
\end{corollary}
\begin{proof}
	We know that \(\cos^2 \frac{\pi}{2} + \sin^2 \frac{\pi}{2} = 1\), and \(\sin \frac{\pi}{2} > 0\), so the result follows.
\end{proof}
\begin{theorem}
	The following standard properties about the periodicity of trigonometric functions hold.
	\begin{enumerate}
		\item \(\sin(z + \frac{\pi}{2}) = \cos z\), and \(\cos(z + \frac{\pi}{2}) = -\sin z\)
		\item \(\sin(z + \pi) = -\sin z\), and \(\cos(z + \pi) = -\cos z\)
		\item \(\sin(z + 2 \pi) = \sin z\), and \(\cos(z + 2\pi) = \cos z\)
	\end{enumerate}
\end{theorem}
\noindent The proofs are immediate from the angle addition formulae.
This then implies that
\[
	e^{iz + 2\pi i} = e^{iz}
\]
Hence \(e^{z}\) is periodic with period \(2 \pi i\).
