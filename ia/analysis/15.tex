\subsection{Proving Infinite Differentiability}
\begin{theorem}
	Let \(f(z) = \sum_0^\infty a_n z^n\) have a radius of convergence \(R\).
	Then \(f\) is complex differentiable at all points with \(\abs{z} < R\), with
	\[
		f'(z) = \sum_1^\infty n a_n z^{n-1}
	\]
	with the same radius of convergence as the original series.
\end{theorem}
\noindent This proof comprises the entire subsection.
This whole subsection is non-examinable, but included for completeness.
First, we will state two lemmas.
\begin{lemma}
	If \(\sum_0^\infty a_n z^n\) has radius of convergence \(R\), then both series
	\[
		\sum_1^\infty n a_n z^{n-1}
	\]
	and
	\[
		\sum_2^\infty n(n-1)a_n z^{n-2}
	\]
	also have radius of convergence \(R\).
\end{lemma}
\begin{proof}
	Let \(R_0\) be such that \(0 < \abs{z} < R_0 < R\).
	Since \(a_0 R_0^n \to 0\), the sequence \(a_0 R_0^n\) is bounded.
	In other words there exists a \(k\) such that \(\abs{a_n R_0^n} \leq k\) for all \(n \geq 0\).
	Thus,
	\[
		\abs{a_n n z^{n-1}} = \frac{n}{\abs{z}}\abs{a_n R_0^n} \abs{\frac{z}{R_0}}^n \leq \frac{kn}{\abs{z}}\abs{\frac{z}{R_0}}^n
	\]
	But
	\[
		\sum n\abs{\frac{z}{R_0}}^n
	\]
	converges by the ratio test, since the ratio is
	\[
		\frac{n+1}{n}\abs{\frac{z}{R_0}}^{n+1} \abs{\frac{R_0}{z}}^n = \frac{n+1}{n}\abs{\frac{z}{R_0}} \to \abs{\frac{z}{R_0}} < 1
	\]
	Hence, the original series \(\sum_1^\infty n a_n z^{n-1}\) is absolutely bounded above by a convergent series, and therefore is absolutely convergent.
	So it is known that the radius of convergence of this derivative series is \textit{at least} \(R\).
	Now, if \(\abs{z} > R\), the series diverges since \(\abs{a_n z^n}\) is unbounded, and hence \(\abs{n a_n z^n}\) is also unbounded.
	The same proof applies to the series for the second derivative.
\end{proof}
\noindent We will need this `second derivative' condition in order to talk about the remainder term after the first derivative, which is related to the second derivative.
\begin{lemma}
	First, for all \(2 \leq r \leq n\).
	\[
		\binom{n}{r} \leq n(n-1)\binom{n-2}{r-2}
	\]
	Further, for all \(z \in \mathbb C\), \(h \in \mathbb C\),
	\[
		\abs{(z + h)^n - z^n - nhz^{n-1}} \leq n(n-1)(\abs{z} + \abs{h})^{n-2}\abs{h}^2
	\]
\end{lemma}
\begin{proof}
	For the first part, we can expand the definitions to get
	\[
		\frac{\binom{n}{r}}{\binom{n-2}{r-2}} = \frac{n(n-1)}{r(r-1)} \leq n(n-1)
	\]
	as required.
	For the second part, we can apply the binomial expansion to cancel the other two terms, and we get
	\begin{align*}
		(z + h)^n - z^n - nhz^{n-1}                  & = \left( \sum_{r=0}^n \binom{n}{r} z^{n-r} h^r \right)  - z^n - nhz^{n-1}                                                               \\
		                                             & = \sum_{r=2}^n \binom{n}{r} z^{n-r} h^r                                                                                                 \\
		\therefore \abs{(z + h)^n - z^n - nhz^{n-1}} & = \abs{\sum_{r=2}^n \binom{n}{r} z^{n-r} h^r}                                                                                           \\
		                                             & \leq \sum_{r=2}^n \abs{\binom{n}{r} z^{n-r} h^r}                                                                                        \\
		                                             & = \sum_{r=2}^n \binom{n}{r} \abs{z}^{n-r} \abs{h}^r                                                                                     \\
		                                             & \leq n(n-1) \underbrace{\left[ \sum_{r=2}^n \binom{n-2}{r-2} \abs{z}^{n-r} \abs{h}^{r-2} \right]}_{(\abs{z} + \abs{h})^{n-2}} \abs{h}^2 \\
		                                             & = n(n-1) (\abs{z} + \abs{h})^{n-2} \abs{h}^2                                                                                            \\
	\end{align*}
	as required.
\end{proof}
\noindent Now, we can prove the original theorem.
\begin{proof}
	By the first lemma, we may define \(f'(z)\) to be
	\[
		f'(z) = \sum_1^\infty n a_n z^{n-1}
	\]
	We now just need to prove that
	\[
		\lim_{h \to 0} I = 0;\quad I = \frac{f(z + h) - f(z) - h f'(z)}{h}
	\]
	We can substitute the expressions we have found for each power series:
	\begin{align*}
		I       & = \frac{\sum_0^\infty a_n (z+h)^n - \sum_0^\infty a_n z^n - h \sum_1^\infty n a_n z^{n-1}}{h}            \\
		        & = \frac{1}{h} \sum_0^\infty \left[ a_n (z+h)^n - a_n z^n - h n a_n z^{n-1} \right]                       \\
		        & = \frac{1}{h} \sum_0^\infty a_n \left[ (z+h)^n - z^n - h n z^{n-1} \right]                               \\
		\abs{I} & = \frac{1}{\abs{h}} \abs{\lim_{N \to \infty} \sum_0^N a_n \left[ (z+h)^n - z^n - h n z^{n-1} \right]}    \\
		\intertext{Since the modulus function is continuous,}
		\abs{I} & = \frac{1}{\abs{h}} \lim_{N \to \infty} \abs{\sum_0^N a_n \left[ (z+h)^n - z^n - h n z^{n-1} \right]}    \\
		        & \leq \frac{1}{\abs{h}} \lim_{N \to \infty} \sum_0^N \abs{a_n \left[ (z+h)^n - z^n - h n z^{n-1} \right]} \\
		        & = \frac{1}{\abs{h}} \sum_0^\infty \abs{a_n} \cdot \abs{(z+h)^n - z^n - h n z^{n-1}}                      \\
		\intertext{By the second part of the second lemma above,}
		\abs{I} & \leq \frac{1}{\abs{h}} \sum_0^\infty \abs{a_n} \cdot n(n-1)(\abs{z} + \abs{h})^{n-2}\abs{h}^2            \\
		        & = \abs{h} \sum_0^\infty \abs{a_n} \cdot n(n-1)(\abs{z} + \abs{h})^{n-2}                                  \\
	\end{align*}
	For \(\abs{h}\) small enough, \((\abs{z} + \abs{h}) < R\).
	Therefore, by the first lemma above,
	\[
		\sum_0^\infty \abs{a_n} \cdot n(n-1)(\abs{z} + \abs{h})^{n-2}
	\]
	converges to some \(A(h)\).
	But \(A(h)\) is monotonically decreasing, so
	\[
		\abs{I} \leq \abs{h} A(h) \leq \abs{h} A(r)
	\]
	for some \(r\) such that \(\abs{z} + r < R\).
	We can now let \(h \to 0\), giving
	\[
		\abs{I} \to 0
	\]
	as required.
\end{proof}

\subsection{Defining Standard Functions}
We can now use this differentiability property to cleanly define the standard exponential, logarithmic and trigonometric functions.
Let \(e \colon \mathbb C \to \mathbb C\) be defined by
\[
	e(z) = \sum_0^\infty \frac{z^n}{n!}
\]
We have already seen that it has infinite radius of convergence.
Straight from the above theorem, \(e\) is infinitely differentiable everywhere, and it is its own derivative.
Note that if a function \(F \colon \mathbb C \to \mathbb C\) has \(F'(z) = 0\) for all \(z \in \mathbb C\), then \(F\) is constant.
Indeed, consider \(g(t) = F(tz) = u(t) + iv(t)\) where \(t, u, v \in \mathbb R\).
Then by the chain rule, \(g'(t) = F'(tz)z = 0\) and hence \(u'(t) + iv'(t) = 0\), giving \(u'(t) = 0\) and \(v'(t) = 0\) everywhere.
We can now apply the real-valued case, showing that \(u\) and \(v\) (and hence \(F\)) are constant everywhere.
Now, let \(a, b \in \mathbb C\), and consider
\[
	F(z) = e(a + b - z)e(z)
\]
Then
\[
	F'(z) = -e(a+b-z)e(z) + e(a+b-z)e(z) = 0
\]
Hence \(e(a + b - z)e(z)\) is constant for all \(z\), hence
\[
	e(a + b - z)e(z) = e(a + b - 0)e(0) = e(a + b)
\]
Since \(z\) is arbitrary, we can set \(z=b\) to recover the familiar relation
\[
	e(a+b-b)e(b) = e(a+b) \implies e(a)e(b) = e(a+b)
\]
