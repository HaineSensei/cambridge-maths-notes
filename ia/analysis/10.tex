\subsection{Differentiating Polynomial Terms}
As an example of the differentiability properties we saw last lecture, we can find the derivative of \(f(x) = x^n\) for \(n \in \mathbb Z\), \(n > 0\).
If \(n=1\), clearly \(f'(x) = 1\).
We can show inductively that \(f'(x) = nx^{n-1}\).
Indeed,
\begin{align*}
	(x^n)' & = x \cdot (x^{n-1})' + (x)' \cdot x^{n-1} \\
	       & = (n-1)x^{n-1} + x^{n-1}                  \\
	       & = nx^{n-1}
\end{align*}
We can now take \(f(x) = x^{-n}\).
Using the reciprocal law,
\begin{align*}
	f'(x) & = \frac{-(x^n)'}{(x^n)^2}  \\
	      & = \frac{-nx^{n-1}}{x^{2n}} \\
	      & = -nx^{-n-1}
\end{align*}

\subsection{Chain Rule}
\begin{theorem}
	Let \(f \colon U \to \mathbb C\) be such that \(f(x) \in V\) for all \(x \in U\).
	If \(f\) is differentiable at \(a \in U\), and \(g \colon V \to \mathbb C\) is differentiable at \(f(a) \in V\), then \(g \circ f\) is differentiable at \(a\) with
	\[
		gf'(a) = f'(a)g'(f(a))
	\]
\end{theorem}
\begin{proof}
	We know that we can write
	\[
		f(x) = f(a) + (x-a)f'(a) + \varepsilon_f(x)(x-a)
	\]
	where \(\lim_{x \to a} \varepsilon_f(x) = 0\).
	Further,
	\[
		g(y) = g(b) + (y-b)g'(b) + \varepsilon_g(y)(y-b)
	\]
	where \(\lim_{y \to b} \varepsilon_g(y) = 0\), and \(b = f(a)\).
	We will set \(\varepsilon_f(a) = 0\) and \(\varepsilon_g(b) = 0\), so they are continuous at \(x=a\) and \(y=b\), so that everything is well-defined when we begin to compose the functions.
	Now, \(y=f(x)\), so
	\begin{align*}
		g(f(x)) & =  g(b) + (f(x) - b)g'(b) + \varepsilon_g(f(x))(f(x) - b)                                                                                                       \\
		        & = g(f(a)) + \left[ (x-a)f'(a) + \varepsilon_f(x)(x-a) \right]\left[ g'(b) + \varepsilon_g(f(x)) \right]                                                         \\
		        & = g(f(a)) + (x-a)f'(a)g'(b) + (x-a)\underbrace{\left[ \varepsilon_f(x) g'(b) + \varepsilon_g(f(x)) \left( f'(a) + \varepsilon_f(x) \right) \right]}_{\sigma(x)} \\
	\end{align*}
	Now, we just need to show that \(\lim_{x \to a} \sigma(x) = 0\) in order to prove the theorem.
	Clearly
	\[
		\sigma(x) = \underbrace{\varepsilon_f(x)}_{\to 0} g'(b) + \underbrace{\varepsilon_g(f(x))}_{\to 0} \left( f'(a) + \varepsilon_f(x) \right)
	\]
	Hence \(\sigma(x) \to 0\) as required.
\end{proof}

\subsection{Rolle's Theorem}
\begin{theorem}
	Let \(f \colon [a,b] \to \mathbb R\) be a continuous function on \([a,b]\) and differentiable on \((a, b)\).
	If \(f(a) = f(b)\), then there exists \(c \in (a,b)\) such that \(f'(c) = 0\).
\end{theorem}
\begin{proof}
	Let \(M\) be the maximum point and \(m\) be the minimum point of the function.
	Recall that in Lecture 8 we proved that any function achieves its bounds.
	Let \(k = f(a)\).
	If \(M=m=k\), then \(f\) must be a constant, and clearly \(f'(c) = 0\) for every value \(c \in (a, b)\).
	Otherwise, either \(M > k\) or \(m < k\).
	Suppose \(M > k\) (the proof is very similar if \(m < k\)).
	Then there exists some value \(c \in (a, b)\) such that \(f(c) = M\).
	We would like to show that \(f'(c) = 0\), so let us suppose that \(f'(c) \neq 0\).
	If \(f'(c) > 0\), then there are values \(d > c\) where \(f(d) > f(c)\).
	Indeed,
	\[
		f(h+c) - f(c) = h\left[ f'(c) + \varepsilon(h) \right]
	\]
	For a small, positive \(h\), this value is positive.
	This contradicts the fact that \(M\) is the maximum.
	Similarly, if \(f'(c) < 0\) there are values \(d < c\) with \(f(d) > f(c)\).
	Hence \(f'(c) = 0\).
\end{proof}

\subsection{Mean Value Theorem}
We can make a small change to Rolle's theorem and obtain the mean value theorem.
\begin{theorem}
	Let \(f \colon [a,b] \to \mathbb R\) be a continuous function on \([a,b]\) and differentiable on \((a, b)\).
	Then there exists \(c \in (a, b)\) such that
	\[
		f(b) - f(a) = f'(c)(b-a)
	\]
\end{theorem}
\begin{proof}
	Let \(\phi\) be a function defined by \(\phi(x) = f(x) - kx\), choosing a \(k\) such that \(\phi(a) = \phi(b)\).
	We can find that
	\[
		f(b) - bk = f(a) - ak \implies k = \frac{f(b) - f(a)}{b - a}
	\]
	By Rolle's theorem, there exists \(c \in (a,b)\) such that \(\phi'(c) = 0\).
	Now, note that \(f'(x) = \phi'(x) + k\), hence there exists \(c\) such that \(f'(c) = k\).
\end{proof}
\begin{remark}
	We will often rewrite the mean value theorem as follows.
	\[
		f(a + h) = f(a) + hf'(a + \theta h)
	\]
	where \(\theta \in (0, 1)\).
	Note, however, that \(\theta\) is a function of \(h\), so if we begin to shrink \(h\) then \(\theta\) may change.
\end{remark}
