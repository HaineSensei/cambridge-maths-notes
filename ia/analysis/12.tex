\subsection{Cauchy's Mean Value Theorem}
\begin{theorem}
	If \(f, g \colon [a,b] \to \mathbb R\) are continuous, and differentiable on \((a, b)\), there exists \(t \in (a,b)\) such that
	\[
		(f(b) - f(a))g'(t) = f'(t)(g(b) - g(a))
	\]
\end{theorem}
\noindent We can recover the normal mean value theorem from Cauchy's generalisation by taking \(g(x) = x\).
\begin{proof}
	Let
	\[
		\phi(x) = \begin{vmatrix}
			1    & 1    & 1    \\
			f(a) & f(x) & f(b) \\
			g(a) & g(x) & g(b)
		\end{vmatrix}
	\]
	Certainly \(\phi(x)\) is continuous on \([a,b]\) and differentiable on \((a, b)\), by using previous results.
	Also, \(\phi(a) = \phi(b) = 0\) by observing the linear dependence of the columns.
	By Rolle's theorem, there exists \(t \in (a, b)\) such that \(\phi'(t) = 0\).
	We can expand \(\phi'(t)\) and this will show the required result.
	\[
		\phi'(x) = f'(x)g(b) - g'(x)f(b) + f(a)g'(x) - g(a)f'(x) = f'(x) [g(b) - g(a)] + g'(x) [f(a) - f(b)]
	\]
\end{proof}

\subsection{Example of L'H\^opital's Rule}
The derivation of L'H\^opital's rule is on an example sheet, so in this subsection we will consider only a special case of it, using Cauchy's mean value theorem.
\[
	\ell = \lim_{x \to 0} \frac{e^x - 1}{\sin x}
\]
We can write
\[
	\ell = \lim_{x \to 0} \frac{e^x - e^0}{\sin x - \sin 0} = \frac{e^t}{\cos t}
\]
for some \(t \in (0, x)\).
So as \(x \to 0\), \(t \to 0\) and hence
\[
	\frac{e^t}{\cos t} \to 1
\]

\subsection{Taylor's Theorem}
\begin{theorem}[Taylor's Theorem with Lagrange's Remainder]
	Suppose \(f\) and its derivatives up to order \(n-1\) are continuous in \([a, a+h]\), and \(f^{(n)}\) exists for \(x \in (a, a+h)\).
	Then
	\[
		f(a+h) = f(a) + hf'(a) + \frac{h^2}{2!} f''(a) + \dots + \frac{h^{n-1}}{(n-1)!}f^{(n-1)}(a) + \frac{h^n}{n!}f^{(n)}(a + \theta h)
	\]
	where \(\theta \in (0, 1)\).
\end{theorem}
\noindent Note that for \(n=1\), this is exactly the mean value theorem, so this can be seen as an \(n\)th order extension of the mean value theorem.
We commonly write \(R_n\) for the final error term \(\frac{h^n}{n!}f^{(n)}(a + \theta h)\).
This is known as Lagrange's form of the remainder.
\begin{proof}
	For \(0 \leq t \leq h\), we define
	\[
		\phi(t) = f(a+t) - f(a) - tf'(a) - \dots - \frac{t^{n-1}}{(n-1)!}f^{(n-1)}(a) - \frac{t^n}{n!}B
	\]
	where we choose \(B\) suitably such that \(\phi(h) = 0\).
	(Recall that in the proof of the mean value theorem, we used \(f(x) - kx\) and picked \(k\) suitably such that this allowed the use of Rolle's theorem.
	This is entirely analogous, but generalised to the \(n\)th derivative).
	Note that
	\[
		\phi(0) = \phi'(0) = \dots = \phi^{(n-1)}(0) = 0
	\]
	We can use Rolle's theorem inductively \(n\) times.
	Since \(\phi(0) = \phi(h) = 0\), there is a point \(0 < h_1 < h\) such that \(\phi'(h_1) = 0\).
	Since \(\phi'(0) = \phi'(h_1) = 0\), there is a point \(0 < h_2 < h_1\) such that \(\phi''(h_2) = 0\).
	This continues until we find a point \(0 < h_n < h\) such that \(\phi^{(n)}(h_n) = 0\).
	Hence \(h_n = \theta h\) for some \(0 < \theta < 1\).
	Now, \(\phi^{(n)}(t) = f^{(n)}(a + t) - B\).
	We can see now that \(B = f^{(n)}(a + \theta h)\), which gives the required result.
\end{proof}
\noindent We can prove an alternative version of Taylor's theorem with a different error term.
\begin{theorem}[Taylor's Theorem with Cauchy's Remainder]
	Suppose (equivalently to before) \(f\) and its derivatives up to order \(n-1\) are continuous in \([a, a+h]\), and \(f^{(n)}\) exists for \(x \in (a, a+h)\).
	Then
	\[
		f(a+h) = f(a) + hf'(a) + \frac{h^2}{2!} f''(a) + \dots + \frac{h^{n-1}}{(n-1)!}f^{(n-1)}(a) + R_n
	\]
	where
	\[
		R_n = \frac{(1 - \theta)^{n-1}h^n f^{(n)}(a + \theta h)}{(n-1)!}
	\]
	for \(\theta \in (0, 1)\).
\end{theorem}
\begin{proof}
	For simplicity, in this proof we let \(a = 0\), although the same argument applies when \(a \neq 0\).
	Let us define
	\[
		F(t) = f(h) - f(t) - (h-t)f'(t) - \dots - \frac{(h-t)^{n-1}f^{(n-1)}(t)}{(n-1)!}
	\]
	for \(t \in [0, h]\).
	Then
	\begin{align*}
		F'(t) & = -f'(t) + f'(t) - (h-t)f''(t) + (h-t)f''(t) - \frac{1}{2} (h-t)^2f'''(t) + \frac{1}{2} (h-t)^2f'''(t) \\
		      & - \dots - \frac{(h-t)^{n-1}}{(n-1)!}f^{(n)}(t)                                                         \\
		      & = - \frac{(h-t)^{n-1}}{(n-1)!}f^{(n)}(t)
	\end{align*}
	Let
	\[
		\phi(t) = F(t) - \left[ \frac{h-t}{h} \right]^p F(0)
	\]
	where \(p \in \mathbb N\) and \(1 \leq p \leq n\).
	Then
	\[
		\phi(0) = \phi(h) = 0
	\]
	By Rolle's theorem, there exists \(\theta \in (0, 1)\) such that
	\[
		\phi'(\theta h) = 0
	\]
	We can compute \(\phi'\) to find
	\[
		\phi'(\theta h) = F'(\theta h) + \frac{p(1-\theta)^{p-1}}{h} F(0) = 0
	\]
	Substituting everything back into \(F\) gives
	\[
		0 = \frac{-h^{n-1}(1-\theta)^{n-1}}{(n-1)!}f^{(n)}(\theta h) + \frac{p(1-\theta)^{p-1}}{h}\left[ f(h) - f(0) - h'(0) - \dots - \frac{h^{n-1}}{(n-1)!}f^{(n-1)}(0) \right]
	\]
	Hence
	\[
		f(h) = f(0) + hf'(0) + \frac{h^2}{2!} f''(0) + \dots + \frac{h^{n-1}}{(n-1)!}f^{(n-1)}(0) + \underbrace{\frac{h^n(1 - \theta)^{n-1}f^{(n)}(\theta h)}{(n-1)!\cdot p(1-\theta)^{p-1}}}_{R_n}
	\]
	By letting \(p = n\), we get Lagrange's remainder.
	If \(p=1\), we get Cauchy's remainder.
\end{proof}
