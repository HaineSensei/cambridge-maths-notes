\subsection{Integration by Parts}
We can use the fundamental theorem of calculus to deduce familiar integration techniques, such as integration by parts, and integration by substitution.
\begin{corollary}
	Suppose \(f', g'\) exist and are continuous on \([a, b]\).
	Then
	\[
		\int_a^b f'g = \eval{fg}_a^b - \int_a^b fg'
	\]
\end{corollary}
\begin{proof}
	By the product rule, we have
	\[
		(fg)' = f'g + fg'
	\]
	Then by the fundamental theorem of calculus,
	\[
		\int_a^b (fg)' = \eval{fg}_a^b = \int_a^b f'g + \int_a^b fg'
	\]
	and the result follows.
\end{proof}

\subsection{Integration by Substitution}
\begin{corollary}
	Let \(g \colon [\alpha, \beta] \to [a, b]\) with \(g(\alpha) = a, g(\beta) = b\) and let \(g'\) exist and be continuous on \([\alpha, \beta]\).
	Let \(f \colon [a, b] \to \mathbb R\) be continuous.
	Then
	\[
		\int_a^b f(x)\dd{x} = \int_\alpha^\beta f(g(t))g'(t)\dd{t}
	\]
\end{corollary}
\begin{proof}
	Let \(F(x) = \int_a^x f(t) \dd{t}\).
	Then let \(h(t) = F(g(t))\).
	This is well defined since \(g\) takes values in \([a, b]\).
	Then,
	\begin{align*}
		\int_\alpha^\beta f(g(t))g'(t)\dd{t} & = \int_\alpha^\beta F'(g(t))g'(t) \dd{t} \\
		                                     & = \int_\alpha^\beta h'(t) \dd{t}         \\
		                                     & = h(\beta) - h(\alpha)                   \\
		                                     & = F(b) - F(a)                            \\
		                                     & = F(b)                                   \\
		                                     & = \int_a^b f(x) \dd{x}
	\end{align*}
\end{proof}

\subsection{Integral Remainder Form of Taylor's Theorem}
\begin{theorem}
	Let \(f\) such that \(f^{(n)}(x)\) is continuous for \(x \in [0, h]\).
	Then
	\[
		f(h) = f(0) + \dots + \frac{h^{n-1}f^{(n-1)}(0)}{(n-1)!} + R_n
	\]
	where
	\[
		R_n = \frac{h^n}{(n-1)!} \int_0^1 (1-t)^{n-1}f^{(n)}(th) \dd{t}
	\]
\end{theorem}
\noindent Note that for this formulation of Taylor's theorem, we require continuity of \(f^{(n)}(x)\), whereas with the previous remainders, the \(n\)th derivative need not be continuous.
\begin{proof}
	First, by substituting \(u = th\), we can see that it is sufficient to show
	\[
		R_n = \frac{1}{(n-1)!} \int_0^h (h - u)^{n-1}f^{(n)}(u) \dd{u}
	\]
	Now, integrating by parts, we have
	\begin{align*}
		R_n & = \frac{-h^{n-1}f^{(n-1)}(0)}{(n-1)!} + \frac{1}{(n-2)!}\int_0^h (h - u)^{n-2}f^{(n-1)}(u) \dd{u} \\
		    & = \frac{-h^{n-1}f^{(n-1)}(0)}{(n-1)!} + R_{n-1}
	\end{align*}
	Hence,
	\[
		R_n = -\frac{h^{n-1}f^{(n-1)}(0)}{(n-1)!} - \frac{h^{n-2}f^{(n-2)}(0)}{(n-2)!} - \dots - \underbrace{\int_0^h f'(u) \dd{u}}_{f(h) - f(0)}
	\]
	which is exactly all the other terms in the Taylor polynomial as required.
\end{proof}

\subsection{Mean Value Theorem for Integrals}
\begin{theorem}
	Let \(f, g \colon [a, b] \to \mathbb R\) be continuous with \(g(x) \neq 0\) for all \(x \in (a, b)\).
	Then
	\[
		\exists c \in (a, b) \st \int_a^b f(x) g(x) \dd{x} = f(c)\int_a^b g(x)\dd{x}
	\]
\end{theorem}
\noindent Note that if we let \(g(x) = 1\), we get
\[
	\int_a^b f(x) \dd{x} = f(c)(b-a)
\]
\begin{proof}
	We will use Cauchy's mean value theorem to get this result.
	Let
	\[
		F(x) = \int_a^x fg;\quad G(x) = \int_a^x g
	\]
	Then there exists an intermediate point \(c\) such that
	\[
		(F(b) - F(a))G'(c) = F'(c)(G(b) - G(a))
	\]
	By the fundamental theorem of calculus,
	\[
		\qty(\int_a^b fg)g(c) = f(c)g(c)\qty(\int_a^b g)
	\]
	Now, since \(g \neq 0\) everywhere,
	\[
		\int_a^b fg = f(c)\int_a^b g
	\]
\end{proof}

\subsection{Deriving Lagrange's and Cauchy's Remainders for Taylor's Theorem}
We can use this new mean value theorem to recover the other forms of the remainders in Taylor's theorem.
We have
\[
	R_n = \frac{h^n}{(n-1)!} \int_0^1 (1-t)^{n-1}f^{(n)}(th) \dd{t}
\]
and we want to show that this is equal to
\[
	\frac{h^n}{n!}f^{(n)}(a + \theta h);\quad \frac{(1 - \theta)^{n-1}h^n f^{(n)}(a + \theta h)}{(n-1)!}
\]
First, let us apply the above mean value theorem with \(g \equiv 1\) and the entire integrand in \(R_n\) as \(f\).
Then
\[
	R_n = \frac{h^n}{(n-1)!} \int_0^1 (1-t)^{n-1}f^{(n)}(th) \dd{t} = \frac{h^n}{(n-1)!} (1-\theta)^{n-1}f^{(n)}(\theta h)
\]
as required for Cauchy's remainder.
To find Lagrange's remainder, we need to use the above mean value theorem with \(g = (1-t)^{n-1}\), which is positive everywhere in \((0, 1)\), and \(f = f^{(n)}(th)\).
Then
\[
	R_n = \frac{h^n}{(n-1)!} f^{(n)}(\theta h) \int_0^1 (1-t)^{n-1}\dd{t}
\]
This integral is simple to find by inspection:
\[
	R_n = \frac{h^n}{(n-1)!} f^{(n)}(\theta h) \frac{1}{n} = \frac{h^n}{n!} f^{(n)}(\theta h)
\]
as required.
