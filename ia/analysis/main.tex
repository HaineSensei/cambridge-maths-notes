\documentclass{article}

\usepackage[UKenglish]{babel}
\usepackage[utf8]{inputenc}
\usepackage[a4paper, margin=20mm]{geometry}
\usepackage{textcomp} % makes the "not defining \perthousand"/"\micro" errors go away by including this first
\usepackage{amsmath}
\usepackage{amssymb}
\usepackage{amsthm}
\usepackage{amsfonts}
\usepackage{wrapfig}
\usepackage{physics}
\usepackage{bm}
\DeclareDocumentCommand\mathbf{m}{\bm{\mathrm{#1}}} % make bold work for greek symbols
\DeclareDocumentCommand\vnabla{}{\nabla} % use non-bold nabla for \grad, \curl etc. Enabled to unify laplacian symbol between vector and scalar forms
\DeclareDocumentCommand\dotproduct{}{\cdot} % use non-bold dot for scalar product to unify notation
\DeclareDocumentCommand\crossproduct{}{\times} % use non-bold dot for scalar product to unify notation
\usepackage{gensymb}
\usepackage{enumerate}
\usepackage{mathtools}
\usepackage{centernot}
\usepackage{relsize}
\usepackage{mathrsfs}
\usepackage{siunitx}
\usepackage{pgfplots}
\pgfplotsset{width=10cm,compat=1.9}
\usepgfplotslibrary{external}
\tikzexternalize[prefix=tikz/]
\usepackage[pdfa]{hyperref}
\hypersetup{
	colorlinks=true,
	linktoc=all,
	linkcolor=black,
}

\numberwithin{equation}{section} % make equations be numbered 1.1 not 1

\newcommand{\tableofcontentsnewpage}{\tableofcontents\newpage}

% create the theorem environments
\theoremstyle{definition}
\newtheorem*{definition}{Definition}

\newtheorem*{claim}{Claim}
\newtheorem*{theorem}{Theorem}
\newtheorem*{proposition}{Proposition}
\newtheorem*{lemma}{Lemma}
\newtheorem*{corollary}{Corollary}

\theoremstyle{remark}
\newtheorem*{note}{Note}
\newtheorem*{remark}{Remark}

\newcommand{\ddempty}{\mathrm{d}}
\newcommand{\dn}[2]{\mathrm{d}^#1#2}
\newcommand{\st}{\text{ s.t. }}
\newcommand{\contradiction}{\(\#\)}
\newcommand{\genset}[1]{\langle{} #1 \rangle}
\newcommand{\nhat}{\vu{n}}
\newcommand{\rdot}{\dot{\vb{r}}}
\newcommand{\rddot}{\ddot{\vb{r}}}
\newcommand{\transpose}{\intercal}
\newcommand{\acts}{\curvearrowright}
\newcommand{\adjugate}[1]{\widetilde{#1}}
\newcommand{\mathhuge}[1]{\mathlarger{\mathlarger{\mathlarger{#1}}}}
\newcommand{\stcomp}[1]{{#1}^c} % consider \complement? Personally I think this looks better, and it's what Wikipedia uses
\newcommand{\prob}[1]{\mathbb{P}\left({#1}\right)}
\newcommand{\psub}[2]{\mathbb{P}_{#1}\left({#2}\right)}
\newcommand{\psubx}[1]{\psub{x}{#1}}
\newcommand{\expect}[1]{\mathbb{E}\left[{#1}\right]}
\newcommand{\esub}[2]{\mathbb{E}_{#1}\left[{#2}\right]}
\newcommand{\esubx}[1]{\esub{x}{#1}}
\newcommand{\Var}[1]{\Varop\left({#1}\right)}
\newcommand{\Cov}[1]{\Covop\left({#1}\right)}
\newcommand{\Corr}[1]{\Corrop\left({#1}\right)}
\newcommand{\convdist}{\xrightarrow{d}}
\newcommand{\convprob}{\xrightarrow{\mathbb{P}}}

\DeclareMathOperator{\vecspan}{span}
\DeclareMathOperator{\HCF}{HCF}
\DeclareMathOperator{\LCM}{LCM}
\DeclareMathOperator{\ord}{ord}
\DeclareMathOperator{\Sym}{Sym}
\DeclareMathOperator{\nullity}{null}
\DeclareMathOperator{\Orb}{Orb}
\DeclareMathOperator{\Stab}{Stab}
\DeclareMathOperator{\ccl}{ccl}
\DeclareMathOperator{\Varop}{Var}
\DeclareMathOperator{\Covop}{Cov}
\DeclareMathOperator{\Corrop}{Corr}

\DeclarePairedDelimiter\ceil{\lceil}{\rceil}
\DeclarePairedDelimiter\floor{\lfloor}{\rfloor}

% for arrows in the middle of the line
\usetikzlibrary{decorations.markings}
\tikzset{->-/.style={decoration={
		markings,
		mark=at position #1 with {\arrow{>}}},postaction={decorate}}}


\title{Analysis}
\author{Cambridge University Mathematical Tripos: Part IA}

\begin{document}
\maketitle

\tableofcontentsnewpage{}

\section{Limits and Convergence: Reviewing Numbers and Sets}
\subsection{Timeline}
\begin{itemize}
	\item (1801--3) Particles were shown to have wave-like properties using Young's double slit experiment.
	\item (1862--4) Electromagnetism was conceived by Maxwell.
	      Light was discovered to be an electromagnetic wave.
	\item (1897) Discovery of the electron by Thomson.
	\item (1900) The Planck law was discovered, which explains black-body radiation.
	\item (1905) The photoelectric effect was discovered by Einstein.
	\item (1909) Wave-light interference patterns were shown to exist with only one photon recorded at a time.
	\item (1911) Rutherford created his atomic model.
	\item (1913) Bohr created his atomic model.
	\item (1923) The Compton experiment showed x-ray scattering off electrons.
	\item (1923--4) De Broglie discovered the concept of wave-particle duality.
	\item (1925--30) The theory of quantum mechanics emerged at this time.
	\item (1927--8) The diffraction experiment was carried out with electrons.
\end{itemize}

\subsection{Particles and Waves in Classical Mechanics}
In classical mechanics, a point-particle is an object with energy and momentum in an infinitesimally small point of space.
Therefore, a particle is determined by the three-dimensional vectors \( \vb x, \vb v = \dot{\vb x} \).
The motion of a particle is governed by Newton's second law,
\[
	m \ddot{\vb x} = \vb F(\vb x, \dot{\vb x})
\]
Solving this equation involves determination of \( \vb x, \dot{\vb x} \) for all \( t > t_0 \), once initial conditions \( \vb x(t_0), \dot{\vb x}(t_0) \) are known.

Waves are classically defined as any real- or complex-valued function with periodicity in time and/or space.
For instance, consider a function \( f \) such that \( f(t + T) = f(t) \), which is a wave with period \( T \).
The frequency \( \nu \) is defined to be \( \frac{1}{T} \), and the angular frequency \( \omega \) is defined as \( 2 \pi \nu = \frac{2\pi}{T} \).
Suppose we have a function in one dimension obeying \( f(x+\lambda) = f(x) \).
This has wavelength \( \lambda \) and wave number \( k = \frac{2\pi}{\lambda} \).

Consider \( f(x) = \exp(\pm i k x) \).
In three dimensions, this becomes \( f(x) = \exp(\pm i \vb k \cdot \vb x) \).
This is called a `plane wave'; the one-dimensional wave number \( k \) has been transformed into a three-dimensional wave vector \( \vb k \).
\( \lambda \) is now defined as \( \frac{2\pi}{\abs{k}} \).

The wave equation in one dimension is
\[
	\pdv[2]{f(x,t)}{t} - c^2 \pdv[2]{f(x,t)}{x} = 0;\quad c \in \mathbb R
\]
The solutions to this equation are
\[
	f_\pm (x,t) = A_\pm \exp(\pm i k x - i \omega t)
\]
where \( \omega = c k; \lambda = \frac{c}{\nu} \).
The two conditions are known as the dispersion relations.
\( A_\pm \) is the amplitude of the waves.

In three dimensions,
\[
	\pdv[2]{f(\vb x,t)}{t} - c^2 \laplacian f(\vb x,t) = 0;\quad c \in \mathbb R
\]
The solution is
\[
	f (\vb x,t) = A \exp(\pm i \vb k \cdot \vb x - i \omega t)
\]
where \( \omega = c \abs{\vb k}; \lambda = \frac{c}{\nu} \).

\begin{note}
	Other kinds of waves are solutions to other governing equations, provided that another dispersion relation \( \omega(\vb k) \) is given.
	Also, for any governing equation linear in \( f \), the superposition principle holds: if \( f_1, f_2 \) are solutions then so is \( f_1 + f_2 \).
\end{note}

\subsection{Black-body Radiation}
Several experiments have shown that light behaves with some particle-like characteristics.
For example, consider a body heated at some temperature \( T \).
Any such body will emit radiation.
The simplest body to study is called a `black-body', which is a totally absorbing surface.
The intensity of light emitted by a black body was modelled as a function of the frequency.
The classical prediction for the spectrum of emitted radiation was that as the frequency increased, the intensity would also increase.
A curve with a clear maximum point was observed.

\section{More on Convergence}
\subsection{Proofs and Non-Proofs Continued}
\begin{claim}
	The solution to the real equation \(x^2-5x+6=0\) is \(x=2\) or \(x=3\).
\end{claim}
\begin{note}
	This is really two assertions:
	\begin{enumerate}[i.]
		\item \(x=2 \lor x=3 \implies x^2 - 5x + 6 = 0\), and
		\item \(x^2 - 5x + 6 = 0 \implies x=2 \lor x=3\)
	\end{enumerate}
	We can denote this using a two-way implication symbol \(\iff\):
	\[
		x=2 \lor x=3 \iff x^2 - 5x + 6 = 0
	\]
\end{note}
\begin{proof}
	We prove case i by expressing the left hand side as a product of factors: \((x-3)(x-2)=0\).
	The other case may be proven using factorisation.
\end{proof}

We can do another kind of proof using \(\iff\) symbols a lot.
However, we need to be absolutely sure that each step really is a bi-implication.
\begin{proof}[Alternative Proof]
	For any real \(x\):
	\begin{align*}
		x^2-5x+6=0 & \iff (x-2)(x-3) = 0       \\
		           & \iff x-2 = 0 \lor x-3 = 0 \\
		           & \iff x=2 \lor x = 3
	\end{align*}
\end{proof}

\begin{claim}
	Every positive real is at least 1.
\end{claim}
\begin{proof}
	Let \(x\) be the smallest positive real.
	We want to prove \(x=1\), so we prove this by contradiction.
	
	Case 1: if \(x < 1\) then \(x^2 < x\) \contradiction{}
	
	Case 2: if \(x > 1\) then \(\sqrt{x} < x\) \contradiction{}
	
	Therefore \(x=1\)
\end{proof}
\begin{note}
	The assertion that there exists a smallest positive real is not justified.
	This means that the proof is invalid in its entirety.
	It is important that every line in a proof must be justified.
\end{note}

\subsection{The Natural Numbers}
Each line in a proof must be justified.
So, in number theory, what are you allowed to assume?
We must begin with a set of axioms.
We define that the natural numbers are a set denoted \(\mathbb N\), that contains an element denoted 1, with an operation \(+1\) satisfying:
\begin{enumerate}
	\item \(\forall n \in \mathbb N, n + 1 \neq 1\)
	\item \(\forall m,n \in \mathbb N, m \neq n \implies m+1 \neq n+1\) (together with the previous rule, this captures the idea that all numbers in \(\mathbb N\) are distinct)
	\item For any property \(p(n)\), if \(p(1)\) is true and \(p(n) \implies p(n+1) \ \forall n \in \mathbb N\), then \(p(n) \ \forall n \in \mathbb N\) (induction axiom).
\end{enumerate}

\noindent This list of rules is known as the Peano axioms.
Note that we did not include 0 in this set.
You can show that the list of natural numbers is complete and has no extras (like the rational number \(3.5\)) by specifying \(p(n)=\) `\(n\) is on the list of natural numbers'.

Note that while numbers are defined as, for example, \(1+1+1+1\), we are free to use whatever names we like, e.g.
4 or the hexadecimal number 0xDEADBEEF = 3735928559.

We may also define our own operations, such as \(+2\), which is defined to be \(+1+1\).
In fact, we can define the operation \(+k\) for any \(k \in \mathbb N\) by stating:
\[
	(n+k)+1 = n+(k+1) \quad(\forall n, k \in \mathbb N)
\]
\noindent and using induction to construct the \(+k\) operator for all \(k\).
We can similarly construct multiplication and exponentiation operators for all natural numbers, although this is omitted here.
We can also prove properties on these operators such as associativity, commutativity and distributivity.

We can also define the \(<\) operator as follows: \(a < b \iff \exists k \in \mathbb N \st a + k = b\).
Of course, we can also prove several properties using this rule, such as transitivity, and the fact that \(a \nless a\), which are omitted here.

\section{Convergence Tests}
\subsection{Differentials and First Order Changes}
Recall that for a function $f(u_1, \dots, u_n)$, we define the differential of $f$, written $\dd{f}$, by
\[ \dd{f} = \frac{\partial f}{\partial u_i} \dd{u}_i \]
noting that the summation convention applies. The $\dd{u}_i$ are called differential forms, which can be thought of as linearly independent objects (if the coordinates $u_1, \dots, u_n$ are independent), i.e. $\alpha_i \dd{u}_i = 0 \implies \alpha_i = 0$ for all $i$. Similarly, if we have a vector $\vb x(u_1, \dots, u_n)$, we define
\[ \dd \vb x = \frac{\partial \vb x}{\partial u_i} \dd{u}_i \]
As an example, let $f(u, v, w) = u^2 + w \sin(v)$. Then
\[ \dd{f} = 2u \dd{u} + w \cos(v) \dd{v} + \sin(v) \dd{w} \]
Similarly, given
\[ \vb x(u, v, w) = \begin{pmatrix}
		u^2 - v^2 \\ w \\ e^v
	\end{pmatrix} \]
we can compute
\[ \dd \vb x = \begin{pmatrix}
		2u \\ 0 \\ 0
	\end{pmatrix} \dd{u} + \begin{pmatrix}
		-2v \\ 0 \\ e^v
	\end{pmatrix} \dd{v} + \begin{pmatrix}
		0 \\ 1 \\ 0
	\end{pmatrix} \dd{w} \]
Differentials encode information about how a function (or vector field) changes when we change the coordinates by a small amount. By calculus,
\[ f(u + \delta u_1, \dots, u_n + \delta u_n) - f(u_1, \dots, u_n) = \frac{\partial f}{\partial u_i} \delta u_i + o(\delta \vb u) \]
So if $\delta f$ denotes the change in $f(u_1, \dots, u_n)$ under this small change in coordinates, we have, to first order,
\[ \delta f \approx \frac{\partial f}{\partial u_i}\delta u_i \]
The analogous result holds for vector vields:
\[ \delta \vb x \approx \frac{\partial \vb x}{\partial u_i}\delta u_i \]

\subsection{Coordinates and Line Elements in $\mathbb R^2$}
We can create multiple different consistent coordinate systems by defining a relationship between them. For example, polar coordinates $(r, \theta)$ and Cartesian coordinates $(x, y)$ can be related by
\[ x = r \cos \theta;\quad y = r \sin \theta \]
Even though this relationship is not bijective (there are multiple polar coordinates mapping to the origin), it's still a useful coordinate system because the vast majority of points work well. Even coordinate systems with a countable amount of badly-behaved points are still useful.

A general set of coordinates $(u, v)$ on $\mathbb R^2$ can be specified by their relationship to the standard Cartesian coordinates $(x, y)$. We must specify smooth, invertible functions $x(u, v)$, $y(u, v)$. We would also like to have a small change in one coordinate system to be equivalent to a small change in the other coordinate system (i.e. the inverse is also smooth). The same principle applies in $\mathbb R^3$ for three coordinates, for example.

Consider the standard Cartesian coordinates in $\mathbb R^2$.
\[ \vb x(x, y) = \begin{pmatrix}
		x \\ y
	\end{pmatrix} = x \vb e_x + y \vb e_y \]
Note that $\{\vb e_x, \vb e_y\}$ are orthonormal, and point in the same direction regardless of the value of $\vb x$: $\vb e_x$ points in the direction of changing $x$ with $y$ held constant, for example. Equivalently,
\[ \vb e_x = \frac{\frac{\partial}{\partial x} \vb x(x, y)}{\abs{\frac{\partial}{\partial x} \vb x(x, y)}};\quad \vb e_y = \frac{\frac{\partial}{\partial y} \vb x(x, y)}{\abs{\frac{\partial}{\partial y} \vb x(x, y)}} \]
Note that
\[ \dd \vb x = \frac{\partial \vb x}{\partial x}\dd{x} + \frac{\partial \vb x}{\partial y} \dd{y} = \dd{x} \,\vb e_x + \dd{y} \,\vb e_y \]
In other words, when applying the change in coordinate $x \mapsto x + \delta x$, the vector changes (to first order) to $\vb x \mapsto \vb x + \delta x \vb e_x$. In fact, in the case of Cartesian coordinates, this change is precisely correct for any size of $\delta$, since the coordinate basis vectors are the same everywhere. We call $\dd \vb x$ the line element; it tells us how small changes in coordinates produce changes in position vectors.

Now, let us consider polar coordinates in two-dimensional space. We can use the same idea as before, giving
\[ \vb e_r = \frac{\frac{\partial}{\partial r} \vb x(r, \theta)}{\abs{\frac{\partial}{\partial r} \vb x(r, \theta)}} = \begin{pmatrix}
		\cos\theta \\ \sin\theta
	\end{pmatrix};\quad \vb e_\theta = \frac{\frac{\partial}{\partial \theta} \vb x(r, \theta)}{\abs{\frac{\partial}{\partial \theta} \vb x(r, \theta)}} = \begin{pmatrix}
		-\sin\theta \\ \cos\theta
	\end{pmatrix} \]
Therefore, we have
\[ \vb x(r, \theta) = \begin{pmatrix}
		r \cos\theta \\ r \sin\theta
	\end{pmatrix} = r\vb e_r \]
Note that $\{\vb e_r, \vb e_\theta\}$ are also orthonormal at each $(r, \theta)$, but their exact values are not the same everywhere. Since the basis vectors are orthogonal, we can call $r$ and $\theta$ orthogonal curvilinear coordinates. Also, we can compute the line element $\dd \vb x$ as
\[ \dd \vb x = \frac{\partial \vb x}{\partial r} \dd{r} + \frac{\partial \vb x}{\partial \theta} \dd \theta = \begin{pmatrix}
		\cos \theta \\ \sin \theta
	\end{pmatrix} \dd{r} + \begin{pmatrix}
		-r \sin \theta \\ r \cos \theta
	\end{pmatrix} \dd \theta = \dd{r} \, \vb e_r + r\, \dd \theta \, \vb e_\theta \]
We see that a change in $\theta$ produces (up to first order) a change $\vb x \mapsto \vb x + r \,\delta \theta \,\vb e_\theta$, a change proportional to $r$. So a small change in $\theta$ could cause quite a large change in Cartesian coordinates.

\subsection{Orthogonal Curvilinear Coordinates}
We say that $(u, v, w)$ are a set of orthogonal curvilinear coordinates if the vectors
\[ \vb e_u = \frac{\frac{\partial \vb x}{\partial u}}{\abs{\frac{\partial \vb x}{\partial u}}};\quad \vb e_v = \frac{\frac{\partial \vb x}{\partial v}}{\abs{\frac{\partial \vb x}{\partial v}}};\quad \vb e_w = \frac{\frac{\partial \vb x}{\partial w}}{\abs{\frac{\partial \vb x}{\partial w}}} \]
form a right-handed, orthonormal basis for each $(u, v, w)$; but not necessarily the same basis over the entire vector field. It is standard to write
\[ h_u = \abs{\frac{\partial \vb x}{\partial u}};\quad h_v = \abs{\frac{\partial \vb x}{\partial v}};\quad h_w = \abs{\frac{\partial \vb x}{\partial w}} \]
We call $h_u, h_v, h_w$ the scale factors.  Note that the line element is
\begin{align*}
	\dd \vb x & = \frac{\partial \vb x}{\partial u}\dd{u} + \frac{\partial \vb x}{\partial v}\dd{v} + \frac{\partial \vb x}{\partial w} \dd{w} \\
	          & = h_u \vb e_u \dd{u} + h_v \vb e_v \dd{v} + h_w \vb e_w \dd{w}
\end{align*}
So the scale factors show how first-order changes in the coordinates are scaled into changes in $\vb x$.

\subsection{Cylindrical Polar Coordinates}
We define $(\rho, \phi, z)$ by
\[ \vb x(\rho, \phi, z) = \begin{pmatrix}
		\rho \cos \phi \\
		\rho \sin \phi \\
		z
	\end{pmatrix} \]
where $0 \leq \rho; 0 \leq \phi < 2 \pi; z \in \mathbb R$. So we can find
\[ \vb e_\rho = \begin{pmatrix}
		\cos \phi \\ \sin \phi \\ 0
	\end{pmatrix};\quad \vb e_\phi = \begin{pmatrix}
		-\sin \phi \\ \cos \phi \\ 0
	\end{pmatrix};\quad \vb e_z = \begin{pmatrix}
		0 \\ 0 \\ 1
	\end{pmatrix} \]
The scale factors are
\[ h_\rho = 1;\quad h_\phi = \rho;\quad h_z = 1 \]
The line element is
\[ \dd \vb x = \dd \rho \, \vb e_\rho + \rho \, \dd \phi \, \vb e_\phi + \dd{z} \, \vb e_z \]
Note that
\[ \vb x = \rho \begin{pmatrix}
		\cos \phi \\ \sin \phi \\ 0
	\end{pmatrix} + z \begin{pmatrix}
		0 \\ 0 \\ 1
	\end{pmatrix} = \rho \vb e_\rho + z \vb e_z \]

\subsection{Spherical Polar Coordinates}
We define $(r, \theta, \phi)$ by
\[ \vb x(r, \theta, \phi) = \begin{pmatrix}
		r \cos \phi \sin \theta \\
		r \sin \phi \sin \theta \\
		r \cos \theta
	\end{pmatrix} \]
where $0 \leq r; 0 \leq \theta < 2 \pi; 0 \leq \phi < 2 \pi$. So we can find
\[ \vb e_r = \begin{pmatrix}
		\cos \phi \sin \theta \\ \sin \phi \sin \theta \\ \cos \theta
	\end{pmatrix};\quad \vb e_\theta = \begin{pmatrix}
		\cos \phi \cos \theta \\ \sin \phi \cos \theta \\ -\sin \theta
	\end{pmatrix};\quad \vb e_\phi = \begin{pmatrix}
		-\sin \phi \\ \cos \phi \\ 0
	\end{pmatrix} \]
The scale factors are
\[ h_r = 1;\quad h_\theta = r;\quad h_\phi = r \sin \theta \]
The line element is
\[ \dd \vb x = \dd{r} \, \vb e_r + r \, \dd \theta \, \vb e_\theta + r \sin \theta \, \dd \phi \, \vb e_\phi \]
Note that
\[ \vb x = r \begin{pmatrix}
		\cos \phi \sin \theta \\ \sin \phi \sin \theta \\ \cos \theta
	\end{pmatrix} = r \vb e_r \]

\section{More Convergence Tests}
\subsection{Definition}
For \(f \colon \mathbb R^3 \to \mathbb R\), we define the gradient of \(f\), written \(\grad f\), by
\begin{equation}
	f(\vb x + \vb h) = f(\vb x) + \grad f(\vb x) \cdot \vb h + o(\vb h)
	\tag{\(\ast\)}
\end{equation}
as \(\abs{\vb h} \to 0\). The directional derivative of \(f\) in the direction \(\vb v\), denoted by \(D_{\vb v} f\) or \(\frac{\partial f}{\partial \vb v}\), is defined by
\[ D_{\vb v} f(\vb x) = \lim_{t \to 0} \frac{f(\vb x + t\vb v) - f(\vb x)}{t} \]
Alternatively,
\begin{equation}
	f(\vb x + t\vb v) = f(\vb x) + t D_{\vb v}f(\vb x) + o(t)
	\tag{\(\dagger\)}
\end{equation}
as \(t \to 0\). Setting \(\vb h = t\vb v\) in \((\ast)\), we have
\[ f(\vb x + t\vb v) = f(\vb x) + t \grad f(\vb x) \cdot \vb v + o(t) \]
This gives another way to interpret the gradient of \(f\). Comparing this result to \((\dagger)\), we see that
\[ D_{\vb v} f = \vb v \cdot \grad f \]
By the Cauchy-Schwarz inequality, the dot product is maximised when the two vectors are parallel. Hence, the directional derivative is maximised when \(\vb v\) points in the direction of \(\grad f\). So \(\grad f\) points in the direction of greatest increase of \(f\). Similarly, \(-\grad f\) points in the direction of greatest decrease of \(f\). For example, suppose \(f(x) = \frac{1}{2}\abs{\vb x}^2\). Then
\[ f(\vb x + \vb h) = \frac{1}{2}(\vb x + \vb h)\cdot (\vb x + \vb h) = \frac{1}{2}\abs{\vb x}^2 + \frac{1}{2}(2\vb x \cdot \vb h) + \frac{1}{2}\abs{\vb h}^2 = f(\vb x) + \vb x \cdot \vb h + o(\vb h) \]
Hence \(\grad f(\vb x) = \vb x\).

\subsection{Gradient on Curves}
Suppose we have a curve \(t \mapsto \vb x(t)\). How does some function \(f\) change when moving along the curve? We will write \(F(t) = f(\vb x(t)), \delta \vb x = \vb x(t + \delta t) - \vb x(t)\).
\begin{align*}
	F(t + \delta t) & = f(\vb x(t + \delta t))                                               \\
	                & = f(\vb x(t) + \delta \vb x)                                           \\
	                & = f(\vb x(t)) + \grad f(\vb x(t)) \cdot \delta \vb x + o(\delta \vb x) \\
	\intertext{Since \(\delta \vb x = \vb x'(t) \,\delta t + o(\delta t)\), we have}
	F(t + \delta t) & = F(t) + \vb x'(t) \cdot \grad f(\vb x(t)) \,\delta t + o(\delta t)
\end{align*}
In other words,
\[ \frac{\dd{F}}{\dd{t}} = \frac{\dd}{\dd{t}}f(\vb x(t)) = \frac{\dd \vb x}{\dd{t}} \cdot \grad f(\vb x(t)) \]

\subsection{Gradient on Surfaces}
Suppose we have a surface \(S\) in \(\mathbb R^3\) defined implicitly by
\[ S = \{ \vb x \in \mathbb R^3 : f(\vb x) = 0 \} \]
If \(t \mapsto \vb x(t)\) is any curve in \(S\), then \(f(\vb x(t)) = 0\) everywhere. So
\[ 0 = \frac{\dd}{\dd{t}}f(\vb x(t)) = \grad f(\vb x(t)) \cdot \frac{\dd \vb x}{\dd{t}} \]
So \(\grad f(\vb x(t))\), the gradient, is orthogonal to \(\frac{\dd \vb x}{\dd{t}}\), the tangent vector of any chosen curve in \(S\). So \(\grad f(\vb x(t))\) is normal to the surface.

\subsection{Coordinate-Independent Representation}
If we are working in an orthogonal curvilinear coordinate system \((u, v, w)\), it is not immediately clear how to compute \(\grad f\), since we need to represent this arbitrary perturbation \(\vb h\) using \((u, v, w)\). In Cartesian coordinates it is simple; to represent the change \(\vb x \mapsto \vb x + \vb h\) we simply add the components of \(\vb x\) and \(\vb h\).
\begin{align*}
	f(\vb x + \vb h) & = f((x + h_1, y + h_2, z + h_3))                                                                                                  \\
	                 & = f(\vb x) + \frac{\partial f}{\partial x} h_1 + \frac{\partial f}{\partial y} h_2 + \frac{\partial f}{\partial z} h_3 + o(\vb h) \\
	                 & = f(\vb x) + \begin{pmatrix}
		\partial f / \partial x \\ \partial f / \partial y \\ \partial f / \partial z
	\end{pmatrix} \cdot h + o(\vb h)                                                                         \\
\end{align*}
So we have
\[ \implies \grad f = \begin{pmatrix}
		\partial f / \partial x \\ \partial f / \partial y \\ \partial f / \partial z
	\end{pmatrix} \]
Or, using suffix notation,
\[ \grad f = \vb e_i \frac{\partial f}{\partial x_i};\quad [\grad f]_i = \frac{\partial f}{\partial x_i} \]
We see that this \(\grad\) is a kind of vector differential operator. In Cartesian coordinates,
\[ \grad = \vb e_x \frac{\partial}{\partial x} + \vb e_y \frac{\partial}{\partial y} + \vb e_z \frac{\partial}{\partial z} \equiv \vb e_i \frac{\partial}{\partial x_i} \]
From our previous example,
\[ f(\vb x) = \frac{1}{2}(x^2 + y^2 + z^2) = \frac{1}{2}\abs{\vb x}^2 \]
\begin{align*}
	[\grad f]_i & = \frac{\partial}{\partial x_i}\left[ \frac{1}{2} x_j x_j \right] \\
	            & = \frac{1}{2} \left[ \delta_{ij} x_j + x_j \delta_{ij} \right]    \\
	            & = x_i                                                             \\
	\grad f     & = \vb e_i x_i
\end{align*}
Let us return back to computing the gradient in the general case. Recall that in Cartesian coordinates, the line element is simple:
\[ \dd \vb x = \dd{x}_i \vb e_i \]
And also, if we have a function on \(\mathbb R^3\) such as \(f(x, y, z)\), it has the differential
\[ \dd{f} = \frac{\partial f}{\partial x_i}\dd{x}_i \]
Then,
\begin{align*}
	\grad f \cdot \dd \vb x & = \left( \vb e_i \frac{\partial f}{\partial x_i} \right) \cdot \left( \vb e_j \dd{x}_j \right) \\
	                        & = \frac{\partial f}{\partial x_i} \left( \vb e_i \cdot \vb e_j \right) \dd{x}_j                \\
	                        & = \frac{\partial f}{\partial x_i} \delta_{ij} \dd{x}_j                                         \\
	                        & = \frac{\partial f}{\partial x_i} \dd{x}_i                                                     \\
	                        & = \dd{f}
\end{align*}
In other words, in \textit{any} set of coordinates,
\[ \grad f \cdot \dd \vb x = \dd{f} \]

\subsection{Computing the Gradient Vector}
\begin{proposition}
	If \((u, v, w)\) are orthogonal curvilinear coordinates, and \(f\) is a function of the position vector \((u, v, w)\), then
	\[ \grad f = \frac{1}{h_u}\frac{\partial f}{\partial u}\vb e_u + \frac{1}{h_v}\frac{\partial f}{\partial v}\vb e_v + \frac{1}{h_w}\frac{\partial f}{\partial w}\vb e_w \]
\end{proposition}
\begin{proof}
	If \(f = f(u, v, w)\) and \(\vb x = \vb x(u, v, w)\), then
	\[ \dd{f} = \frac{\partial f}{\partial u}\dd{u} + \frac{\partial f}{\partial v}\dd{v} + \frac{\partial f}{\partial w}\dd{w} \]
	\[ \dd{x} = h_u \dd{u} \vb e_u + h_v \dd{v} \vb e_v + h_w \dd{w} \vb e_w \]
	Using the above result, we have
	\[ \grad f \cdot \dd \vb x = \dd{f} \]
	\[ \left( (\grad f)_u \vb e_u + (\grad f)_v \vb e_v + (\grad f)_w \vb e_w \right) \cdot \left( h_u \dd{u} \vb e_u + h_v \dd{v} \vb e_v + h_w \dd{w} \vb e_w \right) = \frac{\partial f}{\partial u}\dd{u} + \frac{\partial f}{\partial v}\dd{v} + \frac{\partial f}{\partial w}\dd{w} \]
	\[  (\grad f)_u h_u \dd{u} + (\grad f)_v h_v \dd{v} + (\grad f)_w h_w \dd{w} = \frac{\partial f}{\partial u}\dd{u} + \frac{\partial f}{\partial v}\dd{v} + \frac{\partial f}{\partial w}\dd{w} \]
	Since \(u, v, w\) are independent coordinates, \(\dd{u}, \dd{v}, \dd{w}\) are linearly independent. So we can simply compare coefficients, getting
	\[ \grad f = \frac{1}{h_u}\frac{\partial f}{\partial u}\vb e_u + \frac{1}{h_v}\frac{\partial f}{\partial v}\vb e_v + \frac{1}{h_w}\frac{\partial f}{\partial w}\vb e_w \]
	as required.
\end{proof}
\noindent In cylindrical polar coordinates, we have
\[ \grad f = \frac{\partial f}{\partial \rho} \vb e_\rho + \frac{1}{\rho} \frac{\partial f}{\partial \phi} \vb e_\phi + \frac{\partial f}{\partial z} \vb e_z \]
In spherical polar coordinates, we have
\[ \grad f = \frac{\partial f}{\partial r} \vb e_r + \frac{1}{r} \frac{\partial f}{\partial \theta} \vb e_\theta + \frac{1}{r\sin\theta} \frac{\partial f}{\partial \phi} \vb e_\phi \]
Then using the familiar example \(f(\vb x) = \frac{1}{2}\abs{\vb x}^2\), we have
\[
	f = \begin{cases}
		\frac{1}{2}(x^2 + y^2 + z^2) & \text{in Cartesian coordinates}         \\
		\frac{1}{2}(\rho^2 + z^2)    & \text{in cylindrical polar coordinates} \\
		\frac{1}{2}r^2               & \text{in spherical polar coordinates}   \\
	\end{cases}
\]
Then we can check the value of \(\grad f\) in these different coordinate systems.
\begin{align*}
	\grad f & = \begin{cases}
		x \vb e_x + y \vb e_y + z \vb e_z & \text{in Cartesian coordinates}         \\
		\rho \vb e_\rho + z \vb e_z       & \text{in cylindrical polar coordinates} \\
		r \vb e_r                         & \text{in spherical polar coordinates}   \\
	\end{cases} \\
	        & = \vb x
\end{align*}

\section{Absolute Convergence}
\subsection{Change of Variables}
We can transform derivatives into different coordinate systems to make problems easier to solve. For example, let \(f(x, y)\) be some function with a Cartesian coordinate input. We can rewrite it in terms of polar coordinates \((r, \theta)\). First, rewrite \(f\) as:
\[ f(x(r, \theta), y(r, \theta)) \]
then we can write the derivatives.
\[ \frac{\partial f}{\partial r} = \frac{\partial f}{\partial x}\frac{\partial x}{\partial r} + \frac{\partial f}{\partial y}\frac{\partial y}{\partial r} \]
We can do similar evaluations for \(\frac{\partial f}{\partial \theta}\), for example.

\subsection{Implicit Differentiation}
Consider some surface defined by \(f(x, y, z) = c\). Then \(f\) implicitly defines functions such as \(z(x, y)\) (provided the function is well-behaved). We can find, for example, \(\eval{\frac{\partial z}{\partial x}}_y\) by using the multivariate chain rule in three dimensions.

\[
	\eval{\frac{\partial f}{\partial x}}_y =
	\eval{\frac{\partial f}{\partial x}}_{yz} \underbrace{\eval{\frac{\partial x}{\partial x}}_{y}}_{\mathclap{=1}} +
	\eval{\frac{\partial f}{\partial y}}_{xz} \underbrace{\eval{\frac{\partial y}{\partial x}}_{y}}_{\mathclap{=0}} +
	\eval{\frac{\partial f}{\partial z}}_{xy} \eval{\frac{\partial z}{\partial x}}_{y}
\]
Note that the \(\frac{\partial y}{\partial x}\) term is zero because we hold \(y\) to be fixed. Simplifying, we get
\[
	\eval{\frac{\partial f}{\partial x}}_y =
	\eval{\frac{\partial f}{\partial x}}_{yz} +
	\eval{\frac{\partial f}{\partial z}}_{xy} \eval{\frac{\partial z}{\partial x}}_{y}
\]
The left hand side is zero because on the surface \(z(x, y)\), \(f\) is always equivalent to \(c\) so there is never any \(\delta f\). The \(\eval{\frac{\partial f}{\partial x}}_{yz}\) term, however, is not zero in general because we are not going across the \(z(x, y)\) surface --- just parallel to the \(x\) axis, because we fixed both \(y\) and \(z\). Hence,
\[ \eval{\frac{\partial z}{\partial x}}_y = \frac{-\eval{\frac{\partial f}{\partial x}}_{yz}}{\eval{\frac{\partial f}{\partial z}}_{xy}} \]

\subsection{Reciprocal Rule}
The reciprocal rule for derivatives applies also to partial derivatives so long as the same variables are held fixed. For example, given the function \(f(x(r, \theta), y(r, \theta))\), we have
\[ \eval{\frac{\partial r}{\partial x}}_y = \frac{1}{\eval{\frac{\partial x}{\partial r}}_y} \]
But
\[ \frac{\partial r}{\partial x} \neq \frac{1}{\frac{\partial x}{\partial r}} \]
because the left hand side holds \(y\) constant and the right hand side holds \(\theta\) constant.

\subsection{Differentiating an Integral with Respect to a Parameter}
Consider a family of function \(f(x; \alpha)\) where \(\alpha\) is some parameter. We can say that \(\alpha\) parametrises \(f\). An example of a parametrised function is the logarithm; \(f(x; \alpha) = \log_\alpha x\). We define
\[ I(\alpha) = \int_{a(\alpha)}^{b(\alpha)} f(x; \alpha) \ \dd{x} \]
So, what is \(\frac{\dd{I}}{\dd \alpha}\)? By definition, we have
\begin{align*}
	\frac{\dd{I}}{\dd \alpha} & = \lim_{\delta \alpha \to 0} \frac{I(\alpha + \delta \alpha) - I(\alpha)}{\delta \alpha}                                                                                                                                                                                                                                                  \\
	                          & = \lim_{\delta \alpha \to 0} \frac{1}{\delta\alpha} \left[ \int_{a(\alpha + \delta\alpha)}^{b(\alpha + \delta\alpha)} f(x; \alpha + \delta\alpha)\ \dd{x} - \int_{a(\alpha)}^{b(\alpha)} f(x; \alpha)\ \dd{x} \right]                                                                                                                     \\
	                          & = \lim_{\delta \alpha \to 0} \frac{1}{\delta\alpha} \left[ \int_{a(\alpha)}^{b(\alpha)} f(x; \alpha + \delta\alpha) - f(x; \alpha)\ \dd{x} - \int_{a(\alpha)}^{a(\alpha + \delta)} f(x; \alpha + \delta \alpha)\ \dd{x} + \int_{b(\alpha)}^{b(\alpha + \delta)} f(x; \alpha + \delta \alpha)\ \dd{x} \right]                              \\
	                          & = \int_{a(\alpha)}^{b(\alpha)} \lim_{\delta \alpha \to 0} \frac{f(x; \alpha + \delta\alpha) - f(x; \alpha)}{\delta\alpha}\ \dd{x} - f(a; \alpha) \lim_{\delta \alpha \to 0} \frac{a(\alpha + \delta\alpha) - a(\alpha)}{\delta\alpha} + f(b; \alpha) \lim_{\delta \alpha \to 0} \frac{b(\alpha + \delta\alpha) - b(\alpha)}{\delta\alpha} \\
\end{align*}
Therefore:
\[ \frac{\dd{I}}{\dd \alpha} = \frac{\dd}{\dd \alpha} \int_{a(\alpha)}^{b(\alpha)} f(x; \alpha) \ \dd{x} = \int_{a(\alpha)}^{b(\alpha)} \frac{\partial f}{\partial \alpha} \ \dd{x} + f(b; \alpha) \frac{\dd{b}}{\dd \alpha} - f(a; \alpha) \frac{\dd{a}}{\dd \alpha} \]

\section{Continuity}
\subsection{Discrete Distributions}
In a discrete probability distribution on a probability space \((\Omega, \mathcal F, \mathbb P)\), \(\Omega\) is either finite or countable, i.e.\ \(\Omega = \{ \omega_1, \omega_2, \dots \}\), and as stated before, \(\mathcal F\) is the power set of \(\Omega\).
If we know \(\prob{\{\omega_i\}}\), then this completely determines \(\mathbb P\).
Indeed, let \(A \subseteq \Omega\), then
\[
	\prob{A} = \prob{\bigcup_{i \colon \omega_i \in A} \{ \omega_i \}} = \sum_{i \colon \omega_i \in A}\prob{\{\omega_i\}}
\]
by countable additivity.
We will see later that this is not true if \(\Omega\) is uncountable.
We write \(p_i = \prob{\{\omega_i\}}\), and we then call this a discrete probability distribution.
It has the following key properties:
\begin{itemize}
	\item \(p_i \geq 0\)
	\item \(\sum_i p_i = 1\)
\end{itemize}

\subsection{Bernoulli Distribution}
We model the outcome of a test with two outcomes (e.g.
the toss of a coin) with the Bernoulli distribution.
Let \(\Omega = \{ 0, 1 \}\).
We will denote \(p = p_1\), then clearly \(p_0 = 1 - p\).

\subsection{Binomial Distribution}
The binomial distribution \(B\) has parameters \(N \in \mathbb Z^+, p \in [0, 1]\).
This distribution models a sequence of \(N\) independent Bernoulli distributions of parameter \(p\).
We then count the amount of `successes', i.e.\ trials in which the result was 1.
\(\Omega = \{ 0, 1, \dots, N \}\).
\[
	\prob{\{ k \}} = p_k = \binom{N}{k}p^k(1-p)^{N-k}
\]

\subsection{Multinomial Distribution}
The multinomial distribution is a generalisation of the binomial distribution.
\(M\) has parameters \(N \in \mathbb Z^+\) and \(p_1, p_2, \dots \in [0, 1]\) where \(\sum_{i=1}^k p_i = 1\).
This models a sequence of \(N\) independent trials in which a number from 1 to \(N\) is selected, where the probability of selecting \(i\) is \(p_i\).
\(\Omega = \{ (n_1, \dots, n_k) \in \mathbb N^k \colon \sum_{i=1}^k n_i = N \}\), in other words, ordered partitions of \(N\).
Therefore
\[
	\prob{n_1 \text{ outcomes had value 1}, \dots, n_k \text{ outcomes had value }k} = \prob{(n_1, \dots, n_k)} = \binom{N}{n_1,\dots,n_k}p_1^{n_1}\dots p_k^{n_k}
\]

\subsection{Geometric Distribution}
Consider a Bernoulli distribution of parameter \(p\).
The geometric distribution models running this trial many times independently until the first `success' (i.e.\ the first result of value 1).
Then \(\Omega = \{ 1, 2, \dots \} = \mathbb Z^+\).
Then
\[
	p_k = (1-p)^{k-1}p
\]
We can compute the infinite geometric series \(\sum p_k\) which gives 1.
We could alternatively model the distribution using \(\Omega' = \{ 0, 1, \dots \} = \mathbb N\) which records the amount of failures before the first success.
Then
\[
	p_k' = (1-p)^k p
\]
Again, the sum converges to 1.

\subsection{Poisson Distribution}
This is used to model the number of occurences of an event in a given interval of time.
\(\Omega = \{ 0, 1, 2, \dots \} = \mathbb N\).
This distribution has one parameter \(\lambda \in \mathbb R\).
We have
\[
	p_k = e^{-\lambda} \frac{\lambda^k}{k!}
\]
Then
\[
	\sum_{k=0}^\infty p_k = e^{-\lambda}  \sum_{k=0}^\infty \frac{\lambda^k}{k!} = e^{-\lambda} \cdot e^{\lambda} = 1
\]
Suppose customers arrive into a shop during the time interval \([0, 1]\).
We will subdivide \([0, 1]\) into \(N\) intervals \(\left[ \frac{i-1}{N}, \frac{i}{N} \right]\).
In each interval, a single customer arrives with probability \(p\), independent of other time intervals.
In this example,
\[
	\prob{k \text{ customers arrive}} = \binom{N}{k} p^k (1-p)^{N-k}
\]
Let \(p = \frac{\lambda}{N}\) for \(\lambda > 0\).
We will show that as \(N \to \infty\), this binomial distribution converges to the Poisson distribution.
\begin{align*}
	\binom{N}{k} p^k (1-p)^{N-k} & = \frac{N!}{k!(N-k)!} \left( \frac{\lambda}{n} \right)^k \cdot \left( 1 - \frac{\lambda}{n} \right)^{N-k} \\
	                             & = \frac{\lambda_k}{k!} \cdot \frac{N!}{N^k(N-k)!} \cdot \left( 1 - \frac{\lambda}{N} \right)^{N-k}        \\
	                             & \to \frac{\lambda_k}{k!} \cdot 1 \cdot e^{-\lambda}
\end{align*}
which matches the Poisson distribution.

\subsection{Random Variables}
\begin{definition}
	Consider the probability space \((\Omega, \mathcal F, \mathbb P)\).
	A random variable \(X\) is a function \(X \colon \Omega \to \mathbb R\) satisfying
	\[
		\{ \omega \in \Omega \colon X(\omega) \leq x \} \in \mathcal F
	\]
	for any given \(x\).
\end{definition}
\noindent Suppose \(A \subseteq \mathbb R\).
Then typically we write
\[
	\{ X \in A \} = \{ \omega \colon X(\omega) \in A \}
\]
as shorthand.
Given \(A \in \mathcal F\), we define the indicator of \(A\) to be
\[
	1_A(\omega) = 1(\omega \in A) = \begin{cases}
		1 & \text{if } \omega \in A \\
		0 & \text{otherwise}
	\end{cases}
\]
Because \(A \in \mathbb F\), \(1_A\) is a random variable.
Suppose \(X\) is a random variable.
We define probability distribution function of \(X\) to be
\[
	F_X \colon \mathbb R \to [0, 1];\quad F_X(x) = \prob{X \leq x}
\]
\begin{definition}
	\((X_1, \dots, X_n)\) is called a random variable in \(\mathbb R^n\) if \((X_1, \dots, X_n) \colon \Omega \to \mathbb R^n\), and for all \(x_1, \dots, x_n \in \mathbb R\) we have
	\[
		\{ X_1 \leq x_1, \dots, X_n \leq x_n \} = \{ \omega \colon X_1(\omega) \leq x_1, \dots, X_n(\omega) \leq x_n \} \in \mathcal F
	\]
\end{definition}
\noindent This definition is equivalent to saying that \(X_1, \dots, X_n\) are all random variables in \(\mathbb R\).
Indeed,
\[
	\{ X_1 \leq x_1, \dots, X_n \leq x_n \} = \{ X_1 \leq x_1 \} \cap \dots \cap \{ X_n \leq x_n \}
\]
which, since \(\mathcal F\) is a \(\sigma\)-algebra, is an element of \(\mathcal F\).

\section{Limit of a Function}
\subsection{Definition and Example}
\begin{definition}
	A random variable \(X\) is called discrete if it takes values in a countable set.
	Suppose \(X\) takes values in the countable set \(S\).
	For every \(x \in S\), we write
	\[
		p_x = \prob{X = x} = \prob{\{ \omega \colon X(\omega) = x \}}
	\]
	We call \((p_x)_{x \in S}\) the probability mass function of \(X\), or the distribution of \(X\).
	If \((p_x)\) is Bernoulli for example, then we say that \(X\) is a Bernoulli (or such) random variable, or that \(X\) has the Bernoulli distribution.
\end{definition}
\begin{definition}
	Suppose \(X_1, \dots, X_n\) are discrete random variables taking variables in \(S_1, \dots, S_n\).
	We say that the random variables \(X_1, \dots, X_n\) are independent if
	\[
		\prob{X_1 = x_1, \dots, X_n = x_n} = \prob{X_1 = x_1} \cdots \prob{X_n = x_n}\quad \forall x_1 \in S_1, \dots, x_n \in S_n
	\]
\end{definition}
\noindent As an example, suppose we toss a \(p\)-biased coin \(n\) times independently.
Let \(\Omega = \{ 0, 1 \}^n\).
For every \(\omega \in \Omega\),
\[
	p_\omega = \prod_{k=1}^n p^{\omega_k} (1-p)^{1-\omega_k};\quad \text{where we write } \omega = (\omega_1, \dots, \omega_n)
\]
We define a set of discrete random variables \(X_k(\omega) = \omega_k\).
Then \(X_k\) gives the output of the \(k\)th toss.
We have
\[
	\prob{X_k = 1} = \prob{\omega_k = 1} = p;\quad \prob{X_k = 0} = \prob{\omega_k = 0} = 1-p
\]
So \(X_k\) has the Bernoulli distribution with parameter \(p\).
We can also show that the \(X_i\) are independent.
Let \(x_1, \dots, x_n \in \{ 0, 1 \}\).
Then
\begin{align*}
	\prob{X_1 = x_1, \dots, X_n = x_n} & = \prob{\omega = (x_1, \dots, x_n)}   \\
	                                   & = p_{(x_1, \dots, x_n)}               \\
	                                   & = \prod_{k=1}^N p^{x_k} (1-p)^{1-x_k} \\
	                                   & = \prod_{k=1}^N \prob{X_k = x_k}
\end{align*}
as required.
Now, we define \(S_n(\omega) = X_1(\omega) + \dots + X_n(\omega)\).
This is the number of heads in \(N\) tosses.
So \(S_n \colon \Omega \to \{ 0, \dots, N \}\), and
\[
	\prob{S_n = k} = \binom{n}{k} p^k (1-p)^{n-k}
\]
So \(S_n\) has the binomial distribution with parameters \(n\) and \(p\).

\subsection{Expectation}
Let \((\Omega, \mathcal F, \mathbb P)\) be a probability space such that \(\Omega\) is countable.
Let \(X \colon \Omega \to \mathbb R\) be a random variable, which is necessarily discrete.
We say that \(X\) is non-negative if \(X \geq 0\).
We define the expectation of \(X\) to be
\[
	\expect{X} = \sum_\omega X(\omega) \cdot \prob{\{ \omega \}}
\]
We will write
\[
	\Omega_X = \{ X(\omega) \colon \omega \in \Omega \}
\]
So
\[
	\Omega = \bigcup_{x \in \Omega_X} \{ X = x \}
\]
So we have partitioned \(\Omega\) using \(X\).
Note that
\[
	\expect{X} = \sum_\omega X(\omega) \prob{\{\omega\}} = \sum_{x \in \Omega_X} \sum_{\omega \in \{ X = x\}} X(\omega) \prob{\{\omega\}} = \sum_{x \in \Omega_X} \sum_{\omega \in \{ X = x\}} x \prob{\{\omega\}} = \sum_{x \in \Omega_X} x\prob{\{X = x \}}
\]
which matches the more familiar definition of the expectation; the average of the values taken by \(X\), weighted by the probability of the event occcuring.
So
\[
	\expect{X} = \sum_{x \in \Omega_X} x p_x
\]

\subsection{Expectation of Binomial Distribution}
Let \(X \sim \text{Bin}(N, p)\).
Then
\[
	\forall k = 0, \dots, N,\quad \prob{X = k} = \binom{N}{k} p^k (1-p)^{N-k}
\]
So using the second definition,
\begin{align*}
	\expect{X} & = \sum_{k=0}^N k \prob{X = k}                                                         \\
	           & = \sum_{k=0}^N k \binom{n}{k} p^k (1-p)^{N-k}                                         \\
	           & = \sum_{k=0}^N \frac{k \cdot N!}{k! \cdot (N-k)!} p^k (1-p)^{N-k}                     \\
	           & = \sum_{k=1}^N \frac{(N-1)! \cdot N \cdot p}{(k-1)! \cdot (N-k)!} p^{k-1} (1-p)^{N-k} \\
	           & = Np \sum_{k=1}^N \binom{N-1}{k-1} p^{k-1} (1-p)^{N-k}                                \\
	           & = Np \sum_{k=0}^{N-1} \binom{N-1}{k} p^{k} (1-p)^{N-1-k}                              \\
	           & = Np (p + 1 - p)^{N-1}                                                                \\
	           & = Np
\end{align*}

\subsection{Expectation of Poisson Distribution}
Let \(X \sim \text{Poi}(\lambda)\), so
\[
	\prob{X = k} = e^{-\lambda} \frac{\lambda^k}{k!}
\]
Hence
\begin{align*}
	\expect{X} & = \sum_{k=0}^\infty k e^{-\lambda} \frac{\lambda^k}{k!}              \\
	           & =\sum_{k=1}^\infty e^{-\lambda} \frac{\lambda^{k-1} \lambda}{(k-1)!} \\
	           & = e^{-\lambda} \cdot e^{\lambda} \cdot \lambda                       \\
	           & = \lambda
\end{align*}

\subsection{Expectation of a General Random Variable}
Let \(X\) be a general (not necessarily non-negative) discrete random variable.
Then we define
\[
	X^+ = \max(X, 0);\quad X^- = \max(-X, 0)
\]
Then \(X = X^+ - X^-\).
Note that \(X^+\) and \(X^-\) are non-negative random variables, which has a well-defined expectation.
So if at least one of \(\expect{X^+}, \expect{X^-}\) is finite, we define
\[
	\expect{X} = \expect{X^+} - \expect{X^-}
\]
If both are infinite, then we say that the expectation of \(X\) is not defined.
Whenever we write \(\expect{X}\), it is assumed to be well-defined.
If \(\expect{\abs{X}} < \infty\), we say that \(X\) is integrable.
When \(\expect{X}\) is well-defined, we have again that
\[
	\expect{X} = \sum_{x \in \Omega_x} x \cdot \prob{X = x}
\]

\subsection{Properties of the Expectation}
The following properties follow immediately from the definition.
\begin{enumerate}
	\item If \(X \geq 0\), then \(\expect{X} \geq 0\).
	\item If \(X \geq 0\) and \(\expect{X} = 0\), then \(\prob{X = 0} = 1\).
	\item If \(c \in \mathbb R\), then \(\expect{cX} = c\expect{X}\), and \(\expect{c + X} = c + \expect{X}\).
	\item If \(X\), \(Y\) are two integrable random variables, then \(\expect{X + Y} = \expect{X} + \expect{Y}\).
	\item More generally, let \(c_1, \dots, c_n \in \mathbb R\) and \(X_1, \dots, X_n\) integrable random variables.
	      Then
	      \[
		      \expect{c_1X_1 + \dots + c_n X_n} = c_1 \expect{X_1} + \dots + c_n \expect{X_n}
	      \]
	      So the expectation is a linear operator over finitely many inputs.
\end{enumerate}

\section{Bounds and Inverses}
\subsection{Square Root of \(-1\)}
\begin{theorem}[Wilson's Theorem]
	Let \(p\) be prime.
	Then \((p-1)!
	\equiv -1\ (p)\).
\end{theorem}
\begin{proof}
	Since this is obviously true for \(p=2\), we will suppose that \(p>2\).
	In \(\mathbb Z_p\), let us consider the list \(1, 2, 3 \cdots (p-1)\).
	We can pair each \(a\) with its inverse \(a^{-1}\) for all \(a \neq a^{-1}\).
	Note that \(a = a^{-1} \iff a^2 = 1\) so in this case \(a=1\) or \(a=-1\).
	So let us now multiply each element together, to get
	\[
		(p-1)!
		= (aa^{-1}) (bb^{-1}) \cdots 1 \cdot -1 = (1) \cdot (1) \cdots 1 \cdot -1 = -1
	\]
\end{proof}

\begin{proposition}
	Let \(p>2\) be prime.
	Then \(-1\) is a square number modulo \(p\) if and only if \(p \equiv 1\ (4)\).
\end{proposition}
\begin{proof}
	If \(p>2\) then \(p\) is odd.
	There are therefore two cases, either \(p \equiv 1\) or \(p \equiv 3\) modulo 4.
	Each case is proven individually.
	\begin{itemize}
		\item (\(p = 4k + 3\)) Suppose that \(x^2 = -1\) in \(\mathbb Z_p\).
		      The only thing we know about powers in modular arithmetic is Fermat's Little Theorem, so we will have to use this.
		      So, \(x^{p-1} = x^{4k+2} = 1\).
		      Therefore, \((x^2)^{2k+1} = 1\).
		      But we know that \(x^2=-1\), and we raise this \(-1\) to an odd power, which is \(-1\).
		      So this is a contradiction.
		\item (\(p = 4k + 1\)) By Wilson's Theorem, we know that \((4k)!
		      = -1\).
		      We intend to show that this is a square number in the world of \(\mathbb Z_p\).
		      We will compare the termwise expansion of \((4k)!
		      \) and \([(2k)!]^2\) on consecutive lines.
		      \begin{alignat*}{9}
			      (4k)!
			                & = 1 &  & \cdot 2 &  & \cdot 3 &  & \cdots (2k) &  & \cdot (2k+1) &  & \cdot (2k+2)  &  & \cdots (4k-1)  &  & \cdot (4k)                   \\
			      [(2k)!]^2 & = 1 &  & \cdot 2 &  & \cdot 3 &  & \cdots (2k) &  & \cdot 1      &  & \cdot 2       &  & \cdots (2k-1)  &  & \cdot (2k)                   \\
			      \intertext{By writing each term as an equivalent negative:}
			                & = 1 &  & \cdot 2 &  & \cdot 3 &  & \cdots (2k) &  & \cdot (-4k)  &  & \cdot (-4k+1) &  & \cdots (-2k-2) &  & \cdot (-2k-1)                \\
			      \intertext{Extracting out the negatives:}
			                & = 1 &  & \cdot 2 &  & \cdot 3 &  & \cdots (2k) &  & \cdot (4k)   &  & \cdot (4k-1)  &  & \cdots (2k+2)  &  & \cdot (2k+1) \cdot (-1)^{2k}
		      \end{alignat*}
		      which is equal to the first line by rearranging.
		      So \([(2k)!]^2 = (4k)!
		      = -1\).
		      So \(-1\) is a square number modulo \(p\).
	\end{itemize}
\end{proof}

\subsection{Solving Congruence Equations}
Let us try to solve the equation \(7x \equiv 4\ (30)\).
We take a two-phase approach: first, we will find a single solution, and then we will find all of the other solutions.

Since 7 and 30 are coprime, we can use Euclid's algorithm to find a way of expressing 1 in terms of 7 and 30, in particular \(13 \cdot 7 - 3\cdot 30 = 1\).
This allows us to solve \(7y \equiv 1\ (30)\), by setting \(y=13\).
Then, of course, we can multiply both sides by 4: \(7 y\cdot 4 \equiv 4\ (30)\), so \(x = y \cdot 4 = 13 \cdot 4 = 22\).

We can now find other solutions (apart from trivially adding \(30k\)).
Suppose that there exists some other solution \(x'\), i.e.\ \(7x' \equiv 4\ (30)\).
Then \(7x \equiv 7x'\ (30)\).
As 7 is invertible modulo 30, we can simply multiply by the inverse of 7 to give \(x \equiv x'\ (30)\).
So \(x\) is unique modulo 30.
Alternatively, we could solve the equation without any of this working out by noticing that 7 is invertible!
However, this is not very likely to happen in the general case, since it requires that the coefficient of \(x\) is coprime to the modulus.

Now, let's try a different equation, \(10x = 12\ (34)\).
Since 10 is not invertible, we can't do quite the same thing as above.
We can't also just divide the whole thing by 2, there isn't a rule for that in general.
We can, however, move into \(\mathbb Z\) and manipulate the expression there.
\(10x = 12 + 34y\) for some \(y \in \mathbb Z\), so we can divide the equation by 2 to get \(5x = 6 + 17y\), so \(5x = 6\ (17)\) and we can solve from there.

\subsection{Simultaneous Congruence}
Is there a solution for the simultaneous congruences
\[
	x \equiv 6\ (17);\quad x \equiv 2\ (19)
\]
17 and 19 are coprime, so congruence mod 17 and congruence mod 19 are independent of each other.
How about
\[
	x \equiv 6\ (34);\quad x \equiv 11\ (36)
\]
In this instance, there is obviously no solution; should \(x\) be even or odd?
We can see that, the smallest amount we can adjust \(x\) by in one equation while retaining congruence in the other equation is \(\HCF(34, 36)\), which is 2.
\begin{theorem}[Chinese Remainder Theorem]
	Let \(u, v\) be coprime.
	Then for any \(a, b\), there exists a value \(x\) such that
	\[
		x \equiv a\ (u);\quad x \equiv b\ (v)
	\]
	and that this value is unique modulo \(uv\).
\end{theorem}
\begin{proof}
	We first prove existence of such an \(x\).
	By Euclid's Algorithm, we have \(su + tv = 1\) for some integers \(s, t\).
	Note that therefore:
	\[
		su \equiv 0\ (u);\quad tv \equiv 0\ (v);\quad su \equiv 1\ (v);\quad tv \equiv 1\ (u);
	\]
	Therefore we can make a linear combination of \(su\) and \(tv\) that is the required size in each congruence, specifically
	\[
		x = (su)b + (tv)a
	\]
	Now we prove that this value \(x\) is unique modulo \(uv\).
	Suppose there was some other solution \(x'\).
	Also, \(x' \equiv x\ (u)\) and \(x' \equiv x\ (v)\).
	So we have \(u\mid (x' - x)\) and \(v\mid (x' - x)\) but as \(u\) and \(b\) are coprime we have \(uv\mid (x' - x)\).
	So \(x\) is unique modulo \(uv\).
\end{proof}

\section{Differentiability}
\subsection{Introduction}
A linear map (or linear transformation) is some operation \(T: V \to W\) between vector spaces \(V\) and \(W\) preserving the core vector space structure (specifically, the linearity).
It is defined such that
\[
	T\left(\lambda \vb x + \mu \vb y\right) = \lambda T(\vb x) + \mu T(\vb y)
\]
for all \(\vb x, \vb y \in V\) where the scalars \(\lambda\) and \(\mu\) match up with the scalar field that \(V\) and \(W\) use (so this could be \(\mathbb R\) or \(\mathbb C\) in our examples).
Much of the language used for linear maps between vector spaces is analogous to the language used for homomorphisms between groups.

Note that a linear map is completely determined by its action on a basis \(\{ \vb e_1, \cdots, \vb e_n \}\) where \(n = \dim V\), since
\[
	T\left(\sum_i x_i \vb e_i \right) = \sum_i x_i T(\vb e_i)
\]
We denote \(\vb x' = T(\vb x) \in W\), and define \(\vb x'\) as the image of \(x\) under \(T\).
Further, we define
\[
	\Im (T) = \{ \vb x' \in W : \vb x' =T(\vb x) \text{ for some } \vb x \in V \}
\]
to be the image of \(T\), and we define
\[
	\ker (T) = \{ \vb x \in V : T(\vb x) = \vb 0 \}
\]
to be the kernel of \(T\).

\begin{lemma}
	\(\ker T\) is a subspace of \(V\), and \(\Im T\) is a subspace of \(W\).
\end{lemma}
\begin{proof}
	To verify that some subset is a subspace, it suffices to check that it is non-empty, and that it is closed under linear combinations.

	\(\ker T\) is non-empty because \(\vb 0 \in \ker T\).
	For \(\vb x, \vb y \in \ker T\), we have \(T(\lambda \vb x + \mu \vb y) = \lambda T(\vb x) + \mu T(\vb y) = \vb 0 \in \ker T\) as required.

	\(\Im T\) is non-empty because \(\vb 0 \in \Im T\).
	For \(\vb x, \vb y \in V\), let \(\vb x' = T(\vb x)\) and \(\vb y' = T(\vb y)\), therefore \(\vb x', \vb y' \in \Im T\).
	Now, \(\lambda \vb x' + \mu \vb y' = T(\lambda \vb x + \mu \vb y)\) so it is closed under linear combinations as required.
\end{proof}
Here are some examples of images and kernels.
\begin{enumerate}[(i)]
	\item The zero linear map \(\vb x \mapsto \vb 0\) has:
	      \begin{align*}
		      \Im T  & = \{ \vb 0 \} \\
		      \ker T & = V
	      \end{align*}
	\item The identity linear map \(\vb x \mapsto \vb x\) has:
	      \begin{align*}
		      \Im T  & = V           \\
		      \ker T & = \{ \vb 0 \}
	      \end{align*}
	\item Let \(T: \mathbb R^3 \to \mathbb R^3\), such that
	      \begin{align*}
		      x_1' & = 3x_1 - x_2 + 5x_3 \\
		      x_2' & = -x_1 - 2x_3       \\
		      x_3' & = 2x_1 + x_2 + 3x+3
	      \end{align*}
	      This map has
	      \begin{align*}
		      \Im T  & = \left\{ \lambda \begin{pmatrix} 3 \\ -1 \\ 2 \end{pmatrix} + \mu \begin{pmatrix} 1 \\ 0 \\ 1 \end{pmatrix} : \lambda, \mu \in \mathbb R \right\} \\
		      \ker T & = \left\{ \lambda \begin{pmatrix} 2 \\ -1 \\ -1 \end{pmatrix} : \lambda \in \mathbb R \right\}
	      \end{align*}
\end{enumerate}

\subsection{Rank and nullity}
We define the rank of a linear map to be the dimension of its image, and the nullity of a linear map to be the dimension of its kernel.
\[
	\rank T = \dim \Im T; \quad \nullity T = \dim \ker T
\]
Note that therefore for \(T: V \to W\), we have \(\rank T \leq \dim W\) and \(\ker T \leq \dim V\).
\begin{theorem}
	For some linear map \(T: V \to W\),
	\[
		\rank T + \nullity T = \dim V
	\]
\end{theorem}
\begin{proof}
	This proof is non-examinable (without prompts).
	Let \(\vb e_1, \cdots, \vb e_k\) be a basis for \(\ker T\), so \(T(\vb e_i) = \vb 0\) for all valid \(i\).
	We may extend this basis by adding more vectors \(\vb e_i\) where \(k < i \leq n\) until we have a basis for \(V\), where \(n=\dim V\).
	We claim that the set \(\mathcal B = \{ T(\vb e_{k+1}), \cdots, T(\vb e_n) \}\) is a basis for \(\Im T\).
	If this is true, then clearly the result follows because \(k = \dim \ker T = \nullity T\) and \(n-k = \dim \Im T = \rank T\).

	To prove the claim we need to show that \(\mathcal B\) spans \(\Im T\) and that it is a linearly independent set.
	\begin{itemize}
		\item \(\mathcal B\) spans \(\Im T\) because for any \(\vb x = \sum_{i=1}^n x_i \vb e_i\), we have
		      \[
			      T(\vb x) = \sum_{i=k+1}^n x_i T(\vb e_i) \in \vecspan \mathcal B
		      \]
		\item \(\mathcal B\) is linearly independent.
		      Consider a general linear combination of basis vectors:
		      \[
			      \sum_{i=k+1}^n \lambda_i T(\vb e_i) = 0 \implies T\left( \sum_{i=k+1}^n \lambda_i \vb e_i \right) = 0
		      \]
		      so
		      \[
			      \sum_{i=k+1}^n \lambda_i \vb e_i \in \ker T
		      \]
		      Because this is in the kernel, it may be written in terms of the basis vectors of the kernel.
		      So, we have
		      \[
			      \sum_{i=k+1}^n \lambda_i \vb e_i = \sum_{i=1}^k \mu_i \vb e_i
		      \]
		      This is a linear relation in terms of all basis vectors of \(V\).
		      So all coefficients are zero.
	\end{itemize}
\end{proof}

\subsection{Rotations}
Linear maps are often used to describe geometrical transformations, such as rotations, reflections, projections, dilations and shears.
A convenient way to express these maps is by describing where the basis vectors are mapped to.
In \(\mathbb R^2\), we may describe a rotation anticlockwise around the origin by angle \(\theta\) with
\begin{align*}
	\vb e_1 & \mapsto \cos \theta \vb e_1 + \sin \theta \vb e_2  \\
	\vb e_2 & \mapsto -\sin \theta \vb e_1 + \cos \theta \vb e_2
\end{align*}
In \(\mathbb R^3\) we can construct a similar transformation for a rotation around the \(\vb e_3\) axis with
\begin{align*}
	\vb e_1 & \mapsto \cos \theta \vb e_1 + \sin \theta \vb e_2  \\
	\vb e_2 & \mapsto -\sin \theta \vb e_1 + \cos \theta \vb e_2 \\
	\vb e_3 & \mapsto \vb e_3
\end{align*}
We can extend this to a general rotation in \(\mathbb R^3\) about an axis given by a unit normal vector \(\hat {\vb n}\).
For any vector \(\vb x \in \mathbb R^3\) we can resolve parallel and perpendicular to \(\nhat\) as follows.
\[
	\vb x = \vb x_\parallel + \vb x_\perp;\quad \vb x_\parallel = (\vb x \cdot \nhat) \nhat;\quad \vb x_\perp = \vb x - (\vb x \cdot \nhat) \nhat
\]
Note that \(\nhat\) resembles the \(\vb e_3\) axis here, and \(\vb x_\perp\) resembles the \(\vb e_1\) axis.
So we can compute the equivalent of \(\vb e_2\) using the cross product, \(\nhat \times \vb x_\perp = \nhat \times \vb x\).
Now we may define the map with
\begin{align*}
	\vb x_\parallel & \mapsto \vb x_\parallel                                               \\
	\vb x_\perp     & \mapsto (\cos \theta) \vb x_\perp + (\sin \theta)(\nhat \times \vb x)
\end{align*}
So all together, we have
\[
	\vb x \mapsto (\cos \theta) \vb x + (1 - \cos \theta) (\nhat \cdot \vb x)\nhat + (\sin \theta)(\nhat \times \vb x)
\]

\subsection{Reflections and projections}
For a plane with normal \(\nhat\), we define a projection to be
\begin{align*}
	\vb x_\parallel & \mapsto \vb 0                                           \\
	\vb x_\perp     & \mapsto \vb x_\perp                                     \\
	\vb x           & \mapsto \vb x_\perp = \vb x - (\vb x \cdot \nhat) \nhat
\end{align*}
and a reflection to be
\begin{align*}
	\vb x_\parallel & \mapsto -\vb x_\parallel                                                   \\
	\vb x_\perp     & \mapsto \vb x_\perp                                                        \\
	\vb x           & \mapsto \vb x_\perp - \vb x_\parallel = \vb x - 2(\vb x \cdot \nhat) \nhat
\end{align*}
The same expressions also apply in \(\mathbb R^2\), where we replace the plane with a line.

\subsection{Dilations}
Given scale factors \(\alpha, \beta, \gamma > 0\), we define a dilation along the axes by
\begin{align*}
	\vb e_1 & \mapsto \alpha \vb e_1 \\
	\vb e_2 & \mapsto \beta \vb e_2  \\
	\vb e_3 & \mapsto \gamma \vb e_3
\end{align*}

\subsection{Shears}
Let \(\vb a, \vb b\) be orthogonal unit vectors in \(\mathbb R^3\), i.e.\ \(\abs{\vb a} = \abs{\vb b} = \vb 0\) and \(\vb a \cdot \vb b = 0\), and we define a real parameter \(\lambda\).
A shear is defined as
\begin{align*}
	\vb x & \mapsto \vb x' = \vb x + \lambda \vb a (\vb x \cdot \vb b) \\
	\vb a & \mapsto \vb a                                              \\
	\vb b & \mapsto \vb b + \lambda \vb a
\end{align*}
This definition holds equivalently in \(\mathbb R^2\).

\section{Properties of the Derivative}
\subsection{Equality in Cauchy-Schwarz}
In what cases do we get equality in the Cauchy-Schwarz inequality?
Recall that the inequality states
\[
	\abs{\expect{XY}} \leq \sqrt{\expect{X^2}\cdot\expect{Y^2}}
\]
Recall that in the proof, we considered the random variable \((X - tY)^2\) where \(X\) and \(Y\) were non-negative, and had finite second moments.
The expectation of this random variable was called \(f(t)\), and we found that \(f(t)\) was minimised when \(t = \frac{\expect{XY}}{\expect{Y^2}}\).
We have equality exactly when \(f(t) = 0\) for this value of \(t\).
But \((X - tY)^2\) is a non-negative random variable, with expectation zero, so it must be zero with probability 1.
So we have equality if and only if \(X\) is exactly \(tY\).

\subsection{Jensen's Inequality}
\begin{definition}
	A function \(f\colon \mathbb R \to \mathbb R\) is called convex if \(\forall x, y \in \mathbb R\) and for all \(t \in [0, 1]\),
	\[
		f(tx + (1-t)y) \leq tf(x) + (1-t)f(y)
	\]
	This can be visualised as linearly interpolating the values of the function at two points, \(x\) and \(y\).
	The linear interpolation of those points is always greater than the function applied to the linear interpolation of the input points.
\end{definition}
\begin{theorem}
	Let \(X\) be a random variable, and let \(f\) be a convex function.
	Then
	\[
		\expect{f(X)} \geq f(\expect{X})
	\]
\end{theorem}
\noindent We can remember the direction of this inequality by considering the variance: \(\Var{X} = \expect{(X - \expect{X})^2}\) which is non-negative.
Further, \(\Var{X} = \expect{X^2} - \expect{X}^2\) hence \(\expect{X^2} \geq \expect{X}^2\).
Squaring is an example of a convex function, so Jensen's inequality holds in this case.
We will first prove a basic lemma about convex functions.
\begin{lemma}
	Let \(f \colon \mathbb R \to \mathbb R\) be a convex function.
	Then \(f\) is the supremum of all the lines lying below it.
	More formally, \(\forall m \in \mathbb R\), \(\exists a, b \in \mathbb R\) such that \(f(m) = am + b\) and \(f(x) \geq ax + b\) for all \(x\).
\end{lemma}
\begin{proof}
	Let \(m \in \mathbb R\).
	Let \(x < m < y\).
	Then we can express \(m\) as \(tx + (1-t)y\) for some \(t\) in the interval \([0, 1]\).
	By convexity,
	\[
		f(m) \leq tf(x) + (1-t)f(y)
	\]
	And hence,
	\begin{align*}
		tf(m) + (1-t)f(m)         & \leq tf(x) + (1-t)f(y)       \\
		t(f(m) - f(x))            & \leq (1-t)(f(y) - f(m))      \\
		\frac{f(m) - f(x)}{m - x} & \leq \frac{f(y) - f(m)}{y-m}
	\end{align*}
	So the slope of the line joining \(m\) to a point on its left is smaller than the slope of the line joining \(m\) to a point on its right.
	So we can produce a value \(a \in \mathbb R\) given by
	\[
		a = \sup_{x < m} \frac{f(m) - f(x)}{m - x}
	\]
	such that
	\[
		\frac{f(m) - f(x)}{m - x} \leq a \leq \frac{f(y) - f(m)}{y - m}
	\]
	for all \(x < m < y\).
	We can rearrange this to give
	\[
		f(x) \geq a(x-m) + f(m) = ax + (f(m) - am)
	\]
	for all \(x\).
\end{proof}
\noindent We may now prove Jensen's inequality.
\begin{proof}
	Set \(m = \expect{X}\).
	Then from the lemma above, there exists \(a, b \in \mathbb R\) such that
	\begin{equation}
		f(m) = am + b \implies f(\expect{X}) = a\expect{X} + b \tag{\(\ast\)}
	\end{equation}
	and for all \(x\), we have
	\[
		f(x) \geq ax + b
	\]
	We can now apply this inequality to \(X\) to get
	\[
		f(X) \geq aX + b
	\]
	Taking the expectation, by \((\ast)\) we get
	\[
		\expect{f(X)} \geq a\expect{X} + b = f(\expect{X})
	\]
	as required.
\end{proof}
\noindent Like the Cauchy-Schwarz inequality, we would like to consider the cases of equality.
Let \(X\) be a random variable, and \(f\) be a convex function such that if \(m = \expect{X}\), then \(\exists a, b \in \mathbb R\) such that
\[
	f(m) = am + b;\quad \forall x \neq m,\, f(x) > ax + b
\]
We know that \(f(X) \geq aX + b\), since \(f\) is convex.
Then \(f(X) - (aX+b) \geq 0\) is a non-negative random variable.
Taking expectations,
\[
	\expect{f(X) - (aX+b)} \geq 0
\]
But \(\expect{aX + b} = am + b = f(m) = f(\expect{X})\).
We assumed that \(\expect{f(X)} = f(\expect{X})\), hence \(\expect{aX+b} = \expect{f(X)}\) and \(\expect{f(X) - (aX+b)} = 0\).
But since \(f(X) \geq aX+b\), this forces \(f(X) = aX+b\) everywhere.
By our assumption, for all \(x \neq m\), \(f(x) > ax+b\).
This forces \(X=m\) with probability 1.

\subsection{Arithmetic Mean and Geometric Mean Inequality}
Let \(f\) be a convex function.
Suppose \(x_1, \dots, x_n \in \mathbb R\).
Then, from Jensen's inequality,
\[
	\frac{1}{n} \sum_{k=1}^n f(x_k) \geq f\left( \frac{1}{n} \sum_{k=1}^n x_k \right)
\]
Indeed, we can define a random variable \(X\) to take values \(x_1, \dots, x_n\) all with equal probability.
Then, \(\expect{f(X)}\) gives the left hand side, and \(f(\expect{X})\) gives the right hand side.
Now, let \(f(x) = -\log x\).
This is a convex function as required.
Hence
\begin{align*}
	-\frac{1}{n} \sum_{k=1}^n \log x_k             & \geq -\log\left( \frac{1}{n} \sum_{k=1}^n x_k \right) \\
	\left( \prod_{k=1}^n x_k \right)^{\frac{1}{n}} & \leq \frac{1}{n} \sum_{k=1}^n x_k
\end{align*}
Hence the geometric mean is less than or equal to the arithmetic mean.

\subsection{Conditional Expectation and Law of Total Expectation}
Recall that if \(B \in \mathcal F\) with \(\prob{B} \geq 0\), we defined
\[
	\prob{A \mid B} = \frac{\prob{A \cap B}}{\prob{B}}
\]
Now, let \(X\) be a random variable, and let \(B\) be an event as above with non-zero probability.
We can then define
\[
	\expect{X \mid B} = \frac{\expect{X \cdot 1(B)}}{\prob{B}}
\]
The numerator is notably zero when \(1(B) = 0\), so in essence we are excluding the case where \(X\) is not \(B\).
\begin{theorem}[Law of Total Expectation]
	Suppose \(X \geq 0\).
	Let \((\Omega_n)\) be a partition of \(\Omega\) into disjoint events, so \(\Omega = \bigcup_n \Omega_n\).
	Then
	\[
		\expect{X} = \sum_n \expect{X \mid \Omega_n} \cdot \prob{\Omega_n}
	\]
\end{theorem}
\begin{proof}
	We can write \(X = X \cdot 1(\Omega)\), where
	\[
		1(\Omega) = \sum_n 1(\Omega_n)
	\]
	Taking expectations, we get
	\[
		\expect{X} = \expect{ \sum_{n} X \cdot 1(\Omega_n) }
	\]
	By countable additivity of expectation, we have
	\[
		\expect{X} = \sum_{n} \expect{ X \cdot 1(\Omega_n) } = \sum_n \expect{X \mid \Omega_n} \cdot \prob{\Omega_n}
	\]
	as required.
\end{proof}

\section{Using the Derivative}
\subsection{Joint Distribution}
\begin{definition}
	Let \(X_1, \dots, X_n\) be discrete random variables. Their joint distribution is defined as
	\[ \prob{X_1 = x_1, \dots, X_n = x_n} \]
	for all \(x_i \in \Omega_i\).
\end{definition}
\noindent Now, we have
\[ \prob{X_1 = x_1} = \prob{\{ X_1 = x_1\} \cap \bigcup_{i=2}^n \bigcup_{x_i} \{ X_i = x_i \} } = \sum_{x_2, \dots, x_n} \prob{X_1 = x_1, X_2 = x_2, \dots, X_n = x_n} \]
In general,
\[ \prob{X_i = x_i} = \sum_{x_1, x_2, \dots, x_{i-1}, x_{i+1}, \dots, x_n} \prob{X_1 = x_1, X_2 = x_2, \dots, X_n = x_n} \]
We call \((\prob{X_i = x_i})_i\) the marginal distribution of \(X_i\). Let \(X, Y\) be random variables. The conditional distribution of \(X\) given \(Y = y\) where \(y \in \Omega_y\) is defined to be
\[ \prob{X = x \mid Y = y} = \frac{\prob{X = x, Y = y}}{\prob{Y = y}} \]
We can find
\[ \prob{X = x} = \sum_y \prob{X = x, Y = y} = \sum_y \prob{X = x \mid Y = y} \prob{Y = y} \]
which is the law of total probability.

\subsection{Convolution}
Let \(X\) and \(Y\) be independent, discrete random variables. We would like to find \(\prob{X + Y = z}\). Clearly this is
\begin{align*}
	\prob{X + Y = z} & = \sum_y \prob{X + Y = z, Y = y}           \\
	                 & = \sum_y \prob{X = z-y, Y = y}             \\
	                 & = \sum_y \prob{X = z-y} \cdot \prob{Y = y} \\
\end{align*}
This last sum is called the convolution of the distributions of \(X\) and \(Y\). Similarly,
\[ \prob{X + Y = z} = \sum_x \prob{X = x} \cdot \prob{Y = z-x} \]
As an example, let \(X \sim \mathrm{Poi}(\lambda)\) and \(Y \sim \mathrm{Poi}(\mu)\) be independent. Then
\begin{align*}
	\prob{X + Y = n} & = \sum_{r = 0}^n \prob{X = r} \prob{Y = n - r}                                              \\
	                 & = \sum_{r = 0}^n e^{-\lambda} \frac{\lambda^r}{r!} \cdot e^{-\mu} \frac{\mu^{n-r}}{(n-r)!}  \\
	                 & = e^{-(\lambda+\mu)} \sum_{r = 0}^n \frac{\lambda^r\mu^{n-r}}{r!(n-r)!}                     \\
	                 & = \frac{e^{-(\lambda+\mu)}}{n!} \sum_{r = 0}^n \frac{\lambda^r\mu^{n-r} \cdot n!}{r!(n-r)!} \\
	                 & = \frac{e^{-(\lambda+\mu)}}{n!} \sum_{r = 0}^n \binom{n}{r} \lambda^r\mu^{n-r}              \\
	                 & = \frac{e^{-(\lambda+\mu)}}{n!} (\lambda + \mu)^n                                           \\
\end{align*}
which is the probability mass function of a Poisson random variable with parameter \(\lambda + \mu\). In other words, \(X + Y \sim \mathrm{Poi}(\lambda + \mu)\).

\subsection{Conditional Expectation}
Let \(X\) and \(Y\) be discrete random variables. Then the conditional expectation of \(X\) given that \(Y = y\) is
\[ \expect{X \mid Y = y} = \frac{\expect{X \cdot 1(Y = y)}}{\prob{Y = y}} = \frac{1}{\prob{Y = y}} \sum_x x \cdot \prob{X = x, Y = y} = \sum_x x \cdot \prob{X = x \mid Y = y} \]
Observe that for every \(y \in \Omega_y\), this expectation is purely a function of \(y\). Let \(g(y) = \expect{X \mid Y = y}\). Now, we define the conditional expectation of \(X\) given \(Y\) as \(\expect{X \mid Y} = g(Y)\). Note that \(\expect{X \mid Y}\) is a random variable, dependent only on \(Y\). We have
\begin{align*}
	\expect{X \mid Y} & = g(Y) \cdot 1                                \\
	                  & = g(Y) \sum_y 1(Y = y)                        \\
	                  & = \sum_y g(Y) \cdot 1(Y = y)                  \\
	                  & = \sum_y g(y) \cdot 1(Y = y)                  \\
	                  & = \sum_y \expect{X \mid Y = y} \cdot 1(Y = y)
\end{align*}
This is perhaps a clearer way to see that it depends only on \(Y\). As an example, let us consider tossing a \(p\)-biased coin \(n\) times independently. We write \(X_i\) for the indicator function that the \(i\)th toss was a head. Let \(Y_n = X_1 + \dots + X_n\). What is \(\expect{X_1 \mid Y_n}\)? Let \(g(y) = \expect{X_1 \mid Y_n = y}\). Then \(\expect{X_1 \mid Y_n} = g(Y_n)\). We therefore need to find \(g\). Let \(y \in \{ 0, \dots, n \}\), then
\begin{align*}
	g(y) & = \expect{X_1 \mid Y_n = y}   \\
	     & = \prob{X_1 = 1 \mid Y_n = y} \\
\end{align*}
Clearly if \(y = 0\), then \(\prob{X_1 = 1 \mid Y_n = 0} = 0\). Now, suppose \(y \neq 0\). We have
\begin{align*}
	\prob{X_1 = 1 \mid Y_n = y} & = \frac{\prob{X_1 = 1, Y_n = y}}{\prob{Y_n = y}}                             \\
	                            & = \frac{\prob{X_1 = 1, X_2 + \dots + X_n = y-1}}{\prob{Y_n = y}}             \\
	                            & = \frac{\prob{X_1 = 1} \cdot \prob{X_2 + \dots + X_n = y-1}}{\prob{Y_n = y}} \\
	                            & = \frac{p \cdot \binom{n-1}{y-1} \cdot p^{y-1}(1-p)^{n-y}}{\prob{Y_n = y}}   \\
	                            & = \frac{\binom{n-1}{y-1} \cdot p^y(1-p)^{n-y}}{\binom{n}{y}p^y (1-p)^{n-y}}  \\
	                            & = \frac{\binom{n-1}{y-1}}{\binom{n}{y}}                                      \\
	                            & = \frac{y}{n}
\end{align*}
Hence
\[ g(y) = \frac{y}{n} \]
We can then find that
\[ \expect{X_1 \mid Y_n} = g(Y_n) = \frac{Y_n}{n} \]
which is indeed a random variable dependent only on \(Y_n\).

\subsection{Properties of Conditional Expectation}
The following properties hold.
\begin{itemize}
	\item For all \(c \in \mathbb R\), \(\expect{cX \mid Y} = c\expect{X \mid Y}\), and \(\expect{c \mid Y} = c\).
	\item Let \(X_1, \dots, X_n\) be random variables. Then \(\expect{\sum_{i=1}^n X_i \mid Y} = \sum_{i=1}^n \expect{X_i \mid Y}\).
	\item \(\expect{\expect{X \mid Y}} = \expect{X}\).
\end{itemize}
The last property is not obvious from the definition, so it warrants its own proof. We can see by the standard properties of the expectation that
\begin{align*}
	\expect{X \mid Y}                     & = \sum_y 1(Y = y) \expect{X \mid Y = y}                              \\
	\therefore \expect{\expect{X \mid Y}} & = \sum_y \expect{1(Y = y)} \expect{X \mid Y = y}                     \\
	                                      & = \sum_y \prob{Y = y} \expect{X \mid Y = y}                          \\
	                                      & = \sum_y \prob{Y = y} \frac{\expect{X \cdot 1(Y = y)}}{\prob{Y = y}} \\
	                                      & = \sum_y \expect{X \cdot 1(Y = y)}                                   \\
	                                      & = \expect{\sum_y X \cdot 1(Y = y)}                                   \\
	                                      & = \expect{X \sum_y 1(Y = y)}                                         \\
	                                      & = \expect{X}                                                         \\
\end{align*}
Alternatively, we could expand the inner expectation as a sum:
\[ \sum_y \expect{X \mid Y = y} \cdot \prob{Y = y} = \sum_x \sum_y x \cdot \prob{X = x \mid Y = y} \cdot \prob{Y = y} \]
and the result follows as required. The final property relates conditional probability to independence. Let \(X\) and \(Y\) be independent. Then \(\expect{X \mid Y} = \expect{X}\). Indeed,
\begin{align*}
	\expect{X \mid Y} & = \sum_y 1(Y = y) \expect{X \mid Y = y} \\
	                  & = \sum_y 1(Y = y) \expect{X}            \\
	                  & = \expect{X}                            \\
\end{align*}

\section{Extensions of the Mean Value Theorem}
\subsection{First Isomorphism Theorem}
\begin{theorem}
	Let \(\varphi: G \to H\) be a homomorphism.
	Then \(\frac{G}{\ker \varphi} \cong \Im \varphi\).
\end{theorem}
\begin{proof}
	Define \(\overline \varphi: \frac{G}{\ker \varphi} \to \Im \varphi\) using \(g \ker \varphi \mapsto \varphi(g)\).
	\begin{itemize}
		\item (well-defined) If \(g_1 \ker \varphi = g_2 \ker \varphi\), then \(g_1 = g_2k\), for some \(k \in \ker \varphi\).
		      Hence \(\overline\varphi(g_1 \ker \varphi) = \varphi(g_1) = \varphi(g_2k) = \varphi(g_2)\varphi(k) = \varphi(g_2) = \overline\varphi(g_2 \ker \varphi)\).
		\item (homomorphism) Let \(g, g' \in G\).
		      \(\overline\varphi(g \ker \varphi \cdot g' \ker \varphi) = \overline\varphi(gg' \ker \varphi) = \varphi(gg') = \varphi(g)\varphi(g') = \overline\varphi(g\ker\varphi) \cdot \overline\varphi(g'\ker\varphi)\).
		\item (surjective) All elements of \(\Im \varphi\) are of the form \(\varphi(g)\) for some \(g \in G\), so clearly surjective.
		\item (injective) If \(\overline\varphi(g \ker \varphi) = e = \varphi(g)\) in \(\Im \varphi\) then \(g \in \ker \varphi\), so \(g \ker \varphi = \ker \varphi\).
	\end{itemize}
\end{proof}
This is a useful way to understand the first isomorphism theorem.
Recall that \(\frac{G}{\ker \varphi}\) is really asking the question `how do the copies of \(\ker \varphi\) interact in \(G\)'? Well, as \(\varphi\) is a homomorphism, it represents some property that is true for members of a normal subgroup \(N\) in \(G\), where \(N = \ker \varphi\).
Now, we can imagine the grid analogy from before, laying out several copies of \(N\) as rows.
Let's call the group of these rows \(K\).

Now, multiplying together two rows, i.e.\ two elements from \(K\), we can apply the homomorphism \(\varphi\) to one of the coset representatives for each row to see how the entire row behaves under \(\varphi\).
We know that all coset representatives give equal results, because each element in a given coset \(gN\) can be written as \(gn, n \in N\), so \(\varphi(gn) = \varphi(g)\).
So all elements in the rows behave just like their coset representatives under the homomorphism.
Further, all the cosets give different outputs under \(\varphi\) --- if they gave the same output they'd have to be part of the same coset.
So in some sense, each row represents a distinct output for \(\varphi\).
So the quotient group must be isomorphic to the image of the homomorphism.

Here are some examples.
\begin{enumerate}
	\item \(\det : GL_2(\mathbb R) \to \mathbb R^*\), \(\Im(\det) = \mathbb R^*\), \(\ker(\det) = SL_2(\mathbb R)\).Therefore, \(\frac{GL_2(\mathbb R)}{SL_2(\mathbb R)} \cong \mathbb R^*\).
	\item Consider the map \(\varphi: \mathbb R \to \mathbb C^*, \varphi(r) = e^{2\pi i r}\).
	      This is a homomorphism because \(\varphi(r + s) = e^{2\pi i (r + s)} = e^{2 \pi i r}\cdot e^{2 \pi i s} = \varphi(r) \cdot \varphi(s)\).
	      The image is the unit circle \(\abs{z} = 1\), denoted by \(S_1\); the kernel is \(\mathbb Z\) as \(e^{2 \pi i z}\) for some \(z \in \mathbb Z\), the result is 1.
	      Therefore \(\frac{\mathbb R}{\mathbb Z} = S_1\).
\end{enumerate}

\subsection{Correspondence Theorem}
Now, let's try to understand how subgroups behave inside quotient groups.
\begin{theorem}
	Let \(N \trianglelefteq G\).
	The subgroups of \(\frac{G}{N}\) are in bijective correspondence with subgroups of \(G\) containing \(N\).
\end{theorem}
\begin{proof}
	Given \(N \leq M \leq G\), \(N \trianglelefteq G\), then \(N \trianglelefteq M\) and clearly \(\frac{M}{N} \leq \frac{G}{N}\).
	Conversely, for every subgroup \(H \leq \frac{G}{N}\), we can take the preimage of \(H\) under the quotient map \(\pi : G \to \frac{G}{N}\), i.e.\ \(\pi^{-1}(H) = \{ g \in G : gN \in H \}\).
	This is a subgroup of \(G\):
	\begin{itemize}
		\item (closure) if \(g_1, g_2 \in \pi^{-1}(H)\), then \(g_1g_2N = g_1N\cdot g_2N\) where both elements \(g_1N\) and \(g_2N\) are in \(H\).
		      So \(g_1g_2N \in H\).
		\item (identity, inverses easy to check)
	\end{itemize}
	\(\pi^{-1}(H)\) contains \(N\), since \(\forall n \in N\), \(nN = N \in H\).
	Now we can check that for any \(N \leq M \leq G\), \(\pi^{-1}(\frac{M}{N}) = M\) and for \(H \leq \frac{G}{N}\), \(\frac{\pi^{-1}(H)}{N} = H\).
	So the correspondence is bijective (this satisfies the property that \(ff^{-1}\) and \(f^{-1}f\) are the identity maps on the relevant sets).
\end{proof}
This correspondence preserves lots of structure: for example, indices, normality, containment.
One example is the group \(C_4 \times C_2\), where \(C_4 = \genset a\) and \(C_2 = \genset b\).
The subgroups of this are (TODO draw subgroup lattice)
% TODO draw subgroup lattice
Now, let \(N := \genset{(a^2, b)}\).
Note that this is normal because we are in an abelian group.
Then, according to the above theorem, the subgroup lattice for \(\frac{C_4 \times C_2}{N}\) is bijective with the set of paths on the above lattice that terminate with \(N\) (i.e.\ have \(N\) as a subgroup).
% TODO draw second subgroup lattice
We took the quotient of a group of order 8 by a group of order 2, so \(N\) has order 4, so it must be isomorphic to \(C_4\) (as it has only one subgroup isomorphic to \(C_2\) as can be seen in the lattice, so it cannot be \(C_2 \times C_2\)).

\subsection{Second Isomorphism Theorem}
Let \(H\leq G\) and \(N\trianglelefteq G\), but \(N \nleq H\).
We can actually still make a normal subgroup of \(H\) by intersecting \(H\) with \(N\).
\begin{theorem}
	Let \(H \leq G\) and \(N \trianglelefteq G\).
	Then \(H \cap N \trianglelefteq H\) and \(\frac{H}{H \cap N} \cong \frac{HN}{N}\).
\end{theorem}
\begin{proof}
	When \(N \trianglelefteq G, H \leq G\), then \(HN = \{ hn: h \in H, n \in N \}\) is a subgroup of \(G\), and \(HN = \genset{H, N}\).
	
	Consider the function \(\varphi: H \to \frac{HN}{N}, \varphi(h) := hN\).
	This is a well-defined surjective homomorphism.
	\(\varphi(h) = hN = N \iff h \in N\), but also \(h \in H\), so \(h \in N \cap H\) is the kernel.
	So by the First Isomorphism Theorem, \(\frac{H}{N \cap H} \cong \frac{HN}{N}\) (note that \(\frac{HN}{N} \leq \frac{G}{N}\)).
\end{proof}

\subsection{Third Isomorphism Theorem}
We noted earlier that normality is preserved inside quotient groups.
We can say something analogous about quotients.
\begin{theorem}
	Let \(N \leq M \leq G\) such that \(N \trianglelefteq G\) and \(M \trianglelefteq G\).
	Then \(\frac{M}{N} \trianglelefteq \frac{G}{N}\), and \(\frac{G/N}{M/N} = \frac{G}{M}\).
\end{theorem}
\begin{proof}
	Let us define \(\varphi: \frac{G}{N} \to \frac{G}{M}\) by \(\varphi(gN) = gM\).
	\(\varphi\) is well defined since \(N \leq M\), and it is a surjective homomorphism.
	\(\varphi(gN) = gM = M \iff g \in M\), so its kernel is \(\frac{M}{N}\).
	By the First Isomorphism Theorem, \(\frac{G/N}{M/N} \cong \frac{G}{M}\).
\end{proof}

\section{Applications of Remainders in Taylor's Theorem}
\subsection{Divergence Theorem}
\begin{proof}
	Suppose first that
	\[ \vb F = F_z(x,y,z) \vb e_z \]
	The divergence theorem states that
	\begin{equation}
		\int_V \underbrace{\pdv{F_z}{z}}_{\div{\vb F}} \dd{V} = \int_{\partial V} F_z \vb e_z \cdot \dd{\vb S}
		\tag{$\dagger$}
	\end{equation}
	We would like to show that these two are really the same. First, let us simplify the problem to a convex volume $V$, such that we can split the boundary into two halves, one with normals in the positive $z$ direction ($S_+$) and one with normals in the negative $z$ direction ($S_-$). Then $\partial V = S_+ \cup S_-$. Project the volume into the $x$-$y$ plane, and call this region $A$. This planar region is then the shape of the `cut' between the $S_+$ and $S_-$ halves. We can write
	\[ S_\pm = \left\{ \vb x(x, y) = \begin{pmatrix}
			x \\ y \\ g_\pm (x, y)
		\end{pmatrix} : (x, y) \in A \right\} \]
	We can then say
	\begin{align*}
		\int_V \pdv{F_z}{z} \dd{V} & = \iint_{A} \left[ \int_{z = g_- (x, y)}^{g_+ (x, y)} \pdv{F_z}{z} \dd{z} \right] \dd{x} \dd{y} \\
		                           & = \iint_A \left[ F_z(x, y, g_+ (x, y)) - F_z(x, y, g_- (x, y)) \right] \dd{x}\dd{y}
	\end{align*}
	To calculate right hand side of $(\dagger)$, we need $\dd{\vb S}$:
	\begin{align*}
		\dd{\vb S} & = \pdv{\vb x}{x} \times \pdv{\vb x}{y} \dd{x}\dd{y} \\
		           & = \begin{pmatrix}
			-\pdv*{g_\pm}{x} \\
			-\pdv*{g_\pm}{y} \\
			1                \\
		\end{pmatrix} \dd{x}\dd{y}
	\end{align*}
	Since we want the normal to point `out' of $V$, on $S_\pm$ we have
	\[ \eval{\dd{\vb S}}_{S_\pm} = \pm \begin{pmatrix}
			-\pdv*{g_\pm}{x} \\
			-\pdv*{g_\pm}{y} \\
			1                \\
		\end{pmatrix} \dd{x}\dd{y} \]
	Therefore,
	\begin{align*}
		\int_{\partial V} \vb F \cdot \dd{\vb S} & = \left[ \int_{S_+} + \int_{S_-} \right] F_z \vb e_z \cdot \dd{\vb S}                   \\
		                                         & = \iint_A F_z(x, y, g_+(x, y)) \dd{x}\dd{y} - \iint_A F_z(x, y, g_-(x, y)) \dd{x}\dd{y}
	\end{align*}
	which matches the expression we found for the left hand side of $(\dagger)$ above. In the same way, we can show that
	\begin{align*}
		\int_V \pdv{F_x}{x} \dd{V} & = \int_{\partial V} F_x \vb e_x \cdot \dd{\vb S} \\
		\int_V \pdv{F_y}{y} \dd{V} & = \int_{\partial V} F_y \vb e_y \cdot \dd{\vb S} \\
	\end{align*}
	and because the integrals are linear, we can compute their sum to find
	\[ \int_V \div{\vb F} \dd{V} = \int_{\partial V} \vb F \cdot \dd{\vb S} \]
	which is exactly the divergence theorem.
\end{proof}

\subsection{Green's Theorem}
We can use the two-dimensional divergence theorem to prove Green's theorem.
\begin{proof}
	Let
	\[ \vb F = \begin{pmatrix}
			Q(x, y) \\ -P(x, y)
		\end{pmatrix} \]
	Then
	\[ \iint_A \left( \pdv{Q}{x} - \pdv{P}{y} \right) \dd{x}\dd{y} = \int_A \div{\vb F} \dd{A} = \oint_{\partial A} \vb F \cdot \vb n \dd{s} \]
	If $\partial A$ is parametrised with respect to arc length, which means that the unit tangent vector is
	\[ \vb t = \begin{pmatrix}
			x'(s) \\ y'(s)
		\end{pmatrix} \]
	then the normal vector is
	\[ \vb n = \begin{pmatrix}
			y'(s) \\ -x'(s)
		\end{pmatrix} \]
	Therefore,
	\[ \oint_{\partial A} \vb F \cdot \vb n \dd{s} = \oint_{\partial A} \begin{pmatrix}
			Q \\ -P
		\end{pmatrix} \cdot \begin{pmatrix}
			y'(s) \\ -x'(s)
		\end{pmatrix} \dd{s} = \oint_{\partial A} P \dv{x}{s} \dd{s} + Q \dv{y}{s} \dd{s} = \oint_{\partial A} P \dd{x} + Q \dd{y} \]
\end{proof}

\subsection{Stokes' Theorem}
We can now use Green's theorem to derive Stokes' theorem.
\begin{proof}
	Consider a regular surface
	\[ S = \left\{ \vb x = \vb x(u, v) \colon (u, v) \in A \right\} \]
	Then the boundary is
	\[ \partial S = \left\{ \vb x = \vb x(u, v) \colon (u, v) \in \partial A \right\} \]
	Green's theorem gives
	\begin{equation}
		\oint_{\partial A} P \dd{u} + Q \dd{v} = \iint_A \left( \pdv{Q}{u} - \pdv{P}{v} \right) \dd{u}\dd{v}
		\tag{$\dagger$}
	\end{equation}
	We will now set
	\[ P(u, v) = \vb F(\vb x(u, v)) \cdot \pdv{\vb x}{u};\quad Q(u, v) = \vb F(\vb x(u, v)) \cdot \pdv{\vb x}{v} \]
	Then
	\[ P \dd{u} + Q \dd{v} = \vb F(\vb x(u, v)) \cdot \left( \pdv{\vb x}{u} \dd{u} + \pdv{\vb v} \dd{v} \right) = \vb F(\vb x(u, v)) \cdot \dd{\vb x(u, v)} \]
	And so we can compute the left hand side of $(\dagger)$:
	\[ \oint_{\partial A} P \dd{u} + Q \dd{v} = \oint_{\partial S} \vb F \cdot \dd{\vb x} \]
	For the right hand side, we must first compute some derivatives.
	\[ Q = F_i(\vb x(u, v))\pdv{x_i}{v} \implies \pdv{Q}{u} = \pdv{x_j}{u}\pdv{F_i}{x_j}\pdv{x_i}{v} + F_i \pdv{x_i}{u}{v} \]
	\[ P = F_i(\vb x(u, v))\pdv{x_i}{u} \implies \pdv{Q}{v} = \pdv{x_j}{v}\pdv{F_i}{x_j}\pdv{x_i}{u} + F_i \pdv{x_i}{v}{u} \]
	Hence
	\begin{align*}
		\pdv{Q}{u} - \pdv{P}{v} & = \left( \pdv{x_i}{v}\pdv{x_j}{u} - \pdv{x_i}{u}\pdv{x_j}{v} \right) \pdv{F_i}{x_j}                       \\
		                        & = \left( \delta_{ip} \delta_{jq} - \delta_{iq} \delta_{jp} \right) \pdv{F_i}{x_j}\pdv{x_p}{v}\pdv{x_q}{u} \\
		                        & = \varepsilon_{ijk} \varepsilon_{pqk} \pdv{F_i}{x_j}\pdv{x_p}{v}\pdv{x_q}{u}                              \\
		                        & = \left[ -\curl{\vb F} \right]_k \left( -\pdv{\vb x}{u} \times \pdv{\vb x}{v} \right)_k                   \\
		                        & = \left( \curl{\vb F} \right) \cdot \left( \pdv{\vb x}{u} \times \pdv{\vb x}{v} \right)                   \\
	\end{align*}
	Therefore,
	\[ \iint_{A} \left( \pdv{Q}{u} - \pdv{P}{v} \right) \dd{u}\dd{v} = \iint_{A} \left( \curl{\vb F} \right) \cdot \left( \pdv{\vb x}{u} \times \pdv{\vb x}{v} \right) = \iint_{S} (\curl{\vb F}) \cdot \dd{\vb S} \]
	which gives Stokes' theorem as required.
\end{proof}

\section{Power Series}
\subsection{The Number \(e\)}
We define
\[
	e = 1 + \frac{1}{1!} + \frac{1}{2!} + \frac{1}{3!} + \frac{1}{4!} + \cdots
\]
The partial sums are increasing and bounded above by the powers of two after the first term, so it converges.

\subsection{Algebraic Numbers}
A real \(x\) is called algebraic if it is a root of a nonzero polynomial with integer coefficients.
Otherwise, it is called transcendental.
For example, any rational \(\frac{p}{q}\) is algebraic as it is the root of \(qx-p=0\).
As another example, \(\sqrt 2 + 1\) is algebraic as it is a root of the equation \(x^2 - 2x - 1 = 0\).
The logical next question to ask is whether all reals are algebraic.

\begin{proposition}
	\(e\) is not rational.
\end{proposition}
\begin{proof}
	Suppose that \(e\) is rational, let it be written \(\frac{p}{q}\), where \(q > 1\) (if \(q=1\), rewrite it as \(\frac{2p}{2q}\)).
	Multiplying up by \(q! \) (easier than just \(q\) because then we can compare factorials) gives
	\[
		\sum_{n=0}^\infty \frac{q!}{n!} \in \mathbb Z
	\]
	We know that \(\sum_{n=0}^q \frac{q!}{n!} \in \mathbb Z\).
	The next terms are:
	\begin{align*}
		\frac{q!}{(q+1)!} & = \frac{1}{q+1}                                    \\
		\frac{q!}{(q+2)!} & = \frac{1}{(q+1)(q+2)} \leq \frac{1}{(q+1)^2}      \\
		\frac{q!}{(q+3)!} & = \frac{1}{(q+1)(q+2)(q+3)} \leq \frac{1}{(q+1)^3} \\
		\frac{q!}{(q+n)!} & \leq \frac{1}{(q+1)^n}                             \\
	\end{align*}
	So the next partial sums are bounded above by the geometric series.
	\[
		\sum_{n=q+1}^\infty \frac{q!}{n!} \leq \frac{1}{q} < 1
	\]
	So the whole series multiplied by \(q! \) is a whole number plus a fractional part, which is not an integer \contradiction.
\end{proof}
Ideally now we'd have a proof that \(e\) is transcendental.
However, even though the terms of \(e\) tend to zero quickly, they don't tend to zero quite quickly enough for us to be able to prove it using what we know now.
We instead prove that there exists some transcendental number using a different example, one whose terms tend to zero very quickly indeed.
\begin{theorem}
	Liouville's constant \(c = \sum_{n=1}^\infty \frac{1}{10^{n!}}\) is transcendental.
	As a decimal expansion:
	\[
		c = 0.1100010000000000000000010\cdots
	\]
\end{theorem}
This is a long proof, the hardest in this course.
We will cherry-pick some important results about polynomials in order to make this proof, without a proper introduction to features of polynomials.
\begin{itemize}
	\item For any polynomial \(P\), \(\exists k \in \mathbb R\) such that \(\abs{P(x) - P(y)} \leq k\abs{x-y}\) for all \(0 \leq x, y \leq 1\).
	      Indeed, say \(P(x) = a_d x^d + \cdots + a_0\), then
	      \begin{align*}
		      P(x) - P(y)       & = a_d(x^d - y^d) + a_{d-1}(x^{d-1} - y^{d-1}) + \cdots + a_1(x-y)     \\
		                        & = (x-y) [ a_d(x^{d-1} + x^{d-2}y + \cdots + y^{d-1}) + \cdots + a_1 ] \\
		      \abs{P(x) - P(y)} & \leq \abs{x-y} [ (\abs{a_d} + \abs{a_{d-1}} + \cdots + \abs{a_1})d ]
	      \end{align*}
	      because \(x\) and \(y\) are between 0 and 1.
	\item A nonzero polynomial of degree \(d\) has at most \(d\) roots.
	      Given some polynomial \(P\) of degree \(d\):
	      \begin{itemize}
		      \item If \(P\) has no roots, we are trivially done.
		      \item If \(P\) has some root \(a\), then \(P\) can be written as \((x-a)Q(x)\).
		            Inductively, \(Q(x)\) has at most \(d-1\) roots, so \(P\) has at most \(d\) roots.
	      \end{itemize}
\end{itemize}
Now we can prove the above theorem.
\begin{proof}
	We will write \(c_n = \sum_{k=0}^n \frac{1}{10^{k!}}\), such that \(c_n \to c\).
	Suppose that some polynomial \(P\) has \(c\) as a root.
	Then \(\exists k\) such that \(\abs{P(x) - P(y)} \leq k\abs{x-y}\) when \(0 \leq x, y \leq 1\).
	Let \(P\) have degree \(d\), such that
	\[
		P(x) = a_d x^d + \cdots + a_0
	\]
	Now, \(\abs{c - c_n} = \sum_{k=n+1}^\infty \frac{1}{10^{k!}} \leq \frac{2}{10^{(n+1)!}}\).
	This is a trivial upper bound, of course better upper bounds exist.

	Also, \(c_n = \frac{a}{10^{n!}}\) for some \(a \in \mathbb Z\).
	So \(P(c_n) = \frac{b}{10^{dn!}}\) for some \(b \in \mathbb Z\) (since \(P(\frac{s}{t}) = \frac{q}{t^d}\) for some integer \(q\), where \(\frac{s}{t} \in \mathbb Q\)).

	For \(n\) large enough, \(c_n\) is not a root, because \(P\) only has finitely many roots.
	So
	\[
		\abs{P(c) - P(c_n)} = \abs{P(c_n)} \leq \frac{1}{10^{dn!}}
	\]
	Therefore
	\[
		\frac{1}{10^{dn!}} \leq k\frac{2}{10^{(n+1)!}}
	\]
	which is a contradiction if \(n\) is large enough.
\end{proof}
Here are some remarks about this proof.
\begin{itemize}
	\item This same proof shows that any real \(x\) such that \(\forall n \exists \frac{p}{q}\in \mathbb Q\) with \(0 < \abs{x - \frac{p}{q}} < \frac{1}{q^n}\) is transcendental.
	      Informally, \(x\) has very good rational approximations.
	\item Such \(x\) are often called Liouville numbers; the proof works for all Liouville numbers.
	\item This proof does not show that \(e\) is transcendental (even though it is), because the terms do not go to zero fast enough.
	\item We now know that there exist some transcendental numbers.
	      Another proof of existence of transcendental numbers will be seen in a later lecture.
\end{itemize}

% This really should be part of lecture 15 but it's here for convenience of ordering.
\subsection{Definition of Complex Numbers}
Some polynomials have no real roots, for example \(x^2 + 1\).
We'll try to `force' an \(x\) with the property \(x^2 = -1\).
Note that for example we could not force an \(x\) into existence wih the property \(x^2=2, x^3=3\); how do we know introducing \(i\) will not lead to a contradiction? We will define \(\mathbb C\) to consist of the plane \(\mathbb R^2\), i.e.\ pairs of real numbers, with operations \(+\) and \(\cdot\) which satisfy:
\begin{itemize}
	\item \((a,b)+(c,d) := (a+c, b+d)\)
	\item \((a,b)\cdot(c,d) := (ac-bd, ad+bc)\)
\end{itemize}
We can view \(\mathbb R\) as being contained within \(\mathbb C\) by identifying the real number \(a\) with \((a, 0)\).
Note that the rules of arithmetic of the reals are inherited inside the first element of the complex plane, so there is no contradiction here.
Then let \(i=(0,1)\).
Trivially then, any point \((a, b)\) in the complex numbers may be written as \(a+bi\) where \(a, b \in \mathbb R\).
And, of course, \(i^2 = -1\).

All of the basic rules like associativity and distributivity work in the complex plane.
There are multiplicative inverses: given \(a+bi\), we know that \((a+bi)(a-bi) = a^2 + b^2\) so \(\frac{a-bi}{a^2 + b^2}\) is the inverse (provided the point is nonzero).
This kind of structure with familiar properties is known as a field, for example \(\mathbb C\), \(\mathbb R\), \(\mathbb Q\), \(\mathbb Z_p\) where \(p\) is prime.
The fundamental theorem of algebra states that any nonzero polynomial with complex coefficients has a complex root; this is proven in the IB course Complex Analysis.

\section{Infinite Differentiability of Power Series}
\subsection{The boundary value problem}
Many problems in mathematical physics can be reduced to the form
\[
	\laplacian{\phi} = F
\]
This is called Poisson's equation.
In the case that \(F \equiv 0\), this is called Laplace's equation.
We are interested in solving this equation on \(\Omega \subseteq \mathbb R^n\) for \(n = 2, 3\).
This is too general to solve at the moment, so we will need to supply boundary conditions, which are very common in physical problems.
In other words, \(\phi\) will be known on \(\partial \Omega\), or as \(\abs{\vb x} \to \infty\) if \(\Omega = \mathbb R^n\).
For instance, the Dirichlet problem is
\[
	\laplacian{\phi} = F \text{ inside } \Omega;\quad \phi = f \text{ on } \partial \Omega
\]
The Neumann problem is
\[
	\laplacian{\phi} = F \text{ inside } \Omega;\quad \pdv{\phi}{\vb n} = g \text{ on } \partial \Omega
\]
where \(\vb n\) is the normal to the surface, and \(\pdv{\phi}{\vb n} := \vb n \cdot \grad{\phi}\).
As a further restriction, we must interpret the boundary conditions in an `appropriate' manner; we assume that \(\phi\) (or \(\pdv{\phi}{\vb n}\)) approaches the behaviour at the boundary continuously as \(\vb x \to \partial \Omega\).
More precisely, \(\phi\) and \(\grad{\phi}\) are continuous on \(\Omega \cup \partial\Omega\).
Note that if we are solving some equation \(\laplacian{\phi} = 0\) in \(\Omega\), we must be certain that \(\phi\) is actually well-defined on the entire set.
As a worked example, consider
\[
	\laplacian{\phi} = r \text{ inside } \left\{ r < a \right\};\quad \phi = 1 \text{ on } \left\{ r = a \right\}
\]
We might guess that the solution is of the form \(\phi(r)\).
We can use the formula
\[
	\laplacian{\phi} = \frac{1}{r^2} \dv{r}\left( r^2 \dv{\phi}{r} \right)
\]
to get
\[
	r^3 = \dv{r}\left( r^2 \dv{\phi}{r} \right) \text{ inside } \left\{ r < a \right\};\quad \phi(a) = 1
\]
The general solution to the first part is
\[
	\phi(r) = A + \frac{B}{r} + \frac{1}{12}r^3
\]
The \(\frac{B}{r}\) term is \textit{not} well-defined inside \(\left\{ r < a \right\}\), therefore \(B=0\) to eliminate the problematic term.
By the second part, we can solve for \(A\):
\[
	1 = \phi(a) = A + \frac{1}{12}a^3 \implies A = 1 - \frac{1}{12}a^3
\]
Hence the solution is
\[
	\phi(r) = 1 + \frac{1}{12}\left(r^3 - a^3\right)
\]

\subsection{Uniqueness of solutions}
When solving Poisson's or Laplace's equation, we want to ensure that the solution we find is unique.
If it is unique, then we can apply similar logic to solving differential equations, where we can guess the form of an equation and then derive the solution from that, and we don't need to worry about solutions that do not have this form.
Consider a generic linear problem
\begin{equation}
	L\phi = F \text{ in } \Omega;\quad B \phi = f \text{ on } \partial\Omega \tag{\(\dagger\)}
\end{equation}
where \(L\) and \(B\) are linear differential operators.
If \(\phi_1\) and \(\phi_2\) are both solutions to \((\dagger)\), then consider \(\psi = \phi_1 - \phi_2\).
By linearity,
\[
	L\psi = L\phi_1 - L\phi_2 = F - F = 0 \text{ in } \Omega
\]
and
\[
	B\psi = B\phi_1 - B\phi_2 = f - f = 0 \text{ on } \partial\Omega
\]
If we can show that the only solution to these new equations is \(\psi = 0\), we must conclude that \(\phi_1 = \phi_2\), which means that there is only one solution to \((\dagger)\).
Hence the solution to a linear problem is unique if and only if the only solution to the homogeneous problem is zero.

\begin{proposition}
	The solution to the Dirichlet problem is unique.
	The solution to the Neumann problem is unique up to the addition of an arbitrary constant.
\end{proposition}
\begin{proof}
	Let \(\psi = \phi_1 - \phi_2\) be the difference between two solutions.
	In the Dirichlet case, we want to show that \(\psi = 0\), and in the Neumann case, we want to show that \(\psi\) is an arbitrary constant.
	We know that
	\[
		\laplacian{\psi} = 0 \text{ in } \Omega;\quad B\psi = 0 \text{ on } \partial\Omega
	\]
	where \(B\psi = \psi\) in the Dirichlet problem, or \(B\psi = \pdv{\psi}{\vb n}\) in the Neumann problem.
	Consider the non-negative functional
	\[
		I[\psi] = \int_\Omega \abs{\grad{\psi}}^2 \dd{V} \geq 0
	\]
	Clearly,
	\[
		I[\psi] = 0 \iff \grad{\psi} = 0 \text{ everywhere in } \Omega
	\]
	Now, note that we can apply the divergence theorem to get
	\begin{align*}
		I[\psi] & = \int_\Omega \abs{\grad{\psi}}^2 \dd{V}                                          \\
		        & = \int_\Omega \grad{\psi} \cdot \grad{\psi} \dd{V}                                \\
		        & = \int_\Omega \left( \div(\psi \grad{\psi}) - \psi\laplacian{\psi} \right) \dd{V} \\
		        & = \int_\Omega \div(\psi \grad{\psi}) \dd{V}                                       \\
		        & = \int_{\partial\Omega} \psi \grad{\psi} \cdot\dd{\vb S}                          \\
		        & = \int_{\partial\Omega} \psi \grad{\psi} \cdot \vb n \dd{S}                       \\
		        & = \int_{\partial\Omega} \psi \dv{\psi}{\vb n} \dd{S}                              \\
	\end{align*}
	In the Dirichlet case, \(I[\psi] = 0\) since \(\psi = 0\) on the boundary.
	In the Neumann case, \(I[\psi] = 0\) as well, since \(\dv{\psi}{\vb n} = 0\).
	Hence, in either case, \(\grad{\psi} = 0\) everywhere in \(\Omega\).
	Therefore, \(\psi\) is a constant throughout \(\Omega\).
	In the Dirichlet case, we know that \(\psi = 0\) on the boundary, hence \(\psi = 0\) everywhere as it is continuous.
	However, in the Neumann problem, no such deduction can be made.
\end{proof}
\noindent Here is an example from electrostatics.
Consider the charge density \(\rho\) defined by
\[
	\rho(\vb x) = \begin{cases}
		0    & r < a    \\
		F(r) & r \geq a
	\end{cases}
\]
We can show that there is no electric field in the region \(r < a\).
We know that the electric potential \(\phi\) will satisfy
\[
	\laplacian{\phi} = \frac{-\rho(\vb x)}{\varepsilon_0} = 0 \text{ if } r<a
\]
By symmetry, we will try a \(\phi\) of the form \(\phi(r)\).
Hence, \(\phi(a)\) is constant on the boundary \(r=a\).
Note that the unique solution to
\[
	\laplacian{\phi} = 0 \text{ for } r<a;\quad \phi = \text{constant on } r = a
\]
is exactly that \(\phi\) is constant everywhere.
Hence
\[
	\vb E = -\grad{\psi} = 0 \text{ throughout } r<a
\]
This can be viewed as a version of Newton's shell theorem.

\section{Exponents, Logarithms and Powers}
\subsection{Expectation}
Consider a continuous random variable \(X \colon \Omega \to \mathbb R\), with probability distribution function \(F(x)\) and probability density function \(f(x) = F'(x)\).
We define the expectation of such a \textit{non-negative} random variable as
\[
	\expect{X} = \int_0^\infty x f(x) \dd{x}
\]
In this case, the expectation is either non-negative and finite, or positive infinity.
Now, let \(X\) be a general continuous random variable, that is not necessarily non-negative.
Suppose \(g \geq 0\).
Then,
\[
	\expect{g(X)} = \int_{-\infty}^\infty g(x) f(x) \dd{x}
\]
We can define \(X_+ = \max(X, 0)\) and \(X_- = \min(-X, 0)\).
If at least one of \(\expect{X_+}\) or \(\expect{X_-}\) is finite, then clearly
\[
	\expect{X} := \expect{X_+} - \expect{X_-} = \int_{-\infty}^\infty xf(x) \dd{x}
\]
It is easy to verify that the expectation is a linear function, due to the linearity property of the integral.

\subsection{Computing the Expectation}
\begin{claim}
	Let \(X \geq 0\).
	Then
	\[
		\expect{X} = \int_0^\infty \prob{X \geq x} \dd{x}
	\]
\end{claim}
\begin{proof}
	Using the definition of the expectation,
	\begin{align*}
		\expect{X} & = \int_0^\infty xf(x) \dd{x}                                           \\
		           & = \int_0^\infty \left( \int_0^x \dd{y} \right) f(x) \dd{x}             \\
		           & = \int_0^x \dd{y} \int_y^\infty f(x) \dd{x}                            \\
		           & = \int_0^\infty \dd{y} \left( 1 - \int_{-\infty}^y f(x) \dd{x} \right) \\
		           & = \int_0^\infty \dd{y} \prob{X \geq y}
	\end{align*}
\end{proof}
\noindent Here is an alternative proof.
\begin{proof}
	For every \(\omega \in \Omega\), we can write
	\[
		X(\omega) = \int_0^\infty 1(X(\omega) \geq x) \dd{x}
	\]
	Taking expectations, we get
	\[
		\expect{X} = \expect{\int_0^\infty 1(X(\omega) \geq x) \dd{x}}
	\]
	We will interchange the integral and the expectation, although this step is not justified or rigorous.
	\begin{align*}
		\expect{X} & = \int_0^\infty \expect{1(X(\omega) \geq x)} \dd{x} \\
		           & = \int_0^\infty \prob{X \geq x} \dd{x}
	\end{align*}
\end{proof}

\subsection{Variance}
We define the variance of a continuous random variable as
\[
	\Var{X} = \expect{(X - \expect{X})^2} = \expect{X^2} - \expect{X}^2
\]

\subsection{Uniform Distribution}
Consider the uniform distribution defined by \(a, b \in\mathbb R\).
\[
	f(x) = \begin{cases}
		\frac{1}{b-a} & x \in [a, b]     \\
		0             & \text{otherwise}
	\end{cases}
\]
We write \(X \sim U[a, b]\).
For some \(x \in [a,b]\), we can write
\[
	\prob{X \leq x} = \int_a^x f(y) \dd{y} = \frac{x-a}{b-a}
\]
Hence, for \(x \in [a,b]\),
\[
	F(x) = \begin{cases}
		1               & x > b       \\
		\frac{x-a}{b-a} & x \in [a,b] \\
		0               & x < a
	\end{cases}
\]
Then,
\[
	\expect{X} = \int_a^b \frac{x}{b-a} \dd{x} = \frac{a+b}{2}
\]

\subsection{Exponential Distribution}
The exponential distribution is defined by \(f(x) = \lambda e^{-\lambda x}\) for \(\lambda > 0\), \(x > 0\).
We write \(X \sim \mathrm{Exp}(\lambda)\).
\[
	F(x) = \prob{X \geq x} = \int_0^x \lambda e^{-\lambda y} \dd{y} = 1 - e^{-\lambda x}
\]
Further,
\[
	\expect{X} = \int_0^\infty \lambda x e^{-\lambda x} \dd{x} = \frac{1}{\lambda}
\]
We can view the exponential distribution as a limit of geometric distributions.
Suppose that \(T \sim \mathrm{Exp}(\lambda)\), and let \(T_n = \floor{nT}\) for all \(n \in \mathbb N\).
We have
\[
	\prob{T_n \geq k} = \prob{T \geq \frac{k}{n}} = e^{-\lambda k / n} = \left( e^{-\lambda/n} \right)^k
\]
Hence \(T_n\) is a geometric distribution with parameter \(p_n = e^{-\lambda/n}\).
As \(n \to \infty\), \(p_n \sim \frac{\lambda}{n}\), and \(\frac{T_n}{n} \sim T\).
Hence the exponential distribution is the limit of a scaled version of the geometric distribution.
A key property of the exponential distribution is that it has no memory.
If \(T \sim \mathrm{Exp}(\lambda)\), \(\prob{T > t + s \mid T > s} = \prob{T > t}\).
In fact, the distribution is uniquely characterised by this property.
\begin{proposition}
	Let \(T\) be a positive continuous random variable not identically zero or infinity.
	Then \(T\) has the memoryless property \(\prob{T > t + s \mid T > s} = \prob{T > t}\) if and only if \(T \sim \mathrm{Exp}(\lambda)\) for some \(\lambda > 0\).
\end{proposition}
\begin{proof}
	Clearly if \(T \sim \mathrm{Exp}(\lambda)\), then \(\prob{T > t + s \mid T > s} = e^{-\lambda t} = \prob{T > t}\) as required.
	Now, given that \(T\) has this memoryless property, for all \(s\) and \(t\), we have \(\prob{T > t + s} = \prob{T > t} \prob{T > s}\).
	Let \(g(t) = \prob{T > t}\); we would like to show that \(g(t) = e^{-\lambda t}\).
	Then \(g\) satisfies \(g(t+s) = g(t)g(s)\).
	Then for all \(m \in \mathbb N\), \(g(mt) = (g(t))^m\).
	Setting \(t=1\), \(g(m) = g(1)^m\).
	Now, \(g(m/n)^n = g(mn/n) = g(m)\) hence \(g(m/n) = g(1)^{m/n}\).
	So for all rational numbers \(q \in \mathbb Q\), \(g(q) = g(1)^q\).
	
	Now, \(g(1) = \prob{T > 1} \in (0, 1)\).
	Indeed, \(g(1) \neq 0\) since in this case, for any rational number \(q\) we would have \(g(q) = 0\) contradicting the assumption that \(T\) was not identically zero, and \(g(1) \neq \infty\) because in this case \(T\) would be identically infinity.
	Now, let \(\lambda = -\log\prob{T > 1} > 0\).
	We have now proven that \(g(t) = e^{-\lambda t}\) for all \(t\in\mathbb Q\).
	
	Let \(t \in \mathbb R_+\).
	Then for all \(\varepsilon > 0\), there exist \(r, s \in \mathbb Q\) such that \(r \leq t \leq s\) and \(\abs{r - s} \leq \varepsilon\).
	In this case, \(e^{-\lambda s} = \prob{T > s} \leq \prob{T > t} \leq \prob{T > r} = e^{-\lambda r}\).
	Sending \(\varepsilon \to 0\) finishes the proof, showing that \(g(t) = e^{-\lambda t}\) for all positive reals.
\end{proof}

\subsection{Functions of Continuous Random Variables}
\begin{theorem}
	Suppose that \(X\) is a continuous random variable with density \(f\).
	Let \(g\) be a monotonic continuous function (either strictly increasing or strictly decreasing), such that \(g^{-1}\) is differentiable.
	Then \(g(X)\) is a continuous random variable with density \(fg^{-1}(x) \abs{\dv{x} g^{-1}(x)}\).
\end{theorem}
\begin{proof}
	Suppose that \(g\) is strictly increasing.
	We have
	\[
		\prob{g(X) \leq x} = \prob{X \leq g^{-1}(x)} = F(g^{-1}(x))
	\]
	Hence,
	\[
		\dv{x} \prob{g(X) \leq x} = F'(g^{-1}(x)) \cdot \dv{x} g^{-1}(x) = f(g^{-1}(x)) \dv{x}g^{-1}(x)
	\]
	Note that since \(g\) is strictly increasing, so is \(g^{-1}\).
	Now, suppose the \(g\) is strictly decreasing.
	Since the random variable is continuous,
	\[
		\prob{g(X) \leq x} = \prob{X \geq g^{-1}(x)} = 1 - F(g^{-1}(x))
	\]
	Hence,
	\[
		\dv{x} \prob{g(X) \leq x} = -F'(g^{-1}(x)) \cdot \dv{x} g^{-1}(x) = f(g^{-1}(x)) \abs{\dv{x}g^{-1}(x)}
	\]
	Likewise, in this case, \(g\) is strictly decreasing.
\end{proof}

\subsection{Normal Distribution}
The normal distribution is characterised by \(\mu \in \mathbb R\) and \(\sigma > 0\).
We define
\[
	f(x) = \frac{1}{\sqrt{2 \pi \sigma^2}} \exp\qty{-\frac{(x-\mu)^2}{2\sigma^2}}
\]
\(f(x)\) is indeed a probability density function:
\[
	I = \int_{-\infty}^\infty f(x) \dd{x} = \int_{-\infty}^\infty \frac{1}{\sqrt{2 \pi \sigma^2}} \exp\qty{-\frac{(x-\mu)^2}{2\sigma^2}} \dd{x}
\]
Applying the substitution \(x \mapsto \frac{x-\mu}{\sigma}\), we have
\[
	I = \frac{1}{\sqrt{2 \pi}} \int_{-\infty}^\infty \exp\qty{-\frac{x^2}{2}} \dd{x}
\]
We can evaluate this integral by considering \(I^2\).
\[
	I^2 = \frac{2}{\pi} \int_0^\infty \int_0^\infty e^{\frac{-(u^2 - v^2)}{2}} \dd{u}\dd{v}
\]
Using polar coordinates \(u = r\cos\theta\) and \(v = r\sin\theta\), we have
\[
	I^2 = \frac{2}{\pi} \int_0^\infty \dd{r} \int_0^{\frac{\pi}{2}} \dd{\theta} re^{-\frac{r^2}{2}} = 1 \implies I = \pm 1
\]
But clearly \(I > 0\), so \(I=1\).
Hence \(f\) really is a probability density function.
Now, if \(X \sim \mathrm{N}(\mu, \sigma^2)\),
\begin{align*}
	\expect{X} & = \int_{-\infty}^{\infty} \frac{x}{\sqrt{2\pi\sigma^2}} \exp\qty{-\frac{(x-\mu)^2}{2\sigma^2}} \dd{x}                                                                                                                                                                                                                      \\
	           & = \underbrace{\int_{-\infty}^{\infty} \frac{x - \mu}{\sqrt{2\pi\sigma^2}} \exp\qty{-\frac{(x-\mu)^2}{2\sigma^2}} \dd{x}}_{\text{odd function around } \mu \text{ hence } 0} + \mu\underbrace{\int_{-\infty}^{\infty} \frac{1}{\sqrt{2\pi\sigma^2}} \exp\qty{-\frac{(x-\mu)^2}{2\sigma^2}} \dd{x}}_{I = 1 \text{ by above}} \\
	           & = \mu                                                                                                                                                                                                                                                                                                                      \\
\end{align*}
We can also compute the variance, using the substitution \(u = \frac{x - \mu}{\sigma}\), giving
\begin{align*}
	\Var{X} & = \int_{-\infty}^{\infty} \frac{(x - \mu)^2}{\sqrt{2\pi\sigma^2}} \exp\qty{-\frac{(x-\mu)^2}{2\sigma^2}} \dd{x} \\
	        & = \sigma^2 \int_{-\infty}^{\infty} \frac{u^2}{\sqrt{2\pi}} \exp\qty{-\frac{u^2}{2}} \dd{u}                      \\
	        & = \sigma^2
\end{align*}
In particular, when \(\mu = 0\) and \(\sigma^2 = 1\), we call the distribution \(\mathrm{N}(\mu, \sigma^2) = \mathrm{N}(0, 1)\) the standard normal distribution.
We define
\[
	\Phi(x) = \int_{-\infty}^x \frac{1}{\sqrt{2\pi}} e^{-\frac{u^2}{2}} \dd{u};\quad \phi(x) = \Phi'(x) = \frac{1}{\sqrt{2\pi}} e^{-\frac{x^2}{2}}
\]
Hence \(\Phi(x) = \prob{X \leq x}\) if \(X\) has the standard normal distribution.
Since \(\phi(x) = \phi(-x)\), we have \(\Phi(x) + \Phi(-x) = 1\), hence \(\prob{X \leq x} = 1 - \prob{X \leq -x}\).

\section{Trigonometric Functions}
\subsection{Computing Binomial Coefficients}
\begin{proposition}
	\[
		\binom{n}{k} = \frac{n(n-1)(n-2)\cdots(n-k+1)}{k(k-1)(k-2)\cdots(1)}
	\]
\end{proposition}
\begin{proof}
	The number of ways to name a \(k\)-set is \(n(n-1)(n-2)\cdots(n-k+1)\) because there are \(n\) ways to choose a first element, \(n-1\) ways to choose a second element, and so on.
	We have overcounted the \(k\)-sets, though --- there are \(k(k-1)(k-2)\cdots(1)\) ways to name a given \(k\)-set because you have \(k\) choices for the first element, \(k-1\) choices for the second element, and so on.
	Hence the number of \(k\)-sets in \(\{ 1, 2, \dots, n \}\) is the required result.
\end{proof}
Note that we can also write
\[
	\binom{n}{k} = \frac{n!}{k!(n-k)!}
\]
but this is a very unwieldy formula to use especially by hand, so will be rarely used.
Further, we can make asymptotic approximations using this formula, for example \(\binom{n}{3} \sim \frac{n^3}{6}\) for large \(n\).

\subsection{Binomial Theorem}
\begin{theorem}
	For all \(a, b \in \mathbb R, n \in \mathbb N\), we have
	\[
		(a+b)^n = \binom{n}{0}a^n + \binom{n}{1}a^{n-1}b + \binom{n}{2}a^{n-2}b^2 + \dots + \binom{n}{n}b^n
	\]
\end{theorem}
\begin{proof}
	When we expand \((a+b)^n = (a+b)(a+b)\dots(a+b)\), we obtain terms of the form \(a^k b^{n-k}\).
	To get a single term of this form, we must choose \(k\) brackets for which to take the \(a\) value in the expansion, and the other \(n-k\) brackets will take the \(b\) value.
	The number of terms of the form \(a^k b^{n-k}\) for a fixed \(k\) is therefore the amount of ways of choosing \(k\) brackets out of a total of \(n\), which is \(\binom{n}{k}\).
	So
	\[
		(a+b)^n = \sum_{k=0}^n \binom{n}{k}a^k b^{n-k} = \sum_{k=0}^n \binom{n}{n-k}a^k b^{n-k}
	\]
\end{proof}
For example, we can tell that \((1+x)^n\) reduces to
\[
	1 + nx + \frac{1}{2}n(n-1)x^2 + \frac{1}{3!}n(n-1)(n-2)x^3 + \dots + nx^{n-1} + x^n
\]
So when \(x\) is small, a good approximation to \((1+x)^n\) is \(1 + nx\).

\subsection{Inclusion-Exclusion Theorem}
Given two finite sets \(A\), \(B\), we have
\[
	\abs{A \cup B} = \abs{A} + \abs{B} - \abs{A \cap B}
\]
For three sets, we have
\[
	\abs{A \cup B \cup C} = \abs{A} + \abs{B} + \abs{C} - \abs{A \cap B} - \abs{B \cap C} - \abs{C \cap A} + \abs{A \cap B \cap C}
\]
\begin{theorem}[Inclusion-Exclusion Theorem]
	Let \(S_1, \dots, S_n\) be finite sets.
	Then,
	\[
		\abs{\bigcup_{S \in S_n} S} = \sum_{\abs{A} = 1}\abs{S_A} - \sum_{\abs{A} = 2}\abs{S_A} + \sum_{\abs{A} = 3}\abs{S_A} - \cdots
	\]
	where
	\[
		S_a = \bigcap_{i \in A}S_i
	\]
	and
	\[
		\sum_{\abs{A} = k}
	\]
	is a sum taken over all \(A \subseteq \{ 1, 2, \dots, n \}\) of size \(k\).
\end{theorem}
\begin{proof}
	Let \(x\) be an element of the left hand side.
	We wish to prove that \(x\) is counted exactly once on the right hand side.
	Without loss of generality, let us rename the sets that \(x\) belongs to as \(S_1, S_2, dots, S_k\).

	Then the number of sets \(A\) with \(\abs{A} = 1\) such that \(x \in S_A\) is \(k\).
	The number of sets \(A\) with \(\abs{A} = 2\) such that \(x \in S_a\) is \(\binom{k}{2}\), since we must choose two of the sets \(S_1, \dots, S_k\), so there are \(\binom{k}{2}\) ways to do this.
	So in general, the amount of \(A\) with \(\abs{A} = r\) with \(x \in S_A\) is just \(\binom{k}{r}\).

	So the number of times \(x\) is counted on the right hand side is
	\[
		k - \binom{k}{2} + \binom{k}{3} - \dots + (-1)^{k+1}\binom{k}{k}
	\]
	But \((1 + (-1))^k\) by the binomial expansion is
	\[
		1 - \binom{k}{1} + \binom{k}{2} - \binom{k}{3} + \dots + (-1)^k\binom{k}{k}
	\]
	So the number of times \(x\) is counted on the right hand side is \(1 - (1 + (-1))^k = 1 - 0 = 1\).
\end{proof}

\subsection{Functions}
For sets \(A\) and \(B\), a function \(f\) from \(A\) to \(B\) is a rule that assigns to each \(x \in A\) a unique value \(f(x) \in B\).
More precisely, a function from \(A\) to \(B\) is a set \(f \subseteq A \times B\) such that for every \(x \in A\), there is a unique \(y \in B\) with \((x, y) \in f\).
Of course therefore, if \((x, y) \in f\) then we can write \(f(x) = y\).
Here are some examples.
\begin{enumerate}
	\item \(f\colon \mathbb R \to \mathbb R\) given by \(f(x) = x^2\), or using an alternative notation, \(x \mapsto x^2\) is a function.
	\item A non-example is \(f\colon \mathbb R \to \mathbb R\) given by \(f(x) = \frac{1}{x}\) since it is undefined at \(x=0\).
	\item Another non-example is \(f\colon \mathbb R \to \mathbb R\) given by \(f(x) = \pm \sqrt{\abs{x}}\) since it does not define a unique value in the output space for a given input, such as \(x=2\).
	\item \(f\colon \mathbb R \to \mathbb R\) given by
	      \[
		      f(x) = \begin{cases}
			      1 & x \in \mathbb Q  \\
			      0 & \text{otherwise}
		      \end{cases}
	      \]
	      is a function since it clearly satisfies the second definition.
	      Note that even though we don't know if \(e + \pi\) is rational or not, the function is still well defined since it produces a unique solution for \(f(e + \pi)\), we just don't know which output value it gives.
	\item \(A = \{ 1, 2, 3, 4, 5 \}\), \(B = \{ 1, 2, 3, 4 \}\), and \(f\colon A \to B\) is given by
	      \begin{align*}
		      f(1) & = 1 \\
		      f(2) & = 4 \\
		      f(3) & = 3 \\
		      f(4) & = 3 \\
		      f(5) & = 4
	      \end{align*}
	\item \(A = \{ 1, 2, 3 \}\), \(f\colon A \to A\) is given by
	      \begin{align*}
		      f(1) & = 1 \\
		      f(2) & = 3 \\
		      f(3) & = 2
	      \end{align*}
	\item \(A = \{ 1, 2, 3, 4 \}\), \(f\colon A \to A\) is given by
	      \begin{align*}
		      f(1) & = 1 \\
		      f(2) & = 3 \\
		      f(3) & = 3 \\
		      f(4) & = 4
	      \end{align*}
	\item \(A = \{ 1, 2, 3, 4 \}\), \(B = \{ 1, 2, 3 \}\), \(f\colon A \to B\) is given by
	      \begin{align*}
		      f(1) & = 3 \\
		      f(2) & = 3 \\
		      f(3) & = 2 \\
		      f(4) & = 1
	      \end{align*}
\end{enumerate}

\section{Integration}
\subsection{Conditional density}
We will now define the conditional density of a continuous random variable, given the value of another continuous random variable.
Let \(X\) and \(Y\) be continuous random variables with joint density \(f_{X, Y}\) and marginal densities \(f_X\) and \(f_Y\).
Then we define the conditional density of \(X\) given that \(Y = y\) is defined as
\[
	f_{X \mid Y}(x \mid y) = \frac{f_{X, Y}(x, y)}{f_Y(y)}
\]
Then we can find the law of total probability in the continuous case.
\begin{align*}
	f_X(x) & = \int_{-\infty}^\infty f_{XY}(x, y) \dd{y}                 \\
	       & = \int_{-\infty}^\infty f_{X \mid Y}(x \mid y)f_Y(y) \dd{y}
\end{align*}

\subsection{Conditional expectation}
We want to define \(\expect{X \mid Y}\) to be some function \(g(Y)\) for some function \(g\).
We will define
\[
	g(y) = \int_{-\infty}^\infty xf_{X \mid Y}(x \mid y) \dd{x}
\]
which is the analogous expression to \(\expect{X \mid Y = y}\) from the discrete case.
Then we just set \(\expect{X \mid Y} = g(Y)\) to be the conditional expectation.

\subsection{Transformations of multidimensional random variables}
\begin{theorem}
	Let \(X\) be a continuous random variable with values in \(D \subseteq \mathbb R^d\), with density \(f_X\).
	Now, let \(g\) be a bijection \(D\) to \(g(D)\) which has a continuous derivative, and \(\det g'(x) \neq 0\) for all \(x \in D\).
	Then the random variable \(Y = g(X)\) has density
	\[
		f_Y(y) = f_X(x) \cdot \abs{J} \text{ where } x = g^{-1}(y)
	\]
	where \(J\) is the Jacobian
	\[
		J = \det \left( \left( \pdv{x_i}{y_j} \right)_{i, j = 1}^d \right)
	\]
\end{theorem}
\noindent No proof will be given for this theorem.
As an example, let \(X\) and \(Y\) be independent continuous random variables with the standard normal distribution.
The point \((X, Y)\) in \(\mathbb R^2\) has polar coordinates \((R, \Theta)\).
What are the densities of \(R\) and \(\Theta\)?
We have \(X = R\cos\Theta\) and \(Y = R\sin\Theta\).
The Jacobian is
\[
	J = \det\begin{pmatrix}
		\cos\theta & -r\sin\theta \\
		\sin\theta & r\cos\theta
	\end{pmatrix} = r
\]
Hence,
\begin{align*}
	f_{R, \Theta}(r, \theta) & = f_{X, Y}(r\cos\theta, r\sin\theta) \abs{J}                                                                            \\
	                         & = f_{X, Y}(r\cos\theta, r\sin\theta) r                                                                                  \\
	                         & = f_X(r\cos\theta) f_Y(r\sin\theta) r                                                                                   \\
	                         & = \frac{1}{\sqrt{2\pi}}e^{-\frac{r^2\cos^2\theta}{2}} \cdot \frac{1}{\sqrt{2\pi}}e^{-\frac{r^2\sin^2\theta}{2}} \cdot r \\
	                         & = \frac{1}{2\pi}e^{-\frac{r^2}{2}} \cdot r
\end{align*}
for all \(r > 0\) and \(\theta \in [0, 2\pi]\).
Note that the joint density factorises into marginal densities:
\[
	f_{R, \Theta}(r, \theta) = \underbrace{\frac{1}{2\pi}}_{f_\Theta} \underbrace{re^{-\frac{r^2}{2}}}_{f_R}
\]
so the random variables \(R\) and \(\Theta\) are independent, where \(\Theta \sim U[0, 2\pi]\) and \(R\) has density \(re^{\frac{-r^2}{2}}\) on \((0, \infty)\).

\subsection{Ordered statistics of a random sample}
Let \(X_1, \dots, X_n\) be independent and identically distributed random variables with distribution function \(F\) and density function \(f\).
We can put them in increasing order:
\[
	X_{(1)} \leq X_{(2)} \leq \dots \leq X_{(n)}
\]
and let \(Y_i = X_{(i)}\).
The \((Y_i)\) are the order statistics.
\begin{align*}
	\prob{Y_1 \leq x} & = \prob{\min(X_1, \dots, X_n) \leq x}    \\
	                  & = 1 - \prob{\min(X_1, \dots, X_n) > x}   \\
	                  & = 1 - \prob{X_1 > x}\cdots\prob{X_n > x} \\
	                  & = 1 - (1 - F(x))^n
\end{align*}
Further,
\begin{align*}
	f_{Y_1}(x) & = \dv{x}\left( 1 - (1 - F(x))^n \right) \\
	           & = n (1 - F(x))^{n-1} f(x)
\end{align*}
We can compute an analogous result for the maximum.
\begin{align*}
	\prob{Y_n \leq x} & = (F(x))^n           \\
	f_{Y_n}(x)        & = n(F(x))^{n-1} f(x)
\end{align*}
What are the densities of the other random variables?
First, let \(x_1 < x_2 < \dots < x_n\).
Then, we can first find the joint distribution \(\prob{Y_1 \leq x_1, \dots, Y_n \leq x_n}\).
Note that this is simply the sum over all possible permutations of the \((X_i)\) of \(\prob{X_1 \leq x_1, \dots, X_n \leq x_n}\).
But since the variables are independent and identically distributed, these probabilities are the same.
Hence,
\begin{align*}
	\prob{Y_1 \leq x_1, \dots, Y_n \leq x_n}        & = n!
	\cdot \prob{X_1 \leq x_1, \dots, X_n \leq x_n, X_1 < \dots < X_n}                                               \\
	                                                & = n!
	\int_{-\infty}^{x_1} \int_{u_1}^{x_2} \cdots \int_{u_{n-1}}^{x_n} f(u_1) \cdots f(u_n) \dd{u_1} \cdots \dd{u_n} \\
	\therefore f_{Y_1, \dots, Y_n}(x_1, \dots, x_n) & = n!
	f(x_1) \cdots f(x_n)
\end{align*}
when \(x_1 < x_2 < \dots < x_n\), and the joint density is zero otherwise.
Note that this joint density does not factorise as a product of densities, since we must always consider the indicator function that \(x_1 < x_2 < \dots < x_n\).

\subsection{Ordered statistics on exponential distribution}
Let \(X \sim \mathrm{Exp}(\lambda)\), \(Y \sim \mathrm{Exp}(\mu)\) be independent continuous random variables.
Let \(Z = \min(X, Y)\).
\[
	\prob{Z \geq z} = \prob{X \geq z, Y \geq z} = \prob{X \geq z} \prob{Y \geq z} = e^{-\lambda z} \cdot e^{-\mu z} = e^{-(\lambda + \mu)z}
\]
Hence \(Z\) has the exponential distribution with parameter \(\lambda+\mu\).
More generally, if \(X_1, \dots, X_n\) are independent continuous random variables with \(X_i \sim \mathrm{Exp}(\lambda_i)\), then \(Z = \min(X_1, \dots, X_n)\) has distribution \(\mathrm{Exp}\left( \sum_{i=1}^n \lambda_i \right)\).
Now, let \(X_1, \dots, X_n\) be independent identically distributed random variables with distribution \(\mathrm{Exp}(\lambda)\), and let \(Y_i\) be their order statistics.
Then
\[
	Z_1 = Y_1;\quad Z_2 = Y_2 - Y_1;\quad Z_i = Y_i - Y_{i-1}
\]
So the \(Z_i\) are the `durations between consecutive results' from the \(X_i\).
What is the density of these \(Z_i\)?
First, note that
\[
	Z = \begin{pmatrix}
		Z_1 \\ \vdots \\ Z_n
	\end{pmatrix} = A \begin{pmatrix}
		Y_1 \\ \vdots \\ Y_n
	\end{pmatrix};\quad A = \begin{pmatrix}
		1      & 0      & 0      & \cdots & 0      \\
		-1     & 1      & 0      & \cdots & 0      \\
		0      & -1     & 1      & \cdots & 0      \\
		\vdots & \vdots & \vdots & \ddots & \vdots \\
		0      & 0      & 0      & \cdots & 1
	\end{pmatrix}
\]
Note that \(\det A = 1\), and \(Z = AY\), and note further that
\[
	y_j = \sum_{i=1}^j z_i
\]
Now,
\begin{align*}
	f_{(Z_1, \dots, Z_n)}(z_1, \dots, z_n) & = f_{(Y_1, \dots, Y_n)}(y_1, \dots, y_n) \underbrace{\abs{A}}_{=1} \\
	                                       & = n!
	f(y_1) \cdots f(y_n)                                                                                        \\
	                                       & = n!
	(\lambda e^{-\lambda y_1}) \cdots (\lambda e^{-\lambda y_n})                                                \\
	                                       & = n!
	\lambda^n e^{-\lambda(nz_1 + (n-1)z_2 + \dots + z_n)}                                                       \\
	                                       & = \prod_{i=1}^n (n-i+1) \lambda e^{-\lambda (n-i+1)z_i}
\end{align*}
The density function of the vector \(Z\) factorises into functions of the \(z_i\), so \(Z_1, \dots, Z_n\) are independent and \(Z_i \sim \mathrm{Exp}(\lambda(n-i+1))\).

\section{Classes of Integrable Functions}
\subsection{Real Eigenvalues and Orthogonal Eigenvectors}
Recall that an \(n\times n\) matrix \(A\) is hermitian if and only if \(A^\dagger = \overline{A}^\transpose = A\), or \(\overline{A_{ij}} = A_{ji}\).
If \(A\) is real, then it is hermitian if and only if it is symmetric.
The complex inner product for \(\vb v, \vb w \in \mathbb C^n\) is \(\vb v^\dagger \vb w = \sum_i \overline{v_i}w_i\), and for \(\vb v, \vb w \in \mathbb R^n\), this reduces to the dot product in \(\mathbb R^n\), \(\vb v^\transpose \vb w\).

Here is a key observation.
If \(A\) is hermitian, then
\[
	(A\vb v)^\dagger \vb w = \vb v^\dagger (A \vb w)
\]
\begin{theorem}
	For an \(n \times n\) matrix \(A\) that is hermitian:
	\begin{enumerate}[(i)]
		\item Every eigenvalue \(\lambda\) is real;
		\item Eigenvectors \(\vb v, \vb w\) with different eigenvalues \(\lambda, \mu\) respectively, are orthogonal, i.e.\ \(\vb v^\dagger \vb w = 0\); and
		\item If \(A\) is real and symmetric, then for each eigenvalue \(\lambda\) we can choose a real eigenvector, and part (ii) becomes \(\vb v \cdot \vb w = 0\).
	\end{enumerate}
\end{theorem}
\begin{proof}
	\begin{enumerate}[(i)]
		\item Using the observation above with \(\vb v = \vb w\) where \(\vb v\) is any eigenvector with eigenvalue \(\lambda\), we get
		      \begin{align*}
			      \vb v^\dagger (A\vb v)        & = (A\vb v)^\dagger \vb v                   \\
			      \vb v^\dagger (\lambda\vb v)  & = (\lambda\vb v)^\dagger \vb v             \\
			      \lambda \vb v^\dagger (\vb v) & = \overline{\lambda} (\vb v)^\dagger \vb v \\
			      \intertext{As \(\vb v\) is an eigenvector, it is nonzero, so \(\vb v^\dagger \vb v \neq 0\), so}
			      \lambda                       & = \overline \lambda
		      \end{align*}
		\item Using the same observation,
		      \begin{align*}
			      \vb v^\dagger (A \vb w)   & = (A \vb v)^\dagger \vb w       \\
			      \vb v^\dagger (\mu \vb w) & = (\lambda \vb v)^\dagger \vb w \\
			      \mu \vb v^\dagger \vb w   & = \lambda bm v^\dagger \vb w
		      \end{align*}
		      Since \(\lambda \neq \mu\), \(\vb v^\dagger \vb w = 0\), so the eigenvectors are orthogonal.
		\item Given \(A\vb v = \lambda \vb v\) with \(\vb v \in \mathbb C^n\) but \(A\) is real, let
		      \[
			      \vb v = \vb u + i\vb u';\quad \vb u, \vb u' \in \mathbb R^n
		      \]
		      Since \(\vb v\) is an eigenvector, and this is a linear equation, we have
		      \[
			      A\vb u = \lambda \vb u;\quad A\vb u' = \lambda \vb u'
		      \]
		      So \(\vb u\) and \(\vb u'\) are eigenvectors.
		      \(\vb v \neq 0\) implies that at least one of \(\vb u\) and \(\vb u'\) are nonzero, so there is at least one real eigenvector with this eigenvalue.
	\end{enumerate}
\end{proof}
Case (ii) is a stronger claim for hermitian matrices than just showing that eigenvectors are linearly independent.
Furthermore, previously we considered bases \(\mathcal B_\lambda\) for each eigenspace \(E_\lambda\), and it is now natural to choose bases \(\mathcal B_\lambda\) to be orthonormal when we are considering hermitian matrices.
Here are some examples.
\begin{enumerate}[(i)]
	\item Let
	      \[
		      A = \begin{pmatrix}
			      2 & i \\ -i & 2
		      \end{pmatrix};\quad A^\dagger = A;\quad \lambda = 1, 3;\quad\vb u_1 = \frac{1}{\sqrt{2}} \begin{pmatrix}
			      1 \\i
		      \end{pmatrix};\quad\vb u_2 = \frac{1}{\sqrt{2}} \begin{pmatrix}
			      1 \\-i
		      \end{pmatrix}
	      \]
	      We have chosen coefficients for the vectors \(\vb u_1\) and \(\vb u_2\) such that they are unit vectors.
	      As shown above, they are then orthonormal.
	      We know that having distinct eigenvalues means that a matrix is diagonalisable.
	      So let us set
	      \[
		      P =  \frac{1}{\sqrt{2}} \begin{pmatrix}
			      1 & 1 \\ i & -i
		      \end{pmatrix} \implies P^{-1}AP = D = \begin{pmatrix}
			      1 & 0 \\ 0 & 3
		      \end{pmatrix}
	      \]
	      Since the eigenvectors are orthonormal, so are the columns of \(P\), so \(P^{-1} = P^\dagger\) (i.e.\ \(P\) is unitary).
	\item Let
	      \[
		      A = \begin{pmatrix}
			      0 & 1 & 1 \\ 1 & 0 & 1 \\ 1 & 1 & 0
		      \end{pmatrix}
	      \]
	      \(A\) is real and symmetric, with eigenvalues \(\lambda = -1, 2\) with \(M_{-1} = 2\), \(M_2 = 1\).
	      Further,
	      \[
		      E_{-1} = \vecspan \{ \vb w_1, \vb w_2 \};\quad \vb w_1 = \begin{pmatrix}
			      1 \\ -1 \\ 0
		      \end{pmatrix};\quad \vb w_2 = \begin{pmatrix}
			      1 \\ 0 \\ -1
		      \end{pmatrix}
	      \]
	      So \(m_{-1} = 2\), and the matrix is diagonalisable.
	      Let us choose an orthonormal basis for \(E_{-1}\) by taking
	      \[
		      \vb u_1 = \frac{1}{\abs{\vb w_1}}\vb w_1 = \frac{1}{\sqrt 2}\begin{pmatrix}
			      1 \\ -1 \\ 0
		      \end{pmatrix}
	      \]
	      and we can consider
	      \[
		      \vb w_2' = \vb w_2 - (\vb u_1 \cdot \vb w_2)\vb u_1 = \begin{pmatrix}
			      1/2 \\ 1/2 \\ -1
		      \end{pmatrix}
	      \]
	      so that \(\vb w_2'\) is orthogonal to \(\vb u_1\) by construction.
	      We can then normalise this vector to get
	      \[
		      \vb u_2 = \frac{1}{\abs{\vb w_2'}}\vb w_2' = \frac{1}{\sqrt 6} \begin{pmatrix}
			      1 \\ 1 \\ -2
		      \end{pmatrix}
	      \]
	      and therefore
	      \[
		      \mathcal B_{-1} = \{ \vb u_1, \vb u_2 \}
	      \]
	      is an orthonormal basis.
	      For \(E_2\), let us choose \(\mathcal B_2 = \{ \vb u_3 \}\) where
	      \[
		      \vb u_3 = \frac{1}{\sqrt 3}\begin{pmatrix}
			      1 \\ 1 \\ 1
		      \end{pmatrix}
	      \]
	      Together,
	      \[
		      \mathcal B = \left\{ \frac{1}{\sqrt 2}\begin{pmatrix}
			      1 \\ -1 \\ 0
		      \end{pmatrix}, \frac{1}{\sqrt 6} \begin{pmatrix}
			      1 \\ 1 \\ -2
		      \end{pmatrix}, \frac{1}{\sqrt 3}\begin{pmatrix}
			      1 \\ 1 \\ 1
		      \end{pmatrix} \right\}
	      \]
	      is an orthonormal basis for \(\mathbb R^3\).
	      Let \(P\) be the matrix with columns \(\vb u_1, \vb u_2, \vb u_3\), then \(P^{-1}AP = D\) as required.
	      Since we have chosen an orthonormal basis, \(P\) is orthogonal, so \(P^\transpose AP = D\).
\end{enumerate}

\subsection{Unitary and Orthogonal Diagonalisation}
\begin{theorem}
	Any \(n\times n\) hermitian matrix \(A\) is diagonalisable.
	\begin{enumerate}[(i)]
		\item There exists a basis of eigenvectors \(\vb u_1, \dots, \vb u_n \in \mathbb C^n\) with \(A\vb u_i = \lambda \vb u_i\); equivalently
		\item There exists an \(n \times n\) invertible matrix \(P\) with \(P^{-1}AP = D\) where \(D\) is the matrix with eigenvalues on the diagonal, where the columns of \(P\) are the eigenvectors \(\vb u_i\).
	\end{enumerate}
	In addition, the eigenvectors \(\vb u_i\) can be chosen to be orthonormal, so
	\[
		\vb u^\dagger_i \vb u_j = \delta_{ij}
	\]
	or equivalently, the matrix \(P\) can be chosen to be unitary,
	\[
		P^\dagger = P^{-1} \implies P^\dagger AP = D
	\]
	In the special case that the matrix \(A\) is real, the eigenvectors can be chosen to be real, and so
	\[
		\vb u^\transpose \vb u_j = \vb u_i \cdot \vb u_j = \delta_{ij}
	\]
	so \(P\) is orthogonal, so
	\[
		P^\transpose = P^{-1} \implies P^\transpose AP = D
	\]
\end{theorem}

\section{Properties of the Riemann Integral}
\subsection{Linear maps}
Let \( m, n \in \mathbb N \).
Recall that \( L(\mathbb R^m, \mathbb R^n) \) is the vector space of linear maps from \( \mathbb R^m \) to \( \mathbb R^n \).
This is isomorphic to \( M_{n,m} \), the space of \( n \times m \) real matrices.
There is also an isomorphism to \( \mathbb R^{mn} \).
Let \( e_1, \dots, e_m \) be the standard basis of \( \mathbb R^m \), and similarly let \( e_1', \dots, e_n' \) be the standard basis of \( \mathbb R^n \).
Then \( T \in L(\mathbb R^m, \mathbb R^n) \) is identified with the \( n \times m \) matrix \( (T_{ji}) \) where \( 1 \leq j \leq n \) and \( 1 \leq i \leq m \), such that \( T_{ji} = \inner{T e_i, e_j'} \).
We can therefore view \( L(\mathbb R^m, \mathbb R^n) \) as the \( mn \)-dimensional vector space \( \mathbb R^{mn} \) with the Euclidean norm.
So the norm of a linear map \( T \) is given by
\[
	\norm{T} = \sqrt{\sum_{i=1}^m \sum_{j=1}^n T_{ji}^2} = \sqrt{\sum_{i=1}^m \norm{Te_i}^2}
\]
where \( T e_i \) is the \( i \)th column of \( T \).
Thus, \( L(\mathbb R^m, \mathbb R^n) \) becomes a metric space together with the Euclidean distance \( d(S,T) = \norm{S-T} \).
\begin{lemma}
	For \( T \in L(\mathbb R^m, \mathbb R^n) \) and \( x \in \mathbb R^m \),
	\[
		\norm{Tx} \leq \norm{T} \cdot \norm{x}
	\]
	So \( T \) is a Lipschitz map and hence continuous.
	Further, if \( S \in L(\mathbb R^n, \mathbb R^p) \) then
	\[
		\norm{ST} \leq \norm{S} \cdot \norm{T}
	\]
\end{lemma}
\begin{proof}
	We can write
	\[
		x = \sum_{i=1}^m x_i e_i
	\]
	Hence,
	\[
		Tx = \sum_{i=1}^m x_i T e_i
	\]
	Thus,
	\[
		\norm{Tx} \leq \sum_{i=1}^m \abs{x_i} \norm{T e_i} \leq \qty(\sum_{i=1}^m x_i^2)^{1/2} \cdot \qty(\sum_{i=1}^m \norm{Te_i}^2)^{1/2} = \norm{T} \cdot \norm{x}
	\]
	Further, for \( x,y \in \mathbb R^m \) we have
	\[
		d(Tx, Ty) = \norm{Tx - Ty} = \norm{T(x-y)} \leq \norm{T} \cdot \norm{x-y} = \norm{T} d(x,y)
	\]
	So \( T \) is Lipschitz, and any Lipschitz function is continuous.
	Now,
	\[
		\norm{ST} = \qty(\sum_{i=1}^m \norm{STe_i}^2)^{1/2} \leq \qty(\sum_{i=1}^m \norm{S} \norm{Te_i}^2)^{1/2} = \norm{S} \qty(\sum_{i=1}^m \norm{Te_i}^2)^{1/2} = \norm{S} \cdot \norm{T}
	\]
\end{proof}

\subsection{Differentiation}
Recall from IA Analysis that a function \( f \colon \mathbb R \to \mathbb R \) is \textit{differentiable} at a point \( a \in \mathbb R \) if
\[
	\lim_{h \to 0} \frac{f(a+h) - f(a)}{h}
\]
exists.
The value of this limit is called the \textit{derivative} of \( f \) at \( a \), and denoted \( f'(a) \).
Note that \( f \) is differentiable at \( a \) if and only if there exists \( \lambda \in \mathbb R \) and \( \varepsilon \colon \mathbb R \to \mathbb R \) such that \( \varepsilon(0) = 0 \) and \( \varepsilon \) is continuous at \( 0 \), and
\[
	f(a+h) = f(a) + \lambda h + h \varepsilon(h)
\]
This is because we can define
\[
	\varepsilon(h) = \begin{cases}
		0                                 & h = 0    \\
		\frac{f(a+h) - f(a)}{h} - \lambda & h \neq 0
	\end{cases}
\]
Informally, this \( \varepsilon \) definition states that \( f \) is approximated very well (the error \( h\varepsilon(h) \) shrinks rapidly since \( \varepsilon \to 0 \)) by a linear function in a small neighbourhood of \( a \).
Recall that if \( f \) is \( n \) times differentiable at \( a \), then
\[
	f(a+h) = f(a) + \sum_{k=1}^n \frac{f^{(k)}(a)}{k!}h^k + o(h^n)
\]
\begin{definition}
	Let \( m, n \in \mathbb N \).
	Then \( f \colon \mathbb R^m \to \mathbb R^n \) and \( a \in \mathbb R^m \).
	We say that \( f \) is \textit{differentiable} at \( a \) if there exists a linear map \( T \in L(\mathbb R^m, \mathbb R^n) \) and a function \( \varepsilon \colon \mathbb R^m \to \mathbb R^n \) such that \( \varepsilon(0) = 0 \) and \( \varepsilon \) is continuous at \( 0 \), and
	\[
		f(a+h) = f(a) + T(h) + \norm{h} \varepsilon(h)
	\]
	Note that
	\[
		\varepsilon(h) = \begin{cases}
			0                                     & h = 0    \\
			\frac{f(a+h) - f(a) - T(h)}{\norm{h}} & h \neq 0
		\end{cases}
	\]
	So \( f \) is differentiable at \( a \) if and only if there exists \( T \in L(\mathbb R^m, \mathbb R^n) \) such that
	\[
		\frac{f(a+h) - f(a) - T(h)}{\norm{h}} \to 0
	\]
	as \( h \to 0 \).
	Such a \( T \) is unique.
	Indeed, suppose \( S, T \) satisfy the above limit.
	Then, by subtracting,
	\[
		\frac{S(h) - T(h)}{\norm{h}} \to 0
	\]
	For a fixed \( x \in \mathbb R^m \), \( x \neq 0 \), we have \( \frac{x}{k} \to 0 \) as \( k \to \infty \) so
	\[
		\frac{S\qty(\frac{x}{k}) - T\qty(\frac{x}{k})}{\norm{\frac{x}{k}}} \to 0 \implies \frac{S(x) - T(x)}{\norm{x}} = 0
	\]
	So \( Sx = Tx \).
	It follows that \( S = T \).
	We say that if a function \( f \) is differentiable at a point \( a \), \( T \) is the unique \textit{derivative} of \( f \) at \( a \).
	This is denoted \( f'(a) = Df(a) = \eval{Df}_a \).
	If \( f \colon \mathbb R^m \to \mathbb R^n \) is differentiable at \( a \in \mathbb R^m \) for every \( a \), we say that \( f \) is \textit{differentiable on} \( \mathbb R^m \).
	The function \( f' = D \colon \mathbb R^m \to L(\mathbb R^m, \mathbb R^n) \) mapping \( a \mapsto f'(a) \) is the derivative of \( f \).
\end{definition}
\begin{example}
	Constant functions are differentiable.
	Let \( f \colon \mathbb R^m \to \mathbb R^n \) such that \( f(x) = b \) for \( b \in \mathbb R^n \).
	Then for all \( a \in \mathbb R^m \), we have
	\[
		f(a+h) = f(a) + 0h + 0
	\]
	so \( f \) is differentiable at \( a \) and the derivative is zero.
\end{example}
\begin{example}
	Linear maps are differentiable.
	Let \( f \colon \mathbb R^m \to \mathbb R^n \) be defined by \( f(x) = Tx \) for a linear map \( T \in L(\mathbb R^m, \mathbb R^n) \).
	Then
	\[
		f(a+h) = f(a) + f(h) + 0
	\]
	so \( f \) is differentiable at \( a \) with derivative \( T = f \).
	So \( f' \) is a constant function.
\end{example}
\begin{example}
	Consider
	\[
		f(x) = \norm{x}^2
	\]
	For \( a \in \mathbb R^m \), we can find
	\[
		f(a+h) = \norm{a+h}^2 = \norm{a}^2 + 2\inner{a,h} + \norm{h}^2 = f(a) + 2\inner{a,h} + \norm{h} \varepsilon(h)
	\]
	Hence, \( f \) is differentiable with derivative
	\[
		f'(a)(h) = 2\inner{a,h}
	\]
	Note that \( f' \colon \mathbb R^m \to L(\mathbb R^m \to \mathbb R) \) is linear.
\end{example}
\begin{example}
	Note \( M_n \simeq \mathbb R^{n^2} \).
	The function \( f \colon M_n \to M_n \) given by \( f(A) = A^2 \).
	For a fixed \( A \in M_n \),
	\[
		f(A+H) = (A+H)^2 = A^2 + AH + HA + H^2
	\]
	It suffices to show \( H^2 \) is \( o(\norm{H}) \).
	We have \( \norm{H^2} \leq \norm{H}^2 \), hence
	\[
		\frac{\norm{H^2}}{\norm{H}} \leq \norm{H} \to 0
	\]
	So \( f \) is differentiable at \( A \) and the derivative is given by
	\[
		f'(A)(H) = AH + HA
	\]
\end{example}
\begin{example}
	Suppose \( f \colon \mathbb R^m \times \mathbb R^n \to \mathbb R^p \) is bilinear.
	Let \( (a, b) \in \mathbb R^m \times \mathbb R^n \).
	Then,
	\[
		f((a,b) + (h,k)) = f((a+h, b+k)) = f(a,b) + f(a,k) + f(h,b) + f(h,k)
	\]
	The map \( \mathbb R^m \times \mathbb R^n \to \mathbb R^p \) given by \( (h,k) \mapsto f(a,k) + f(h,b) \) is linear as the sum of two linear maps.
	So it suffices to show \( f(h,k) \) is \( o(\norm{(h,k)}) \).
	\[
		h = \sum_{i=1}^m h_i e_i;\quad k = \sum_{j=1}^n k_j e_j'
	\]
	Hence,
	\[
		f(h,k) = \sum_{i=1}^m \sum_{j=1}^n h_i k_j f(e_i, e_j') \implies \norm{f(h,k)} \leq \sum_{i=1}^m \sum_{j=1}^n \abs{h_i} \cdot \abs{k_j} \cdot \norm{f(e_i, e_j')} \leq C \norm{(h,k)}^2
	\]
	for some constant \( C \), since \( \abs{h_i} \leq \norm{(h,k)}^2 \) and similarly for \( \abs{k_j} \).
	So
	\[
		\frac{\norm{f(h,k)}}{\norm{(h,k)}} \leq C \norm{(h,k)} \to 0
	\]
	Hence \( f \) is differentiable with
	\[
		f'(a,b)(h,k) = f(a,k) + f(h,b)
	\]
\end{example}

\section{Fundamental Theorem of Calculus}
\subsection{Conjugation Action of \(GL_n(\mathbb F)\)}
Recall from Vectors and Matrices: if \(\alpha\colon \mathbb F^n \to \mathbb F^n\) is a linear map, we can represent \(\alpha\) as a matrix \(A\) with respect to a basis \(\{ \vb e_1, \dots, \vb e_n \}\).
If we choose a different basis \(\{ \vb f_1, \dots, \vb f_n \}\) then \(\alpha\) can also be written as a matrix with respect to this new basis, by the matrix \(P^{-1}AP\) where \(P\) is the change of basis matrix, defined by
\[
	\vb f_j = P_{ij}\vb e_i
\]
This is an example of conjugation.
\begin{proposition}
	\(GL_n(\mathbb F)\) acts on \(M_{n \times n}(\mathbb F)\) by conjugation.
	The orbit of a matrix \(A \in M_{n \times n}(\mathbb F)\) is the set of matrices representing the same linear map as \(A\) with respect to different bases.
\end{proposition}
\begin{proof}
	This is an action:
	\begin{itemize}
		\item \(P(A) = PAP^{-1} \in M_{n \times n}(\mathbb F)\) for any chosen matrix \(A \in M_{n \times n}(\mathbb F)\), \(P \in GL_n(\mathbb F)\)
		\item \(I(A) = IAI^{-1} = A\)
		\item \(Q(P(A)) = QPAP^{-1}Q^{-1} = (QP)A(QP)^{-1} = (QP)(A)\)
	\end{itemize}
	As shown in the discussion above, \(A\) and \(B\) are in the same orbit if and only if \(A = PBP^{-1} \iff B = P^{-1}AP\), which is equivalent to this conjugation action.
\end{proof}

\subsection{Orbits of Conjugation Action: Jordan Normal Form}
Recall from Vectors and Matrices that any matrix in \(M_{2 \times 2}(\mathbb C)\) is conjugate to a matrix in Jordan Normal Form, i.e.\ to one of the following types of matrix:
\[
	\begin{pmatrix}
		\lambda_1 & 0 \\ 0 & \lambda_2
	\end{pmatrix};\quad \begin{pmatrix}
		\lambda & 0 \\ 0 & \lambda
	\end{pmatrix};\quad \begin{pmatrix}
		\lambda & 1 \\ 0 & \lambda
	\end{pmatrix}
\]
In the first case, the values \(\lambda_1, \lambda_2\) are uniquely determined by the matrix we are trying to conjugate (specifically its eigenvalues).
But of course, the order of the eigenvalues is not determined uniquely.
Other than this, no two matrices on this list of possible Jordan Normal Forms are conjugate.
\begin{itemize}
	\item \(\begin{pmatrix}
		      \lambda_1 & 0 \\ 0 & \lambda_2
	      \end{pmatrix}\) is characterised by having two distinct eigenvalues, a property independent of the chosen basis, so it cannot be conjugate to the others.
	\item \(\begin{pmatrix}
		      \lambda & 0 \\ 0 & \lambda
	      \end{pmatrix}\) is only conjugate to itself since it is \(\lambda I\).
	\item \(\begin{pmatrix}
		      \lambda & 1 \\ 0 & \lambda
	      \end{pmatrix}\) is characterised by having a repeated eigenvalue \(\lambda\), but only a one dimensional eigenspace (independent of the basis we choose).
\end{itemize}
This gives a complete description of the orbits of \(GL_n(\mathbb C) \acts M_{n \times n}(\mathbb C)\).

\subsection{Stabilisers of Conjugation Action}
Clearly we have
\[
	P \in \Stab(A) \iff PAP^{-1} = A \iff PA = AP
\]
So if two matrices commute, they stabilise each other.
Let us consider the three cases as above.
\begin{itemize}
	\item For \(A = \begin{pmatrix}
		      \lambda_1 & 0 \\ 0 & \lambda_2
	      \end{pmatrix}\):
	      \begin{align*}
		      \begin{pmatrix}
			      a & b \\ c & d
		      \end{pmatrix}\begin{pmatrix}
			      \lambda_1 & 0 \\ 0 & \lambda_2
		      \end{pmatrix} & = \begin{pmatrix}
			      \lambda_1 a & \lambda_2 b \\
			      \lambda_1 c & \lambda_2 d
		      \end{pmatrix} \\
		      \begin{pmatrix}
			      \lambda_1 & 0 \\ 0 & \lambda_2
		      \end{pmatrix}\begin{pmatrix}
			      a & b \\ c & d
		      \end{pmatrix} & = \begin{pmatrix}
			      \lambda_1 a & \lambda_1 b \\
			      \lambda_2 c & \lambda_2 d
		      \end{pmatrix}
	      \end{align*}
	      So this matrix is in the stabiliser if and only if \(b = c = 0\).
	      \[
		      \Stab\begin{pmatrix}
			      \lambda_1 & 0 \\ 0 & \lambda_2
		      \end{pmatrix} = \left\{ \begin{pmatrix}
			      a & 0 \\ 0 & d
		      \end{pmatrix} \in GL_2(\mathbb C) \right\}
	      \]
	\item For \(A = \begin{pmatrix}
		      \lambda & 0 \\ 0 & \lambda
	      \end{pmatrix}\), clearly its stabiliser is \(GL_2(\mathbb C)\) since \(A = \lambda I\), and so it commutes with any matrix.
	\item For \(A = \begin{pmatrix}
		      \lambda & 1 \\ 0 & \lambda
	      \end{pmatrix}\), the stabiliser is
	      \[
		      \Stab \begin{pmatrix}
			      \lambda & 1 \\ 0 & \lambda
		      \end{pmatrix} = \left\{ \begin{pmatrix}
			      a & b \\ 0 & a
		      \end{pmatrix} \in GL_2(\mathbb C) \right\}
	      \]
	      (Proof as exercise)
\end{itemize}

\subsection{Geometry of Orthogonal Groups}
We will look more closely at the orthogonal group and special orthogonal group, and then focus on symmetries of \(\mathbb R^2\) and \(\mathbb R^3\).
Let us consider the standard inner product in \(\mathbb R^n\):
\[
	\vb x \cdot \vb y = x_i y_i = \vb x^\transpose \vb y
\]
If we consider the columns \(\vb p_1, \dots, \vb p_n\) of an orthogonal matrix \(P \in O_n\), we have
\[
	(P^\transpose P)_{ij} = \vb p_i^\transpose \vb p_j = \vb p_i \cdot \vb p_j
\]
So since \(P \in O_n \iff P^\transpose P = I\), we have
\[
	\vb p_i \cdot \vb p_j = \delta_{ij}
\]
\begin{proposition}
	\(P \in O_n\) if and only if the columns of \(P\) form an orthonormal basis.
\end{proposition}
This has been proven by the above discussion.
Thinking of \(P \in O_n\) as a change of basis matrix, we get the following result.
\begin{proposition}
	Consider \(O_n \acts M_{n \times n}(\mathbb R)\) by conjugation.
	Two matrices are in the same orbit if and only if they represent the same linear map with respect to two orthonormal bases.
\end{proposition}
\begin{proposition}
	\(P \in O_n\) if and only if \(P \vb x \cdot P \vb y = \vb x \cdot \vb y\), i.e.\ the matrix preserves the inner product.
\end{proposition}
\begin{proof}
	In the forward direction:
	\[
		(P\vb x) \cdot (P \vb y) = (P \vb x)^\transpose (P \vb y) = \vb x^\transpose P^\transpose P \vb y = \vb x^\transpose \vb y = \vb x \cdot \vb y
	\]
	In the backward direction: if \(P\vb x \cdot P\vb y = \vb x \cdot \vb y\) for all \(\vb x, \vb y \in \mathbb R^n\), then taking the standard basis vectors \(\vb e_i, \vb e_j\) we have
	\[
		P\vb e_i \cdot P\vb e_j = \vb e_i \cdot \vb e_j = \delta_{ij}
	\]
	So the vectors \(P\vb e_1, \dots, P\vb e_n\) are orthonormal.
	These are the columns of \(P\), so \(P \in O_n\).
\end{proof}
\begin{corollary}
	For \(P \in O_n\), \(\vb x, \vb y \in \mathbb R^n\), we have
	\begin{enumerate}[(i)]
		\item \(\abs{P\vb x} = \abs{\vb x}\) (\(P\) preserves length)
		\item \(P\vb x \angle P\vb y = \vb x \angle \vb y\) (\(P\) preserves angles between vectors)
	\end{enumerate}
	\begin{proof}
		\begin{enumerate}[(i)]
			\item Follows from the fact that the inner product is preserved, by taking the inner product of a vector with itself under the transformation.
			\item Angles are also defined using the inner product,
			      \[
				      \cos (\vb x \angle \vb y) = \frac{\vb x \cdot \vb y}{\abs{\vb x}\abs{\vb y}}
			      \]
			      Since the inner product and the lengths are preserved, the cosine of the angle is therefore preserved.
			      Since \(\cos\colon [0, \pi] \to [-1, 1]\) is injective, \(\vb x \angle \vb y = P\vb x \angle P\vb y\).
		\end{enumerate}
	\end{proof}
\end{corollary}

\subsection{Reflections in \(O_n\)}
We will consider what the elements of these groups look like when acting upon \(\mathbb R^n\).
\begin{definition}
	If \(\vb a \in \mathbb R^n\) with \(\abs{\vb a} = 1\), then the reflection in the plane normal to \(\vb a\) is the linear map
	\[
		R_{\vb a} \colon \mathbb R^n \to \mathbb R^n;\quad \vb x \mapsto \vb x - 2 (\vb x \cdot \vb a) \vb a
	\]
\end{definition}
\begin{lemma}
	\(R_{\vb a}\) lies in \(O_n\).
\end{lemma}
\begin{proof}
	Let \(\vb x, \vb y \in \mathbb R^n\).
	\begin{align*}
		R_{\vb a}(\vb x) \cdot R_{\vb a}(\vb y) & = (\vb x - 2 (\vb x \cdot \vb a) \vb a) \cdot (\vb y - 2 (\vb y \cdot \vb a) \vb a)                                                                      \\
		                                        & = \vb x \cdot \vb y - 2(\vb x \cdot \vb a)(\vb a \cdot \vb y) - 2(\vb y\cdot \vb a)(\vb x \cdot a) + 4(x\cdot a)(y \cdot a)\underbrace{(a \cdot a)}_{=1} \\
		                                        & = \vb x \cdot \vb y
	\end{align*}
	So it preserves the inner product, so it is an orthogonal matrix.
\end{proof}
As we might expect, conjugates of reflections by orthogonal matrices are also reflections.
\begin{lemma}
	Given \(P \in O_n\), \(PR_{\vb a}P^{-1} = R_{P\vb a}\).
\end{lemma}
\begin{proof}
	We have
	\begin{align*}
		PR_{\vb a}P^{-1}(\vb x) & = P(P^{-1}(\vb x) - 2 (P^{-1}(\vb x) \cdot \vb a) \vb a) \\
		                        & = \vb x - 2(P^{-1}(\vb x)\cdot\vb a)(P\vb a)             \\
		                        & = \vb x - 2(P^\transpose (\vb x)\cdot\vb a)(P\vb a)      \\
		                        & = \vb x - 2(\vb x^\transpose P \vb a)(P\vb a)            \\
		                        & = \vb x - 2(\vb x \cdot P\vb a)(P\vb a)
	\end{align*}
	which by inspection is the reflection of \(\vb x\) by the plane with normal \(P\vb a\).
\end{proof}
We know that no reflection matrix can be in \(SO_n\), since this requires the determinant to be \(+1\), which is the product of the eigenvalues.
The \(n-1\) eigenvectors with eigenvalue \(+1\) are \(n-1\) linearly independent vectors spanning the plane, and the single eigenvector with eigenvalue \(-1\) is the normal to the plane.
So the determinant is \(-1\).

\section{Integration Techniques and Integrals in Taylor's Theorem}
\subsection{All Transformations in \(O_2\)}
\begin{theorem}
	Every element of \(SO_2\) is of the form
	\[
		\begin{pmatrix}
			\cos\theta & -\sin\theta \\
			\sin\theta & \cos\theta
		\end{pmatrix}
	\]
	for some \(\theta \in [0, 2\pi)\).
	
	This is an anticlockwise rotation of \(\mathbb R^2\) about the origin by angle \(\theta\).
	Conversely, every such element lies in \(SO_2\).
\end{theorem}
\begin{proof}
	Let
	\[
		A = \begin{pmatrix}
			a & b \\ c & d
		\end{pmatrix} \in SO_2
	\]
	We have \(A^\transpose A = I\) and \(\det A = 1\).
	So
	\[
		A^\transpose = A^{-1} \implies \begin{pmatrix}
			a & c \\ b & d
		\end{pmatrix} = \frac{1}{1} \begin{pmatrix}
			d & -b \\ -c & a
		\end{pmatrix}
	\]
	So \(a=d, b=-c\).
	Since \(ad-bc=1\), \(a^2+c^2=1\).
	Then we can write \(a = \cos \theta\) and \(c = \sin \theta\) for a unique \(\theta \in [0, 2\pi)\).
	
	Conversely, the determinant of this matrix is 1, and is in \(O_2\), so this element lies in \(SO_2\).
\end{proof}
\begin{theorem}
	The elements of \(O_2 \setminus SO_2\) are the reflections in lines through the origin.
\end{theorem}
\begin{proof}
	Let
	\[
		A = \begin{pmatrix}
			a & b \\ c & d
		\end{pmatrix} \in O_2 \setminus SO_2
	\]
	So \(A^\transpose A = I\) and \(\det A = -1\).
	\[
		A^\transpose = A^{-1} \implies \begin{pmatrix}
			a & c \\ b & d
		\end{pmatrix} = \frac{1}{-1} \begin{pmatrix}
			d & -b \\ -c & a
		\end{pmatrix}
	\]
	So \(a=-d, b=c\).
	Together with \(ad-bc=-1\), we have \(a^2 + c^2 = 1\).
	So let \(a = \cos \theta\), \(c = \sin \theta\) like before, so
	\[
		A = \begin{pmatrix}
			\cos \theta & \sin \theta  \\
			\sin \theta & -\cos \theta
		\end{pmatrix}
	\]
	which can be shown to be a reflection using double angle formulas such that
	\[
		A \begin{pmatrix}
			\sin \frac{\theta}{2} \\ \cos \frac{\theta}{2}
		\end{pmatrix} = -\begin{pmatrix}
			\sin \frac{\theta}{2} \\ \cos \frac{\theta}{2}
		\end{pmatrix};\quad A\begin{pmatrix}
			\cos \frac{\theta}{2} \\ \sin \frac{\theta}{2}
		\end{pmatrix} = \begin{pmatrix}
			\cos \frac{\theta}{2} \\ \sin \frac{\theta}{2}
		\end{pmatrix}
	\]
	So \(A\) is a reflection in the plane orthogonal to the vector \(\begin{pmatrix}
		\sin \frac{\theta}{2} \\ \cos \frac{\theta}{2}
	\end{pmatrix}\).
	Conversely, any reflection in a line through the origin has this form, so it will be in \(O_2 \setminus SO_2\).
\end{proof}

\begin{corollary}
	Every element of \(O_2\) is the composition of at most two reflections.
\end{corollary}
\begin{proof}
	Every element of \(O_2 \setminus SO_2\) is a reflection, so this is trivial.
	If \(A \in SO_2\), then we can write
	\[
		A = \underbrace{A \begin{pmatrix}
				-1 & 0 \\ 0 & 1
			\end{pmatrix}}_{\det = -1} \underbrace{\begin{pmatrix}
				-1 & 0 \\ 0 & 1
			\end{pmatrix}}_{\det = -1}
	\]
	So we have expressed \(A\) as the product of two reflections.
\end{proof}

\subsection{All Transformations in \(O_3\)}
\begin{theorem}
	If \(A \in SO_3\), then there exists some unit vector \(\vb v \in \mathbb R^3\) with \(A\vb v = \vb v\), i.e.\ there exists an eigenvector with eigenvalue 1.
\end{theorem}
\begin{proof}
	It is sufficient to show that 1 is an eigenvalue of \(A\), since this guarantees that there is some nonzero eigenvector for this eigenvalue which we can then normalise.
	This is equivalent to showing that \(\det (A - I) = 0\).
	\begin{align*}
		\det(A - I) & = \det(A - AA^\transpose)       \\
		            & = \det(A)\det(I - A^\transpose) \\
		            & = \det(I - A^\transpose)        \\
		            & = \det((I - A)^\transpose)      \\
		            & = \det(I - A)                   \\
		            & = (-1)^3\det(A - I)
	\end{align*}
	So \(2\det(A - I) = 0 \implies \det(A - I) = 0\).
\end{proof}
\begin{corollary}
	Every element \(A \in SO_3\) is conjugate (in \(SO_3\)) to a matrix of the form
	\[
		\begin{pmatrix}
			1 & 0           & 0            \\
			0 & \cos \theta & -\sin \theta \\
			0 & \sin \theta & \cos \theta
		\end{pmatrix}
	\]
\end{corollary}
\begin{proof}
	By the above theorem, there exists some unit vector \(\vb v_1\) which is an eigenvector of eigenvalue 1.
	We can extend this vector to an orthonormal basis \(\{ \vb v_1, \vb v_2, \vb v_3 \}\) of \(\mathbb R^3\).
	Then, for \(i=2,3\), we have
	\[
		A\vb v_i \cdot \vb v_1 = A\vb v_i \cdot A\vb v_1 = \vb v_i \cdot \vb v_1 = 0
	\]
	So \(A\vb v_2, A\vb v_3\) lie in the subspace generated by \(\vb v_2, \vb v_3\), i.e.\ \(\vecspan \{ \vb v_2, \vb v_3 \} = \genset{\vb v_2, \vb v_3}\).
	So \(A\) maps this subspace to itself, and we can thus consider the restriction of \(A\) to this subspace.
	The matrix in this new basis will have form
	\[
		\begin{pmatrix}
			1 & 0 & 0 \\
			0 & a & b \\
			0 & c & d
		\end{pmatrix}
	\]
	The smaller matrix in the bottom right will still have determinant 1, since we can expand the determinant here by the first row.
	So \(A\) restricted to this subspace is an element of \(SO_2\), so its matrix must be of the form
	\[
		\begin{pmatrix}
			a & b \\ c & d
		\end{pmatrix} = \begin{pmatrix}
			\cos \theta & -\sin \theta \\
			\sin \theta & \cos \theta
		\end{pmatrix}
	\]
	So \(A\) has the required form with respect to this new basis \(\{ \vb v_1, \vb v_2, \vb v_3 \}\).
	The change of basis matrix \(P\) lies in \(O_3\) since \(\{ bm v_1, \vb v_2, \vb v_3 \}\) is an orthonormal basis.
	If \(P \notin SO_3\), then we can use the basis \(\{ -\vb v_1, \vb v_2, \vb v_3 \}\) instead, which will invert the determinant of \(P\).
	So in either case \(P \in SO_3\).
\end{proof}
This tells us in particular that every element in \(SO_3\) is a rotation about some axis, here \(\vb v_1\).

\begin{corollary}
	Every element of \(O_3\) is the composition of at most three reflections.
\end{corollary}
\begin{proof}
	\begin{itemize}
		\item If \(A \in SO_3\), then \(\exists P \in SO_3\) such that \(PAP^{-1} = B\), where \(B\) is of the form
		      \[
			      B = \begin{pmatrix}
				      1 & 0           & 0            \\
				      0 & \cos \theta & -\sin \theta \\
				      0 & \sin \theta & \cos \theta
			      \end{pmatrix}
		      \]
		      Since this smaller matrix
		      \[
			      \begin{pmatrix}
				      \cos \theta & -\sin \theta \\
				      \sin \theta & \cos \theta
			      \end{pmatrix}
		      \]
		      is a composition of at most two reflections, then \(B\) is also a composition of at most two reflections, i.e.\ \(B = B_1 B_2\).
		      Since \(A\) is a conjugate of \(B\), it is also a composition of at most two reflections, as the conjugate of a reflection is a reflection, and \(A = P^{-1}BP = (P^{-1}B_1P)(P^{-1}B_2P)\).
		\item If \(A \in O_3 \setminus SO_3\), then \(\det A = -1\) and we can construct
		      \[
			      A = \underbrace{A\begin{pmatrix}
					      -1 & 0 & 0 \\
					      0  & 1 & 0 \\
					      0  & 0 & 1
				      \end{pmatrix}}_{\det = 1}\underbrace{\begin{pmatrix}
					      -1 & 0 & 0 \\
					      0  & 1 & 0 \\
					      0  & 0 & 1
				      \end{pmatrix}}_{\det = -1}
		      \]
		      So the left-hand product lies in \(SO_3\), so it is a composition of at most two reflections.
		      The final element is a reflection in the \(y\)--\(z\) plane, so the combined product is a composition of at most three reflections.
	\end{itemize}
\end{proof}

\subsection{Symmetries of the Cube (revisited)}
We can think of symmetry groups of the Platonic solids as subgroups of \(O_3\) by placing the solid at the origin.
By question 11 on example sheet 4, we have that \(O_3 \cong SO_3 \times C_2\), where \(C_2\) is generated by the map \(\vb v \mapsto -\vb v\).
So if \(\vb v\mapsto -\vb v\) is a symmetry of our platonic solid, then this group of symmetries will also split as the direct product of \(G^+ \times C_2\) where \(G^+\) is the group of rotations (proof as exercise).

So we have that the group of symmetries of the cube is \(G^+ \times C_2 \cong S_4 \times C_2\) by the results from earlier.



\section{Improper Integration}
\subsection{Quadrics in General}
A quadric in \(\mathbb R^n\) is a hypersurface defined by an equation of the form
\[ Q(\vb x) = \vb x^\transpose A \vb x + \vb b^\transpose \vb x + c = 0 \]
for some nonzero, symmetric, real \(n \times n\) matrix \(A\), \(b \in \mathbb R^n\), \(c \in \mathbb R\). In components,
\[ Q(\vb x) = A_{ij}x_ix_j + b_ix_i + c = 0 \]
We will clasify solutions for \(\vb x\) up to geometrical equivalence, so we will not distinguish between solutions here which are related by isometries in \(\mathbb R^n\) (distance-preserving maps, i.e. translations and orthogonal transformations about the origin).

Note that \(A\) is invertible if and only if it has no zero eigenvalues. In this case, we can complete the sequare in the equation \(Q(\vb x) = 0\) by setting \(\vb y = \vb x + \frac{1}{2}A^{-1} \vb b\). This is essentially a translation isometry, moving the origin to \(\frac{1}{2}A^{-1} \vb b\).
\begin{align*}
	\vb y^\transpose A \vb y & = (\vb x + \frac{1}{2}A^{-1}\vb b)^\transpose A (\vb x + \frac{1}{2}A^{-1}\vb b)                         \\
	                         & = (\vb x^\transpose + \frac{1}{2}\vb b^\transpose(A^{-1})^\transpose) A (\vb x + \frac{1}{2}A^{-1}\vb b) \\
	                         & = \vb x^\transpose A \vb x + \vb b^\transpose \vb x + \frac{1}{4}\vb b^\transpose A^{-1}\vb b
\end{align*}
since \((A^\transpose)^{-1} = (A^{-1})^\transpose\). Then,
\[ Q(\vb x) = 0 \iff \mathcal F(\vb y) = k \]
with
\[ \mathcal F(\vb y) = \vb y^\transpose A \vb y \]
which is a quadratic form with respect to a new origin \(\vb y = \vb 0\), and where \(k = \frac{1}{4}\vb b^\transpose A^{-1}\vb b - c\). Now we can diagonalise \(\mathcal F\) as in the above section, in particular, orthonormal eigenvectors give the principal axes, and the eigenvalues of \(A\) and the value of \(k\) determine the geometrical nature of the solution of the quadric. In \(\mathbb R^3\), the geometrical possibilities are (as we saw before):
\begin{enumerate}[(i)]
	\item eigenvalues positive, \(k\) positive gives an ellipsoid;
	\item eigenvalues different signs, \(k\) nonzero gives a hyperboloid
\end{enumerate}
If \(A\) has one or more zero eigenvalues, then the analysis we have just provided changes, since we can no longer construct such a \(\vb y\) vector, since \(A^{-1}\) does not exist. The simplest standard form of \(Q\) may have both linear and quadratic terms.

\subsection{Conics as Quadrics}
Quadrics in \(\mathbb R^2\) are curves called conics. Let us first consider the case where \(\det A \neq 0\). By completing the square and diagonalising \(A\), we get a standard form
\[ \lambda_1 {x'_1}^2 + \lambda_2 {x'_2}^2 = k \]
The variables \(x'_i\) correspond to the principal axes and the new origin. We have the following cases.
\begin{itemize}
	\item (\(\lambda_1, \lambda_2 > 0\)) This is an ellipse for \(k>0\), and a point for \(k=0\). There are no solutions for \(k<0\).
	\item (\(\lambda_1 > 0, \lambda_2 < 0\)) This gives a hyperbola for \(k>0\), and a hyperbola in the other axis if \(k<0\). If \(k=0\), this is a pair of lines. For instance, \({x'_1}^2 - {x'_2}^2 = 0 \implies (x'_1 - x'_2)(x'_1 + x'_2) = 0\).
\end{itemize}
If \(\det A = 0\), then there is exactly one zero eigenvalue since \(A \neq 0\). Then:
\begin{itemize}
	\item (\(\lambda_1 > 0, \lambda_2 = 0\)) We will diagonalise \(A\) in the original expression for the quadric. This gives
	      \[ \lambda_1 {x'_1}^2 + b'_1 x'_1 + b'_2 x'_2 + c = 0 \]
	      This is a new equation in the coordinate system defined by \(A\)'s principal axes. Completing the square here in the \(x'_1\) term, we have
	      \[ \lambda_1 {x''_1}^2 + b'_2x'_2 + c' = 0 \]
	      where \(x''_1 = x'_1 + \frac{1}{2\lambda_1}b'_1\), and \(c' = c - \frac{{b'_1}^2}{4\lambda_1^2}\). If \(b'_2 = 0\), then \(x_2\) can take any value; and we get a pair of lines if \(c'<0\), a single line if \(c'=0\), and no solutions if \(c'>0\). Otherwise, \(b'_2 \neq 0\), and the equation becomes
	      \[ \lambda_1 {x''_1}^2 + b'_2x''_2 = 0 \]
	      where \(x_2'' = x'_2 + \frac{1}{b_2'}c'\), and clearly this equation is a parabola.
\end{itemize}
All changes of coordinates correspond to translations (shifts of the origin) or orthogonal transformations, both of which preserve distance and angles.

\subsection{Standard Forms for Conics}
The general forms of conics can be written in terms of lengths \(a, b\) (the semi-major and semi-minor axes), or equivalently a length scale \(\ell\) and a dimensionless eccentricity constant \(e\).
\begin{itemize}
	\item First, let us consider Cartesian coordinates. The formulas are:

	      \medskip\noindent\begin{tabular}{c|c|c|c}
		      conic     & formula                                 & eccentricity                       & foci       \\\hline
		      ellipse   & \(\frac{x^2}{a^2} + \frac{y^2}{b^2} = 1\) & \(b^2=a^2(1-e^2)\), and \(e<1\)        & \(x=\pm ae\) \\
		      parabola  & \(y^2 = 4ax\)                             & one quadratic term vanishes, \(e=1\) & \(x = +a\)   \\
		      hyperbola & \(\frac{x^2}{a^2} - \frac{y^2}{b^2} = 1\) & \(b^2=a^2(e^2-1)\), and \(e<1\)        & \(x=\pm ae\)
	      \end{tabular}

	\item Polar coordinates are a convenient alternative to Cartesian coordinates. In this coordinate system, we set the origin to be at a focus. Then, the formulas are
	      \[ r = \frac{\ell}{1 + e\cos \theta} \]
	      \begin{itemize}
		      \item For the ellipse, \(e<1\) and \(\ell = a(1-e^2)\);
		      \item For the parabola, \(e=1\) and \(\ell = 2a\); and
		      \item For the hyperbola, \(e>1\) and \(\ell = a(e^2 - 1)\). There is only one branch for the hyperbola given by this polar form.
	      \end{itemize}
\end{itemize}

%TODO draw graphs for all of these curves in both coordinate systems

\subsection{Conics as Sections of a Cone}
The equation for a cone in \(\mathbb R^3\) given by an apex \(\vb c\), an axis \(\nhat\), and an angle \(\alpha < \frac{\pi}{2}\), is
\[ (\vb x - \vb c)\cdot\nhat = \abs{\vb x - \vb c}\cos \alpha \]
Less formally, the angle of \(\vb x\) away from \(\nhat\) must be \(\alpha\). By squaring this equation, we can essentially define two cones which stretch out infinitely far and meet at the centre point \(\vb c\).
\[ \left( (\vb x - \vb c)\cdot\nhat \right)^2 = \abs{\vb x - \vb c}^2\cos^2 \alpha \]
Let us choose a set of coordinate axes so that our equations end up slightly easier. Let \(\vb c = c\vb e_3, \nhat = \cos\beta \vb e_1 - \sin\beta \vb e_3\). Then essentially the cone starts at \((0, 0, c)\) and points `downwards' in the \(\vb e_1\)--\(\vb e_3\) plane. Then the conic section is the intersection of this cone with the \(\vb e_1\)--\(\vb e_2\) plane, i.e. \(x_3 = 0\).
\[ (x_1\cos\beta - c\sin\beta)^2 = (x_1^2 + x_2^2 + c^2)\cos^2\alpha \]
\[ \iff (\cos^2\alpha - \cos^2\beta)x_1^2 + (\cos^2\alpha)x_2^2 + 2x_1c\sin\beta\cos\beta = \text{const.} \]
Now we can compare the signs of the \(x_1^2\) and \(x_2^2\) terms. Clearly the \(x_2^2\) term is always positive, so we consider the sign of the \(x_1^2\) term.
\begin{itemize}
	\item If \(\cos^2 \alpha > \cos^2\beta\) (i.e. \(\alpha < \beta\)), then we have an ellipse.
	\item If \(\cos^2 \alpha = \cos^2\beta\) (i.e. \(\alpha = \beta\)), then we have a parabola.
	\item If \(\cos^2 \alpha < \cos^2\beta\) (i.e. \(\alpha > \beta\)), then we have a hyperbola.
\end{itemize}

\section{Generalising Riemann Integrability}
\subsection{Special relativity with particle physics}
In Newtonian physics, when two particles collide, we must consider the conservation of 3-momentum.
In special relativity however, we must instead consider the conservation of 4-momentum:
\[
	P = \begin{pmatrix}
		\frac{E}{c} \\ \vb p
	\end{pmatrix}
\]
It is often convenient, when dealing with systems of particles, to let the origin of our frame of reference be the centre of momentum.
This is the frame such that the total 3-momentum of the system is zero.
However, this cannot be done when dealing with massless particles since there does not exist such a rest frame.

\subsection{Particle decay}
Consider a particle of mass \(m_1\) with 3-momentum \(\vb p_1\) which decays into two particles of mass \(m_2\) and \(m_3\) with 3-momenta \(\vb p_2, \vb p_3\).
Since 4-momentum is conserved, we get \(P_1 = P_2 + P_3\).
First, consider the 0 component (the timelike component) of \(P\).
\[
	E_1 = E_2 + E_3
\]
Now, consider the \(1, 2, 3\) components (the spacelike components) of the 4-momentum.
We have
\[
	\vb p_1 = \vb p_2 + \vb p_3
\]
Let us look at this in the centre of momentum frame, so \(\vb p_1 = 0\).
Hence
\[
	\vb p_2 = -\vb p_3
\]
Because we are in the centre of momentum frame, we have \(E_1 = m_1 c^2\) hence
\[
	\frac{E_1}{c} = m_1 c = \frac{E_2}{c} + \frac{E_3}{c}
\]
Further,
\[
	\frac{E_2}{c} = \sqrt{\vb p_2^2 + m_2^2 c^2};\quad \frac{E_3}{c} = \sqrt{\vb p_3^2 + m_3^2 c^2}
\]
Hence,
\[
	m_1 c = \sqrt{\vb p_2^2 + m_2^2 c^2} + \sqrt{\vb p_3^2 + m_3^2 c^2} \geq m_2 c + m_3 c
\]
Hence the rest mass of the initial particle must be \textit{at least} the sum of the rest masses of the particles that result from the decay.

\subsection{Higgs to photon decay}
Consider the decay of the Higgs particle \(h\) into two photons \(\gamma\).
By conservation of 4-momentum,
\[
	P_h = P_{\gamma_1} + P_{\gamma_2}
\]
In the Higgs rest frame,
\[
	P_h = \begin{pmatrix}
		m_h c \\ \vb 0
	\end{pmatrix} =
	\begin{pmatrix}
		\frac{E_{\gamma_1}}{c} \\ \vb p_{\gamma_1}
	\end{pmatrix}
	+
	\begin{pmatrix}
		\frac{E_{\gamma_2}}{c} \\ \vb p_{\gamma_2}
	\end{pmatrix}
\]
Looking at the \(1, 2, 3\) components we find
\[
	\vb p_{\gamma_1} = -\vb p_{\gamma_2}
\]
Looking at the 0 component we find
\[
	m_h c = \frac{E_{\gamma_1}}{c} + \frac{E_{\gamma_2}}{c}
\]
Since \(\frac{E^2}{c^2} = \vb p^2 + m^2c^2\), because the photons have zero rest mass we have
\[
	\frac{E_{\gamma_1}}{c} = \abs{\vb p_{\gamma_1}} = \abs{\vb p_{\gamma_2}} = \frac{E_{\gamma_2}}{c}
\]
Hence,
\[
	E_{\gamma_1} = E_{\gamma_2} = \frac{1}{2}m_h c^2
\]
Note that mass has been lost, but kinetic energy has been gained.

\subsection{Particle scattering}
Consider two identical particles colliding, without decaying into new particles.
In frame \(S\), particle 1 is moving horizontally with 3-velocity \(\vb u\), and particle 2 starts at rest.
After the collision, particle 1 has 3-velocity \(\vb q\) and particle 2 has 3-velocity \(\vb r\), where \(\vb q\) has angle \(\theta\) to the horizontal and \(\vb r\) has angle \(\phi\) to the horizontal.
In the centre of momentum frame \(S'\), particles 1 and 2 move towards each other horizontally with 3-momenta \(\vb p_1\) and \(\vb p_2 = -\vb p_1\).
After the collision, particle 1 moves with 3-momentum \(\vb p_3\) and particle 2 moves with 3-momentum \(\vb p_4 = -\vb p_3\).
The angle of deflection is \(\theta'\).
By conservation of 4-momentum,
\[
	P_1 + P_2 = P_3 + P_4
\]
Since particles 1 and 2 have the same mass, their speeds (in \(S'\)) are equal both before and after the collision.
Let the speed before the collision be \(v\) and the speed after the collision be \(w\).
\[
	P_1' = \begin{pmatrix}
		m\gamma_v c \\
		m\gamma_v v \\
		0           \\
		0
	\end{pmatrix};\quad P_2' = \begin{pmatrix}
		m\gamma_v c  \\
		-m\gamma_v v \\
		0            \\
		0
	\end{pmatrix};\quad P_3' = \begin{pmatrix}
		m\gamma_w c             \\
		m\gamma_w w \cos\theta' \\
		m\gamma_w w \sin\theta' \\
		0
	\end{pmatrix};\quad P_4' = \begin{pmatrix}
		m\gamma_w c              \\
		-m\gamma_w w \cos\theta' \\
		-m\gamma_w w \sin\theta' \\
		0
	\end{pmatrix}
\]
Looking at the 0 component,
\[
	2 m\gamma_v c = 2m\gamma_w c
\]
Since \(m\) is the same on both sides,
\[
	v = w
\]
Now we will apply a Lorentz transformation to return to \(S\).
\[
	\Lambda = \begin{pmatrix}
		\gamma_v             & \gamma_v \frac{v}{c} & 0 & 0 \\
		\gamma_v \frac{v}{c} & \gamma_v             & 0 & 0 \\
		0                    & 0                    & 1 & 0 \\
		0                    & 0                    & 0 & 1
	\end{pmatrix}
\]
Now, since \(u\) is the initial velocity of particle 1 in \(S\),
\[
	P_1 = \Lambda P_1' = \begin{pmatrix}
		m\gamma_v^2 \qty(c + \frac{v^2}{c}) \\
		m\gamma_v^2 (v+v)                   \\
		0                                   \\
		0
	\end{pmatrix} = \begin{pmatrix}
		m\gamma_u c \\
		m\gamma_u u \\
		0           \\
		0
	\end{pmatrix}
\]
After the collision, as seen in \(S\), particle 1's 4-momentum is
\[
	P_3 = \Lambda P_3' = \begin{pmatrix}
		m\gamma_v^2 \qty(c + \frac{v^2}{c}\cos\theta') \\
		m\gamma_v^2 \qty(v + v\cos\theta')             \\
		m\gamma_v v\sin\theta'                         \\
		0
	\end{pmatrix} = \begin{pmatrix}
		m\gamma_q c           \\
		m\gamma_q q\cos\theta \\
		m\gamma_q q\sin\theta \\
		0
	\end{pmatrix}
\]
By dividing the 1 and 2 components on both sides, we deduce
\[
	\tan\theta = \frac{m\gamma_v v\sin\theta'}{m\gamma_v^2 v(1 + \cos\theta')} = \frac{1}{\gamma_v} \tan\frac{1}{2}\theta'
\]
For the second particle, we can do the same calculation to get
\[
	\tan\phi = \frac{m\gamma_v v\sin\theta'}{m\gamma_v^2 v(1 - \cos\theta')} = \frac{1}{\gamma_v} \cot\frac{1}{2}\theta'
\]
So given the knowledge of the exact setup of the particles, we can find the angles between the particles as viewed in a different reference frame.
In particular,
\[
	\tan\theta \cdot \tan\phi = \frac{1}{\gamma_v^2} = \frac{2}{1+\gamma_u} \leq 1
\]
This is a generalisation of the Newtonian result, where \(\gamma_u = 1\) giving
\[
	\tan\theta \cdot \tan\phi = 1
\]
So the angle between the trajectories in the Newtonian case is \(\frac{\pi}{2}\).

\subsection{Particle creation}
Consider equal particles 1 and 2 of mass \(m\) moving towards each other horizontally with speed \(v\) in \(S\), with 4-momenta \(P_1\) and \(P_2\).
After the collision, particles 1 and 2 have 4-momenta \(P_3\) and \(P_4\), and a new particle 3 with 4-momentum \(P_5\) is created with mass \(M\).
Note that \(S\) is the centre of momentum frame.
By conservation of 4-momentum, we have
\[
	P_1 + P_2 = P_3 + P_4 + P_5
\]
We have
\[
	P_2 + P_2 = \begin{pmatrix}
		2m\gamma_v c \\ \vb 0
	\end{pmatrix} = \begin{pmatrix}
		\frac{E_3}{c} + \frac{E_4}{c} + \frac{E_5}{c} \\
		\vb 0
	\end{pmatrix}
\]
Certainly we have
\[
	2m\gamma_v c^2 = E_3 + E_4 + E_5 \geq (m + m + M)c^2 = (2m + M)c^2
\]
Hence, for the particle's creation to be possible, we must have
\[
	\gamma_v \geq 1 + \frac{M}{2m}
\]
So the initial kinetic energy in \(S\) must satisfy
\[
	2m(\gamma_v - 1)c^2 \geq Mc^2
\]
Consider some other reference frame \(S'\) where one particle moves with speed \(u\) and the other is at rest.
Then
\[
	u = \frac{2v}{1 + \frac{v^2}{c^2}}
\]
Hence, by the result above in the particle scattering experiment,
\[
	\gamma_u = 2(\gamma_v^2 - 1) \geq 2\qty(1 + \frac{M}{2m})^2 - 1 = 1 + \frac{2M}{m} + \frac{M^2}{2m^2}
\]
Hence, in this frame, the kinetic energy \(mc^2(\gamma_u - 1)\) must satisfy
\[
	mc^2(\gamma_u - 1) \geq mc^2\qty(\frac{2M}{m} + \frac{M^2}{2m^2}) \geq 2Mc^2 + \frac{M^2c^2}{2m}
\]
This extra \(\frac{M^2c^2}{2m}\) term (compared to the \(Mc^2\) expression in \(S\)) is produced by the transformation between frames.
So in a frame where one particle is at rest, we require significantly more kinetic energy.
So a particle accelerator is most efficiently utilised by accelerating two particles into each other, rather than by accelerating one particle into a fixed target.


\end{document}
