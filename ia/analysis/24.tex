\subsection{Using the Integral Test}
First, consider the sum \(\sum_1^\infty \frac{1}{n^k}\).
By the integral test, this converges if and only if \(k > 1\).
As a more complicated example, consider \(\sum_2^\infty \frac{1}{n\log n}\).
Let \(f(x) = \frac{1}{x\log x}\), and
\[
	\int_2^R \frac{\dd{x}}{x\log x} = \eval{\log(\log x)}_2^R
\]
which diverges, so by the integral test the series diverges.
\begin{corollary}[Euler-Mascheroni Constant]
	As \(n \to \infty\),
	\[
		\sum_1^n \frac{1}{n} - \int_1^n \frac{1}{n} = 1 + \frac{1}{2} + \dots + \frac{1}{n} - \log n \to \gamma
	\]
	where \(\gamma \in [0, 1]\).
	This is known as the Euler-Mascheroni constant.
	It is unknown whether \(\gamma\) is irrational.
\end{corollary}

\subsection{Piecewise Continuous Functions}
\begin{definition}
	A function \(f \colon [a, b] \to \mathbb R\) is piecewise continuous if there is a dissection \(\mathcal D\) such that \(f\) is continuous on all intervals defined by this dissection, and that the one-sided limits
	\[
		\lim_{x \to x_{j-1}^+} f(x);\quad \lim_{x \to x_{j-1}^-} f(x)
	\]
	exist.
\end{definition}
\noindent We can extend the class of Riemann integrable functions to include piecewise continuous functions as well.
This is true since we use this dissection to construct the upper and lower sums.
The one-sided limits are here to ensure that the function is bounded near these discontinuities.
We might now ask how large the discontinuity set is allowed to be in order for \(f\) to still be Riemann integrable.
As we have seen from examples before, it is possible to have a function which has countably many discontinuity points, but is still Riemann integrable.
