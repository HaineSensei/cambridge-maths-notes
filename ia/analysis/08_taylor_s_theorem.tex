\subsection{Lagrange's and Cauchy's remainders}
\begin{theorem}[Taylor's Theorem with Lagrange's Remainder]
	Suppose \(f\) and its derivatives up to order \(n-1\) are continuous in \([a, a+h]\), and \(f^{(n)}\) exists for \(x \in (a, a+h)\).
	Then
	\[
		f(a+h) = f(a) + hf'(a) + \frac{h^2}{2!} f''(a) + \dots + \frac{h^{n-1}}{(n-1)!}f^{(n-1)}(a) + \frac{h^n}{n!}f^{(n)}(a + \theta h)
	\]
	where \(\theta \in (0, 1)\).
\end{theorem}
\noindent Note that for \(n=1\), this is exactly the mean value theorem, so this can be seen as an \(n\)th order extension of the mean value theorem.
We commonly write \(R_n\) for the final error term \(\frac{h^n}{n!}f^{(n)}(a + \theta h)\).
This is known as Lagrange's form of the remainder.
\begin{proof}
	For \(0 \leq t \leq h\), we define
	\[
		\phi(t) = f(a+t) - f(a) - tf'(a) - \dots - \frac{t^{n-1}}{(n-1)!}f^{(n-1)}(a) - \frac{t^n}{n!}B
	\]
	where we choose \(B\) suitably such that \(\phi(h) = 0\).
	(Recall that in the proof of the mean value theorem, we used \(f(x) - kx\) and picked \(k\) suitably such that this allowed the use of Rolle's theorem.
	This is entirely analogous, but generalised to the \(n\)th derivative).
	Note that
	\[
		\phi(0) = \phi'(0) = \dots = \phi^{(n-1)}(0) = 0
	\]
	We can use Rolle's theorem inductively \(n\) times.
	Since \(\phi(0) = \phi(h) = 0\), there is a point \(0 < h_1 < h\) such that \(\phi'(h_1) = 0\).
	Since \(\phi'(0) = \phi'(h_1) = 0\), there is a point \(0 < h_2 < h_1\) such that \(\phi''(h_2) = 0\).
	This continues until we find a point \(0 < h_n < h\) such that \(\phi^{(n)}(h_n) = 0\).
	Hence \(h_n = \theta h\) for some \(0 < \theta < 1\).
	Now, \(\phi^{(n)}(t) = f^{(n)}(a + t) - B\).
	We can see now that \(B = f^{(n)}(a + \theta h)\), which gives the required result.
\end{proof}
\noindent We can prove an alternative version of Taylor's theorem with a different error term.
\begin{theorem}[Taylor's Theorem with Cauchy's Remainder]
	Suppose (equivalently to before) \(f\) and its derivatives up to order \(n-1\) are continuous in \([a, a+h]\), and \(f^{(n)}\) exists for \(x \in (a, a+h)\).
	Then
	\[
		f(a+h) = f(a) + hf'(a) + \frac{h^2}{2!} f''(a) + \dots + \frac{h^{n-1}}{(n-1)!}f^{(n-1)}(a) + R_n
	\]
	where
	\[
		R_n = \frac{(1 - \theta)^{n-1}h^n f^{(n)}(a + \theta h)}{(n-1)!}
	\]
	for \(\theta \in (0, 1)\).
\end{theorem}
\begin{proof}
	For simplicity, in this proof we let \(a = 0\), although the same argument applies when \(a \neq 0\).
	Let us define
	\[
		F(t) = f(h) - f(t) - (h-t)f'(t) - \dots - \frac{(h-t)^{n-1}f^{(n-1)}(t)}{(n-1)!}
	\]
	for \(t \in [0, h]\).
	Then
	\begin{align*}
		F'(t) & = -f'(t) + f'(t) - (h-t)f''(t) + (h-t)f''(t) - \frac{1}{2} (h-t)^2f'''(t) + \frac{1}{2} (h-t)^2f'''(t) \\
		      & - \dots - \frac{(h-t)^{n-1}}{(n-1)!}f^{(n)}(t)                                                         \\
		      & = - \frac{(h-t)^{n-1}}{(n-1)!}f^{(n)}(t)
	\end{align*}
	Let
	\[
		\phi(t) = F(t) - \left[ \frac{h-t}{h} \right]^p F(0)
	\]
	where \(p \in \mathbb N\) and \(1 \leq p \leq n\).
	Then
	\[
		\phi(0) = \phi(h) = 0
	\]
	By Rolle's theorem, there exists \(\theta \in (0, 1)\) such that
	\[
		\phi'(\theta h) = 0
	\]
	We can compute \(\phi'\) to find
	\[
		\phi'(\theta h) = F'(\theta h) + \frac{p(1-\theta)^{p-1}}{h} F(0) = 0
	\]
	Substituting everything back into \(F\) gives
	\[
		0 = \frac{-h^{n-1}(1-\theta)^{n-1}}{(n-1)!}f^{(n)}(\theta h) + \frac{p(1-\theta)^{p-1}}{h}\left[ f(h) - f(0) - h'(0) - \dots - \frac{h^{n-1}}{(n-1)!}f^{(n-1)}(0) \right]
	\]
	Hence
	\[
		f(h) = f(0) + hf'(0) + \frac{h^2}{2!} f''(0) + \dots + \frac{h^{n-1}}{(n-1)!}f^{(n-1)}(0) + \underbrace{\frac{h^n(1 - \theta)^{n-1}f^{(n)}(\theta h)}{(n-1)!\cdot p(1-\theta)^{p-1}}}_{R_n}
	\]
	By letting \(p = n\), we get Lagrange's remainder.
	If \(p=1\), we get Cauchy's remainder.
\end{proof}

\subsection{Bounding error terms}
Recall that Lagrange's remainder is
\[
	R_n = \frac{h^n}{n!}f^{(n)}(a + \theta h)
\]
and Cauchy's remainder is
\[
	R_n = \frac{(1 - \theta)^{n-1}h^n f^{(n)}(a + \theta h)}{(n-1)!}
\]
and that we can write
\[
	f(h) = P_{n-1}(h) + R_n
\]
where \(P_{n-1}\) is the Taylor polynomial to \((n-1)\)th order.
To get a Taylor series for a function \(f\), we need to prove that the \(R_n\) tend to zero as \(n \to \infty\).
In general, this requires estimates for the \(R_n\) and it could take a lot of effort to prove whether this limit is zero or not.
Note also that the theorems deducing the remainder terms work equally well in an interval \([a+h, a]\) where \(h < 0\).

\subsection{Binomial series}
\begin{proposition}
	Let
	\[
		f(x) = (1 + x)^r
	\]
	for some \(r \in \mathbb Q\).
	If \(\abs{x} < 1\), then
	\[
		f(x) = 1 + \binom{r}{1}x + \dots + \binom{r}{n}x^n + \dots
	\]
	where
	\[
		\binom{r}{n} = \frac{r(r-1)\cdots(r-n+1)}{n!}
	\]
\end{proposition}
\begin{proof}
	Clearly,
	\[
		f^{(n)}(x) = r(r-1)\cdots(r-n+1)(1+x)^{r-n}
	\]
	These coefficients correspond exactly with that of the Taylor polynomial.
	If \(r \in \mathbb N\), then \(f^{(r+1)}(x) \equiv 0\), so clearly the \(R_n\) are zero as \(n \to \infty\).
	In general, using Lagrange's form of the remainder,
	\[
		R_n = \frac{x^n}{n!} f^{(n)}(\theta x) = \binom{r}{n} \frac{x^n}{(1 + \theta x)^{n-r}}
	\]
	Note that in principle, \(\theta\) depends both on \(x\) and \(n\).
	For \(0 < x < 1\), \((1 + \theta x)^{n - r} > 1\) for \(n > r\).
	Now observe that the series given by
	\[
		\sum \binom{r}{n} x^n
	\]
	is absolutely convergent for \(\abs{x} < 1\).
	Indeed, we can apply the ratio test and find that
	\[
		\abs{\frac{a_{n+1}}{a_n}} = \abs{\frac{(r-n)x}{n+1}}
	\]
	which tends to \(\abs{x}\) as \(n \to \infty\).
	In particular therefore, the terms \(\binom{r}{n}x^n\) tend to zero for \(\abs{x} < 1\).
	Hence for \(n > r\) and \(0 < x < 1\), we have
	\[
		\abs{R_n} \leq \abs{\binom{r}{n}x^n} \to 0
	\]
	So the claim is proven in the range \(0 \leq x < 1\).
	If \(x < 0\), then the step when we compare \((1 + \theta x)^{n-r}\) with 1 breaks down.
	Let us instead use the Cauchy form of the remainder to bypass this step.
	\[
		R_n = \frac{(1 - \theta)^{n-1}x^n f^{(n)}(\theta x)}{(n-1)!} = \frac{(1-\theta)^{n-1} r(r-1)\cdots(r-n+1)(1+\theta x)^{r-n} x^n}{(n-1)!}
	\]
	By regrouping terms, we get
	\[
		R_n = \frac{r(r-1)\cdots(r-n+1)}{(n-1)!} \cdot \frac{(1-\theta)^{n-1}}{(1 + \theta x)^{n-r}} x^n = r\binom{r-1}{n-1}x^n (1+\theta x)^{r-1} \left( \underbrace{\frac{1-\theta}{1 + \theta x}}_{<1} \right)^{n-1}
	\]
	Hence
	\[
		\abs{R_n} \leq \abs{r \binom{r-1}{n-1}x^n} (1+\theta x)^{r-1}
	\]
	This will then tend to zero, after a bit more effort; we can bound the \((1 + \theta x)^{r-1}\) term by the maximum of \(1\) and \((1 + x)^{r-1}\), which is independent of \(n\), and then the result will follow.
\end{proof}
