\subsection{Geometry of Trigonometric Functions}
Recall that given any two vectors \(\vb x\) and \(\vb y\) in \(\mathbb R^2\), we can define the dot (scalar) product by
\[
	\vb x \cdot \vb y = (x_1, x_2) \cdot (y_1, y_2) = x_1y_1 + x_2y_2
\]
By the Cauchy-Schwarz inequality, we have
\[
	\abs{\vb x \cdot \vb y} \leq \norm{\vb x} \norm{\vb y}
\]
where we define the Euclidean norm in the normal way.
Thus, for \(\vb x \neq 0\), \(\vb y \neq 0\), we have
\[
	-1 \leq \frac{\vb x \cdot \vb y}{\norm{\vb x} \norm{\vb y}} \leq 1
\]
We now define the angle between two vectors \(\vb x\) and \(\vb y\) as exactly the unique number \(\theta \in [0, \pi]\) such that
\[
	\cos\theta = \frac{\vb x \cdot \vb y}{\norm{\vb x} \norm{\vb y}}
\]

\subsection{Hyperbolic Functions}
We define the functions \(\cosh\) and \(\sinh\) as follows.
\[
	\cosh z = \frac{1}{2}\qty(e^z + e^{-z})
\]
\[
	\sinh z = \frac{1}{2}\qty(e^z - e^{-z})
\]
Hence
\[
	\cosh z = \cos(iz);\quad \sinh z = -i\sin(iz)
\]
We can then show that
\[
	\dv{z} \cosh z = \sinh z;\quad \dv{z} \sinh z = \cosh z
\]
and further,
\[
	\cosh^2 z - \sinh^2 z \equiv 1
\]

\subsection{Defining the Riemann Integral}
\begin{definition}
	A \textit{dissection} or \textit{partition} \(\mathcal D\) of \([a, b]\) is a finite subset of \([a, b]\) containing the end points \(a\) and \(b\).
	We write
	\[
		\mathcal D = \{ x_0, x_1, \dots, x_n \}
	\]
	with \(a = x_0 < x_1 < \dots < x_{n-1} < x_n = b\).
\end{definition}
\begin{definition}
	We define the \textit{upper sum} of a bounded function \(f\) associated with a partition \(\mathcal D\) by
	\[
		S(f, \mathcal D) = \sum_{j=1}^n (x_j - x_{j-1}) \sup_{x \in [x_{j-1}, x_j]} f(x)
	\]
	The \textit{lower sum} is defined similarly,
	\[
		s(f, \mathcal D) = \sum_{j=1}^n (x_j - x_{j-1}) \inf_{x \in [x_{j-1}, x_j]} f(x)
	\]
\end{definition}
\noindent Clearly then \(S \geq s\) for all \(\mathcal D\).
\begin{lemma}
	If \(\mathcal D\) and \(\mathcal D'\) are dissections with \(\mathcal D' \supseteq \mathcal D\) (\(\mathcal D'\) is a refinement of \(\mathcal D\)), then
	\[
		S(f, \mathcal D) \underset{(\mathrm{i})}{\geq} S(f, \mathcal D') \underset{(\mathrm{ii})}{\geq} s(f, \mathcal D') \underset{(\mathrm{iii})}{\geq} s(f, \mathcal D)
	\]
\end{lemma}
\begin{proof}
	Inequality (ii) is obvious, we have already shown this to be true.
	Now, suppose \(\mathcal D'\) contains a single extra point \(y\) compared to \(\mathcal D\), where \(y \in (x_{r-1}, x_r)\).
	Clearly,
	\[
		\sup_{x \in [x_{r-1}, y]} f(x), \sup_{x \in [y, x_r]} f(x) \leq \sup_{x \in [x_{r-1}, x_r]}
	\]
	Then
	\[
		(x_r - x_{r-1}) \sup_{x \in [x_{r-1}, x_r]} f(x) \geq (y-x_{r-1}) \sup_{x \in [r_{r-1}, y]} f(x) + (x_r - y) \sup_{x \in [y, r]} f(x)
	\]
	Hence,
	\[
		S(f, \mathcal D) \geq S(f, \mathcal D')
	\]
	The same proof holds for inequality (iii), and inductively we can show that this works for any amount of extra points.
\end{proof}
\begin{lemma}
	If \(\mathcal D_1, \mathcal D_2\) are arbitrary dissections, then
	\[
		S(f, \mathcal D_1) \geq S(f, \mathcal D_1 \cup \mathcal D_2) \geq s(f, \mathcal D_1 \cup \mathcal D_2) \geq s(f, \mathcal D_2)
	\]
	In particular, \(S(f, \mathcal D_1) \geq s(f, \mathcal D_2)\).
\end{lemma}
\begin{proof}
	Let \(\mathcal D' = \mathcal D_1 \cup \mathcal D_2\), which is a refinement of both \(\mathcal D_1\) and \(\mathcal D_2\), and apply the previous lemma.
\end{proof}
\begin{definition}
	The \textit{upper integral} of \(f\) is
	\[
		I^\star(f) = \inf_{\mathcal D} S(f, \mathcal D)
	\]
	Note that such an integral always exists, since the upper sums are always bounded below by an arbitrary lower sum.
	Hence the infimum does indeed exist and is finite.
	Similarly,
	\[
		I_\star(f) = \sup_{\mathcal D} s(f, \mathcal D)
	\]
\end{definition}
\noindent Then by the lemmas above, \(I^\star(f) \geq I_\star(f)\), since \(S(f, \mathcal D_2) \geq s(f, \mathcal D_1)\) for arbitrary dissections \(\mathcal D_1\) and \(\mathcal D_2\).
\begin{definition}
	A bounded function \(f \colon [a, b] \to \mathbb R\) is (Riemann) integrable if \(I^\star(f) = I_\star(f)\).
	If this equality holds, we write
	\[
		\int_a^b f(x) \dd{x} = I^\star(f) = I_\star(f) = \int_a^b f
	\]
\end{definition}
