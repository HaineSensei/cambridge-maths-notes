\subsection{Definitions}
Let \(f \colon E \subseteq \mathbb C \to \mathbb C\).
Mostly we will take \(E\) to be an interval in the real numbers, or a disc in the complex plane.
\begin{definition}
	Let \(x \in E\) be a point such that there exists a sequence \(x_n \in E\) with \(x_n \neq x\), but \(x_n \to x\), i.e.\ \(x\) is a limit point.
	\(f\) is said to be differentiable at \(x\) with derivative \(f'(x)\) if
	\[
		\lim_{y \to x} \frac{f(y) - f(x)}{y - x} = f'(x)
	\]
\end{definition}
\noindent If \(f\) is differentiable at each point in \(E\), we say that \(f\) is differentiable on \(E\).
\begin{remark}
	One interpretation of the definition is to write it in the form
	\[
		\varepsilon(h) := f(x+h) - f(x) - hf'(x);\quad \lim_{h \to 0} \frac{\varepsilon(h)}{h} = 0
	\]
	so \(\varepsilon\) is \(o(h)\).
	Hence,
	\[
		f(x+h) = f(x) + hf'(x) + \varepsilon(h)
	\]
	We could have made an alternative definition for differentiability.
	\(f\) is differentiable at \(x\) if there exists \(A\) and \(\varepsilon\) such that
	\[
		f(x+h) = f(x) + hA + \varepsilon(h) \text{ where } \lim_{h \to 0} \frac{\varepsilon(h)}{h} = 0
	\]
	If such an \(A\) exists, then it is unique, since \(A\) is the limit
	\[
		A = \lim_{h \to 0} \frac{f(x + h) - f(x)}{h}
	\]
	We could have alternatively written the definition as
	\[
		f(x+h) = f(x) + hf'(x) + h\varepsilon_f(h) \text{ where } \lim_{h \to 0} \varepsilon_f(h) = 0
	\]
	or perhaps
	\[
		f(x) = f(a) + (x-a)f'(a) + (x-a)\varepsilon_f(x) \text{ where } \lim_{x \to a} \varepsilon_f(x) = 0
	\]
	Note further that if \(f\) is differentiable at \(x\), \(f\) is certainly continuous at \(x\).
	This follows from the fact that \(\varepsilon(h) \to 0\), and hence \(f(x+h) \to f(x)\) as \(h \to 0\).
\end{remark}
\noindent As an example, let us consider \(f(x) = \abs{x}\) for \(f \colon \mathbb R \to \mathbb R\).
Is the function at the point \(x=0\) differentiable?
If \(x > 0\), we have \(f'(x) = 1\), but if \(x < 0\), we have \(f'(x) = -1\).
These results can be checked directly using the definitions above.
But we have produced two sequences for \(h \to 0\) which give different values, so the derivative is not defined here.

\subsection{Differentiation of sums and products}
\begin{proposition}
	\begin{enumerate}
		\item If \(f(x) = c\) for all \(x \in E\), then \(f\) is differentiable with \(f'(x) = 0\).
		\item If \(f\) and \(g\) are differentiable at \(x\), then so is \(f+g\), where \((f+g)'(x) = f'(x) + g'(x)\).
		\item If \(f\) and \(g\) are differentiable at \(x\), then so is \(fg\), where \((fg)'(x) = f'(x)g(x) + g'(x)f(x)\).
		\item If \(f\) is differentiable at \(x\) and \(f(x) \neq 0\), then so is \(\frac{1}{f}\), where \((\frac{1}{f})'(x) = \frac{-f'(x)}{(f(x))^2}\).
	\end{enumerate}
\end{proposition}
\begin{proof}
	\begin{enumerate}
		\item \(\lim_{h \to 0} \frac{c-c}{h} = 0\) as required.
		\item Since all relevant limits are well-defined,
		      \[
			      \lim_{h \to 0} \frac{f(x+h) + g(x+h) - f(x) - g(x)}{h} = \lim_{h \to 0} \frac{f(x+h) - f(x)}{h} + \lim_{h \to 0} \frac{g(x+h) - g(x)}{h} = f'(x) + g'(x)
		      \]
		\item Let \(\phi(x) = f(x)g(x)\).
		      Then, since \(f\) is continuous at \(x\),
		      \begin{align*}
			      \lim_{h \to 0} \frac{\phi(x+h) - \phi(x)}{h} & = \lim_{h \to 0} \frac{f(x+h)g(x+h) - f(x)g(x)}{h}                            \\
			                                                   & = \lim_{h \to 0} f(x+h) \frac{g(x+h) - g(x)}{h} + g(x)\frac{f(x+h) - f(x)}{h} \\
			                                                   & = \lim_{h \to 0} f(x) \frac{g(x+h) - g(x)}{h} + g(x)\frac{f(x+h) - f(x)}{h}   \\
			                                                   & = f(x)g'(x) + g(x)f'(x)
		      \end{align*}
		\item Let \(\phi(x) = \frac{1}{f(x)}\).
		      Then,
		      \begin{align*}
			      \lim_{h \to 0} \frac{\phi(x+h) - \phi(x)}{h} & = \lim_{h \to 0} \frac{\frac{1}{f(x+h)} - \frac{1}{f(x)}}{h} \\
			                                                   & = \lim_{h \to 0} \frac{f(x) - f(x+h)}{hf(x)f(x+h)}           \\
			                                                   & = \frac{-f'(x)}{f(x)f(x)}                                    \\
		      \end{align*}
	\end{enumerate}
\end{proof}
\begin{remark}
	From (iii) and (iv), we can immediately find the quotient rule,
	\[
		\left( \frac{f(x)}{g(x)} \right)' = \frac{g(x)f'(x) - f(x)g'(x)}{(g(x))^2}
	\]
\end{remark}

\subsection{Differentiating polynomial terms}
As an example of the differentiability properties we saw last lecture, we can find the derivative of \(f(x) = x^n\) for \(n \in \mathbb Z\), \(n > 0\).
If \(n=1\), clearly \(f'(x) = 1\).
We can show inductively that \(f'(x) = nx^{n-1}\).
Indeed,
\begin{align*}
	(x^n)' & = x \cdot (x^{n-1})' + (x)' \cdot x^{n-1} \\
	       & = (n-1)x^{n-1} + x^{n-1}                  \\
	       & = nx^{n-1}
\end{align*}
We can now take \(f(x) = x^{-n}\).
Using the reciprocal law,
\begin{align*}
	f'(x) & = \frac{-(x^n)'}{(x^n)^2}  \\
	      & = \frac{-nx^{n-1}}{x^{2n}} \\
	      & = -nx^{-n-1}
\end{align*}

\subsection{Chain rule}
\begin{theorem}
	Let \(f \colon U \to \mathbb C\) be such that \(f(x) \in V\) for all \(x \in U\).
	If \(f\) is differentiable at \(a \in U\), and \(g \colon V \to \mathbb C\) is differentiable at \(f(a) \in V\), then \(g \circ f\) is differentiable at \(a\) with
	\[
		gf'(a) = f'(a)g'(f(a))
	\]
\end{theorem}
\begin{proof}
	We know that we can write
	\[
		f(x) = f(a) + (x-a)f'(a) + \varepsilon_f(x)(x-a)
	\]
	where \(\lim_{x \to a} \varepsilon_f(x) = 0\).
	Further,
	\[
		g(y) = g(b) + (y-b)g'(b) + \varepsilon_g(y)(y-b)
	\]
	where \(\lim_{y \to b} \varepsilon_g(y) = 0\), and \(b = f(a)\).
	We will set \(\varepsilon_f(a) = 0\) and \(\varepsilon_g(b) = 0\), so they are continuous at \(x=a\) and \(y=b\), so that everything is well-defined when we begin to compose the functions.
	Now, \(y=f(x)\), so
	\begin{align*}
		g(f(x)) & =  g(b) + (f(x) - b)g'(b) + \varepsilon_g(f(x))(f(x) - b)                                                                                                       \\
		        & = g(f(a)) + \left[ (x-a)f'(a) + \varepsilon_f(x)(x-a) \right]\left[ g'(b) + \varepsilon_g(f(x)) \right]                                                         \\
		        & = g(f(a)) + (x-a)f'(a)g'(b) + (x-a)\underbrace{\left[ \varepsilon_f(x) g'(b) + \varepsilon_g(f(x)) \left( f'(a) + \varepsilon_f(x) \right) \right]}_{\sigma(x)} \\
	\end{align*}
	Now, we just need to show that \(\lim_{x \to a} \sigma(x) = 0\) in order to prove the theorem.
	Clearly
	\[
		\sigma(x) = \underbrace{\varepsilon_f(x)}_{\to 0} g'(b) + \underbrace{\varepsilon_g(f(x))}_{\to 0} \left( f'(a) + \varepsilon_f(x) \right)
	\]
	Hence \(\sigma(x) \to 0\) as required.
\end{proof}

\subsection{Rolle's theorem}
\begin{theorem}
	Let \(f \colon [a,b] \to \mathbb R\) be a continuous function on \([a,b]\) and differentiable on \((a, b)\).
	If \(f(a) = f(b)\), then there exists \(c \in (a,b)\) such that \(f'(c) = 0\).
\end{theorem}
\begin{proof}
	Let \(M\) be the maximum point and \(m\) be the minimum point of the function.
	Recall that in Lecture 8 we proved that any function achieves its bounds.
	Let \(k = f(a)\).
	If \(M=m=k\), then \(f\) must be a constant, and clearly \(f'(c) = 0\) for every value \(c \in (a, b)\).
	Otherwise, either \(M > k\) or \(m < k\).
	Suppose \(M > k\) (the proof is very similar if \(m < k\)).
	Then there exists some value \(c \in (a, b)\) such that \(f(c) = M\).
	We would like to show that \(f'(c) = 0\), so let us suppose that \(f'(c) \neq 0\).
	If \(f'(c) > 0\), then there are values \(d > c\) where \(f(d) > f(c)\).
	Indeed,
	\[
		f(h+c) - f(c) = h\left[ f'(c) + \varepsilon(h) \right]
	\]
	For a small, positive \(h\), this value is positive.
	This contradicts the fact that \(M\) is the maximum.
	Similarly, if \(f'(c) < 0\) there are values \(d < c\) with \(f(d) > f(c)\).
	Hence \(f'(c) = 0\).
\end{proof}

\subsection{Mean value theorem}
We can make a small change to Rolle's theorem and obtain the mean value theorem.
\begin{theorem}
	Let \(f \colon [a,b] \to \mathbb R\) be a continuous function on \([a,b]\) and differentiable on \((a, b)\).
	Then there exists \(c \in (a, b)\) such that
	\[
		f(b) - f(a) = f'(c)(b-a)
	\]
\end{theorem}
\begin{proof}
	Let \(\phi\) be a function defined by \(\phi(x) = f(x) - kx\), choosing a \(k\) such that \(\phi(a) = \phi(b)\).
	We can find that
	\[
		f(b) - bk = f(a) - ak \implies k = \frac{f(b) - f(a)}{b - a}
	\]
	By Rolle's theorem, there exists \(c \in (a,b)\) such that \(\phi'(c) = 0\).
	Now, note that \(f'(x) = \phi'(x) + k\), hence there exists \(c\) such that \(f'(c) = k\).
\end{proof}
\begin{remark}
	We will often rewrite the mean value theorem as follows.
	\[
		f(a + h) = f(a) + hf'(a + \theta h)
	\]
	where \(\theta \in (0, 1)\).
	Note, however, that \(\theta\) is a function of \(h\), so if we begin to shrink \(h\) then \(\theta\) may change.
\end{remark}

\subsection{Properties of a function from its derivative}
We can deduce certain facts about a function by observing the properties its derivative exhibits.
These results are mostly trivial corollaries to the mean value theorem, proven in the last lecture.
\begin{corollary}
	Let \(f \colon [a,b] \to \mathbb R\) be continuous, and differentiable on \((a, b)\).
	Then we have
	\begin{enumerate}
		\item If \(f'(x) > 0\) for all \(x \in (a, b)\), then \(f\) is strictly increasing on \([a, b]\);
		\item If \(f'(x) \geq 0\) for all \(x \in (a, b)\), then \(f\) is increasing on \([a, b]\);
		\item If \(f'(x) = 0\) for all \(x \in (a, b)\), then \(f\) is constant on \([a, b]\).
	\end{enumerate}
\end{corollary}
\noindent Part (iii) of this corollary is essentially solving the most simple differential equation; we are showing that the only possible solutions to this equation are the constant functions.
Note that similar statements about decreasing functions hold.
\begin{proof}
	\begin{enumerate}
		\item We have \(f(y) - f(x) = f'(c)(y-x)\) for some \(c \in (x, y)\).
		      If \(f'(c) > 0\), then \(f(y) - f(x) > 0\).
		\item Analogously to before, \(f(y) - f(x) = f'(c)(y-x)\) for some \(c \in (x, y)\).
		      If \(f'(c) \geq 0\), then  \(f(y) - f(x) \geq 0\).
		\item By the mean value theorem on \([a, x]\), if \(f'(c) = 0\), then \(f(x) - f(a) = 0\).
	\end{enumerate}
\end{proof}

\subsection{Inverse function theorem}
\begin{theorem}
	Let \(f \colon [a, b] \to \mathbb R\) be a continuous function on \([a, b]\) and differentiable on \((a, b)\), with \(f'(x) > 0\) everywhere on \((a, b)\).
	Let \(f(a) = c, f(b) = d\).
	Then the function \(f \colon [a,b] \to [c,d]\) is bijective, and \(f^{-1} \colon [c,d] \to [a,b]\) is differentiable on \((c, d)\) with
	\[
		\left( f^{-1} \right)' (x) = \frac{1}{f'\left(f^{-1}(x)\right)}
	\]
\end{theorem}
\noindent Note, in lecture 8 it was proven that a continuous strictly increasing function has a continous inverse.
This strengthens that claim to include the differentiability property if the original function was differentiable.
\begin{proof}
	We know from lecture 8 that there exists \(g \colon [c,d] \to [a,b]\) which is a strictly increasing continuous function, which is the inverse of \(f\).
	We must now show that \(g\) is differentiable and that its derivative has the required form as stated in the claim.
	Now, let \(y = f(x)\).
	Given \(k \neq 0\), let \(h\) be given by
	\[
		y + k = f(x+h)
	\]
	Alternatively, written in terms of \(g\),
	\[
		x + h = g(y + k)
	\]
	So clearly \(h \neq 0\).
	Since \(g\) is continuous, if \(k \to 0\) then \(h \to 0\).
	Then
	\begin{align*}
		\frac{g(y + k) - g(y)}{k}                           & = \frac{x + h - x}{f(x+h) - y}           \\
		                                                    & = \frac{h}{f(x+h) - f(x)}                \\
		\therefore \lim_{k \to 0} \frac{g(y + k) - g(y)}{k} & = \lim_{h \to 0} \frac{h}{f(x+h) - f(x)} \\
		                                                    & = \frac{1}{f'(x)}
	\end{align*}
	as required.
\end{proof}

\subsection{Derivative of rational powers}
First, let \(g(x) = x^{1/q}\) for some positive integer \(q\).
We can find that \(f(x) = x^q\) has the derivative \(f'(x) = qx^{q-1}\).
By the inverse function theorem, \(g'(x) = \frac{1}{q}x^{1/q - 1}\).
Now, if \(g(x) = x^{p/q}\), where \(p\) is an integer and \(q\) is a positive integer, then by the chain rule \(g'(x) = \frac{p}{q}x^{p/q - 1}\) which matches the expected result.

\subsection{Mean value theorem applied to limits}
Suppose \(f, g \colon [a,b] \to \mathbb R\) are continuous, and differentiable on \((a, b)\).
Suppose further that \(g(a) \neq g(b)\).
The mean value theorem can be applied to both functions, and will give two points \(s, t \in (a, b)\) such that
\[
	\frac{f(b) - f(a)}{g(b) - g(a)} = \frac{(b-a)f'(s)}{(b-a)g'(t)} = \frac{f'(s)}{g'(t)}
\]
This gives us a way to simplify a limit of the form of the left hand side (as \(b \to a\)) by instead considering the right hand side.
We can apply Cauchy's mean value theorem, seen in the next lecture.

\subsection{Cauchy's mean value theorem}
\begin{theorem}
	If \(f, g \colon [a,b] \to \mathbb R\) are continuous, and differentiable on \((a, b)\), there exists \(t \in (a,b)\) such that
	\[
		(f(b) - f(a))g'(t) = f'(t)(g(b) - g(a))
	\]
\end{theorem}
\noindent We can recover the normal mean value theorem from Cauchy's generalisation by taking \(g(x) = x\).
\begin{proof}
	Let
	\[
		\phi(x) = \begin{vmatrix}
			1    & 1    & 1    \\
			f(a) & f(x) & f(b) \\
			g(a) & g(x) & g(b)
		\end{vmatrix}
	\]
	Certainly \(\phi(x)\) is continuous on \([a,b]\) and differentiable on \((a, b)\), by using previous results.
	Also, \(\phi(a) = \phi(b) = 0\) by observing the linear dependence of the columns.
	By Rolle's theorem, there exists \(t \in (a, b)\) such that \(\phi'(t) = 0\).
	We can expand \(\phi'(t)\) and this will show the required result.
	\[
		\phi'(x) = f'(x)g(b) - g'(x)f(b) + f(a)g'(x) - g(a)f'(x) = f'(x) [g(b) - g(a)] + g'(x) [f(a) - f(b)]
	\]
\end{proof}

\subsection{Example of l'H\^opital's rule}
The derivation of l'H\^opital's rule is on an example sheet, so in this subsection we will consider only a special case of it, using Cauchy's mean value theorem.
\[
	\ell = \lim_{x \to 0} \frac{e^x - 1}{\sin x}
\]
We can write
\[
	\ell = \lim_{x \to 0} \frac{e^x - e^0}{\sin x - \sin 0} = \frac{e^t}{\cos t}
\]
for some \(t \in (0, x)\).
So as \(x \to 0\), \(t \to 0\) and hence
\[
	\frac{e^t}{\cos t} \to 1
\]
