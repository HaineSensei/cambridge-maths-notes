\subsection{Absolute convergence}
\begin{definition}
	Let \(a_n \in \mathbb C\).
	Then if \(\sum_{n=1}^\infty \abs{a_n}\) converges, then the series is called absolutely convergent.
\end{definition}
\begin{remark}
	Since \(\abs{a_n} \geq 0\), we can use the previous tests to check for absolute convergence.
\end{remark}
\begin{theorem}
	Let \(a_n \in \mathbb C\).
	If this series is absolutely convergent, it is convergent.
\end{theorem}
\begin{proof}
	Suppose first that \(a_n\) is a sequence of real numbers.
	Then let
	\[
		v_n = \begin{cases}
			a_n & \text{if } a_n \geq 0 \\
			0   & \text{if } a_n < 0
		\end{cases};\quad w_n = \begin{cases}
			0    & \text{if } a_n \geq 0 \\
			-a_n & \text{if } a_n < 0
		\end{cases}
	\]
	Hence,
	\[
		v_n = \frac{\abs{a_n} + a_n}{2};\quad w_n = \frac{\abs{a_n} - a_n}{2}
	\]
	Clearly, \(v_n, w_n \geq 0\), and \(a_n = v_n - w_n\), and \(\abs{a_n} = v_n + w_n\).
	If \(\sum \abs{a_n}\) converges, then by comparison \(\sum v_n\) and \(\sum w_n\) also converge, and hence \(\sum a_n\) converges.
	Now, let us consider the case where \(a_n\) is complex.
	Then we can write \(a_n = x_n + iy_n\) where \(x_n, y_n\) are real sequences.
	Note that \(\abs{x_n}, \abs{y_n} \leq \abs{a_n}\).
	So by comparison \(x_n\) and \(y_n\) converge, so \(a_n\) converges.
\end{proof}
Here are some examples.
\begin{enumerate}
	\item The alternating harmonic series \(\sum \frac{(-1)^n}{n}\) is convergent, but not absolutely convergent.
	\item \(\sum \frac{z^n}{2^n}\) is absolutely convergent when \(\abs{z} < \abs{2}\), because it reduces to a real geometric series.
	      If \(\abs{z} \geq 2\), then \(\abs{a_n} \geq 1\), so we do not have absolute convergence.
\end{enumerate}

\subsection{Conditional convergence and rearrangement}
If the series is convergent but not absolutely convergent, it is called \textit{conditionally} convergent.
The sum to which a series converges depends on the order in which the terms are added.
\begin{definition}
	Let \(\sigma\) be a bijection of the positive integers to itself, then
	\[
		a_n' = a_{\sigma(n)}
	\]
	is a rearrangement of \(a_n\).
\end{definition}
\begin{theorem}
	If \(\sum_1^\infty a_n\) is absolutely convergent, then every rearrangement of this series converges to the same value.
\end{theorem}
\begin{proof}
	First, let us consider the real case.
	Let \(\sum a_n'\) be a rearrangement of \(\sum a_n\).
	Let \(s_n = \sum_1^n a_n\), and \(t_n = \sum_1^n a_n'\).
	Let \(s_n\) converge to \(s\).
	Suppose first that \(a_n \geq 0\).
	Then given any \(n \in \mathbb N\), we can find some \(q \in \mathbb N\) such that \(s_q\) contains every term of \(t_n\).
	Since the \(a_n \geq 0\),
	\[
		t_n \leq s_q \leq s
	\]
	As \(n \to \infty\), the \(t_n\) is an increasing sequence bounded above, so it must tend to a limit \(t\), where \(t \leq s\).
	Note, however, that this argument is symmetric; we can equally derive that \(s \leq t\).
	Therefore \(s = t\).

	Now, let us drop the condition that \(a_n \geq 0\).
	We can now consider \(v_n, w_n\) from above:
	\[
		v_n = \frac{\abs{a_n} + a_n}{2};\quad w_n = \frac{\abs{a_n} - a_n}{2}
	\]
	Since \(\sum\abs{a_n}\) converges, both \(\sum v_n, \sum w_n\) converge.
	Since all \(v_n, w_n \geq 0\), we can deduce that \(\sum v_n = \sum v_n'\) and \(\sum w_n' = \sum w_n\).
	The claim follows since \(a_n = v_n - w_n\).

	For the case \(a_n \in \mathbb C\), we can write \(a_n = x_n + iy_n\), noting that \(\abs{x_n}, \abs{y_n} \geq \abs{a_n}\).
	By comparison, the series \(\sum x_n, \sum y_n\) are absolutely convergent, and by the previous case, \(\sum x_n = \sum x_n'\) and \(\sum y_n' = \sum y_n'\).
	Since \(a_n' = x_n' + y_n'\), \(\sum a_n = \sum a_n'\) as required.
\end{proof}
