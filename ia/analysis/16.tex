\subsection{Exponential and Logarithmic Functions}
Last lecture, we covered the power series form of the exponential function \(e \colon \mathbb C \to \mathbb C\).
Note that if we input a real number, the output is also real.
Hence, \(e \colon \mathbb R \to \mathbb R\).
This restricted definition of the function has the following properties.
\begin{theorem}
	\begin{enumerate}[(i)]
		\item \(e \colon \mathbb R \to \mathbb R\) is everywhere differentiable, and \(e'(x) = e(x)\).
		\item \(e(x+y) = e(x)e(y)\).
		\item \(e(x) > 0\).
		\item \(e\) is strictly increasing.
		\item \(e(x) \to \infty\) as \(x \to \infty\), and \(e(x) \to 0\) as \(x \to -\infty\).
		\item \(e \colon \mathbb R \to (0, \infty)\) is a bijection.
	\end{enumerate}
\end{theorem}
\begin{proof}
	The first two properties follow from the last lecture.
	\begin{enumerate}[(i)]
		\setcounter{enumi}{2}
		\item Clearly, \(e(x) > 0\) for all \(x \geq 0\) by considering the power series, which contains only positive terms for \(x>0\), and also \(e(0) = 1\).
		      Also, \(e(0) = e(x - x) = e(x)e(-x)\), hence for all negative \(x\), \(e(x) > 0\).
		\item Since \(e'(x) = e(x)\), \(e'(x) = e(x) > 0\) everywhere.
		\item By considering partial sums, if \(x>0\) we have \(e(x) > 1+x\), so if \(x \to \infty\), \(e(x) \to \infty\).
		      When \(x \to -\infty\), \(e(x) = \frac{1}{e(x)} \to 0\).
		\item Injectivity follows from being strictly increasing.
		      For surjectivity, we need to show that given any \(y \in (0, \infty)\) there exists some \(x\) such that \(e(x) = y\).
		      Due to property (v) above, we can certainly find real numbers \(a\) and \(b\) such that \(e(a) < y < e(b)\).
		      By the intermediate value theorem, there exists \(x \in \mathbb R\) such that \(e(x) = y\).
	\end{enumerate}
\end{proof}
\begin{remark}
	We have essentially proven that \(e \colon (\mathbb R, +) \to ((0, \infty), \times)\) is a group isomorphism.
	This is exactly the same as showing that it is a bijection.
	Since \(e\) is a function, there exists an inverse function \(\ell \colon ((0, \infty), \times) \to (\mathbb R, +)\).
\end{remark}
\begin{theorem}
	\begin{enumerate}[(i)]
		\item \(\ell \colon (0, \infty) \to \mathbb R\) is a bijection, and \(\ell(e(x)) = x\) for all \(x \in \mathbb R\), and \(e(\ell(x)) = x\) for all \(x \in (0, \infty)\).
		\item \(\ell\) is differentiable and its derivative is \(\ell'(t) = \frac{1}{t}\).
		\item \(\ell(xy) = \ell(x) + \ell(y)\).
	\end{enumerate}
\end{theorem}
\begin{proof}
	\begin{enumerate}[(i)]
		\item This first propety is obvious from the definition.
		\item By the inverse function theorem, \(\ell\) is differentiable everywhere and \(\ell'(t) = \frac{1}{t}\) as required.
		\item From IA Groups, if \(e\) is an isomorphism, so is its inverse.
	\end{enumerate}
\end{proof}

\subsection{Real Numbered Exponents}
We will now define for \(\alpha \in \mathbb R\) and \(x > 0\) the function
\[
	r_\alpha(x) = e(\alpha \ell(x))
\]
This can be taken as the definition of \(x\) raised to the power \(\alpha\).
\begin{theorem}
	Suppose \(x, y > 0\) and \(\alpha, \beta \in \mathbb R\).
	Then
	\begin{enumerate}[(i)]
		\item \(r_\alpha(xy) = r_\alpha(x)r_\alpha(y)\)
		\item \(r_{\alpha + \beta}(x) = r_\alpha(x) r_\beta(x)\)
		\item \(r_\alpha(r_\beta(x)) = r_{\alpha\beta}(x)\)
		\item \(r_1(x) = x\), and \(r_0(x) = 1\)
	\end{enumerate}
\end{theorem}
\begin{proof}
	\begin{enumerate}[(i)]
		\item \(r_\alpha(xy) = e(\alpha \ell(xy)) = e(\alpha \ell(x) + \alpha \ell(y)) = e(\alpha \ell(x))e(\alpha\ell(y)) = r_\alpha(x)r_\alpha(y)\)
		\item \(r_{\alpha + \beta}(x) = e((\alpha + \beta) \ell(x)) = e(\alpha\ell(x) + \beta\ell(x)) = e(\alpha\ell(x))e(\beta\ell(x)) = r_\alpha(x) r_\beta(x)\)
		\item \(r_\alpha(r_\beta(x)) = e(\alpha \ell[e(\beta \ell(x))]) = e(\alpha \beta \ell(x)) = r_{\alpha\beta}(x)\)
		\item \(r_1(x) = e(\ell(x)) = x\), and \(r_0(x) = e(0 \ell(x)) = e(0) = 1\)
	\end{enumerate}
\end{proof}
\noindent Suppose we want to compute \(r_n(x)\), where \(n \in\mathbb Z\).
Then \(r_n(x) = r_{1 + \dots + 1}(x) = x \cdots x\), so we have aggreement between \(r_n(x)\) and our previous definition of \(x^n\).
Similarly, since \(r_1(x) r_{-1}(x) = 1\), we have \(r_{-1}(x) = \frac{1}{x}\).
Further, \(r_{\frac{1}{q}}(x) = x^\frac{1}{q}\).
Therefore, \(r_{\frac{p}{q}}(x) = x^{\frac{p}{q}}\).
So this definition is simply a more general definition for exponentiation by a real number.

From now, we will let \(\exp(x) \equiv e(x)\), \(\log(x) \equiv \ell(x)\), and \(x^\alpha \equiv r_\alpha(x)\).
In fact, \(\exp(x) = e^x\) for a suitable number \(e\), since \(e(x) = e(x \log(e)) = r_x(e) = e^x\) where \(e := e(1) = \sum_0^\infty \frac{1}{n!}\).

Finally, we can compute the derivative of \(x^\alpha\) using the chain rule.
\[
	(x^\alpha)' = \left( e^{\alpha \log x} \right)' = e^{\alpha \log x} \alpha \frac{1}{x} = \alpha x^\alpha x^{-1} = \alpha x^{\alpha - 1}
\]
as expected.
Further, if \(f(x) = a^x\), we can find
\[
	f'(x) = \left( e^{x \log a} \right)' = e^{x \log a} \log a = a^x \log a
\]
