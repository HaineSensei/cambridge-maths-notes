\subsection{Breaking an Interval}
Let $f$ be integrable on $[a, b]$. If $a < c < b$, then $f$ is integrable over $[a, c]$ and $[c, b]$, with
\[ \int_a^b f = \int_a^c f + \int_c^b f \]
Conversely, if $f$ is integrable on $[a, c]$ and $[c, b]$, then $f$ is integrable over $[a, b]$ and the same equality holds for the combination of the integrals.
\begin{proof}
	We first make two observations. First, if $\mathcal D_1$ is a dissection of $[a, c]$ and $\mathcal D_2$ is a dissection of $[c, b]$, then $\mathcal D = \mathcal D_1 \cup \mathcal D_2$ is a dissection of $[a, b]$, and
	\begin{equation}
		S(f, \mathcal D_1 \cup \mathcal D_2) = S\qty(\eval{f}_{[a, c]}, \mathcal D_1) + S\qty(\eval{f}_{[c, b]}, \mathcal D_2)
		\tag{$\ast$}
	\end{equation}
	Also, if $\mathcal D$ is a dissection of $[a, b]$, then
	\begin{equation}
		S(f, \mathcal D) \geq S(f, \mathcal D \cup \{ c \}) = S\qty(\eval{f}_{[a, c]}, \mathcal D_1) + S\qty(\eval{f}_{[c, b]}, \mathcal D_2)
		\tag{$\dagger$}
	\end{equation}
	Now,
	\[ (\ast) \implies I^\star(f) \leq I^\star\qty(\eval{f}_{[a, c]}) + I^\star\qty(\eval{f}_{[c, b]}) \]
	Further,
	\[ (\dagger) \implies I^\star(f) \geq I^\star\qty(\eval{f}_{[a, c]}) + I^\star\qty(\eval{f}_{[c, b]}) \]
	Hence,
	\[ I^\star(f) = I^\star\qty(\eval{f}_{[a, c]}) + I^\star\qty(\eval{f}_{[c, b]}) \]
	This argument also applies for the lower integral, therefore
	\[ 0 \leq I^\star(f) - I_\star(f) = \underbrace{I^\star\qty(\eval{f}_{[a, c]}) - I_\star\qty(\eval{f}_{[a, c]})}_{A} + \underbrace{I^\star\qty(\eval{f}_{[c, b]}) + I_\star\qty(\eval{f}_{[c, b]})}_{B} \]
	Note that $A, B \geq 0$. If $f$ is integrable on $[a, c]$ and $[c, b]$, then $A = B = 0$, hence $I^\star(f) = I_\star(f)$ and it is integrable on $[a, b]$. If $f$ is integrable on $[a, b]$, then we know $I^\star(f) = I_\star(f)$, so $A = B = 0$ so $f$ is integrable on $[a, c]$ and $[c, b]$.
\end{proof}
\noindent Note that we take the following convention:
\[ \int_a^b f = -\int_b^a f \]
and if $a=b$, then this value is zero. With this convention, if $f$ is bounded with $\abs{f} \leq k$, then
\[ \abs{\int_a^b f} \leq k\abs{b - a} \]

\subsection{Fundamental Theorem of Calculus}
Suppose a function $f \colon [a, b] \to \mathbb R$ is bounded and integrable. Then since it is integrable on any sub-interval, we can define
\[ F(x) = \int_a^x f(t) \dd{t} \]
for $x \in [a, b]$.
\begin{theorem}
	$F$ is continuous.
\end{theorem}
\begin{proof}
	We know that
	\[ F(x + h) - F(x) = \int_x^{x+h} f(t)\dd{t} \]
	We want this quantity to vanish as $h \to 0$. We find, given that $f$ is bounded by $k$,
	\[ \abs{F(x+h) - F(x)} = \abs{\int_x^{x+h} f(t) \dd{t}} \leq k\abs{h} \]
	So the result follows as $h \to 0$.
\end{proof}
\begin{theorem}
	If in addition $f$ is continuous at $x$, then $F$ is differentiable, with $F'(x) = f(x)$.
\end{theorem}
\begin{proof}
	Consider
	\[ \abs{\frac{F(x + h) - F(x)}{h} - f(x)} \]
	If this tends to zero, then the theorem holds.
	\[ \abs{\frac{F(x + h) - F(x)}{h} - f(x)} = \frac{1}{\abs{h}} \abs{\int_x^{x+h} f(t) \dd{t} - hf(x)} = \frac{1}{\abs{h}} \abs{\int_x^{x+h} [f(t) - f(x)] \dd{t}} \]
	Since $f$ is continuous at $x$, given $\varepsilon > 0$, $\exists \delta > 0$ such that $\abs{t - x} - \delta \implies \abs{f(t) - f(x)} < \varepsilon$. If $\abs{h} < \delta$, then the integrand is bounded by $\varepsilon$. Hence,
	\[ \abs{\frac{F(x + h) - F(x)}{h} - f(x)} \leq \frac{1}{\abs{h}} \abs{h\varepsilon} = \varepsilon \]
	So we can make this value as small as we like. So the theorem holds.
\end{proof}
\noindent For example, consider the function
\[ f(x) = \begin{cases}
		-1 & x \in [-1, 0] \\
		1  & x \in (0, 1]
	\end{cases} \]
This is a bounded, integrable function, with
\[ F(x) = -1 + \abs{x} \]
Note that this $F$ is not differentiable at $x = 0$.
\begin{corollary}
	If $f = g'$ is a continuous function on $[a, b]$, then
	\[ \int_a^x f(t) \dd{t} = g(x) - g(a) \]
	is a differentiable function on $[a, b]$.
\end{corollary}
\begin{proof}
	From above, $F - g$ has zero derivative in $[a, b]$, hence $F - g$ is constant. Since $F(a) = 0$, we get $F(x) = g(x) - g(a)$.
\end{proof}
\noindent Note that every continuous function $f$ has an `indefinite' integral (or `antiderivative') written $\int f(x)\dd{x}$, which is determined uniquely up to the addition of a constant. Note further that we have now essentially solved the differential equation
\[ \left\{\begin{array}{l}
		y'(x) = f(x) \\
		y(a) = y_0
	\end{array}\right. \]
and shown that there is a unique solution to this ordinary differential equation.
