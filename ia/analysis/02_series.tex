\subsection{Definition}
Let \(a_n\) be a real or complex sequence.
We say that \(\sum_{j=1}^\infty a_j\) converges to \(s\) if the sequence of partial sums \(s_N\) converges to \(s\) as \(N \to \infty\), i.e.
\[
	s_N = \sum_{j=1}^N a_j \to s
\]
If the sequence of partial sums does not converge, then we say that the series diverges.
Note that any problem on series can be turned into a problem on sequences, by considering their partial sums.
\begin{lemma}
	\begin{enumerate}[(i)]
		\item If \(\sum_{j=1}^\infty a_j\) and \(\sum_{j=1}^\infty b_j\) converge, then so does \(\sum_{j=1}^\infty (\lambda a_j + \mu b_j)\), where \(\lambda, \mu \in \mathbb C\).
		\item Suppose \(\exists N\) such that \(a_j = b_j\) for all \(j \geq N\).
		      Then either \(\sum_{j=1}^\infty a_j\) and \(\sum_{j=1}^\infty b_j\) both converge, or they both diverge.
		      In other words, the initial terms do not matter for considering convergence (but the sum will change).
	\end{enumerate}
\end{lemma}
\begin{proof}
	\begin{enumerate}[(i)]
		\item We have
		      \begin{align*}
			      s_N            & = \sum_{j=1}^N (\lambda a_j + \mu b_j)            \\
			                     & = \sum_{j=1}^N \lambda a_j + \sum_{j=1}^N \mu b_j \\
			                     & = \lambda c_N + \mu d_N                           \\
			      \therefore s_N & \to \lambda c + \mu d
		      \end{align*}
		\item For any \(n \geq N\), we have
		      \begin{align*}
			      s_N & = \sum_{j=1}^n a_j = \sum_{j=1}^{N-1} a_j + \sum_{j=n}^N a_j \\
			      d_N & = \sum_{j=1}^n b_j = \sum_{j=1}^{N-1} b_j + \sum_{j=n}^N b_j \\
		      \end{align*}
		      Taking the difference, we get
		      \[
			      s_N - d_N = \sum_{j=1}^{N-1} a_j - \sum_{j=1}^{N-1} b_j
		      \]
		      which is finite.
		      So \(s_N\) converges if and only if \(d_N\) also converges.
	\end{enumerate}
\end{proof}

\subsection{Geometric series}
Let \(a_n = x^{n-1}\), where \(n \geq 1\).
Then
\[
	s_n = \sum_{j=1}^n a_j = 1 + x + x^2 + \dots + x^{n-1}
\]
Then
\[
	s_n = \begin{cases}
		\frac{1 - x^n}{1 - x} & \text{if } x \neq 1 \\
		n                     & \text{if } x = 1
	\end{cases}
\]
This can be shown by observing that
\[
	x s_n = x + x^2 + \dots + x^n = s_n - 1 + x^n \implies s_n(1-x) = 1-x^n
\]
If \(\abs{x} < 1\), then \(x^n \to 0\) as \(x \to \infty\).
So \(s_n \to \frac{1}{1-x}\).
If \(x > 1\), then \(x^n \to \infty\) and so \(s_n \to \infty\).
If \(x < -1\), \(s_n\) oscillates.
For completeness, if \(x=-1\), \(s_n\) oscillates between 0 and 1.

Note that the statement \(s_n \to \infty\) means that given \(a \in \mathbb R\), \(\exists N\) such that \(s_n > a\) for all \(n \geq N\), and a similar statement holds for negative infinity (swapping the inequality).
If \(s_n\) does not converge or tend to \(\pm \infty\), we say that \(s_n\) oscillates.

Thus the geometric series converges if and only if \(\abs{x} < 1\).
Note that to prove that \(x^n \to 0\) if \(\abs{x} < 1\), we can consider the case \(0 < x < 1\) and write \(1/x = 1 + \delta\) for some positive \(\delta\).
Then \(x^n = \frac{1}{(1 + \delta)^n} \leq \frac{1}{1 + \delta n}\) from the binomial expansion, and this tends to zero as required.

\begin{lemma}
	If \(\sum_{j=1}^\infty a_j\) converges, then \(\lim_{j \to \infty} a_j = 0\).
\end{lemma}
\begin{proof}
	Given \(s_n = \sum_{j=1}^n a_j\), we have \(a_n = s_n - s_{n-1}\).
	If \(s_n \to a\), then \(a_n \to 0\) since \(s_{n-1}\) also tends to \(a\).
\end{proof}
\begin{remark}
	The converse is not true.
	For example, the harmonic series diverges, but the terms approach zero.
	Consider
	\begin{align*}
		s_{2n} & = s_n + \frac{1}{n+1} + \frac{1}{n+2} + \dots + \frac{1}{2n} \\
		       & > s_n + \frac{1}{2n} + \frac{1}{2n} + \dots + \frac{1}{2n}   \\
		       & = s_n + \frac{1}{2}
	\end{align*}
	So as \(n \to \infty\), if the sequence is convergent then the sequences \(s_n\) and \(s_{2n}\) tend to the same limit, but they clearly do not.
\end{remark}
