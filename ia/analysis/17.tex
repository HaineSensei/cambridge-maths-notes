\subsection{Power Series Definitions}
We define
\begin{align*}
	\cos z & = 1 - \frac{z^2}{2!} + \frac{z^4}{4!} - \frac{z^6}{6!} + \dots = \sum_0^\infty \frac{(-1)^k z^{2k}}{(2k)!}     \\
	\sin z & = z - \frac{z^3}{3!} + \frac{z^5}{5!} - \frac{z^7}{7!} + \dots = \sum_0^\infty \frac{(-1)^k z^{2k+1}}{(2k+1)!} \\
\end{align*}
Both power series have infinite radius of convergence, by the ratio test (the same proof from the exponential function can be used here). Hence \(\cos z\) and \(\sin z\) are differentiable, and \(\dv{z}\cos z = -\sin z\) and \(\dv{z}\sin z = \cos z\) as expected, by termwise differentiation. Further, we can deduce that
\[ e^{iz} = \sum_0^\infty \frac{(iz)^n}{n!} = \sum_0^\infty \frac{(iz)^{2k}}{(2k)!} + \sum_0^\infty \frac{(iz)^{2k+1}}{(2k+1)!} \]
Note that
\[ (iz)^{2k} = (-1)^k z^{2k};\quad (iz)^{2k+1} = i (-1)^k z^{2k+1} \]
Hence,
\[ e^{iz} = \cos z + i \sin z \]
Similarly,
\[ e^{-iz} = \cos z - i \sin z \]
We can then write
\[ \cos z = \frac{1}{2}\qty( e^{iz} + e^{-iz} );\quad \sin z = \frac{1}{2i}\qty( e^{iz} - e^{-iz} ) \]
Many common trigonometric identities follow from this, such as the identity \(\cos^2 z + \sin^2 z \equiv 1\). However, we have not deduced the period of the functions. Now, restricted to the real case, \(\sin x, \cos x \in \mathbb R\), and the identity \(\cos^2 z + \sin^2 z \equiv 1\) gives that \(\abs{\sin x} \leq 1\) and \(\abs{\cos x} \leq 1\) for all real \(x\).

\subsection{Definition of \(\pi\)}
\begin{proposition}
	There is a smallest positive number \(\pi\) such that
	\[ \cos \frac{\pi}{2} = 0 \]
	and we have \(\sqrt{2} < \pi < \sqrt{3}\).
\end{proposition}
\begin{proof}
	If \(0 < x < 2\),
	\[ \sin x = \qty(x - \frac{x^3}{3!}) + \qty(\frac{x^5}{5!} - \frac{x^7}{7!}) + \cdots \]
	For this range of values, each parenthesised block is positive, so \(\sin x > 0\). So in this range,
	\[ \dv{x} \cos x < 0 \]
	Hence, \(\cos x\) is a strictly decreasing function on this interval. Now,
	\[ \cos \frac{\sqrt{2}}{2} = \qty(\frac{\sqrt{2}^4}{4!} - \frac{\sqrt{2}^6}{6!}) + \cdots > 0 \]
	since each bracketed block is positive.
	\[ \cos \frac{\sqrt{3}}{2} = 1 - \frac{\sqrt{3}^2}{2!} + \frac{\sqrt{3}^4}{4!} - \qty(\frac{\sqrt{3}^6}{6!} - \frac{\sqrt{3}^8}{8!}) + \cdots < 0 \]
	since all the bracketed terms are positive, and being subtracted from a negative number. By the intermediate value theorem, the existence of such a \(\pi\) follows.
\end{proof}
\begin{corollary}
	We have that \(\sin \frac{\pi}{2} = 1\).
\end{corollary}
\begin{proof}
	We know that \(\cos^2 \frac{\pi}{2} + \sin^2 \frac{\pi}{2} = 1\), and \(\sin \frac{\pi}{2} > 0\), so the result follows.
\end{proof}
\begin{theorem}
	The following standard properties about the periodicity of trigonometric functions hold.
	\begin{enumerate}[(i)]
		\item \(\sin(z + \frac{\pi}{2}) = \cos z\), and \(\cos(z + \frac{\pi}{2}) = -\sin z\)
		\item \(\sin(z + \pi) = -\sin z\), and \(\cos(z + \pi) = -\cos z\)
		\item \(\sin(z + 2 \pi) = \sin z\), and \(\cos(z + 2\pi) = \cos z\)
	\end{enumerate}
\end{theorem}
\noindent The proofs are immediate from the angle addition formulae. This then implies that
\[ e^{iz + 2\pi i} = e^{iz} \]
Hence \(e^{z}\) is periodic with period \(2 \pi i\).
