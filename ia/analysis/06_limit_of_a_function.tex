\subsection{Definition}
Let \(f \colon E \subseteq \mathbb C \to \mathbb C\).
We would like to define what is meant by \(\lim_{z \to a} f(z)\), even when \(a \notin E\).
Further, if we have a set with an isolated point, for example \(E = \{ 0 \} \cup [1, 2]\), it does not make sense to talk about limits tending to 0 since there are no points in \(E\) close to 0.
\begin{definition}
	Let \(E \subseteq \mathbb C,\, a \in \mathbb C\).
	\(a\) is a limit point of \(E\) if for any \(\delta > 0\), there exists \(z \in E\) such that \(0 < \abs{z - a} < \delta\).
\end{definition}
\noindent First, note that \(a\) is a limit point if and only if there exists a sequence \(z_n \in E\) such that \(z_n \to a\), but notably \(z_n \neq a\) for all \(n\).
\begin{definition}
	Let \(f \colon E \subseteq \mathbb C \to \mathbb C\), and let \(a \in \mathbb C\) be a limit point of \(E\).
	We say that \(f \to \ell\) as \(z \to a\), if given \(\varepsilon > 0\) there exists \(\delta > 0\) such that whenever \(0 < \abs{z - a} < \delta\) and \(z \in E\), \(\abs{f(z) - \ell} < \varepsilon\).
	Equivalently, \(f(z_n) \to \ell\) for every sequence \(z_n \in E\), such that \(z_n \to a\) but \(z_n \neq a\).
\end{definition}
\noindent Therefore if \(a \in E\) is a limit point, then \(\lim_{z \to a} f(z) = f(a)\) if and only if \(f\) is continuous at \(a\).
If \(a \in E\) is isolated (not a limit point) then \(f\) at \(a\) is trivially continuous, since there are no points near \(a\) but \(a\) itself.

\subsection{Properties}
The limit of a function has very similar properties when compared to the limit of a sequence.
\begin{enumerate}
	\item It is unique.
	      \(f(z) \to A\), \(f(z) \to B\) implies \(A = B\).
	\item \(f(z) \to A\), \(g(z) \to B\) implies
	      \begin{enumerate}
		      \item \(f(z) + g(z) \to A + B\)
		      \item \(f(z)\cdot g(z) \to AB\)
		      \item If \(B \neq 0\), \(\frac{f(z)}{g(z)} \to \frac{A}{B}\)
	      \end{enumerate}
\end{enumerate}

\subsection{Intermediate value theorem}
\begin{theorem}
	Let \(f \colon [a, b] \to \mathbb R\) be a continuous function where \(f(a) \neq f(b)\).
	Then \(f\) takes all values in the interval \([f(a), f(b)]\).
\end{theorem}
\begin{proof}
	Without loss of generality, let us assume \(f(a) < f(b)\).
	Let us take an \(\eta\) such that \(f(a) < \eta < f(b)\).
	We want to prove that there exists some value \(c \in [a, b]\) with \(f(c) = \eta\).
	Let \(s\) be the set of points defined by
	\[
		s = \{ x \in [a, b] \colon f(x) < \eta \}
	\]
	\(a \in s\) therefore the set \(s\) is non-empty.
	The set is also clearly bounded above by \(b\).
	So there is a supremum of this set, say \(\sup s = c\) where \(c \leq b\).
	This point \(c\) can be visualised as the last point at which \(y=f(x)\) crosses the line \(y=c\).
	We intend to show that the function at this rightmost point is \(\eta\).

	By the definition of the supremum, given \(n\) there exists \(x_n \in s\) such that \(c - \frac{1}{n} < x_n \leq c\).
	So the sequence \(x_n\) tends to \(c\).
	We know that \(f(x_n) < \eta\) for all \(x_n\) by definition of the set \(s\).
	By the continuity of \(f\), \(f(x_n) \to f(c)\).
	Thus,
	\begin{equation}
		f(c) \leq \eta \tag{\(\ast\)}
	\end{equation}
	Now, let us consider the fact that \(c \neq b\).
	If \(c = b\), then \(f(b) \leq \eta\) which is a contradiction since \(\eta < f(b)\).
	So for a large \(n\), we can ensure that \(c + \frac{1}{n} \in [a,b]\).
	So by continuity of the function, \(f(c + \frac{1}{n}) \to f(c)\).
	But since \(c + \frac{1}{n} > c\), then necessarily \(f(c + \frac{1}{n}) \geq \eta\) because \(c\) is the supremum of \(s\).
	Thus
	\[
		f(c) \geq \eta
	\]
	Combining this with \((\ast)\) we get \(f(c) = \eta\).
\end{proof}
\noindent This theorem is very useful for finding zeroes and fixed points.
For example, we can prove the existence of the \(N\)th root of a positive real number \(y\).
Let
\[
	f(x) = x^N
\]
Then \(f\) is certainly continuous on the interval \([0, 1+y]\), since
\[
	0 = f(0) < y < (1+y)^N = f(1 + y)
\]
By the intermediate value theorem, there exists a point \(c \in (0, 1+y)\) such that \(f(c) = c^N = y\).
So \(c\) is a positive \(N\)th root of \(y\).
We can also prove the uniqueness of such a point.
Suppose \(d^N = y\) with \(d>0\) and \(d \neq c\).
Without loss of generality, suppose \(d < c\).
Then \(d^N < c^N\) so \(d^N \neq y\), which is a contradiction.

\subsection{Bounds of a continuous function}
\begin{theorem}
	Let \(f \colon [a, b] \to \mathbb R\) be continuous.
	Then the function is bounded, i.e.\ there exists \(k \in \mathbb R\) such that \(\abs{f(x)} \leq k\) for every point \(x \in [a, b]\).
\end{theorem}
\begin{proof}
	Suppose that such a function \(f\) is not bounded.
	Then in particular, given any integer \(n \geq 1\), there exists \(x_n \in [a, b]\) such that \(\abs{f(x_n)} > n\).
	By the Bolzano-Weierstrass theorem, the sequence \(x_n\), which is bounded by \(a \leq x_n \leq b\), has a convergent subsequence \(x_{n_j} \to x\), such that \(x \in [a, b]\).
	Then by continuity of \(f\), \(f(x_{n_j}) \to f(x)\).
	But \(\abs{f(x_{n_j})} > n_j \to \infty\).
	This is a contradiction.
\end{proof}
\noindent We can actually improve this statement.
\begin{theorem}
	Suppose \(f \colon [a, b] \to \mathbb R\) is a continuous function.
	Then there exist \(x_1, x_2 \in [a, b]\) such that
	\[
		f(x_1) \leq f(x) \leq f(x_2)
	\]
	for all \(x \in [a, b]\).
	In other words, a continuous function on a closed bounded interval is bounded and attains its bounds.
\end{theorem}
\begin{proof}
	Let \(A = \{ f(x) \colon x \in [a, b] \}\) be the image of \([a, b]\) under \(f\).
	By the above theorem, \(A\) is bounded.
	It is also non-empty, hence it has a supremum \(M = \sup A\) (and analogously an infimum \(\inf A\), whose proof is almost identical).
	Then by the definition of the supremum, given an integer \(n \geq 1\) there exists \(x_n \in [a, b]\) such that \(M - \frac{1}{n} < f(x_n) \leq M\).
	By the Bolzano-Weierstrass theorem, there exists a convergent subsequence \(x_{n_j} \to x \in [a, b]\).
	Since \(f(x_{n_j}) \to M\), then by continuity, \(f(x) = M\).
\end{proof}
\noindent Here is an alternative proof of the same theorem.
\begin{proof}
	As before, let \(A\) be the image of \(f\), and \(M\) be the supremum of \(A\).
	Suppose there is no \(x_2 \in [a, b]\) such that \(f(x_2) = M\).
	Then let \(g(x) = \frac{1}{M - f(x)}\) for \(x \in [a, b]\).
	Since there exists no \(x\) such that \(M = f(x)\), \(g(x)\) is continuous since we are never dividing by zero.
	So \(g\) is bounded.
	So by the previous theorem, there is some \(k > 0\) such that \(g(x) \leq k\) for all \(x \in [a, b]\).
	This means that \(f(x) \leq M - \frac{1}{k}\) on \([a, b]\) for this \(k\), but this cannot happen since \(M\) is the supremum.
\end{proof}
\noindent Note that these theorems are certainly false if the interval is not closed: consider the counterexample \((0, 1]\) and the function \(x \mapsto x^{-1}\).


\subsection{Inverse functions}
\begin{definition}
	\(f\) is increasing for \(x \in [a, b]\) if \(f(x_1) \leq f(x_2)\) for all \(x_1 \leq x_2 \in [a, b]\).
	If \(f(x_1) < f(x_2)\) then the function is strictly increasing.
	A function may be called decreasing or strictly decreasing analogously.
\end{definition}
\begin{definition}
	A function \(f\) is called monotone if it is either increasing or decreasing.
\end{definition}
\begin{theorem}
	Let \(f \colon [a, b] \to \mathbb R\) be continuous and strictly increasing for \(x \in [a, b]\).
	Let \(c = f(a)\), \(d = f(b)\).
	Then \(f \colon [a, b] \to [c, d]\) is bijective, and the inverse \(g := f^{-1} \colon [c, d] \to [a, b]\) is continuous and strictly increasing.
\end{theorem}
\noindent A similar theorem holds for strictly decreasing functions.
\begin{proof}
	Let \(c < k < d\).
	From the intermediate value theorem, there exists \(h\) such that \(f(h) = k\).
	This \(h\) must be unique since the function is strictly increasing.
	Then we can define \(g(k) = h\), giving us an inverse \(g \colon [c, d] \to [a, b]\) for \(f\).

	First, note that \(g\) is strictly increasing.
	Indeed, for \(y_1 < y_2\) then \(y_1 = f(x_1)\), \(y_2 = f(x_2)\).
	This means that if \(x_2 \geq x_1\), then since \(f\) is increasing \(y_2 \leq y_1\) which is a contradiction.

	Now, note that \(g\) is continuous.
	Indeed, given \(\varepsilon > 0\), we can let \(k_1 = f(h - \varepsilon)\) and \(k_2 = f(h + \varepsilon)\).
	If \(f\) is strictly increasing, then \(k_1 < k < k_2\).
	Then \(h - \varepsilon < g(y) < h + \varepsilon\).
	So let \(\delta = \min(k_2 - k, k - k_1)\) where \(k \in (c, d)\), establishing continuity as claimed.
\end{proof}
