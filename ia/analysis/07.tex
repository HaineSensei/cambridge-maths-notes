\subsection{Definition}
Let \(f \colon E \subseteq \mathbb C \to \mathbb C\). We would like to define what is meant by \(\lim_{z \to a} f(z)\), even when \(a \notin E\). Further, if we have a set with an isolated point, for example \(E = \{ 0 \} \cup [1, 2]\), it does not make sense to talk about limits tending to 0 since there are no points in \(E\) close to 0.
\begin{definition}
	Let \(E \subseteq \mathbb C,\, a \in \mathbb C\). \(a\) is a limit point of \(E\) if for any \(\delta > 0\), there exists \(z \in E\) such that \(0 < \abs{z - a} < \delta\).
\end{definition}
\noindent First, note that \(a\) is a limit point if and only if there exists a sequence \(z_n \in E\) such that \(z_n \to a\), but notably \(z_n \neq a\) for all \(n\).
\begin{definition}
	Let \(f \colon E \subseteq \mathbb C \to \mathbb C\), and let \(a \in \mathbb C\) be a limit point of \(E\). We say that \(f \to \ell\) as \(z \to a\), if given \(\varepsilon > 0\) there exists \(\delta > 0\) such that whenever \(0 < \abs{z - a} < \delta\) and \(z \in E\), \(\abs{f(z) - \ell} < \varepsilon\). Equivalently, \(f(z_n) \to \ell\) for every sequence \(z_n \in E\), such that \(z_n \to a\) but \(z_n \neq a\).
\end{definition}
\noindent Therefore if \(a \in E\) is a limit point, then \(\lim_{z \to a} f(z) = f(a)\) if and only if \(f\) is continuous at \(a\). If \(a \in E\) is isolated (not a limit point) then \(f\) at \(a\) is trivially continuous, since there are no points near \(a\) but \(a\) itself.

\subsection{Properties}
The limit of a function has very similar properties when compared to the limit of a sequence.
\begin{enumerate}
	\item It is unique. \(f(z) \to A\), \(f(z) \to B\) implies \(A = B\).
	\item \(f(z) \to A\), \(g(z) \to B\) implies
	      \begin{enumerate}
		      \item \(f(z) + g(z) \to A + B\)
		      \item \(f(z)\cdot g(z) \to AB\)
		      \item If \(B \neq 0\), \(\frac{f(z)}{g(z)} \to \frac{A}{B}\)
	      \end{enumerate}
\end{enumerate}

\subsection{Intermediate Value Theorem}
\begin{theorem}
	Let \(f \colon [a, b] \to \mathbb R\) be a continuous function where \(f(a) \neq f(b)\). Then \(f\) takes all values in the interval \([f(a), f(b)]\).
\end{theorem}
\begin{proof}
	Without loss of generality, let us assume \(f(a) < f(b)\). Let us take an \(\eta\) such that \(f(a) < \eta < f(b)\). We want to prove that there exists some value \(c \in [a, b]\) with \(f(c) = \eta\). Let \(s\) be the set of points defined by
	\[ s = \{ x \in [a, b] \colon f(x) < \eta \} \]
	\(a \in s\) therefore the set \(s\) is non-empty. The set is also clearly bounded above by \(b\). So there is a supremum of this set, say \(\sup s = c\) where \(c \leq b\). This point \(c\) can be visualised as the last point at which \(y=f(x)\) crosses the line \(y=c\). We intend to show that the function at this rightmost point is \(\eta\).

	By the definition of the supremum, given \(n\) there exists \(x_n \in s\) such that \(c - \frac{1}{n} < x_n \leq c\). So the sequence \(x_n\) tends to \(c\). We know that \(f(x_n) < \eta\) for all \(x_n\) by definition of the set \(s\). By the continuity of \(f\), \(f(x_n) \to f(c)\). Thus,
	\begin{equation}
		f(c) \leq \eta \tag{\(\ast\)}
	\end{equation}
	Now, let us consider the fact that \(c \neq b\). If \(c = b\), then \(f(b) \leq \eta\) which is a contradiction since \(\eta < f(b)\). So for a large \(n\), we can ensure that \(c + \frac{1}{n} \in [a,b]\). So by continuity of the function, \(f(c + \frac{1}{n}) \to f(c)\). But since \(c + \frac{1}{n} > c\), then necessarily \(f(c + \frac{1}{n}) \geq \eta\) because \(c\) is the supremum of \(s\). Thus
	\[ f(c) \geq \eta \]
	Combining this with \((\ast)\) we get \(f(c) = \eta\).
\end{proof}
\noindent This theorem is very useful for finding zeroes and fixed points. For example, we can prove the existence of the \(N\)th root of a positive real number \(y\). Let
\[ f(x) = x^N \]
Then \(f\) is certainly continuous on the interval \([0, 1+y]\), since
\[ 0 = f(0) < y < (1+y)^N = f(1 + y) \]
By the intermediate value theorem, there exists a point \(c \in (0, 1+y)\) such that \(f(c) = c^N = y\). So \(c\) is a positive \(N\)th root of \(y\). We can also prove the uniqueness of such a point. Suppose \(d^N = y\) with \(d>0\) and \(d \neq c\). Without loss of generality, suppose \(d < c\). Then \(d^N < c^N\) so \(d^N \neq y\), which is a contradiction.
