\subsection{Complicated Integrable Functions}
In principle, many functions that are not continuous or monotonic can be integrated using the Riemann integral.
For example, the function \(f\colon [0, 1] \to \mathbb R\) defined by
\[
	f(x) = \begin{cases}
		\frac{1}{q} & x = \frac{p}{q} \in (0, 1] \text{ in its lowest form } \\ % chktex 9
		0           & \text{otherwise}
	\end{cases}
\]
is Riemann integrable.
We know that \(s(f, \mathcal D) = 0\) for all \(\mathcal D\), since any interval will contain irrational numbers.
We will show that given \(\varepsilon > 0\), there exists \(\mathcal D\) such that \(S(f, \mathcal D) < \varepsilon\).
If this is true, then this function \(f\) really is Riemann integrable, with \(\int f = 0\).
We will choose \(N \in \mathbb N\) such that \(\frac{1}{N} < \frac{\varepsilon}{2}\).
Then
\[
	S = \qty{ x \in [0, 1] \colon f(x) \geq \frac{1}{N} } = \qty{ \frac{p}{q} \colon 1 \leq q \leq N, 1 \leq p \leq q }
\]
This set \(S\) is a finite set, hence
\[
	S = \qty{ 0, t_1, \dots, t_R };\quad 0 < t_1 < \dots < t_R = 1
\]
Consider a dissection \(\mathcal D\) such that
\begin{enumerate}[(1)]
	\item Each \(t_k\) is in some interval \([x_{j-1}, x_j]\), and
	\item For all \(k\), the unique interval containing \(t_k\) has length at most \(\frac{\varepsilon}{2R}\).
\end{enumerate}
Such a dissection can certainly be constructed.
Then, in any interval that does not contain a \(t_k\), \(f\) in this interval is less than \(\frac{1}{N}\).
In any interval that does contain a \(t_k\), \(f \geq \frac{1}{N}\) but \(f < 1\) everywhere.
Since there are \(R\) such intervals, each of which with length \(\frac{\varepsilon}{2R}\), we have
\[
	S(f, \mathcal D) \leq \frac{1}{N} + \frac{\varepsilon}{2} < \varepsilon
\]

\subsection{Basic Properties}
Consider functions \(f\) and \(g\) which are bounded and integrable on \([a, b]\).
\begin{enumerate}[(1)]
	\item If \(f \leq g\) on \([a, b]\), then \(\int f \leq \int g\).
	\item \(f + g\) is integrable on \([a, b]\), and \(\int (f + g) = \int f + \int g\).
	\item For any constant \(k\), \(kf\) is integrable, and \(\int kf = k\int f\).
	\item \(\abs{f}\) is integrable, and \(\abs{\int f} \leq \int \abs{f}\).
	\item \(fg\) is integrable.
\end{enumerate}
\begin{proof}
	We will see proofs for some of these properties.
	\begin{enumerate}[(1)]
		\item If \(f \leq g\), then
		      \[
			      \int f = I^\star(f) \leq S(f, \mathcal D) \leq S(g, \mathcal D)
		      \]
		      Hence,
		      \[
			      \int f = I^\star(f) \leq I^\star(g) = \int g
		      \]
		\item We have
		      \[
			      \sup_{[x_{j-1}, x_j]} (f + g) \leq \sup_{[x_{j-1}, x_j]} f + \sup_{[x_{j-1}, x_j]} g
		      \]
		      Therefore,
		      \[
			      S(f + g, \mathcal D) \leq S(f, \mathcal D) + S(g, \mathcal D)
		      \]
		      Now, consider two dissections \(\mathcal D_1, \mathcal D_2\).
		      Now,
		      \[
			      I^\star(f + g) \leq S(f + g, \mathcal D_1 \cup \mathcal D_2) \leq S(f, \mathcal D_1 \cup \mathcal D_2) + S(g, \mathcal D_1 \cup \mathcal D_2) \leq S(f, \mathcal D_1) + S(g, \mathcal D_2)
		      \]
		      We can then fix \(\mathcal D_1\) and take the infimum over \(\mathcal D_2\) to get
		      \[
			      I^\star(f + g) \leq S(f, \mathcal D_1) + I^\star(g)
		      \]
		      Taking the infimum over \(\mathcal D_1\) gives
		      \[
			      I^\star(f + g) \leq I^\star(f) + I^\star(g) = \int f + \int g
		      \]
		      A completely similar argument will show that
		      \[
			      I_\star(f + g) \geq \int f + \int g
		      \]
		      Combining this, \(f+g\) must be integrable, since \(I^\star(f + g) \geq I_\star(f + g)\).
		      This integral is then exactly \(\int f + \int g\).
		      \setcounter{enumi}{3}
		\item Consider first \(f_+(x) = \max(f(x), 0)\).
		      We want to show that \(f_+\) is integrable.
		      We can check that
		      \[
			      \sup_{[x_{j-1}, x_j]}f_+ - \inf_{[x_{j-1}, x_j]}f_+ \leq \sup_{[x_{j-1}, x_j]}f - \sup_{[x_{j-1}, x_j]}f
		      \]
		      We know that given \(\varepsilon > 0\), there exists \(\mathcal D\) such that
		      \[
			      S(f, \mathcal D) - s(f, \mathcal D) < \varepsilon
		      \]
		      Hence,
		      \[
			      S(f_+, \mathcal D) - s(f_+, \mathcal D) \leq S(f, \mathcal D) - s(f, \mathcal D) < \varepsilon
		      \]
		      Therefore \(f_+\) is integrable.
		      But \(\abs{f} = 2f_+ - f\), hence \(\abs{f}\) is integrable by properties (2) and (3).
		      Since \(-\abs{f} \leq f \leq \abs{f}\), we can use monotonicity from (1) to find that
		      \[
			      \abs{\int f} \leq \int \abs{f}
		      \]
		      as claimed.
		\item Let \(f\) be integrable and positive.
		      Then we can check that
		      \[
			      \sup_{[x_{j-1}, x_j]} f^2 = \qty(\underbrace{\sup_{[x_{j-1}, x_j]} f}_{M_j})^2;\quad \inf_{[x_{j-1}, x_j]} f^2 = \qty(\underbrace{\inf_{[x_{j-1}, x_j]}}_{m_j})^2
		      \]
		      Then,
		      \begin{align*}
			      S(f^2, \mathcal D) - s(f^2, \mathcal D) & = \sum_{j=1}^n (x_j - x_{j-1})(M_j^2 - m_j^2)        \\
			                                              & = \sum_{j=1}^n (x_j - x_{j-1})(M_j - m_j)(M_j + m_j)
		      \end{align*}
		      The function \(f\) is bounded by some constant \(k\), therefore the bracket \((M_j + m_j)\) is bounded by \(2k\), which gives
		      \[
			      S(f^2, \mathcal D) - s(f^2, \mathcal D) \leq 2k\qty(S(f, \mathcal D) - s(f, \mathcal D))
		      \]
		      So \(f^2\) is integrable.
		      Now, considering any \(f\), \(\abs{f} \geq 0\) is a non-negative integrable function.
		      Since \(f^2 = \abs{f^2}\), we deduce that \(f^2\) is integrable for any integrable \(f\).
		      Finally, for \(fg\), note that
		      \[
			      4fg = (f + g)^2 - (f - g)^2
		      \]
		      The right hand side is integrable, so the left hand side is integrable.
	\end{enumerate}
\end{proof}

% chktex 17
