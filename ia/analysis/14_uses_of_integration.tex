\subsection{Improper integration}
\begin{definition}
	Suppose \(f \colon [a, \infty) \to \mathbb R\) is integrable (and therefore bounded) on every interval of the form \([a, R]\), and further, as \(R \to \infty\), we have \(\int_a^R f \to \ell\).

	Then we say that the integral \(\int_a^\infty f\) exists (or converges), and its value is \(\ell\).
	If this integral does not tend to a limit, we say that \(\int_a^\infty f\) diverges.
\end{definition}
We can similarly define the integral \(\int_{-\infty}^a f\).
If \(\int_a^\infty f = \ell_1\) and \(\int_{-\infty}^a f = \ell_2\), we can write
\[
	\int_{-\infty}^\infty f = \ell_1 + \ell_2
\]
Note that this last condition is \textit{not} the same as saying that \(\lim_{R \to \infty}\int_{-R}^R f\) exists.
For this two-sided improper integral to exist, we need the stronger condition that the one-sided improper integrals exist on either side.
For example, consider \(f(x) = x\).
Clearly \(\int_{-R}^R f = 0\), but this function is not improper integrable.
For example, consider
\[
	\int_1^\infty \frac{\dd{x}}{x^k}
\]
This converges if and only if \(k > 1\).
Indeed, if \(k \neq 1\),
\[
	\int_1^R \frac{\dd{x}}{x^k} = \eval{\frac{x^{1-k}}{1-k}}_1^R = \frac{R^{1-k} - 1}{1-k}
\]
which is clearly finite in the limit if and only if \(k > 1\).
If \(k=1\), then we can find
\[
	\int_1^R \frac{\dd{x}}{x} = \log R \to \infty
\]
as expected.
Note the following observations.
\begin{enumerate}[(1)]
	\item \(\frac{1}{\sqrt{x}}\) is continuous (and bounded) on \([\delta, 1]\) for all \(\delta > 0\), and
	      \[
		      \int_\delta^1 \frac{\dd{x}}{\sqrt{x}} = \eval{2\sqrt{x}}_\delta^1 = 2 - 2\sqrt{\delta}
	      \]
	      So as \(\delta \to 0\), this integral tends to 2.
	      This integral is defined, even though the value of the function at zero is unbounded.
	      So we commonly write
	      \[
		      \int_0^1 \frac{\dd{x}}{\sqrt{x}} = 2
	      \]
	\item Similarly, we write
	      \[
		      \int_0^1 \frac{\dd{x}}{x} = \lim_{\delta \to 0} \int_\delta^1 \frac{\dd{x}}{x} = \lim_{\delta \to 0} \eval{\log x}_\delta^1 = \lim_{\delta \to 0} (\log 1 - \log \delta)
	      \]
	      Since this limit does not exist, the integral \(\int_0^1 \frac{\dd{x}}{x}\) does not exist.
	\item If \(f \geq 0\) and \(g \geq 0\) for \(x \geq a\), and \(f(x) \leq kg(x)\), where \(k\) is a constant for \(x \geq a\), then
	      \[
		      \int_a^\infty g \text{ converges} \implies \int_a^\infty f \text{ converges, and } \int_a^\infty f \leq \int_a^\infty g
	      \]
	      This is similar to the comparison test for series.
	      First, note that \(\int_a^R f \leq k \int_a^R g\).
	      Further, \(\int_a^R f\) is an increasing function of \(R\) since \(f \geq 0\), and bounded above, since \(\int_a^\infty g\) converges.
	      Let
	      \[
		      \ell = \sup_{R \geq a}\int_a^R f < \infty
	      \]
	      Then we want to show that \(\lim_{R \to \infty}\int_a^R f = \ell\).
	      Given \(\varepsilon > 0\), by the definition of the supremum \(\exists R_0\) such that
	      \[
		      \int_a^{R_0} f \geq \ell - \varepsilon
	      \]
	      Thus for all \(R\geq R_0\) we have
	      \[
		      \int_a^R f \geq \int_a^{R_0} f \geq \ell - \varepsilon
	      \]
	      Hence,
	      \[
		      0 \leq \ell - \int_a^R f \leq \varepsilon
	      \]
	      As an example, consider the integral
	      \[
		      \int_0^\infty \exp(\frac{-x^2}{2}) \dd{x}
	      \]
	      Now, for \(x \geq 1\), we can bound the integrand by \(\exp(-\frac{x}{2})\), and the integral of this bound is clearly bounded.
	      Hence the original integral converges.
	\item If \(\sum a_n\) converges, then \(a_n \to 0\).
	      However, with improper integrals, this is not necessarily the case.
	      Consider a convergent series \(a_n\), where \(0 < a_n < 1\) for all \(n\).
	      Then define the function \(f\) defined by
	      \[
		      f(n + r) = \begin{cases}
			      1 & \text{if } r < a_n \\
			      0 & \text{otherwise}
		      \end{cases}
	      \]
	      where the input \(x\) is split into the integer part \(n\) and the remainder \(r\).
	      This function is essentially a sequence of rectangles of height 1 and width \(a_n\), spaced so that each rectangle starts at an integer value of \(x\).
	      Clearly, we have
	      \[
		      \int_0^n f = \sum_0^n a_n
	      \]
	      where \(n\) is an integer.
	      So the integral converges, but the integrand does not tend to zero.
\end{enumerate}

\subsection{Integral test for series convergence}
\begin{theorem}
	Let \(f(x)\) be a positive decreasing function for \(x \geq 1\).
	Then,
	\begin{enumerate}[(1)]
		\item The integral \(\int_1^\infty f(x) \dd{x}\) and the series \(\sum_1^\infty f(x)\) both converge or diverge.
		      (Note that such a function is always Riemann integrable on a closed interval since it is bounded and decreasing.)
		\item As \(n \to \infty\), \(\sum_{r=1}^n f(r) - \int_1^n f(x) \dd{x}\) tends to a limit \(\ell\) such that \(0 \leq \ell \leq f(1)\).
	\end{enumerate}
\end{theorem}
\begin{proof}
	If \(n-1 \leq x \leq n\), then
	\[
		f(n-1) \geq f(x) \geq f(n)
	\]
	Hence,
	\[
		f(n-1) \geq \int_{n-1}^n f(x) \dd{x} \geq f(n)
	\]
	Adding up such integrals, we get
	\[
		\sum_1^{n-1} f(r) \geq \int_1^n f(x) \dd{x} \geq \sum_2^{n} f(r)
	\]
	Then the first claim is obvious.
	For the second claim, let
	\[
		\phi(n) = \sum_1^n f(r) - \int_1^n f(x) \dd{x}
	\]
	Then, using the inequalities established above,
	\[
		\phi(n) - \phi(n-1) = f(n) - \int_{n-1}^n f(x) \dd{x} \leq 0
	\]
	So \(\phi\) is a decreasing sequence.
	Further,
	\[
		0 \leq \phi(n) \leq f(1)
	\]
	\(\phi\) is bounded, so it converges to some limit \(\ell\).
\end{proof}
\begin{example}
	First, consider the sum \(\sum_1^\infty \frac{1}{n^k}\).
	By the integral test, this converges if and only if \(k > 1\).
	As a more complicated example, consider \(\sum_2^\infty \frac{1}{n\log n}\).
	Let \(f(x) = \frac{1}{x\log x}\), and
	\[
		\int_2^R \frac{\dd{x}}{x\log x} = \eval{\log(\log x)}_2^R
	\]
	which diverges, so by the integral test the series diverges.
\end{example}
\begin{corollary}[Euler-Mascheroni Constant]
	As \(n \to \infty\),
	\[
		\sum_1^n \frac{1}{n} - \int_1^n \frac{1}{n} = 1 + \frac{1}{2} + \dots + \frac{1}{n} - \log n \to \gamma
	\]
	where \(\gamma \in [0, 1]\).
	This is known as the Euler-Mascheroni constant.
	It is unknown whether \(\gamma\) is irrational.
\end{corollary}

\subsection{Piecewise continuous functions}
\begin{definition}
	A function \(f \colon [a, b] \to \mathbb R\) is piecewise continuous if there is a dissection \(\mathcal D\) such that \(f\) is continuous on all intervals defined by this dissection, and that the one-sided limits
	\[
		\lim_{x \to x_{j-1}^+} f(x);\quad \lim_{x \to x_{j-1}^-} f(x)
	\]
	exist.
\end{definition}
We can extend the class of Riemann integrable functions to include piecewise continuous functions as well.
This is true since we use this dissection to construct the upper and lower sums.
The one-sided limits are here to ensure that the function is bounded near these discontinuities.
We might now ask how large the discontinuity set is allowed to be in order for \(f\) to still be Riemann integrable.
As we have seen from examples before, it is possible to have a function which has countably many discontinuity points, but is still Riemann integrable.
