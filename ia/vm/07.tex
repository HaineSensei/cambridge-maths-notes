\subsection{Definitions}
A subspace of a real vector space \(V\) is a subset \(U \subseteq V\) that is a vector space.
Equivalently, if all pairs of vectors \(v, w \in U\) satisfy \(\lambda v + \mu w \in U\), then \(U\) is a subspace of \(V\).
Note that the span generated from a set of vectors is a subspace, as it is characterised by this equivalent definition.
Also, note that the origin must be part of any subspace, because multiplying a vector by zero must yield the origin.

\subsection{Linear Dependence}
In some real vector space \(V\), let \(\vb v_1, \vb v_2 \cdots \vb v_r\) be vectors in \(V\).
Now consider the linear relation
\[
	\lambda_1 \vb v_1 + \lambda_2 \vb v_2 + \cdots + \lambda_r \vb v_r = 0
\]
Then we call the set of vectors a linearly independent set if the only solution is where all \(\lambda\) values are zero.
Otherwise, it is a linearly dependent set.

\subsection{Inner Product Spaces}
An inner product is an extra structure that we can have on a real vector space \(V\), which is often denoted by angle brackets or parentheses.
It can also be characterised by axioms (specifically the ones in Section 6.2).
Features like the norm of a vector, and theorems like the Cauchy-Schwarz inequality, follow from these axioms.

For example, let us consider the vector space
\[
	V = \{ f: [0, 1] \to \mathbb R: f \text{ smooth}; f(0) = f(1) = 0 \}
\]
We can define the inner product to be
\[
	f \cdot g = \langle f, g \rangle = \int_0^1 f(x)g(x)\dd{x}
\]
Then by the Cauchy-Schwarz inequality, we have
\begin{align*}
	\abs{\langle f, g \rangle}               & \leq \norm{f} \cdot \norm{g}                                    \\
	\therefore \abs{\int_0^1 f(x)g(x)\dd{x}} & \leq \sqrt{\int_0^1 f(x)^2 \dd{x}}\sqrt{\int_0^1 g(x)^2 \dd{x}}
\end{align*}

\begin{lemma}
	In any real inner product space \(V\), if \(\vb v_1 \cdots v_r \neq \vb 0\) are orthogonal, they are linearly independent.
\end{lemma}
\begin{proof}
	If \(\sum_i \alpha_i \vb v_i = 0\), then
	\begin{align*}
		\left\langle \vb v_j, \sum_i \alpha_i \vb v_i \right\rangle     & = 0 \\
		\intertext{And because each vector that is not \(\vb v_j\) is orthogonal to it, those terms cancel, leaving}
		\therefore \left\langle \vb v_j, \alpha_j \vb v_j \right\rangle & = 0 \\
		\alpha_j \left\langle \vb v_j, \vb v_j \right\rangle            & = 0 \\
		\alpha_j = 0
	\end{align*}
	So they are linearly independent.
\end{proof}

\subsection{Bases and Dimensions}
In a vector space \(V\), a basis is a set \(\mathcal B = \{ \vb e_1 \cdots \vb e_n \}\) such that
\begin{itemize}
	\item \(\mathcal B\) spans \(V\); and
	\item \(\mathcal B\) is linearly independent, which implies that the coefficients on these basis vectors are unique for any vector in \(V\), since it is impossible to write one vector in terms of the others
\end{itemize}

\begin{theorem}
	If \(\{\vb e_1 \cdots \vb e_n \}\) and \(\{ \vb f_1 \cdots \vb f_m \}\) are bases for a real vector space \(V\), then \(n=m\), which we call the dimension of \(V\).
\end{theorem}
\begin{proof}
	This proof is non-examinable (without prompts).
	We can write each basis vector in terms of the others, since they all span the same vector space.
	Thus:
	\[
		\vb f_a = \sum_i A_{ai} \vb e_i;\quad \vb e_i = \sum_a B_{ia} \vb f_a
	\]
	Note that indices \(i,j\) span from 1 to \(n\), while \(a,b\) span from 1 to \(m\).
	We can substitute one expression into the other, forming:
	\begin{align*}
		\vb f_a & = \sum_i A_{ai} \left( \sum_b B_{ib}\vb f_b \right)  \\
		\vb f_a & = \sum_b \left( \sum_i A_{ai} B_{ib} \right) \vb f_b
	\end{align*}
	Note that we have now written \(\vb f_a\) as a linear combination of \(\vb f_b\) for all valid \(b\).
	But since they are linearly independent, the coefficient of \(\vb f_b\) must be zero if \(a \neq b\), and one of \(a = b\).
	Therefore, we have
	\[
		\delta_{ab} = \sum_i A_{ai} B_{ib}
	\]
	We can make a similar statement about \(\vb e_i\):
	\[
		\delta_{ij} = \sum_a B_{ia} A_{aj} = \sum_a A_{aj} B_{ia}
	\]
	Now, assigning \(a=b\) and \(i=j\), summing over both, and substituting into our two previous expressions for \(\delta\), we have:
	\begin{alignat*}{2}
		\sum_{ia} A_{ai} B_{ia} & = \sum_a \delta_{aa} &  & = \sum_i \delta_{ii} \\
		                        & = m                  &  & = n
	\end{alignat*}
\end{proof}
