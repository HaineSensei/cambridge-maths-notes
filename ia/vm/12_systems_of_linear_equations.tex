\subsection{Introduction and nature of solutions}
Consider a system of \(n\) linear equations in \(n\) unknowns \(x_i\) written in matrix-vector form:
\[
	A\vb x = \vb b,\quad \vb x, \vb b \in \mathbb R^n,
\]
where \(A\) is an \(n \times n\) matrix.
There are three possibilities:
\begin{enumerate}[(i)]
	\item \(\det A \neq 0 \implies A^{-1}\) exists so there is a unique solution \(\vb x = A^{-1} \vb b\)
	\item \(\det A = 0\) and \(b \notin \Im A\) means that there is no solution
	\item \(\det A = 0\) and \(b \in \Im A\) means that there are infinitely many solutions of the form
	      \[
		      \vb x = \vb x_0 + \vb u
	      \]
	      where \(\vb u \in \ker A\) and \(\vb x_0\) is a particular solution
\end{enumerate}
A solution therefore exists if and only if \(A\vb x_0 = \vb b\) for some \(\vb x_0\), which is true if and only if \(\vb b \in \Im A\).
Then \(\vb x\) is also a solution if and only if \(\vb u = \vb x - \vb x_0\) satisfies
\[
	A\vb u = \vb 0
\]
This equation is known as the equivalent homogeneous problem.
Now, \(\det A \neq 0 \iff \Im A = \mathbb R^n \iff \ker A = \{ \vb 0 \}\).
So in case (i), there is always a unique solution for any \(\vb b\).
But \(\det A = 0 \iff \rank(A) < n \iff \nullity A > 0\).
Then either \(b \notin \Im A\) as in case (ii), or \(b \in \Im A\) as in case (iii).

If \(\vb u_1, \dots, \vb u_k\) is a basis for \(\ker A\), then the general solution to the homogeneous problem is some linear combination of these basis vectors, i.e.
\[
	\vb u = \sum_{i=1}^k \lambda_i \vb u_i,\quad k = \nullity A
\]
This is similar to the complementary function and particular integral technique used to solve linear differential equations.

For example, in \(A\vb x = \vb b\), let
\[
	A = \begin{pmatrix}
		1 & 1 & a \\ a & 1 & 1 \\ 1 & a & 1
	\end{pmatrix};\quad \vb b = \begin{pmatrix}
		1 \\ c \\ 1
	\end{pmatrix};\quad a, c \in \mathbb R
\]
We have previously found that \(\det A = (a-1)^2(a+2)\).
So the cases are:
\begin{itemize}
	\item (\(a \neq 1, a \neq -2\)) \(\det A \neq 0\) and \(A^{-1}\) exists; we previously found this to be
	      \[
		      A^{-1} = \frac{1}{(1-a)(2+a)}\begin{pmatrix}
			      1 & 1+a & 1 \\ 1 & 1 & -1-a \\ -1-a & 1 & 1
		      \end{pmatrix}
	      \]
	      For these values of \(a\), there is a unique solution for any \(c\), demonstrating case (i) above:
	      \[
		      \vb x = A^{-1} \vb b = \frac{1}{(1-a)(2+a)}\begin{pmatrix}
			      2-c-ca \\ c-a \\ c-a
		      \end{pmatrix}
	      \]
	      Geometrically, this solution is simply a point.
	\item (\(a = 1\)) In this case, the matrix is simply
	      \[
		      A = \begin{pmatrix}
			      1 & 1 & 1 \\ 1 & 1 & 1 \\ 1 & 1 & 1
		      \end{pmatrix} \implies \Im A = \vecspan \left\{ \begin{pmatrix}
			      1 \\ 1 \\ 1
		      \end{pmatrix} \right\} = \left\{ \lambda\begin{pmatrix}
			      1 \\ 1 \\ 1
		      \end{pmatrix} \right\};\quad \ker A = \vecspan\left\{ \begin{pmatrix}
			      -1 \\ 1 \\ 0
		      \end{pmatrix}, \begin{pmatrix}
			      -1 \\ 0 \\ 1
		      \end{pmatrix} \right\}
	      \]
	      Note that \(\vb b \in \Im A\) if and only if \(c=1\), where a particular solution is
	      \[
		      \vb x_0 = \begin{pmatrix}
			      1 \\ 0 \\ 0
		      \end{pmatrix}
	      \]
	      So the general solution is given by
	      \[
		      \vb x = \vb x_0 + \vb u = \begin{pmatrix}
			      1 - \lambda - \mu \\ \lambda \\ \mu
		      \end{pmatrix}
	      \]
	      In summary, for \(a=1\), \(c=1\) we have case (iii).
	      Geometrically this is a plane.
	      For \(a=1\), \(c \neq 1\), we have case (ii) where there are no solutions.

	\item (\(a=-2\)) The matrix becomes
	      \[
		      A = \begin{pmatrix}
			      1 & 1 & -2 \\ -2 & 1 & 1 \\ 1 & -2 & 1
		      \end{pmatrix} \implies \Im A = \vecspan \left\{ \begin{pmatrix}
			      1 \\ -2 \\ 1
		      \end{pmatrix}, \begin{pmatrix}
			      1 \\ 1 \\ -2
		      \end{pmatrix} \right\};\quad \ker A = \left\{ \lambda\begin{pmatrix}
			      1 \\ 1 \\ 1
		      \end{pmatrix} \right\}
	      \]
	      Now, \(\vb b \in \Im A\) if and only if \(c = -2\), the particular solution is
	      \[
		      \vb x_0 = \begin{pmatrix}
			      1 \\ 0 \\ 0
		      \end{pmatrix}
	      \]
	      The general solution is therefore
	      \[
		      \vb x = \vb x_0 + \vb u = \begin{pmatrix}
			      1 + \lambda \\ \lambda \\ \lambda
		      \end{pmatrix}
	      \]
	      In summary, for \(a=-2\) and \(c=-2\) we have case (iii).
	      Geometrically this is a line.
	      For \(a=-2\), \(c \neq -2\), we have case (ii) where there are no solutions.
\end{itemize}

\subsection{Geometrical interpretation in \(\mathbb R^3\)}
Let \(\vb R_1, \vb R_2, \vb R_3\) be the rows of the \(3 \times 3\) matrix \(A\).
Then the rows represent the normals of planes.
This is clear by expanding the matrix multiplication of the homogeneous form:
\begin{align*}
	A\vb u = \vb 0 \iff & \vb R_1 \cdot \vb u = 0 \\
	                    & \vb R_2 \cdot \vb u = 0 \\
	                    & \vb R_3 \cdot \vb u = 0
\end{align*}
So the solution of the homogeneous problem (i.e.\ finding the general solution) amounts to determining where the planes intersect.
\begin{itemize}
	\item (\(\rank A = 3\)) The rows are linearly independent, so the three planes' normals are linearly independent and the planes intersect at \(\vb 0\) only.
	\item (\(\rank A = 2\)) The normals span a plane, so the planes intersect in a line.
	\item (\(\rank A = 1\)) The normals are parallel and therefore the planes coincide.
	\item (\(\rank A = 0\)) The normals are all zero, so any vector in \(\mathbb R^3\) solves the equation.
\end{itemize}
Now, let us consider instead the original problem \(A \vb x = \vb b\):
\begin{align*}
	A\vb b = \vb 0 \iff & \vb R_1 \cdot \vb u = b_1 \\
	                    & \vb R_2 \cdot \vb u = b_2 \\
	                    & \vb R_3 \cdot \vb u = b_3
\end{align*}
The planes still have normals \(\vb R_i\) as before, but they do not necessarily pass through the origin.
\begin{itemize}
	\item (\(\rank A = 3\)) The planes' normals are linearly independent and the planes intersect at a point; this is the unique solution.
	\item (\(\rank A < 3\)) The existence of a solution depends on the value of \(\vb b\).
	      \begin{itemize}
		      \item (\(\rank A = 2\)) The planes may intersect in a line as before, but they may instead form a sheaf (the planes pairwise intersect in lines but they do not as a triple), or two planes could be parallel and not intersect each other at all.
		      \item (\(\rank A = 1\)) The normals are parallel, so the planes may coincide or they might be parallel.
		            There is no solution unless all three planes coincide.
	      \end{itemize}
\end{itemize}
