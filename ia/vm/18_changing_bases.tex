\subsection{Change of basis formula}
Recall that given a linear map \(T\colon V \to W\) where \(V\) and \(W\) are real or complex vector spaces, and choices of bases \(\{ \vb e_i \}\) for \(i = 1, \dots, n\) and \(\{\vb f_a \}\) for \(a = 1, \dots, m\), then the \(m \times n\) matrix \(A\) with respect to these bases is defined by
\[
	T(\vb e_i) = \sum_a \vb f_a A_{ai}
\]
So the entries in column \(i\) of \(A\) are the components of \(T(\vb e_i)\) with respect to the basis \(\{ \vb f_a \}\).
This is chosen to ensure that the statement \(\vb y = T(\vb x)\) is equivalent to the statement that \(y_a = A_{ai}x_i\), where \(\vb y = \sum_a y_a \vb f_a\) and \(\vb x = \sum_i x_i \vb e_i\).
This equivalence holds since
\[
	T\left(\sum_i x_i \vb e_i \right) = \sum_i x_i T(\vb e_i) = \sum_i x_i \left( \sum_a \vb f_a A_{ai} \right) = \sum_a \underbrace{\left( \sum_i A_{ai} x_i \right)}_{y_a} \vb f_a
\]
as required.
For the same linear map \(T\), there is a different matrix representation \(A'\) with respect to different bases \(\{ \vb e'_i \}\) and \(\{ \vb f'_a \}\).
To relate \(A\) with \(A'\), we need to understand how the new bases relate to the original bases.
The change of base matrices \(P\) (\(n \times n\)) and \(Q\) (\(m \times m\)) are defined by
\[
	\vb e_i' = \sum_j \vb e_j P_{ji};\quad \vb f_a' = \sum_b \vb f_b Q_{ba}
\]
The entries in column \(i\) of \(P\) are the components of the new basis \(\vb e_i'\) in terms of the old basis vectors \(\{ \vb e_j \}\), and similarly for \(Q\).
Note, \(P\) and \(Q\) are invertible, and in the relation above we could exchange the roles of \(\{ \vb e_i \}\) and \(\{ \vb e'_i \}\) by replacing \(P\) with \(P^{-1}\), and similarly for \(Q\).

\begin{proposition}[Change of base formula for a linear map]
	With the definitions above,
	\[
		A' = Q^{-1}AP
	\]
\end{proposition}
\noindent First we will consider an example, then we will construct a proof.
Let \(n=2, m=3\), and
\begin{align*}
	T(\vb e_1) & = \vb f_1 + 2\vb f_2 - \vb f_3 = \sum_a \vb f_a A_{a1}  \\
	T(\vb e_2) & = -\vb f_1 + 2\vb f_2 + \vb f_3 = \sum_a \vb f_a A_{a2}
\end{align*}
Therefore,
\[
	A = \begin{pmatrix}
		1 & -1 \\ 2 & 2 \\ -1 & 1
	\end{pmatrix}
\]
Consider a new basis for \(V\), given by
\begin{align*}
	\vb e'_1 & = \vb e_1 - \vb e_2 = \sum_i \vb e_i P_{i1} \\
	\vb e'_2 & = \vb e_1 + \vb e_2 = \sum_i \vb e_i P_{i2}
\end{align*}
\[
	P = \begin{pmatrix}
		1 & 1 \\ -1 & 1
	\end{pmatrix}
\]
Consider further a new basis for \(W\), given by
\begin{align*}
	\vb f'_1 & = \vb f_1 - \vb f_3 = \sum_a \vb f_a Q_{a1} \\
	\vb f'_2 & = \vb f_2 = \sum_a \vb f_a Q_{a2}           \\
	\vb f'_3 & = \vb f_1 + \vb f_3 = \sum_a \vb f_a Q_{a3}
\end{align*}
\[
	Q = \begin{pmatrix}
		1  & 0 & 1 \\
		0  & 1 & 0 \\
		-1 & 0 & 1
	\end{pmatrix}
\]
From the change of base formula,
\begin{align*}
	A' & = Q^{-1}AP                                                                       \\
	   & = \begin{pmatrix}
		1/2 & 0 & -1/2 \\
		0   & 1 & 0    \\
		1/2 & 0 & 1/2
	\end{pmatrix}\begin{pmatrix}
		1 & -1 \\ 2 & 2 \\ -1 & 1
	\end{pmatrix}\begin{pmatrix}
		1 & 1 \\ -1 & 1
	\end{pmatrix} \\
	   & = \begin{pmatrix}
		2 & 0 \\ 0 & 4 \\ 0 & 0
	\end{pmatrix}
\end{align*}
Now checking this result directly,
\begin{align*}
	T(\vb e'_1) & = 2\vb f_1 - 2\vb f_3 = 2\vb f_1' \\
	T(\vb e'_2) & = 4\vb f_2 = 4\vb f_2'
\end{align*}
which matches the content of the matrix as required.
Now, let us prove the proposition in general.
\begin{proof}
	\begin{align*}
		T(\vb e'_i) & = T\left( \sum_j \vb e_j P_{ji} \right)              \\
		            & = \sum_j T(\vb e_j) P_{ji}                           \\
		            & = \sum_j \left( \sum_a \vb f_a A_{aj} \right) P_{ji} \\
		            & = \sum_{ja} \vb f_a A_{aj} P_{ji}
	\end{align*}
	But on the other hand,
	\begin{align*}
		T(\vb e'_i) & = \sum_b \vb f'_b A'_{bi}                             \\
		            & = \sum_b \left( \sum_a \vb f_a Q_{ab} \right) A'_{bi} \\
		            & = \sum_{ab} \vb f_a Q_{ab} A'_{bi}
	\end{align*}
	which is a sum over the same set of basis vectors, so we may equate coefficients of \(\vb f_a\).
	\begin{align*}
		\sum_j A_{aj} P_{ji} & = \sum_b Q_{ab} A'_{bi} \\
		(AP)_{ai}            & = (QA')_{ai}
	\end{align*}
	Therefore
	\[
		AP = QA' \implies A' = Q^{-1}AP
	\]
	as required.
\end{proof}

\subsection{Changing bases of vector components}
Here is another way to arrive at the formula \(A' = Q^{-1}AP\).
Consider changes in vector components
\begin{align*}
	\vb x        & = \sum_i x_i \vb e_i = \sum_j x_j' \vb e'_j       \\
	             & = \sum_i\left( \sum_j P_{ij} x'_j \right) \vb e_i \\
	\implies x_i & = P_{ij} x'_j
\end{align*}
We will write
\[
	X = \begin{pmatrix}
		x_1 \\ \vdots \\ x_n
	\end{pmatrix};\quad X' = \begin{pmatrix}
		x'_1 \\ \vdots \\ x'_n
	\end{pmatrix}
\]
Then \(X = PX'\) or \(X' = P^{-1}X\).
Similarly,
\begin{align*}
	\vb y        & = \sum_a y_a \vb f_a = \sum_b y_b' \vb f'_b \\
	\implies y_a & = Q_{ab} y'_b
\end{align*}
Then \(Y = QY'\) or \(Y' = Q^{-1}Y\).
So the matrices are defined to ensure that
\[
	Y = AX;\quad Y' = A'X'
\]
Therefore,
\[
	QY' = APX' \implies Y' = (Q^{-1}AP)X' \implies A' = Q^{-1}AP
\]

\subsection{Specialisations of changes of basis}
Now, let us consider some special cases (in increasing order of specialisation).
\begin{enumerate}[(i)]
	\item Let \(V=W\) with \(\vb e_i = \vb f_i\) and \(\vb e'_i = \vb f'_i\).
	      So \(P=Q\) and the change of basis is
	      \[
		      A' = P^{-1}AP
	      \]
	      Matrices representing the same linear map but with respect to different bases are similar.
	      Conversely, if \(A, A'\) are similar, then we can construct an invertible change of basis matrix \(P\) which relates them, so they can be regarded as representing the same linear map.
	      In an earlier section we noted that \(\tr(A') = \tr(A)\), \(\det(A') = \det(A)\) and \(\chi_A(t) = \chi_{A'}(t)\).
	      so these are intrinsic properties of the linear map, not just the particular matrix we choose to represent it.
	\item Let \(V=W=\mathbb R^n\) or \(\mathbb C^n\) where \(\vb e_i\) is the standard basis, with respect to which, \(T\) has matrix \(A\).
	      If there exists a basis of eigenvectors, \(\vb e'_i = \vb v_i\) with \(A\vb v_i = \lambda_i\vb v_i\).
	      Then
	      \[
		      A' = P^{-1}AP = D = \begin{pmatrix}
			      \lambda_1 & 0         & \cdots & 0         \\
			      0         & \lambda_2 & \cdots & 0         \\
			      \vdots    & \vdots    & \ddots & \vdots    \\
			      0         & 0         & \cdots & \lambda_n
		      \end{pmatrix}
	      \]
	      and
	      \[
		      \vb v_i = \sum_k \vb e_j P_{ji}
	      \]
	      so the eigenvectors are the columns of \(P\).
	\item Let \(A\) be hermitian, i.e.\ \(A^\dagger = A\), then we always have a basis of orthonormal eigenvectors \(\vb e'_i = \vb u_i\).
	      Then the relations in (ii) apply, and \(P\) is unitary, \(P^\dagger = P^{-1}\).
\end{enumerate}

\subsection{Jordan normal form}
Also known as the (Jordan) Canonical Form, this result classifies \(n\times n\) complex matrices up to similarity.
\begin{proposition}
	Any \(2\times 2\) complex matrix \(A\) is similar to one of the following:
	\begin{enumerate}[(i)]
		\item \(A' = \begin{pmatrix}
			      \lambda_1 & 0         \\
			      0         & \lambda_2
		      \end{pmatrix}\) with \(\lambda_1 \neq \lambda_2\), so \(\chi_A(t) = (t - \lambda_1)(t - \lambda_2)\).
		\item \(A' = \begin{pmatrix}
			      \lambda & 0       \\
			      0       & \lambda
		      \end{pmatrix}\), so \(\chi_A(t) = (t - \lambda)^2\).
		\item \(A' = \begin{pmatrix}
			      \lambda & 1       \\
			      0       & \lambda
		      \end{pmatrix}\), so \(\chi_A(t) = (t - \lambda)^2\) as in case (ii).
	\end{enumerate}
\end{proposition}
\begin{proof}
	\(\chi_A(t)\) has two roots over \(\mathbb C\).
	\begin{enumerate}[(i)]
		\item For distinct roots \(\lambda_1, \lambda_2\), we have \(M_{\lambda_1} = m_{\lambda_1} = M_{\lambda_2} = m_{\lambda_2} = 1\).
		      So the eigenvectors \(\vb v_1, \vb v_2\) provide a basis.
		      Hence \(A' = P^{-1}AP\) with the eigenvectors as the columns of \(P\).
		\item For a repeated root \(\lambda\) with \(M_\lambda = m_\lambda = 2\), the same argument applies.
		\item For a repeated root \(\lambda\) with \(M_\lambda = 2\), \(m_\lambda = 1\), we do not have a basis of eigenvectors so we cannot diagonalise the matrix.
		      We only have one linearly independent eigenvector, which we will call \(\vb v\).
		      Let \(\vb w\) be any other vector such that \(\{ \vb v, \vb w \}\) are linearly indepdendent.
		      Then
		      \begin{align*}
			      A\vb v & = \lambda \vb v              \\
			      A\vb w & = \alpha \vb v + \beta \vb w
		      \end{align*}
		      The matrix representing this linear map with respect to the basis vectors \(\{ \vb v, \vb w \}\) is therefore
		      \[
			      \begin{pmatrix}
				      \lambda & \alpha \\
				      0       & \beta
			      \end{pmatrix}
		      \]
		      Let us solve for some of these unknowns.
		      We know that the characteristic polynomial of this matrix must be \((t - \lambda)^2\), so \(\beta = \lambda\).
		      Also, \(\alpha \neq 0\), otherwise we have case (ii) above.
		      So now we can set \(\vb u = \alpha \vb v\), so
		      \begin{align*}
			      A(\alpha \vb v) & = \lambda (\alpha \vb v)     \\
			      A\vb w          & = \alpha \vb v + \beta \vb w
		      \end{align*}
		      So with respect to the basis \(\{ \vb u, \vb w \}\) we get the matrix \(A\) to be
		      \[
			      A' = \begin{pmatrix}
				      \lambda & 1       \\
				      0       & \lambda
			      \end{pmatrix}
		      \]
	\end{enumerate}
\end{proof}
\begin{proof}[Alternative Proof]
	Here is an alternative appproach for case (iii).
	If \(A\) has characteristic polynomial
	\[
		\chi_A(t) = (t - \lambda)^2
	\]
	but \(A \neq \lambda I\), then there exists some vector \(\vb w\) for which \(\vb u = (A - \lambda I)\vb w \neq \vb 0\).
	So \((A - \lambda I)\vb u = (A - \lambda I)^2 \vb w  = \vb 0\) by the Cayley-Hamilton theorem.
	So
	\begin{align*}
		A\vb u & = \lambda \vb u         \\
		A\vb w & = \vb u + \lambda \vb w
	\end{align*}
	So with basis \(\{ \vb u, \vb w \}\) we have the matrix
	\[
		A' = \begin{pmatrix}
			\lambda & 1       \\
			0       & \lambda
		\end{pmatrix}
	\]
\end{proof}
Here is a concrete example using this alternative proof method.
\[
	A = \begin{pmatrix}
		1 & 4 \\ -1 & 5
	\end{pmatrix} \implies \chi_A(t) = (t - 3)^2
\]
So
\[
	A - 3I = \begin{pmatrix}
		-2 & 4 \\ -1 & 2
	\end{pmatrix}
\]
We will choose \(\vb w = \begin{pmatrix}
	1 \\ 0
\end{pmatrix}\) and we find \(\vb u = (A - 3I)\vb w = \begin{pmatrix}
	-2 \\ -1
\end{pmatrix}\).
\(\vb w\) is not an eigenvector, as required for the construction.
By the reasoning in the alternative argument above, \(\vb u\) is an eigenvector by construction.
\begin{align*}
	A\vb u & = 3\vb u         \\
	A\vb w & = \vb u + 3\vb w
\end{align*}
So
\[
	P = \begin{pmatrix}
		-2 & 1 \\ -1 & 0
	\end{pmatrix} \implies P^{-1} = \begin{pmatrix}
		0 & -1 \\ 1 & -2
	\end{pmatrix}
\]
and we can check that
\[
	P^{-1}AP = \begin{pmatrix}
		3 & 1 \\ 0 & 3
	\end{pmatrix} = A'
\]

\subsection{Jordan normal forms in \(n\) dimensions}
To extend the arguments above to larger matrices, consider the \(n\times n\) matrix
\[
	N = \begin{pmatrix}
		0      & 1      & 0      & \cdots & 0      \\
		0      & 0      & 1      & \cdots & 0      \\
		0      & 0      & 0      & \cdots & 0      \\
		\vdots & \vdots & \vdots & \ddots & \vdots \\
		0      & 0      & 0      & \cdots & 0
	\end{pmatrix}
\]
When applied to the standard basis vectors in \(\mathbb C^n\), the action of this matrix sends \(\vb e_n \mapsto \vb e_{n-1} \mapsto \dots \mapsto \vb e_1 \mapsto \vb 0\).
This is consistent with the property that \(N^n = 0\).
The kernel of this matrix has dimension 1.
Now consider the matrix \(J = \lambda I + N\), as follows:
\[
	N = \begin{pmatrix}
		\lambda & 1       & 0       & \cdots & 0       \\
		0       & \lambda & 1       & \cdots & 0       \\
		0       & 0       & \lambda & \cdots & 0       \\
		\vdots  & \vdots  & \vdots  & \ddots & \vdots  \\
		0       & 0       & 0       & \cdots & \lambda
	\end{pmatrix}
\]
This matrix has
\[
	\chi_J(t) = (\lambda - t)^n
\]
with \(M_\lambda = n\) and \(m_\lambda = 1\), since the kernel of \(J - \lambda I = N\) has dimension 1 as before.
The general result is as follows.
\begin{theorem}
	Any \(n\times n\) complex matrix \(A\) is similar to a matrix of the form
	\[
		A' = \left( \begin{array}{c|c|c|c}
				J_{n_1}(\lambda_1) & 0                  & \cdots & 0                  \\\hline
				0                  & J_{n_2}(\lambda_2) & \cdots & 0                  \\\hline
				\vdots             & \vdots             & \ddots & \vdots             \\\hline
				0                  & 0                  & \cdots & J_{n_r}(\lambda_r)
			\end{array} \right)
	\]
	where each diagonal block is a Jordan block \(J_{n_r}(\lambda_r)\) which is an \(n_r \times n_r\) matrix \(J\) with eigenvalue \(\lambda_r\).
	\(\lambda_1, \dots, \lambda_r\) are eigenvalues of \(A\) and \(A'\), and the same eigenvalue may appear in different blocks.
	Further, \(n_1 + n_2 + \dots + n_r = n\) so we end up with an \(n \times n\) matrix.
	\(A\) is diagonalisable if and only if \(A'\) consists entirely of \(1 \times 1\) blocks.
	The expression above is the Jordan Normal Form.
\end{theorem}
The proof is non-examinable and depends on the Part IB courses Linear Algebra, and Groups, Rings and Modules, so is not included here.
