\subsection{Definitions}
Consider a linear map \(T: \mathbb R^n \to \mathbb R^m\), with standard bases \(\{ \vb e_i \} \in \mathbb R^n\), \(\{ \vb f_a \}, \in \mathbb R^m\), and with \(T(\vb x) = \vb x'\).
Let further

\[ \vb x = \sum_i x_i \vb e_i = \begin{pmatrix}
		x_1 \\ x_2 \\ \vdots \\ x_n
	\end{pmatrix};\quad x' = \sum_a x_a' \vb f_a = \begin{pmatrix}
		x_1' \\ x_2' \\ \vdots \\ x_m'
	\end{pmatrix} \]

Linearity implies that \(T\) is fixed by specifying
\[ T(\vb e_i) = \vb e_i' = \vb C_i \in \mathbb R^m \]
We take these \(\vb C\) as columns of an \(m \times n\) array or matrix \(M\), with rows denoted as \(\vb R_a \in \mathbb R^n\).

\[ \begin{pmatrix}
		\uparrow   &        & \uparrow   \\
		\vb C_1    & \cdots & \vb C_n    \\
		\downarrow &        & \downarrow
	\end{pmatrix} = M = \begin{pmatrix}
		\leftarrow & \vb R_1 & \rightarrow \\
		           & \vdots  &             \\
		\leftarrow & \vb R_m & \rightarrow
	\end{pmatrix} \]

\(M\) has entries \(M_{ai} \in \mathbb R\), where \(a\) labels rows and \(i\) labels columns, so
\[ (\vb C_i)_a = M_{ai} = (\vb R_a)_i \]
The action of \(T\) is then given by the matrix \(M\) multiplying the vector \(\vb x\) in the following way:
\[ \vb x' = M \vb x \]
defined by
\[ x_a' = M_{ai}x_i \]
or explicitly:
\[
	\begin{pmatrix}
		x_1' \\ x_2' \\ \vdots \\ x_m'
	\end{pmatrix}
	=
	\begin{pmatrix}
		M_{11} & M_{12} & \cdots & M_{1n} \\
		M_{21} & M_{22} & \cdots & M_{2n} \\
		\vdots & \vdots & \ddots & \vdots \\
		M_{m1} & M_{m2} & \cdots & M_{mn}
	\end{pmatrix}
	\begin{pmatrix}
		x_1 \\ x_2 \\ \vdots \\ x_n
	\end{pmatrix}
	=
	\begin{pmatrix}
		M_{11} x_1 + M_{12} x_2 + \cdots + M_{1n} x_n \\
		M_{21} x_1 + M_{22} x_2 + \cdots + M_{2n} x_n \\
		\vdots                                        \\
		M_{m1} x_1 + M_{m2} x_2 + \cdots + M_{mn} x_n
	\end{pmatrix}
\]
To check that the matrix multiplication above gives the action of \(T\), we can plug in a generic value \(\vb x\), and we get
\[ \vb x' = T\left(\sum_i x_i \vb e_i\right) = \sum_i x_i T(\vb e_i) = \sum_i x_i \vb C_i \]
and by taking component \(a\) of the vector, we have
\[ x_a' = \sum_i x_i (\vb C_i)_a = \sum_i x_i M_{ai} \]
as required. Note also that
\[ x_a' = M_{ai}x_i = (\vb R_a)_i x_i = \vb R_a \cdot \vb x \]
We can now regard the properties of \(T\) as properties of \(M\) (suitably interpreted). For example:
\begin{itemize}
	\item \(\Im(T) = \Im(M) = \vecspan \{ \vb C_1, \cdots, \vb C_n \}\). In words, the image of a matrix is the span of its columns.
	\item \(\ker(T) = \ker(M) = \{ \vb x: \forall a, \vb R_a \cdot \vb x = 0 \}\). In some sense, the kernel of \(M\) is the subspace perpendicular to all of its rows.
\end{itemize}

\subsection{Examples}
\begin{enumerate}[(i)]
	\item The zero map \(\mathbb R^n \to \mathbb R^m\) corresponds to the zero matrix
	      \[ M = 0 \text{ with } M_{ai} = 0 \]
	\item The identity map \(\mathbb R^n \to \mathbb R^n\) corresponds to the identity (or unit) matrix
	      \[ M = I \text{ with } I_{ij} = \delta_{ij} \]
	\item The map \(\mathbb R^3 \to \mathbb R^3\) given by \(\vb x' = T(\vb x) = M\vb x\) with
	      \[ M = \begin{pmatrix}
			      3  & 1 & 5  \\
			      -1 & 0 & -2 \\
			      2  & 1 & 3
		      \end{pmatrix} \]
	      gives
	      \[
		      \begin{pmatrix}
			      x_1' \\ x_2' \\ x_3'
		      \end{pmatrix}
		      =
		      \begin{pmatrix}
			      3x_1 + x_2 + 5x_3 \\
			      -x_1 - 2x_3       \\
			      2x_1 + x_2 + 3x_3
		      \end{pmatrix}
	      \]
	      In this case, we may read off the column vectors \(\vb C_a\) from the matrix. Note that since they form a linearly dependent set, we have
	      \[ \Im(T) = \Im(M) = \vecspan \{ \vb C_1, \vb C_2, \vb C_3 \} = \vecspan \{ \vb C_1, \vb C_2 \} \]
	      Here, \(\vb R_2 \times \vb R_3 = \begin{pmatrix}
			      2 & -1 & -1
		      \end{pmatrix} = \vb u\) is actually perpendicular to all rows as they form a linearly dependent set. So
	      \[ \ker(T) = \ker(M) = \{ \lambda \vb u \} \]
	\item A rotation through \(\theta\) in \(\mathbb R^2\) is given by (building from the images of the basis vectors):
	      \[ \begin{pmatrix}
			      \cos \theta & -\sin \theta \\
			      \sin \theta & \cos \theta
		      \end{pmatrix} \]
	\item A dilation \(\vb x' = M \vb x\) with scale factors \(\alpha, \beta, \gamma\) along axes in \(\mathbb R^3\) is given by
	      \[ \begin{pmatrix}
			      \alpha & 0     & 0      \\
			      0      & \beta & 0      \\
			      0      & 0     & \gamma
		      \end{pmatrix} \]
	\item A reflection in a plane perpendicular to a unit vector \(\nhat\) is given by a matrix \(H\) that must have the property that
	      \begin{align*}
		      \vb x' & = H \vb x = \vb x - 2(\vb x - \nhat) \nhat \\
		      x_i'   & = x_i - 2x_jn_jn_i = H_{ij}x_j             \\
		      \intertext{And by comparing coefficients of \(x_j\), and using \(\delta\) to rewrite \(x_i\) using the \(j\) index, we have}
		      H_{ij} & = \delta_ij - 2n_in_j
	      \end{align*}
	      For example, with \(\nhat = \frac{1}{\sqrt 3}\begin{pmatrix}
			      1 & 1 & 1
		      \end{pmatrix}\), then \(n_in_j = \frac{1}{3}\) for all \(i, j\), so
	      \[ H = \frac{1}{3}\begin{pmatrix}
			      1  & -2 & -2 \\
			      -2 & 1  & -2 \\
			      -2 & -2 & 1
		      \end{pmatrix} \]
	\item A shear is defined by a matrix \(S\) such that
	      \[ \vb x' = S\vb x = \vb x + \lambda(\vb b \cdot \vb x)\vb a \]
	      where \(\vb a\), \(\vb b\) are unit vectors with \(\vb a \perp \vb b\), and where \(\lambda\) is a real scale factor. Therefore:
	      \begin{align*}
		      x_i'              & = x_i + \lambda b_j x_j a_i = S_{ij}x_j \\
		      \therefore S_{ij} & = \delta_{ij} + \lambda a_i b_j
	      \end{align*}
	      For example in \(\mathbb R^2\) with \(\vb a = \begin{pmatrix}
			      1 \\ 0
		      \end{pmatrix}\) and \(\vb b = \begin{pmatrix}
			      0 \\ 1
		      \end{pmatrix}\), we have
	      \[ S = \begin{pmatrix}
			      1 & \lambda \\ 0 & 1
		      \end{pmatrix} \]
	\item A rotation matrix \(R\) in \(\mathbb R^3\) with axis \(\nhat\) and angle \(\theta\) must satisfy
	      \begin{align*}
		      \vb x'            & = R\vb x = (\cos \theta)\vb x + (1 - \cos \theta)(\nhat \cdot \vb x)\nhat + (\sin \theta)(\nhat \times \vb x) \\
		      x_i'              & = (\cos \theta)x_i + (1 - \cos \theta)n_j x_j n_i - (\sin \theta) \varepsilon_{ijk}x_j n_k = R_{ij} x_j       \\
		      \therefore R_{ij} & = \delta_{ij}(\cos \theta) - (1 - \cos \theta)n_in_j - (\sin \theta)\varepsilon_{ijk} n_k
	      \end{align*}
\end{enumerate}

\subsection{Matrix of a General Linear Map}
Consider a linear map \(T: V \to W\) between general real or complex vector spaces of dimension \(n, m\) respectively. We will choose bases \(\{ \vb e_i \}\) for \(V\) and \(\{ \vb f_a \}\) for \(W\). The matrix representing the linear map \(T\) with respect to these bases is an \(m \times n\) array with entries \(M_{ai} \in \mathbb R\) or \(\mathbb C\) as appropriate, defined by
\[ T(\vb e_i) = \sum_a \vb f_a M_{ai} \]
Then
\[ \vb x' = T(\vb x) \iff x_a' = \sum_i M_{ai}x_i = M_{ai}x_i \]
where
\[ \vb x = \sum_i x_i \vb e_i;\quad \vb x' = \sum_a x_a \vb f_a \]
Note therefore that (in real vector spaces) given choices of bases \(\{ \vb e_i \}\) and \(\{ \vb f_a \}\), \(V\) is identified with \(\mathbb R_n\) in the sense that any vector has \(n\) real components, and that \(W\) is identified with \(R_m\) analogously, and that therefore \(T\) is identified with an \(m\times n\) real matrix \(M\). Note further that entries in column \(i\) of \(M\) are components of \(T(\vb e_i)\) with respect to basis \(\{ \vb f_a \}\).
