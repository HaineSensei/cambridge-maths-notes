\subsection{Kronecker $\delta$}
The Kronecker $\delta$ is defined by
\[ \delta_{ij} = \begin{cases}
		1 & \text{if } i = j    \\
		0 & \text{if } i \neq j
	\end{cases} \]
Then $\vb e_i \vb e_j = \delta_{ij}$. We can also use $\delta$ to rewrite indices: $\sum_i \delta_{ij} \vb a_i = \vb a_j$. So
\begin{align*}
	\vb a \cdot \vb b & = \left( \sum_i \vb a_i \vb e_i \right) \cdot \left( \sum_j \vb b_j \vb e_j \right) \\
	                  & = \sum_{ij} \vb a_i \vb b_j (\vb e_i \cdot \vb e_j)                                 \\
	                  & = \sum_{ij} \vb a_i \vb b_j \delta_{ij}                                             \\
	                  & = \sum_i \vb a_i \vb b_i
\end{align*}

\subsection{Levi-Civita $\varepsilon$}
The Levi-Civita $\varepsilon$ is defined by
\[
	\varepsilon_{ijk} = \begin{cases}
		+1 & \text{if } ijk \text{ is an even permutation of } [1, 2, 3] \\
		-1 & \text{if } ijk \text{ is an odd permutation of } [1, 2, 3]  \\
		0  & \text{otherwise}
	\end{cases}
\]
Then
\begin{align*}
	\varepsilon_{123} = \varepsilon_{231} = \varepsilon_{312} & = +1 \\
	\varepsilon_{132} = \varepsilon_{321} = \varepsilon_{213} & = -1
\end{align*}
and all other permutations of $[1, 2, 3]$ yield 0. This shows that $\varepsilon$ is totally antisymmetric; exchanging any pair of indices changes the sign. We now have:
\begin{align*}
	\vb e_i \times \vb e_j & = \sum_k \varepsilon_{ijk} \vb e_k                                                   \\
	\intertext{And:}
	\vb a \times \vb b     & = \left( \sum_i \vb a_i \vb e_i \right) \times \left( \sum_j \vb b_j \vb e_j \right) \\
	\vb a \times \vb b     & = \sum_{ij} \vb a_i \vb b_j \left( \vb e_i \times \vb e_j \right)                    \\
	\vb a \times \vb b     & = \sum_{ijk} \vb a_i \vb b_j \varepsilon_{ijk} \vb e_k
\end{align*}
So the individual terms of the cross product can be written
\[ (\vb a \times \vb b)_k = \sum_{ij} \vb a_i \vb b_j \varepsilon_{ijk} \]

\subsection{Summation Convention}
We use the `summation convention' to abbreviate the many summation symbols used throughout linear algebra.
\begin{enumerate}
	\item An index which occurs exactly once in some term, denoted a `free index', must appear once in every term in that equation.
	\item An index which occurs exactly twice in a given term, denoted a `repeated/contracted/dummy index', is implicitly summed over.
	\item No index can occur more than twice in a given term.
\end{enumerate}

\subsection{$\varepsilon\varepsilon$ Identities}
The most general $\varepsilon\varepsilon$ identity is as follows:
\begin{align*}
	\varepsilon_{ijk} \varepsilon_{pqr}
	 & = \delta_{ip}\delta_{jq}\delta_{kr} - \delta_{jp}\delta_{iq}\delta_{kr} \\
	 & + \delta_{jp}\delta_{kq}\delta_{ir} - \delta_{kp}\delta_{jq}\delta_{ir} \\
	 & = \delta_{kp}\delta_{iq}\delta_{jr} - \delta_{ip}\delta_{kq}\delta_{jr}
\end{align*}
This is, however, very verbose and not used often throughout the course. It is provable by noting the total antisymmetry in $i,j,k$ and $p,q,r$ on both sides of the equation implies that both sides agree up to a constant factor. We can check that this factor is 1 by substituting in values such as $i=p=1$, $j=q=2$ and $k=r=3$.

The next most generic form is a very useful identity.
\[ \varepsilon_{ijk}\varepsilon_{pqk} = \delta_{ip}\delta_{jq} - \delta_{iq}\delta_{jp} \]
This is essentially the first line of the above identity, noting that $k=r$. We can prove this is true by observing the antisymmetry, and that both sides vanish under $i=j$ or $p=q$. So it suffices to check two cases: $i=p, j=q$ and $i=q, j=p$.

We can now continue making more indices equal to each other to get even more specific identities:
\[ \varepsilon_{ijk}\varepsilon_{pjk} = 2\delta_{ip} \]
This is easy to prove by noting that $\delta_{jj} = \sum_j \delta_{jj} = 3$, and using the $\delta$ rewrite rule.

Finally, we have
\[ \varepsilon_{ijk}\varepsilon_{ijk} = 6 \]
No indices are free here, so the values of $i, j, k$ themselves are predetermined by the fact that we are in three-dimensional space.

\subsection{Vector Triple Product Identity}
Using the summation convention (as will now be implied for the remainder of the course), we can prove
\begin{align*}
	\left[ \vb a \times (\vb b \times \vb c) \right]_i
	 & = \varepsilon_{ijk} \vb a_j (\vb b \times \vb c)_k                                                  \\
	 & = \varepsilon_{ijk} \vb a_j \varepsilon_{pqk} \vb b_p \vb c_q                                       \\
	 & = \varepsilon_{ijk}\varepsilon_{pqk} \vb a_j \vb b_p \vb c_q                                        \\
	 & = (\delta_{ip}\delta_{jq})\vb a_j \vb b_p \vb c_q - (\delta_{iq}\delta_{jp})\vb a_j \vb b_p \vb c_q \\
	 & = (\vb a \cdot \vb c) \vb b_i - (\vb a \cdot \vb b) \vb c_i
\end{align*}
