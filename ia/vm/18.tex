\subsection{Linear Independence of Eigenvectors}
\begin{proposition}
	Let $\vb v_1, \vb v_2, \dots, \vb v_r$ be eigenvectors of an $n\times n$ matrix $A$ with eigenvalues $\lambda_1, \lambda_2,\dots,\lambda_r$. If the eigenvalues are distinct, then the eigenvectors are linearly independent.
\end{proposition}
\begin{proof}
	Note that if we take some linear combination $\vb w = \sum_{j=1}^r \alpha_j\vb v_j$, then $(A - \lambda I)\vb w = \sum_{j=1}^r \alpha_j(\lambda_j - \lambda)\vb v_j$. Here are two methods for getting this proof.
	\begin{enumerate}[(i)]
		\item Suppose the eigenvectors are linearly dependent, so there exist linear combinations $\vb w = \vb 0$ where some $\alpha$ are nonzero. Let $p$ be the amount of nonzero $\alpha$ values. So, $2 \leq p \leq r$. Now, pick such a $\vb w$ for which $p$ is least. Without loss of generality, let $\alpha_1$ be one of the nonzero coefficients. Then
		      \[ (A - \lambda_1 I)\vb w = \sum_{j=2}^r \alpha_j(\lambda_j - \lambda_1)\vb v_j = \vb 0 \]
		      This is a linear relation with $p-1$ nonzero coefficients \contradiction.
		\item Alternatively, given a linear relation $\vb w=\vb 0$,
		      \[ \prod_{j \neq k} (A - \lambda_j I) \vb w = \alpha_k \prod_{j \neq k} (\lambda_k - \lambda_j) \vb v_k = \vb 0 \]
		      for some fixed $k$. So $\alpha_k = 0$. So the eigenvectors are linearly independent as claimed.
	\end{enumerate}
\end{proof}
\begin{corollary}
	With conditions as in the proposition above, let $\mathcal B_{\lambda_i}$ be a basis for the eigenspace $E_{\lambda_i}$. Then $\mathcal B = \mathcal B_{\lambda_1} \cup \mathcal B_{\lambda_2} \cup \dots \cup \mathcal B_{\lambda_r}$ is linearly independent.
\end{corollary}
\begin{proof}
	Consider a general linear combination of all these vectors, it has the form
	\[ \vb w = \vb w_1 + \vb w_2 + \dots + \vb w_r \]
	where each $\vb w_i \in E_i$. Applying the same arguments as in the proposition, we find that
	\[ \vb w = 0 \implies \forall i\,\vb w_i = 0 \]
	So each $\vb w_i$ is the trivial linear combination of elements of $\mathcal B_{\lambda_i}$ and the result follows.
\end{proof}

\subsection{Diagonalisability and Similarity}
\begin{proposition}
	For an $n \times n$ matrix $A$ acting on $V = \mathbb R^n$ or $\mathbb C^n$, the following conditions are equivalent:
	\begin{enumerate}[(i)]
		\item there exists a basis of eigenvectors of $A$ for $V$, named $\vb v_1, \vb v_2, \dots, \vb v_n$ which $A\vb v_i = \lambda_i\vb v_i$ for each $i$; and
		\item there exists an $n \times n$ invertible matrix $P$ with the property that
		      \[ P^{-1}AP = D = \begin{pmatrix}
				      \lambda_1 & 0         & \cdots & 0         \\
				      0         & \lambda_2 & \cdots & 0         \\
				      \vdots    & \vdots    & \ddots & \vdots    \\
				      0         & 0         & \cdots & \lambda_n
			      \end{pmatrix} \]
	\end{enumerate}
	If either of these conditions hold, then $A$ is diagonalisable.
\end{proposition}
\begin{proof}
	Note that for any matrix $P$, $AP$ has columns $A\vb C_i(P)$, and $PD$ has columns $\lambda_i \vb C_i(P)$. Then (i) and (ii) are related by choosing $\vb v_i = \vb C_i(P)$. Then $P^{-1}AP = D \iff AP = PD \iff A\vb v_i = \lambda_i\vb v_i$.

	In essence, given a basis of eigenvectors as in (i), the relation above defines $P$, and if the eigenvectors are linearly independent then $P$ is invertible. Conversely, given a matrix $P$ as in (ii), its columns are a basis of eigenvectors.
\end{proof}
Let's try some examples.
\begin{enumerate}[(i)]
	\item Let
	      \[ A = \begin{pmatrix}
			      1 & 1 \\ 0 & 1
		      \end{pmatrix} \implies E_1 = \left\{ \alpha\begin{pmatrix}
			      1 \\ 0
		      \end{pmatrix} \right\} \]
	      This is a single eigenvalue $\lambda = 1$ with one linearly independent eigenvector. So there is no basis of eigenvectors for $\mathbb R^2$ or $\mathbb C^2$, so $A$ is not diagonalisable.
	\item Let
	      \[ U = \begin{pmatrix}
			      \cos \theta & -\sin \theta \\
			      \sin \theta & \cos \theta
		      \end{pmatrix} \implies E_{e^{i\theta}} = \left\{ \alpha\begin{pmatrix}
			      1 \\ -i
		      \end{pmatrix} \right\};\quad E_{e^{-i\theta}} = \left\{ \beta\begin{pmatrix}
			      1 \\ i
		      \end{pmatrix} \right\} \]
	      which are two linearly independent complex eigenvectors. So,
	      \[ P = \begin{pmatrix}
			      1 & 1 \\ -i & i
		      \end{pmatrix};\quad P^{-1} = \frac{1}{2}\begin{pmatrix}
			      1 & i \\ 1 & -i
		      \end{pmatrix};\quad P^{-1}UP = \begin{pmatrix}
			      e^{i\theta} & 0 \\ 0 & e^{i\theta}
		      \end{pmatrix} \]
	      So $U$ is diagonalisable over $\mathbb C^2$ but not over $\mathbb R^2$.
\end{enumerate}

\subsection{Criteria for Diagonalisability}
\begin{proposition}
	Consider an $n \times n$ matrix $A$.
	\begin{enumerate}[(i)]
		\item $A$ is diagonalisable if it has $n$ distinct eigenvalues (sufficient condition).
		\item $A$ is diagonalisable if and only if for every eigenvalue $\lambda$, $M_\lambda = m_\lambda$ (necessary and sufficient condition).
	\end{enumerate}
\end{proposition}
\begin{proof}
	Use the proposition and corollary above.
	\begin{enumerate}[(i)]
		\item If we have $n$ distinct eigenvalues, then we have $n$ linearly independent eigenvectors. Hence they form a basis.
		\item If $\lambda_i$ are all the distinct eigenvalues, then $\mathcal B_{\lambda_1} \cup \dots \cup \mathcal B_{\lambda_r}$ are linearly independent. The number of elements in this new basis is $\sum_{i} m_{\lambda_i} = \sum_{i} M_{\lambda_i} = n$ which is the degree of the characteristic polynomial. So we have a basis.
	\end{enumerate}
	Note that case (i) is just a specialisation of case (ii) where both multiplicities are 1.
\end{proof}
Let us consider some examples.
\begin{enumerate}[(i)]
	\item Let
	      \[ A = \begin{pmatrix}
			      -2 & 2 & -3 \\ 2 & 1 & -6 \\ -1 & -2 & 0
		      \end{pmatrix} \implies \lambda = 5, -3;\quad M_5=m_5=1;\quad M_{-3}=m_{-3}=2 \]
	      So $A$ is diagonalisable by case (ii) above, and moreover
	      \[ P = \begin{pmatrix}
			      1  & -2 & 3 \\
			      2  & 1  & 0 \\
			      -1 & 0  & 1
		      \end{pmatrix};\quad P^{-1} = \frac{1}{8}\begin{pmatrix}
			      1  & 2 & -3 \\
			      -2 & 4 & 6  \\
			      1  & 2 & 5
		      \end{pmatrix} \implies P^{-1}AP = \begin{pmatrix}
			      5 & 0  & 0  \\
			      0 & -3 & 0  \\
			      0 & 0  & -3
		      \end{pmatrix} \]
	\item Let
	      \[ A = \begin{pmatrix}
			      -3 & -1 & 1 \\
			      -1 & -3 & 1 \\
			      -2 & 2  & 0
		      \end{pmatrix} \implies \lambda = -2;\quad M_{-2}=3 > m_{-2} = 2 \]
	      So $A$ is not diagonalisable. As a check, if it were diagonalisable, then there would be some matrix $P$ such that $P^{-1}AP = -2I \implies A = P(-2I)P^{-1} = -2I$ \contradiction.
\end{enumerate}

\subsection{Similarity}
Matrices $A$ and $B$ (both $n \times n$) are similar if $B = P^{-1}AP$ for some invertible $n\times n$ matrix $P$. This is an equivalence relation.
\begin{proposition}
	If $A$ and $B$ are similar, then
	\begin{enumerate}[(i)]
		\item $\tr B = \tr A$
		\item $\det B = \det A$
		\item $\chi_B = \chi_A$
	\end{enumerate}
\end{proposition}
\begin{proof}
	\begin{enumerate}[(i)]
		\item \begin{align*}
			      \tr B & = \tr (P^{-1}AP) \\&= \tr(APP^{-1}) \\&= \tr A
		      \end{align*}
		\item \begin{align*}
			      \det B & = \det (P^{-1}AP) \\&= \det P^{-1} \det A \det P \\&= \det A
		      \end{align*}
		\item \begin{align*}
			      \det(B - tI) & = \det(P^{-1}AP - tI) \\&= \det(P^{-1}AP - tP^{-1}P) \\&= \det(P^{-1}(A - tI)P) \\&= \det P^{-1} \det(A - tI) \det P \\&= \det(A - tI)
		      \end{align*}
	\end{enumerate}
\end{proof}
