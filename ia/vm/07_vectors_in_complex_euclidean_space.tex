\subsection{Definition}
We define \(\mathbb C^n\) by
\[
	\mathbb C^n := \{ \vb z = (z_1, z_2, \cdots, z_n): \forall i, z_i \in \mathbb C \}
\]
We define addition and scalar multiplication in obvious ways.
Note that we have a choice over what the scalars are allowed to be.
If we only allow scalars that are real numbers, \(\mathbb C^n\) can be considered a real vector space with bases \((0, \cdots, 1, \cdots, 0)\) and \((0, \cdots, i, \cdots, 0)\) and dimension \(2n\).
Alternatively, if we let the scalars be any complex numbers, we don't need to have imaginary bases, thus giving us a complex vector space with bases \((0, \cdots, 1, \cdots, 0)\) and dimension \(n\).
We can say that \(\mathbb C^n\) has dimension \(2n\) over \(\mathbb R\), and dimension \(n\) over \(\mathbb C\).
From here on, unless stated otherwise, we treat \(\mathbb C^n\) to be a complex vector space.

\subsection{Inner product in \(\mathbb C^n\)}
We can define the inner product by
\[
	\langle \vb z, \vb w \rangle := \sum_j \overline{z_j} w_j
\]
The conjugate over the \(z\) terms ensures that the inner product is positive definite.
It has these properties, analogous to the properties of the inner product in the real vector space \(\mathbb R^n\):
\begin{itemize}
	\item (Hermitian) \(\langle \vb z, \vb w \rangle = \overline{\langle \vb w, \vb z \rangle}\)
	\item (linear/antilinear) \(\langle \vb z, \lambda \vb w + \lambda' \vb w' \rangle = \lambda \langle \vb z, \vb w \rangle + \lambda' \langle \vb z, \vb w' \rangle\) and \(\langle \lambda \vb z + \lambda' \vb z', w \rangle = \overline{\lambda} \langle \vb z, \vb w \rangle + \overline{\lambda'} \langle \vb z', \vb w \rangle\)
	\item (positive definite) \(\langle \vb z, \vb z \rangle = \sum_j \abs{z_j}^2\) which is real and greater than or equal to zero, where the equality holds if and only if \(\vb z = \vb 0\).
\end{itemize}
We can also define the norm of \(\vb z\) to satisfy \(\abs{\vb z} \geq 0\) and \(\abs{\vb z}^2 = \langle \vb z, \vb z \rangle\).
Note that the standard basis for \(\mathbb C^n\) is orthonormal, since the inner product of any two basis vectors \(\vb e_j\) and \(\vb e_k\) is given by \(\delta_{jk}\).

\subsection{Inner product in complex plane}
Here is an example of the use of the complex inner product on \(\mathbb C^1 = \mathbb C\).
Note first that \(\langle z, w \rangle = \overline z w\).
Let \(z = a_1 + ia_2\) and \(w = b_1 + ib_2\) where \(a_1, a_2, b_1, b_2 \in \mathbb R\).
Then
\begin{align*}
	\langle z, w \rangle & = \overline z w                              \\
	                     & = (a_1 b_1 + a_2 b_2) + i(a_1 b_2 - a_2 b_1) \\
	                     & = (z \cdot w) + i[z, w]
\end{align*}
We can therefore use the inner product to compute two different scalar products at the same time.
