\subsection{Definitions}
We construct the complex numbers from $\mathbb R$ by adding an element $i$ such that $i^2 = -1$. By definition, any complex number $z \in \mathbb C = x + iy$ where $x, y \in \mathbb R$. We use the notation $x = \Re z$ and $y = \Im z$ to query the components of a complex number. The complex numbers contains the set of real numbers, due to the fact that $x = x + i0$. We define the operations of addition and multiplication in familiar ways, which lets us state that $\mathbb C$ is a field.

We also define the complex conjugate $\overline{z}$ as negating the imaginary part of $z$. Trivially we can see facts such as $\overline{\left( \overline{z}\right) } = z$; $\overline{z + w} = \overline z + \overline w$ and $\overline{zw} = \overline z \cdot \overline w$.

\subsection{The Fundamental Theorem of Algebra}
The Fundamental Theorem of Algebra states that a polynomial of degree $n$ can be written as a product of $n$ linear factors:
\[ c_nz^n + \cdots + c_1z^1 + c_0z^0 = c_n(z-\alpha_1)(z-\alpha_2) \cdots (z-\alpha_n)\quad (\text{where } c_i, \alpha_i \in \mathbb C) \]

We can reformulate this statement as follows: a polynomial of degree $n$ has $n$ solutions $\alpha_i$, counting repeats. This theorem is not proved in this course.

\subsection{Properties of Modulus}
The modulus of complex numbers $z_1, z_2$ satisfies:
\begin{itemize}
	\item (composition) $\abs{z_1 z_2} = \abs{z_1} \abs{z_2}$, and
	\item (triangle inequality) $\abs{z_1 + z_2} \leq \abs{z_1} + \abs{z_2}$
\end{itemize}
\begin{proof}
	The composition property is trivial. To prove the triangle inequality, we square both sides and compare.
	\begin{align*}
		\text{LHS} & = \abs{z_1 + z_2}^2                                                 \\
		           & = (z_1 + z_2)\overline{(z_1 + z_2)}                                 \\
		           & = \abs{z_1}^2 + \overline{z_1}z_2 + z_1\overline{z_2} + \abs{z_2}^2 \\
		\text{RHS} & = \abs{z_1}^2 + 2 \abs{z_1}\abs{z_2} + \abs{z_2}^2
	\end{align*}
	Note that
	\begin{align*}
		\overline{z_1}z_2 + z_1\overline{z_2}                                           & \leq 2 \abs{z_1}\abs{z_2}     \\
		\iff \frac{1}{2}\left( \overline{z_1}z_2 + \overline{\overline{z_1}z_2} \right) & \leq \abs{z_1}\abs{z_2}       \\
		\iff \Re (\overline{z_1} z_2)                                                   & \leq \abs{\overline{z_1} z_2}
	\end{align*}
	which is true.
\end{proof}

We can alternatively use the map $z_2 \to z_2 - z_1$ to write the triangle inequality as
\begin{align*}
	\abs{z_2 - z_1}            & \geq \abs{z_2} - \abs{z_1}       \\
	\text{or } \abs{z_2 - z_1} & \geq \abs{z_1} - \abs{z_2}       \\
	\therefore \abs{z_2 - z_1} & \geq \abs{\abs{z_2} - \abs{z_1}} \\
\end{align*}

\subsection{De Moivre's Theorem}
De Moivre's Theorem states that
\[ (\cos \theta + i \sin \theta)^n = \cos n \theta + i \sin n \theta \quad(\forall n \in \mathbb Z) \]
We can prove this using induction for $n \geq 0$. To show the negative case, simply use the positive result and raise it to the power of $-1$.
