\subsection{Linear Combinations}
If $T: V \to W$ and $S: V \to W$, between real or complex vector spaces $V, W$ of dimension $n, m$ respectively, are linear, then
\[ \alpha T + \beta S: V \to W \]
is also a linear map, where
\[ (\alpha T + \beta S)(\vb x) = \alpha T(\vb x) + \beta S(\vb x) \]
for any $\vb x \in V$. So the set of linear maps is a vector space. If $M$ and $N$ are the $m\times N$ matrices for $T, S$ then $\alpha M + \beta N$ is the $m\times n$ matrix for the linear combination above, where
\[ (\alpha M + \beta N)_{ai} + \alpha M_{ai} + \beta N_{ai};\quad a = 1, \cdots, m;\quad i = 1, \cdots, n \]
with respect to the same bases.

\subsection{Matrix Multiplication}
If $A$ is an $m\times n$ matrix with entries $A_{ai}$, and $B$ is an $n \times p$ matrix with entries $B_{ir}$, then we define $AB$ to be an $m \times p$ matrix with entries
\[ (AB)_{ar} = A_{ai}B_{ir};\quad a = 1, \cdots, m;\quad i = 1, \cdots, n;\quad r = 1, \cdots, p \]
The product is not defined unless the amount of columns of $A$ matches the number of rows of $B$.

Matrix multiplication corresponds to composition of linear maps. Consider linear maps:
\begin{align*}
	S: \mathbb R^p \to \mathbb R^n                  & ;\; S(\vb x) = B \vb x,\, \vb x \in \mathbb R^p \\
	T: \mathbb R^n \to \mathbb R^m                  & ;\; T(\vb x) = A \vb x,\, \vb x \in \mathbb R^n \\
	\implies T \circ S: \mathbb R^p \to \mathbb R^m & ;\; (T\circ S)(\vb x) = (AB)x
\end{align*}
since
\[ \left[ (AB)\vb x \right]_a = (AB)_{ar}x_r \]
and
\[ A(B(\vb x)) = A_{ai} (B\vb x)_i = A_{ai} B_{ir} x_r = (AB)_{ar}x_r \]
as required. The definition of matrix multiplication ensures that these answers agree. Of course, this proof works for complex or general vector spaces.

\subsection{Properties of Matrix Product}
Whenever the products are defined, then for any scalars $\lambda$ and $\mu$:
\begin{itemize}
	\item $(\lambda M + \mu N)P = \lambda MP + \mu NP$
	\item $P(\lambda M + \mu N) = \lambda PM + \mu PN$
	\item $(MN)P = M(NP)$
	\item $IM = MI = M$ where $I_{ij} = \delta_{ij}$
\end{itemize}
We may view matrix multiplication in the following ways.
\begin{enumerate}[(i)]
	\item Regarding a vector $\vb x \in \mathbb R^n$ as a column vector (an $n \times 1$ matrix), then the matrix-vector and matrix-matrix multiplication rules agree.
	\item Consider the product $AB$ where $A$ is an $m \times n$ matrix and $B$ is an $n \times p$, with columns $\vb C_r(B) \in \mathbb R^n$ and columns $\vb C_r(AB) \in \mathbb R^m$, where $1 \leq r \leq p$. The columns are related by $\vb C_r(AB) = A \vb C_r(B)$. Less formally, eavh column in the right matrix is acted on by the left matrix as if it were a vector, then the resultant vectors are combined into the output matrix.
	\item In terms of rows and columns,
	      \[ AB = \begin{pmatrix}
			                 & \vdots     &             \\
			      \leftarrow & \vb R_n(A) & \rightarrow \\
			                 & \vdots     &
		      \end{pmatrix} \begin{pmatrix}
			             & \uparrow   &        \\
			      \cdots & \vb C_r(B) & \cdots \\
			             & \downarrow &
		      \end{pmatrix} \]
	      gives
	      \begin{align*}
		      (AB)_{ar} & = \left[ \vb R_a(A) \right]_i \left[ \vb C_r(B) \right]_i                                              \\
		                & = \vb R_a(A) \cdot \vb C_r(B) \text{ for real matrices, where the $\cdot$ is the dot product in $R^n$}
	      \end{align*}
\end{enumerate}

\subsection{Matrix Inverses}
If $A$ is an $m \times n$ then $B$, an $n \times m$ matrix, is a left inverse of $A$ if $BA = I$ (the $n \times n$ identity matrix). $C$ is a right inverse of $A$ if $AC = I$ (the $m \times m$ identity matrix). If $m = n$ ($A$ is square), then one of these implies the other; there is no distinction between left and right inverses. We say that $B = C = A^{-1}$, \textit{the} inverse of the matrix $A$, such that $AA^{-1} = A^{-1}A = I$. Not every matrix has an inverse. If such an inverse exists, $A$ is called invertible, or non-singular.

Consider $\vb x, \vb x' \in \mathbb R^n$ or $\mathbb C^n$, and $M$ is an $n \times n$ matrix. If $M^{-1}$ exists, we can solve the equation $\vb x' = M \vb x$ for $\vb x$, given $\vb x'$, because we can apply the matrix inverse on the left. For example, where $n=2$, we have
\[ M = \begin{pmatrix}
		M_{11} & M_{12} \\
		M_{21} & M_{22}
	\end{pmatrix} \]
and
\begin{align*}
	x_1' & = M_{11}x_1 + M_{12}x_2 \\
	x_2' & = M_{21}x_1 + M_{22}x_2
\end{align*}
We can solve these simultaneous equations to construct the general matrix inverse.
\begin{align*}
	M_{22} x_1' - M_{12}x_2'  & = (\det M)x_1 \\
	-M_{21} x_1' + M_{11}x_2' & = (\det M)x_2
\end{align*}
where $\det M = M_{11} M_{22} - M_{12} M_{21}$, called the determinant of the matrix. Where the determinant is nonzero, the matrix inverse
\[ M^{-1} = \frac{1}{\det M}\begin{pmatrix}
		M_{22}  & -M_{12} \\
		-M_{21} & M_{11}
	\end{pmatrix} \]
exists. Note that
\begin{align*}
	\vb C_1     & = M \vb e_1 = \begin{pmatrix} M_{11} \\ M_{21} \end{pmatrix}                            \\
	\vb C_2     & = M \vb e_2 = \begin{pmatrix} M_{12} \\ M_{22} \end{pmatrix}                            \\
	\iff \det M & = [\vb C_1, \vb C_2] = [M\vb e_1, M\vb e_2] \text{ in } \mathbb R^2
\end{align*}
So the determinant gives the signed factor by which areas are scaled under the action of $M$. $\det M$ is nonzero if and only if $M\vb e_1$ and $M\vb e_2$ are linearly independent, which is true if and only if the image of $M$ has dimension 2, i.e. $M$ has maximal rank. For example, a shear
\[ S(\lambda) = \begin{pmatrix}
		1 & \lambda \\ 0 & 1
	\end{pmatrix} \]
has determinant 1, so areas are preserved. In particular, in this case,
\[ S^{-1}(\lambda) = \begin{pmatrix}
		1 & -\lambda \\ 0 & 1
	\end{pmatrix} = S(-\lambda) \]
As another example, we know that a matrix $R(\theta)$ for a rotation about a fixed axis $\nhat$ through angle $\theta$ has formula
\begin{align*}
	R(\theta)_{ij} R(-\theta)_{jk} & = (\delta_{ij}\cos \theta + (1 - \cos \theta) n_i n_j - \varepsilon_{ijp}n_p \sin \theta) \times (\delta_{jk}\cos \theta + (1 - \cos \theta) n_j n_k + \varepsilon_{jkq}n_q \sin \theta) \\
	\intertext{Expanding out, noting that $n_in_i = 1$ as $\nhat$ is a unit vector, and cancelling:}
	                               & = \delta_{ik} \cos^2 \theta + 2\cos \theta(1 - \cos \theta) n_in_k + (1 - \cos \theta)^2n_in_k - \varepsilon_{ijp}\varepsilon_{jkq} n_p n_q \sin^2 \theta                                \\
	\intertext{By using an $\varepsilon\varepsilon$ identity:}
	                               & = \delta_{ik}\cos^2\theta + (1 - \cos^2 \theta)n_in_k + \delta_{ik}n_pn_p \sin^2 \theta - (\sin^2 \theta)n_in_k                                                                          \\
	                               & = \delta_{ik}\cos^2\theta + \delta_{ik}n_pn_p \sin^2 \theta                                                                                                                              \\
	                               & = \delta_{ik}\cos^2\theta + \delta_{ik} \sin^2 \theta                                                                                                                                    \\
	                               & = \delta_{ik}
\end{align*}
as required.
