\subsection{Complex function definitions}
For \(z \in \mathbb C\), we can define:
\begin{align*}
	\exp z & = \sum_{n=0}^{\infty} \frac{1}{n!}z^n          \\
	\cos z & = \frac{1}{2} \left( e^{iz} + e^{-iz} \right)  \\
	\sin z & = \frac{1}{2i} \left( e^{iz} - e^{-iz} \right)
\end{align*}
By defining \(\log z = w \st e^w = z\), we have a complex logarithm function.
By expanding the definition, we get that \(\log z = \log r + i\theta\) where \(r = \abs{z}\) and \(\theta = \arg{z}\).
Note that because the argument of a complex number is multi-valued, so is the logarithm.

We can define exponentiation in the general case by defining \(z^\alpha = e^{\alpha \log z}\).
Depending on the choice of \(\alpha\), we have three cases:
\begin{itemize}
	\item If \(\alpha = p \in \mathbb Z\) then the result of \(z^p\) is unambiguous because
	      \[
		      z^p = e^{p \log z} = e^{p (\log r + i \theta + 2 \pi i n)}
	      \]
	      which has a factor of \(e^{2 \pi i p n}\) which is 1.
	\item For a similar reason, a rational exponent has finitely many values.
	\item But in the general case, there are infinitely many values.
\end{itemize}
We can calculate results such as the square root of a complex number, which have two results as you might expect.

\begin{note}
	We can't use facts like \(z^\alpha z^\beta = z^{\alpha + \beta}\) in the complex case because the left and right hand sides both have infinite sets of answers, which may not be the same.
\end{note}

\subsection{Transformations and primitives}
We can represent a line passing through \(x_0\in \mathbb C\) parallel to \(w \in \mathbb C\) using the formula:
\[
	z = z_0 + \lambda w\quad(\lambda \in \mathbb R)
\]
We can eliminate the dependency on \(\lambda\) by computing the conjugate of both sides:
\begin{align*}
	\overline{z}                  & = \overline{z_0} + \lambda \overline{w} \\
	\overline{w}z - w\overline{z} & = \overline{w}z_0 - w\overline{z_0}
\end{align*} % TODO prove this probably
We can also write the equation for a circle with centre \(c \in \mathbb C\) and radius \(\rho \in \mathbb R\):
\[
	z = c + \rho e^{i\alpha}
\]
or equivalently:
\[
	\abs{z - c} = \abs{\rho e^{i\alpha}} = \rho
\]
or by squaring both sides:
\[
	\abs{z}^2 - c\overline{z} - \overline{c}z = \rho^2 - \abs{c}^2
\]
