\subsection{Eigenvalues and eigenvectors}
For a linear map \(T\colon V \to V\), a vector \(\vb v \in V\) with \(\vb v \neq 0\) is called an eigenvector of \(T\) with eigenvalue \(\lambda\) if \(T(\vb v) = \lambda \vb v\).
If \(V = \mathbb R^n\) or \(\mathbb C^n\), and \(T\) is given by an \(n \times n\) matrix \(A\), then
\[
	A\vb v = \lambda v \iff (A - \lambda I)\vb v = \vb 0
\]
and for a given \(\lambda\), this holds for some \(\vb v \neq 0\) if and only if
\[
	\det(A - \lambda I) = 0
\]
This is called the characteristic equation for \(A\).
So \(\lambda\) is an eigenvalue if and only if it is a root of the characteristic polynomial
\[
	\chi_A(t) = \det(A - tI) = \begin{vmatrix}
		A_{11} - t & A_{12}     & \cdots & A_{1n}     \\
		A_{21}     & A_{22} - t & \cdots & A_{2n}     \\
		\vdots     & \vdots     & \ddots & \vdots     \\
		A_{n1}     & A_{n2}     & \cdots & A_{nn} - t
	\end{vmatrix}
\]
We can look for eigenvalues as roots of the characteristic polynomial or characteristic equation, and then determine the corresponding eigenvectors once we've deduced what the possibilities are.
Here are a few examples.
\begin{enumerate}
	\item \(V = \mathbb C^2\):
	      \[
		      A = \begin{pmatrix}
			      2 & i \\ -i & 2
		      \end{pmatrix} \implies \det(A - \lambda I) = (2-\lambda)^2 - 1 = 0
	      \]
	      So we have \((2 - \lambda)^2 = 1\) so \(\lambda = 1\) or 3.
	      \begin{itemize}
		      \item (\(\lambda = 1\))
		            \[
			            (A - I)\vb v = \begin{pmatrix}
				            1 & i \\ -i & 1
			            \end{pmatrix}\begin{pmatrix}
				            v_1 \\ v_2
			            \end{pmatrix} = \vb 0 \implies \vb v = \alpha\begin{pmatrix}
				            1 \\ i
			            \end{pmatrix}
		            \]
		            for any \(\alpha \neq 0\).
		      \item (\(\lambda = 3\))
		            \[
			            (A - 3I)\vb v = \begin{pmatrix}
				            -1 & i \\ -i & -1
			            \end{pmatrix}\begin{pmatrix}
				            v_1 \\ v_2
			            \end{pmatrix} = \vb 0 \implies \vb v = \beta\begin{pmatrix}
				            1 \\ -i
			            \end{pmatrix}
		            \]
		            for any \(\beta \neq 0\).
	      \end{itemize}
	\item \(V = \mathbb R^2\):
	      \[
		      A = \begin{pmatrix}
			      1 & 1 \\ 0 & 1
		      \end{pmatrix} \implies \det(A - \lambda I) = (1-\lambda)^2 = 0
	      \]
	      So \(\lambda = 1\) only, a repeated root.
	      \[
		      (A - I)\vb v = \begin{pmatrix}
			      0 & 1 \\ 0 & 0
		      \end{pmatrix}\begin{pmatrix}
			      v_1 \\ v_2
		      \end{pmatrix} = \vb 0 \implies \vb v = \alpha\begin{pmatrix}
			      1 \\ 0
		      \end{pmatrix}
	      \]
	      for any \(\alpha \neq 0\).
	      There is only one (linearly independent) eigenvector here.
	\item \(V = \mathbb R^2\) or \(\mathbb C^2\):
	      \[
		      U = \begin{pmatrix}
			      \cos\theta & -\sin\theta \\ \sin\theta & \cos\theta
		      \end{pmatrix} \implies \chi_U(t) = \det(U - tI) = t^2 - 2t\cos\theta + 1
	      \]
	      The eigenvalues \(\lambda\) are \(e^{\pm i \theta}\).
	      The eigenvectors are
	      \[
		      \vb v = \alpha \begin{pmatrix}
			      1 \\ \mp i
		      \end{pmatrix};\quad \alpha \neq 0
	      \]
	      So there are no real eigenvalues or eigenvectors except when \(\theta = n \pi\).
	\item \(V = \mathbb C^n\):
	      \[
		      A = \begin{pmatrix}
			      \lambda_1 & 0         & \cdots & 0         \\
			      0         & \lambda_2 & \cdots & 0         \\
			      \vdots    & \vdots    & \ddots & \vdots    \\
			      0         & 0         & \cdots & \lambda_n
		      \end{pmatrix} \implies \chi_A(t) = \det(A - tI) = (\lambda_1 - t)(\lambda_2 - t)(\lambda_3 - t)\dots(\lambda_n - t)
	      \]
	      So the eigenvalues are all the \(\lambda_i\), and the eigenvectors are \(\vb v = \alpha \vb e_i\) (\(\alpha \neq 0\)) for each \(i\).
\end{enumerate}

\subsection{The characteristic polynomial}
For an \(n \times n\) matrix \(A\), the characteristic polynomial \(\chi_A(t)\) has degree \(n\):
\[
	\chi_A(t) = \sum_{j = 0}^n c_j t^j = (-1)^n(t-\lambda_1)\dots(t-\lambda_n)
\]
\begin{enumerate}
	\item There exists at least one eigenvalue (solution to \(\chi_A\)), due to the fundamental theorem of algebra, or \(n\) roots counted with multiplicity.
	\item \(\tr(A) = A_{ii} = \sum_{i=1}^n \lambda_i\), the sum of the eigenvalues.
	      Compare terms of degree \(n-1\) in \(t\), and from the determinant we get
	      \[
		      (-t)^{n-1}A_{11} + (-t)^{n-1}A_{22} + \dots + (-t)^{n-1}A_{nn}
	      \]
	      The overall sign matches with the expansion of \((-1)^n(t-\lambda_1)(t-\lambda_2)\dots(t-\lambda_n)\).
	\item \(\det(A) = \chi_A(0) = \prod_{i=1}^n \lambda_i\), the product of the eigenvalues.
	\item If \(A\) is real, then the coefficients \(c_i\) in the characteristic polynomial are real, so \(\chi_A(\lambda) = 0 \iff \chi_A(\overline\lambda) = 0\).
	      So the non-real roots occur in conjugate pairs if \(A\) is real.
\end{enumerate}

\subsection{Eigenspaces and multiplicities}
For an eigenvalue \(\lambda\) of a matrix \(A\), we define the eigenspace
\[
	E_\lambda = \{ \vb v : A \vb v = \lambda \vb v \} = \ker (A - \lambda I)
\]
All nonzero vectors in this space are eigenvectors.
The geometric multiplicity is
\[
	m_\lambda = \dim E_\lambda = \nullity (A - \lambda I)
\]
equivalent to the number of linearly independent eigenvectors with the given eigenvalue \(\lambda\).
The algebraic multiplicity is
\[
	M_\lambda = \text{the multiplicity of } \lambda \text{ as a root of } \chi_A(t)
\]
i.e.\ \(\chi_A(t) = (t - \lambda)^{M_t} f(t)\), where \(f(\lambda) \neq 0\).

\begin{proposition}
	\(M_\lambda \geq m_\lambda\) (and \(m_\lambda \geq 1\) since \(\lambda\) is an eigenvalue).
	The proof of this proposition is delayed until the next section where we will then have the tools to prove it.
\end{proposition}
Here are some examples.
\begin{enumerate}
	\item
	      \[
		      A = \begin{pmatrix}
			      -2 & 2 & -3 \\ 2 & 1 & -6 \\ -1 & -2 & 0
		      \end{pmatrix} \implies \chi_A(t) = \det(A - tI) = (5-t)(t+3)^2
	      \]
	      So \(\lambda = 5, -3\).
	      \(M_5 = 1\), \(M_{-3} = 2\).
	      We will now find the eigenspaces.
	      \begin{itemize}
		      \item (\(\lambda = 5\))
		            \[
			            E_5 = \left\{ \alpha\begin{pmatrix}
				            1 \\ 2 \\ -1
			            \end{pmatrix} \right\}
		            \]
		      \item (\(\lambda = -3\))
		            \[
			            E_{-3} = \left\{ \alpha\begin{pmatrix}
				            -2 \\ 1 \\ 0
			            \end{pmatrix} + \beta\begin{pmatrix}
				            3 \\ 0 \\ 1
			            \end{pmatrix} \right\}
		            \]
	      \end{itemize}
	      Note that to compute the eigenvectors, we just need to solve the equation \((A - \lambda I)\vb x = \vb 0\).
	      In the case of \(\lambda = -3\), for example, we then have
	      \[
		      \begin{pmatrix}
			      1 & 2 & -3 \\ 2 & 4 & -6 \\ -1 & -2 & 3
		      \end{pmatrix} \begin{pmatrix}
			      x_1 \\ x_2 \\ x_3
		      \end{pmatrix} = \vb 0
	      \]
	      We can use the first line of the matrix to get a linear combination for \(x_1, x_2, x_3\), specifically \(x_1 + 2x_2 = 3x_3 = 0\), so we can eliminate one of the variables (here, \(x_1\)) to get
	      \[
		      \vb x = \begin{pmatrix}
			      -2x_2 + 3x_3 \\ x_2 \\ x_3
		      \end{pmatrix} = \vb 0
	      \]
	      Now, \(\dim E_5 = m_5 = 1 = M_5\).
	      Similarly, \(\dim E_{-3} = m_{-3} = 2 = M_{-3}\).

	\item
	      \[
		      A = \begin{pmatrix}
			      -3 & -1 & 1 \\ -1 & -3 & 1 \\ -2 & -2 & 0
		      \end{pmatrix} \implies \chi_A(t) = \det(A - tI) = -(t + 2)^3
	      \]
	      We have a root \(\lambda = -2\) with \(M_{-2} = 3\).
	      To find the eigenspace, we will look for solutions of:
	      \[
		      (A + 2I)\vb x = \begin{pmatrix}
			      -1 & -1 & 1 \\ -1 & -1 & 1 \\ -2 & -2 & 2
		      \end{pmatrix} \begin{pmatrix}
			      x_1 \\ x_2 \\ x_3
		      \end{pmatrix} = \vb 0 \implies \vb x = \begin{pmatrix}
			      -x_2 + x_3 \\ x_2 \\ x_3
		      \end{pmatrix}
	      \]
	      So
	      \[
		      E_{-2} = \left\{ \alpha\begin{pmatrix}
			      -1 \\ 1 \\ 0
		      \end{pmatrix} + \beta\begin{pmatrix}
			      1 \\ 0 \\ 1
		      \end{pmatrix} \right\}
	      \]
	      Further, \(m_{-2} = 2 < 3 = M_{-2}\).

	\item A reflection in a plane through the origin with unit normal \(\nhat\) satisfies
	      \[
		      H\nhat = -\nhat;\quad \forall \vb u \perp \nhat, H \vb u = \vb u
	      \]
	      The eigenvalues are therefore \(\pm 1\) and \(E_{-1} = \{ \alpha \nhat \}\), and \(E_1 = \{ \vb x: \vb x \cdot \nhat = 0 \}\).
	      The multiplicities are given by \(M_{-1} = m_{-1} = 1, M_1 = m_1 = 2\).

	\item A rotation about an axis \(\nhat\) through angle \(\theta\) in \(\mathbb R^3\) satisfies
	      \[
		      R\nhat = \nhat
	      \]
	      So the axis of rotation is the eigenvector with eigenvalue 1.
	      There are no other real eigenvalues unless \(\theta = n\pi\).
	      The rotation restricted to the plane perpendicular to \(\nhat\) has eigenvalues \(e^{\pm i \theta}\) as shown above.
\end{enumerate}

\subsection{Linear independence of eigenvectors}
\begin{proposition}
	Let \(\vb v_1, \vb v_2, \dots, \vb v_r\) be eigenvectors of an \(n\times n\) matrix \(A\) with eigenvalues \(\lambda_1, \lambda_2,\dots,\lambda_r\).
	If the eigenvalues are distinct, then the eigenvectors are linearly independent.
\end{proposition}
\begin{proof}
	Note that if we take some linear combination \(\vb w = \sum_{j=1}^r \alpha_j\vb v_j\), then \((A - \lambda I)\vb w = \sum_{j=1}^r \alpha_j(\lambda_j - \lambda)\vb v_j\).
	Here are two methods for getting this proof.
	\begin{enumerate}
		\item Suppose the eigenvectors are linearly dependent, so there exist linear combinations \(\vb w = \vb 0\) where some \(\alpha\) are nonzero.
		      Let \(p\) be the amount of nonzero \(\alpha\) values.
		      So, \(2 \leq p \leq r\).
		      Now, pick such a \(\vb w\) for which \(p\) is least.
		      Without loss of generality, let \(\alpha_1\) be one of the nonzero coefficients.
		      Then
		      \[
			      (A - \lambda_1 I)\vb w = \sum_{j=2}^r \alpha_j(\lambda_j - \lambda_1)\vb v_j = \vb 0
		      \]
		      This is a linear relation with \(p-1\) nonzero coefficients \contradiction.
		\item Alternatively, given a linear relation \(\vb w=\vb 0\),
		      \[
			      \prod_{j \neq k} (A - \lambda_j I) \vb w = \alpha_k \prod_{j \neq k} (\lambda_k - \lambda_j) \vb v_k = \vb 0
		      \]
		      for some fixed \(k\).
		      So \(\alpha_k = 0\).
		      So the eigenvectors are linearly independent as claimed.
	\end{enumerate}
\end{proof}
\begin{corollary}
	With conditions as in the proposition above, let \(\mathcal B_{\lambda_i}\) be a basis for the eigenspace \(E_{\lambda_i}\).
	Then \(\mathcal B = \mathcal B_{\lambda_1} \cup \mathcal B_{\lambda_2} \cup \dots \cup \mathcal B_{\lambda_r}\) is linearly independent.
\end{corollary}
\begin{proof}
	Consider a general linear combination of all these vectors, it has the form
	\[
		\vb w = \vb w_1 + \vb w_2 + \dots + \vb w_r
	\]
	where each \(\vb w_i \in E_i\).
	Applying the same arguments as in the proposition, we find that
	\[
		\vb w = 0 \implies \forall i\,\vb w_i = 0
	\]
	So each \(\vb w_i\) is the trivial linear combination of elements of \(\mathcal B_{\lambda_i}\) and the result follows.
\end{proof}

\subsection{Diagonalisability}
\begin{proposition}
	For an \(n \times n\) matrix \(A\) acting on \(V = \mathbb R^n\) or \(\mathbb C^n\), the following conditions are equivalent:
	\begin{enumerate}
		\item there exists a basis of eigenvectors of \(A\) for \(V\), named \(\vb v_1, \vb v_2, \dots, \vb v_n\) which \(A\vb v_i = \lambda_i\vb v_i\) for each \(i\); and
		\item there exists an \(n \times n\) invertible matrix \(P\) with the property that
		      \[
			      P^{-1}AP = D = \begin{pmatrix}
				      \lambda_1 & 0         & \cdots & 0         \\
				      0         & \lambda_2 & \cdots & 0         \\
				      \vdots    & \vdots    & \ddots & \vdots    \\
				      0         & 0         & \cdots & \lambda_n
			      \end{pmatrix}
		      \]
	\end{enumerate}
	If either of these conditions hold, then \(A\) is diagonalisable.
\end{proposition}
\begin{proof}
	Note that for any matrix \(P\), \(AP\) has columns \(A\vb C_i(P)\), and \(PD\) has columns \(\lambda_i \vb C_i(P)\).
	Then (i) and (ii) are related by choosing \(\vb v_i = \vb C_i(P)\).
	Then \(P^{-1}AP = D \iff AP = PD \iff A\vb v_i = \lambda_i\vb v_i\).

	In essence, given a basis of eigenvectors as in (i), the relation above defines \(P\), and if the eigenvectors are linearly independent then \(P\) is invertible.
	Conversely, given a matrix \(P\) as in (ii), its columns are a basis of eigenvectors.
\end{proof}
Let's try some examples.
\begin{enumerate}
	\item Let
	      \[
		      A = \begin{pmatrix}
			      1 & 1 \\ 0 & 1
		      \end{pmatrix} \implies E_1 = \left\{ \alpha\begin{pmatrix}
			      1 \\ 0
		      \end{pmatrix} \right\}
	      \]
	      This is a single eigenvalue \(\lambda = 1\) with one linearly independent eigenvector.
	      So there is no basis of eigenvectors for \(\mathbb R^2\) or \(\mathbb C^2\), so \(A\) is not diagonalisable.
	\item Let
	      \[
		      U = \begin{pmatrix}
			      \cos \theta & -\sin \theta \\
			      \sin \theta & \cos \theta
		      \end{pmatrix} \implies E_{e^{i\theta}} = \left\{ \alpha\begin{pmatrix}
			      1 \\ -i
		      \end{pmatrix} \right\};\quad E_{e^{-i\theta}} = \left\{ \beta\begin{pmatrix}
			      1 \\ i
		      \end{pmatrix} \right\}
	      \]
	      which are two linearly independent complex eigenvectors.
	      So,
	      \[
		      P = \begin{pmatrix}
			      1 & 1 \\ -i & i
		      \end{pmatrix};\quad P^{-1} = \frac{1}{2}\begin{pmatrix}
			      1 & i \\ 1 & -i
		      \end{pmatrix};\quad P^{-1}UP = \begin{pmatrix}
			      e^{i\theta} & 0 \\ 0 & e^{i\theta}
		      \end{pmatrix}
	      \]
	      So \(U\) is diagonalisable over \(\mathbb C^2\) but not over \(\mathbb R^2\).
\end{enumerate}

\subsection{Criteria for diagonalisability}
\begin{proposition}
	Consider an \(n \times n\) matrix \(A\).
	\begin{enumerate}
		\item \(A\) is diagonalisable if it has \(n\) distinct eigenvalues (sufficient condition).
		\item \(A\) is diagonalisable if and only if for every eigenvalue \(\lambda\), \(M_\lambda = m_\lambda\) (necessary and sufficient condition).
	\end{enumerate}
\end{proposition}
\begin{proof}
	Use the proposition and corollary above.
	\begin{enumerate}
		\item If we have \(n\) distinct eigenvalues, then we have \(n\) linearly independent eigenvectors.
		      Hence they form a basis.
		\item If \(\lambda_i\) are all the distinct eigenvalues, then \(\mathcal B_{\lambda_1} \cup \dots \cup \mathcal B_{\lambda_r}\) are linearly independent.
		      The number of elements in this new basis is \(\sum_{i} m_{\lambda_i} = \sum_{i} M_{\lambda_i} = n\) which is the degree of the characteristic polynomial.
		      So we have a basis.
	\end{enumerate}
	Note that case (i) is just a specialisation of case (ii) where both multiplicities are 1.
\end{proof}
Let us consider some examples.
\begin{enumerate}
	\item Let
	      \[
		      A = \begin{pmatrix}
			      -2 & 2 & -3 \\ 2 & 1 & -6 \\ -1 & -2 & 0
		      \end{pmatrix} \implies \lambda = 5, -3;\quad M_5=m_5=1;\quad M_{-3}=m_{-3}=2
	      \]
	      So \(A\) is diagonalisable by case (ii) above, and moreover
	      \[
		      P = \begin{pmatrix}
			      1  & -2 & 3 \\
			      2  & 1  & 0 \\
			      -1 & 0  & 1
		      \end{pmatrix};\quad P^{-1} = \frac{1}{8}\begin{pmatrix}
			      1  & 2 & -3 \\
			      -2 & 4 & 6  \\
			      1  & 2 & 5
		      \end{pmatrix} \implies P^{-1}AP = \begin{pmatrix}
			      5 & 0  & 0  \\
			      0 & -3 & 0  \\
			      0 & 0  & -3
		      \end{pmatrix}
	      \]
	\item Let
	      \[
		      A = \begin{pmatrix}
			      -3 & -1 & 1 \\
			      -1 & -3 & 1 \\
			      -2 & 2  & 0
		      \end{pmatrix} \implies \lambda = -2;\quad M_{-2}=3 > m_{-2} = 2
	      \]
	      So \(A\) is not diagonalisable.
	      As a check, if it were diagonalisable, then there would be some matrix \(P\) such that \(P^{-1}AP = -2I \implies A = P(-2I)P^{-1} = -2I\) \contradiction.
\end{enumerate}

\subsection{Similarity}
Matrices \(A\) and \(B\) (both \(n \times n\)) are similar if \(B = P^{-1}AP\) for some invertible \(n\times n\) matrix \(P\).
This is an equivalence relation.
\begin{proposition}
	If \(A\) and \(B\) are similar, then
	\begin{enumerate}
		\item \(\tr B = \tr A\)
		\item \(\det B = \det A\)
		\item \(\chi_B = \chi_A\)
	\end{enumerate}
\end{proposition}
\begin{proof}
	\begin{enumerate}
		\item \begin{align*}
			      \tr B & = \tr (P^{-1}AP) \\&= \tr(APP^{-1}) \\&= \tr A
		      \end{align*}
		\item \begin{align*}
			      \det B & = \det (P^{-1}AP) \\&= \det P^{-1} \det A \det P \\&= \det A
		      \end{align*}
		\item \begin{align*}
			      \det(B - tI) & = \det(P^{-1}AP - tI) \\&= \det(P^{-1}AP - tP^{-1}P) \\&= \det(P^{-1}(A - tI)P) \\&= \det P^{-1} \det(A - tI) \det P \\&= \det(A - tI)
		      \end{align*}
	\end{enumerate}
\end{proof}

\subsection{Real eigenvalues and orthogonal eigenvectors}
Recall that an \(n\times n\) matrix \(A\) is hermitian if and only if \(A^\dagger = \overline{A}^\transpose = A\), or \(\overline{A_{ij}} = A_{ji}\).
If \(A\) is real, then it is hermitian if and only if it is symmetric.
The complex inner product for \(\vb v, \vb w \in \mathbb C^n\) is \(\vb v^\dagger \vb w = \sum_i \overline{v_i}w_i\), and for \(\vb v, \vb w \in \mathbb R^n\), this reduces to the dot product in \(\mathbb R^n\), \(\vb v^\transpose \vb w\).

Here is a key observation.
If \(A\) is hermitian, then
\[
	(A\vb v)^\dagger \vb w = \vb v^\dagger (A \vb w)
\]
\begin{theorem}
	For an \(n \times n\) matrix \(A\) that is hermitian:
	\begin{enumerate}
		\item Every eigenvalue \(\lambda\) is real;
		\item Eigenvectors \(\vb v, \vb w\) with different eigenvalues \(\lambda, \mu\) respectively, are orthogonal, i.e.\ \(\vb v^\dagger \vb w = 0\); and
		\item If \(A\) is real and symmetric, then for each eigenvalue \(\lambda\) we can choose a real eigenvector, and part (ii) becomes \(\vb v \cdot \vb w = 0\).
	\end{enumerate}
\end{theorem}
\begin{proof}
	\begin{enumerate}
		\item Using the observation above with \(\vb v = \vb w\) where \(\vb v\) is any eigenvector with eigenvalue \(\lambda\), we get
		      \begin{align*}
			      \vb v^\dagger (A\vb v)        & = (A\vb v)^\dagger \vb v                   \\
			      \vb v^\dagger (\lambda\vb v)  & = (\lambda\vb v)^\dagger \vb v             \\
			      \lambda \vb v^\dagger (\vb v) & = \overline{\lambda} (\vb v)^\dagger \vb v \\
			      \intertext{As \(\vb v\) is an eigenvector, it is nonzero, so \(\vb v^\dagger \vb v \neq 0\), so}
			      \lambda                       & = \overline \lambda
		      \end{align*}
		\item Using the same observation,
		      \begin{align*}
			      \vb v^\dagger (A \vb w)   & = (A \vb v)^\dagger \vb w       \\
			      \vb v^\dagger (\mu \vb w) & = (\lambda \vb v)^\dagger \vb w \\
			      \mu \vb v^\dagger \vb w   & = \lambda \bm v^\dagger \vb w
		      \end{align*}
		      Since \(\lambda \neq \mu\), \(\vb v^\dagger \vb w = 0\), so the eigenvectors are orthogonal.
		\item Given \(A\vb v = \lambda \vb v\) with \(\vb v \in \mathbb C^n\) but \(A\) is real, let
		      \[
			      \vb v = \vb u + i\vb u';\quad \vb u, \vb u' \in \mathbb R^n
		      \]
		      Since \(\vb v\) is an eigenvector, and this is a linear equation, we have
		      \[
			      A\vb u = \lambda \vb u;\quad A\vb u' = \lambda \vb u'
		      \]
		      So \(\vb u\) and \(\vb u'\) are eigenvectors.
		      \(\vb v \neq 0\) implies that at least one of \(\vb u\) and \(\vb u'\) are nonzero, so there is at least one real eigenvector with this eigenvalue.
	\end{enumerate}
\end{proof}
Case (ii) is a stronger claim for hermitian matrices than just showing that eigenvectors are linearly independent.
Furthermore, previously we considered bases \(\mathcal B_\lambda\) for each eigenspace \(E_\lambda\), and it is now natural to choose bases \(\mathcal B_\lambda\) to be orthonormal when we are considering hermitian matrices.
Here are some examples.
\begin{enumerate}
	\item Let
	      \[
		      A = \begin{pmatrix}
			      2 & i \\ -i & 2
		      \end{pmatrix};\quad A^\dagger = A;\quad \lambda = 1, 3;\quad\vb u_1 = \frac{1}{\sqrt{2}} \begin{pmatrix}
			      1 \\i
		      \end{pmatrix};\quad\vb u_2 = \frac{1}{\sqrt{2}} \begin{pmatrix}
			      1 \\-i
		      \end{pmatrix}
	      \]
	      We have chosen coefficients for the vectors \(\vb u_1\) and \(\vb u_2\) such that they are unit vectors.
	      As shown above, they are then orthonormal.
	      We know that having distinct eigenvalues means that a matrix is diagonalisable.
	      So let us set
	      \[
		      P =  \frac{1}{\sqrt{2}} \begin{pmatrix}
			      1 & 1 \\ i & -i
		      \end{pmatrix} \implies P^{-1}AP = D = \begin{pmatrix}
			      1 & 0 \\ 0 & 3
		      \end{pmatrix}
	      \]
	      Since the eigenvectors are orthonormal, so are the columns of \(P\), so \(P^{-1} = P^\dagger\) (i.e.\ \(P\) is unitary).
	\item Let
	      \[
		      A = \begin{pmatrix}
			      0 & 1 & 1 \\ 1 & 0 & 1 \\ 1 & 1 & 0
		      \end{pmatrix}
	      \]
	      \(A\) is real and symmetric, with eigenvalues \(\lambda = -1, 2\) with \(M_{-1} = 2\), \(M_2 = 1\).
	      Further,
	      \[
		      E_{-1} = \vecspan \{ \vb w_1, \vb w_2 \};\quad \vb w_1 = \begin{pmatrix}
			      1 \\ -1 \\ 0
		      \end{pmatrix};\quad \vb w_2 = \begin{pmatrix}
			      1 \\ 0 \\ -1
		      \end{pmatrix}
	      \]
	      So \(m_{-1} = 2\), and the matrix is diagonalisable.
	      Let us choose an orthonormal basis for \(E_{-1}\) by taking
	      \[
		      \vb u_1 = \frac{1}{\abs{\vb w_1}}\vb w_1 = \frac{1}{\sqrt 2}\begin{pmatrix}
			      1 \\ -1 \\ 0
		      \end{pmatrix}
	      \]
	      and we can consider
	      \[
		      \vb w_2' = \vb w_2 - (\vb u_1 \cdot \vb w_2)\vb u_1 = \begin{pmatrix}
			      1/2 \\ 1/2 \\ -1
		      \end{pmatrix}
	      \]
	      so that \(\vb w_2'\) is orthogonal to \(\vb u_1\) by construction.
	      We can then normalise this vector to get
	      \[
		      \vb u_2 = \frac{1}{\abs{\vb w_2'}}\vb w_2' = \frac{1}{\sqrt 6} \begin{pmatrix}
			      1 \\ 1 \\ -2
		      \end{pmatrix}
	      \]
	      and therefore
	      \[
		      \mathcal B_{-1} = \{ \vb u_1, \vb u_2 \}
	      \]
	      is an orthonormal basis.
	      For \(E_2\), let us choose \(\mathcal B_2 = \{ \vb u_3 \}\) where
	      \[
		      \vb u_3 = \frac{1}{\sqrt 3}\begin{pmatrix}
			      1 \\ 1 \\ 1
		      \end{pmatrix}
	      \]
	      Together,
	      \[
		      \mathcal B = \left\{ \frac{1}{\sqrt 2}\begin{pmatrix}
			      1 \\ -1 \\ 0
		      \end{pmatrix}, \frac{1}{\sqrt 6} \begin{pmatrix}
			      1 \\ 1 \\ -2
		      \end{pmatrix}, \frac{1}{\sqrt 3}\begin{pmatrix}
			      1 \\ 1 \\ 1
		      \end{pmatrix} \right\}
	      \]
	      is an orthonormal basis for \(\mathbb R^3\).
	      Let \(P\) be the matrix with columns \(\vb u_1, \vb u_2, \vb u_3\), then \(P^{-1}AP = D\) as required.
	      Since we have chosen an orthonormal basis, \(P\) is orthogonal, so \(P^\transpose AP = D\).
\end{enumerate}

\subsection{Unitary and orthogonal diagonalisation}
\begin{theorem}
	Any \(n\times n\) hermitian matrix \(A\) is diagonalisable.
	\begin{enumerate}
		\item There exists a basis of eigenvectors \(\vb u_1, \dots, \vb u_n \in \mathbb C^n\) with \(A\vb u_i = \lambda \vb u_i\); equivalently
		\item There exists an \(n \times n\) invertible matrix \(P\) with \(P^{-1}AP = D\) where \(D\) is the matrix with eigenvalues on the diagonal, where the columns of \(P\) are the eigenvectors \(\vb u_i\).
	\end{enumerate}
	In addition, the eigenvectors \(\vb u_i\) can be chosen to be orthonormal, so
	\[
		\vb u^\dagger_i \vb u_j = \delta_{ij}
	\]
	or equivalently, the matrix \(P\) can be chosen to be unitary,
	\[
		P^\dagger = P^{-1} \implies P^\dagger AP = D
	\]
	In the special case that the matrix \(A\) is real, the eigenvectors can be chosen to be real, and so
	\[
		\vb u^\transpose \vb u_j = \vb u_i \cdot \vb u_j = \delta_{ij}
	\]
	so \(P\) is orthogonal, so
	\[
		P^\transpose = P^{-1} \implies P^\transpose AP = D
	\]
\end{theorem}
