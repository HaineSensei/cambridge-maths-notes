We use the normal Euclidean notions of points, lines, planes, length, angles and so on.
By choosing an (arbitrary) origin point \(O\), we may write positions as position vectors with respect to that origin point.

\subsection{Vector addition and scalar multiplication}
We define vector addition using the shape of a parallelogram with points \(\vb 0, \vb a, \vb a + \vb b, \vb b\).
We define scalar multiplication of a vector using the line \(\overrightarrow{OA}\) and setting the length to be multiplied by the constant.
Note that this vector space is an abelian group under addition.
\begin{definition}
	\(\vb a\) and \(\vb b\) are defined to be parallel if and only if \(\vb a = \lambda \vb b\) or \(\vb b = \lambda \vb a\) for some \(\lambda \in \mathbb R\).
	This is denoted \(\vb a \parallel \vb b\).
	Note that the vectors may be zero, in particular the zero vector is parallel to all vectors.
\end{definition}
\begin{definition}
	The span of a set of vectors is defined as \(\vecspan \{\vb a, \vb b, \cdots, \vb c\} = \{ \alpha \vb a + \beta \vb b + \cdots + \gamma \vb c: \alpha, \beta, \gamma \in \mathbb R \}\).
	This is the line/plane/volume etc.
	containing the vectors.
	The span has an amount of dimensions at most equal to the amount of vectors in the input set.
	For example, the span of a set of two vectors may be a point, line or plane containing the vectors.
\end{definition}

\subsection{Scalar product}
\begin{definition}
	Given two vectors \(\vb a, \vb b\), let \(\theta\) be the angle between the two vectors.
	Then, we define
	\[
		\vb a \cdot \vb b = \abs{\vb a} \abs{\vb b} \cos \theta
	\]
	Note that if either of the vectors is zero, \(\theta\) is undefined.
	However, the dot product is zero anyway here, so this is irrelevant.
\end{definition}
\begin{definition}
	Two vectors \(\vb a\) and \(\vb b\) are defined to be parallel (or orthogonal) if and only if \(\vb a \cdot \vb b = 0\).
	This is denoted \(\vb a \perp \vb b\).
	This is true in two cases:
	\begin{enumerate}
		\item \(\cos \theta = 0 \iff \theta = \frac{\pi}{2} \mod \pi\), or
		\item \(\vb a = 0\) or \(\vb b = 0\).
	\end{enumerate}
	Therefore, the zero vector is perpendicular to all vectors.
\end{definition}
\begin{definition}
	We can decompose a vector \(\vb b\) into components relative to \(\vb a\):
	\[
		\vb b = \vb b_\parallel + \vb b_\perp
	\]
	where \(\vb b_\parallel\) is the component of \(\vb b\) parallel to \(\vb a\), and \(\vb b_\perp\) is the component of \(\vb b\) perpendicular to \(\vb a\).
	In particular, we have that
	\[
		\vb a \cdot \vb b = \vb a \cdot \vb b_\parallel
	\]
\end{definition}

\subsection{Vector product}
\begin{definition}
	Given two vectors \(\vb a, \vb b\), let \(\theta\) be the angle between the two vectors measured with respect to an arbitrary normal \(\vb{\hat n}\).
	Then, we define
	\[
		\vb a \wedge \vb b = \vb a \times \vb b = \abs{\vb a} \abs{\vb b} \vb{\hat n} \sin \theta
	\]
	Note that by swapping the sign of \(\vb{\hat n}\), \(\theta\) changes to \(2 \pi - \theta\), leaving the result unchanged.
	There are two degenerate cases:
	\begin{itemize}
		\item \(\theta\) is undefined if \(\vb a\) or \(\vb b\) is the zero vector, but the result is zero anyway because we multiply by the magnitudes of both vectors.
		\item \(\vb{\hat n}\) is undefined if \(\vb a \parallel \vb b\), but here \(\sin \theta = 0\) so the result is zero anyway.
	\end{itemize}
\end{definition}
\noindent We can provide several useful interpretations of the cross product:
\begin{itemize}
	\item The magnitude of \(\vb a \times \vb b\) is the vector area of the parallelogram defined by the points \(\vb 0, \vb a, \vb a + \vb b, \vb b\).
	\item By fixing a vector \(\vb a\), we can consider the plane perpendicular to it.
	      If \(\vb x\) is another vector in the plane, \(\vb x \mapsto \vb a \times \vb x\) rotates \(\vb x\) by \(\frac{\pi}{2}\) in the plane, scaling it by the magnitude of \(\vb a\).
\end{itemize}
Note that by resolving a vector \(\vb b\) perpendicular to another vector \(\vb a\), we have that
\[
	\vb a \times \vb b = \vb a \times \vb b_\perp
\]
A final useful property of the cross product is that since the result is perpendicular to both input vectors, we have
\[
	\vb a \cdot (\vb a \times \vb b) = \vb b \cdot(\vb a \times \vb b) = 0
\]
