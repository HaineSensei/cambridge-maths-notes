\subsection{Quadrics in General}
A quadric in \(\mathbb R^n\) is a hypersurface defined by an equation of the form
\[
	Q(\vb x) = \vb x^\transpose A \vb x + \vb b^\transpose \vb x + c = 0
\]
for some nonzero, symmetric, real \(n \times n\) matrix \(A\), \(b \in \mathbb R^n\), \(c \in \mathbb R\).
In components,
\[
	Q(\vb x) = A_{ij}x_ix_j + b_ix_i + c = 0
\]
We will clasify solutions for \(\vb x\) up to geometrical equivalence, so we will not distinguish between solutions here which are related by isometries in \(\mathbb R^n\) (distance-preserving maps, i.e.\ translations and orthogonal transformations about the origin).

Note that \(A\) is invertible if and only if it has no zero eigenvalues.
In this case, we can complete the sequare in the equation \(Q(\vb x) = 0\) by setting \(\vb y = \vb x + \frac{1}{2}A^{-1} \vb b\).
This is essentially a translation isometry, moving the origin to \(\frac{1}{2}A^{-1} \vb b\).
\begin{align*}
	\vb y^\transpose A \vb y & = (\vb x + \frac{1}{2}A^{-1}\vb b)^\transpose A (\vb x + \frac{1}{2}A^{-1}\vb b)                         \\
	                         & = (\vb x^\transpose + \frac{1}{2}\vb b^\transpose(A^{-1})^\transpose) A (\vb x + \frac{1}{2}A^{-1}\vb b) \\
	                         & = \vb x^\transpose A \vb x + \vb b^\transpose \vb x + \frac{1}{4}\vb b^\transpose A^{-1}\vb b
\end{align*}
since \((A^\transpose)^{-1} = (A^{-1})^\transpose\).
Then,
\[
	Q(\vb x) = 0 \iff \mathcal F(\vb y) = k
\]
with
\[
	\mathcal F(\vb y) = \vb y^\transpose A \vb y
\]
which is a quadratic form with respect to a new origin \(\vb y = \vb 0\), and where \(k = \frac{1}{4}\vb b^\transpose A^{-1}\vb b - c\).
Now we can diagonalise \(\mathcal F\) as in the above section, in particular, orthonormal eigenvectors give the principal axes, and the eigenvalues of \(A\) and the value of \(k\) determine the geometrical nature of the solution of the quadric.
In \(\mathbb R^3\), the geometrical possibilities are (as we saw before):
\begin{enumerate}[(i)]
	\item eigenvalues positive, \(k\) positive gives an ellipsoid;
	\item eigenvalues different signs, \(k\) nonzero gives a hyperboloid
\end{enumerate}
If \(A\) has one or more zero eigenvalues, then the analysis we have just provided changes, since we can no longer construct such a \(\vb y\) vector, since \(A^{-1}\) does not exist.
The simplest standard form of \(Q\) may have both linear and quadratic terms.

\subsection{Conics as Quadrics}
Quadrics in \(\mathbb R^2\) are curves called conics.
Let us first consider the case where \(\det A \neq 0\).
By completing the square and diagonalising \(A\), we get a standard form
\[
	\lambda_1 {x'_1}^2 + \lambda_2 {x'_2}^2 = k
\]
The variables \(x'_i\) correspond to the principal axes and the new origin.
We have the following cases.
\begin{itemize}
	\item (\(\lambda_1, \lambda_2 > 0\)) This is an ellipse for \(k>0\), and a point for \(k=0\).
	      There are no solutions for \(k<0\).
	\item (\(\lambda_1 > 0, \lambda_2 < 0\)) This gives a hyperbola for \(k>0\), and a hyperbola in the other axis if \(k<0\).
	      If \(k=0\), this is a pair of lines.
	      For instance, \({x'_1}^2 - {x'_2}^2 = 0 \implies (x'_1 - x'_2)(x'_1 + x'_2) = 0\).
\end{itemize}
If \(\det A = 0\), then there is exactly one zero eigenvalue since \(A \neq 0\).
Then:
\begin{itemize}
	\item (\(\lambda_1 > 0, \lambda_2 = 0\)) We will diagonalise \(A\) in the original expression for the quadric.
	      This gives
	      \[
		      \lambda_1 {x'_1}^2 + b'_1 x'_1 + b'_2 x'_2 + c = 0
	      \]
	      This is a new equation in the coordinate system defined by \(A\)'s principal axes.
	      Completing the square here in the \(x'_1\) term, we have
	      \[
		      \lambda_1 {x''_1}^2 + b'_2x'_2 + c' = 0
	      \]
	      where \(x''_1 = x'_1 + \frac{1}{2\lambda_1}b'_1\), and \(c' = c - \frac{{b'_1}^2}{4\lambda_1^2}\).
	      If \(b'_2 = 0\), then \(x_2\) can take any value; and we get a pair of lines if \(c'<0\), a single line if \(c'=0\), and no solutions if \(c'>0\).
	      Otherwise, \(b'_2 \neq 0\), and the equation becomes
	      \[
		      \lambda_1 {x''_1}^2 + b'_2x''_2 = 0
	      \]
	      where \(x_2'' = x'_2 + \frac{1}{b_2'}c'\), and clearly this equation is a parabola.
\end{itemize}
All changes of coordinates correspond to translations (shifts of the origin) or orthogonal transformations, both of which preserve distance and angles.

\subsection{Standard Forms for Conics}
The general forms of conics can be written in terms of lengths \(a, b\) (the semi-major and semi-minor axes), or equivalently a length scale \(\ell\) and a dimensionless eccentricity constant \(e\).
\begin{itemize}
	\item First, let us consider Cartesian coordinates.
	      The formulas are:

	      \medskip\noindent\begin{tabular}{c|c|c|c}
		      conic     & formula                                   & eccentricity                         & foci         \\\hline
		      ellipse   & \(\frac{x^2}{a^2} + \frac{y^2}{b^2} = 1\) & \(b^2=a^2(1-e^2)\), and \(e<1\)      & \(x=\pm ae\) \\
		      parabola  & \(y^2 = 4ax\)                             & one quadratic term vanishes, \(e=1\) & \(x = +a\)   \\
		      hyperbola & \(\frac{x^2}{a^2} - \frac{y^2}{b^2} = 1\) & \(b^2=a^2(e^2-1)\), and \(e<1\)      & \(x=\pm ae\)
	      \end{tabular}

	\item Polar coordinates are a convenient alternative to Cartesian coordinates.
	      In this coordinate system, we set the origin to be at a focus.
	      Then, the formulas are
	      \[
		      r = \frac{\ell}{1 + e\cos \theta}
	      \]
	      \begin{itemize}
		      \item For the ellipse, \(e<1\) and \(\ell = a(1-e^2)\);
		      \item For the parabola, \(e=1\) and \(\ell = 2a\); and
		      \item For the hyperbola, \(e>1\) and \(\ell = a(e^2 - 1)\).
		            There is only one branch for the hyperbola given by this polar form.
	      \end{itemize}
\end{itemize}

%TODO draw graphs for all of these curves in both coordinate systems

\subsection{Conics as Sections of a Cone}
The equation for a cone in \(\mathbb R^3\) given by an apex \(\vb c\), an axis \(\nhat\), and an angle \(\alpha < \frac{\pi}{2}\), is
\[
	(\vb x - \vb c)\cdot\nhat = \abs{\vb x - \vb c}\cos \alpha
\]
Less formally, the angle of \(\vb x\) away from \(\nhat\) must be \(\alpha\).
By squaring this equation, we can essentially define two cones which stretch out infinitely far and meet at the centre point \(\vb c\).
\[
	\left( (\vb x - \vb c)\cdot\nhat \right)^2 = \abs{\vb x - \vb c}^2\cos^2 \alpha
\]
Let us choose a set of coordinate axes so that our equations end up slightly easier.
Let \(\vb c = c\vb e_3, \nhat = \cos\beta \vb e_1 - \sin\beta \vb e_3\).
Then essentially the cone starts at \((0, 0, c)\) and points `downwards' in the \(\vb e_1\)--\(\vb e_3\) plane.
Then the conic section is the intersection of this cone with the \(\vb e_1\)--\(\vb e_2\) plane, i.e.\ \(x_3 = 0\).
\[
	(x_1\cos\beta - c\sin\beta)^2 = (x_1^2 + x_2^2 + c^2)\cos^2\alpha
\]
\[
	\iff (\cos^2\alpha - \cos^2\beta)x_1^2 + (\cos^2\alpha)x_2^2 + 2x_1c\sin\beta\cos\beta = \text{const.}
\]
Now we can compare the signs of the \(x_1^2\) and \(x_2^2\) terms.
Clearly the \(x_2^2\) term is always positive, so we consider the sign of the \(x_1^2\) term.
\begin{itemize}
	\item If \(\cos^2 \alpha > \cos^2\beta\) (i.e.\ \(\alpha < \beta\)), then we have an ellipse.
	\item If \(\cos^2 \alpha = \cos^2\beta\) (i.e.\ \(\alpha = \beta\)), then we have a parabola.
	\item If \(\cos^2 \alpha < \cos^2\beta\) (i.e.\ \(\alpha > \beta\)), then we have a hyperbola.
\end{itemize}
