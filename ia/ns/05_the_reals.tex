\subsection{Motivation for the reals}
Why do we need the real numbers in the first place?
Well, we introduce new sets of numbers when there are equations that we cannot solve using our current number system.
For example, the equation \(x+2=0\) is not solvable in \(\mathbb N\), so we constructed \(\mathbb Z\).
Then we could not solve equations like \(2x = 3\), so we created the rationals, \(\mathbb Q\).
Now, we cannot solve equations such as \(x^2 = 2\), so we must create a new set of numbers that contains this solution.

\begin{proposition}
	There does not exist a \(q \in \mathbb Q\) such that \(q^2 = 2\).
	Note that in this proposition we make no assumption that \(q^2 = 2\) is solvable, or that a solution if one exists does not lie within \(\mathbb Q\); we simply state that confined to the realm of \(\mathbb Q\) the equation is unsolvable.
\end{proposition}
\begin{proof}[Proof 1]
	Suppose that such a \(q \in \mathbb Q\) exists, such that \(q^2 = 2\).
	Without loss of generality, we will assume that \(q>0\) because \((-q)^2 = q^2\).
	So let \(q\) be written as \(a/b\) where \(a, b \in \mathbb N\).
	Then \(a^2/b^2 = 2\), so \(a^2 = 2b^2\).
	If we factorise each side as a product of primes, the exponent of the prime 2 on the left hand side must be even, but on the right hand side it must be odd.
	This contradicts the unique factorisation of natural numbers.
	So such a \(q\) does not exist.
\end{proof}
\begin{proof}[Proof 2]
	Suppose that there exists some \(q \in \mathbb Q\) written similarly to above as \(a/b\).
	Note that for any \(c, d \in \mathbb Z\), \(cq + d\) is of the form \(e/b\) for some integer \(e\).
	Therefore, if \(cq+d>0\) then \(cq+d \geq 1/b\).

	Now, note that \(0 < (q - 1) < 1\), so for a suitably large \(n\), we have \(0 < (q - 1)^n < 1/b\).
	However, \((q-1)^n\) is of the form \(cq+d\) because \(q^2 = 1\) so we can eliminate all exponents.
	This is a contradiction so such a \(q\) does not exist.
\end{proof}

\subsection{Least upper bound}
We can see from the proofs above that \(\mathbb Q\) has a `gap' at \(\sqrt 2\).
How can we express this fact without mentioning \(\mathbb R\)?
We can't just say plainly that \(\sqrt 2 \notin \mathbb Q\) because as far as we know from \(\mathbb Q\), there is no reason to assume that such a number called \(\sqrt 2\) even exists!
We need to find a way to express the concept of \(\sqrt 2\) in the language of \(\mathbb Q\).
One way to do this is by creating sone set \(S = \{ q \in \mathbb Q: q^2 < 2 \}\).
Then we can write down some upper bounds for this set.
For example, 2 is a trivial upper bound, as is \(1.5\), and as is \(1.42\).
In fact, we can continue making smaller and smaller upper bounds.
We can see therefore that there exists no least upper bound in \(\mathbb Q\).

\subsection{Assumptions about the reals}
We define the reals as follows: the reals are a set written \(\mathbb R\) with elements 0 and 1 with \(0 \neq 1\); with operations \(+\) and \(\cdot\); and an ordering \(<\); such that:
\begin{enumerate}
	\item \(+\) is commutative, associative, has identity 0, and there are inverses for all elements;
	\item \(\cdot\) is commutative, associative, has identity 1, and there are inverses for all nonzero elements;
	\item \(\cdot\) is distributive over \(+\);
	\item for all \(a\) and \(b\) in \(\mathbb R\), exactly one of \(a<b\), \(a=b\) and \(a>b\) are true, and that \(a<b\) and \(b<c\) implies \(a<c\);
	\item for all \(a, b, c \in \mathbb R\), \(a<b\) implies \(a + c < b + c\), and \(a<b\) implies \(ac < bc\) when \(c > 0\); and
	\item for any set \(S\) of reals that is non-empty and bounded above, \(S\) has a least upper bound.
\end{enumerate}

\subsection{Immediate remarks}
There are some notable immediate remarks about the definitions of the reals.
\begin{itemize}
	\item We can contain the rationals inside the reals: \(\mathbb Q \subset \mathbb R\)
	\item The least upper bound axiom is false in \(\mathbb Q\), which is why it's so important in \(\mathbb R\).
	\item Why did we specify `non-empty' and `bounded above' in the least upper bound axiom?
	      Of course, if a set is not bounded above, then it has no upper bound, so clearly it can have no least upper bound.
	      If a set is empty, then every real is an upper bound for this set, and as there is no least real number, there is no least upper bound.
	\item It is possible to construct \(\mathbb R\) out of \(\mathbb Q\), and check that the above axioms hold.
	      However, this is a rare example where the construction of \(\mathbb R\) is complicated and irrelevant, so it is not covered here.
\end{itemize}

\subsection{Axiom of Archimedes}
The reals do not contain infinitely big or infinitesimally small elements.
\begin{proposition}
	\(\mathbb N\) is not bounded above in \(\mathbb R\).
\end{proposition}
\begin{proof}
	If there were some upper bound \(c = \sup \mathbb N\), then \(c-1\) is clearly not an upper bound for \(\mathbb N\).
	So there exists some natural number \(n\) such that \(n > c-1\).
	But then clearly \(n+1 \in \mathbb N > c\) contradicting the existence of this upper bound.
\end{proof}
\begin{corollary}
	For each \(t \in \mathbb R > 0\), \(\exists n \in \mathbb N\) such that \(\frac{1}{n} < t\).
\end{corollary}
\begin{proof}
	We have some \(n \in \mathbb N\) with \(n > \frac{1}{t}\) by the above proposition.
	So \(\frac{1}{n} < t\).
\end{proof}

\subsection{Examples of sets and least upper bounds}
Note that a common way to write `least upper bound' is the word supremum, denoted \(\sup S\).
\begin{enumerate}
	\item Let \(S = \{ x \in \mathbb R: 0 \leq x \leq 1 \} = [0, 1]\).
	      The least upper bound of \(S\) is 1, because:
	      \begin{itemize}
		      \item 1 is an upper bound for \(S\); \(\forall x \in S, x\leq1 \); and
		      \item Every upper bound \(y\) must have \(y \geq 1\) because \(1 \in S\).
	      \end{itemize}
	\item Let \(S = \{ x \in \mathbb R: 0 < x < 1 \} = (0, 1)\).
	      \(\sup S = 1\) because:
	      \begin{itemize}
		      \item 1 is an upper bound for \(S\); \(\forall x \in S, x \leq 1\); and
		      \item No upper bound \(c\) has \(c<1\).
		            Indeed, certainly \(c>0\) (\(c > \frac{1}{2}\) since \(\frac{1}{2} \in S\)).
		            So if \(c<1\), then \(0<c<1\), so the number \(\frac{1+c}{2} \in S\) and is larger than \(c\), so it is not an upper bound.
	      \end{itemize}
	\item Let \(S = \{ 1 - \frac{1}{n}: n \in \mathbb N \}\).
	      \(\sup S = 1\) because:
	      \begin{itemize}
		      \item 1 is clearly an upper bound.
		      \item Let us suppose \(c < 1\) is an upper bound.
		            Then \(\forall n \in \mathbb N, 1 - \frac{1}{n} < c\) so \(1 - c < \frac{1}{n}\).
		            From the corollary of the Axiom of Archimedes above, this is a contradiction.
	      \end{itemize}
\end{enumerate}
\begin{remark}
	If \(S\) has a greatest element, then this element is the supremum of the set: \(\sup S \in S\).
	But if \(S\) does not have a greatest element, then \(\sup S \notin S\).
	Also, we do not need any kind of `greatest lower bound' axiom --- if \(S\) is a non-empty, bounded below set of reals, then the set \(\{ -x: x \in S \}\) is non-empty and bounded above, and so has a least upper bound, so \(S\) has a greatest lower bound equivalent to its additive inverse.
	This is commonly called the `infimum', or \(\inf S\).
\end{remark}

\subsection{Elements of the reals}
\begin{theorem}
	\(\exists x \in \mathbb R\) with \(x^2 = 2\).
\end{theorem}
\begin{proof}
	Let \(S\) be the set of all real numbers such that \(x^2 < 2\).
	Of course, it is non-empty (try \(x=0\)) and bounded above (try \(x=2\)).
	So let \(c = \sup S\); we want to show that \(c^2 = 2\).
	We prove this by eliminating all alternatives; clearly either \(c^2 < 2\), \(c^2 = 2\) or \(c^2 > 2\).
	\begin{itemize}
		\item (\(c^2 < 2\)) We want to prove that \((c+t)^2 < 2\) for some small \(t\).
		      For \(0<t<1\), we have \((c+t)^2 = c^2 + 2ct + t^2 \leq c^2 + 5t\), since \(c\) is at most 2, and \(t^2\) is at most \(t\).
		      So this value is less than 2 for some suitably small \(t\), contradicting the least upper bound --- we have just shown that \((c+t) \in S\).
		\item (\(c^2 > 2\)) We want to prove that \((c-t)^2 > 2\) for some small \(t\).
		      For \(0<t<1\), we have \((c-t)^2 = c^2 - 2ct + t^2 \geq c^2 - 4t\), since \(c\) is at most 2, and \(t^2\) is at least zero.
		      So this value is greater than 2 for some suitably small \(t\), contradicting the least upper bound --- we have just created a lower upper bound.
	\end{itemize}
	So \(c^2 = 2\).
\end{proof}
This same kind of proof works for a lot of real values, for example \(\sqrt[n]{x}\) for \(n \in \mathbb N\), \(x\in \mathbb R, x < 0\).
Reals that are not rational are called irrational.
This is a negative statement however, so it is better in proofs to suppose that something is rational, and then show a contradiction.

Also, the rationals are `dense'; for any \(a, b \in \mathbb R\), there is another rational between them.
We may assume without loss of generality that they are both non-negative and that \(a<b\).
Then pick some \(n \in \mathbb N\) with \(\frac{1}{n} < b-a\).
Among the list \(\frac{0}{n}, \frac{1}{n}, \frac{2}{n}, \cdots\), there is a final one that is less than or equal to \(a\), which we will denote \(\frac{q}{n}\) (otherwise \(a\) is an upper bound to this list, contradicting the axiom of Archimedes).
So \(a < \frac{q + 1}{n} < b\) as required.

The irrationals are also dense; for any reals \(a\) and \(b\) with the same conditions above, these exists some irrational \(c\) with \(a<c<b\).
We know that there exists a rational \(c\) with \(a\sqrt{2} < c < b\sqrt{2}\), so \(a < \frac{c}{\sqrt{2}} < b\).
