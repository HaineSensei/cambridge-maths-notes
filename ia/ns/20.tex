\subsection{Equivalence Classes as Partitions}
\begin{proposition}
	Let \(R\) be an equivalence relation on a set \(X\).
	Then the equivalence classes of \(R\) partition \(X\).
\end{proposition}
\begin{proof}
	Each equivalence class \([x]\) is non-empty, since \(x = x\).
	Further,
	\[
		\bigcup_{x \in X} = X
	\]
	since \(x \in [x]\) for all \(x \in X\).
	Now we must show that the classes are disjoint, or are equal.
	Given \(x, y\) with \([x] \cap [y] \neq \varnothing\), we need to show that \([x] = [y]\).
	Choose some \(z\) such that \(z \in [x] \cap [y]\).
	Then, \(zRx\) and \(zRy\), so \(xRy\).
	Thus for any \(t\), \(tRx \implies tRy\) due to transitivity, and \(tRy \implies tRx\) for the same reason.
	So \([x] = [y]\).
\end{proof}
As an example, does there exist an equivalence relation on \(\mathbb N\) with three equivalence classes, two of which are infinite, and one of which is finite? Yes --- we can break up \(\mathbb N\) into three parts, for example positive numbers, negative numbers and zero.
This defines an equivalence relation.

\subsection{Quotients}
Given an equivalence relation \(R\) on a set \(X\), the quotient of \(X\) by \(R\) is
\[
	X/R = \{ [x]: x \in X \}
\]
The map \(q\colon X\to X/R\) given by \(x \mapsto [x]\) is called the `quotient map' or `projection map'.
As an example, on \(\mathbb Z \times \mathbb N\), let us define \((a, b)R(c, d)\) to be true if \(ad=bc\).
This is an equivalence relation that demonstrates equivalence of rational numbers, where \(a, c\) are the numerators and \(b, d\) are denominators.
Here, \(\mathbb Z \times \mathbb N / R\) is a copy of \(\mathbb Q\), associating \([(a, b)]\) with \(a/b\).
Then, \(q\colon \mathbb Z \times \mathbb N \to \mathbb Q\) would map \((a, b)\) to \(a/b\).

\subsection{Countability}
We have a notion of `size' for finite sets.
Is there such an analogous notion for infinite sets? We will say that a set \(X\) is countable if \(X\) is finite, or it bijects with \(\mathbb N\).
Equivalently, we can list out the elements of the set, and each element will appear in the list.
Here are some examples.
\begin{enumerate}[(i)]
	\item Clearly any finite set is countable.
	\item \(\mathbb N\) is countable.
	\item \(\mathbb Z\) is countable, let us construct the list of numbers
	      \[
		      0, 1, -1, 2, -2, 3, -3, 4, -4, \dots
	      \]
\end{enumerate}
It makes sense now to consider two sets to have the same size if they biject with each other.

\subsection{Countability under Injections}
\begin{proposition}
	A set \(X\) is countable if and only if it injects into \(\mathbb N\).
\end{proposition}
\begin{proof}
	The forward implication is trivial: if \(X\) is finite, then there must be an injection in to \(\mathbb N\), and if it bijects with \(\mathbb N\) then that bijection is a valid injection.
	This encompasses both cases of countable sets.
	
	Now let us consider the reverse implication.
	We may assume \(X\) is infinite, since if \(X\) is finite then by definition \(X\) is countable.
	We know that \(X\) injects onto \(\mathbb N\) under some injective function \(f\), so \(X\) bijects with \(\Im f\).
	So it is enough to show that the image \(\Im f\) is countable.
	We will now set \(a_1\) to be the least element of \(\Im f\), and \(a_2\) to be the least element not equal to \(a_1\), and so on.
	In general, \(a_n = \min (\Im f \setminus \{ a_i : 0 \leq i < n \})\).
	Then \(\Im f\) is the set \(\{ a_1, a_2, \dots \}\).
	There are no extra elements that we have not covered, since each \(a \in X\) is \(a_n\) for some \(n\), because \(a=a_n, n \leq a\).
	So we have listed elements of \(\Im f\), so \(\Im f\) is countable, so \(X\) is countable.
\end{proof}
Thus, we can view countability as being `at most as large as \(\mathbb N\)'.
For instance, any subset of a countable set is also countable.

\begin{remark}
	In \(\mathbb R\), let
	\[
		X = \left\{ \frac{1}{2}, \frac{2}{3}, \frac{3}{4}, \dots \right\} \cup \{ 1 \}
	\]
	Then \(X\) is countable, as we can list it as
	\[
		1, \frac{1}{2}, \frac{2}{3}, \frac{3}{4}, \dots
	\]
	But if we counted from `least element' to `most element', we would never reach the element 1 in countable time.
	Note further that if we find it difficult to construct a list for a set, it does not mean it is uncountable, it could just mean that we haven't found the right list yet.
\end{remark}
