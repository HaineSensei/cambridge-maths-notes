\subsection{Proofs}
\begin{definition}[Proof]
	A proof is a logical argument that establishes a conclusion.
\end{definition}

Clearly there are some things missing from this definition; we have not yet defined a `logical argument' or a `conclusion'; however we have to start somewhere, and assuming understanding of logic is a good place to start.
There is a 3rd year course called `Logic and Set Theory' that rigorously defines this.

\subsection{Motivation for proof}
There are two main reasons to want to prove things.
\begin{enumerate}
	\item To be sure that they are true; and
	\item to understand why they are true.
\end{enumerate}

For the first point, it is easy to make a contrived example that shows why we need to prove statements even though they appear to be true for small \(n\), for example: `all positive integers \(n\) are not equal to 100 trillion'.
Understanding the reasoning behind why a statement is true is also very important; an example of this is at the end of this lecture.

\subsection{Examples of proofs and non-proofs}
\begin{claim}
	For any positive integer \(n\), \(n^3-n\) is a multiple of 3.
\end{claim}
\begin{proof}
	Given some positive integer \(n\), we have

	\[
		n^3 - n = (n-1)n(n+1)
	\]

	One of \(n-1,\,n,\,n+1\) must be a multiple of 3 as they are 3 consecutive integers.

	Therefore, \((n-1)n(n+1)\) must be a multiple of 3.
\end{proof}

There are a couple of things to note about this proof.
\begin{itemize}
	\item The phrase `given a positive integer' is important; we need to know where this variable \(n\) came from.
	\item We used the fact that three consecutive numbers contain a multiple of 3 here, but this was not proven.
	      We must prove this fact elsewhere, or we cannot use it in this course!
	\item It is important to write proofs legibly and linearly down the page; don't just write a long line of symbols.
\end{itemize}

\begin{claim}
	For any positive integer \(n\), if \(n^2\) is even then \(n\) is even.
\end{claim}
\begin{proof}
	Given a positive integer \(n\) that is even, we have \(n=2k\) for some integer \(k\).

	Thus \(n^2 = (2k)^2 = 4k^2 = 2(2k^2)\),

	so \(n^2\) is even.
\end{proof}
\begin{note}
	This is a false proof.
	We proved that \(B \implies A\), but we want \(A \implies B\).
	Our result wasn't false, but it didn't show what we set out to prove.
	The words `for some integer \(k\)' are important: we must specify which set \(k\) belongs to.
	Our proof would be incorrect if we did not state this, as it would be unclear that \(2(2k^2)\) is an even number.
\end{note}

\begin{claim}
	For any positive integer \(n\), if \(n^2\) is a multiple of 9 then \(n\) is a multiple of 9.
\end{claim}
\begin{proof}
	Given a positive integer \(n\) that is a multiple of 9, we have \(n=9k\) for some integer \(k\).

	Therefore, \(n^2 = (9k)^2 = 81k^2 = 9(9k^2)\),

	so \(n^2\) is a multiple of 9.
\end{proof}
\begin{note}
	Not only does this fall for the same trap as the previous proof, but the original claim is false (e.g.
	\(n=6\))!
	It's entirely irrelevant that the claim is true for some positive integers, because even one counterexample disproves the claim.
\end{note}

Let's return now to the previous incorrect example: `if \(n^2\) even then \(n\) even for all positive integers \(n\)'.

\begin{proof}
	Suppose that \(n\) is odd.

	We have \(n = 2k+1\) for some integer \(k\).

	Therefore, \(n^2 = (2k+1)^2 = 4k^2 + 4k + 1 = 2(2k^2 + 2k) + 1\)

	\(n^2\) is odd \contradiction{}

	Therefore \(n\) is even.
\end{proof}
\begin{itemize}
	\item We prove things to show \textit{why} something is true.
	      We can see why this claim was true here --- it's really a statement about the properties of odd numbers, not the properties of even numbers.
	\item We started by saying that we need something tangible to work with: just stating that `\(n^2\) is even' is really hard to work with because square roots just get messy and don't yield any result.
	      So we had to choose a clever first step.
	\item The symbol \contradiction{} shows that we have a contradiction.
\end{itemize}

This was a kind of proof by contradiction.
Essentially, \(A \implies B\) is the same as saying \(\neg B \implies \neg A\).
This is because:
\begin{itemize}
	\item \(A \implies B\) means that there is no case such that \(A\) is false and \(B\) is true.
	\item \(\neg B \implies \neg A\) means that there is no case such that \(\neg B\) is false and \(\neg A\) is true.
	      In other words, there is no case such that \(B\) is true and \(A\) is false.
	      This is equivalent to the case with \(A \implies B\).
\end{itemize}

\begin{claim}
	The solution to the real equation \(x^2-5x+6=0\) is \(x=2\) or \(x=3\).
\end{claim}
\begin{note}
	This is really two assertions:
	\begin{enumerate}[i.]
		\item \(x=2 \lor x=3 \implies x^2 - 5x + 6 = 0\), and
		\item \(x^2 - 5x + 6 = 0 \implies x=2 \lor x=3\)
	\end{enumerate}
	We can denote this using a two-way implication symbol \(\iff\):
	\[
		x=2 \lor x=3 \iff x^2 - 5x + 6 = 0
	\]
\end{note}
\begin{proof}
	We prove case i by expressing the left hand side as a product of factors: \((x-3)(x-2)=0\).
	The other case may be proven using factorisation.
\end{proof}

We can do another kind of proof using \(\iff\) symbols a lot.
However, we need to be absolutely sure that each step really is a bi-implication.
\begin{proof}[Alternative Proof]
	For any real \(x\):
	\begin{align*}
		x^2-5x+6=0 & \iff (x-2)(x-3) = 0       \\
		           & \iff x-2 = 0 \lor x-3 = 0 \\
		           & \iff x=2 \lor x = 3
	\end{align*}
\end{proof}

\begin{claim}
	Every positive real is at least 1.
\end{claim}
\begin{proof}
	Let \(x\) be the smallest positive real.
	We want to prove \(x=1\), so we prove this by contradiction.

	Case 1: if \(x < 1\) then \(x^2 < x\) \contradiction{}

	Case 2: if \(x > 1\) then \(\sqrt{x} < x\) \contradiction{}

	Therefore \(x=1\)
\end{proof}
\begin{note}
	The assertion that there exists a smallest positive real is not justified.
	This means that the proof is invalid in its entirety.
	It is important that every line in a proof must be justified.
\end{note}
