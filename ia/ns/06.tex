\subsection{Highest Common Factor}
The common factors of two numbers \(m = p_1^{a_1} \cdots p_k^{a_k}\) and \(n = p_1^{b_1} \cdots p_k^{b_k}\) where \(a\) and \(b\) are zero or above is given by \(p_1^{c_1} \cdots p_k^{c_k}\) where \(c_i \leq \min(a_i, b_i)\) So the highest common factor is given by \(c_i = \min(a_i, b_i)\).

\subsection{Lowest Common Multiple}
The common multiples of those two numbers is given by \(d_i \geq \max(a_i, b_i)\). So analogously the lowest common multiple is given by \(d_i = \max(a_i, b_i)\).

We have an interesting property that \(\HCF(m, n) \LCM(m, n) = mn\). This is true because any term \(p_i\) is given by \(p_i^{\min(a_i, b_i)}p_i^{\max(a_i, b_i)} = p_i^{a_i + b_i}\).

\subsection{Modular Arithmetic}
In modular arithmetic, we need to prove that things like addition and multiplication are valid. In order to do this, we need to show that if \(a \equiv a' \mod n\) and \(b \equiv b' \mod n\) then, for example, \(ab \equiv a'b'\). We can prove these statements trivially by writing \(a' = a + kn\) where \(k\) is some integer, then evaluating the left and right hand sides in \(\mathbb Z\).

Many rules of arithmetic are inherited from \(\mathbb Z\); for example, addition is commutative. This is easy to realise: to prove that \(a + b = b + a\) in \(\mathbb Z_n\) it is sufficient to prove the statement is true in the whole of \(\mathbb Z\).

As another example, we can transform the unique prime factorisation lemma into \(\mathbb Z_p\). In \(\mathbb Z_p\) where \(p\) is prime,
\[ ab = 0 \implies (a = 0) \lor (b = 0) \]
In general, \(\mathbb Z_p\) where \(p\) is prime is a very well behaved and convenient-to-use subset of \(\mathbb Z\).

\subsection{Inverses}
For any \(a, b \in \mathbb Z_n\), \(b\) is an inverse of \(a\) if \(ab=1\). Note that unlike in group theory, it is not necessarily the case that all elements will have inverses. For example, in \(\mathbb Z_{10}\), the elements 3 and 7 are inverses, but 4 has no inverse. Note that:
\begin{itemize}
	\item Invertible integers are cancellable. For example, \(ab=ac \implies b=c\) if \(a\) is invertible (by left-multiplying by its inverse).
	\item In general, you cannot simply cancel an integer multiple in the realm of modular arithmetic. For example \(4 \cdot 5 = 2 \cdot 5 \centernot\implies 4 = 2\).
	\item Invertible numbers are also called `units'.
\end{itemize}
