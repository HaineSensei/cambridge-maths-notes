\subsection{Injection, Surjection and Bijection}
\begin{definition}
	A function \(f\colon A \to B\) is
	\begin{itemize}
		\item injective, if \(\forall a, a' \in A\), we have \(a \neq a' \implies f(a) \neq f(a')\), or equivalently, \(f(a) = f(a') \implies a = a'\), or in words, `different points stay different' (e.g.
		      example 6 above).
		\item surjective, if \(\forall b \in B\), \(\exists a \in A\) such that \(f(a) = b\), or in words, `everything in \(B\) is hit' (e.g.
		      examples 6 and 8).
		\item bijective, if it is injective and surjective, or in words, `everything in \(B\) is hit exactly once', or `\(f\) pairs up elements of \(A\) and elements of \(B\)' (e.g.
		      example 6, or \(f\colon \mathbb R \to \mathbb R\) given by \(f(x) = x^3\)).
	\end{itemize}
\end{definition}
\begin{definition}
	For a function \(f\colon A \to B\), \(A\) is the domain, \(B\) is the range, and \(\{ b \in B : \exists a \in A \st f(a) = b \}\) is the image.
\end{definition}
We must always provide the domain and range of a function; a function's properties depend on this.
For example, is the function \(f\) defined by \(f(x) = f^2\) injective?
If \(f\colon \mathbb N \to \mathbb N\), then it is injective, but if \(f\colon \mathbb Z \to \mathbb Z\), then it is not.

There are a number of properties that hold specifically for finite sets \(A\), \(B\):
\begin{enumerate}
	\item There is no surjection \(A \to B\) if \(\abs{B} > \abs{A}\).
	\item There is no injection \(A \to B\) if \(\abs{A} > \abs{B}\).
	\item For a function \(f\colon A \to A\), \(f\) injective \(\iff\) \(f\) surjective.
	      Hence, if \(f\) is either injective or surjective, it is bijective.
	\item There is no bijection from \(A\) to any proper subset of \(A\).
\end{enumerate}
As counterexamples for infinite sets:
\begin{enumerate}
	\item We define \(f_0\colon \mathbb N \to \mathbb N\) by \(f_0(x) = x+1\).
	      Then, \(f_0\) is injective but not surjective.
	\item We define \(f_1\colon \mathbb N \to \mathbb N\) by \(f_0(x) = x-1\), or 1 if \(x=1\).
	      Then, \(f_0\) is surjective but not injective.
	\item We define \(g\colon \mathbb N \to \mathbb N \setminus \{ 1 \}\) by \(g(x) = x+1\).
	      Then, \(g\) is bijective between \(\mathbb N\) and a proper subset of \(\mathbb N\).
\end{enumerate}

\subsection{More Examples of Functions}
\begin{enumerate}
	\item For any set \(X\) we have \(1_X\colon X \to X\) defined by \(1_X(x) = x\).
	      This is known as the identity function on \(X\).
	\item For any set \(X\), and \(A \subset X\), we have an indicator function (or characteristic function) \(\chi_A\colon X \to \{ 0, 1 \}\) defined by
	      \[
		      \chi_A(x) = \begin{cases}
			      0 & x \notin A \\
			      1 & x \in A
		      \end{cases}
	      \]
	\item A sequence of reals \(x_1, x_2, \dots\) is a function \(f\colon \mathbb N \to \mathbb R\) defined by \(f(n) = x_n\).
	\item The operation \(+\) on \(\mathbb N\) is a function \(\mathbb N^2 \to \mathbb N\).
	\item A set \(X\) has size \(n\) \(\iff\) there is a bijection between \(X\) and \(\{ 1, 2, \dots, n \}\).
\end{enumerate}

\subsection{Composition of Functions}
Given \(f\colon A \to B\) and \(g\colon B \to C\), we define the composition \(g\circ f \colon A \to C\), given by \((g\circ f)(a) = g(f(a))\).
For example, if \(f\colon \mathbb R \to \mathbb R\), \(f(x) = 2x\), \(g\colon \mathbb R \to \mathbb R\), \(g(x) = x+1\), then \((f \circ g)(x) = 2(x+1)\), and \((g \circ f)(x) = 2x + 1\).

In general, the operation \(\circ\) is not commutative, as we can see from this example.
However, \(\circ\) is associative.
Given \(f\colon A \to B\), \(g\colon B \to C\), \(h\colon C \to D\), we have \(h \circ (g \circ f) = (h \circ g) \circ f\).
Indeed, for any input \(x \in A\),
\[
	(h \circ (g \circ f))(x) = h((g \circ f)(x)) = h(g(f(x))) = (h \circ g)(f(x)) = ((h \circ g)\circ f)(x)
\]
Thus \((h \circ (g \circ f))(x) = ((h \circ g)\circ f)(x)\) for every \(x \in A\), so \(h \circ (g \circ f) = (h \circ g)\circ f\).
