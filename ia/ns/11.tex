\subsection{Remarks}
If \(S\) has a greatest element, then this element is the supremum of the set: \(\sup S \in S\).
But if \(S\) does not have a greatest element, then \(\sup S \notin S\).
Also, we do not need any kind of `greatest lower bound' axiom --- if \(S\) is a non-empty, bounded below set of reals, then the set \(\{ -x: x \in S \}\) is non-empty and bounded above, and so has a least upper bound, so \(S\) has a greatest lower bound equivalent to its additive inverse.
This is commonly called the `infimum', or \(\inf S\).

\subsection{Elements of the reals}
\begin{theorem}
	\(\exists x \in \mathbb R\) with \(x^2 = 2\).
\end{theorem}
\begin{proof}
	Let \(S\) be the set of all real numbers such that \(x^2 < 2\).
	Of course, it is non-empty (try \(x=0\)) and bounded above (try \(x=2\)).
	So let \(c = \sup S\); we want to show that \(c^2 = 2\).
	We prove this by eliminating all alternatives; clearly either \(c^2 < 2\), \(c^2 = 2\) or \(c^2 > 2\).
	\begin{itemize}
		\item (\(c^2 < 2\)) We want to prove that \((c+t)^2 < 2\) for some small \(t\).
		      For \(0<t<1\), we have \((c+t)^2 = c^2 + 2ct + t^2 \leq c^2 + 5t\), since \(c\) is at most 2, and \(t^2\) is at most \(t\).
		      So this value is less than 2 for some suitably small \(t\), contradicting the least upper bound --- we have just shown that \((c+t) \in S\).
		\item (\(c^2 > 2\)) We want to prove that \((c-t)^2 > 2\) for some small \(t\).
		      For \(0<t<1\), we have \((c-t)^2 = c^2 - 2ct + t^2 \geq c^2 - 4t\), since \(c\) is at most 2, and \(t^2\) is at least zero.
		      So this value is greater than 2 for some suitably small \(t\), contradicting the least upper bound --- we have just created a lower upper bound.
	\end{itemize}
	So \(c^2 = 2\).
\end{proof}
This same kind of proof works for a lot of real values, for example \(\sqrt[n]{x}\) for \(n \in \mathbb N\), \(x\in \mathbb R, x < 0\).
Reals that are not rational are called irrational.
This is a negative statement however, so it is better in proofs to suppose that something is rational, and then show a contradiction.

Also, the rationals are `dense'; for any \(a, b \in \mathbb R\), there is another rational between them.
We may assume without loss of generality that they are both non-negative and that \(a<b\).
Then pick some \(n \in \mathbb N\) with \(\frac{1}{n} < b-a\).
Among the list \(\frac{0}{n}, \frac{1}{n}, \frac{2}{n}, \cdots\), there is a final one that is less than or equal to \(a\), which we will denote \(\frac{q}{n}\) (otherwise \(a\) is an upper bound to this list, contradicting the axiom of Archimedes).
So \(a < \frac{q + 1}{n} < b\) as required.

The irrationals are also dense; for any reals \(a\) and \(b\) with the same conditions above, these exists some irrational \(c\) with \(a<c<b\).
We know that there exists a rational \(c\) with \(a\sqrt{2} < c < b\sqrt{2}\), so \(a < \frac{c}{\sqrt{2}} < b\).

\subsection{Sequences and limits}
How can we ascribe meaning to expressions like this?
\[
	1 + \frac{1}{2} + \frac{1}{4} + \frac{1}{8} + \cdots
\]
Certainly, we have a concept of addition, and we can keep adding as many terms as we like, but there is no implicit definition of an infinite sum from the aforementioned axioms.

A definition that makes sense would involve partial sums \(x_n\) of this infinite series.
However, we could not just say that the partial sums get progressively closer to a value, because then trivially something like \(\frac{1}{2}, \frac{2}{3}, \frac{3}{4}, \frac{4}{5}, \cdots\) tends to 107, even though they're clearly getting closer.

A more accurate definition would be to state that we can get arbitrarily close (within some given \(\varepsilon\)) to a `limit value' \(c\) by taking some amount of terms \(n\) of this series: \(c - \varepsilon < x_n < c + \varepsilon\).
But this is still wrong: the sequence \(\frac{1}{2}, 10, \frac{2}{3}, 10, \frac{3}{4}, 10, \frac{4}{5}, 10, \cdots\) could then tend to 1 even though every other term is 10.

The best definition would state that the sequence of partial sums would \textit{stay} within \(\varepsilon\) of \(c\) for all \(x_k\) where \(k \geq n\) for some \(n \in \mathbb N\).
In less formal words, for any \(\varepsilon > 0\), \(x_n\) will eventually stay within \(\varepsilon\) of \(c\).
Equivalently, \(\forall \varepsilon > 0, \exists N \in \mathbb N\) such that \(\forall n > N\) we have \(\abs{x_n - c} < \varepsilon\).
