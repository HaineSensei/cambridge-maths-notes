\subsection{Binomial coefficients}
For \(n \in \mathbb N_0\) and \(0 \leq k \leq n\), we can write \(\binom{n}{k}\) for the number of subsets of an \(n\)-set that are of size \(k\).
\[
	\binom{n}{k} = \abs{\left\{ S \subseteq \{ 1, 2, \dots, n \}: \abs{S} = k \right\}}
\]
For example, there are six 2-sets in a 4-set.
There is a formula for this, but generally this definition is a lot easier to use.
Note that \(\binom{n}{0} = 1\), \(\binom{n}{n} = 1\), and \(\binom{n}{1}=n\) where \(n>0\).

Note that \(\binom{n}{0} + \binom{n}{1} + \dots + \binom{n}{n} = 2^n\) as each side counts the number of subsets in an \(n\)-set.
Also:
\begin{enumerate}
	\item \(\binom{n}{k} = \binom{n}{n-k}\) (\(\forall n \in N_0, 0 \leq k \leq n\)).
	      Indeed, specifying which \(k\) members to pick for a subset is equivalent to specifying which \(n-k\) members not to pick.
	\item \(\binom{n}{k} = \binom{n-1}{k-1} + \binom{n-1}{k}\) (\(\forall n \in \mathbb N, 0 < k < n\)).
	      Indeed, the number of \(k\)-subsets of \(\{ 1, 2, \dots, n \}\) without \(n\) is \(\binom{n-1}{k}\).
	      The number of \(k\)-subsets of \(\{ 1, 2, \dots, n \}\) that do contain \(n\) is \(\binom{n-1}{k-1}\) as we must pick the remaining \(k-1\) elements of this new subset.
	      So in total, \(\binom{n-1}{k-1} + \binom{n-1}{k}\) encapsulates both possibilities.
\end{enumerate}
This last point illustrates that Pascal's Triangle will give all the binomial coefficients since it perfectly encapsulates the relationship between a given element of the triangle with two elements from the previous row.
The exact proof follows from the other known properties of the binomial coefficients.

\subsection{Computing binomial coefficients}
\begin{proposition}
	\[
		\binom{n}{k} = \frac{n(n-1)(n-2)\cdots(n-k+1)}{k(k-1)(k-2)\cdots(1)}
	\]
\end{proposition}
\begin{proof}
	The number of ways to name a \(k\)-set is \(n(n-1)(n-2)\cdots(n-k+1)\) because there are \(n\) ways to choose a first element, \(n-1\) ways to choose a second element, and so on.
	We have overcounted the \(k\)-sets, though --- there are \(k(k-1)(k-2)\cdots(1)\) ways to name a given \(k\)-set because you have \(k\) choices for the first element, \(k-1\) choices for the second element, and so on.
	Hence the number of \(k\)-sets in \(\{ 1, 2, \dots, n \}\) is the required result.
\end{proof}
Note that we can also write
\[
	\binom{n}{k} = \frac{n!}{k!(n-k)!}
\]
but this is a very unwieldy formula to use especially by hand, so will be rarely used.
Further, we can make asymptotic approximations using this formula, for example \(\binom{n}{3} \sim \frac{n^3}{6}\) for large \(n\).

\subsection{Binomial theorem}
\begin{theorem}
	For all \(a, b \in \mathbb R, n \in \mathbb N\), we have
	\[
		(a+b)^n = \binom{n}{0}a^n + \binom{n}{1}a^{n-1}b + \binom{n}{2}a^{n-2}b^2 + \dots + \binom{n}{n}b^n
	\]
\end{theorem}
\begin{proof}
	When we expand \((a+b)^n = (a+b)(a+b)\dots(a+b)\), we obtain terms of the form \(a^k b^{n-k}\).
	To get a single term of this form, we must choose \(k\) brackets for which to take the \(a\) value in the expansion, and the other \(n-k\) brackets will take the \(b\) value.
	The number of terms of the form \(a^k b^{n-k}\) for a fixed \(k\) is therefore the amount of ways of choosing \(k\) brackets out of a total of \(n\), which is \(\binom{n}{k}\).
	So
	\[
		(a+b)^n = \sum_{k=0}^n \binom{n}{k}a^k b^{n-k} = \sum_{k=0}^n \binom{n}{n-k}a^k b^{n-k}
	\]
\end{proof}
For example, we can tell that \((1+x)^n\) reduces to
\[
	1 + nx + \frac{1}{2}n(n-1)x^2 + \frac{1}{3!}n(n-1)(n-2)x^3 + \dots + nx^{n-1} + x^n
\]
So when \(x\) is small, a good approximation to \((1+x)^n\) is \(1 + nx\).

\subsection{Inclusion-exclusion theorem}
Given two finite sets \(A\), \(B\), we have
\[
	\abs{A \cup B} = \abs{A} + \abs{B} - \abs{A \cap B}
\]
For three sets, we have
\[
	\abs{A \cup B \cup C} = \abs{A} + \abs{B} + \abs{C} - \abs{A \cap B} - \abs{B \cap C} - \abs{C \cap A} + \abs{A \cap B \cap C}
\]
\begin{theorem}[Inclusion-Exclusion Theorem]
	Let \(S_1, \dots, S_n\) be finite sets.
	Then,
	\[
		\abs{\bigcup_{S \in S_n} S} = \sum_{\abs{A} = 1}\abs{S_A} - \sum_{\abs{A} = 2}\abs{S_A} + \sum_{\abs{A} = 3}\abs{S_A} - \cdots
	\]
	where
	\[
		S_A = \bigcap_{i \in A}S_i
	\]
	and
	\[
		\sum_{\abs{A} = k}
	\]
	is a sum taken over all \(A \subseteq \{ 1, 2, \dots, n \}\) of size \(k\).
\end{theorem}
\begin{proof}
	Let \(x\) be an element of the left hand side.
	We wish to prove that \(x\) is counted exactly once on the right hand side.
	Without loss of generality, let us rename the sets that \(x\) belongs to as \(S_1, S_2, \dots, S_k\).

	Then the number of sets \(A\) with \(\abs{A} = 1\) such that \(x \in S_A\) is \(k\).
	The number of sets \(A\) with \(\abs{A} = 2\) such that \(x \in S_a\) is \(\binom{k}{2}\), since we must choose two of the sets \(S_1, \dots, S_k\), so there are \(\binom{k}{2}\) ways to do this.
	So in general, the amount of \(A\) with \(\abs{A} = r\) with \(x \in S_A\) is just \(\binom{k}{r}\).

	So the number of times \(x\) is counted on the right hand side is
	\[
		k - \binom{k}{2} + \binom{k}{3} - \dots + (-1)^{k+1}\binom{k}{k}
	\]
	But \((1 + (-1))^k\) by the binomial expansion is
	\[
		1 - \binom{k}{1} + \binom{k}{2} - \binom{k}{3} + \dots + (-1)^k\binom{k}{k}
	\]
	So the number of times \(x\) is counted on the right hand side is \(1 - (1 + (-1))^k = 1 - 0 = 1\).
\end{proof}
