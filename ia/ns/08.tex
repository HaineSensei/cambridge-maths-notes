\subsection{Square root of \(-1\)}
\begin{theorem}[Wilson's Theorem]
	Let \(p\) be prime.
	Then \((p-1)!
	\equiv -1\ (p)\).
\end{theorem}
\begin{proof}
	Since this is obviously true for \(p=2\), we will suppose that \(p>2\).
	In \(\mathbb Z_p\), let us consider the list \(1, 2, 3 \cdots (p-1)\).
	We can pair each \(a\) with its inverse \(a^{-1}\) for all \(a \neq a^{-1}\).
	Note that \(a = a^{-1} \iff a^2 = 1\) so in this case \(a=1\) or \(a=-1\).
	So let us now multiply each element together, to get
	\[
		(p-1)!
		= (aa^{-1}) (bb^{-1}) \cdots 1 \cdot -1 = (1) \cdot (1) \cdots 1 \cdot -1 = -1
	\]
\end{proof}

\begin{proposition}
	Let \(p>2\) be prime.
	Then \(-1\) is a square number modulo \(p\) if and only if \(p \equiv 1\ (4)\).
\end{proposition}
\begin{proof}
	If \(p>2\) then \(p\) is odd.
	There are therefore two cases, either \(p \equiv 1\) or \(p \equiv 3\) modulo 4.
	Each case is proven individually.
	\begin{itemize}
		\item (\(p = 4k + 3\)) Suppose that \(x^2 = -1\) in \(\mathbb Z_p\).
		      The only thing we know about powers in modular arithmetic is Fermat's Little Theorem, so we will have to use this.
		      So, \(x^{p-1} = x^{4k+2} = 1\).
		      Therefore, \((x^2)^{2k+1} = 1\).
		      But we know that \(x^2=-1\), and we raise this \(-1\) to an odd power, which is \(-1\).
		      So this is a contradiction.
		\item (\(p = 4k + 1\)) By Wilson's Theorem, we know that \((4k)!
		      = -1\).
		      We intend to show that this is a square number in the world of \(\mathbb Z_p\).
		      We will compare the termwise expansion of \((4k)!
		      \) and \([(2k)!]^2\) on consecutive lines.
		      \begin{alignat*}{9}
			      (4k)!
			                & = 1 &  & \cdot 2 &  & \cdot 3 &  & \cdots (2k) &  & \cdot (2k+1) &  & \cdot (2k+2)  &  & \cdots (4k-1)  &  & \cdot (4k)                   \\
			      [(2k)!]^2 & = 1 &  & \cdot 2 &  & \cdot 3 &  & \cdots (2k) &  & \cdot 1      &  & \cdot 2       &  & \cdots (2k-1)  &  & \cdot (2k)                   \\
			      \intertext{By writing each term as an equivalent negative:}
			                & = 1 &  & \cdot 2 &  & \cdot 3 &  & \cdots (2k) &  & \cdot (-4k)  &  & \cdot (-4k+1) &  & \cdots (-2k-2) &  & \cdot (-2k-1)                \\
			      \intertext{Extracting out the negatives:}
			                & = 1 &  & \cdot 2 &  & \cdot 3 &  & \cdots (2k) &  & \cdot (4k)   &  & \cdot (4k-1)  &  & \cdots (2k+2)  &  & \cdot (2k+1) \cdot (-1)^{2k}
		      \end{alignat*}
		      which is equal to the first line by rearranging.
		      So \([(2k)!]^2 = (4k)!
		      = -1\).
		      So \(-1\) is a square number modulo \(p\).
	\end{itemize}
\end{proof}

\subsection{Solving congruence equations}
Let us try to solve the equation \(7x \equiv 4\ (30)\).
We take a two-phase approach: first, we will find a single solution, and then we will find all of the other solutions.

Since 7 and 30 are coprime, we can use Euclid's algorithm to find a way of expressing 1 in terms of 7 and 30, in particular \(13 \cdot 7 - 3\cdot 30 = 1\).
This allows us to solve \(7y \equiv 1\ (30)\), by setting \(y=13\).
Then, of course, we can multiply both sides by 4: \(7 y\cdot 4 \equiv 4\ (30)\), so \(x = y \cdot 4 = 13 \cdot 4 = 22\).

We can now find other solutions (apart from trivially adding \(30k\)).
Suppose that there exists some other solution \(x'\), i.e.\ \(7x' \equiv 4\ (30)\).
Then \(7x \equiv 7x'\ (30)\).
As 7 is invertible modulo 30, we can simply multiply by the inverse of 7 to give \(x \equiv x'\ (30)\).
So \(x\) is unique modulo 30.
Alternatively, we could solve the equation without any of this working out by noticing that 7 is invertible!
However, this is not very likely to happen in the general case, since it requires that the coefficient of \(x\) is coprime to the modulus.

Now, let's try a different equation, \(10x = 12\ (34)\).
Since 10 is not invertible, we can't do quite the same thing as above.
We can't also just divide the whole thing by 2, there isn't a rule for that in general.
We can, however, move into \(\mathbb Z\) and manipulate the expression there.
\(10x = 12 + 34y\) for some \(y \in \mathbb Z\), so we can divide the equation by 2 to get \(5x = 6 + 17y\), so \(5x = 6\ (17)\) and we can solve from there.

\subsection{Simultaneous congruence}
Is there a solution for the simultaneous congruences
\[
	x \equiv 6\ (17);\quad x \equiv 2\ (19)
\]
17 and 19 are coprime, so congruence mod 17 and congruence mod 19 are independent of each other.
How about
\[
	x \equiv 6\ (34);\quad x \equiv 11\ (36)
\]
In this instance, there is obviously no solution; should \(x\) be even or odd?
We can see that, the smallest amount we can adjust \(x\) by in one equation while retaining congruence in the other equation is \(\HCF(34, 36)\), which is 2.
\begin{theorem}[Chinese Remainder Theorem]
	Let \(u, v\) be coprime.
	Then for any \(a, b\), there exists a value \(x\) such that
	\[
		x \equiv a\ (u);\quad x \equiv b\ (v)
	\]
	and that this value is unique modulo \(uv\).
\end{theorem}
\begin{proof}
	We first prove existence of such an \(x\).
	By Euclid's Algorithm, we have \(su + tv = 1\) for some integers \(s, t\).
	Note that therefore:
	\[
		su \equiv 0\ (u);\quad tv \equiv 0\ (v);\quad su \equiv 1\ (v);\quad tv \equiv 1\ (u);
	\]
	Therefore we can make a linear combination of \(su\) and \(tv\) that is the required size in each congruence, specifically
	\[
		x = (su)b + (tv)a
	\]
	Now we prove that this value \(x\) is unique modulo \(uv\).
	Suppose there was some other solution \(x'\).
	Also, \(x' \equiv x\ (u)\) and \(x' \equiv x\ (v)\).
	So we have \(u\mid (x' - x)\) and \(v\mid (x' - x)\) but as \(u\) and \(b\) are coprime we have \(uv\mid (x' - x)\).
	So \(x\) is unique modulo \(uv\).
\end{proof}
