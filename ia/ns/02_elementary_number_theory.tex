\subsection{The natural numbers}
Each line in a proof must be justified.
So, in number theory, what are you allowed to assume?
We must begin with a set of axioms.
We define that the natural numbers are a set denoted \(\mathbb N\), that contains an element denoted 1, with an operation \(+1\) satisfying:
\begin{enumerate}
	\item \(\forall n \in \mathbb N, n + 1 \neq 1\)
	\item \(\forall m,n \in \mathbb N, m \neq n \implies m+1 \neq n+1\) (together with the previous rule, this captures the idea that all numbers in \(\mathbb N\) are distinct)
	\item For any property \(p(n)\), if \(p(1)\) is true and \(p(n) \implies p(n+1) \ \forall n \in \mathbb N\), then \(p(n) \ \forall n \in \mathbb N\) (induction axiom).
\end{enumerate}

\noindent This list of rules is known as the Peano axioms.
Note that we did not include 0 in this set.
You can show that the list of natural numbers is complete and has no extras (like the rational number \(3.5\)) by specifying \(p(n)=\) `\(n\) is on the list of natural numbers'.

Note that while numbers are defined as, for example, \(1+1+1+1\), we are free to use whatever names we like, e.g.
4 or the hexadecimal number 0xDEADBEEF = 3735928559.

We may also define our own operations, such as \(+2\), which is defined to be \(+1+1\).
In fact, we can define the operation \(+k\) for any \(k \in \mathbb N\) by stating:
\[
	(n+k)+1 = n+(k+1) \quad(\forall n, k \in \mathbb N)
\]
\noindent and using induction to construct the \(+k\) operator for all \(k\).
We can similarly construct multiplication and exponentiation operators for all natural numbers, although this is omitted here.
We can also prove properties on these operators such as associativity, commutativity and distributivity.

We can also define the \(<\) operator as follows: \(a < b \iff \exists k \in \mathbb N \st a + k = b\).
Of course, we can also prove several properties using this rule, such as transitivity, and the fact that \(a \nless a\), which are omitted here.

\subsection{Strong induction}
The induction axiom states that if we know
\begin{itemize}
	\item \(p(1)\) is true, and
	\item \(p(n) \implies p(n+1)\) for any \(n \in \mathbb N\)
\end{itemize}
then we can conclude that \(p(n)\) is true for all \(n \in \mathbb N\).
We can in fact prove a stronger statement using this axiom, known as `strong induction'.
\begin{claim}
	If we know that
	\begin{itemize}
		\item \(p(1)\) is true, and
		\item the fact that \(p(k)\) is true for all \(k < n\) implies that \(p(n)\) is true
	\end{itemize}
	then \(p(n)\) is true for all \(n \in \mathbb N\).
\end{claim}
\begin{proof}
	Consider the predicate \(q(n)\) defined as: `\(p(k)\) is true for all \(k < n\)'.
	Given that \(p(1)\) is true, \(q(1)\) is trivially true since there are no \(k\) below 1.
	Since \(q(n) \implies q(n+1)\), we can use the induction axiom, showing that \(q(n)\) is true for all \(n\), so \(p(n)\) is true for all \(n\).
\end{proof}
This provides a very useful alternative way of looking at induction.
Instead of just considering a process from \(n\) to \(n+1\), we can inject an inductive viewpoint into any proof.
When proving something on the natural numbers, we can always assume that the hypothesis is true for smaller \(n\) than what we are currently using.
This allows us to write very powerful proofs because in the general case we are allowed to refer back to other smaller cases --- but not just \(n-1\), any \(k\) less than \(n\).

We may rewrite the principle of strong induction in the following ways:
\begin{enumerate}
	\item If \(p(n)\) is false for some \(n\), there must be some \(m\) where \(p(m)\) is false and \(p(k)\) is true for all \(k<m\).
	      In other words, if a counterexample exists, there must exist a minimal counterexample.
	\item If \(p(n)\) is true for some \(n\), then there is a smallest \(n\) where \(p(n)\).
	      In other words, if an example exists, there must exist a minimal example.
	      This is known as the `well-ordering principle'.
\end{enumerate}

\subsection{The integers}
The integers \(\mathbb Z\) consist of the set of natural numbers \(\mathbb N\), their additive inverses, and an identity element denoted 0.
In other words, \((\mathbb Z, +)\) is the group generated by \(\mathbb N\) and the addition operator: \(\mathbb Z = \genset{\mathbb N}\).

We define operations in a familiar way, for example \(a < b \iff \exists c \in \mathbb N \st a+c = b\).

\subsection{The rationals}
The rational numbers \(\mathbb Q\) consist of all expressions denoted \(\frac{a}{b}\) where \(a, b \in \mathbb Z\) with \(b \neq 0\); with \(\frac{a}{b}\) regarded as the same as \(\frac{c}{d}\) if and only if \(ad=bc\).

We define, for example,
\[
	\frac{a}{b} + \frac{c}{d} = \frac{ad + bc}{bd}
\]
Note that is important to verify with each operation that it does not matter how you write a given rational number.
For example, \(\frac{1}{2} + \frac{1}{2} = \frac{2}{4} + \frac{3}{6}\).
This means that operations such as \(\frac{a}{b} \to \frac{a^3}{b^2}\) cannot exist because then it would depend on how you write the rational number.
