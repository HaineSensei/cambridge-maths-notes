\subsection{Ordered Pairs}
For any $a, b$, we can form the ordered pair $(a, b)$, where equality is checked component-wise. For sets $A, B$, we can form their product $A \times B = \{ (a, b) : a \in A, b \in B \}$. For example, $\mathbb R^2 = \mathbb R \times \mathbb R$ can be viewed as a plane. We can form other sizes of tuples similarly.

\subsection{Power Sets}
For any set $X$, we can form the power set $\mathcal P(X)$ consisting of all subsets of $X$.
\[ \mathcal P(X) = \{ Y: Y \subseteq X \} \]
For example:
\[ \mathcal P(\{ 1, 2 \}) = \{ \varnothing, \{ 1 \}, \{ 2 \}, \{ 1, 2\} \} \]

\subsection{Russell's Paradox}
For a set $A$, we can always form the set $\{ x \in A: p(x) \}$ for any property $p$. We cannot, however, form the set $\{ x: p(x) \}$. Suppose we could form such a set, then we could form the set $X = \{ x: x \notin x \}$. Now, is $X \in X$? If this is true, then it fails the defining property $x \notin x$. If this is false, then the defining property is true, so it must be in the set. This is a contradiction in both cases.

Similarly, there is no `universal' set $\mathscr E$, meaning $\forall x, x \in \mathscr E$. Otherwise we could form the $X$ above by $\{ x \in \mathscr E: p(x) \}$. To guarantee that a given set exists, we need to obtain it in some way from known sets.

\subsection{Finite Sets}
We will write $\mathbb N_0 = \mathbb N \cup \{ 0 \}$. For $n \in \mathbb N_0$, we can say that a set $A$ has size $n$ if we can write $A = \{ a_1, a_2, \cdots, a_n \}$ where the $a_i$ are distinct. A set is called finite if it has a size $n \in \mathbb N_0$.

Note that a set cannot have size $n$ and size $m$ for $n \neq m$. Suppose that $A$ has size $n$ and size $m$ where $n, m > 0$. Then, removing an element, we obtain a set that has size $n-1$ and $m-1$. By induction on the larger of $n$ and $m$, we will eventually reach a size of both zero and non-zero which is a contradiction.

\begin{proposition}
	A set of size $n$ has exactly $2^n$ subsets.
\end{proposition}
\begin{proof}[Proof 1]
	We may assume that our set is simply $\{ 1, 2, \cdots, n \}$ by relabelling. When constructing a subset $S$ from this set, there are $n$ independent binary choices for whether a given element should be within this subset, since for example either $1 \in S$ or $1 \notin S$ must be true. So there are $2^n$ distinct choices of subset you can make.
\end{proof}
\begin{proof}[Proof 2]
	We will prove this inductively on $n$, noting that $n=0$ is trivial. For any subset $T \subseteq \{ 1, 2, \cdots n-1 \}$, how many $S \subseteq \{ 1, \cdots, n \}$ have $S \cap \{ 1, 2, \cdots n-1 \} = T$? Exactly two: $T$ and $T \cup \{ n \}$. So there are two choices for how to extend this subset to the new element $n$. So the number of subsets is $2 \cdot 2^{n-1} = 2^n$.
\end{proof}
\noindent In some sense Proof 2 is a more `formal' version of Proof 1, using induction rather than intuition. We sometimes say that if $A$ has size $n$, then $\abs{A} = n$, and that $A$ is an $n$-set.

\subsection{Binomial Coefficients}
For $n \in \mathbb N_0$ and $0 \leq k \leq n$, we can write $\binom{n}{k}$ for the number of subsets of an $n$-set that are of size $k$.
\[ \binom{n}{k} = \abs{\left\{ S \subseteq \{ 1, 2, \dots, n \}: \abs{S} = k \right\}} \]
For example, there are six 2-sets in a 4-set. There is a formula for this, but generally this definition is a lot easier to use. Note that $\binom{n}{0} = 1$, $\binom{n}{n} = 1$, and $\binom{n}{1}=n$ where $n>0$.

Note that $\binom{n}{0} + \binom{n}{1} + \dots + \binom{n}{n} = 2^n$ as each side counts the number of subsets in an $n$-set. Also:
\begin{enumerate}
	\item $\binom{n}{k} = \binom{n}{n-k}$ ($\forall n \in N_0, 0 \leq k \leq n$). Indeed, specifying which $k$ members to pick for a subset is equivalent to specifying which $n-k$ members not to pick.
	\item $\binom{n}{k} = \binom{n-1}{k-1} + \binom{n-1}{k}$ ($\forall n \in \mathbb N, 0 < k < n$). Indeed, the number of $k$-subsets of $\{ 1, 2, \dots, n \}$ without $n$ is $\binom{n-1}{k}$. The number of $k$-subsets of $\{ 1, 2, \dots, n \}$ that do contain $n$ is $\binom{n-1}{k-1}$ as we must pick the remaining $k-1$ elements of this new subset. So in total, $\binom{n-1}{k-1} + \binom{n-1}{k}$ encapsulates both possibilities.
\end{enumerate}
This last point illustrates that Pascal's Triangle will give all the binomial coefficients since it perfectly encapsulates the relationship between a given element of the triangle with two elements from the previous row. The exact proof follows from the other known properties of the binomial coefficients.
