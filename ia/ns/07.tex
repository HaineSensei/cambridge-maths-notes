\subsection{Invertibility}
\begin{proposition}
	Let $n \geq 2$. Then every $a \centernot\equiv 0\ (n)$ is invertible modulo $n$ if and only if $(a, n) = 1$. Note that the parenthesis notation means the highest common factor of the parameters. In particular, if $n$ is prime, then all $1 \leq a < n$ are invertible.
\end{proposition}
\begin{proof}
	This first proof uses Euclid's algorithm. If $a$ and $n$ satisfy $(a, n) = 1$ then $ax + ny = 1$ for some $x, y \in \mathbb Z$. So $ax = 1 - ny$, so $ax \equiv 1\ (n)$. So $x$ is the inverse of $a$.
\end{proof}
\begin{proof}
	This alternate proof only works for $n=p$ where $p$ is a prime; our whole proof lies entirely within $\mathbb Z_p$. Consider $0a, 1a, 2a, \cdots, (p-1)a$. Take two numbers $i, j$ between 0 and $p-1$, then consider the condition $ia = ja$. This implies that $(i - j)a = 0$, but $a \neq 0$, so $i=j$. So this list $0a, 1a, \cdots$ contains all distinct elements, all of which must be between 0 and $p-1$. Therefore, by the pigeonhole principle, one of these elements must be equal to 1. Therefore there exists an inverse for $a$.
\end{proof}

\subsection{Euler's Totient Function}
\begin{definition}
	Let $\varphi(n)$ be the amount of natural numbers less than or equal to $n$ that are coprime to $n$.
\end{definition}
Here are some examples.
\begin{itemize}
	\item If $p$ is prime, then $\varphi(p) = p - 1$ since all naturals less than $p$ are coprime to it.
	\item $\varphi(p^2) = p^2 - p$ because there are $p$ numbers in this range who shares the common factor $p$ with $p^2$, specifically the numbers $p, 2p, 3p, \cdots, (p-1)p, p^2$.
	\item If $a, b$ are coprime, $\varphi(ab) = ab - a - b + 1$. There are $ab$ numbers in total to pick from. There are $a$ multiples of $b$ and $b$ multiples of $a$, and since we discounted $ab$ itself twice we need to count it again. Note that $\varphi(ab) = \varphi(a)\varphi(b)$.
\end{itemize}

\begin{theorem}[Fermat's Little Theorem]
	Let $p$ be a prime. Then in $\mathbb Z_p$, $a \neq 0 \implies a^{p-1} = 1$.
\end{theorem}
\noindent This is actually a special case of the following theorem:
\begin{theorem}[Fermat-Euler Theorem]
	Let $n \geq 2$. Then in $\mathbb Z_n$, any unit $a$ satisfies $a^{\varphi(n)} = 1$.
\end{theorem}
\begin{proof}
	Let the set of units $X \in \mathbb Z_n = \{ x_1, x_2, \cdots, x_{\varphi(n)} \}$. Consider multiplying each unit by $a$. We have $Y = \{ ax_1, ax_2, \cdots, ax_{\varphi(n)} \}$. Since $a$ is invertible, this set is comprised of distinct elements. Further, since they are all products of units, they are all units. So $Y$ is a list of $\varphi(n)$ distinct units, so this list must be equal to $X$. Now, since the lists are the same, the product of all their elements must be the same. So $\prod X = \prod Y = a^{\varphi(n)}\prod X$. We can cancel the factor of $\prod X$ because it is a product of invertibles, leaving $1 = a^{\varphi(n)}$ as required.
\end{proof}
If alternatively we wanted to prove this just for $p$ prime, then we could replace $\varphi(n)$ with $p-1$, and $\prod X$ with $(p-1)!$.

\subsection{Simple Quadratic}
\begin{lemma}
	Let $p$ be prime. Then in $\mathbb Z_p$, $x^2 = 1$ has solutions $1$ and $-1$ only.
\end{lemma}
\begin{note}
	In $\mathbb Z_8$, for example, we have $1^2 = 3^2 = 5^2 = 7^2 = 1$, so obviously this does not hold in the general case.
\end{note}
\begin{proof}
	$x^2 = 1$ implies that $(x-1)(x+1) = 0$. Because of the $p\mid ab\implies (p\mid a) \lor (p\mid b)$ lemma, we know that $(x-1) = 0$ or $(x+1) = 0$, so $-1$ and $1$ are the only solutions.
\end{proof}
