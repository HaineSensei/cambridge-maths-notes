\subsection{Sets}
A set is any* collection of mathematical objects. $(\forall x, x \in A \iff x \in B) \iff (A = B)$. In words, two sets which have the same members are considered to be the same; order of members is not important in a set. There is no `multiple membership' of a set, $\{ a, a \} = \{ a \}$.

\subsection{Subsets}
Given a set $A$ and a property $p(x)$, we can form $\{ x \in A: p(x) \}$; the subset of all members of $A$ with property $p$. This is sometimes called the `subset selection' rule or axiom. We can say that $B$ is a subset of $A$ if $\forall x, x \in B \implies x \in A$, written $B \subseteq A$. Further, $A = B \iff A \subseteq B, B \subseteq A$.

\subsection{Unions and Intersections}
Given sets $A$ and $B$, we can form their union $A \cup B = \{ x: x \in A \lor x \in B \}$. We can also form their intersection $A \cap B = \{ x: x \in A \wedge x \in B \}$. If $A \cap B = \varnothing$, we say $A$ and $B$ are disjoint. Note that we could consider $A \cap B$ as a special case of subset selection; the subset of $A$ with the property that the element is in $B$. Therefore, $A \cap B \subseteq A$, and $A \cap B \subseteq B$. We define the set difference $A \setminus B = \{ x \in A: x \notin B \}$.

Note that $\cap$ and $\cup$ are commutative and associative. Also, $\cup$ is distributive over $\cap$, and $\cap$ is distributive over $\cup$. For example, let us prove that $A \cap (B \cup C) = (A \cap B) \cup (A \cap C)$.
\begin{itemize}
	\item (LHS $\subseteq$ RHS) Given $x \in A \cap (B \cup C)$, we have $x \in A$ and also either $x \in B$ or $x \in C$. If $x \in B$ then $x \in A \cap B$ so $x \in (A \cap B) \cup (A \cap C)$; and vice versa for $C$.
	\item (RHS $\subseteq$ LHS) Given $x \in (A \cap B) \cup (A \cap C)$, either $x \in A \cap B$ or $x \in A \cap C$. If $x \in A \cap B$ then $x \in A$ and $x \in B \cup C$ as required; and vice versa for the other case.
\end{itemize}
As the union is associative, we can have bigger unions of more sets. For example, if we let $A_n = \{ n^2, n^3 \}$ for each $n \in \mathbb N$, the infinite union
\[ A_1 \cup A_2 \cup A_3 \cup \cdots = \bigcup_{n=1}^\infty A_n = \bigcup_{n \in \mathbb N} A_n = \{ x \in N: x \text{ is a square or a cube} \} \]
When we use the $n \in \mathbb N$ on the large union symbol, we call $\mathbb N$ the `index set'. Note that the infinite union is not defined as a limit of finite unions; it is simply defined using set comprehension. In general, given a set $I$, and sets $A_i$, $i \in I$, we can form
\[ \bigcup_{i \in I}A_i = \{ x: \exists i \in I, x \in A_i \} \]
and
\[ \bigcap_{i \in I}A_i = \{ x: \forall i \in I, x \in A_i \} \]
Note that we cannot form an intersection when $I = \varnothing$, as will be explained later.
