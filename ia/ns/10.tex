\subsection{Immediate Remarks}
There are some notable immediate remarks about the definitions of the reals.
\begin{itemize}
	\item We can contain the rationals inside the reals: \(\mathbb Q \subset \mathbb R\)
	\item The least upper bound axiom is false in \(\mathbb Q\), which is why it's so important in \(\mathbb R\).
	\item WHy did we specify `non-empty' and `bounded above' in the least upper bound axiom? Of course, if a set is not bounded above, then it has no upper bound, so clearly it can have no least upper bound. If a set is empty, then every real is an upper bound for this set, and as there is no least real number, there is no least upper bound.
	\item It is possible to construct \(\mathbb R\) out of \(\mathbb Q\), and check that the above axioms hold. However, this is a rare example where the construction of \(\mathbb R\) is complicated and irrelevant, so it is not covered here.
\end{itemize}

\subsection{Axiom of Archimedes}
The reals do not contain infinitely big or infinitesimally small elements.
\begin{proposition}
	\(\mathbb N\) is not bounded above in \(\mathbb R\).
\end{proposition}
\begin{proof}
	If there were some upper bound \(c = \sup \mathbb N\), then \(c-1\) is clearly not an upper bound for \(\mathbb N\). So there exists some natural number \(n\) such that \(n > c-1\). But then clearly \(n+1 \in \mathbb N > c\) contradicting the existence of this upper bound.
\end{proof}
\begin{corollary}
	For each \(t \in \mathbb R > 0\), \(\exists n \in \mathbb N\) such that \(\frac{1}{n} < t\).
\end{corollary}
\begin{proof}
	We have some \(n \in \mathbb N\) with \(n > \frac{1}{t}\) by the above proposition. So \(\frac{1}{n} < t\).
\end{proof}

\subsection{Examples of Sets and Least Upper Bounds}
Note that a common way to write `least upper bound' is the word supremum, denoted \(\sup S\).
\begin{enumerate}
	\item Let \(S = \{ x \in \mathbb R: 0 \leq x \leq 1 \} = [0, 1]\). The least upper bound of \(S\) is 1, because:
	      \begin{itemize}
		      \item 1 is an upper bound for \(S\); \(\forall x \in S, x\leq1 \); and
		      \item Every upper bound \(y\) must have \(y \geq 1\) because \(1 \in S\).
	      \end{itemize}
	\item Let \(S = \{ x \in \mathbb R: 0 < x < 1 \} = (0, 1)\). \(\sup S = 1\) because:
	      \begin{itemize}
		      \item 1 is an upper bound for \(S\); \(\forall x \in S, x \leq 1\); and
		      \item No upper bound \(c\) has \(c<1\). Indeed, certainly \(c>0\) (\(c > \frac{1}{2}\) since \(\frac{1}{2} \in S\)). So if \(c<1\), then \(0<c<1\), so the number \(\frac{1+c}{2} \in S\) and is larger than \(c\), so it is not an upper bound.
	      \end{itemize}
	\item Let \(S = \{ 1 - \frac{1}{n}: n \in \mathbb N \}\). \(\sup S = 1\) because:
	      \begin{itemize}
		      \item 1 is clearly an upper bound.
		      \item Let us suppose \(c < 1\) is an upper bound. Then \(\forall n \in \mathbb N, 1 - \frac{1}{n} < c\) so \(1 - c < \frac{1}{n}\). From the corollary of the Axiom of Archimedes above, this is a contradiction.
	      \end{itemize}
\end{enumerate}
