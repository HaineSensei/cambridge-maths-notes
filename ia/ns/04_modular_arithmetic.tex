\subsection{Introduction}
In modular arithmetic, we need to prove that things like addition and multiplication are valid.
In order to do this, we need to show that if \(a \equiv a' \mod n\) and \(b \equiv b' \mod n\) then, for example, \(ab \equiv a'b'\).
We can prove these statements trivially by writing \(a' = a + kn\) where \(k\) is some integer, then evaluating the left and right hand sides in \(\mathbb Z\).

Many rules of arithmetic are inherited from \(\mathbb Z\); for example, addition is commutative.
This is easy to realise: to prove that \(a + b = b + a\) in \(\mathbb Z_n\) it is sufficient to prove the statement is true in the whole of \(\mathbb Z\).

As another example, we can transform the unique prime factorisation lemma into \(\mathbb Z_p\).
In \(\mathbb Z_p\) where \(p\) is prime,
\[
	ab = 0 \implies (a = 0) \lor (b = 0)
\]
In general, \(\mathbb Z_p\) where \(p\) is prime is a very well behaved and convenient-to-use subset of \(\mathbb Z\).

\subsection{Inverses}
For any \(a, b \in \mathbb Z_n\), \(b\) is an inverse of \(a\) if \(ab=1\).
Note that unlike in group theory, it is not necessarily the case that all elements will have inverses.
For example, in \(\mathbb Z_{10}\), the elements 3 and 7 are inverses, but 4 has no inverse.
Note that:
\begin{itemize}
	\item Invertible integers are cancellable.
	      For example, \(ab=ac \implies b=c\) if \(a\) is invertible (by left-multiplying by its inverse).
	\item In general, you cannot simply cancel an integer multiple in the realm of modular arithmetic.
	      For example \(4 \cdot 5 = 2 \cdot 5 \centernot\implies 4 = 2\).
	\item Invertible numbers are also called `units'.
\end{itemize}

\subsection{Invertibility}
\begin{proposition}
	Let \(n \geq 2\).
	Then every \(a \centernot\equiv 0\ (n)\) is invertible modulo \(n\) if and only if \((a, n) = 1\).
	Note that the parenthesis notation means the highest common factor of the parameters.
	In particular, if \(n\) is prime, then all \(1 \leq a < n\) are invertible.
\end{proposition}
\begin{proof}
	This first proof uses Euclid's algorithm.
	If \(a\) and \(n\) satisfy \((a, n) = 1\) then \(ax + ny = 1\) for some \(x, y \in \mathbb Z\).
	So \(ax = 1 - ny\), so \(ax \equiv 1\ (n)\).
	So \(x\) is the inverse of \(a\).
\end{proof}
\begin{proof}
	This alternate proof only works for \(n=p\) where \(p\) is a prime; our whole proof lies entirely within \(\mathbb Z_p\).
	Consider \(0a, 1a, 2a, \cdots, (p-1)a\).
	Take two numbers \(i, j\) between 0 and \(p-1\), then consider the condition \(ia = ja\).
	This implies that \((i - j)a = 0\), but \(a \neq 0\), so \(i=j\).
	So this list \(0a, 1a, \cdots\) contains all distinct elements, all of which must be between 0 and \(p-1\).
	Therefore, by the pigeonhole principle, one of these elements must be equal to 1.
	Therefore there exists an inverse for \(a\).
\end{proof}

\subsection{Euler's totient function}
\begin{definition}
	Let \(\varphi(n)\) be the amount of natural numbers less than or equal to \(n\) that are coprime to \(n\).
\end{definition}
Here are some examples.
\begin{itemize}
	\item If \(p\) is prime, then \(\varphi(p) = p - 1\) since all naturals less than \(p\) are coprime to it.
	\item \(\varphi(p^2) = p^2 - p\) because there are \(p\) numbers in this range who shares the common factor \(p\) with \(p^2\), specifically the numbers \(p, 2p, 3p, \cdots, (p-1)p, p^2\).
	\item If \(a, b\) are coprime, \(\varphi(ab) = ab - a - b + 1\).
	      There are \(ab\) numbers in total to pick from.
	      There are \(a\) multiples of \(b\) and \(b\) multiples of \(a\), and since we discounted \(ab\) itself twice we need to count it again.
	      Note that \(\varphi(ab) = \varphi(a)\varphi(b)\).
\end{itemize}

\subsection{Fermat's little theorem and Fermat-Euler theorem}
\begin{theorem}
	Let \(p\) be a prime.
	Then in \(\mathbb Z_p\), \(a \neq 0 \implies a^{p-1} = 1\).
\end{theorem}
\noindent This is actually a special case of the following theorem:
\begin{theorem}[Fermat-Euler Theorem]
	Let \(n \geq 2\).
	Then in \(\mathbb Z_n\), any unit \(a\) satisfies \(a^{\varphi(n)} = 1\).
\end{theorem}
\begin{proof}
	Let the set of units \(\mathbb Z_n \supset X = \{ x_1, x_2, \cdots, x_{\varphi(n)} \}\).
	Consider multiplying each unit by \(a\).
	We have \(Y = \{ ax_1, ax_2, \cdots, ax_{\varphi(n)} \}\).
	Since \(a\) is invertible, this set is comprised of distinct elements.
	Further, since they are all products of units, they are all units.
	So \(Y\) is a list of \(\varphi(n)\) distinct units, so this list must be equal to \(X\).
	Now, since the lists are the same, the product of all their elements must be the same.
	So \(\prod X = \prod Y = a^{\varphi(n)}\prod X\).
	We can cancel the factor of \(\prod X\) because it is a product of invertibles, leaving \(1 = a^{\varphi(n)}\) as required.
\end{proof}
If alternatively we wanted to prove this just for \(p\) prime, then we could replace \(\varphi(n)\) with \(p-1\), and \(\prod X\) with \((p-1)!
\).

\subsection{Square roots of one}
\begin{lemma}
	Let \(p\) be prime.
	Then in \(\mathbb Z_p\), \(x^2 = 1\) has solutions \(1\) and \(-1\) only.
\end{lemma}
\begin{note}
	In \(\mathbb Z_8\), for example, we have \(1^2 = 3^2 = 5^2 = 7^2 = 1\), so obviously this does not hold in the general case.
\end{note}
\begin{proof}
	\(x^2 = 1\) implies that \((x-1)(x+1) = 0\).
	Because of the \(p\mid ab\implies (p\mid a) \lor (p\mid b)\) lemma, we know that \((x-1) = 0\) or \((x+1) = 0\), so \(-1\) and \(1\) are the only solutions.
\end{proof}

\subsection{Square roots of negative one}
\begin{theorem}[Wilson's Theorem]
	Let \(p\) be prime.
	Then \((p-1)!
	\equiv -1\ (p)\).
\end{theorem}
\begin{proof}
	Since this is obviously true for \(p=2\), we will suppose that \(p>2\).
	In \(\mathbb Z_p\), let us consider the list \(1, 2, 3 \cdots (p-1)\).
	We can pair each \(a\) with its inverse \(a^{-1}\) for all \(a \neq a^{-1}\).
	Note that \(a = a^{-1} \iff a^2 = 1\) so in this case \(a=1\) or \(a=-1\).
	So let us now multiply each element together, to get
	\[
		(p-1)!
		= (aa^{-1}) (bb^{-1}) \cdots 1 \cdot -1 = (1) \cdot (1) \cdots 1 \cdot -1 = -1
	\]
\end{proof}

\begin{proposition}
	Let \(p>2\) be prime.
	Then \(-1\) is a square number modulo \(p\) if and only if \(p \equiv 1\ (4)\).
\end{proposition}
\begin{proof}
	If \(p>2\) then \(p\) is odd.
	There are therefore two cases, either \(p \equiv 1\) or \(p \equiv 3\) modulo 4.
	Each case is proven individually.
	\begin{itemize}
		\item (\(p = 4k + 3\)) Suppose that \(x^2 = -1\) in \(\mathbb Z_p\).
		      The only thing we know about powers in modular arithmetic is Fermat's Little Theorem, so we will have to use this.
		      So, \(x^{p-1} = x^{4k+2} = 1\).
		      Therefore, \((x^2)^{2k+1} = 1\).
		      But we know that \(x^2=-1\), and we raise this \(-1\) to an odd power, which is \(-1\).
		      So this is a contradiction.
		\item (\(p = 4k + 1\)) By Wilson's Theorem, we know that \((4k)!
		      = -1\).
		      We intend to show that this is a square number in the world of \(\mathbb Z_p\).
		      We will compare the termwise expansion of \((4k)!
		      \) and \([(2k)!]^2\) on consecutive lines.
		      \begin{alignat*}{9}
			      (4k)!
			                & = 1 &  & \cdot 2 &  & \cdot 3 &  & \cdots (2k) &  & \cdot (2k+1) &  & \cdot (2k+2)  &  & \cdots (4k-1)  &  & \cdot (4k)                   \\
			      [(2k)!]^2 & = 1 &  & \cdot 2 &  & \cdot 3 &  & \cdots (2k) &  & \cdot 1      &  & \cdot 2       &  & \cdots (2k-1)  &  & \cdot (2k)                   \\
			      \intertext{By writing each term as an equivalent negative:}
			                & = 1 &  & \cdot 2 &  & \cdot 3 &  & \cdots (2k) &  & \cdot (-4k)  &  & \cdot (-4k+1) &  & \cdots (-2k-2) &  & \cdot (-2k-1)                \\
			      \intertext{Extracting out the negatives:}
			                & = 1 &  & \cdot 2 &  & \cdot 3 &  & \cdots (2k) &  & \cdot (4k)   &  & \cdot (4k-1)  &  & \cdots (2k+2)  &  & \cdot (2k+1) \cdot (-1)^{2k}
		      \end{alignat*}
		      which is equal to the first line by rearranging.
		      So \([(2k)!]^2 = (4k)!
		      = -1\).
		      So \(-1\) is a square number modulo \(p\).
	\end{itemize}
\end{proof}

\subsection{Solving congruence equations}
Let us try to solve the equation \(7x \equiv 4\ (30)\).
We take a two-phase approach: first, we will find a single solution, and then we will find all of the other solutions.

Since 7 and 30 are coprime, we can use Euclid's algorithm to find a way of expressing 1 in terms of 7 and 30, in particular \(13 \cdot 7 - 3\cdot 30 = 1\).
This allows us to solve \(7y \equiv 1\ (30)\), by setting \(y=13\).
Then, of course, we can multiply both sides by 4: \(7 y\cdot 4 \equiv 4\ (30)\), so \(x = y \cdot 4 = 13 \cdot 4 = 22\).

We can now find other solutions (apart from trivially adding \(30k\)).
Suppose that there exists some other solution \(x'\), i.e.\ \(7x' \equiv 4\ (30)\).
Then \(7x \equiv 7x'\ (30)\).
As 7 is invertible modulo 30, we can simply multiply by the inverse of 7 to give \(x \equiv x'\ (30)\).
So \(x\) is unique modulo 30.
Alternatively, we could solve the equation without any of this working out by noticing that 7 is invertible!
However, this is not very likely to happen in the general case, since it requires that the coefficient of \(x\) is coprime to the modulus.

Now, let's try a different equation, \(10x = 12\ (34)\).
Since 10 is not invertible, we can't do quite the same thing as above.
We can't also just divide the whole thing by 2, there isn't a rule for that in general.
We can, however, move into \(\mathbb Z\) and manipulate the expression there.
\(10x = 12 + 34y\) for some \(y \in \mathbb Z\), so we can divide the equation by 2 to get \(5x = 6 + 17y\), so \(5x = 6\ (17)\) and we can solve from there.

\subsection{Chinese remainder theorem}
Is there a solution for the simultaneous congruences
\[
	x \equiv 6\ (17);\quad x \equiv 2\ (19)
\]
17 and 19 are coprime, so congruence mod 17 and congruence mod 19 are independent of each other.
How about
\[
	x \equiv 6\ (34);\quad x \equiv 11\ (36)
\]
In this instance, there is obviously no solution; should \(x\) be even or odd?
We can see that, the smallest amount we can adjust \(x\) by in one equation while retaining congruence in the other equation is \(\HCF(34, 36)\), which is 2.
\begin{theorem}
	Let \(u, v\) be coprime.
	Then for any \(a, b\), there exists a value \(x\) such that
	\[
		x \equiv a\ (u);\quad x \equiv b\ (v)
	\]
	and that this value is unique modulo \(uv\).
\end{theorem}
\begin{proof}
	We first prove existence of such an \(x\).
	By Euclid's Algorithm, we have \(su + tv = 1\) for some integers \(s, t\).
	Note that therefore:
	\[
		su \equiv 0\ (u);\quad tv \equiv 0\ (v);\quad su \equiv 1\ (v);\quad tv \equiv 1\ (u);
	\]
	Therefore we can make a linear combination of \(su\) and \(tv\) that is the required size in each congruence, specifically
	\[
		x = (su)b + (tv)a
	\]
	Now we prove that this value \(x\) is unique modulo \(uv\).
	Suppose there was some other solution \(x'\).
	Also, \(x' \equiv x\ (u)\) and \(x' \equiv x\ (v)\).
	So we have \(u\mid (x' - x)\) and \(v\mid (x' - x)\) but as \(u\) and \(b\) are coprime we have \(uv\mid (x' - x)\).
	So \(x\) is unique modulo \(uv\).
\end{proof}

\subsection{RSA encryption}
A practical use of number theory is RSA encryption, which is an asymmetric encryption protocol that allows encryption by using a public and private key pair.
We will begin by first choosing two large distinct primes \(p\) and \(q\).
By large, we mean primes that are hundreds of digits long; in practice, these primes are between around 512 bits and 2048 bits long when represented in binary.
Let \(n=pq\), and pick a `coding exponent' \(e\).
Our message that we want to send must be an element of \(\mathbb Z_n\), so if it is not representable in this form we must break it apart into several smaller messages, or perhaps use RSA to share some kind of small symmetric key for another encryption algorithm.
Let this message be \(x\), so \(x < n\).

To encode \(x\), we raise it to the power \(e\) in \(\mathbb Z_n\).
To efficiently compute large powers of \(x\), we can use a repeated squaring technique.
For example, we can find \(x, x^2, x^4, x^8, x^{16}\) through repeated squaring, and then for example we can calculate \(x^{19} = x^{16} x^{2} x^{1}\).

To decode \(x^e\), we ideally want some number \(d\) such that \((x^e)^d = x\).
By the Fermat-Euler Theorem, we have \(x^{\varphi(n)} = 1\), so clearly \(x^{k\varphi(n) + 1} = x\).
In other words, we want \(ed \equiv 1 \mod \varphi(n)\).
By running Euclid's algorithm on \(e\) and \(\varphi(n)\), we can find such a \(d\).
Note that this requires \(e\) and \(\varphi(n)\) to be coprime; in practice we would choose \(e\) after we have chosen \(n\) such that this is the case.

Now, we can see that to encode a message, all you need is \(n\) and \(e\).
However, to decode, you need to also know \(d\), which means you need to know \(\varphi(n) = \varphi(pq) = pq - p - q + 1\) which requires that you know the original \(p\) and \(q\).
If we pick sufficiently large \(p\) and \(q\), our \(n\) will be so big as to be almost impossible to factorise in any decent length of time.
So we can publish \(n\) and \(e\) as our public key, and anyone may use these numbers to encrypt a message that then only we can decode.
