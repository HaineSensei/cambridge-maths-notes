\subsection{Countability of Power Sets}
\begin{theorem}
	The power set \(\mathcal P(\mathbb N)\) is uncountable.
\end{theorem}
\begin{proof}
	Suppose the subsets of \(\mathbb N\) are listed as \(S_1, S_2, S_3, \dots\) then we want to construct another set \(S\) that is not equal to any of the other sets \(S_i\). So for each set \(S_i\), we must ensure that \(S\) and \(S_i\) differ for at least one value. An easy way to do this is to include the number \(i\) in the subset if \(S_i\) does not contain the number, and to exclude \(i\) if \(i \in S_i\). Then \(S\) differs from \(S_i\) at position \(i\). This is the same logic as the diagonal argument above. We have:
	\[ S = \{ n \in \mathbb N : n \notin S_n \} \]
	So \(S\) is not on the list \(S_1, S_2, S_3, \dots\) no matter what way we choose to list the elements, so \(\mathcal P(\mathbb N)\) is uncountable.
\end{proof}
\begin{remark}
	Alternatively, we could just inject \((0, 1)\) into \(\mathcal P(\mathbb N)\). For example, consider \(x \in (0, 1)\) represented as \(0.x_1x_2x_3x_4\dots\) in binary where the \(x_1, x_2, \dots\) are zero or one (not ending with an infinite amount of 1s). We can convert this \(x\) into a subset of \(\mathbb N\) by considering the set \(\{ n \in \mathbb N : x_n = 1 \}\). Then the uncountability follows.
\end{remark}
In fact, our proof of this theorem shows the following.
\begin{theorem}
	For any set \(X\), there is no bijection from \(X\) to the power set \(\mathcal P(X)\).
\end{theorem}
For example, \(\mathbb R\) does not biject with \(\mathcal P(\mathbb R)\). The proof in fact will show that there is no surjection from \(X\) to its power set; i.e. the power set is `larger' than \(X\).
\begin{proof}
	Given any function \(f\colon X \to \mathcal P(X)\), we will show \(f\) is not surjective. Let \(S = \{ x \in X: x \notin f(x) \}\). Then \(S\) does not belong to the image of \(f\) because they differ at element \(x\); for all \(x\) we have \(S \neq f(x)\).
\end{proof}
\begin{remark}
	Note that:
	\begin{enumerate}
		\item This is similar in some sense to Russell's paradox.
		\item This theorem gives another proof that there is no universal set \(\mathscr E\), since its power set \(\mathcal P(\mathscr E) \subseteq \mathscr E\). But of course, there is always a surjection from a set to a subset. This is a contradiction.
	\end{enumerate}
\end{remark}

\subsection{Disjoint Real Intervals}
This is an example on countability. Let \(A_i, i \in I\) be a family of open, pairwise disjoint intervals. Must this family be countable? Note that it is not as simple as just listing from left to right, for example consider
\[ \left(\frac{1}{2}, 1\right), \left(\frac{1}{3}, \frac{1}{2}\right), \left(\frac{1}{4}, \frac{1}{3}\right), \dots, (-1, 0) \]
Then the leftmost interval is \((-1, 0)\), but there is no `next interval' just after it. Also consider
\[ \left( 0, \frac{1}{2} \right), \left( \frac{1}{2}, \frac{2}{3} \right), \left( \frac{2}{3}, \frac{3}{4} \right), \dots, (1, 2) \]
Then we can list the first infinitely many intervals, but we will never reach \((1, 2)\). The answer turns out to be true; the family is countable. Here are two important proofs.
\begin{proof}[Proof 1]
	Each interval \(A_i\) contains a rational number \(a_i\). The rationals \(\mathbb Q\) are countable. So let us just list the \(a_i\). The family is therefore countable.
\end{proof}
\begin{proof}[Proof 2]
	\(\{ i \in I: A_i \text{ has length } \leq 1\}\) is certainly countable, since it injects into \(\mathbb Z\) (here, as all \(A_i\) contain some integer). Further, \(\left\{ i \in I: A_i \text{ has length } \leq \frac{1}{2} \right\}\) is countable for the same reason. Essentially, for all \(n\), \(\left\{ i \in I: A_i \text{ has length } \leq \frac{1}{n} \right\}\) is countable. We have written all the intervals as a countable union (over \(n\)) of countable sets.
\end{proof}

\subsection{Summary of Countability}
To show a set \(X\) is uncountable:
\begin{enumerate}
	\item Run a diagonal argument; or
	\item Inject an uncountable set into \(X\)
\end{enumerate}
To show a set \(X\) is countable:
\begin{enumerate}
	\item List all the elements (usually fiddly); or
	\item Inject \(X\) into \(\mathbb N\) (or another countable set); or
	\item Express \(X\) as a countable union of countable sets (usually the best); or
	\item If \(X\) is `in' or `near' \(\mathbb R\), consider \(\mathbb Q\) (see Proof 2 above).
\end{enumerate}
