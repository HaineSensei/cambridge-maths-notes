\subsection{Invertibility}
We say that a function \(f\colon A \to B\) is invertible if there exists some \(g\colon B \to A\) such that \(g \circ f = 1_A\) and \(f \circ g = 1_B\). For example \(f\colon \mathbb R \to \mathbb R\) given by \(f(x)=2x+1\) has inverse \(g\colon \mathbb R \to \mathbb R\) given by \(g(x)=\frac{x-1}{2}\). We can prove that this is correct by showing for all real numbers that \((g\circ f)(x) = x\) and vice versa as required.

As an example, consider \(f_0\colon \mathbb N \to \mathbb N\) given by \(f_0(x)=x+1\), and \(f_1\colon \mathbb N \to \mathbb N\) given by \(f_1(x) = x-1\) if \(x\neq 1\) and 1 if \(x=1\). \(f_1\circ f_0 = 1_{\mathbb N}\) but \(f_0\circ f_1 \neq 1_{\mathbb N}\) because they disagree at 1. So we must check inverses both ways.

In fact, if \(f\colon A \to B\) is invertible if and only if it is a bijection.
\begin{itemize}
	\item (forward implication) Let \(g\) be the inverse of \(f\). It is surjective because \(\forall b \in B\), we have \(b=f(g(b))\). It is injective because given two elements \(a,a'\) such that \(f(a) = f(a')\), we have \(g(f(a)) = g(f(a')) = a = a'\) as required. So it is bijective.
	\item (backward implication) Let \(g(b)\) be the unique point \(a \in A\) with \(f(a) = b\) for all \(b \in B\).
\end{itemize}

\subsection{Relations}
A relation on a set \(X\) is a subset of \(R \subseteq X \times X\). We usually write \(aRb\) to denote \((a, b) \in R\). Here are some examples.
\begin{enumerate}
	\item On \(\mathbb N\), \(aRb\) if \(a \equiv b\ (5)\). For example, \(2R12\) but not \(2R11\).
	\item On \(\mathbb N\), \(aRb\) if \(a \mid b\).
	\item On \(\mathbb N\), \(aRb\) if \(a \neq b\).
	\item On \(\mathbb N\), \(aRb\) if \(a=b \pm 1\).
	\item On \(\mathbb N\), \(aRb\) if \(\abs{a-b} \leq 2\).
	\item On \(\mathbb N\), \(aRb\) if either \(a, b \leq 6\) or \(a, b > 6\).
\end{enumerate}
A relation may have a number of important properties:
\begin{itemize}
	\item (reflexive) If \(\forall x \in X\), \(xRx\), e.g. examples 1, 2, 5, 6.
	\item (symmetric) If \(\forall x, y \in X\), \(xRy \implies yRx\), e.g. examples 1, 3, 4, 5, 6.
	\item (transitive) If \(\forall x, y, z \in X\), \(xRy, yRz \implies xRz\), e.g. examples 1, 2, 6.
\end{itemize}
An equivalence relation is a relation that is reflexive, symmetric and transitive. Examples 1, 6 above are equivalence relations. Here are some more examples.
\begin{enumerate}
	\item On \(\mathbb N\), \(xRy\) if \(x=y\).
	\item Considering a partition of set \(X\) into subsets \(C_1, C_2, \dots, i \in I\) where the \(C_i\) are non-empty and disjoint, and their union is \(X\). Then consider the relation \(aRb\) if \(\exists i\) such that \(a \in C_i\) and \(b \in C_i\). \(aRb\) is an equivalence relation. In fact, all equivalence relations can be considered to be in this form; we will prove this shortly.
\end{enumerate}
For an equivalence relation \(R\) on a set \(X\), and \(x \in X\), we define the equivalence class \([x] = \{ y \in X: y R x \}\). In the first example 1 above, \([2] = \{ y \in \mathbb N : y \equiv 2\ (5) \}\).
