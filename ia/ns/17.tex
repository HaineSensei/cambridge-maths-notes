\subsection{Computing Binomial Coefficients}
\begin{proposition}
	\[
		\binom{n}{k} = \frac{n(n-1)(n-2)\cdots(n-k+1)}{k(k-1)(k-2)\cdots(1)}
	\]
\end{proposition}
\begin{proof}
	The number of ways to name a \(k\)-set is \(n(n-1)(n-2)\cdots(n-k+1)\) because there are \(n\) ways to choose a first element, \(n-1\) ways to choose a second element, and so on.
	We have overcounted the \(k\)-sets, though --- there are \(k(k-1)(k-2)\cdots(1)\) ways to name a given \(k\)-set because you have \(k\) choices for the first element, \(k-1\) choices for the second element, and so on.
	Hence the number of \(k\)-sets in \(\{ 1, 2, \dots, n \}\) is the required result.
\end{proof}
Note that we can also write
\[
	\binom{n}{k} = \frac{n!}{k!(n-k)!}
\]
but this is a very unwieldy formula to use especially by hand, so will be rarely used.
Further, we can make asymptotic approximations using this formula, for example \(\binom{n}{3} \sim \frac{n^3}{6}\) for large \(n\).

\subsection{Binomial Theorem}
\begin{theorem}
	For all \(a, b \in \mathbb R, n \in \mathbb N\), we have
	\[
		(a+b)^n = \binom{n}{0}a^n + \binom{n}{1}a^{n-1}b + \binom{n}{2}a^{n-2}b^2 + \dots + \binom{n}{n}b^n
	\]
\end{theorem}
\begin{proof}
	When we expand \((a+b)^n = (a+b)(a+b)\dots(a+b)\), we obtain terms of the form \(a^k b^{n-k}\).
	To get a single term of this form, we must choose \(k\) brackets for which to take the \(a\) value in the expansion, and the other \(n-k\) brackets will take the \(b\) value.
	The number of terms of the form \(a^k b^{n-k}\) for a fixed \(k\) is therefore the amount of ways of choosing \(k\) brackets out of a total of \(n\), which is \(\binom{n}{k}\).
	So
	\[
		(a+b)^n = \sum_{k=0}^n \binom{n}{k}a^k b^{n-k} = \sum_{k=0}^n \binom{n}{n-k}a^k b^{n-k}
	\]
\end{proof}
For example, we can tell that \((1+x)^n\) reduces to
\[
	1 + nx + \frac{1}{2}n(n-1)x^2 + \frac{1}{3!}n(n-1)(n-2)x^3 + \dots + nx^{n-1} + x^n
\]
So when \(x\) is small, a good approximation to \((1+x)^n\) is \(1 + nx\).

\subsection{Inclusion-Exclusion Theorem}
Given two finite sets \(A\), \(B\), we have
\[
	\abs{A \cup B} = \abs{A} + \abs{B} - \abs{A \cap B}
\]
For three sets, we have
\[
	\abs{A \cup B \cup C} = \abs{A} + \abs{B} + \abs{C} - \abs{A \cap B} - \abs{B \cap C} - \abs{C \cap A} + \abs{A \cap B \cap C}
\]
\begin{theorem}[Inclusion-Exclusion Theorem]
	Let \(S_1, \dots, S_n\) be finite sets.
	Then,
	\[
		\abs{\bigcup_{S \in S_n} S} = \sum_{\abs{A} = 1}\abs{S_A} - \sum_{\abs{A} = 2}\abs{S_A} + \sum_{\abs{A} = 3}\abs{S_A} - \cdots
	\]
	where
	\[
		S_A = \bigcap_{i \in A}S_i
	\]
	and
	\[
		\sum_{\abs{A} = k}
	\]
	is a sum taken over all \(A \subseteq \{ 1, 2, \dots, n \}\) of size \(k\).
\end{theorem}
\begin{proof}
	Let \(x\) be an element of the left hand side.
	We wish to prove that \(x\) is counted exactly once on the right hand side.
	Without loss of generality, let us rename the sets that \(x\) belongs to as \(S_1, S_2, \dots, S_k\).

	Then the number of sets \(A\) with \(\abs{A} = 1\) such that \(x \in S_A\) is \(k\).
	The number of sets \(A\) with \(\abs{A} = 2\) such that \(x \in S_a\) is \(\binom{k}{2}\), since we must choose two of the sets \(S_1, \dots, S_k\), so there are \(\binom{k}{2}\) ways to do this.
	So in general, the amount of \(A\) with \(\abs{A} = r\) with \(x \in S_A\) is just \(\binom{k}{r}\).

	So the number of times \(x\) is counted on the right hand side is
	\[
		k - \binom{k}{2} + \binom{k}{3} - \dots + (-1)^{k+1}\binom{k}{k}
	\]
	But \((1 + (-1))^k\) by the binomial expansion is
	\[
		1 - \binom{k}{1} + \binom{k}{2} - \binom{k}{3} + \dots + (-1)^k\binom{k}{k}
	\]
	So the number of times \(x\) is counted on the right hand side is \(1 - (1 + (-1))^k = 1 - 0 = 1\).
\end{proof}

\subsection{Functions}
For sets \(A\) and \(B\), a function \(f\) from \(A\) to \(B\) is a rule that assigns to each \(x \in A\) a unique value \(f(x) \in B\).
More precisely, a function from \(A\) to \(B\) is a set \(f \subseteq A \times B\) such that for every \(x \in A\), there is a unique \(y \in B\) with \((x, y) \in f\).
Of course therefore, if \((x, y) \in f\) then we can write \(f(x) = y\).
Here are some examples.
\begin{enumerate}
	\item \(f\colon \mathbb R \to \mathbb R\) given by \(f(x) = x^2\), or using an alternative notation, \(x \mapsto x^2\) is a function.
	\item A non-example is \(f\colon \mathbb R \to \mathbb R\) given by \(f(x) = \frac{1}{x}\) since it is undefined at \(x=0\).
	\item Another non-example is \(f\colon \mathbb R \to \mathbb R\) given by \(f(x) = \pm \sqrt{\abs{x}}\) since it does not define a unique value in the output space for a given input, such as \(x=2\).
	\item \(f\colon \mathbb R \to \mathbb R\) given by
	      \[
		      f(x) = \begin{cases}
			      1 & x \in \mathbb Q  \\
			      0 & \text{otherwise}
		      \end{cases}
	      \]
	      is a function since it clearly satisfies the second definition.
	      Note that even though we don't know if \(e + \pi\) is rational or not, the function is still well defined since it produces a unique solution for \(f(e + \pi)\), we just don't know which output value it gives.
	\item \(A = \{ 1, 2, 3, 4, 5 \}\), \(B = \{ 1, 2, 3, 4 \}\), and \(f\colon A \to B\) is given by
	      \begin{align*}
		      f(1) & = 1 \\
		      f(2) & = 4 \\
		      f(3) & = 3 \\
		      f(4) & = 3 \\
		      f(5) & = 4
	      \end{align*}
	\item \(A = \{ 1, 2, 3 \}\), \(f\colon A \to A\) is given by
	      \begin{align*}
		      f(1) & = 1 \\
		      f(2) & = 3 \\
		      f(3) & = 2
	      \end{align*}
	\item \(A = \{ 1, 2, 3, 4 \}\), \(f\colon A \to A\) is given by
	      \begin{align*}
		      f(1) & = 1 \\
		      f(2) & = 3 \\
		      f(3) & = 3 \\
		      f(4) & = 4
	      \end{align*}
	\item \(A = \{ 1, 2, 3, 4 \}\), \(B = \{ 1, 2, 3 \}\), \(f\colon A \to B\) is given by
	      \begin{align*}
		      f(1) & = 3 \\
		      f(2) & = 3 \\
		      f(3) & = 2 \\
		      f(4) & = 1
	      \end{align*}
\end{enumerate}
