\subsection{Introduction}
Intuitively, we might think that:
\begin{itemize}
	\item `\(A\) bijects with \(B\)' means `\(A\) has the same size as \(B\)'.
	\item `\(A\) injects into \(B\)' means `\(A\) is at most as large as \(B\)'.
	\item `\(A\) surjects onto \(B\)' means `\(A\) is at least as large as \(B\)'.
\end{itemize}
Of course, these analogies break down where \(B\) is zero, since there are no functions from \(A\) to \(B\) in this case.
For these to make sense, we require (for \(A, B\neq\varnothing\)) `\(A\) injects into \(B\)' to be true if and only if `\(B\) surjects onto \(A\)', and vice versa.
\begin{itemize}
	\item In the forward direction, we are given an injection \(f\colon A \to B\).
	      Pick some point \(a_0\) in \(A\), and define a surjective function \(g\colon B \to A\) given by
	      \[
		      b \mapsto \begin{cases}
			      a   & \text{if } \exists!\ a \in A, f(a) = b \\
			      a_0 & \text{otherwise}
		      \end{cases}
	      \]
	      Since the mapping \(f\) is injective, the first case will always provide a unique value of \(a\).
	\item Proving the converse, we are given a surjection \(g\colon B \to A\).
	      For each \(a\) in \(A\), we have some \(a' \in B\) with \(g(a') = a\) since \(g\) is a surjection.
	      Let \(f(a) = a'\) for each \(a\in A\), and \(f\) is injective.
\end{itemize}

\subsection{Schr\"oder-Bernstein theorem}
Further, we must also have that if `\(A\) is at most as large as \(B\)' and `\(B\) is at most as large as \(A\)', then they must be the same size.
Otherwise this size intuition would not make sense.
\begin{theorem}[Schr\"oder-Bernstein Theorem]
	If \(f\colon A\to B\) and \(g\colon B\to A\) are injections, then there exists a bijection \(h\colon A\to B\).
\end{theorem}
\begin{proof}
	For \(a\in A\), we will write \(g^{-1}(a)\) to denote the unique \(b \in B\) such that \(g(b) = a\), if such a \(b\) exists (and similarly for \(f^{-1}(b)\)).
	The `ancestor sequence' of \(a \in A\) is \(g^{-1}(a), f^{-1}g^{-1}(a), g^{-1}f^{-1}g^{-1}(a), \dots\) which may terminate.
	So for any ancestor, after undergoing the relevant function \(f\) or \(g\) repeatedly, we will end up at \(a\).
	There are three possible behaviours:
	\begin{itemize}
		\item Let \(A_0\) be the subset of \(A\) such that the ancestor sequence stops at even time, i.e.\ the last ancestor is in \(A\);
		\item Let \(A_1\) be the subset of \(A\) such that the ancestor sequence stops at odd time, i.e.\ the last ancestor is in \(B\); and
		\item Let \(A_\infty\) be the subset of \(A\) such that the ancestor sequence does not terminate.
	\end{itemize}
	We specify 0 to be even, i.e.\ if \(a\in A\) has no ancestor \(g^{-1}(a)\), then \(a \in A_0\).
	We define similar subsets of \(B\): \(B_0\), \(B_1\), \(B_\infty\).
	Now:
	\begin{itemize}
		\item \(f\colon A \to B\) is a bijection between \(A_0\) and \(B_1\).
		      Clearly if some element \(a\) has an even number of ancestors, the ancestors of \(f(a)\) are exactly \(a\) and all of its ancestors, i.e.\ an odd number.
		      It is surjective because every element in \(B_1\) has an inverse \(f^{-1}(b) \in A_0\) by construction.
		\item \(g\colon B \to A\) is a bijection between \(B_0\) and \(A_1\) due to the same argument.
		\item \(f\) (or \(g\), both functions work for this proof) bijects \(A_\infty\) and \(B_\infty\).
		      It is surjective because for every element \(b \in B\), it has some ancestor \(f^{-1}(b) \in A_\infty\).
	\end{itemize}
	So the function \(h\colon A \to B\) is given by
	\[
		h(a) = \begin{cases}
			f(a)      & \text{if } a \in A_0      \\
			g^{-1}(a) & \text{if } a \in A_1      \\
			f(a)      & \text{if } a \in A_\infty
		\end{cases}
	\]
	is a bijection.
\end{proof}
Let us consider an example of this theorem in action.
Do \([0, 1]\) and \([0,1]\cup[2,3]\) biject?
All we need is to find an injection both ways.
\begin{itemize}
	\item Let \(f\colon [0,1] \to [0,1] \cup [2,3]\) be the identity map \(f(x) = x\).
	\item Let \(g\colon [0,1] \cup [2,3] \to [0,1]\) be given by \(g(x) = x/3\).
\end{itemize}

It would also be nice to have that, for any sets \(A\) and \(B\), either \(A\) injects into \(B\) or \(B\) injects into \(A\).
Then we can create a total ordering, rather than a partial ordering; we can compare any two sets.
This is proven to be true in the Part II course Logic and Set Theory.

\subsection{Injections into power sets}
We have the sets
\[
	\mathbb N, \mathcal P(\mathbb N), \mathcal P(\mathcal P(\mathbb N)), \dots, \mathcal P^k(\mathbb N), \dots
\]
Does every set \(X\) inject into one of those?
It seems like this might be true, but the set
\[
	X = \mathbb N \cup \mathcal P(\mathbb N) \cup \mathcal P(\mathcal P(\mathbb N)) \cup \dots
\]
is a counterexample.
Let us continue further with this approach.
\[
	X' = X \cup \mathcal P(X) \cup \mathcal P(\mathcal P(X)) \cup \dots
\]
\[
	X'' = X' \cup \mathcal P(X') \cup \mathcal P(\mathcal P(X')) \cup \dots
\]
and so on.
Now, does every set inject into one of these sets?
No, consider
\[
	Y = X \cup X' \cup X'' \cup X''' \cup \dots
\]
We can keep going forever.
So we can't construct a set that all sets inject into.

\subsection{What happens next?}
This is the end of the Numbers and Sets course.
Here are a few of the courses that feed from this course.
\begin{itemize}
	\item Factorisation is taken further in the IB Groups, Rings and Modules course.
	\item Fermat's Little Theorem, squares modulo \(p\) etc.\ are taken further in II Number Theory.
	\item The analysis chapter is extended by IA Analysis.
	\item Countability and sizes of sets are taken further in the II Logic and Set Theory course.
\end{itemize}
