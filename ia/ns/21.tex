\subsection{Products of Countably Infinite Sets}
\begin{theorem}
	\(\mathbb N \times \mathbb N\) is countable.
\end{theorem}
\begin{proof}[Proof 1]
	We will define \(a_1 = (1, 1)\), and inductively define
	\[ a_n = \begin{cases}
			(p-1, q+1) & \text{if } p > 1 \\
			(q+1, 1)   & \text{if } p = 1
		\end{cases} \]
	where \(a_{n-1} = (p, q)\). Therefore, we are essentially moving across antidiagonals of the plane. This does hit every point \((x, y) \in \mathbb N \times \mathbb N\), for example by induction on \(x+y\), so we have listed all elements of \(\mathbb N \times \mathbb N\).
\end{proof}
\begin{proof}[Proof 2]
	If we can define an injective function \(\mathbb N \times \mathbb N \to \mathbb N\), then it is countable. For example, let \(f = 2^x 3^y\). \(f\) is injective, so \(\mathbb N \times \mathbb N\) is countable.
\end{proof}

\subsection{Countable Unions of Countable Sets}
Proof 2 is also a way to show the following theorem:
\begin{theorem}
	Let \(A_1, A_2, A_3, \dots\) be countable sets. Then \(A_1 \cup A_2 \cup A_3 \cup \dots\) is countable. Less formally, `a countable union of countable sets is countable'.
\end{theorem}
\begin{proof}
	For each \(i\), \(A_i\) is countable, so we can list \(A_i\) as \(a_{i1}, a_{i2}, a_{i3}, \dots\) which may or may not terminate. We can then define
	\[ f\colon \bigcup_{n \in \mathbb N}A_n \to \mathbb N;\quad f(x) = 2^i 3^j \]
	where \(x = a_{ij}\). If \(x\) is in more than one set, just take the least \(i\) that is valid. Then \(f\) is an injection so the union is countable.
\end{proof}

\subsection{Partitioning into Countable Subsets}
Here are some examples of using this theorem by partitioning sets as a countable union of countable subsets.
\begin{enumerate}
	\item \(\mathbb Q\) is countable, since it is a countable union of countable sets:
	      \[ \mathbb Q = \mathbb Z \cup \frac{1}{2}\mathbb Z \cup \frac{1}{3}\mathbb Z \cup \dots \]
	      Each \(\frac{1}{n}\mathbb Z\) is countable, since they biject with \(\mathbb Z\) which is a countable set. It doesn't matter if we've counted an element in \(\mathbb Q\) twice; the above theorem works even with intersecting sets.
	\item The set \(\mathbb A\) of all algebraic numbers is countable. It is enough to show that the set of integer polynomials is countable, since each polynomial has a finite amount of roots and then \(\mathbb A\) is a countable union of finite sets. Now, to show that the set of integer polynomials is countable, it is enough to show that for each degree \(d\) it is countable, since it is a countable union of all polynomials of degree \(d\) (again using the above theorem). To specify a polynomial of degree \(d\) you must name its coefficients, so this set injects into \(\mathbb Z^{d+1}\), so we must just show that \(\mathbb Z^{d+1}\) is countable (not a bijection since the first term of the polynomial must be nonzero). We know that \(\mathbb Z^n\) is countable because we can inductively show that \(\mathbb Z^2, \mathbb Z^3, \mathbb Z^4, \dots\) are countable inductively.
\end{enumerate}

\subsection{Uncountable Sets}
\begin{definition}
	A set is uncountable if there is no way to count the set.
\end{definition}
\begin{theorem}
	\(\mathbb R\) is uncountable.
\end{theorem}
\begin{proof}[Proof (Cantor's Diagonal Argument)]
	We will show that \((0, 1)\) is uncountable, then clearly \(\mathbb R\) is uncountable. Suppose \((0, 1)\) is countable. Then given a sequence \(r_1, r_2, \dots\) in \((0, 1)\), we just need to find some number \(s \in (0, 1)\) not contained within this sequence. For each \(r_n\), we have a decimal expansion \(r_n = 0.r_{n1}r_{n2}r_{n3}\dots\). Let us now write all of these numbers in a matrix-style form:
	\begin{align*}
		r_1 & = 0.r_{11}r_{12}r_{13}\dots \\
		r_2 & = 0.r_{21}r_{22}r_{23}\dots \\
		r_3 & = 0.r_{31}r_{32}r_{33}\dots \\
		\vdots
	\end{align*}
	We just need to construct some number \(s\) that is not in this list. So, let us simply make sure that for any given \(r\) value, there is at least one digit that does not match. The easiest way to construct such a number is
	\[ s = 0.s_1 s_2 s_3 \dots \]
	where \(s_1 \neq r_{11}\), \(s_2 \neq r_{22}\), \(s_3 \neq r_{33}\) and so on. We can pick any numbers we like according to these constraints, but we should avoid picking digits 0 and 9 since \(0.1000\dots = 0.0999\dots\) for example, which could cause unnecessary ambiguity. Then \(s \neq r_1, s \neq r_2, \dots\) since there is at least one mismatched digit in the expansion for each \(r_i\); they differ in decimal digit \(i\). So \(\mathbb R\) is uncountable.
\end{proof}
This is another proof that transcendental numbers exist. \(\mathbb R\) is uncountable and \(\mathbb A\) is countable, so there exists a transcendental number. Indeed, `most' numbers are transcendental, i.e. \(\mathbb R \setminus \mathbb A\) is uncountable (because if \(\mathbb R \setminus \mathbb A\) were countable, then \(\mathbb R\) would be \((\mathbb R \setminus \mathbb A) \cup \mathbb A\) which is a finite union of countable sets \contradiction).
