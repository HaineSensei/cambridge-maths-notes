\subsection{Proofs and Non-Proofs Continued}
\begin{claim}
	The solution to the real equation \(x^2-5x+6=0\) is \(x=2\) or \(x=3\).
\end{claim}
\begin{note}
	This is really two assertions:
	\begin{enumerate}[i.]
		\item \(x=2 \lor x=3 \implies x^2 - 5x + 6 = 0\), and
		\item \(x^2 - 5x + 6 = 0 \implies x=2 \lor x=3\)
	\end{enumerate}
	We can denote this using a two-way implication symbol \(\iff\):
	\[
		x=2 \lor x=3 \iff x^2 - 5x + 6 = 0
	\]
\end{note}
\begin{proof}
	We prove case i by expressing the left hand side as a product of factors: \((x-3)(x-2)=0\).
	The other case may be proven using factorisation.
\end{proof}

We can do another kind of proof using \(\iff\) symbols a lot.
However, we need to be absolutely sure that each step really is a bi-implication.
\begin{proof}[Alternative Proof]
	For any real \(x\):
	\begin{align*}
		x^2-5x+6=0 & \iff (x-2)(x-3) = 0       \\
		           & \iff x-2 = 0 \lor x-3 = 0 \\
		           & \iff x=2 \lor x = 3
	\end{align*}
\end{proof}

\begin{claim}
	Every positive real is at least 1.
\end{claim}
\begin{proof}
	Let \(x\) be the smallest positive real.
	We want to prove \(x=1\), so we prove this by contradiction.
	
	Case 1: if \(x < 1\) then \(x^2 < x\) \contradiction{}
	
	Case 2: if \(x > 1\) then \(\sqrt{x} < x\) \contradiction{}
	
	Therefore \(x=1\)
\end{proof}
\begin{note}
	The assertion that there exists a smallest positive real is not justified.
	This means that the proof is invalid in its entirety.
	It is important that every line in a proof must be justified.
\end{note}

\subsection{The Natural Numbers}
Each line in a proof must be justified.
So, in number theory, what are you allowed to assume?
We must begin with a set of axioms.
We define that the natural numbers are a set denoted \(\mathbb N\), that contains an element denoted 1, with an operation \(+1\) satisfying:
\begin{enumerate}
	\item \(\forall n \in \mathbb N, n + 1 \neq 1\)
	\item \(\forall m,n \in \mathbb N, m \neq n \implies m+1 \neq n+1\) (together with the previous rule, this captures the idea that all numbers in \(\mathbb N\) are distinct)
	\item For any property \(p(n)\), if \(p(1)\) is true and \(p(n) \implies p(n+1) \ \forall n \in \mathbb N\), then \(p(n) \ \forall n \in \mathbb N\) (induction axiom).
\end{enumerate}

\noindent This list of rules is known as the Peano axioms.
Note that we did not include 0 in this set.
You can show that the list of natural numbers is complete and has no extras (like the rational number \(3.5\)) by specifying \(p(n)=\) `\(n\) is on the list of natural numbers'.

Note that while numbers are defined as, for example, \(1+1+1+1\), we are free to use whatever names we like, e.g.
4 or the hexadecimal number 0xDEADBEEF = 3735928559.

We may also define our own operations, such as \(+2\), which is defined to be \(+1+1\).
In fact, we can define the operation \(+k\) for any \(k \in \mathbb N\) by stating:
\[
	(n+k)+1 = n+(k+1) \quad(\forall n, k \in \mathbb N)
\]
\noindent and using induction to construct the \(+k\) operator for all \(k\).
We can similarly construct multiplication and exponentiation operators for all natural numbers, although this is omitted here.
We can also prove properties on these operators such as associativity, commutativity and distributivity.

We can also define the \(<\) operator as follows: \(a < b \iff \exists k \in \mathbb N \st a + k = b\).
Of course, we can also prove several properties using this rule, such as transitivity, and the fact that \(a \nless a\), which are omitted here.
