\newcommand{\yearnumber}{IA}
\documentclass{scrbook}

\usepackage{fontspec}
\let\nocolourboxes 1
\usepackage[book]{../../util}
\usepackage{subfiles}
\usepackage{minitoc}
\usepackage{tocloft}

\renewcommand{\thechapter}{\Roman{chapter}}
\renewcommand{\thesection}{\arabic{section}}

% Define new font family with roman font and monospaced, oldstyled numbers
\newfontfamily{\msf}{TeX Gyre Pagella}[Numbers={OldStyle,Monospaced}, Ligatures=TeX]
% switch all headings and pagenumbers in the document
\addtokomafont{disposition}{\msf}
\addtokomafont{pagehead}{\msf}
\addtokomafont{pagenumber}{\msf}
% switch entries in toc
\DeclareTOCStyleEntry[entryformat=\msf\bfseries,pagenumberformat=\msf\bfseries]{default}{chapter}

\addtokomafont{pagenumber}{\oldstylenums}
\addtokomafont{chapterentrypagenumber}{\oldstylenums}
\RedeclareSectionCommand[tocpagenumberformat=\oldstylenums]{section}
\RedeclareSectionCommand[tocpagenumberformat=\oldstylenums]{subsection}

\renewcommand{\mtcSfont}{\msf\bfseries}
\renewcommand{\mtcSSfont}{\msf}

% add spacing between roman course numbers and course names
\DeclareTOCStyleEntries[numwidth=3em]{tocline}{chapter}

\begin{document}

\begin{titlepage}
	\begin{center}
		\vspace*{1cm}

		\Huge
		\textbf{Notes on the Mathematical Tripos}

		\vspace{0.5cm}
		\LARGE
		Sky Wilshaw

		\vfill

		\Huge
		\textsc{Part \yearnumber}

		\vfill

		\Large
		University of Cambridge\\
		2020--\the\year{}

	\end{center}
\end{titlepage}

\dominitoc{}

\setcounter{tocdepth}{0}
\cleardoubleoddpage
\tableofcontents
\newpage
\setcounter{tocdepth}{3}

\chapter*{Introduction}
% TODO: Write introduction
Introduction.
\iffalse
This book contains notes for the maths courses at Cambridge University.
Please note that while efforts have been made to ensure completeness and correctness, no guarantees can be made; this is simply a reasonably complete way of collating information about the courses.

This book can be downloaded in PDF form for free at \url{https://thirdsgames.co.uk/maths.html}, and the source code (for the book itself and for the individual courses) can be accessed at \url{https://github.com/zeramorphic/cambridge-maths-notes}.

You are given the right to download the PDF of the book (or its component parts) for private use.
You are permitted to download and modify the source code of the repository (the book and the course notes it contains), but may not distribute these modifications (including object files such as PDFs generated from these modifications) to others.
However, you are permitted to make public forks of the repository in order to create pull requests, but this does not grant you permission to distribute object files created from these forked repositories.
It must be clear when visiting it that such a repository is a fork of \url{https://github.com/zeramorphic/cambridge-maths-notes}, and must include a link to the original repository.
(Forks created on GitHub satisfy this requirement, as the title contains the words `forked from' and then a link to the original repository.)

This project makes use of free software in accordance with their license terms, including the \texttt{cobra} CLI interface generator \\ \url{https://pkg.go.dev/mod/github.com/spf13/cobra@v1.1.3}.
For more information, read the licenses of each dependency.
\fi

\let\maketitle\ignorespaces{}
\renewcommand{\tableofcontentsnewpage}{\newpage\minitoc\newpage}


\chapter{Numbers and Sets}
This course gives an introduction to university level maths.
We begin by presenting some basic rules, called axioms.
Then, we carefully prove logical statements that follow from these axioms.
We build upon our previous results iteratively until we have proven some important theorems.
Almost every other course in an undergraduate maths course will follow this pattern, and this course acts as a prototypical example.

In the first part of the course, we rigorously define the natural numbers, integers, rationals, and reals.
Using the axioms, we can prove facts about things like prime numbers, modular arithmetic, and limits.
In the second half of the course, we establish the notion of a set, and define what concepts like functions are.
At the end, we use the rules of sets to prove that there are different sizes of infinity.

\subfile{../ns/main.tex}

\chapter{Differential Equations}
A differential equation is an equation involving one or more unknown functions and their derivatives.
These equations arise in many fields of study, such as physics and biology.
In this course, we explore many different ways to solve some common types of differential equations.

In many cases, it is not possible to solve differential equations, so it is important to classify various cases that we can solve, and explore them in depth.
Heuristically, a differential equation is often easier to solve if it involves fewer variables, and if the derivatives involved have a lower order.
First, we will study differential equations in only one variable: the `ordinary' differential equations.
Towards the end of the course, we study `partial' differential equations, which can involve more than one variable.

\subfile{../de/main.tex}

\chapter{Groups}
Many mathematical objects have lots of symmetry.
To study symmetry in an abstract way, we define the notion of a group.
Groups allow us to characterise all of the possible symmetries of an object, so understanding groups allows us to understand symmetry itself.
Shapes, numbers, and matrices all give rise to their own groups, which provide insight into how the objects are structured.

As well as studying groups on their own, we also study the ways in which groups can interact.
One example is a particular kind of function called a homomorphism, which preserves the structure of the groups in question.
The homomorphisms between groups allow us to study each group in more detail.

\subfile{../groups/main.tex}

\chapter{Vectors and Matrices}
The complex numbers can be viewed as a kind of two-dimensional analogue to the real numbers, with a real coordinate and an imaginary coordinate.
Euclidean space is a three-dimensional version of the reals, with three coordinates to represent each point.
In this course, we generalise these examples, and study vector spaces which can have any dimension.

\subfile{../vm/main.tex}

\chapter{Dynamics and Relativity}
\subfile{../dr/main.tex}
\chapter{Probability}
\subfile{../probability/main.tex}
\chapter{Vector Calculus}
\subfile{../vc/main.tex}
\chapter{Analysis I}
\subfile{../analysis/main.tex}

\end{document}
