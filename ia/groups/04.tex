\subsection{Images and Kernels}
\begin{definition}
	Let \(\varphi: H \to G\) be a homomorphism.
	Then the image of \(\varphi\), denoted \(\Im \varphi\), is defined as \(\{ g \in G : g = \varphi(h) \text{ for some } h \in H \}\).
	The kernel of \(\varphi\), denoted \(\ker \varphi\), is defined as \(\{ h \in H : \varphi(h) = e_G \}\).
\end{definition}
Informally, we can say:
\begin{itemize}
	\item The image of \(\varphi\) is the set of outputs of \(\varphi\).
	\item The kernel of \(\varphi\) is the set of inputs that map to the identity element.
\end{itemize}
\begin{proposition}
	\(\Im \varphi \leq G\) and \(\ker \varphi \leq H\).
\end{proposition}
\begin{proof}
	To prove that \(\Im \varphi \leq G\), we check the group axioms (apart from associativity, since this is implicit).
	\begin{itemize}
		\item (closure) If \(a, b \in \Im \varphi\) then there exist some \(x, y \in H\) such that \(\varphi(x) = a\) and \(\varphi(y) = b\).
		      Therefore, \(\varphi(x)\varphi(y)=\varphi(xy)\) which is in the image by definition.
		\item (identity) \(\varphi(e_H) = e_G\)
		\item (inverses) Let \(x \in H\) such that \(\varphi(x) = a\).
		      Then, because \(x^{-1}\in H\), we know that \(\varphi(x^{-1}) = \varphi(x)^{-1} \in \Im H\) as required.
	\end{itemize}
	Now we prove a similar result for the kernel.
	\begin{itemize}
		\item (closure) If \(x, y \in \ker H\) then \(\varphi(xy) = \varphi(x) \varphi(y) = e_G e_G = e_G\), which is the requirement for being in the kernel, so \(xy \in \ker \varphi\).
		\item (identity) \(\varphi(e_H) = e_G\) so the identity element \(e_H\) is in the kernel.
		\item (inverses) \(\varphi(x^{-1}) = \varphi(x)^{-1}\).
		      So if \(x \in \ker \varphi\) then \(\varphi(x^{-1}) = e_G^{-1} = e_G\) so \(\varphi^{-1}\) is also in the kernel.
	\end{itemize}
\end{proof}
Here are a few examples of kernels and images of homomorphisms.
\begin{enumerate}
	\setcounter{enumi}{-1}
	\item If \(\varphi: H \to G\) is the trivial homomorphism (mapping every element to the identity) then:
	      \[
		      \Im \varphi = \{ e_G \};\quad \ker \varphi = H
	      \]
	\item If \(H \leq G\) then the inclusion homomorphism \(\iota: H \to G\) has
	      \[
		      \Im \iota = H;\quad \ker \iota = e_H
	      \]
	\item \(\varphi: \mathbb Z \to \mathbb Z_n\) where operations are performed modulo \(n\) has
	      \[
		      \Im \varphi = \mathbb Z_n;\quad \ker \varphi = n\mathbb Z
	      \]
\end{enumerate}

\begin{proposition}
	Let \(\varphi: H \to G\) be a homomorphism.
	Then
	\begin{itemize}
		\item \(\varphi\) is surjective if and only if \(\Im \varphi = G\); and
		\item \(\varphi\) is injective if and only if \(\ker \varphi = \{ e_H \}\).
	\end{itemize}
\end{proposition}
\begin{proof}
	The first case is trivial.
	After all, the definition of surjectivity is that all outputs are mapped onto by something, which means that image is equal to this output set.
	Now, let us prove the injectivity part.
	We start in the forward direction, then we prove the converse.

	Suppose that \(\varphi\) is injective.
	Then \(\varphi(a) = \varphi(b) \implies a = b\).
	We have that \(\varphi(e_H) = e_G\), so \(e_H\) must be the only element sent to \(e_G\).
	Therefore the kernel is simply \(\{ e_H \}\).

	Conversely, suppose that the kernel of \(\varphi\) is simply the identity element.
	Then, let us suppose there are two elements \(a, b\) in \(H\) such that \(\varphi(a) = \varphi(b)\).
	Then, \(\varphi(a b^{-1})=\varphi(a)\varphi(b)^{-1} = \varphi(b)\varphi(b)^{-1} = e_G\).
	Therefore, \(ab^{-1} = e_H\), so \(a = b\).
	So \(\varphi\) is injective.
\end{proof}

\subsection{Direct Products of Groups}
\begin{definition}
	The direct product of two groups \(G\) and \(H\) is written \(G \times H\), and defined to be \(\{ (g, h) : g \in G, h \in H \}\), where the group operation is defined by
	\[
		(g_1, h_1) \ast_{G \times H} (g_2, h_2) = (g_1 \ast_G g_2, h_1 \ast_H h_2)
	\]
\end{definition}
We will now prove that this really is a group.
\begin{proof}
	We prove each axiom.
	\begin{itemize}
		\item (closure) For a pair of elements \((g_1, h_1)\) and \((g_2, h_2)\) in \(G \times H\), the product \((g_1 \ast_G g_2, h_1 \ast_H h_2)\) is clearly in \(G \times H\), because the first entry is in \(G\) and the second entry is in \(H\), which is the requirement for being a member of \(G \times H\).
		\item (identity) The element \((e_G, e_H)\) is an identity.
		\item (inverses) Given an element \((g, h) \in G \times H\), the element \((g^{-1}, h^{-1})\) satisfies \((g^{-1}, h^{-1})(g, h) = (e_G, e_H) = e_{G \times H}\).
		\item (associativity) Given three elements \((g_i, h_i)\), \(i \in \{1, 2, 3\}\), we have
		      \begin{align*}
			      ((g_1, h_1) \ast (g_2, h_2)) \ast (g_3, h_3) & = (g_1 \ast g_2, h_1 \ast h_2) \ast (g_3, h_3)       \\
			                                                   & = ((g_1 \ast g_2) \ast g_3, (h_1 \ast h_2) \ast h_3) \\
			                                                   & = (g_1 \ast (g_2 \ast g_3), h_1 \ast (h_2 \ast h_3)) \\
			                                                   & = (g_1, h_1) \ast (g_2 \ast g_3, h_2 \ast h_3)       \\
			                                                   & = (g_1, h_1) \ast ((g_2, h_2) \ast (g_3, h_3))
		      \end{align*}
	\end{itemize}
\end{proof}
\noindent \(G \times H\) contains subgroups \(G \times {e_H}\) and \({e_G} \times H\) which are isomorphic to \(G\) and \(H\) respectively.
We name these subgroups simply \(G\) and \(H\) because they are isomorphic.
\begin{note}
	In \(G \times H\), everything in \(G\) commutes with everything in \(H\).
	\[
		\forall g \in G, \forall h \in H, (g, e_H) \ast (e_G, h) = (e_G, h) \ast (g, e_H) = (g, h)
	\]
\end{note}

\begin{theorem}[Direct Product Theorem]
	Let \(H, K\) be subgroups of \(G\) such that
	\begin{itemize}
		\item \(H \cap K = \{ e \}\) (the groups intersect only in \(e\))
		\item \(\forall h \in H, \forall k \in K, hk = kh\) (\(H\) and \(K\) commute in \(G\))
		\item \(\forall g \in G, \exists h \in H, \exists k \in K \st g = hk\) (\(G = HK\))
	\end{itemize}
	Then \(G \cong H \times K\).
\end{theorem}
\begin{proof}
	Consider \(\varphi: H \times K \to G\) where \(\varphi((h, k)) = hk\).
	We now prove that \(\varphi\) is a homomorphism.
	\begin{align*}
		\varphi((h_1, k_1)(h_2, k_2)) & = \varphi((h_1h_2, k_1k_2))              \\
		                              & = h_1h_2k_1k_2                           \\
		                              & = h_1k_1h_2k_2                           \\
		                              & = \varphi((h_1, k_1))\varphi((h_2, k_2))
	\end{align*}
	Note that by the third property in the theorem we know that \(\varphi\) is surjective.
	We now prove that \(\varphi\) is also injective.

	Suppose that \((h, k) \in \ker \varphi\).
	Then \(\varphi((h, k)) = e_G\) so \(hk = e_G\).
	So \(h = k^{-1}\).
	This means that there is some element that is part of both \(H\) and \(K\), for example \(h\).
	But by the first property in the theorem, this value must be \(e\), so \(\ker \varphi = \{ e_G \}\), so \(\varphi\) is injective.

	\(\varphi\) is an injective, surjective homomorphism, so it is an isomorphism.
	So \(G\) is isomorphic to \(H \times K\).
\end{proof}

\noindent Now, we can consider direct products in two distinct lenses: a combination of smaller groups to form a large one, or a partition of a large group into two that combine to produce the original.

\subsection{Cyclic Groups}
\begin{definition}
	Let \(G\) be a group, and let \(X \subseteq G\) be some subset.
	If \(\genset{X} = G\) then \(X\) is a generating set of \(G\).
	We say that \(G\) is cyclic if there exists some element \(a\) in \(G\) such that \(\genset{a} = G\).
	\(a\) is called a generator of \(G\).
\end{definition}
\begin{enumerate}
	\setcounter{enumi}{-1}
	\item The trivial group \(\{ e \}\) is generated by its element.
	\item \((\mathbb Z, +)\) is a cyclic group generated by \(\mathbb Z = \genset{-1} = \genset{1}\).
	\item \((\mathbb Z_n, +)\), where addition is modulo \(n\), is generated by \(\mathbb Z_n = \genset{k}\) where \(k\) and \(n\) are coprime.
\end{enumerate}
