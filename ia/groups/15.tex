\subsection{Symmetries of the Tetrahedron}
A tetrahedron has 4 faces (regular, equilateral triangles), 4 vertices, and 6 edges. We will label the vertices \(1, 2, 3, 4\). Let \(G\) be the group of symmetries of the tetrahedron. Clearly \(G\) acts transitively on the vertices (we can get from any vertex to any other through a symmetry). There is no non-trivial symmetry that fixes all the vertices, so \(\rho\colon G \to S_4\) is an injective homomorphism.

\(\Orb(1) = \{ 1, 2, 3, 4 \}\) as \(G\) is transitive. \(\Stab(1) =\) all of the symmetries of the face \(\{2,3,4\}\), i.e.
\[ \Stab(1) = \{ e, (2\ 3\ 4), (2\ 4\ 3), (2\ 3), (3\ 4), (2\ 4) \} \cong D_6 \cong S_3 \]
Then \(\abs{G} = \abs{\Orb(1)} \cdot \abs{\Stab(1)} = 4 \cdot 6 = 24 = \abs{S_4}\). Since \(G \leq S_4\) and their orders match, \(G = S_4\).

Now let \(G^+\) be the subgroup of \(G\) formed only of the rotations in \(G\). Again, \(\Orb(1) = \{ 1, 2, 3, 4 \}\). Now, \(\Stab(1) = \{ e, (2\ 3\ 4), (2\ 4\ 3) \}\). So \(\abs{G^+} = \abs{\Orb(1)} \cdot \abs{\Stab(1)} = 4 \cdot 3 = 12\). Since \(G^+ \leq G = S_4\), then we know that \(G^+ = A_4\). Indeed, we have all 3-cycles (since these are rotations through vertices), and all elements of the form \((1\ 2)(3\ 4)\) since these are rotations in the axis through the midpoints of opposite edges.

\subsection{Symmetries of the Cube}
We label the vertices from 1 to 8 here, and let \(G\) be the group of symmetries of the cube acting on the vertices. Clearly the action is transitive, so \(\abs{\Orb(1)} = 8\). \(\Stab(1) = \{ e, r, r^2, s_1, s_2, s_3 \}\) where \(r\) and \(r^2\) are the rotations through the axis that passes through vertex 1, and where the \(s_i\) are the reflections through three planes containing vertex 1. So \(\abs{\Stab(1)} = 6\), so \(\abs{G} = 48\). We will determine this group completely later on.

Let \(G^+\) be the subgroup of \(G\) containing the rotations of \(G\). Then, the action is still transitive, and \(\abs{\Stab(1)} = 3\), since we are only looking at the rotations. So \(\abs{G^+} = 24\).

Now, to determine this group, let \(G^+\) act on the 4 diagonals in the cube. This gives us a homomorphism \(\rho\colon G^+ \to S_4\). We have all 4-cycles in \(\Im \rho\), since rotating the cube by quarter turns through the \(x, y, z\) axes permute the diagonals in this way. We also have all transpositions (2-cycles) by rotating the cube by a half turn through the plane of two diagonals. In Example Sheet 2, we prove that \(\genset{(1\ 2), (1\ 2\ 3\ 4)} = S_4\), so \(\rho\) is surjective. But since the orders match, \(G^+ \cong S_4\).

\subsection{Platonic Solids}
The aforementioned solids are two of the five Platonic solids; the solids in \(\mathbb R^3\) that have polygonal faces, straight edges and vertices such that their group of symmetries acts transitively on triples (vertex, incident edge, incident face). These are therefore particularly symmetric solids for having this transitive action. The other solids are the octahedron, dodecahedron and icosahedron. The cube and octahedron are `dual', i.e. they can be inscribed in each other with vertices placed in the centres of faces. The dodecahedron and icosahedron are also dual. Dual solids have the same symmetry groups, so there are only three symmetry groups of Platonic solids.

\subsection{Cauchy's Theorem}
\begin{theorem}
	Let \(G\) be a finite group, \(p\) a prime such that \(p \mid \abs{G}\). Then \(G\) has an element of order \(p\).
\end{theorem}
\begin{proof}
	Let \(p \mid \abs{G}\). Consider \(G^p = G \times G \times \cdots \times G\). This is the group formed of \(p\)-tuples of elements of \(G\) with coordinate-wise composition. Consider the subset \(X \subseteq G^p\), given by
	\[ X := \{ (g_1, g_2, \cdots, g_p) \in G^p: g_1g_2\cdots g_p = e \} \]
	which can be described as `\(p\)-tuples multiplying to \(e\)'. Note that if \(g \in G\) has order \(p\), then \((g, g, \cdots, g) \in X\); and that if \((g, g, \cdots, g) \in X\) where \(g \neq e\), then \(g\) has order \(p\).

	Now take a cyclic group \(C_p = \genset a\), and let \(C_p \acts X\) by `cycling':
	\[ a(g_1, g_2, \cdots, g_p) = (g_2, \cdots, g_p, g_1) \]
	This really is an action:
	\begin{itemize}
		\item If \(g_1g_2 \cdots g_p = e\), then \(e = g_1^{-1} e g_1 = g_1^{-1}g_1g_2 \cdots g_p g_1 = g_2 \cdots g_pg_1\) as required. Of course, this applies inductively for any power of \(a\).
		\item \(e(g_1, \cdots, g_p) = (g_1, \cdots, g_p)\) as required.
		\item \(a^k(g_1, \cdots, g_p) = (g_{k+1}, \cdots, g_k) = a \cdot a \cdots a(g_1, \cdots, g_k)\).
	\end{itemize}
	Since orbits partition \(X\), the sum of the sizes of the orbits must be \(\abs{X}\). We know that \(\abs{X} = \abs{G}^{p-1}\), since all choices of \(g_i\) are free apart from the last one, which must be the inverse of the product of the other elements. So we have \(p-1\) choices of \(\abs{G}\) elements, so \(\abs{X} = \abs{G}^{p-1}\).

	So since \(p \mid \abs{G}\), then \(p \mid \abs{X}\). By the Orbit-Stabiliser theorem:
	\[ \abs{\Orb((g_1, \cdots, g_p))} \cdot \abs{\Stab((g_1, \cdots, g_p))} = \abs{C_p} = p \]
	So any orbit has size 1 or \(p\), and they sum to \(\abs{X} = pk\) for some \(k \in \mathbb N\). So
	\[ \abs{X} = pk = \sum_{\text{orbits of size 1}} 1 + \sum_{\text{orbits of size \(p\)}} p \]
	Clearly, \(\abs{\Orb((e, e, \cdots, e))} = 1\). So there must be some other orbits of size 1, so that \(p\) divides the amount of orbits of size 1. But orbits of size 1 must be of the form \(\Orb((g, g, \cdots, g))\) in order to have the same form under the action of \(a\). So there exists some \(g \neq e \in G\) such that \((g, g, \cdots g) \in X\), i.e. \(g^p = e\), so \(o(g) = p\).
\end{proof}
