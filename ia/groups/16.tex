\subsection{Left multiplication actions}
\begin{lemma}
	Let \(G\) be a group.
	\(G\) acts on itself by left multiplication.
	This action is faithful and transitive.
\end{lemma}
\begin{proof}
	\begin{itemize}
		\item For any \(g, x \in G\), \(gx \in G\)
		\item \(e(x) = e \cdot x = x\)
		\item \((g_1 g_2) x = g_1 (g_2 x)\)
	\end{itemize}
	So it really is an action.
	It is faithful because \(g(x) = gx = x\) implies \(g = e\).
	It is transitive, because given any \(x, y \in G\), the action \(g = yx^{-1}\) gives \(g(x) = y\).
\end{proof}
\begin{definition}
	This left-multiplication action of a group on itself is known as the left regular action.
\end{definition}

\subsection{Cayley's theorem}
\begin{theorem}
	Every group is isomorphic to a subgroup of a symmetric group.
\end{theorem}
\begin{proof}
	Let \(G \acts G\) by the left regular action.
	This gives a homomorphism \(\rho\colon G \to \Sym(G)\), with \(\ker \rho = \{ e \}\) since the action is faithful.
	So, by the First Isomorphism Theorem, \(\frac{G}{\ker \rho} = G \cong \Im \rho \leq \Sym(G)\).
\end{proof}

\begin{proposition}
	Let \(H \leq G\).
	Then \(G\) acts on the set of left cosets of \(H\) in \(G\) by left multiplication, and this action is transitive.
	(This is called the `left coset action').
\end{proposition}
\begin{proof}
	We check the conditions for actions.
	\begin{itemize}
		\item \(g(g_1H) = gg_1H\), so \(g(g_1H)\) is a left coset.
		\item \(e(g_1G) = eg_1H = g_1H\)
		\item \((gg')(g_1H) = gg'g_1H = g(g'(g_1H))\)
	\end{itemize}
	So this is an action.
	Given two cosets \(g_1H\) and \(g_2H\), the element \((g_1g_2^{-1})\) acts on \(g_2H\) to give \(g_1H\), so it is transitive.
\end{proof}
Note:
\begin{itemize}
	\item This is the left regular action if \(H = \{ e \}\).
	\item This induces actions of \(G\) on its quotient groups \(\frac{G}{N}\).
\end{itemize}

\subsection{Conjugation actions}
\begin{definition}
	Given \(g, h \in G\), the element \(hgh^{-1}\) is the conjugate of \(g\) by \(h\).
\end{definition}
We should think of conjugate elements as doing the same thing but from different `points of view' --- we change perspective by doing \(h^{-1}\), then do the action \(g\), then reset the perspective back to normal using \(h\).

Here is an example using \(D_{10}\), where the vertices of the regular pentagon are \(v_1 \dots v_5\) clockwise.
Consider the conjugates \(s\) and \(rsr^{-1}\), where \(s\) is a reflection through \(v_1\) and the centre, and \(r\) is a rotation by \(\frac{2\pi}{5}\) clockwise.
So \(rsr^{-1}\) ends up being just a reflection through \(v_2\) and the centre.
So the result of conjugating the reflection by a rotation is still a reflection, just from a different point of view.

Another example is in matrix groups such as \(GL_n(\mathbb R)\) where a conjugate matrix represents the same transformation but with respect to a different basis.
This will be covered in more detail later.

As a general principle, conjugate elements can be expected to have similar properties.
We will now prove some of these such properties.
\begin{proposition}
	A group \(G\) acts on itself by conjugation.
\end{proposition}
\begin{proof}
	\begin{itemize}
		\item \(g(x) = gxg^{-1} \in G\) for any \(g, x \in G\)
		\item \(e(x) = exe^{-1} = x\) for any \(x \in G\)
		\item \(g(h(x)) = ghxh^{-1}g^{-1} = (gh)(x)\)
	\end{itemize}
\end{proof}
\begin{definition}
	The kernel, orbits and stabilisers have special names:
	\begin{itemize}
		\item The kernel of the conjugation action of \(G\) on itself is the centre \(Z(G)\):
		      \[
			      Z(G) := \{ g \in G : \forall h \in G,\, ghg^{-1} = h \iff gh=hg \}
		      \]
		      In less formal terms, \(Z(G)\) is the set of `elements that commute with everything'.
		\item An orbit of this action is called a conjugacy class:
		      \[
			      \ccl(h) := \{ ghg^{-1}: g \in G \}
		      \]
		      Sometimes this is written \(\ccl_G(h)\) to clarify which group we're working on.
		\item Stabilisers are called centralisers:
		      \[
			      C_G(h) := \{ g \in G : ghg^{-1} = h \iff gh = hg \}
		      \]
		      This is the set of `elements that commute with \(h\)'.
	\end{itemize}
\end{definition}
Exercise: \(Z(G) = \bigcap_{h \in G} C_G(h)\).

\begin{definition}
	If \(H \leq G, g \in G\), then the conjugate of \(H\) by \(g\) is:
	\[
		gHg^{-1} = \{ ghg^{-1} : h \in H \}
	\]
\end{definition}
\begin{proposition}
	Let \(H \leq G, g \in G\).
	Then \(gHg^{-1}\) is also a subgroup of \(G\).
\end{proposition}
\begin{proof}
	We check the group axioms.
	\begin{itemize}
		\item (closure) If \(gh_1g^{-1}, gh_2g^{-1} \in gHg^{-1}\), then
		      \[
			      (gh_1g^{-1})(gh_2g^{-1}) = gh_1(g^{-1}g)h_2g^{-1} = g(h_1h_2)g^{-1} \in gHg^{-1}
		      \]
		\item (identity) \(geg^{-1} = e \in gHg^{-1}\)
		\item (inverses) Given \(ghg^{-1} \in gHg^{-1}\), the inverse is \(gh^{-1}g^{-1}\), which of course is an element of \(gHg^{-1}\).
	\end{itemize}
\end{proof}
Note that \(gHg^{-1}\) is isomorphic to \(H\) (proof as exercise).
\begin{proposition}
	A group \(G\) acts by conjugation on the set of its subgroups.
	The singleton orbits are the normal subgroups.
\end{proposition}
Proof as exercise.
(Recall that \(N \trianglelefteq G \iff \forall g \in G,\, gNg^{-1} = N\), which is the same as being stable under conjugation)
