\subsection{Order of Permutations}
The set of cycle lengths of the disjoint cycle decomposition of \(\sigma\) is called the cycle type of \(\sigma\).
For example, \(\sigma = (1\ 2\ 3)(5\ 6)\) has cycle type \(3, 2\) (or equivalently \(2, 3\)).

\begin{theorem}
	The order of \(\sigma \in S_n\) is the least common multiple of the cycle lengths in its cycle type.
\end{theorem}
\begin{proof}
	The order of a \(k\)-cycle is \(k\).
	Let us decompose \(\sigma\) into a product of disjoint cycles such that \(\sigma = \tau_1 \tau_2 \cdots \tau_r\).
	Then \(\sigma^m = \tau_1^m \tau_2^m \cdots \tau_r^m\) since disjoint cycles commute.

	Let each \(\tau_i\) be a \(k_i\)-cycle.
	Then if \(\sigma^m = e\), \(\tau_1^m \tau_2^m \cdots \tau_r^m = e\), and so \(\tau_1^m = \tau_2^{-m} \tau_3^{-m} \cdots \tau_r^{-m}\).
	Note that the right hand side and left hand side permute different elements, so they must both be the identity element \(e\).
	Repeating this style of argument with every \(\tau\) shows that \(\tau_i^m = e\) so \(k_i | m\).

	So clearly the lowest common multiple of all of the \(k_i\) divides the order of the permutation, \(o(\sigma)\).
	Now, we check that it is actually equal to \(o(\sigma)\).
	Let \(L\) be this lowest common multiple.
	Then \(\sigma^L = \tau_1^L \tau_2^L \cdots \tau_r^L = (\tau_1^{k_1})^{L/k_1} (\tau_2^{k_2})^{L/k_2} \cdots (\tau_r^{k_r})^{L/k_r}\).
	All of these exponents are integers because \(L\) is a multiple of each \(k_i\).
	So we have \(e \cdot e \cdots e = e\).
	So the order of \(\sigma\) is \(L\).
\end{proof}

\subsection{Products of Transpositions}
\begin{proposition}
	Let \(\sigma \in S_n\).
	Then \(\sigma\) is a product of transpositions.
\end{proposition}
\begin{proof}
	It is enough to prove this for just a cycle, then we can use the disjoint cycle decomposition to create a transposition product for the whole \(\sigma\).
	\((a_1\ a_2\cdots a_n) = (a_1\ a_2)(a_2\ a_3)\cdots(a_{n-1}\ a_n)\), so the result is immediate.
\end{proof}
\noindent Note that this decomposition is not unique in general.

\subsection{Permutation Parity}
A permutation may be considered even if its transposition decomposition has an even number of terms, or odd otherwise.
Note that an even-length cycle has odd parity, and an odd-length cycle has even parity.
\begin{proposition}
	The parity of a permutation is well-defined, regardless of exactly how you write a permutation.
\end{proposition}
\begin{proof}
	Let us denote the amount of cycles in the disjoint cycle decomposition of \(\sigma\) with \(\#(\sigma)\).
	Let \(\tau = (c\ d)\).
	Then the effects of multiplying \(\sigma\) by \(\tau\) (on the right) have two cases, since it only affects \(c\) and \(d\).
	\begin{itemize}
		\item If \(c\) and \(d\) are in the same cycle in \(\sigma\), we get the following conversion:
		      \[
			      (c\ a_2 \cdots a_{k-1}\ d\ a_{k+1} \cdots a_l) \mapsto (c\ a_{k+1} \cdots a_l) (d\ a_2 \cdots a_{k-1})
		      \]
		      So \(\#(\sigma\tau) = \#(\sigma) + 1\).
		\item Otherwise, \(c\) and \(d\) are in different cycles (possibly singletons) in \(\sigma\), so we get the following conversion:
		      \[
			      (c\ a_2 \cdots a_k) (d\ b_2 \cdots a_l) \mapsto (c\ b_2 \cdots b_l\ d\ a_2 \cdots a_k)
		      \]
		      So \(\#(\sigma\tau) = \#(\sigma) - 1\).
	\end{itemize}
	In either case, parity is flipped.
	Now, suppose that \(\sigma\) is written as two products of transpositions, where one has \(m\) transpositions, and one has \(n\) transpositions.
	Therefore we have \(\#(\sigma) \equiv \#(e) + m \mod 2\), and \(\#(\sigma) \equiv \#(e) + n \mod 2\).
	But \(\#(\sigma)\) is uniquely determined by \(\sigma\), so both equations match, so \(m \equiv n \mod 2\), so the parity is well-defined.
\end{proof}

\begin{definition}
	Writing \(\sigma\) as a product of transpositions, the sign of \(\sigma\) is defined as 1 if the permutation is even, and \(-1\) if it is odd.
\end{definition}
Note that the function \(\mathrm{sign}(\sigma)\) is a homomorphism from \(S_n\) to \((\{-1, 1\}, \cdot)\).

\begin{definition}
	The alternating group \(A_n\) is defined as the kernel of the sign homomorphism on \(S_n\).
	In other words, it is the set of even permutations of \(S_n\).
\end{definition}
