\subsection{Definitions}
Which elements of \(X\) can we `get to' from a certain \(x \in X\) using the action of \(G\)?
\begin{definition}
	Let \(G \acts X\), \(x \in X\).
	The orbit of \(x\) is
	\[
		\Orb (x) = G(x) := \{ g(x) : g \in G \} \subseteq X
	\]
\end{definition}
\noindent Which group elements leave a given \(x\) unchanged?
\begin{definition}
	The stabiliser of \(x\) is defined by
	\[
		\Stab(x) = G_x := \{ g \in G : g(x) = x \} \subseteq G
	\]
\end{definition}
\begin{definition}
	An action is transitive if \(\Orb(x) = X\), i.e.\ we can get to any element from any other element.
\end{definition}
As an example, let \(G = S_3\).
Then we could say, for example, \(G \acts \{ 1, 2, 3, 4 \}\).
\begin{itemize}
	\item \(\Orb(1) = \Orb(2) = \Orb(3) = \{ 1, 2, 3 \}\)
	\item \(\Orb(4) = \{ 4 \}\)
	\item \(\Stab(1) = \{ e, (2\ 3) \}\)
	\item \(\Stab(2) = \{ e, (1\ 3) \}\)
	\item \(\Stab(3) = \{ e, (1\ 2) \}\)
	\item \(\Stab(4) = G\)
\end{itemize}

\subsection{Properties}
\begin{lemma}
	For any \(x \in X\), \(\Stab(x) \leq G\).
\end{lemma}
\begin{proof}
	Associativity is inherited.
	\begin{itemize}
		\item (closure) \(g, h \in \Stab(x)\) implies that \((gh)(x) = g(h(x)) = g(x) = x\) so \(gh \in \Stab(x)\).
		\item (identity) \(e(x) = x\) by definition, so \(e \in \Stab(x)\).
		\item (inverses) if \(g \in \Stab(x)\) then \(g(x) = x\), and therefore \(x = g^{-1}(x)\), so \(g^{-1} \in \Stab(x)\).
	\end{itemize}
\end{proof}
Recall from Numbers and Sets: a partition of a set \(X\) is a set of subsets of \(X\) such that each \(x \in X\) belongs to exactly one subset in the partition.
\begin{lemma}
	Let \(G \acts X\).
	Then the orbits partition \(X\).
\end{lemma}
\begin{proof}
	\begin{itemize}
		\item Firstly, for any \(x \in X\), \(x \in \Orb(x)\).
		      So the union of all orbits is \(X\).
		\item Suppose that the orbits are not all disjoint.
		      Let \(z \in \Orb(x) \cap \Orb(y)\).
		      Then \(\exists g_1 \in G\) such that \(g_1(x) = z\), and also \(\exists g_2 \in G\) such that \(g_2(y) = z\), i.e.\ \(y = g_2^{-1}(z)\).
		      So \(y = g_2^{-1}g_1(x)\).
		      Thus, for any \(g \in G\), \(g(y) = gg_2^{-1}g_1(x) \in \Orb(x)\) so \(\Orb(y) \subseteq \Orb(x)\).
		      Vice versa, \(\Orb(x) \subseteq \Orb(y)\), so \(\Orb(x) = \Orb(y)\).
		      Thus orbits are either disjoint or equal.
	\end{itemize}
\end{proof}
Recall the proof of disjoint cycle notation for \(\sigma \in S_n\): we were really finding the orbits in \(\{ 1, 2, \cdots, n \}\) under \(\genset \sigma\), which are disjoint.
Note that the sizes of orbits can be different (unlike cosets, where the sizes are always the same).
\begin{theorem}[Orbit-Stabiliser Theorem]
	Let \(G \acts X\), \(G\) finite.
	Then for any \(x \in X\),
	\[
		\abs{G} = \abs{\Orb x} \cdot \abs{\Stab x}
	\]
\end{theorem}
\begin{proof}
	\(g(x) = h(x) \iff h^{-1}g(x) = x \iff h^{-1}g \in \Stab(x)\).
	By the proposition in section 9.2, this statement is true if and only if
	\(g \Stab(x) = h \Stab(x)\) as cosets.
	So distinct points in the orbit of \(x\) are in bijection with distinct cosets of the stabiliser.
	So \(\abs{\Orb x} = \abs{G : \Stab x}\) and the result follows.
\end{proof}
In particular, notice that all elements in a given coset \(g \Stab(x)\) do the same thing to \(x\) as \(g\): an element of this coset has the form \(gh\) where \(h \in \Stab(x)\).
Then \(gh(x) = g(x)\).

This theorem is very powerful, we can use it for investigating groups further.
For example, we can construct another proof that \(\abs{D_{2n}} = 2n\) using the Orbit-Stabiliser theorem.
\(D_{2n}\) acts transitively on \(\{1, 2, \cdots, n\}\) so \(\abs{\Orb(1)} = n\).
\(\abs{\Stab(1)} = 2\) because only the identity and the reflection through this point stabilise the point.
So \(\abs{D_{2n}} = 2n\).
