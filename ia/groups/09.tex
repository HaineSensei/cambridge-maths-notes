\subsection{Lagrange's Theorem}
\begin{definition}
	We define the index of a subgroup \(H \leq G\) in \(G\), written \(\abs{G : H}\), to be the number of distinct cosets of \(H\) in \(G\).
\end{definition}
\begin{theorem}[Lagrange's Theorem]
	Let \(H \leq G\) be a subgroup of a finite group \(G\). Then:
	\begin{enumerate}[(i)]
		\item \(\abs{H} = \abs{gH}\) for any \(g \in G\);
		\item for any \(g_1, g_2 \in G\), either \(g_1 H = g_2 H\) or \(g_1 H \cap g_2 H = \varnothing\); and
		\item \(G = \bigcup_{g \in G} gH\)
	\end{enumerate}
	And in particular, \(\abs{G} = \abs{G : H} \cdot \abs{H}\).
\end{theorem}
\begin{proof}
	We prove each statement independently.
	\begin{enumerate}[(i)]
		\item The function \(H \to gH\), defined by \(h \mapsto gh\), defines a bijection between \(H\) and \(gH\), so \(\abs{H} = \abs{gH}\).
		\item Suppose \(g_1 H \cap g_2 H \neq \varnothing\). Then \(\exists g \in g_1 H \cap g_2 H\). So \(g = g_1 h_1 = g_2 h_2\) for some \(h_1, h_2 \in H\). So \(g_1 = g_2 h_2 h_1^{-1}\). So for any \(h \in H\), we have
		      \[ g_1 h = g_2 \underbrace{h_2 h_1^{-1} h}_{\mathclap{\in H}} \]
		      So certainly \(g_1 H \subseteq g_2 H\). Employing a symmetric argument for the other way round, we have \(g_1 H = g_2 H\).
		\item Given some \(g \in G\) then \(g \in gH\), since \(e \in H\). So \(G \subseteq \bigcup_{g \in G} gH\). But also, \(gH \subseteq G\), so \(\bigcup_{g \in G} gH \subseteq G\). So \(G = \bigcup_{g \in G} gH\).
	\end{enumerate}
	So now that we know that \(G\) is composed of a union of disjoint cosets, all of which are the same size, we know that \(\abs{G}\) is just the number of these cosets multiplied by the size of such a coset, or in other words
	\[ \abs{G} = \abs{G : H} \cdot \abs{H} \]
\end{proof}
Note that we could equivalently have used right cosets in place of left cosets. Remember that in general, \(gH \neq Hg\), and the set of left cosets is not equal to the set of right cosets.

\subsection{Consequences}
\begin{proposition}
	\(g_1 H = g_2 H \iff g_1^{-1} g_2 \in H\).
\end{proposition}
\begin{proof}
	We first consider the forwards case. Clearly \(g_1\) is an element of \(g_1 H\), as \(H\) contains \(e\). Also, \(g_2\) is an element of \(g_2\). So \(g_1^{-1} g_2 \in H\). Now for the backwards case. Clearly, \(g_2 H\) contains the element \(g_2\), as \(e\) maps to it. Also, since \(H\) contains \(g_1^{-1} g_2\), \(g_1 H\) contains the element \(g_1 \ast (g_1^{-1} g_2) = g_2\). As cosets are either disjoint or equal, and they clearly share the element \(g_2\), then they are equal.
\end{proof}
Note further that \(g' \in gH\) implies \(g'H = gH\). We may therefore take a single element from each of these distinct cosets, and we will call them \(g_1, g_2, \cdots, g_{\abs{G:H}}\). Then
\[ G = \bigsqcup_{i=1}^{\abs{G:H}} g_i H \]
where the \(\bigsqcup\) symbol denotes a disjoint union of sets. These \(g_i\) are called coset representatives of \(H\) in \(G\).

\begin{corollary}
	Let \(G\) be a finite group and \(g \in G\). Then \((\ord g) \mid \abs{G}\).
\end{corollary}
\begin{proof}
	Recall that \(\ord g\) is defined as the smallest \(n\) such that \(g^n = e\). We define the subgroup \(H \leq G\) as \(H = \genset g\). Then \(\ord g = \abs{H}\). By Lagrange's Theorem, we know that \(\abs{H} \mid \abs{G}\).
\end{proof}

\begin{corollary}
	Let \(G\) be a finite group, and let \(g \in G\). Then \(g^{\abs{G}} = e\).
\end{corollary}
\begin{proof}
	This follows directly from the previous corollary. \(g^{\abs{G}} = g^{n \cdot \ord g}\) for some natural number \(n\), so this simply reduces to \(e\).
\end{proof}

\begin{corollary}
	Groups of prime order are cyclic, and are generated by any non-identity element.
\end{corollary}
\begin{proof}
	Let \(\abs{G} = p\), where \(p\) is a prime. We will take some \(g \in G\), and generate a group from it. By Lagrange's Theorem, \(\abs{\genset{g}} \mid \abs{G}\), so \(\abs{\genset g}\) is either 1 or \(p\). Now, note that \(e\) and \(g\) are both elements of \(\genset{g}\), so if \(g \neq e\) then clearly \(\abs{\genset{g}} > 1\), so \(\abs{\genset{g}} = p\).
\end{proof}

\subsection{Number Theoretic Implications}
We can take Lagrange's theorem into the world of number theory, and specifically modular arithmetic, where we are dealing with finite groups. Clearly, \(\mathbb Z_n\) is a group under addition modulo \(n\), but what happens with multiplication modulo \(n\)? Clearly this is not a group --- for a start, 0 has no inverse. By removing all elements of the group that have no inverse, we obtain \(\mathbb Z_n^*\).

Note that for any \(x \in \mathbb Z_n\), \(x\) has a multiplicative inverse if and only if \(\HCF(x, n) = 1\), i.e. if \(x\) and \(n\) are coprime. This follows directly from the fact that we can write 1 as a linear combination of \(x\) and \(n\), i.e. \(xy + mn = 1\), thus defining \(y\) as the multiplicative inverse of \(x\) modulo \(n\). From this, it is simple to check that \(\mathbb Z_n^*\) forms a group under multiplication.

We may also create an equivalent group-theoretic definition of Euler's totient function \(\varphi\) as follows: \(\varphi(n) := \abs{\mathbb Z_n^*}\). We can now use Lagrange's theorem to prove the Fermat-Euler theorem (that is, \(\HCF(N, n) = 1 \implies N^{\varphi(n)} \equiv 1 \mod n\)) as follows.
\begin{proof}
	If \(N\) and \(n\) are coprime, then there is an element, here denoted \(a\), in \(\mathbb Z_n\) corresponding to \(N\). So \(a^{\varphi(n)} = a^{\abs{\mathbb Z_n^*}} = 1\) in \(\mathbb Z_n\). Since \(N = a + kn\), we may expand \(N^{\varphi(n)} = a^{\varphi(n)} + n(\cdots) \equiv a^{\varphi(n)} \equiv 1 \mod n\).
\end{proof}
