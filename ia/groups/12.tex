\subsection{First Isomorphism Theorem}
\begin{theorem}
	Let \(\varphi: G \to H\) be a homomorphism. Then \(\frac{G}{\ker \varphi} \cong \Im \varphi\).
\end{theorem}
\begin{proof}
	Define \(\overline \varphi: \frac{G}{\ker \varphi} \to \Im \varphi\) using \(g \ker \varphi \mapsto \varphi(g)\).
	\begin{itemize}
		\item (well-defined) If \(g_1 \ker \varphi = g_2 \ker \varphi\), then \(g_1 = g_2k\), for some \(k \in \ker \varphi\). Sp \(\overline\varphi(g_1 \ker \varphi) = \varphi(g_1) = \varphi(g_2k) = \varphi(g_2)\varphi(k) = \varphi(g_2) = \overline\varphi(g_2 \ker \varphi)\).
		\item (homomorphism) Let \(g, g' \in G\). \(\overline\varphi(g \ker \varphi \cdot g' \ker \varphi) = \overline\varphi(gg' \ker \varphi) = \varphi(gg') = \varphi(g)\varphi(g') = \overline\varphi(g\ker\varphi) \cdot \overline\varphi(g'\ker\varphi)\).
		\item (surjective) All elements of \(\Im \varphi\) are of the form \(\varphi(g)\) for some \(g \in G\), so clearly surjective.
		\item (injective) If \(\overline\varphi(g \ker \varphi) = e = \varphi(g)\) in \(\Im \varphi\) then \(g \in \ker \varphi\), so \(g \ker \varphi = \ker \varphi\).
	\end{itemize}
\end{proof}
This is a useful way to understand the first isomorphism theorem. Recall that \(\frac{G}{\ker \varphi}\) is really asking the question `how do the copies of \(\ker \varphi\) interact in \(G\)?' Well, as \(\varphi\) is a homomorphism, it represents some property that is true for members of a normal subgroup \(N\) in \(G\), where \(N = \ker \varphi\). Now, we can imagine the grid analogy from before, laying out several copies of \(N\) as rows. Let's call the group of these rows \(K\).

Now, multiplying together two rows, i.e. two elements from \(K\), we can apply the homomorphism \(\varphi\) to one of the coset representatives for each row to see how the entire row behaves under \(\varphi\). We know that all coset representatives give equal results, because each element in a given coset \(gN\) can be written as \(gn, n \in N\), so \(\varphi(gn) = \varphi(g)\). So all elements in the rows behave just like their coset representatives under the homomorphism. Further, all the cosets give different outputs under \(\varphi\) --- if they gave the same output they'd have to be part of the same coset. So in some sense, each row represents a distinct output for \(\varphi\). So the quotient group must be isomorphic to the image of the homomorphism.

Here are some examples.
\begin{enumerate}
	\item \(\det : GL_2(\mathbb R) \to \mathbb R^*\), \(\Im(\det) = \mathbb R^*\), \(\ker(\det) = SL_2(\mathbb R)\).Therefore, \(\frac{GL_2(\mathbb R)}{SL_2(\mathbb R)} \cong \mathbb R^*\).
	\item Consider the map \(\varphi: \mathbb R \to \mathbb C^*, \varphi(r) = e^{2\pi i r}\). This is a homomorphism because \(\varphi(r + s) = e^{2\pi i (r + s)} = e^{2 \pi i r}\cdot e^{2 \pi i s} = \varphi(r) \cdot \varphi(s)\). The image is the unit circle \(\abs{z} = 1\), denoted by \(S_1\); the kernel is \(\mathbb Z\) as \(e^{2 \pi i z}\) for some \(z \in \mathbb Z\), the result is 1. Therefore \(\frac{\mathbb R}{\mathbb Z} = S_1\).
\end{enumerate}

\subsection{Correspondence Theorem}
Now, let's try to understand how subgroups behave inside quotient groups.
\begin{theorem}
	Let \(N \trianglelefteq G\). The subgroups of \(\frac{G}{N}\) are in bijective correspondence with subgroups of \(G\) containing \(N\).
\end{theorem}
\begin{proof}
	Given \(N \leq M \leq G\), \(N \trianglelefteq G\), then \(N \trianglelefteq M\) and clearly \(\frac{M}{N} \leq \frac{G}{N}\). Conversely, for every subgroup \(H \leq \frac{G}{N}\), we can take the preimage of \(H\) under the quotient map \(\pi : G \to \frac{G}{N}\), i.e. \(\pi^{-1}(H) = \{ g \in G : gN \in H \}\). This is a subgroup of \(G\):
	\begin{itemize}
		\item (closure) if \(g_1, g_2 \in \pi^{-1}(H)\), then \(g_1g_2N = g_1N\cdot g_2N\) where both elements \(g_1N\) and \(g_2N\) are in \(H\). So \(g_1g_2N \in H\).
		\item (identity, inverses easy to check)
	\end{itemize}
	\(\pi^{-1}(H)\) contains \(N\), since \(\forall n \in N\), \(nN = N \in H\). Now we can check that for any \(N \leq M \leq G\), \(\pi^{-1}(\frac{M}{N}) = M\) and for \(H \leq \frac{G}{N}\), \(\frac{\pi^{-1}(H)}{N} = H\). So the correspondence is bijective (this satisfies the property that \(ff^{-1}\) and \(f^{-1}f\) are the identity maps on the relevant sets).
\end{proof}
This correspondence preserves lots of structure: for example, indices, normality, containment. One example is the group \(C_4 \times C_2\), where \(C_4 = \genset a\) and \(C_2 = \genset b\). The subgroups of this are (TODO draw subgroup lattice)
% TODO draw subgroup lattice
Now, let \(N := \genset{(a^2, b)}\). Note that this is normal because we are in an abelian group. Then, according to the above theorem, the subgroup lattice for \(\frac{C_4 \times C_2}{N}\) is bijective with the set of paths on the above lattice that terminate with \(N\) (i.e. have \(N\) as a subgroup).
% TODO draw second subgroup lattice
We took the quotient of a group of order 8 by a group of order 2, so \(N\) has order 4, so it must be isomorphic to \(C_4\) (as it has only one subgroup isomorphic to \(C_2\) as can be seen in the lattice, so it cannot be \(C_2 \times C_2\)).

\subsection{Second Isomorphism Theorem}
Let \(H\leq G\) and \(N\trianglelefteq G\), but \(N \nleq H\). We can actually still make a normal subgroup of \(H\) by intersecting \(H\) with \(N\).
\begin{theorem}
	Let \(H \leq G\) and \(N \trianglelefteq G\). Then \(H \cap N \trianglelefteq H\) and \(\frac{H}{H \cap N} \cong \frac{HN}{N}\).
\end{theorem}
\begin{proof}
	When \(N \trianglelefteq G, H \leq G\), then \(HN = \{ hn: h \in H, n \in N \}\) is a subgroup of \(G\), and \(HN = \genset{H, N}\).

	Consider the function \(\varphi: H \to \frac{HN}{N}, \varphi(h) := hN\). This is a well-defined surjective homomorphism. \(\varphi(h) = hN = N \iff h \in N\), but also \(h \in H\), so \(h \in N \cap H\) is the kernel. So by the First Isomorphism Theorem, \(\frac{H}{N \cap H} \cong \frac{HN}{N}\) (note that \(\frac{HN}{N} \leq \frac{G}{N}\)).
\end{proof}

\subsection{Third Isomorphism Theorem}
We noted earlier that normality is preserved inside quotient groups. We can say something analogous about quotients.
\begin{theorem}
	Let \(N \leq M \leq G\) such that \(N \trianglelefteq G\) and \(M \trianglelefteq G\). Then \(\frac{M}{N} \trianglelefteq \frac{G}{N}\), and \(\frac{G/N}{M/N} = \frac{G}{M}\).
\end{theorem}
\begin{proof}
	Let us define \(\varphi: \frac{G}{N} \to \frac{G}{M}\) by \(\varphi(gN) = gM\). \(\varphi\) is well defined since \(N \leq M\), and it is a surjective homomorphism. \(\varphi(gN) = gM = M \iff g \in M\), so its kernel is \(\frac{M}{N}\). By the First Isomorphism Theorem, \(\frac{G/N}{M/N} \cong \frac{G}{M}\).
\end{proof}
