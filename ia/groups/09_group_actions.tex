\subsection{Definition}
For many of the examples of groups that we have encountered, we have identified elements of that group by their effect on some set, for example the symmetric group \(S_n\) permuting the set \(\{ 1, \cdots, n \}\), and the M\"obius group being functions \(\Chat \to \Chat\), and the dihedral group \(D_{2n}\) being symmetries of an \(n\)-gon.
While we can study groups purely algebraically, it can be very useful to see how a group acts on other objects.

\begin{definition}
	Let \(G\) be a group, \(X\) be a set.
	An action of \(G\) on \(X\) is a function \(\alpha: G \times X \to X\), written
	\[
		\alpha(g, x) = \alpha_g(x)
	\]
	satisfying:
	\begin{itemize}
		\item \(\alpha_g(x) \in X\) (implied by the function's type)
		\item \(\alpha_e(x) = x;\; \forall x \in X\)
		\item \(\alpha_g \circ \alpha_h(x) = \alpha_{gh}(x);\; \forall g, h \in G, \forall x \in X\)
	\end{itemize}
	We can write \(G \acts X\).
\end{definition}
Here are some examples.
\begin{enumerate}
	\item Take any \(G\), \(X\) and define the trivial action \(\alpha_g(x) = x\).
	\item \(S_n \acts \{ 1, 2, \cdots, n \}\) by permutation.
	\item \(D_{2n} \acts \{ \text{vertices of a regular \(n\)-gon} \}\), and labelling the vertices as 1 to \(n\), we have \(D_{2n} \acts \{ 1, 2, \cdots, n \}\).
	\item \(\mathcal M \acts \Chat\) via M\"obius maps.
	\item Symmetries of a cube act on the set of vertices, the set of edges, and even (for example) the set of pairs of opposite faces of the cube.
\end{enumerate}
Examples 1, 2 show that more than one group can act on a given set.
Example 4 shows that one group can act on many sets.
Group actions help us deduce information about the group.

\begin{lemma}
	\(\forall g \in G, \alpha_g: X \to X, x \mapsto \alpha_g(x)\) is a bijection.
\end{lemma}
\begin{proof}
	We have that \(\alpha_g(\alpha_{g^{-1}}(x)) = \alpha_{g g^{-1}}(x) = \alpha_e(x) = x\).
	Similarly, \(\alpha_{g^{-1}}(\alpha_g(x)) = \alpha_{g^{-1} g}(x) = \alpha_e(x) = x\).
	So the composition \(\alpha_g \circ \alpha_{g^{-1}}\) is the identity on \(X\), and \(\alpha_{g^{-1}} \circ \alpha_g\) is also the identity on \(X\), so \(\alpha_g\) is a bijection.
\end{proof}
We can also define actions by linking \(G\) to \(\Sym(X)\).
\begin{proposition}
	Let \(G\) be a group, \(X\) a set.
	Then \(\alpha\colon G \times X \to X\) is an action if and only if the function \(\rho\colon G \to \Sym(X)\) where \(\rho(g) = \alpha_g\) is a homomorphism.
\end{proposition}
\begin{proof}
	\(\alpha\) is an action.
	By the above lemma, \(\alpha_g\) is a bijection from \(X \to X\).
	So \(\alpha_g \in \Sym(X)\).
	Now, we want to show that \(\rho\) is a homomorphism.
	\(\rho(gh) = \alpha_{gh}\), and for all \(x \in X\), \(\alpha_{gh}(x) = \alpha_g \circ \alpha_h (x)\), so \(\rho(gh) = \alpha_{gh} = \rho(g)\circ\rho(h)\).
	So \(\rho\) is a homomorphism.

	In the other direction, given that \(\rho\) is a homomorphism \(G \to \Sym(X)\), we can define an action \(\alpha\colon G \times X \to X\) by \(\alpha(g, x) = \alpha_g(x) := \rho(g)(x)\).
	\(\alpha\) is an action because \(\alpha_g \circ \alpha_h = \rho(g)\rho(h) = \rho(gh) = \alpha_{gh}\), and the identity element \(\rho(e)\) is the identity element in \(\Sym(X)\), so \(\alpha_e(x) = \rho(e)(x) = x\) as required.
\end{proof}
Sometimes we write \(g(x)\) instead of the more verbose \(\alpha_g(x)\).

\begin{definition}
	The kernel of an action \(\alpha\colon G \times X \to X\) is the kernel of the homomorphism \(\rho\colon G \to \Sym(X)\).
	These are all the elements of \(G\) that preserve every element of \(X\).
\end{definition}
Note that \(\frac{G}{\ker \rho} \cong \Im \rho \leq \Sym(X)\).
So in particular, if the kernel is trivial, then \(G \leq \Sym(X)\).
\begin{enumerate}
	\item \(D_{2n}\) acting on the vertices \(\{ 1, \cdots, n \}\) of an \(n\)-gon has \(\ker\rho = \{ e \}\).
	      Every non-trivial element of \(D_{2n}\) moves at least one vertex.
	      So \(D_{2n} \leq S_n\) by the First Isomorphism Theorem.
	\item Let \(G\) be symmetries of a cube, and consider \(X = \{ \text{unordered pairs of opposite faces} \}\).
	      Then \(\abs{X} = 3\) as there are three unordered pairs of opposite faces.
	      So \(\rho\colon G \to S_3\).
	      Clearly there are symmetries of the cube that realise all the permutations of \(X\), so \(\rho\) is surjective.
	      So \(\frac{G}{\ker \rho} \cong S_3\).
	      Note that there are clearly non-trivial symmetries (e.g.
	      reflection) that preserve \(X\), so the kernel is non-trivial.
\end{enumerate}
\begin{definition}
	An action \(G \acts X\) is called faithful if \(\ker \rho = \{ e \}\).
\end{definition}
Then \(G\) is isomorphic to a subgroup of \(\Sym X\) by the First Isomorphism Theorem.

\subsection{Orbits and stabilisers}
Which elements of \(X\) can we `get to' from a certain \(x \in X\) using the action of \(G\)?
\begin{definition}
	Let \(G \acts X\), \(x \in X\).
	The orbit of \(x\) is
	\[
		\Orb (x) = G(x) := \{ g(x) : g \in G \} \subseteq X
	\]
\end{definition}
\noindent Which group elements leave a given \(x\) unchanged?
\begin{definition}
	The stabiliser of \(x\) is defined by
	\[
		\Stab(x) = G_x := \{ g \in G : g(x) = x \} \subseteq G
	\]
\end{definition}
\begin{definition}
	An action is transitive if \(\Orb(x) = X\), i.e.\ we can get to any element from any other element.
\end{definition}
As an example, let \(G = S_3\).
Then we could say, for example, \(G \acts \{ 1, 2, 3, 4 \}\).
\begin{itemize}
	\item \(\Orb(1) = \Orb(2) = \Orb(3) = \{ 1, 2, 3 \}\)
	\item \(\Orb(4) = \{ 4 \}\)
	\item \(\Stab(1) = \{ e, (2\ 3) \}\)
	\item \(\Stab(2) = \{ e, (1\ 3) \}\)
	\item \(\Stab(3) = \{ e, (1\ 2) \}\)
	\item \(\Stab(4) = G\)
\end{itemize}

\subsection{Properties}
\begin{lemma}
	For any \(x \in X\), \(\Stab(x) \leq G\).
\end{lemma}
\begin{proof}
	Associativity is inherited.
	\begin{itemize}
		\item (closure) \(g, h \in \Stab(x)\) implies that \((gh)(x) = g(h(x)) = g(x) = x\) so \(gh \in \Stab(x)\).
		\item (identity) \(e(x) = x\) by definition, so \(e \in \Stab(x)\).
		\item (inverses) if \(g \in \Stab(x)\) then \(g(x) = x\), and therefore \(x = g^{-1}(x)\), so \(g^{-1} \in \Stab(x)\).
	\end{itemize}
\end{proof}
Recall from Numbers and Sets: a partition of a set \(X\) is a set of subsets of \(X\) such that each \(x \in X\) belongs to exactly one subset in the partition.
\begin{lemma}
	Let \(G \acts X\).
	Then the orbits partition \(X\).
\end{lemma}
\begin{proof}
	\begin{itemize}
		\item Firstly, for any \(x \in X\), \(x \in \Orb(x)\).
		      So the union of all orbits is \(X\).
		\item Suppose that the orbits are not all disjoint.
		      Let \(z \in \Orb(x) \cap \Orb(y)\).
		      Then \(\exists g_1 \in G\) such that \(g_1(x) = z\), and also \(\exists g_2 \in G\) such that \(g_2(y) = z\), i.e.\ \(y = g_2^{-1}(z)\).
		      So \(y = g_2^{-1}g_1(x)\).
		      Thus, for any \(g \in G\), \(g(y) = gg_2^{-1}g_1(x) \in \Orb(x)\) so \(\Orb(y) \subseteq \Orb(x)\).
		      Vice versa, \(\Orb(x) \subseteq \Orb(y)\), so \(\Orb(x) = \Orb(y)\).
		      Thus orbits are either disjoint or equal.
	\end{itemize}
\end{proof}
Recall the proof of disjoint cycle notation for \(\sigma \in S_n\): we were really finding the orbits in \(\{ 1, 2, \cdots, n \}\) under \(\genset \sigma\), which are disjoint.
Note that the sizes of orbits can be different (unlike cosets, where the sizes are always the same).
\begin{theorem}[Orbit-Stabiliser Theorem]
	Let \(G \acts X\), \(G\) finite.
	Then for any \(x \in X\),
	\[
		\abs{G} = \abs{\Orb x} \cdot \abs{\Stab x}
	\]
\end{theorem}
\begin{proof}
	\(g(x) = h(x) \iff h^{-1}g(x) = x \iff h^{-1}g \in \Stab(x)\).
	By a previous result, this statement is true if and only if
	\(g \Stab(x) = h \Stab(x)\) as cosets.
	So distinct points in the orbit of \(x\) are in bijection with distinct cosets of the stabiliser.
	So \(\abs{\Orb x} = \abs{G : \Stab x}\) and the result follows.
\end{proof}
In particular, notice that all elements in a given coset \(g \Stab(x)\) do the same thing to \(x\) as \(g\): an element of this coset has the form \(gh\) where \(h \in \Stab(x)\).
Then \(gh(x) = g(x)\).

This theorem is very powerful, we can use it for investigating groups further.
For example, we can construct another proof that \(\abs{D_{2n}} = 2n\) using the Orbit-Stabiliser theorem.
\(D_{2n}\) acts transitively on \(\{1, 2, \cdots, n\}\) so \(\abs{\Orb(1)} = n\).
\(\abs{\Stab(1)} = 2\) because only the identity and the reflection through this point stabilise the point.
So \(\abs{D_{2n}} = 2n\).

\subsection{Symmetries of the tetrahedron}
A tetrahedron has 4 faces (regular, equilateral triangles), 4 vertices, and 6 edges.
We will label the vertices \(1, 2, 3, 4\).
Let \(G\) be the group of symmetries of the tetrahedron.
Clearly \(G\) acts transitively on the vertices (we can get from any vertex to any other through a symmetry).
There is no non-trivial symmetry that fixes all the vertices, so \(\rho\colon G \to S_4\) is an injective homomorphism.

\(\Orb(1) = \{ 1, 2, 3, 4 \}\) as \(G\) is transitive.
\(\Stab(1) =\) all of the symmetries of the face \(\{2,3,4\}\), i.e.
\[
	\Stab(1) = \{ e, (2\ 3\ 4), (2\ 4\ 3), (2\ 3), (3\ 4), (2\ 4) \} \cong D_6 \cong S_3
\]
Then \(\abs{G} = \abs{\Orb(1)} \cdot \abs{\Stab(1)} = 4 \cdot 6 = 24 = \abs{S_4}\).
Since \(G \leq S_4\) and their orders match, \(G = S_4\).

Now let \(G^+\) be the subgroup of \(G\) formed only of the rotations in \(G\).
Again, \(\Orb(1) = \{ 1, 2, 3, 4 \}\).
Now, \(\Stab(1) = \{ e, (2\ 3\ 4), (2\ 4\ 3) \}\).
So \(\abs{G^+} = \abs{\Orb(1)} \cdot \abs{\Stab(1)} = 4 \cdot 3 = 12\).
Since \(G^+ \leq G = S_4\), then we know that \(G^+ = A_4\).
Indeed, we have all 3-cycles (since these are rotations through vertices), and all elements of the form \((1\ 2)(3\ 4)\) since these are rotations in the axis through the midpoints of opposite edges.

\subsection{Symmetries of the cube}
We label the vertices from 1 to 8 here, and let \(G\) be the group of symmetries of the cube acting on the vertices.
Clearly the action is transitive, so \(\abs{\Orb(1)} = 8\).
\(\Stab(1) = \{ e, r, r^2, s_1, s_2, s_3 \}\) where \(r\) and \(r^2\) are the rotations through the axis that passes through vertex 1, and where the \(s_i\) are the reflections through three planes containing vertex 1.
So \(\abs{\Stab(1)} = 6\), so \(\abs{G} = 48\).
We will determine this group completely later on.

Let \(G^+\) be the subgroup of \(G\) containing the rotations of \(G\).
Then, the action is still transitive, and \(\abs{\Stab(1)} = 3\), since we are only looking at the rotations.
So \(\abs{G^+} = 24\).

Now, to determine this group, let \(G^+\) act on the 4 diagonals in the cube.
This gives us a homomorphism \(\rho\colon G^+ \to S_4\).
We have all 4-cycles in \(\Im \rho\), since rotating the cube by quarter turns through the \(x, y, z\) axes permute the diagonals in this way.
We also have all transpositions (2-cycles) by rotating the cube by a half turn through the plane of two diagonals.
In example sheet 2, we prove that \(\genset{(1\ 2), (1\ 2\ 3\ 4)} = S_4\), so \(\rho\) is surjective.
But since the orders match, \(G^+ \cong S_4\).

\subsection{Platonic solids}
The aforementioned solids are two of the five Platonic solids; the solids in \(\mathbb R^3\) that have polygonal faces, straight edges and vertices such that their group of symmetries acts transitively on triples (vertex, incident edge, incident face).
These are therefore particularly symmetric solids for having this transitive action.
The other solids are the octahedron, dodecahedron and icosahedron.
The cube and octahedron are `dual', i.e.\ they can be inscribed in each other with vertices placed in the centres of faces.
The dodecahedron and icosahedron are also dual.
Dual solids have the same symmetry groups, so there are only three symmetry groups of Platonic solids.

\subsection{Cauchy's theorem}
\begin{theorem}
	Let \(G\) be a finite group, \(p\) a prime such that \(p \mid \abs{G}\).
	Then \(G\) has an element of order \(p\).
\end{theorem}
\begin{proof}
	Let \(p \mid \abs{G}\).
	Consider \(G^p = G \times G \times \cdots \times G\).
	This is the group formed of \(p\)-tuples of elements of \(G\) with coordinate-wise composition.
	Consider the subset \(X \subseteq G^p\), given by
	\[
		X := \{ (g_1, g_2, \cdots, g_p) \in G^p: g_1g_2\cdots g_p = e \}
	\]
	which can be described as `\(p\)-tuples multiplying to \(e\)'.
	Note that if \(g \in G\) has order \(p\), then \((g, g, \cdots, g) \in X\); and that if \((g, g, \cdots, g) \in X\) where \(g \neq e\), then \(g\) has order \(p\).

	Now take a cyclic group \(C_p = \genset a\), and let \(C_p \acts X\) by `cycling':
	\[
		a(g_1, g_2, \cdots, g_p) = (g_2, \cdots, g_p, g_1)
	\]
	This really is an action:
	\begin{itemize}
		\item If \(g_1g_2 \cdots g_p = e\), then \(e = g_1^{-1} e g_1 = g_1^{-1}g_1g_2 \cdots g_p g_1 = g_2 \cdots g_p g_1\) as required.
		      Of course, this applies inductively for any power of \(a\).
		\item \(e(g_1, \cdots, g_p) = (g_1, \cdots, g_p)\) as required.
		\item \(a^k(g_1, \cdots, g_p) = (g_{k+1}, \cdots, g_k) = a \cdot a \cdots a(g_1, \cdots, g_k)\).
	\end{itemize}
	Since orbits partition \(X\), the sum of the sizes of the orbits must be \(\abs{X}\).
	We know that \(\abs{X} = \abs{G}^{p-1}\), since all choices of \(g_i\) are free apart from the last one, which must be the inverse of the product of the other elements.
	So we have \(p-1\) choices of \(\abs{G}\) elements, so \(\abs{X} = \abs{G}^{p-1}\).

	So since \(p \mid \abs{G}\), then \(p \mid \abs{X}\).
	By the Orbit-Stabiliser theorem:
	\[
		\abs{\Orb((g_1, \cdots, g_p))} \cdot \abs{\Stab((g_1, \cdots, g_p))} = \abs{C_p} = p
	\]
	So any orbit has size 1 or \(p\), and they sum to \(\abs{X} = pk\) for some \(k \in \mathbb N\).
	So
	\[
		\abs{X} = pk = \sum_{\text{orbits of size 1}} 1 + \sum_{\text{orbits of size \(p\)}} p
	\]
	Clearly, \(\abs{\Orb((e, e, \cdots, e))} = 1\).
	So there must be some other orbits of size 1, so that \(p\) divides the amount of orbits of size 1.
	But orbits of size 1 must be of the form \(\Orb((g, g, \cdots, g))\) in order to have the same form under the action of \(a\).
	So there exists some \(g \neq e \in G\) such that \((g, g, \cdots g) \in X\), i.e.\ \(g^p = e\), so \(o(g) = p\).
\end{proof}

\subsection{Left regular action}
\begin{lemma}
	Let \(G\) be a group.
	\(G\) acts on itself by left multiplication.
	This action is faithful and transitive.
\end{lemma}
\begin{proof}
	\begin{itemize}
		\item For any \(g, x \in G\), \(gx \in G\)
		\item \(e(x) = e \cdot x = x\)
		\item \((g_1 g_2) x = g_1 (g_2 x)\)
	\end{itemize}
	So it really is an action.
	It is faithful because \(g(x) = gx = x\) implies \(g = e\).
	It is transitive, because given any \(x, y \in G\), the action \(g = yx^{-1}\) gives \(g(x) = y\).
\end{proof}
\begin{definition}
	This left-multiplication action of a group on itself is known as the left regular action.
\end{definition}

\subsection{Cayley's theorem}
\begin{theorem}
	Every group is isomorphic to a subgroup of a symmetric group.
\end{theorem}
\begin{proof}
	Let \(G \acts G\) by the left regular action.
	This gives a homomorphism \(\rho\colon G \to \Sym(G)\), with \(\ker \rho = \{ e \}\) since the action is faithful.
	So, by the First Isomorphism Theorem, \(\frac{G}{\ker \rho} = G \cong \Im \rho \leq \Sym(G)\).
\end{proof}

\begin{proposition}
	Let \(H \leq G\).
	Then \(G\) acts on the set of left cosets of \(H\) in \(G\) by left multiplication, and this action is transitive.
	(This is called the `left coset action').
\end{proposition}
\begin{proof}
	We check the conditions for actions.
	\begin{itemize}
		\item \(g(g_1H) = gg_1H\), so \(g(g_1H)\) is a left coset.
		\item \(e(g_1G) = eg_1H = g_1H\)
		\item \((gg')(g_1H) = gg'g_1H = g(g'(g_1H))\)
	\end{itemize}
	So this is an action.
	Given two cosets \(g_1H\) and \(g_2H\), the element \((g_1g_2^{-1})\) acts on \(g_2H\) to give \(g_1H\), so it is transitive.
\end{proof}
Note:
\begin{itemize}
	\item This is the left regular action if \(H = \{ e \}\).
	\item This induces actions of \(G\) on its quotient groups \(\frac{G}{N}\).
\end{itemize}
