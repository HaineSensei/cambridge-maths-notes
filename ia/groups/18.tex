\subsection{Splitting Conjugacy Classes}
Some conjugacy classes of \(S_n\) are split into smaller conjugacy classes in \(A_n\), since some elements require elements of \(S_n \setminus A_n\) to conjugate with each other.
By the orbit-stabiliser theorem,
\[
	\abs{S_n} = \abs{\ccl_{S_n}(\sigma)} \cdot \abs{C_{S_n}(\sigma)}
\]
\[
	\abs{A_n} = \abs{\ccl_{A_n}(\sigma)} \cdot \abs{C_{A_n}(\sigma)}
\]
But \(\abs{S_n} = 2\abs{A_n}\), and \(\abs{\ccl_{S_n}(\sigma)} \geq \abs{\ccl_{A_n}(\sigma)}\).
So either:
\begin{itemize}
	\item \(\ccl_{S_n}(\sigma) = \ccl_{A_n}(\sigma)\) and \(\abs{C_{S_n}(\sigma)} = 2\abs{C_{A_n}(\sigma)}\), or
	\item \(\abs{\ccl_{S_n}(\sigma)} = 2 \abs{\ccl_{A_n}(\sigma)}\) and \(C_{S_n}(\sigma) = C_{A_n}(\sigma)\)
\end{itemize}
\begin{definition}
	When \(\abs{\ccl_{S_n}(\sigma)} = 2 \abs{\ccl_{A_n}(\sigma)}\), we say that the conjugacy class of \(\sigma\) splits in \(A_n\).
\end{definition}
When does a conjugacy class split in \(A_n\)?
\begin{proposition}
	The conjugacy class of \(\sigma \in A_n\) splits in \(A_n\) if and only if there are no odd permutations that commute with \(\sigma\).
\end{proposition}
\begin{proof}
	\[
		\abs{\ccl_{S_n}(\sigma)} = 2 \abs{\ccl_{A_n}(\sigma)} \iff C_{S_n}(\sigma) = C_{A_n}(\sigma)
	\]
	\[
		C_{A_n}(\sigma) = A_n \cap C_{S_n}(\sigma)
	\]
	\[
		A_n \cap C_{S_n}(\sigma) = C_{S_n}(\sigma) \iff C_{S_n}(\sigma)\text{ contains no odd elements}
	\]
	So no odd permutation is in this centraliser.
\end{proof}
Let us consider an example for conjugacy classes in \(A_4\).\medskip

\noindent\begin{tabular}{ccccc}
	cycle type  & example element        & odd element in \(C_{S_4}\)?
	            & size of \(\ccl_{S_4}\) & size of \(\ccl_{A_4}\)                                  \\\midrule
	\(1,1,1,1\) & \(e\)                  & yes, e.g.
	\((1\ 2)\)  & 1                      & 1                                                       \\
	\(2,2\)     & \((1\ 2)(3\ 4)\)       & yes, e.g.
	\((1\ 2)\)  & 3                      & 3                                                       \\
	\(3,1\)     & \((1\ 2\ 3)\)          & no                          & 8 & two classes of size 4 \\
\end{tabular}

\medskip\noindent There is no odd element in \(C_{S_4}(1\ 2\ 3)\) because \(\abs{C_{S_4}(1\ 2\ 3)} = 3\) and clearly \(C_{S_4}\) contains \(\genset{(1\ 2\ 3)}\), which is a set of 3 elements, so \(C_{S_4} = \genset{(1\ 2\ 3)}\) which are all even elements.

Let us now consider conjugacy classes in \(A_5\).\medskip

\noindent\begin{tabular}{ccccc}
	cycle type    & example element        & odd element in \(C_{S_5}\)?
	              & size of \(\ccl_{S_5}\) & size of \(\ccl_{A_5}\)                                    \\ \midrule
	\(1,1,1,1,1\) & \(e\)                  & yes, e.g.
	\((1\ 2)\)    & 1                      & 1                                                         \\
	\(2,2,1\)     & \((1\ 2)(3\ 4)\)       & yes, e.g.
	\((1\ 2)\)    & 15                     & 15                                                        \\
	\(3,1,1\)     & \((1\ 2\ 3)\)          & yes, e.g.
	\((4\ 5)\)    & 20                     & 20                                                        \\
	\(5\)         & \((1\ 2\ 3\ 4\ 5)\)    & no                          & 24 & two classes of size 12 \\
\end{tabular}

\medskip\begin{lemma}
	\(C_{S_5}(1\ 2\ 3\ 4\ 5) = \genset{(1\ 2\ 3\ 4\ 5)}\).
\end{lemma}
\begin{proof}
	\[
		\abs{\ccl_{S_5}(1\ 2\ 3\ 4\ 5)} = \frac{5 \cdot 4 \cdot 3 \cdot 2}{5} = 24
	\]
	By the orbit-stabiliser theorem,
	\[
		\abs{S_5} = 120 = 24 \abs{C_{S_5}(1\ 2\ 3\ 4\ 5)} \implies \abs{C_{S_5}(1\ 2\ 3\ 4\ 5)} = 5
	\]
	Clearly \(\genset{(1\ 2\ 3\ 4\ 5)} \subseteq C_{S_5}(1\ 2\ 3\ 4\ 5)\) so \(\genset{(1\ 2\ 3\ 4\ 5)} = C_{S_5}(1\ 2\ 3\ 4\ 5)\).
	Note, this contains only even elements.
\end{proof}

\begin{theorem}
	\(A_5\) is a simple group.
\end{theorem}
\begin{proof}
	Normal subgroups must be unions of conjugacy classes, they must contain \(e\), and their order must divide the order of the group \(\abs{A_5} = 60\).
	The sizes of conjugacy classes we have are \(1, 15, 20, 12, 12\) from the example above.
	The only ways of adding 1 plus some of the other numbers to get a divisor of 60 are
	\begin{itemize}
		\item (1) which can only be the trivial subgroup
		\item (\(1+15+20+12+12=60\)) which can only be the group itself
	\end{itemize}
	So those are the only possible normal subgroups, so it is simple.
\end{proof}
\begin{remark}
	All \(A_n\) for \(n \geq 5\) are simple.
\end{remark}

\subsection{The M\"obius Group}
We can now study the action of the M\"obius group \(\mathcal M\), which is the group of M\"obius maps
\[
	f\colon \hat{\mathbb C} \to \hat{\mathbb C};\quad f(z) = \frac{az+b}{cz+d};\quad a,b,c,d\in\mathbb C;\quad ad-bc\neq 0;\quad \frac{1}{0}=\infty;\quad\frac{1}{\infty}=0
\]
\begin{remark}
	The above definition defines an action \(M \acts \hat{\mathbb C}\).
\end{remark}
\begin{proposition}
	The action \(M \acts \hat{\mathbb C}\) is faithful (the only elements acting as the identity are the identity), and so \(\mathcal M \leq \Sym(\hat{\mathbb C})\).
\end{proposition}
\begin{proof}
	Consider \(\rho\colon \mathcal M \to \Sym(\hat{\mathbb C})\) given by \(\rho(f)(z) = f(z)\).
	Then if \(\rho(f) = e_{\Sym(\hat{\mathbb C})}\) (the function \(z \mapsto z\)) then \(f\) is the identity \(e_{\mathcal M}\).
	So \(\rho\) is injective and the action is faithful.
\end{proof}
\begin{definition}
	A fixed point of a M\"obius map \(f\) is a point \(z\) such that \(f(z) = z\).
\end{definition}
\begin{theorem}
	A M\"obius map with at least three fixed points is the identity.
\end{theorem}
\begin{proof}
	Let \(f(z) = \frac{az+b}{cz+d}\) have at least three fixed points.
	\begin{itemize}
		\item If \(\infty\) is not a fixed point, then the equation \(\frac{az+b}{cz+d} = z\) is true for at least three complex numbers.
		      Rewritten,
		      \[
			      cz^2 + (d-a)z-b=0
		      \]
		      By the fundamental theorem of algebra, this can only have at most two distinct roots.
		      So we must have \(c=b=0, d=a\), i.e.\ \(f(z) = z\).
		\item If \(\infty\) is a fixed point, then \(\frac{a\infty + b}{c\infty + d} = \frac{a}{c} = \infty\) so \(c = 0\).
		      So for the other two fixed points, \(\frac{az+b}{d} = z\) for at least two complex numbers.
		      Rewritten,
		      \[
			      (a-d)z+b=0
		      \]
		      By the fundamental theorem of algebra, this can only have one root.
		      So we must have \(a=d,b=0\), i.e.\ \(f(z) = z\).
	\end{itemize}
\end{proof}
\begin{corollary}
	If two M\"obius maps coincide on three distinct points in \(\hat{\mathbb C}\), then they must be equal.
\end{corollary}
\begin{proof}
	Let \(f, g \in \mathcal M\) be such that \(f(z_1) = g(z_1)\), \(f(z_2) = g(z_2)\), \(f(z_3) = g(z_3)\) for three distinct points \(z_1, z_2, z_3 \in \hat{\mathbb C}\).
	Then \(g^{-1}f(z_i) = z_i\) for the same three distinct points.
	So \(g^{-1}f\) is the identity by the theorem above, so \(g = f\).
\end{proof}
In less formal words, we can say `knowing what a M\"obius map does to 3 points determines it'.
