\subsection{All Transformations in \(O_2\)}
\begin{theorem}
	Every element of \(SO_2\) is of the form
	\[
		\begin{pmatrix}
			\cos\theta & -\sin\theta \\
			\sin\theta & \cos\theta
		\end{pmatrix}
	\]
	for some \(\theta \in [0, 2\pi)\).
	
	This is an anticlockwise rotation of \(\mathbb R^2\) about the origin by angle \(\theta\).
	Conversely, every such element lies in \(SO_2\).
\end{theorem}
\begin{proof}
	Let
	\[
		A = \begin{pmatrix}
			a & b \\ c & d
		\end{pmatrix} \in SO_2
	\]
	We have \(A^\transpose A = I\) and \(\det A = 1\).
	So
	\[
		A^\transpose = A^{-1} \implies \begin{pmatrix}
			a & c \\ b & d
		\end{pmatrix} = \frac{1}{1} \begin{pmatrix}
			d & -b \\ -c & a
		\end{pmatrix}
	\]
	So \(a=d, b=-c\).
	Since \(ad-bc=1\), \(a^2+c^2=1\).
	Then we can write \(a = \cos \theta\) and \(c = \sin \theta\) for a unique \(\theta \in [0, 2\pi)\).
	
	Conversely, the determinant of this matrix is 1, and is in \(O_2\), so this element lies in \(SO_2\).
\end{proof}
\begin{theorem}
	The elements of \(O_2 \setminus SO_2\) are the reflections in lines through the origin.
\end{theorem}
\begin{proof}
	Let
	\[
		A = \begin{pmatrix}
			a & b \\ c & d
		\end{pmatrix} \in O_2 \setminus SO_2
	\]
	So \(A^\transpose A = I\) and \(\det A = -1\).
	\[
		A^\transpose = A^{-1} \implies \begin{pmatrix}
			a & c \\ b & d
		\end{pmatrix} = \frac{1}{-1} \begin{pmatrix}
			d & -b \\ -c & a
		\end{pmatrix}
	\]
	So \(a=-d, b=c\).
	Together with \(ad-bc=-1\), we have \(a^2 + c^2 = 1\).
	So let \(a = \cos \theta\), \(c = \sin \theta\) like before, so
	\[
		A = \begin{pmatrix}
			\cos \theta & \sin \theta  \\
			\sin \theta & -\cos \theta
		\end{pmatrix}
	\]
	which can be shown to be a reflection using double angle formulas such that
	\[
		A \begin{pmatrix}
			\sin \frac{\theta}{2} \\ \cos \frac{\theta}{2}
		\end{pmatrix} = -\begin{pmatrix}
			\sin \frac{\theta}{2} \\ \cos \frac{\theta}{2}
		\end{pmatrix};\quad A\begin{pmatrix}
			\cos \frac{\theta}{2} \\ \sin \frac{\theta}{2}
		\end{pmatrix} = \begin{pmatrix}
			\cos \frac{\theta}{2} \\ \sin \frac{\theta}{2}
		\end{pmatrix}
	\]
	So \(A\) is a reflection in the plane orthogonal to the vector \(\begin{pmatrix}
		\sin \frac{\theta}{2} \\ \cos \frac{\theta}{2}
	\end{pmatrix}\).
	Conversely, any reflection in a line through the origin has this form, so it will be in \(O_2 \setminus SO_2\).
\end{proof}

\begin{corollary}
	Every element of \(O_2\) is the composition of at most two reflections.
\end{corollary}
\begin{proof}
	Every element of \(O_2 \setminus SO_2\) is a reflection, so this is trivial.
	If \(A \in SO_2\), then we can write
	\[
		A = \underbrace{A \begin{pmatrix}
				-1 & 0 \\ 0 & 1
			\end{pmatrix}}_{\det = -1} \underbrace{\begin{pmatrix}
				-1 & 0 \\ 0 & 1
			\end{pmatrix}}_{\det = -1}
	\]
	So we have expressed \(A\) as the product of two reflections.
\end{proof}

\subsection{All Transformations in \(O_3\)}
\begin{theorem}
	If \(A \in SO_3\), then there exists some unit vector \(\vb v \in \mathbb R^3\) with \(A\vb v = \vb v\), i.e.\ there exists an eigenvector with eigenvalue 1.
\end{theorem}
\begin{proof}
	It is sufficient to show that 1 is an eigenvalue of \(A\), since this guarantees that there is some nonzero eigenvector for this eigenvalue which we can then normalise.
	This is equivalent to showing that \(\det (A - I) = 0\).
	\begin{align*}
		\det(A - I) & = \det(A - AA^\transpose)       \\
		            & = \det(A)\det(I - A^\transpose) \\
		            & = \det(I - A^\transpose)        \\
		            & = \det((I - A)^\transpose)      \\
		            & = \det(I - A)                   \\
		            & = (-1)^3\det(A - I)
	\end{align*}
	So \(2\det(A - I) = 0 \implies \det(A - I) = 0\).
\end{proof}
\begin{corollary}
	Every element \(A \in SO_3\) is conjugate (in \(SO_3\)) to a matrix of the form
	\[
		\begin{pmatrix}
			1 & 0           & 0            \\
			0 & \cos \theta & -\sin \theta \\
			0 & \sin \theta & \cos \theta
		\end{pmatrix}
	\]
\end{corollary}
\begin{proof}
	By the above theorem, there exists some unit vector \(\vb v_1\) which is an eigenvector of eigenvalue 1.
	We can extend this vector to an orthonormal basis \(\{ \vb v_1, \vb v_2, \vb v_3 \}\) of \(\mathbb R^3\).
	Then, for \(i=2,3\), we have
	\[
		A\vb v_i \cdot \vb v_1 = A\vb v_i \cdot A\vb v_1 = \vb v_i \cdot \vb v_1 = 0
	\]
	So \(A\vb v_2, A\vb v_3\) lie in the subspace generated by \(\vb v_2, \vb v_3\), i.e.\ \(\vecspan \{ \vb v_2, \vb v_3 \} = \genset{\vb v_2, \vb v_3}\).
	So \(A\) maps this subspace to itself, and we can thus consider the restriction of \(A\) to this subspace.
	The matrix in this new basis will have form
	\[
		\begin{pmatrix}
			1 & 0 & 0 \\
			0 & a & b \\
			0 & c & d
		\end{pmatrix}
	\]
	The smaller matrix in the bottom right will still have determinant 1, since we can expand the determinant here by the first row.
	So \(A\) restricted to this subspace is an element of \(SO_2\), so its matrix must be of the form
	\[
		\begin{pmatrix}
			a & b \\ c & d
		\end{pmatrix} = \begin{pmatrix}
			\cos \theta & -\sin \theta \\
			\sin \theta & \cos \theta
		\end{pmatrix}
	\]
	So \(A\) has the required form with respect to this new basis \(\{ \vb v_1, \vb v_2, \vb v_3 \}\).
	The change of basis matrix \(P\) lies in \(O_3\) since \(\{ \vb v_1, \vb v_2, \vb v_3 \}\) is an orthonormal basis.
	If \(P \notin SO_3\), then we can use the basis \(\{ -\vb v_1, \vb v_2, \vb v_3 \}\) instead, which will invert the determinant of \(P\).
	So in either case \(P \in SO_3\).
\end{proof}
This tells us in particular that every element in \(SO_3\) is a rotation about some axis, here \(\vb v_1\).

\begin{corollary}
	Every element of \(O_3\) is the composition of at most three reflections.
\end{corollary}
\begin{proof}
	\begin{itemize}
		\item If \(A \in SO_3\), then \(\exists P \in SO_3\) such that \(PAP^{-1} = B\), where \(B\) is of the form
		      \[
			      B = \begin{pmatrix}
				      1 & 0           & 0            \\
				      0 & \cos \theta & -\sin \theta \\
				      0 & \sin \theta & \cos \theta
			      \end{pmatrix}
		      \]
		      Since this smaller matrix
		      \[
			      \begin{pmatrix}
				      \cos \theta & -\sin \theta \\
				      \sin \theta & \cos \theta
			      \end{pmatrix}
		      \]
		      is a composition of at most two reflections, then \(B\) is also a composition of at most two reflections, i.e.\ \(B = B_1 B_2\).
		      Since \(A\) is a conjugate of \(B\), it is also a composition of at most two reflections, as the conjugate of a reflection is a reflection, and \(A = P^{-1}BP = (P^{-1}B_1P)(P^{-1}B_2P)\).
		\item If \(A \in O_3 \setminus SO_3\), then \(\det A = -1\) and we can construct
		      \[
			      A = \underbrace{A\begin{pmatrix}
					      -1 & 0 & 0 \\
					      0  & 1 & 0 \\
					      0  & 0 & 1
				      \end{pmatrix}}_{\det = 1}\underbrace{\begin{pmatrix}
					      -1 & 0 & 0 \\
					      0  & 1 & 0 \\
					      0  & 0 & 1
				      \end{pmatrix}}_{\det = -1}
		      \]
		      So the left-hand product lies in \(SO_3\), so it is a composition of at most two reflections.
		      The final element is a reflection in the \(y\)--\(z\) plane, so the combined product is a composition of at most three reflections.
	\end{itemize}
\end{proof}

\subsection{Symmetries of the Cube (revisited)}
We can think of symmetry groups of the Platonic solids as subgroups of \(O_3\) by placing the solid at the origin.
By question 11 on example sheet 4, we have that \(O_3 \cong SO_3 \times C_2\), where \(C_2\) is generated by the map \(\vb v \mapsto -\vb v\).
So if \(\vb v\mapsto -\vb v\) is a symmetry of our platonic solid, then this group of symmetries will also split as the direct product of \(G^+ \times C_2\) where \(G^+\) is the group of rotations (proof as exercise).

So we have that the group of symmetries of the cube is \(G^+ \times C_2 \cong S_4 \times C_2\) by the results from earlier.
