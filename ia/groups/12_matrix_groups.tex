\subsection{Definitions}
We will look at various groups of matrices, their related actions, and study distance-preserving maps on \(\mathbb R^2\) and \(\mathbb R^3\).
Here are some examples of matrix groups.
\begin{itemize}
	\item \(M_{n \times n}(\mathbb F)\) is the set of \(n \times n\) matrices over the field \(\mathbb F\).
	\item \(GL_n(\mathbb F)\) is the set of \(n \times n\) matrices over \(\mathbb F\) which are invertible.
	      This is known as the general linear group over \(\mathbb F\).
	      \begin{itemize}
		      \item \(GL_n(\mathbb F)\) is a group under multiplication.
		      \item \(\det\colon GL_n(\mathbb F) \to \mathbb F^\times := \mathbb F \setminus \{ 0 \}\) is a surjective homomorphism.
		      \item Given \(A \in GL_n(\mathbb R)\), \(A^\transpose\) is the matrix with entries \((A^\transpose)_{ij} = A_{ji}\).
		            It satisfies
		            \begin{itemize}
			            \item \((AB)^\transpose = B^\transpose A^\transpose\)
			            \item \((A^{-1})^\transpose = (A^\transpose)^{-1}\)
			            \item \(AA^\transpose = I \iff A^\transpose A = I \iff A^\transpose = A^{-1}\)
			            \item \(\det A^\transpose = \det A\)
		            \end{itemize}
	      \end{itemize}
	\item \(SL_n(\mathbb F) \leq GL_n(\mathbb F)\) is the kernel of the \(\det\) homomorphism.
	      This is the special linear group.
	\item \(O_n = O_n(\mathbb R) := \{ A \in GL_n(\mathbb R): A^\transpose A = I \}\) is the orthogonal group.
	      We can check the group axioms to verify it is a subgroup of \(GL_n(\mathbb R)\).
	\item \(SO_n \leq O_n\) is the kernel of the \(\det\) homomorphism.
	      This is the special orthogonal group.
\end{itemize}
\begin{proposition}
	\(\det \colon O_n \to \{ \pm 1 \}\) is a surjective homomorphism.
\end{proposition}
\begin{proof}
	If \(A \in O_n\), then \(A^\transpose A = I\).
	So \((\det A)^2 = \det A^\transpose \cdot \det A = \det(A^\transpose A) = \det I = 1\).
	So \(\det A = \pm 1\).
	It is surjective since \(\det I = 1\), and the determinant of the matrix similar to the identity but one of the diagonal entries is \(-1\) has determinant \(-1\).
\end{proof}

\subsection{Matrix encoding of M\"obius maps}
\begin{proposition}
	The function \(\varphi\colon SL_2(\mathbb C) \to \mathcal M\) mapping
	\[
		\begin{pmatrix}
			a & b \\ c & d
		\end{pmatrix} \mapsto f;\quad f(z) = \frac{az + b}{cz + d}
	\]
	is a surjective homomorphism with kernel \(\{ I, -I \}\).
\end{proposition}
\begin{proof}
	Firstly, \(\varphi\) is a homomorphism.
	If \(f_1(z) = \frac{a_1z+b_1}{c_1z+d_1}, f_2(z) = \frac{a_2z+b_2}{c_2z+d_2}\), then we have seen that \(f_2(f_1(z))\) can be written in the form \(\frac{az+b}{cz+d}\) where
	\[
		\begin{pmatrix}
			a & b \\ c & d
		\end{pmatrix} = \begin{pmatrix}
			a_2 & b_2 \\ c_2 & d_2
		\end{pmatrix}\begin{pmatrix}
			a_1 & b_1 \\ c_1 & d_1
		\end{pmatrix}
	\]
	So
	\[
		\varphi\left( \begin{pmatrix}
				a_2 & b_2 \\ c_2 & d_2
			\end{pmatrix}\begin{pmatrix}
				a_1 & b_1 \\ c_1 & d_1
			\end{pmatrix} \right) = \varphi \begin{pmatrix}
			a_2 & b_2 \\ c_2 & d_2
		\end{pmatrix} \cdot \varphi \begin{pmatrix}
			a_1 & b_1 \\ c_1 & d_1
		\end{pmatrix}
	\]
	Secondly, \(\varphi\) is surjective.
	If \(\frac{az+b}{cz+d}\) is a M\"obius map, then
	\[
		\begin{pmatrix}
			a & b \\ c & d
		\end{pmatrix} \in GL_2(\mathbb C)
	\]
	since \(ad-bc\neq 0\).
	But
	\[
		\det \begin{pmatrix}
			a & b \\ c & d
		\end{pmatrix}
	\]
	may not be 1, so we will take \(D^2\) to be this determinant, then we can consider
	\[
		\begin{pmatrix}
			a/D & b/D \\ c/D & d/D
		\end{pmatrix}
	\]
	This new matrix has determinant 1 and is equal to the original M\"obius map, so we have a matrix in \(SL_2(\mathbb C)\) that maps to any given M\"obius map.
	Finally, we want to find the kernel.
	\[
		\varphi \begin{pmatrix}
			a & b \\ c & d
		\end{pmatrix} = \text{id} \in \mathbb M \implies \frac{az+b}{cz+d} = z \iff c = d = 0; a = d
	\]
	But since this matrix has determinant 1, \(a = d = \pm 1\), and thus \(\ker \varphi = \{ I, -I \}\).
\end{proof}
\begin{corollary}
	\[
		\mathcal M \cong \faktor{SL_2(\mathbb C)}{\{ I, -I \}}
	\]
\end{corollary}
\begin{proof}
	This is an immediate consequence of the first isomorphism theorem.
\end{proof}
The quotient \(\faktor{SL_2(\mathbb C)}{\{ I, -I \}}\) is known as the projective special linear group \(PSL_2(\mathbb C)\).

\subsection{Actions of matrices on vector spaces}
All of the groups defined above act on the corresponding vector spaces.
For example, \(GL_n(\mathbb F) \acts \mathbb F^n\).
As an example, let \(G \leq GL_2(\mathbb R) \acts \mathbb R^2\).
What are the orbits of this action?
Clearly, \(\{ \vb 0 \}\) is a singleton orbit since we are acting by linear maps.
\begin{itemize}
	\item If \(G = GL_2(\mathbb R)\), \(G\) acts transitively on \(\mathbb R^2 \setminus \{ 0 \}\).
	      We can complete any \(\vb v \neq 0\) to a basis and therefore we have an invertible change of basis matrix sending any basis to any basis.
	      So there are two orbits: \(\mathbb R^2 \setminus \{ 0 \}\) and \(\{ 0 \}\) itself.
	\item If \(G\) is the set of upper triangular matrices given by
	      \[
		      G = \left\{ \begin{pmatrix}
			      a & b \\ 0 & d
		      \end{pmatrix} \in GL_2(\mathbb R) \right\} = \left\{ \begin{pmatrix}
			      a & b \\ 0 & d
		      \end{pmatrix}: a, d \neq 0 \right\}
	      \]
	      We know that \(\Orb(\vb 0) = \{ \vb 0 \}\).
	      Further:
	      \[
		      \Orb\begin{pmatrix}
			      1 \\ 0
		      \end{pmatrix} = \left\{ \begin{pmatrix}
			      a & b \\ 0 & d
		      \end{pmatrix} \begin{pmatrix}
			      1 \\ 0
		      \end{pmatrix}: \begin{pmatrix}
			      a & b \\ 0 & d
		      \end{pmatrix} \in G \right\} = \left\{ \begin{pmatrix}
			      a \\ 0
		      \end{pmatrix}: a \neq 0 \right\}
	      \]
	      We haven't found all of the orbits yet so let us consider another point.
	      \[
		      \Orb\begin{pmatrix}
			      0 \\ 1
		      \end{pmatrix} = \left\{ \begin{pmatrix}
			      a & b \\ 0 & d
		      \end{pmatrix} \begin{pmatrix}
			      0 \\ 1
		      \end{pmatrix}: \begin{pmatrix}
			      a & b \\ 0 & d
		      \end{pmatrix} \in G \right\} = \left\{ \begin{pmatrix}
			      b \\ d
		      \end{pmatrix}: d \neq 0 \right\}
	      \]
	      We have found all of the orbits since the union gives \(\mathbb R^2\).
\end{itemize}

\subsection{Conjugation action of general linear group}
Recall from Vectors and Matrices: if \(\alpha\colon \mathbb F^n \to \mathbb F^n\) is a linear map, we can represent \(\alpha\) as a matrix \(A\) with respect to a basis \(\{ \vb e_1, \dots, \vb e_n \}\).
If we choose a different basis \(\{ \vb f_1, \dots, \vb f_n \}\) then \(\alpha\) can also be written as a matrix with respect to this new basis, by the matrix \(P^{-1}AP\) where \(P\) is the change of basis matrix, defined by
\[
	\vb f_j = P_{ij}\vb e_i
\]
This is an example of conjugation.
\begin{proposition}
	\(GL_n(\mathbb F)\) acts on \(M_{n \times n}(\mathbb F)\) by conjugation.
	The orbit of a matrix \(A \in M_{n \times n}(\mathbb F)\) is the set of matrices representing the same linear map as \(A\) with respect to different bases.
\end{proposition}
\begin{proof}
	This is an action:
	\begin{itemize}
		\item \(P(A) = PAP^{-1} \in M_{n \times n}(\mathbb F)\) for any chosen matrix \(A \in M_{n \times n}(\mathbb F)\), \(P \in GL_n(\mathbb F)\)
		\item \(I(A) = IAI^{-1} = A\)
		\item \(Q(P(A)) = QPAP^{-1}Q^{-1} = (QP)A(QP)^{-1} = (QP)(A)\)
	\end{itemize}
	As shown in the discussion above, \(A\) and \(B\) are in the same orbit if and only if \(A = PBP^{-1} \iff B = P^{-1}AP\), which is equivalent to this conjugation action.
\end{proof}

Recall from Vectors and Matrices that any matrix in \(M_{2 \times 2}(\mathbb C)\) is conjugate to a matrix in Jordan Normal Form, i.e.\ to one of the following types of matrix:
\[
	\begin{pmatrix}
		\lambda_1 & 0 \\ 0 & \lambda_2
	\end{pmatrix};\quad \begin{pmatrix}
		\lambda & 0 \\ 0 & \lambda
	\end{pmatrix};\quad \begin{pmatrix}
		\lambda & 1 \\ 0 & \lambda
	\end{pmatrix}
\]
In the first case, the values \(\lambda_1, \lambda_2\) are uniquely determined by the matrix we are trying to conjugate (specifically its eigenvalues).
But of course, the order of the eigenvalues is not determined uniquely.
Other than this, no two matrices on this list of possible Jordan Normal Forms are conjugate.
\begin{itemize}
	\item \(\begin{pmatrix}
		      \lambda_1 & 0 \\ 0 & \lambda_2
	      \end{pmatrix}\) is characterised by having two distinct eigenvalues, a property independent of the chosen basis, so it cannot be conjugate to the others.
	\item \(\begin{pmatrix}
		      \lambda & 0 \\ 0 & \lambda
	      \end{pmatrix}\) is only conjugate to itself since it is \(\lambda I\).
	\item \(\begin{pmatrix}
		      \lambda & 1 \\ 0 & \lambda
	      \end{pmatrix}\) is characterised by having a repeated eigenvalue \(\lambda\), but only a one dimensional eigenspace (independent of the basis we choose).
\end{itemize}
This gives a complete description of the orbits of \(GL_n(\mathbb C) \acts M_{n \times n}(\mathbb C)\).

\subsection{Stabilisers of conjugation action}
Clearly we have
\[
	P \in \Stab(A) \iff PAP^{-1} = A \iff PA = AP
\]
So if two matrices commute, they stabilise each other.
Let us consider the three cases as above.
\begin{itemize}
	\item For \(A = \begin{pmatrix}
		      \lambda_1 & 0 \\ 0 & \lambda_2
	      \end{pmatrix}\):
	      \begin{align*}
		      \begin{pmatrix}
			      a & b \\ c & d
		      \end{pmatrix}\begin{pmatrix}
			      \lambda_1 & 0 \\ 0 & \lambda_2
		      \end{pmatrix} & = \begin{pmatrix}
			      \lambda_1 a & \lambda_2 b \\
			      \lambda_1 c & \lambda_2 d
		      \end{pmatrix} \\
		      \begin{pmatrix}
			      \lambda_1 & 0 \\ 0 & \lambda_2
		      \end{pmatrix}\begin{pmatrix}
			      a & b \\ c & d
		      \end{pmatrix} & = \begin{pmatrix}
			      \lambda_1 a & \lambda_1 b \\
			      \lambda_2 c & \lambda_2 d
		      \end{pmatrix}
	      \end{align*}
	      So this matrix is in the stabiliser if and only if \(b = c = 0\).
	      \[
		      \Stab\begin{pmatrix}
			      \lambda_1 & 0 \\ 0 & \lambda_2
		      \end{pmatrix} = \left\{ \begin{pmatrix}
			      a & 0 \\ 0 & d
		      \end{pmatrix} \in GL_2(\mathbb C) \right\}
	      \]
	\item For \(A = \begin{pmatrix}
		      \lambda & 0 \\ 0 & \lambda
	      \end{pmatrix}\), clearly its stabiliser is \(GL_2(\mathbb C)\) since \(A = \lambda I\), and so it commutes with any matrix.
	\item For \(A = \begin{pmatrix}
		      \lambda & 1 \\ 0 & \lambda
	      \end{pmatrix}\), the stabiliser is
	      \[
		      \Stab \begin{pmatrix}
			      \lambda & 1 \\ 0 & \lambda
		      \end{pmatrix} = \left\{ \begin{pmatrix}
			      a & b \\ 0 & a
		      \end{pmatrix} \in GL_2(\mathbb C) \right\}
	      \]
	      (Proof as exercise)
\end{itemize}

\subsection{Geometry of orthogonal groups}
We will look more closely at the orthogonal group and special orthogonal group, and then focus on symmetries of \(\mathbb R^2\) and \(\mathbb R^3\).
Let us consider the standard inner product in \(\mathbb R^n\):
\[
	\vb x \cdot \vb y = x_i y_i = \vb x^\transpose \vb y
\]
If we consider the columns \(\vb p_1, \dots, \vb p_n\) of an orthogonal matrix \(P \in O_n\), we have
\[
	(P^\transpose P)_{ij} = \vb p_i^\transpose \vb p_j = \vb p_i \cdot \vb p_j
\]
So since \(P \in O_n \iff P^\transpose P = I\), we have
\[
	\vb p_i \cdot \vb p_j = \delta_{ij}
\]
\begin{proposition}
	\(P \in O_n\) if and only if the columns of \(P\) form an orthonormal basis.
\end{proposition}
This has been proven by the above discussion.
Thinking of \(P \in O_n\) as a change of basis matrix, we get the following result.
\begin{proposition}
	Consider \(O_n \acts M_{n \times n}(\mathbb R)\) by conjugation.
	Two matrices are in the same orbit if and only if they represent the same linear map with respect to two orthonormal bases.
\end{proposition}
\begin{proposition}
	\(P \in O_n\) if and only if \(P \vb x \cdot P \vb y = \vb x \cdot \vb y\), i.e.\ the matrix preserves the inner product.
\end{proposition}
\begin{proof}
	In the forward direction:
	\[
		(P\vb x) \cdot (P \vb y) = (P \vb x)^\transpose (P \vb y) = \vb x^\transpose P^\transpose P \vb y = \vb x^\transpose \vb y = \vb x \cdot \vb y
	\]
	In the backward direction: if \(P\vb x \cdot P\vb y = \vb x \cdot \vb y\) for all \(\vb x, \vb y \in \mathbb R^n\), then taking the standard basis vectors \(\vb e_i, \vb e_j\) we have
	\[
		P\vb e_i \cdot P\vb e_j = \vb e_i \cdot \vb e_j = \delta_{ij}
	\]
	So the vectors \(P\vb e_1, \dots, P\vb e_n\) are orthonormal.
	These are the columns of \(P\), so \(P \in O_n\).
\end{proof}
\begin{corollary}
	For \(P \in O_n\), \(\vb x, \vb y \in \mathbb R^n\), we have
	\begin{enumerate}
		\item \(\abs{P\vb x} = \abs{\vb x}\) (\(P\) preserves length)
		\item \(P\vb x \angle P\vb y = \vb x \angle \vb y\) (\(P\) preserves angles between vectors)
	\end{enumerate}
\end{corollary}
\begin{proof}
	\begin{enumerate}
		\item Follows from the fact that the inner product is preserved, by taking the inner product of a vector with itself under the transformation.
		\item Angles are also defined using the inner product,
			  \[
				  \cos (\vb x \angle \vb y) = \frac{\vb x \cdot \vb y}{\abs{\vb x}\abs{\vb y}}
			  \]
			  Since the inner product and the lengths are preserved, the cosine of the angle is therefore preserved.
			  Since \(\cos\colon [0, \pi] \to [-1, 1]\) is injective, \(\vb x \angle \vb y = P\vb x \angle P\vb y\).
	\end{enumerate}
\end{proof}

\subsection{Reflections in \(O_n\)}
We will consider what the elements of these groups look like when acting upon \(\mathbb R^n\).
\begin{definition}
	If \(\vb a \in \mathbb R^n\) with \(\abs{\vb a} = 1\), then the reflection in the plane normal to \(\vb a\) is the linear map
	\[
		R_{\vb a} \colon \mathbb R^n \to \mathbb R^n;\quad \vb x \mapsto \vb x - 2 (\vb x \cdot \vb a) \vb a
	\]
\end{definition}
\begin{lemma}
	\(R_{\vb a}\) lies in \(O_n\).
\end{lemma}
\begin{proof}
	Let \(\vb x, \vb y \in \mathbb R^n\).
	\begin{align*}
		R_{\vb a}(\vb x) \cdot R_{\vb a}(\vb y) & = (\vb x - 2 (\vb x \cdot \vb a) \vb a) \cdot (\vb y - 2 (\vb y \cdot \vb a) \vb a)                                                                                                  \\
		                                        & = \vb x \cdot \vb y - 2(\vb x \cdot \vb a)(\vb a \cdot \vb y) - 2(\vb y\cdot \vb a)(\vb x \cdot \vb a) + 4(\vb x\cdot \vb a)(\vb y \cdot \vb a)\underbrace{(\vb a \cdot \vb a)}_{=1} \\
		                                        & = \vb x \cdot \vb y
	\end{align*}
	So it preserves the inner product, so it is an orthogonal matrix.
\end{proof}
As we might expect, conjugates of reflections by orthogonal matrices are also reflections.
\begin{lemma}
	Given \(P \in O_n\), \(PR_{\vb a}P^{-1} = R_{P\vb a}\).
\end{lemma}
\begin{proof}
	We have
	\begin{align*}
		PR_{\vb a}P^{-1}(\vb x) & = P(P^{-1}(\vb x) - 2 (P^{-1}(\vb x) \cdot \vb a) \vb a) \\
		                        & = \vb x - 2(P^{-1}(\vb x)\cdot\vb a)(P\vb a)             \\
		                        & = \vb x - 2(P^\transpose (\vb x)\cdot\vb a)(P\vb a)      \\
		                        & = \vb x - 2(\vb x^\transpose P \vb a)(P\vb a)            \\
		                        & = \vb x - 2(\vb x \cdot P\vb a)(P\vb a)
	\end{align*}
	which by inspection is the reflection of \(\vb x\) by the plane with normal \(P\vb a\).
\end{proof}
We know that no reflection matrix can be in \(SO_n\), since this requires the determinant to be \(+1\), which is the product of the eigenvalues.
The \(n-1\) eigenvectors with eigenvalue \(+1\) are \(n-1\) linearly independent vectors spanning the plane, and the single eigenvector with eigenvalue \(-1\) is the normal to the plane.
So the determinant is \(-1\).

\subsection{Classifying elements of \(O_2\)}
\begin{theorem}
	Every element of \(SO_2\) is of the form
	\[
		\begin{pmatrix}
			\cos\theta & -\sin\theta \\
			\sin\theta & \cos\theta
		\end{pmatrix}
	\]
	for some \(\theta \in [0, 2\pi)\).

	This is an anticlockwise rotation of \(\mathbb R^2\) about the origin by angle \(\theta\).
	Conversely, every such element lies in \(SO_2\).
\end{theorem}
\begin{proof}
	Let
	\[
		A = \begin{pmatrix}
			a & b \\ c & d
		\end{pmatrix} \in SO_2
	\]
	We have \(A^\transpose A = I\) and \(\det A = 1\).
	So
	\[
		A^\transpose = A^{-1} \implies \begin{pmatrix}
			a & c \\ b & d
		\end{pmatrix} = \frac{1}{1} \begin{pmatrix}
			d & -b \\ -c & a
		\end{pmatrix}
	\]
	So \(a=d, b=-c\).
	Since \(ad-bc=1\), \(a^2+c^2=1\).
	Then we can write \(a = \cos \theta\) and \(c = \sin \theta\) for a unique \(\theta \in [0, 2\pi)\).

	Conversely, the determinant of this matrix is 1, and is in \(O_2\), so this element lies in \(SO_2\).
\end{proof}
\begin{theorem}
	The elements of \(O_2 \setminus SO_2\) are the reflections in lines through the origin.
\end{theorem}
\begin{proof}
	Let
	\[
		A = \begin{pmatrix}
			a & b \\ c & d
		\end{pmatrix} \in O_2 \setminus SO_2
	\]
	So \(A^\transpose A = I\) and \(\det A = -1\).
	\[
		A^\transpose = A^{-1} \implies \begin{pmatrix}
			a & c \\ b & d
		\end{pmatrix} = \frac{1}{-1} \begin{pmatrix}
			d & -b \\ -c & a
		\end{pmatrix}
	\]
	So \(a=-d, b=c\).
	Together with \(ad-bc=-1\), we have \(a^2 + c^2 = 1\).
	So let \(a = \cos \theta\), \(c = \sin \theta\) like before, so
	\[
		A = \begin{pmatrix}
			\cos \theta & \sin \theta  \\
			\sin \theta & -\cos \theta
		\end{pmatrix}
	\]
	which can be shown to be a reflection using double angle formulas such that
	\[
		A \begin{pmatrix}
			\sin \frac{\theta}{2} \\ \cos \frac{\theta}{2}
		\end{pmatrix} = -\begin{pmatrix}
			\sin \frac{\theta}{2} \\ \cos \frac{\theta}{2}
		\end{pmatrix};\quad A\begin{pmatrix}
			\cos \frac{\theta}{2} \\ \sin \frac{\theta}{2}
		\end{pmatrix} = \begin{pmatrix}
			\cos \frac{\theta}{2} \\ \sin \frac{\theta}{2}
		\end{pmatrix}
	\]
	So \(A\) is a reflection in the plane orthogonal to the vector \(\begin{pmatrix}
		\sin \frac{\theta}{2} \\ \cos \frac{\theta}{2}
	\end{pmatrix}\).
	Conversely, any reflection in a line through the origin has this form, so it will be in \(O_2 \setminus SO_2\).
\end{proof}

\begin{corollary}
	Every element of \(O_2\) is the composition of at most two reflections.
\end{corollary}
\begin{proof}
	Every element of \(O_2 \setminus SO_2\) is a reflection, so this is trivial.
	If \(A \in SO_2\), then we can write
	\[
		A = \underbrace{A \begin{pmatrix}
				-1 & 0 \\ 0 & 1
			\end{pmatrix}}_{\det = -1} \underbrace{\begin{pmatrix}
				-1 & 0 \\ 0 & 1
			\end{pmatrix}}_{\det = -1}
	\]
	So we have expressed \(A\) as the product of two reflections.
\end{proof}

\subsection{Classifying elements of \(O_3\)}
\begin{theorem}
	If \(A \in SO_3\), then there exists some unit vector \(\vb v \in \mathbb R^3\) with \(A\vb v = \vb v\), i.e.\ there exists an eigenvector with eigenvalue 1.
\end{theorem}
\begin{proof}
	It is sufficient to show that 1 is an eigenvalue of \(A\), since this guarantees that there is some nonzero eigenvector for this eigenvalue which we can then normalise.
	This is equivalent to showing that \(\det (A - I) = 0\).
	\begin{align*}
		\det(A - I) & = \det(A - AA^\transpose)       \\
		            & = \det(A)\det(I - A^\transpose) \\
		            & = \det(I - A^\transpose)        \\
		            & = \det((I - A)^\transpose)      \\
		            & = \det(I - A)                   \\
		            & = (-1)^3\det(A - I)
	\end{align*}
	So \(2\det(A - I) = 0 \implies \det(A - I) = 0\).
\end{proof}
\begin{corollary}
	Every element \(A \in SO_3\) is conjugate (in \(SO_3\)) to a matrix of the form
	\[
		\begin{pmatrix}
			1 & 0           & 0            \\
			0 & \cos \theta & -\sin \theta \\
			0 & \sin \theta & \cos \theta
		\end{pmatrix}
	\]
\end{corollary}
\begin{proof}
	By the above theorem, there exists some unit vector \(\vb v_1\) which is an eigenvector of eigenvalue 1.
	We can extend this vector to an orthonormal basis \(\{ \vb v_1, \vb v_2, \vb v_3 \}\) of \(\mathbb R^3\).
	Then, for \(i=2,3\), we have
	\[
		A\vb v_i \cdot \vb v_1 = A\vb v_i \cdot A\vb v_1 = \vb v_i \cdot \vb v_1 = 0
	\]
	So \(A\vb v_2, A\vb v_3\) lie in the subspace generated by \(\vb v_2, \vb v_3\), i.e.\ \(\vecspan \{ \vb v_2, \vb v_3 \} = \genset{\vb v_2, \vb v_3}\).
	So \(A\) maps this subspace to itself, and we can thus consider the restriction of \(A\) to this subspace.
	The matrix in this new basis will have form
	\[
		\begin{pmatrix}
			1 & 0 & 0 \\
			0 & a & b \\
			0 & c & d
		\end{pmatrix}
	\]
	The smaller matrix in the bottom right will still have determinant 1, since we can expand the determinant here by the first row.
	So \(A\) restricted to this subspace is an element of \(SO_2\), so its matrix must be of the form
	\[
		\begin{pmatrix}
			a & b \\ c & d
		\end{pmatrix} = \begin{pmatrix}
			\cos \theta & -\sin \theta \\
			\sin \theta & \cos \theta
		\end{pmatrix}
	\]
	So \(A\) has the required form with respect to this new basis \(\{ \vb v_1, \vb v_2, \vb v_3 \}\).
	The change of basis matrix \(P\) lies in \(O_3\) since \(\{ \vb v_1, \vb v_2, \vb v_3 \}\) is an orthonormal basis.
	If \(P \notin SO_3\), then we can use the basis \(\{ -\vb v_1, \vb v_2, \vb v_3 \}\) instead, which will invert the determinant of \(P\).
	So in either case \(P \in SO_3\).
\end{proof}
This tells us in particular that every element in \(SO_3\) is a rotation about some axis, here \(\vb v_1\).

\begin{corollary}
	Every element of \(O_3\) is the composition of at most three reflections.
\end{corollary}
\begin{proof}
	\begin{itemize}
		\item If \(A \in SO_3\), then \(\exists P \in SO_3\) such that \(PAP^{-1} = B\), where \(B\) is of the form
		      \[
			      B = \begin{pmatrix}
				      1 & 0           & 0            \\
				      0 & \cos \theta & -\sin \theta \\
				      0 & \sin \theta & \cos \theta
			      \end{pmatrix}
		      \]
		      Since this smaller matrix
		      \[
			      \begin{pmatrix}
				      \cos \theta & -\sin \theta \\
				      \sin \theta & \cos \theta
			      \end{pmatrix}
		      \]
		      is a composition of at most two reflections, then \(B\) is also a composition of at most two reflections, i.e.\ \(B = B_1 B_2\).
		      Since \(A\) is a conjugate of \(B\), it is also a composition of at most two reflections, as the conjugate of a reflection is a reflection, and \(A = P^{-1}BP = (P^{-1}B_1P)(P^{-1}B_2P)\).
		\item If \(A \in O_3 \setminus SO_3\), then \(\det A = -1\) and we can construct
		      \[
			      A = \underbrace{A\begin{pmatrix}
					      -1 & 0 & 0 \\
					      0  & 1 & 0 \\
					      0  & 0 & 1
				      \end{pmatrix}}_{\det = 1}\underbrace{\begin{pmatrix}
					      -1 & 0 & 0 \\
					      0  & 1 & 0 \\
					      0  & 0 & 1
				      \end{pmatrix}}_{\det = -1}
		      \]
		      So the left-hand product lies in \(SO_3\), so it is a composition of at most two reflections.
		      The final element is a reflection in the \(y\)--\(z\) plane, so the combined product is a composition of at most three reflections.
	\end{itemize}
\end{proof}

\begin{example}[symmetries of the cube, revisited]
	We can think of symmetry groups of the Platonic solids as subgroups of \(O_3\) by placing the solid at the origin.
	By question 11 on example sheet 4, we have that \(O_3 \cong SO_3 \times C_2\), where \(C_2\) is generated by the map \(\vb v \mapsto -\vb v\).
	So if \(\vb v\mapsto -\vb v\) is a symmetry of our platonic solid, then this group of symmetries will also split as the direct product of \(G^+ \times C_2\) where \(G^+\) is the group of rotations (proof as exercise).

	So we have that the group of symmetries of the cube is \(G^+ \times C_2 \cong S_4 \times C_2\) by the results from earlier.
\end{example}
