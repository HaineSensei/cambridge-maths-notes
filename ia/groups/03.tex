\subsection{Properties of Generated Groups}
We can make a more precise definition of generated groups as follows:
\begin{itemize}
	\item \(\genset{X}\) contains \(e\)
	\item \(\genset{X}\) contains the set \(X\)
	\item \(\genset{X}\) contains all possible products of \(X\) and their inverses
\end{itemize}

\begin{proposition}
	Let \(X \subseteq G, X \neq \varnothing\).
	Then \(\genset{X}\) is the set of elements of \(G\) of the form \(x_1^{\alpha_1} x_2^{\alpha_2} x_3^{\alpha_3} \cdots x_k^{\alpha_k}\) where \(x_i \in X\) (not necessarily distinct), \(\alpha_i = \pm 1\), and \(k \geq 0\).
	By convention, the empty product \(k=0\) is defined to be \(e\).
\end{proposition}
\begin{proof}
	Let \(T\) be the set of such elements of the given form.
	Clearly, \(T \subseteq \genset{X}\).
	Also, \(T\) is a subgroup of \(G\), and \(X \subseteq T\), so \(\genset{X} \subseteq T\).
	Because both \(T \subseteq \genset{X}\) and \(\genset{X} \subseteq T\), we have \(T = \genset{X}\).
\end{proof}
Note that generating sets are not necessarily unique.
For example, the group of integers under addition generated by \(\genset{1}\) is equivalent to \(\genset{2, 3}\), both of which are equivalent to \(\mathbb Z\), for example.
As a discrete example, \(\mathbb Z_5\) can be generated by any element in the set apart from zero, for example: \(\mathbb Z_5 = \genset{1} = \genset{2} = \genset{3} = \genset{4} \neq \genset{0}\).

\subsection{Homomorphisms}
\begin{definition}
	Let \((G, \ast_G)\), \((H, \ast_H)\) be groups.
	A function \(\varphi: H \to G\) is a homomorphism if
	\[
		\forall a, b \in H,\quad \varphi(a \ast_H b) = \varphi(a) \ast_G \varphi(b)
	\]
\end{definition}
A homomorphism \(\varphi: H \to G\) may have the following descriptions:
\begin{itemize}
	\item injective, if \(\varphi(a) = \varphi(b) \implies a = b\);
	\item surjective, if \(\forall g \in G, \exists h \in H \st \varphi(h) = g\); and
	\item bijective, if it is both injective and surjective.
\end{itemize}
A more intuitive interpretation of the descriptions is:
\begin{itemize}
	\item A function is injective if the outputs are unique;
	\item A function is surjective if all outputs are used;
	\item A function is bijective if there is a one-to-one relation between every element in the input and output sets.
\end{itemize}
Here are some examples, without proofs.
\begin{enumerate}
	\setcounter{enumi}{-1}
	\item Given any two groups \(G\) and \(H\), \(\varphi: H \to G\) defined by \(\varphi(h) = e_G\) is a homomorphism.
	\item The inclusion function \(\iota: H \to G\) where \(H \leq G\) is an injective homomorphism.
	      The inclusion function is defined as the identity function, simply transferring elements from a subgroup into the supergroup.
	\item \(\varphi: \mathbb Z \to \mathbb Z_n\) given that \(\varphi(k) = k \mod n\) is a surjective homomorphism.
	\item \(\varphi: (\mathbb R, +) \to (\mathbb R_{>0}, \cdot)\) where \(R_{>0} = \{r \in \mathbb R : r > 0\}\) and \(\varphi(x) = e^x\) is a bijective homomorphism, otherwise known as an isomorphism.
	\item \(\det : GL_2(\mathbb R) \to (\mathbb R^*, \cdot)\) is a surjective homomorphism.
\end{enumerate}

\subsection{Properties of Homomorphisms}
\begin{proposition}
	Let \(\varphi: H \to G\) be a homomorphism.
	Then, for all \(h \in H\):
	\begin{enumerate}[i.]
		\item \(\varphi(e_H) = e_G\)
		\item \(\varphi(h^{-1}) = \varphi(h)^{-1}\)
		\item Given another homomorphism \(\psi: G \to K\), \(\psi \circ \varphi: H \to K\) is a homomorphism.
	\end{enumerate}
\end{proposition}
\begin{proof}
	We prove each result in order.
	\begin{enumerate}[i.]
		\item Given the identity element of \(H\) is \(e_H\) and similarly for \(G\),
		      \begin{align*}
			      \varphi(e_H \ast e_H) & = \varphi(e_H) \ast \varphi(e_H) \\
			      \implies \varphi(e_H) & = \varphi(e_H) \ast \varphi(e_H) \\
			      e_G                   & = \varphi(e_H)
		      \end{align*}
		\item Consider \(\varphi(h) \ast \varphi(h^{-1}) = \varphi(h \ast h^{-1}) = \varphi(e_H) = e_G\) which is the defining property of the inverse.
		\item For all \(a, b \in H\):
		      \begin{align*}
			      (\psi \circ \varphi) (a \ast b) & = \psi(\varphi(a \ast b))                           \\
			                                      & = \psi(\varphi(a) + \varphi(b))                     \\
			                                      & = \psi(\varphi(a)) + \psi(\varphi(b))               \\
			                                      & = (\psi \circ \varphi)(a) + (\psi \circ \varphi)(b)
		      \end{align*}
	\end{enumerate}
\end{proof}

\subsection{Isomorphisms}
A bijective homomorphism is called an isomorphism.
If there exists an isomorphism \(\varphi: H \to G\), we say that \(H\) is isomorphic to \(G\), or \(H \cong G\).
\begin{enumerate}
	\item Consider a group \(G\) defined as \(\{ e^{\frac{2 \pi i k}{n}} : k \in \mathbb Z_n \}\) under multiplication.
	      Then, \((G, \cdot) \cong (\mathbb Z_n, +)\) where \(\varphi: \mathbb Z_n \to G\) is defined as \(\varphi(k) = e^{\frac{2\pi i k}{n}}\).
	\item \(\varphi: \mathbb Z \to n\mathbb Z\) for \(n \in \mathbb N\) given by \(\varphi(k) = nk\).
	      Note that all non-trivial subgroups of \(\mathbb Z\) are isomorphic to \(\mathbb Z\).
\end{enumerate}

\begin{proposition}
	Let \(\varphi: H \to G\) be an isomorphism.
	Then \(\varphi^{-1}: G \to H\) is an isomorphism.
\end{proposition}
\begin{proof}
	For all \(a, b \in G\),
	\begin{align*}
		\varphi^{-1}(a \ast b) & = \varphi^{-1}\left[ \varphi(\varphi^{-1}(a)) \ast \varphi(\varphi^{-1}(b)) \right] \\
		                       & = \varphi^{-1}\left[ \varphi(\varphi^{-1}(a) \ast \varphi^{-1}(b)) \right]          \\
		                       & = \varphi^{-1}(a) \ast \varphi^{-1}(b)
	\end{align*}
	So \(\varphi^{-1}\) is a homomorphism.
	But since \(\varphi\) is bijective, so is \(\varphi^{-1}\).
	So \(\varphi^{-1}\) is an isomorphism.
\end{proof}
