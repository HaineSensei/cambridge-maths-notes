\subsection{Examples of isomorphism theorems}
\begin{enumerate}
	\item Consider \(\mathbb Z\), \(H = 3 \mathbb Z\), \(N = 5 \mathbb Z\).
	      Then by the Second Isomorphism Theorem, we have
	      \[
		      H \cap N \trianglelefteq H \implies 15\mathbb Z \trianglelefteq 3\mathbb Z
	      \]
	      and, since \(HN = \genset{H, N} = \mathbb Z\) as 3 and 5 are coprime,
	      \[
		      \frac{H}{H \cap N} \cong \frac{HN}{N} \implies \frac{3\mathbb Z}{15\mathbb Z} \cong \frac{\mathbb Z}{5\mathbb Z} \cong \mathbb Z_5
	      \]
	\item (TODO see \(C_4 \times C_2\) example from last time) Let \(C_4 = \genset a\), \(C_2 = \genset b\), \(G = C_4 \times C_2\), \(N = \genset{(a^2, b)}\), \(M = \genset{(e, b), (a^2, e)}\).
	      Then \(N \leq M \leq G\).
	      By the Third Isomorphism Theorem,
	      \[
		      \frac{(C_4 \times C_2) / N}{M / N} = \frac{C_4 \times C_2}{M} = C_2
	      \]
\end{enumerate}

\subsection{Simple groups}
\begin{definition}
	A group \(G\) is simple if its only normal subgroups are trivial \(\{ e \}\) and \(G\) itself.
\end{definition}
\begin{itemize}
	\item \(C_p\) where \(p\) is prime is a simple group.
	\item \(A_5\) is simple.
	      A proof of this will be shown later in the course.
\end{itemize}

\subsection{Group actions}
For many of the examples of groups that we have encountered, we have identified elements of that group by their effect on some set, for example the symmetric group \(S_n\) permuting the set \(\{ 1, \cdots, n \}\), and the M\"obius group being functions \(\hat{\mathbb C} \to \hat{\mathbb C}\), and the dihedral group \(D_{2n}\) being symmetries of an \(n\)-gon.
While we can study groups purely algebraically, it can be very useful to see how a group acts on other objects.

\begin{definition}
	Let \(G\) be a group, \(X\) be a set.
	An action of \(G\) on \(X\) is a function \(\alpha: G \times X \to X\), written
	\[
		\alpha(g, x) = \alpha_g(x)
	\]
	satisfying:
	\begin{itemize}
		\item \(\alpha_g(x) \in X\) (implied by the function's type)
		\item \(\alpha_e(x) = x;\; \forall x \in X\)
		\item \(\alpha_g \circ \alpha_h(x) = \alpha_{gh}(x);\; \forall g, h \in G, \forall x \in X\)
	\end{itemize}
	We can write \(G \acts X\).
\end{definition}
Here are some examples.
\begin{enumerate}
	\setcounter{enumi}{-1}
	\item Take any \(G\), \(X\) and define the trivial action \(\alpha_g(x) = x\).
	\item \(S_n \acts \{ 1, 2, \cdots, n \}\) by permutation.
	\item \(D_{2n} \acts \{ \text{vertices of a regular \(n\)-gon} \}\), and labelling the vertices as 1 to \(n\), we have \(D_{2n} \acts \{ 1, 2, \cdots, n \}\).
	\item \(\mathcal M \acts \hat{\mathbb C}\) via M\"obius maps.
	\item Symmetries of a cube act on the set of vertices, the set of edges, and even (for example) the set of pairs of opposite faces of the cube.
\end{enumerate}
Examples 1, 2 show that more than one group can act on a given set.
Example 4 shows that one group can act on many sets.
Group actions help us deduce information about the group.

\begin{lemma}
	\(\forall g \in G, \alpha_g: X \to X, x \mapsto \alpha_g(x)\) is a bijection.
\end{lemma}
\begin{proof}
	We have that \(\alpha_g(\alpha_{g^{-1}}(x)) = \alpha_{g g^{-1}}(x) = \alpha_e(x) = x\).
	Similarly, \(\alpha_{g^{-1}}(\alpha_g(x)) = \alpha_{g^{-1} g}(x) = \alpha_e(x) = x\).
	So the composition \(\alpha_g \circ \alpha_{g^{-1}}\) is the identity on \(X\), and \(\alpha_{g^{-1}} \circ \alpha_g\) is also the identity on \(X\), so \(\alpha_g\) is a bijection.
\end{proof}
We can also define actions by linking \(G\) to \(\Sym(X)\).
\begin{proposition}
	Let \(G\) be a group, \(X\) a set.
	Then \(\alpha\colon G \times X \to X\) is an action if and only if the function \(\rho\colon G \to \Sym(X)\) where \(\rho(g) = \alpha_g\) is a homomorphism.
\end{proposition}
\begin{proof}
	\(\alpha\) is an action.
	By the above lemma, \(\alpha_g\) is a bijection from \(X \to X\).
	So \(\alpha_g \in \Sym(X)\).
	Now, we want to show that \(\rho\) is a homomorphism.
	\(\rho(gh) = \alpha_{gh}\), and for all \(x \in X\), \(\alpha_{gh}(x) = \alpha_g \circ \alpha_h (x)\), so \(\rho(gh) = \alpha_{gh} = \rho(g)\circ\rho(h)\).
	So \(\rho\) is a homomorphism.

	In the other direction, given that \(\rho\) is a homomorphism \(G \to \Sym(X)\), we can define an action \(\alpha\colon G \times X \to X\) by \(\alpha(g, x) = \alpha_g(x) := \rho(g)(x)\).
	\(\alpha\) is an action because \(\alpha_g \circ \alpha_h = \rho(g)\rho(h) = \rho(gh) = \alpha_{gh}\), and the identity element \(\rho(e)\) is the identity element in \(\Sym(X)\), so \(\alpha_e(x) = \rho(e)(x) = x\) as required.
\end{proof}
Sometimes we write \(g(x)\) instead of the more verbose \(\alpha_g(x)\).

\begin{definition}
	The kernel of an action \(\alpha\colon G \times X \to X\) is the kernel of the homomorphism \(\rho\colon G \to \Sym(X)\).
	These are all the elements of \(G\) that preserve every element of \(X\).
\end{definition}
Note that \(\frac{G}{\ker \rho} \cong \Im \rho \leq \Sym(X)\).
So in particular, if the kernel is trivial, then \(G \leq \Sym(X)\).
\begin{enumerate}
	\item \(D_{2n}\) acting on the vertices \(\{ 1, \cdots, n \}\) of an \(n\)-gon has \(\ker\rho = \{ e \}\).
	      Every non-trivial element of \(D_{2n}\) moves at least one vertex.
	      So \(D_{2n} \leq S_n\) by the First Isomorphism Theorem.
	\item Let \(G\) be symmetries of a cube, and consider \(X = \{ \text{unordered pairs of opposite faces} \}\).
	      Then \(\abs{X} = 3\) as there are three unordered pairs of opposite faces.
	      So \(\rho\colon G \to S_3\).
	      Clearly there are symmetries of the cube that realise all the permutations of \(X\), so \(\rho\) is surjective.
	      So \(\frac{G}{\ker \rho} \cong S_3\).
	      Note that there are clearly non-trivial symmetries (e.g.
	      reflection) that preserve \(X\), so the kernel is non-trivial.
\end{enumerate}
\begin{definition}
	An action \(G \acts X\) is called faithful if \(\ker \rho = \{ e \}\).
\end{definition}
Then \(G\) is isomorphic to a subgroup of \(\Sym X\) by the First Isomorphism Theorem.
