\documentclass{article}

\usepackage[UKenglish]{babel}
\usepackage[utf8]{inputenc}
\usepackage[a4paper, margin=20mm]{geometry}
\usepackage{textcomp} % makes the "not defining \perthousand"/"\micro" errors go away by including this first
\usepackage{amsmath}
\usepackage{amssymb}
\usepackage{amsthm}
\usepackage{amsfonts}
\usepackage{wrapfig}
\usepackage{physics}
\usepackage{bm}
\DeclareDocumentCommand\mathbf{m}{\bm{\mathrm{#1}}} % make bold work for greek symbols
\DeclareDocumentCommand\vnabla{}{\nabla} % use non-bold nabla for \grad, \curl etc. Enabled to unify laplacian symbol between vector and scalar forms
\DeclareDocumentCommand\dotproduct{}{\cdot} % use non-bold dot for scalar product to unify notation
\DeclareDocumentCommand\crossproduct{}{\times} % use non-bold dot for scalar product to unify notation
\usepackage{gensymb}
\usepackage{enumerate}
\usepackage{mathtools}
\usepackage{centernot}
\usepackage{relsize}
\usepackage{mathrsfs}
\usepackage{siunitx}
\usepackage{pgfplots}
\pgfplotsset{width=10cm,compat=1.9}
\usepgfplotslibrary{external}
\tikzexternalize[prefix=tikz/]
\usepackage[pdfa]{hyperref}
\hypersetup{
	colorlinks=true,
	linktoc=all,
	linkcolor=black,
}

\numberwithin{equation}{section} % make equations be numbered 1.1 not 1

\newcommand{\tableofcontentsnewpage}{\tableofcontents\newpage}

% create the theorem environments
\theoremstyle{definition}
\newtheorem*{definition}{Definition}

\newtheorem*{claim}{Claim}
\newtheorem*{theorem}{Theorem}
\newtheorem*{proposition}{Proposition}
\newtheorem*{lemma}{Lemma}
\newtheorem*{corollary}{Corollary}

\theoremstyle{remark}
\newtheorem*{note}{Note}
\newtheorem*{remark}{Remark}

\newcommand{\ddempty}{\mathrm{d}}
\newcommand{\dn}[2]{\mathrm{d}^#1#2}
\newcommand{\st}{\text{ s.t. }}
\newcommand{\contradiction}{\(\#\)}
\newcommand{\genset}[1]{\langle{} #1 \rangle}
\newcommand{\nhat}{\vu{n}}
\newcommand{\rdot}{\dot{\vb{r}}}
\newcommand{\rddot}{\ddot{\vb{r}}}
\newcommand{\transpose}{\intercal}
\newcommand{\acts}{\curvearrowright}
\newcommand{\adjugate}[1]{\widetilde{#1}}
\newcommand{\mathhuge}[1]{\mathlarger{\mathlarger{\mathlarger{#1}}}}
\newcommand{\stcomp}[1]{{#1}^c} % consider \complement? Personally I think this looks better, and it's what Wikipedia uses
\newcommand{\prob}[1]{\mathbb{P}\left({#1}\right)}
\newcommand{\psub}[2]{\mathbb{P}_{#1}\left({#2}\right)}
\newcommand{\psubx}[1]{\psub{x}{#1}}
\newcommand{\expect}[1]{\mathbb{E}\left[{#1}\right]}
\newcommand{\esub}[2]{\mathbb{E}_{#1}\left[{#2}\right]}
\newcommand{\esubx}[1]{\esub{x}{#1}}
\newcommand{\Var}[1]{\Varop\left({#1}\right)}
\newcommand{\Cov}[1]{\Covop\left({#1}\right)}
\newcommand{\Corr}[1]{\Corrop\left({#1}\right)}
\newcommand{\convdist}{\xrightarrow{d}}
\newcommand{\convprob}{\xrightarrow{\mathbb{P}}}

\DeclareMathOperator{\vecspan}{span}
\DeclareMathOperator{\HCF}{HCF}
\DeclareMathOperator{\LCM}{LCM}
\DeclareMathOperator{\ord}{ord}
\DeclareMathOperator{\Sym}{Sym}
\DeclareMathOperator{\nullity}{null}
\DeclareMathOperator{\Orb}{Orb}
\DeclareMathOperator{\Stab}{Stab}
\DeclareMathOperator{\ccl}{ccl}
\DeclareMathOperator{\Varop}{Var}
\DeclareMathOperator{\Covop}{Cov}
\DeclareMathOperator{\Corrop}{Corr}

\DeclarePairedDelimiter\ceil{\lceil}{\rceil}
\DeclarePairedDelimiter\floor{\lfloor}{\rfloor}

% for arrows in the middle of the line
\usetikzlibrary{decorations.markings}
\tikzset{->-/.style={decoration={
		markings,
		mark=at position #1 with {\arrow{>}}},postaction={decorate}}}


\title{Dynamics and Relativity}
\author{Cambridge University Mathematical Tripos: Part IA}

\begin{document}
\maketitle

\tableofcontents
\newpage

\section{Basic Definitions and Newton's Laws}
\subsection{Basic Concepts}
\begin{definition}
    A particle is an object which has negligible size. It therefore does not have an alignment or rotation. It has a finite mass $m > 0$, and perhaps an electric charge $q$ (which may be positive or negative). The position of the particle is described by a position vector $\vb r(t)$ or $\vb x(t)$, with respect to an origin $O$.
\end{definition}
\begin{definition}
    The Cartesian components of this vector $\vb r(t)$ are given by $(x, y, z)$, where $\vb r = x \hat{\vb \imath} + y \hat{\vb \jmath} + z \hat{\vb k}$, with $\hat{\vb \imath}, \hat{\vb \jmath}, \hat{\vb k}$ orthonormal. The choice of coordinate axes defines a frame of reference $S$.
\end{definition}
\begin{definition}
    The velocity of a particle is $\vb u(t) = \dot{\vb r} = \frac{\dd}{\dd{t}}\vb r(t)$. The velocity is tangential to the path, or \textit{trajectory}, of the particle.
\end{definition}
\begin{definition}
    The momentum of a particle is $\vb p = m \vb u$.
\end{definition}
\begin{definition}
    The acceleration of a particle is $\vb a = \dot{\vb u} = \ddot{\vb r}$.
\end{definition}
\begin{note}
    The time derivative of $\vb u(t)$, for example, is defined using the limit definition:
    \[ \dot{\vb u}(t) = \lim_{h \to 0} \frac{\vb u(t + h) - \vb u(t)}{h} \]
    with $\vb u \to \vb u_0$ if and only if $\abs{\vb u - \vb u_0} \to 0$. With Cartesian basis vectors, we can evaluate derivatives componentwise, bringing the differential operator inside each vector component.
\end{note}
The derivatives of scalar and vector functions interoperate as expected. Suppose we have a scalar function $f(t)$ and vector functions $\vb g(t), \vb h(t)$, then for example we have
\[ \frac{\dd}{\dd{t}}(f \vb g) = \frac{\dd{f}}{\dd{t}} \vb g + f \frac{\dd \vb g}{\dd{t}} \]
\[ \frac{\dd}{\dd{t}}(\vb g \cdot \vb h) = \frac{\dd \vb g}{\dd{t}}\cdot \vb h + \vb g \cdot \frac{\dd \vb h}{\dd{t}} \]
\[ \frac{\dd}{\dd{t}}(\vb g \times \vb h) = \frac{\dd \vb g}{\dd{t}}\times \vb h + \vb g \times \frac{\dd \vb h}{\dd{t}} \]
Take note of the ordering of the terms involving $\vb g$ and $\vb h$ when using the vector product.

\subsection{Newton's Laws of Motion}
\begin{enumerate}
    \item (Galileo's Law of Inertia) There exist inertial frames of reference in which a particle remains at rest or moves in a straight line at constant speed (i.e. at constant velocity), unless it is acted on by a force.
    \item In an inertial frame of reference, the rate of change of momentum of a particle is equal to the force acting on it.
    \item To every action, there is an equal and opposite reaction. The forces exerted between two particles are equal in magnitude and opposite in direction.
\end{enumerate}
Note that the second law is a statement about vectors. All of these statements that we have made about particles can also be extended to finite bodies, composed of many particles.

\section{Galilean Invariance}
\subsection{Boosts}
In an inertial frame, the acceleration of a particle is zero if the force acting on the particle is zero.
\[ \ddot{\vb r} = \vb 0 \iff \vb F = \vb 0 \]
There is no unique inertial frame of reference. If $S$ is an inertial frame, then any other frame $S'$ moving at constant velocity relative to $S$ is also an inertial frame. For example, suppose that $S'$ is moving at speed $v$ in the $x$ direction. Then here
\[ x'=x-vt;\quad y'=y;\quad z'=z;\quad t'=t \]
and
We can generalise this to $S'$ moving in an arbitrary direction relative to $S$, i.e.
\[ \vb r' = \vb r - \vb v t \]
where $\vb v$ is the velocity of $S'$ relative to $S$. This type of transformation is known as a `boost'. For a particle with position vector $\vb r(t)$ in $S$ (and position vector $\vb r'(t)$ in $S'$), we can compute the velocity $\vb u = \dot{\vb r}$ and acceleration $\vb a = \ddot{\vb r}$ as follows:
\[ \vb u' = \vb u - \vb v;\quad \vb a' = \vb a \]
This can be seen by taking the derivative of the `boost' formula.

\subsection{Galilean Transformations}
A general Galilean Transformation is any transformation that preserves inertial frames. They are combinations of:
\begin{itemize}
    \item boosts $\vb r' = \vb r - \vb vt$ where $\vb v$ is constant,
    \item translations of space (moving the origin) $\vb r' = \vb r - \vb r_0$ where $r_0$ is constant,
    \item translations of time $t' = t - t_0$ where $t_0$ is constant,
    \item rotations and reflections in space $\vb r' = R \vb r$ where $R$ is a constant orthogonal matrix.
\end{itemize}
This set generates the Galilean group. For any Galilean transformation we have
\[ \ddot{\vb r} = \vb 0 \iff \ddot{(\vb r')} = \vb 0 \]

The principle of Galilean relativity is that the laws of Newtonian physics are the same in all inertial frames. In other words, the laws of physics are always the same:
\begin{itemize}
    \item at any point in space
    \item at any point in time
    \item in any direction
    \item at any constant velocity
\end{itemize}
Any set of equations which describe Newtonian physics must preserve this Galilean invariant. This shows that measurement of velocity cannot be absolute, it must be relative to a specific inertial frame of reference --- but conversely, measurement of acceleration \textit{is} absolute.

\subsection{Newton's Second Law}
For any particle subject to a force $\vb F$, the momentum $\vb p$ of the particle satisfies
\[ \frac{\dd \vb p}{\dd{t}} = \vb F \]
where $\vb p = m \vb u$. For this part of the course, let us assume that $m$ is constant. Then $\vb F = \dot{\vb p} = m \vb a$. We can interpret this value $m$ as a measure of `reluctance to accelerate', i.e. its inertia. If $\vb F$ is specified as a function of $\vb r, \dot{\vb r}, t$, then we have a second order differential equation for $\vb r$. In order to solve this equation, we must provide two of initial conditions $\vb r_0$ and $\dot{\vb r}_0$ at some initial time $t_0$. The trajectory of the particle is then determined for all future and past times.

\subsection{Gravitational Force}
Consider two particles, one at $\vb r_1$ and one at $\vb r_2$. Newton's law of gravitation states that the gravitational force on $\vb r_1$ is given by
\[ \vb F_1 = \frac{- G m_1 m_2 (\vb r_1 - \vb r_2)}{\abs{\vb r_1 - \vb r_2}^3} \]
where $G$ is the gravitational constant, and $\vb F_2$ is given by $-\vb F_1$. Note that:
\begin{itemize}
    \item This is known as an inverse square law, since the magnitude of the output is proportional to the inverse of the square of the distance between the particles.
    \item This is an attractive force, since it is in the direction $\vb r_2 - \vb r_1$.
    \item This obeys Newton's Third Law, since $\vb F_2 = - \vb F_1$.
    \item By inspection, $G$ must have dimension $L^3 \cdot M^{-1} \cdot T^{-2}$, i.e. length cubed over mass over time squared.
\end{itemize}

\subsection{Electromagnetic Force}
Consider a particle with electric charce $q$, in the presence of an electric field $\vb E(\vb r, t)$ and a magnetic field $\vb B(\vb r, t)$. The Lorentz force law states that
\[ \vb F(\vb r, \dot{\vb r}, t) = q\left( \vb E + \dot{\vb r} \times \vb B \right) \]
As an example, let $\vb E = \vb 0$ everywhere, and let $\vb B$ be a constant vector. Then
\[ m \ddot{\vb r} = q \dot{\vb r} \times \vb B \]
We can solve this differential equation for $\vb r$. Let us choose axes such that $\vb B = B \hat{\vb z}$, i.e. $\vb B$ is in the $z$ direction. Evaluating the cross product, $m \ddot{z} = 0$, so $z = z_0 + ut$ where $z_0$ and $u$ are constants. Further,
\[ m \ddot x = qB\dot y;\quad m \ddot y = -qB\dot x \]
For convenience, let us define $\omega = qB/m$, and then
\[ x = x_0 - \alpha \cos(\omega(t - t_0));\quad x = y_0 + \alpha \sin(\omega(t - t_0)) \]
This describes circles in the $x$--$y$ plane, and constant velocity motion in the $z$ direction. This results in a helix in the direction of the magnetic field, clockwise when viewed from the direction of $\vb B$.

\section{Dimensional Analysis}
\subsection{Choice of Units}
Many problems in dynamics involve three basic dimensional quantities: length, mass and time. These are commonly referred to using the symbols $L$, $M$ and $T$, to be generic over the choice of unit system. The dimensions of some quantity $x$ can therefore be expressed in terms of powers of $L$, $M$, $T$. So the dimension of density is $M \cdot L^{-3}$. The dimension of force is $M \cdot L \cdot T^{-2}$.

Only `power law' functions of these quantities are allowed; we are not allowed to exponentiate a dimensional quantity, for example. This is because $e^L = 1 + L + \frac{1}{2}L^2 + \dots$ would be comparing a dimensionless constant 1 with some length, and some area, and so forth. This comparison does not make any sense.

We can choose a unit system that is convenient, for example SI units. It defines the metre for $L$, the kilogram for $M$ and the second for $T$. So many other physical quantities can be formed from these. For example, the SI unit for the gravitational constant is \si{\metre\cubed\per\kilogram\per\second\squared}. In this unit system, we can say $G = \SI{6.67e-11}{\metre\cubed\per\kilogram\per\second\squared}$.

As a general principle, dynamical and physical equations must work for any consistent choice of units. If, however, we used SI units for length, mass and time, but the imperial unit pound-force as the unit for force, the equations would be inconsistent.

\subsection{Scaling and Dimensional Independence}
Suppose that a dimensional quantity $Y$ depends on a set of dimensional quantities $X_1, X_2, \dots, X_n$, so the dimension of $Y$ is $L^a M^b T^c$ and the dimension of the $X_i$ are $L^{a_i} M^{b_i} T^{c_i}$.

If $n \leq 3$, then $Y = C \cdot X_1^{p_1}X_2^{p_2}X_3^{p_3}$, and $p_1, p_2, p_3$ can be found by balancing the dimensions. Hence $a = a_1p_1 + a_2p_2 + a_3p_3$ and so forth for $b$ and $c$. This yields a unique solution for $p_1, p_2, p_3$ if these three equations are linearly independent, i.e. if the dimensions of $X_1, X_2, X_3$ are independent.

If $n > 3$, then this property of dimensional independence does not hold; it is always possible to express one of the four (or more) dimensions in terms of the other three. So let us choose $X_1, X_2, X_3$ to be dimensionally independent, and then we can incorporate $X_4, X_5$ and so on as dimensionless quantities:
\[ \lambda_1 = \frac{X_4}{X_1^{q_{11}}X_2^{q_{12}}X_3^{q_{13}}};\quad \lambda_2 = \frac{X_5}{X_1^{q_{21}}X_2^{q_{22}}X_3^{q_{23}}} \cdots \]
where the powers $q_{ij}$ have been chosen such that the $\lambda$ are dimensionless. Then
\[ Y = X_1^{p_1}X_2^{p_2}X_3^{p_3} \cdot C(\lambda_1, \lambda_2, \dots, \lambda_{n-3}) \]
This is known as Bridgman's Theorem.

\subsection{Simple Pendulum}
As an example, let us consider a simple pendulum with a string of length $\ell$, released from rest, when the horizontal distance from the end of the pendulum to the rest position is $d$. How does the period $P$ of the pendulum depend on the four dimensional quantities $m, \ell, d, g$?

We know that the dimension of the period is $T$, time. The dimension of $m$ is $M$, the dimension of $g$ is $L \cdot T^{-2}$, and the dimensions of $\ell$ and $d$ are both $L$. We will form one dimensionless group, since $n=4$ in this case. A simple way of doing so is letting $\lambda = d/\ell$. So $P = m^{p_1} \ell^{p_2} g^{p_3} \cdot f(d/\ell)$. Comparing units, we have $T = M^{p_1} L^{p_2} (L \cdot T^{-2})^{p_3}$. Solving, we get $p_1 = 0, p_2 = \frac{1}{2}, p_3 = \frac{-1}{2}$. Applying Bridgman's Theorem, we have $P = \sqrt{\ell / g} \cdot f(d/\ell)$. This does not completely specify the formula, but it does provide useful insights. For example, doubling both $d$ and $\ell$, $P \mapsto \sqrt{2} P$, since $d/\ell$ does not change.

\subsection{Taylor's Estimate of Energy of First Atomic Explosion}
Taylor used publicly available data on the fireball's growth over time in order to estimate the energy released in the first atomic explosion. Let $R(t)$ be the radius of the fireball as a function of time, which has dimension $L$. The time $t$ has dimension $T$. The density of air $\rho$ has dimension $M \cdot L^{-3}$. The energy of the explosion is $E$ which has dimension $M \cdot L^2 \cdot T^{-2}$. Then, $R = C \cdot t^\alpha \rho^\beta E^\gamma$. By balancing dimensions, we have $\alpha = \frac{2}{5}, \beta = \frac{-1}{5}, \gamma = \frac{1}{5}$. Then, $R(t) = C \cdot t^{\frac{2}{5}} \rho^{\frac{-1}{5}} \gamma^{\frac{1}{5}}$.

Taylor then verified this $\frac{2}{5}$ power law, and estimated the value of $E$ as $\frac{\rho R^5}{C^5 t^2}$. It was observed that $\frac{R^5}{t^2} \sim \SI{6.7e13}{\metre\tothe{5}\per\second}$, and $\rho\sim\SI{1.25}{\kilogram\per\metre\cubed}$. Then if $C \sim 1$ then $E \sim \SI{1e14}{\joule}$, which is approximately \SI{2.4e4}{\tonne} of TNT.

\section{Forces and Potential Energy in One Spatial Dimension}
\subsection{Forces}
Consider a particle of mass $m$ at position $x(t)$ in one spatial dimension. Let us consider the action of a force $F(x)$ on the particle, i.e. a force dependent entirely on the position and not the velocity or time. We define the potential energy $V(x)$ by
\[ F(x) = -\frac{\dd{V}}{\dd{x}} \]
Hence,
\[ V(x) = - \int^x F(u) \dd{u} \]
The lower limit is unspecified to give an arbitrary constant in $V(x)$. By Newton's Second Law,
\[ m\ddot{x} = -\frac{\dd{V}}{\dd{x}} \]
We define the kinetic energy $T = \frac{1}{2}m\dot x^2$. The total energy in the system $E$ is defined as $T + V = \frac{1}{2} m \dot x^2 + V(x)$. We will show that total energy is conserved: $\frac{\dd{E}}{\dd{t}} = 0$.
\begin{proof}
    \begin{align*}
        \frac{\dd{E}}{\dd{t}} & = \frac{\dd}{\dd{t}}\left( \frac{1}{2}m\dot x^2 + V(x) \right) \\
                              & = m\dot x \ddot x + \frac{\dd{V}}{\dd{x}} \dot x               \\
                              & = \dot x\left( m \ddot x + \frac{\dd{V}}{\dd{x}} \right)       \\
                              & = \dot x ( 0 )                                                 \\
                              & = 0
    \end{align*}
\end{proof}
\noindent In general, in order to conserve a total energy $\frac{1}{2}m\dot x^2 + \Phi$, we require that
\[ \dot x F = -\frac{\dd \Phi}{\dd{t}} \]
It is usually the case that there exists no such $\Phi$ if $F$ depends on $\dot x$ or $t$.

\subsection{Force in the Harmonic Oscillator}
Let us consider the example of the harmonic oscillator, i.e.
\[ F(x) = -kx \]
Then we can construct
\[ V(x) = -\int^x -ku \dd{u} = \int^x ku \dd{u} = \frac{1}{2} kx^2 \]
where we have chosen the arbitrary constant conveniently. Note that we can explicitly solve the second order ordinary differential equation to compute $x$ as a function of $t$:
\[ x(t) = A\cos \omega t + B\sin \omega t;\quad \dot x(t) = -\omega A \sin \omega t + \omega B \cos \omega t \]
where $\omega = \sqrt{\frac{k}{m}}$. We can check that energy $E$ is conserved:
\begin{align*}
    E & = \frac{1}{2}m\dot x^2 + \frac{1}{2}kx^2                                                                                                         \\
      & = \frac{1}{2}m \left( -\omega A \cos \omega t + \omega B \sin \omega t \right)^2 + \frac{1}{2}k \left( A\sin \omega t + B\cos \omega t \right)^2 \\
      & = \frac{1}{2}k(A^2 + B^2)
\end{align*}

\subsection{More General Potentials}
Note that conservation of energy is a first integral of Newton's Second Law. In one dimension, conservation of energy gives useful information about a particle's motion that can help in deriving $x$ as a function of $t$. In the previous example, we verified that conservation of energy holds having already solved the differential equation, but it can often be more useful to consider energy while solving the equation.
\[ E = \frac{1}{2}m\dot x^2 + V(x) \]
Hence,
\[ \dot x = \pm \sqrt{\frac{2}{m}(E - V(x))} \]
Therefore,
\[ \int_{x_0}^x \frac{\dd{u}}{\sqrt{\frac{2}{m}(E - V(u))}} = t - t_0 \]
where $x(t_0) = x_0$. This gives $t$ as a function of $x$; we can invert this function to give $x$ as a function of $x$. Realistically, this integral is mostly useful to get structural insight rather than actually solving $x$ as a function of time, since it is difficult to do this analytically. As an example, let
\[ V(x) = \lambda(x^3 - 3 \beta^2 x) \]
where $\lambda, \beta$ are positive constants. What happens if we release the particle from rest at $x=x_0$? We can draw the graph of $V(x)$ and imagine the height of the graph as the height of a `rail' that the particle sits on, acted on under gravity, i.e. the particle `falls' from higher $V(x)$ to lower $V(x)$, gaining kinetic energy as it falls. Since we start at rest, $E = V(x_0)$ at $t=0$, and in the subsequent motion $E \leq V(x_0)$. We have a few cases:
\begin{enumerate}[{Case} 1:]
    \item ($x_0 < -\beta$) $x_0 = -\beta$ is a maximum point on the graph. The particle will move to the left with $x(t) \to -\infty$ as $t \to \infty$.
    \item ($-\beta < x_0 < 2\beta$) Note that $V(-\beta) = V(2\beta)$; they are the same height on the graph. Since there is no friction in this model, the particle's motion is confined to the region $-\beta < x < 2\beta$ and will oscillate forever.
    \item ($2\beta < x_0$) The particle will move to the left, reaching $x=-\beta$, and then will continue to the left, since it has kinetic energy at this point. So $x \to -\infty$ as $t \to \infty$.
\end{enumerate}
We also have special cases on the turning points $\pm\beta$, where the particle does not move. There is another case at $x_0 = 2\beta$: the particle will move to the left, accelerating until $x=\beta$, then decelerating until $x=-\beta$, where it will then stop moving at this maximum point. How long does it take for the particle to move from $x_0=2\beta$ to $x=-\beta$, where it rests? We can use the integral above to compute this, letting $t_0 = 0$ and $x(0) = 2\beta$.
\begin{align*}
    \int_{x(t)}^{2\beta} \frac{\dd \widetilde x}{\sqrt{\frac{2\lambda}{m}(2\beta^3 - \widetilde x^3 + 3 \beta^2 \widetilde x)}} & = t \\
    \int_{x(t)}^{2\beta} \frac{\dd \widetilde x}{\sqrt{\frac{2\lambda}{m}(\widetilde x + \beta)^2(2\beta - \widetilde x)}}      & = t \\
    \int_{x(t)}^{2\beta} \frac{\dd \widetilde x}{(\widetilde x + \beta)\sqrt{\frac{2\lambda}{m}(2\beta - \widetilde x)}}        & = t \\
\end{align*}
This integral diverges as $\widetilde x \to -\beta$, so it takes an infinite amount of time to come to rest at this maximum point; specifically it exhibits logarithmic behaviour.

\section{Equilibrium Points and Higher Dimensional Forces}
\subsection{Equilibrium Points}
An equilibrium point is defined as a point where the potential is stationary, in other words where the force on the particle is zero. So the particle stays at rest for all time. In the example in the previous lecture, $x = \pm \beta$ were the equilibrium points. We can analyse the motion close to the equilibrium point in order to work out whether the equilibrium point is stable or unstable. Let $x_0$ be an equilibrium point, so $V'(x_0) = 0$. We can expand $V(x)$ as a series, assuming that $x-x_0$ is small.
\[ V(x) = V(x_0) + \frac{1}{2}(x-x_0)^2V''(x_0) + o((x-x_0)^2) \]
In the neighbourhood of $x_0$,
\[ m\ddot x = -V'(x) \approx -(x-x_0)V''(x_0) \]
\begin{itemize}
    \item If $V''(x_0) > 0$, we have a local minimum of potential, which gives rise to a stable equilibrium point. The equation of motion of a particle near $x_0$ is a harmonic oscillator. The angular frequency of oscillation is $\omega = \sqrt{\frac{V''(x_0)}{m}}$.
    \item If $V''(x_0) < 0$, we have a local maximum of potential, which gives rise to an unstable equilibrium point. Any perturbation from this point will cause an increased deviation from the point. The equation of motion near this point is exponential; almost always exponentially increasing rather than decreasing. The growth rate is $\gamma = \sqrt{\frac{-V''(x_0)}{m}}$.
    \item If $V''(x) = 0$, we must use higher-order terms from the Taylor series in order to determine the behaviour.
\end{itemize}
Let us consider the example of a simple pendulum with a mass $m$ held by a rigid beam of length $\ell$. Let the angle between the beam and the vertical direction be $\theta$. By Newton's second law,
\[ F(x = \ell \theta) = m \ell \ddot \theta = -mg \sin \theta \]
We can derive an energy equation by using $F(x) = -V'(x)$.
\[ V(x = \ell \theta) = -\int_0^{\ell\theta} F(u) \dd{u} = -mg \ell \cos \theta \]
The kinetic energy $T$ is given by
\[ T = \frac{1}{2}m\ell^2\dot\theta^2 \]
We can check that $\frac{\dd{E}}{\dd{t}} = 0$ at all $t$. The stationary points of $V$ are at $\theta = 0$ and $\theta = \pi$ (assuming $0 \leq \theta < 2\pi$). The $\theta=0$ point is stable, since $V''(\theta = 0) > 0$. The $\theta=\pi$ point is unstable. If $-mg\ell < E < mg\ell$, the pendulum will oscillate between two values since it cannot continue spinning in circles. In particular, this oscillation occurs about a position of stable equilibrium. However, if we add additional energy into this system, either $\dot\theta > 0$ or $\dot\theta < 0$ for all time. It is impossible to have $E < -mg\ell$ since this is the minimum value of the potential.

Now, let us consider the period $P$ of the oscillation of $\theta$ after releasing the particle from rest at some initial angle $\theta_0$. Note that the oscillation consists of $\theta_0 \to 0 \to -\theta_0 \to 0 \to \theta_0$. By symmetry, this period is four times the time it takes to go from $\theta_0$ to 0. From the energy equation, we can deduce
\begin{align*}
    P & = 4 \int_0^{\theta_0} \frac{\dd \theta}{\sqrt{\frac{2g\ell}{\ell^2}(\cos \theta - \cos \theta_0)}}  \\
      & = 4\sqrt{\frac{\ell}{g}} \int_0^{\theta_0} \frac{\dd \theta}{\sqrt{2\cos \theta - 2 \cos \theta_0}} \\
      & = 4\sqrt{\frac{\ell}{g}} F(\theta_0)
\end{align*}
where $f$ is notably a function only of $\theta_0$. Recall from the dimensional analysis lecture that
\[ P = \sqrt{\frac{l}{g}}H\left( \frac{d}{\ell} \right) \]
noting that $d/\ell$ and $\theta$ both define the initial condition. So we have deduced this unknown function $H$. This integral is difficult to compute exactly; however, we can compute an approximation when $\theta_0$ (and hence $\theta$) is small.
\begin{align*}
    F(\theta_0) & = \int_0^{\theta_0} \frac{\dd \theta}{\sqrt{\theta_0^2 - \theta^2}} \\
                & = \frac{\pi}{2}
\end{align*}
which is independent of $\theta_0$. Hence, for small angles,
\[ P \approx 2\pi \sqrt{\frac{\ell}{g}} \]

\subsection{Force and Potential in Three Spatial Dimensions}
Consider a particle moving in three spatial dimensions under a force $\vb F$. Then Newton's second law states
\[ m \ddot {\vb r} = \vb F \]
We define the kinetic energy by
\[ T = \frac{1}{2}\abs{\dot {\vb r}}^2 = \frac{1}{2}\abs{\vb u}^2 \]
Then
\[ \frac{\dd{T}}{\dd{t}} = m \dot {\vb r} \cdot \ddot {\vb r} = \vb F \cdot \dot{\vb r} = \vb F \cdot \vb u \]
This is the rate of working of the force on the particle. Let us consider the total work done by a force on a particle as it moves along a finite curve $C$ from $t_1$ to $t_2$. Then the total work done is the line integral
\[
    W = \int_{t_1}^{t_2} \vb F \cdot \vb u \dd{t}
    = \int_{t_1}^{t_2} \vb F \cdot \dot {\vb r} \dd{t}
    = \int_{\vb r(t_1)}^{\vb r(t_2)} \vb F \cdot \dd \vb r
\]
Note that we must specify that this integral acts along the curve $C$, since any other curve could connect the points $\vb r(t_1)$ and $\vb r(t_2)$. We can write this integral in terms of coordinates:
\[ \int_{\vb r(t_1)}^{\vb r(t_2)} F_x \dd{x} + F_y \dd{y} + F_z \dd{z} \]
Now, if force is only a function of the position $\vb r$, then we say that $\vb F(\vb r)$ defines a force field. A \textit{conservative} force field is such that
\[ \vb F(\vb r) = -\grad V(\vb r) \]
for some function $V(\vb r)$. In component form, this is equivalent to
\[ F_i = -\frac{\partial V}{\partial x_i} \]
If the force is conservative, then the energy $E = T + V(\vb r)$ is conserved.
\begin{proof}
    \[ \frac{\dd{E}}{\dd{t}} = \frac{\dd{T}}{\dd{t}} + \frac{\dd}{\dd{t}}V(\vb r) = m \dot{\vb r} \cdot \ddot{\vb r} + \grad V \cdot \dot{\vb r} = m\dot{\vb r} \cdot \ddot{\vb r} - m\ddot{\vb r} \cdot \dot{\vb r} = 0 \]
\end{proof}
\noindent Let us consider the total work done on the particle under a conservative force. From the properties of the gradient vector,
\[ W = \int_C \vb F \cdot \dd \vb r = -\int_C \grad V \cdot \dd \vb r = V(\vb r_1) - V(\vb r_2) \]
Note that this is dependent only on the end points of the curve; it is irrelevant of the path taken. Hence, if $C$ is closed, then no net work is done by the force. Note that in general, $F(\vb r)$ is not conservative, so in general there is no $V(\vb r)$ such that $\vb F = -\grad V$. In fact, $\vb F(\vb r)$ is conservative if
\[ \grad \times \vb F(\vb r) = \vb 0 \]

\section{Gravitational and Electromagnetic Forces}
\subsection{Gravity}
The gravitational force experienced by a mass $m$ at position vector $\vb r$ relative to a mass $M$ is given by
\[ \vb F = \frac{- G M m}{\abs{\vb r}^3} \cdot \vb r = \frac{- G M m}{\abs{\vb r}^2} \cdot \hat{\vb r} \]
This is a conservative force:
\[ \vb F(\vb r) = -\grad V(\vb r);\quad V(\vb r) = \frac{-GMm}{r} \]
To remove the factor of $m$, we define the `gravitational potential' $\Phi_g$ to be
\[ \Phi_g(\vb r) = \frac{-GM}{r} \]
We further define the gravitational field
\[ \vb g(\vb r) = -\grad \Phi_g(\vb r) = \frac{-GM}{r^2}\hat{\vb r} \]
Note that this is dependent only on $M$, and not $m$. These quantities are related to $\vb F$ and $V$ by scale factors of $m$.
\[ V(\vb r) = m \Phi_g(\vb r);\quad \vb F(\vb r) = m\vb g \]
We can generalise these expressions to define the gravitational potential associated with many point masses $M_i$ for $i = 1, \dots, n$. Then,
\[ \Phi_g(\vb r) = -\sum_{i=1}^n \frac{GM_i}{\abs{\vb r - \vb r_i}} \]
\[ \vb g(\vb r) = -\sum_{i=1}^n \frac{GM_i}{\abs{\vb r - \vb r_i}^3}(\vb r - \vb r_i) \]
We can extend this to a continuous mass distribution by generalising the summation into an integral. In particular, for a uniform spherical distribution of mass centred at the origin, we have that outside the sphere
\[ \Phi_G(\vb r) = \frac{-GM}{r} \]
which is equivalent to the formula for a point mass at the origin. So we can represent any spherical distribution of mass as a particle, provided we never consider behaviour inside the sphere.

\subsection{Gravitational and Inertial Mass}
Note that in the equations for gravitational force, mass plays two roles.
\begin{itemize}
    \item Inertial mass: In Newton's second law, $m \ddot{\vb r} = \vb F$ shows that the mass encapsulates the resistance to motion
    \item Gravitational mass: In the law of gravitation, $F = \frac{-GMm}{r^2}\hat{\vb r}$, showing the scale factor by which the mass affects the force.
\end{itemize}
It turns out that these `masses' are not exactly the same; they differ by a factor of around \num{1e-12}. In this course, we will consider these masses to be identical since the factor is very small.

\subsection{One-Dimensional Approximation to Gravity}
Let us consider a one-dimensional approximation. Consider a mass $m$ at some height $z$ above the surface of a planet of mass $M$ and radius $R$, where $z \ll R$. Using the binomial expansion, the potential is approximated by
\[ V(R + z) = \frac{-GMm}{R + z} \approx \frac{-GMm}{R} + \frac{GMmz}{R^2} - \dots \]
The first term in the expansion is a constant, and the second term is $mgz$ where $g$ is a constant. So when $z \ll R$,
\[ V(R + z) \approx mgz;\quad g = \frac{GM}{R^2} \approx \SI{9.8}{\metre\per\second\squared} \]

\subsection{Escape Velocity}
Consider a particle leaving the surface of a planet of mass $M$ and radius $R$, starting with velocity $v$. Can this particle escape the gravitational attraction of the planet, and fly off to infinity? By conservation of energy,
\[ E = T + V = \frac{1}{2}mv^2 - \frac{GMm}{r} \]
If $E < 0$, the particle does not have sufficient energy to leave the `potential well' $V$. If $E > 0$, the particle can escape to infinity. The critical velocity $v_0$ at which the particle can escape with lowest energy (the escape velocity) is therefore computed by setting $E = 0$ at $r=R$, i.e.
\[ \frac{1}{2}v_0^2 = \frac{GM}{R} \implies v_0 = \sqrt{\frac{2GM}{R}} \]
Note that light has a finite velocity, $c$. Therefore it must be possible that a mass is large enough that even the speed of light is insufficient for a particle to escape from a given radius. This describes a black hole. Of course, at this point we would need to invoke Einstein's theory of relativity in order to properly describe the behaviour of such an object.

\subsection{Electromagnetism}
We know that the force $\vb F$ acting on a particle with charge $q$ is
\[ \vb F = q\vb E + q\dot{\vb r} \times \vb B \]
where $\vb E, \vb B$ are functions of $\vb r$ and $t$. This is known as the Lorentz force law. Let us first consider time-independent fields $\vb E(\vb r), \vb B(\vb r)$ as a simplification. In this case, we can write
\[ \vb E = -\grad \Phi_e(\vb r) \]
where $\Phi_e$ is the electrostatic potential. The force $q\vb E$ is therefore conservative. We now prove that for time independent $\vb E(\vb r)$ and $\vb B(\vb r)$, $\vb F$ is conservative.
\begin{proof}
    \begin{align*}
        E                     & = \frac{1}{2}m \abs{\dot{\vb r}}^2 + q\Phi_e(\vb r)                         \\
        \frac{\dd{E}}{\dd{t}} & = m \dot{\vb r} \cdot \ddot{\vb r} + q\dot{\vb r} \cdot \grad \Phi_e(\vb r) \\
                              & = \dot{\vb r} \cdot (m \ddot{\vb r} + q\grad \Phi_e)                        \\
                              & = \dot{\vb r} \cdot (q \vb E + q \dot{\vb r} \times \vb B + q \grad \Phi_e) \\
                              & = \dot{\vb r} \cdot (q \dot{\vb r} \times \vb B)                            \\
                              & = 0
    \end{align*}
    since this is a triple product where two of the vectors are parallel. Since $\vb B$ acts perpendicular to the velocity, it does not do work on the particle.
\end{proof}
\noindent Analogously to point masses, we may consider point charges. A particle with charge $Q$ located at the origin generates an electrostatic potential and electric field
\[ \Phi_e(\vb r) = \frac{Q}{4\pi\varepsilon_0 r};\quad \vb E(\vb r) = -\grad \Phi_e = \frac{Q}{4\pi\varepsilon_0 r^2}\hat{\vb r} \]
where $\varepsilon_0 = \SI{8.85e-12}{\per\metre\cubed\per\kilogram\second\squared\coulomb\squared}$ is the electric constant. So the force on a particle of charge $q$ located at $\vb r$ is given by
\[ \vb F = -q\grad \Phi_e = \frac{Qq}{4\pi\varepsilon_0 r^2}\hat{\vb r} \]
This is called the Coulomb force. A negative sign is an attractive force; a positive sign is a repulsive force. This can be seen by considering a perturbation from the origin.

\section{Friction}
\subsection{Definition}
Friction is a contact force, unlike the forces we have discussed previously. It is a convenient encapsulation of many complicated molecular phenomena; it is not a fundamental force.

\subsection{Dry Friction}
The friction associated with solid bodies in contact is called `dry' friction. It has two associated forces: the normal force $\vb N$ perpendicular to the contact surface, which prevents objects from passing through each other, and the tangential force $\vb F$ parallel to the contact surface, which resists the relative tangential motion of the bodies in contact. When the two bodies are static, the empirically-derived formula relating the forces is
\[ \abs{\vb F} \leq \mu_s \abs{\vb N} \]
where $\mu_s$ is the coefficient of static friction. If the objects start to move relative to each other, this is kinetic friction. In this case,
\[ \abs{\vb F} = \mu_k \abs{\vb N} \]
where $\mu_k$ is the coefficient of kinetic friction. Generally $\mu_s > \mu_k > 0$.

\subsection{Fluid Drag}
When a solid body moves through a fluid (a liquid or a gas), it experiences a drag force. There are two important equations that model fluid drag. The linear drag formula is
\[ \vb F = -k_1 \vb u \]
This formula is most relevant to `small' objects, moving through a viscous fluid. Stokes' drag law for a moving sphere states that
\[ k_1 = 6 \pi \eta R \]
where $\eta$ is the viscosity of the sphere, and $R$ is the radius of the sphere. The quadratic drag formula is
\[ \vb F = -k_2 \abs{\vb u} \vb u \]
This formula is more relevant to `large' objects, moving through a less viscous fluid. Of course, $k_1 \neq k_2$ since they have different dimensions. Typically, we have
\[ k_2 = \rho_{\text{fluid}} C_D R^2 \]
where $C_D$ is the drag coefficient, and $R^2$ is the size of the cross section.

\subsection{Work Done by Friction}
Note that since friction always acts in a direction opposite to a component of motion. So if there is a frictional force, the body loses kinetic energy if the fluid (or other body) is assumed to be at rest. The rate of work under a fluid's drag force is
\[ \vb F \cdot \vb u = \begin{cases}
        -k_1 \abs{\vb u}^2 & \text{linear drag}    \\
        -k_2 \abs{\vb u}^3 & \text{quadratic drag}
    \end{cases} \]
In the latter case, the total work done is proportional to $\abs{\vb u}^2$ multiplied by the total distance travelled. The fluid gains energy, which may manifest as heat.

\subsection{Projectiles Experiencing Linear Drag}
Let us consider the example of a projectile moving through the air, under uniform gravity and a linear drag force.
\[ m \frac{\dd \vb u}{\dd{t}} = m\vb g - k\vb u \]
Solving with an integrating factor, we have
\[ \frac{\dd}{\dd{t}}\left( \vb u e^{kt/m} \right) = m\vb g e^{kt/m} \]
\[ \vb u = \frac{m\vb g}{k} + \vb C e^{-kt/m} \]
We can find $\vb C$ using the initial conditions, say at $t=0$, $\vb x = 0, \vb u = \vb U$.
\[ \vb u = \frac{m\vb g}{k} + \left( \vb U - \frac{-\vb g}{k} \right) e^{-kt/m} \]
Then
\begin{align*}
    \vb x & = \frac{m\vb g}{k}t - \frac{m}{k}\left( \vb U - \frac{-\vb g}{k} \right) e^{-kt/m} + D               \\
          & =\frac{m\vb g}{k}t + \frac{m}{k}\left( \vb U - \frac{-\vb g}{k} \right) \left( 1 - e^{-kt/m} \right)
\end{align*}
Now, considering the components of $\vb x = (x, y, z)$ and $\vb u = (u, v, w)$, we can choose
\[ \vb U = (U \cos \theta, 0, U\sin\theta);\quad \vb g = (0, 0, -g) \]
Then
\[ \begin{pmatrix}
        u \\ v \\ w
    \end{pmatrix} = \begin{pmatrix}
        U\cos\theta e^{-kt/m} \\
        0                     \\
        \left(U \sin\theta + \frac{mg}{k}\right)e^{-kt/m} - \frac{mg}{k}
    \end{pmatrix} \]
Note that the terminal velocity is $(0, 0, -mg/k)$, achieved on a time scale of $m/k$ (as seen from the exponential term). Further,
\[ \begin{pmatrix}
        x \\ y \\ z
    \end{pmatrix} = \begin{pmatrix}
        \frac{mU\cos\theta}{k}\left( 1 - e^{-kt/m} \right) \\
        0                                                  \\
        \frac{m}{k}\left( U \sin\theta + \frac{mg}{k} \right)\left( 1 - e^{-kt/m} \right) - \frac{mgt}{k}
    \end{pmatrix} \]
There exists a range $R$ of this particle, since initially the particle moves upwards, but as time increases the particle begins moving downwards again. $R$ is a function of $U, \theta, m, k, g$. We can construct the dimensionless group $\frac{kU}{mg} = \frac{U/g}{m/k}$, which can be thought of as the gravitational time scale divided by the frictional time scale. Dimensional analysis shows that
\[ R = \frac{U^2}{g}F\left(\theta, \frac{kU}{mg}\right) \]
When $\frac{kU}{mg} \ll 1$, this is very small friction.
\[ R = \frac{U^2}{g}\cdot 2\sin\theta\cos\theta \]
When $\frac{ku}{mg} \gg 1$, this is very large friction.
\[ R = \frac{U^2}{g} \left( \frac{mg}{kU}cos\theta \right) \]
$R$ is a decreasing function of $\frac{kU}{mg}$.

\section{Angular Motion and Orbits}
\subsection{Angular Momentum}
We define the angular momentum for a particle with position vector $\vb r(t)$, of mass $m$, moving under the influence of a force $F$ as
\[ \vb L = \vb r \times p = \vb r \times m \dot{\vb r} \]
Then
\[ \dot{\vb L} = m\dot{\vb r} \times \dot{\vb r} + m\vb r \times \ddot{\vb r} = \vb r \times \vb F = \vb G \]
This term $\vb r \times \vb F = \vb G$ is sometimes called the torque or the moment of the force. The values of $\vb L$ and $\vb G$ depend on the choice of origin, so we typically refer to the angular momentum about a particular point. If $\vb r \times \vb F = \vb 0$, then the angular momentum is conserved. The angular momentum around some suitably chosen point may be constant; this may help with calculations since we are free to choose the origin.

\subsection{Orbits}
We will begin the topic of orbits by considering the problem of gravitational orbits. Let
\[ m\ddot{\vb r} = -\grad V(r) \]
This represents a particle moving in a conservative force that is a function only of the radius from the origin. For this problem, we are assuming that the `central' mass remains fixed at the origin. This is a good approximation if the central mass is significantly larger than $m$.

\subsection{Central Forces}
We define a central force as a conservative force with the potential $V(r)$ being a function only of the radius from the origin. Consequently,
\[ \vb F = -\grad V(r) = -\grad V(\abs{\vb r}) = -\frac{\dd{V}}{\dd{r}}\hat{\vb r} \]
Consider the angular momentum $L$ about the origin, given by
\[ \dot{\vb L} = \vb r \times \vb F = \vb r \times \left(-\frac{\dd{V}}{\dd{r}}\hat{\vb r}\right) = 0 \]
So angular momentum about the origin is conserved for any central force. Further, from the definition of $\vb L$,
\[ \vb L \cdot \vb r = 0 \]
Hence, the motion of the particle is confined to the plane through the origin, perpendicular to $\vb L$. This reduces a three-dimensional problem into a two-dimensional problem.

\subsection{Polar Coordinates in the Plane}
A convenient choice of coordinates to use is the set of two-dimensional polar coordinates, by choosing the $z$ axis such that the orbit lies in the plane $z=0$. Then
\[ x = r\cos\theta;\quad y = r\sin\theta \]
Then, relative to the Cartesian axes,
\[ \vb e_r = \hat{\vb r} = \begin{pmatrix}
        \cos\theta \\ \sin\theta
    \end{pmatrix};\quad \vb e_\theta = \begin{pmatrix}
        -\sin\theta \\ \cos\theta
    \end{pmatrix} \]
At any point, $\vb e_r$, $\vb e_\theta$ form an orthonormal basis, but the basis can point in different directions for different values of $\theta$. In other words, they form a set of orthonormal curvilinear coordinates. We have
\[ \frac{\dd}{\dd \theta}\vb e_r = \vb e_\theta;\quad \frac{\dd}{\dd \theta}\vb e_\theta = -\vb e_r \]
Note that for a moving particle, $r$ and $\theta$ are functions of position, and hence functions of time. So we can use the following results:
\[ \frac{\dd \vb e_r}{\dd{t}} = \frac{\dd \theta}{\dd{t}} \frac{\dd \vb e_r}{\dd \theta} = \vb e_\theta \frac{\dd \theta}{\dd{t}};\quad \frac{\dd \vb e_\theta}{\dd{t}} = \frac{\dd \theta}{\dd{t}} \frac{\dd \vb e_\theta}{\dd \theta} = -\vb e_r \frac{\dd \theta}{\dd{t}} \]
We can compute expressions for velocity and acceleration in terms of these new coordinates.
\begin{align*}
    \vb r                  & = r \vb e_r                                    \\
    \therefore \dot{\vb r} & = \dot r \vb e_r + r \frac{\dd}{\dd{t}}\vb e_r \\
                           & = \dot r \vb e_r + r \dot\theta \vb e_\theta
\end{align*}
So $\dot r$ is the radial component of the velocity, and $r\dot\theta$ is the angular component of the velocity. Note that $\dot\theta$ is the angular velocity. Further:
\begin{align*}
    \ddot{\vb r} & = \frac{\dd}{\dd{t}}\left( \dot r \vb e_r + r \dot \theta \vb e_\theta \right)                                                                                        \\
                 & = \ddot r \vb e_r + \dot r \dot{\vb e}_r + \dot r \dot \theta \vb e_\theta + r \ddot \theta \vb e_\theta + r \dot \theta \dot{\vb e}_\theta                           \\
                 & = \ddot r \vb e_r + \dot r \dot \theta \vb e_\theta + \dot r \dot \theta \vb e_\theta + r \ddot \theta \vb e_\theta + r \dot \theta \left(-\dot \theta \vb e_r\right) \\
                 & = \left(\ddot r - r \dot \theta^2\right) \vb e_r + \left(2\dot r\dot \theta + r\ddot \theta\right) \vb e_\theta
\end{align*}
Again we can read off the radial and angular components of the acceleration.

\subsection{Circular Motion}
Let us consider the example of circular motion with constant angular velocity. Then we can set $r = a$, $\dot\theta = \omega$, and let $\dot r = \ddot r = \ddot \theta = 0$. We can find that
\[ \dot {\vb r} = a \omega \vb e_\theta ;\quad \ddot{\vb r} = -a\omega^2 \vb e_r \]
The acceleration is in the inward radial direction, which constrains the particle to follow a circlar path instead of flying off tangentially towards infinity. Therefore, by Newton's second law, there is a constant force in this direction.

\section{Orbits and Stability}
\subsection{Motion in a Central Force Field}
By Newton's second law, the force in a central force field is given by
\[ m\ddot{\vb r} = \vb F = -\grad V = -\frac{\dd{V}}{\dd{r}} \vb e_r \]
The results from the previous lecture give
\begin{equation}
    m\left(\ddot r - r\dot\theta^2\right) \vb e_r + m\left(2\dot r \dot \theta + r \ddot \theta\right) \vb e_\theta = \frac{-\dd{V}}{\dd{r}} \vb e_r \tag{$\ast$}
\end{equation}
But the right hand side has no angular component, so $m\left(2\dot r \dot \theta + r \ddot \theta\right) = 0$. Then
\[ \frac{m}{r} \frac{\dd}{\dd{t}}\left(r^2 \dot\theta\right) = 0 \]
So the quantity $h = r^2 \dot\theta$ is constant. Note that the angular momentum $\vb L$ is given by
\[ \vb L = m\vb r \times \dot{\vb r} = mr\vb e_r \times \left( \dot r\vb e_r + r\dot\theta \vb e_\theta \right) = mr^2 \dot\theta \vb e_z \]
Hence the magnitude of the angular momentum is constant. Now, let us consider the radial component in $(\ast)$.
\begin{align*}
    m\ddot r - mr\dot\theta^2 & = -\frac{\dd{V}}{\dd{r}}                    \\
    m\ddot r                  & = -\frac{\dd{V}}{\dd{r}} + \frac{mh^2}{r^3} \\
    m\ddot r                  & = -\frac{\dd{V}_\text{eff}}{\dd{r}}
\end{align*}
where
\[ V_\text{eff}(r) = V(r) + \frac{mh^2}{2r^2} \]
where $V_\text{eff}$ is called the effective potential. In other words, the motion of the particle is equivalent to one-dimensional motion under the influence of the effective potential. The energy of the particle is given by
\[ T + V(r) = \frac{1}{2}m\left( \dot r^2 + r^2 \dot\theta^2 \right) + V(r) = \frac{1}{2}m\dot r^2 + \frac{mh^2}{2r^2} + V(r) = \frac{1}{2}m\dot r^2 + V_\text{eff}(r) \]
which is consistent with our description of the effective potential.

\subsection{Orbits under Gravity}
As an example, let us consider
\[ V(r) = \frac{-GMm}{r};\quad V_\text{eff}(r) = \frac{-GMm}{r} + \frac{mh^2}{2r^2} \]
The effective potential has a single minimum point at $r_\ast$, and a single root at $r_0$. In other words, $V'_\text{eff}(r_\ast) = 0$ and $V_\text{eff}(r_0) = 0$. We can compute that
\[ r_0 = \frac{h^2}{2GM};\quad r_\ast = \frac{h^2}{GM} \]
The minimum energy is therefore
\[ E_\text{min} = \frac{-m(GM)^2}{2h^2} \]
What is the possible motion of the particle? At $E = E_\text{min}$, we have $r(t) = r_\ast$, an equilibrium position. Further, $\dot\theta = \frac{h}{r_\ast^2}$ everywhere. At $E_\text{min} < E < 0$, then $r(t)$ oscillates between a minimum point (periapsis/perihelion/perigee) and a maximum point (apoapsis/aphelion/apogee), and $\dot\theta$ varies. If $E_\text{min} \geq 0$, the particle escapes to infinity. This is sometimes called an unbound orbit.

\subsection{Stability of Circular Orbits}
Consider a general potential $V(r)$. Does a circular orbit exist, and is it stable? We will assume that the angular momentum is given and non-zero. For a circular orbit, the radius is a constant value $r_\ast$, so $\ddot r = 0$ and hence $V'_\text{eff}(r_\ast) = 0$. We know that we have a stable equilibrium if $V_\text{eff}$ has a minimum at this point. Correspondingly, it is unstable if this is a maximum. So, for instance, it is stable if $V''_\text{eff}(r_\ast) > 0$. Now, let us rewrite these conditions in terms of $V(r)$.
\[ V'(r_\ast) - \frac{mh^2}{r_\ast^3} = 0;\quad V''(r_\ast) = V''(r_\ast) + \frac{3mh^2}{r_\ast^4} > 0 \]
We can combine these to give the condition for stability as
\[ V''(r_\ast) + \frac{3V'(r_\ast)}{r_\ast} > 0 \]
Now let us consider an example,
\[ V(r) = \frac{-km}{r^p} \]
where $p>0$, $k>0$. If $p=1$, this is an example of an inverse square law. We have a circular orbit if
\[ \frac{pkm}{r_\ast^{p+1}} - \frac{mh^2}{r_\ast^3} = 0 \]
Hence,
\[ r_\ast^{p-2} = \frac{pkm}{mh^2} \implies r_\ast = \left( \frac{pkm}{mh^2} \right)^{\frac{1}{p-2}} \]
So there exists a circular orbit for all $h$ provided $p \neq 2$. Is this a stable orbit?
\[ V''(r_\ast) + \frac{3V'(r_\ast)}{r_\ast} = \frac{-kmp(p+1)}{r_\ast^{p+2}} + \frac{3kmp}{r_\ast^{p+2}} = \frac{p(2-p)km}{r_\ast^{p+2}} \]
So this is greater than zero (stable) if $0 < p < 2$ and less than zero (unstable) if $p > 2$.

\subsection{The Orbit Equation}
What shape does a non-circular orbit trace out? We could in principle find $r(t)$ by the energy equation
\[ E = \frac{1}{2}m \dot r^2 + V_\text{eff}(r) = \text{constant} \]
Hence
\[ t = \pm \frac{m}{2} \int^r \frac{\dd{u}}{\sqrt{E - V_\text{eff}(u)}} \]
Then we can use $r(t)^2\dot \theta = h$ to deduce $\theta(t)$. However in practice, this is not useful. An analytic solution is only possible for a small family of effective potential functions. It is somewhat more convenient to find $r$ in terms of $\theta$, not in terms of $t$. We can write
\[ \frac{\dd}{\dd{t}} = \frac{\dd \theta}{\dd{t}} \frac{\dd}{\dd \theta} = \frac{h}{r^2} \frac{\dd}{\dd \theta} \]
Applying this to Newton's second law, we have
\[ m\frac{h}{r^2} \frac{\dd}{\dd \theta}\left( \frac{h}{r^2} \frac{\dd}{\dd \theta} r \right) - \frac{mh^2}{r^3} = F(r) \]
The $\frac{h}{r^2} \frac{\dd}{\dd \theta} r$ term suggests using the substitution $u = \frac{1}{r}$. Then
\[ mhu^2 \frac{\dd}{\dd \theta}\left( -h\frac{\dd{u}}{\dd \theta} \right) - mh^2u^3 = F(u^{-1}) \]
\[ \frac{\dd^2 u}{\dd \theta^2} + u = \frac{-1}{mh^2u^2}F(u^{-1}) \]
This is known as the orbit equation. We can solve this for $u$ as a function of $\theta$.

\end{document}