\subsection{Independence of events}
\begin{definition}
	Let \((\Omega, \mathcal F, \mathbb P)\) be a probability space.
	Let \(A, B \in \mathcal F\).
	\(A\) and \(B\) are called independent if
	\[
		\prob{A \cap B} = \prob{A} \cdot \prob{B}
	\]
	We write \(A \perp B\), or \(A \perp\!\!\!\perp B\).
	A countable collection of events \((A_n)\) is said to be independent if for all distinct \(i_1, \dots, i_k\), we have
	\[
		\prob{A_{i_1} \cap \dots \cap A_{i_k}} = \prod_{j=1}^k \prob{A_{i_j}}
	\]
\end{definition}
\begin{remark}
	To show that a collection of events is independent, it is insufficient to show that events are pairwise independent.
	For example, consider tossing a fair coin twice, so \(\Omega = \{ (0, 0), (0, 1), (1, 0), (1, 1) \}\).
	\(\prob{\{\omega\}} = \frac{1}{4}\).
	Consider the events \(A, B, C\) where
	\[
		A = \{ (0, 0), (0, 1) \};\quad B = \{ (0, 0), (1, 0) \};\quad C = \{ (1, 0), (0, 1) \}
	\]
	\[
		\prob{A} = \prob{B} = \prob{C} = \frac{1}{2}
	\]
	\begin{align*}
		\prob{A \cap B} = \prob{\{ (0, 0) \}} & = \frac{1}{4} = \prob{A} \cdot \prob{B} \\
		\prob{A \cap C} = \prob{\{ (0, 1) \}} & = \frac{1}{4} = \prob{A} \cdot \prob{C} \\
		\prob{B \cap C} = \prob{\{ (1, 0) \}} & = \frac{1}{4} = \prob{B} \cdot \prob{C}
	\end{align*}
	\[
		\prob{A \cap B \cap C} = \prob{\varnothing} = 0
	\]
\end{remark}
\begin{claim}
	If \(A \perp B\), then \(A \perp \stcomp{B}\).
\end{claim}
\begin{proof}
	\begin{align*}
		\prob{A \cap \stcomp{B}} & = \prob{A} - \prob{A \cap B}           \\
		                         & = \prob{A} - \prob{A} \cdot \prob{B}   \\
		                         & = \prob{A} \left( 1 - \prob{B} \right) \\
		                         & = \prob{A} \prob{\stcomp{B}}
	\end{align*}
	as required.
\end{proof}

\subsection{Conditional probability}
\begin{definition}
	Let \((\Omega, \mathcal F, \mathbb P)\) be a probability space.
	Let \(B \in \mathcal F\) with \(\prob{B} > 0\).
	We define the conditional probability of \(A\) given \(B\), written \(\prob{A \mid B}\) as
	\[
		\prob{A \mid B} = \frac{\prob{A \cap B}}{\prob{B}}
	\]
\end{definition}
\noindent Note that if \(A\) and \(B\) are independent, then
\[
	\prob{A \mid B} = \frac{\prob{A \cap B}}{\prob{B}} = \frac{\prob{A} \cdot \prob{B}}{\prob{B}} = \prob{A}
\]
\begin{claim}
	Suppose that \((A_n)\) is a disjoint sequence in \(\mathcal F\).
	Then
	\[
		\prob{\bigcup A_n \mathrel{\Big|} B} = \sum_{n} \prob{A_n \mid B}
	\]
	This is the countable additivity property for conditional probability.
\end{claim}
\begin{proof}
	\begin{align*}
		\prob{\bigcup A_n \mathrel{\Big|} B} & = \frac{\prob{(\bigcup A_n) \cap B}}{\prob{B}} \\
		                                     & = \frac{\prob{\bigcup (A_n \cap B)}}{\prob{B}} \\
		\intertext{By countable additivity, since that \((A_n \cap B)\) are disjoint,}
		                                     & = \sum_n \frac{\prob{A_n \cap B}}{\prob{B}}    \\
		                                     & = \sum_n \prob{A_n \mid B}
	\end{align*}
\end{proof}
\noindent We can think of \(\prob{\dots \mid B}\) as a new probability measure for the same \(\Omega\).

\subsection{Law of total probability}
\begin{claim}
	Suppose \((B_n)\) is a disjoint collection of events in \(\mathcal F\), such that \(\bigcup B = \Omega\), and for all \(n\), we have \(\prob{B_n} > 0\).
	If \(A \in \mathcal F\) then
	\[
		\prob{A} = \sum_n \prob{A \mid B_n} \cdot \prob{B_n}
	\]
\end{claim}
\begin{proof}
	\begin{align*}
		\prob{A} & = \prob{A \cap \Omega}                   \\
		         & = \prob{A \cap \left(\bigcup B_n\right)} \\
		         & = \prob{\bigcup(A \cap B_n)}             \\
		\intertext{By countable additivity,}
		         & = \sum_n \prob{A \cap B_n}               \\
		         & = \sum_n \prob{A \mid B_n} \prob{B_n}
	\end{align*}
\end{proof}

\subsection{Bayes' formula}
\begin{claim}
	Suppose \((B_n)\) is a disjoint sequence of events with \(\bigcup B_n = \Omega\) and \(\prob{B_n} > 0\) for all \(n\).
	Then
	\[
		\prob{B_n \mid A} = \frac{\prob{A \mid B_n} \prob{B_n}}{\sum_k \prob{A \mid B_k} \prob{B_k}}
	\]
\end{claim}
\begin{proof}
	\begin{align*}
		\prob{B_n \mid A} & = \frac{\prob{B_n \cap A}}{\prob{A}}                                       \\
		                  & = \frac{\prob{A \mid B_n} \prob{B_n}}{\prob{A}}                            \\
		\intertext{By the Law of Total Probability,}
		                  & = \frac{\prob{A \mid B_n} \prob{B_n}}{\sum_k \prob{A \mid B_k} \prob{B_k}}
	\end{align*}
\end{proof}
\noindent Note that on the right hand side, the numerator appears somewhere in the denominator.
This formula is the basis of Bayesian statistics.
It allows us to reverse the direction of a conditional probability --- knowing the probabilities of the events \((B_n)\), and given a model of \(\prob{A \mid B_n}\), we can calculuate the posterior probabilities of \(B_n\) given that \(A\) occurs.

\subsection{Bayes' formula for medical tests}
Consider the probability of getting a false positive on a test for a rare condition.
Suppose 0.1\% of the population have condition \(A\), and we have a test which is positive for 98\% of the affected population, and 1\% of those unaffected by the disease.
Picking an individual at random, what is the probability that they suffer from \(A\), given that they have a positive test?

We define \(A\) to be the set of individuals suffering from the condition, and \(P\) is the set of individuals testing positive.
Then by Bayes' formula,
\[
	\prob{A \mid P} = \frac{\prob{P \mid A}\prob{A}}{\prob{P \mid A}\prob{A} + \prob{P \mid \stcomp{A}}\prob{\stcomp{A}}} = \frac{0.98 \cdot 0.001}{0.98 \cdot 0.001 + 0.01 \cdot 0.999} \approx 0.09 = 9\%
\]
Why is this so low?
We can rewrite this instance of Bayes' formula as
\[
	\prob{A \mid P} = \frac{1}{1 + \frac{\prob{P \mid \stcomp{A}}\prob{\stcomp{A}}}{\prob{P \mid A}\prob{A}}}
\]
Here, \(\prob{\stcomp{A}} \approx 1, \prob{P \mid A} \approx 1\).
So
\[
	\prob{A \mid P} \approx \frac{1}{1 + \frac{\prob{P \mid \stcomp{A}}}{\prob{A}}}
\]
So this is low because the probability that \(\prob{P \mid \stcomp{A}} \gg \prob{A}\).
Suppose that there is a population of 1000 people and about 1 suffers from the disease.
Among the 999 not suffering from \(A\), about 10 will test positive.
So there will be about 11 people who test positive, and only 1 out of 11 (9\%) of those actually has the disease.

\subsection{Probability changes under extra knowledge}
Consider these three statements:
\begin{enumerate}[(a)]
	\item I have two children, (at least) one of whom is a boy.
	\item I have two children, and the eldest one is a boy.
	\item I have two children, one of whom is a boy born on a Thursday.
\end{enumerate}
What is the probability that I have two boys, given \(a\), \(b\) or \(c\)?
Since no further information is given, we will assume that all outcomes are equally likely.
We define:
\begin{itemize}
	\item \(BG\) is the event that the elder sibling is a boy, and the younger is a girl;
	\item \(GB\) is the event that the elder sibling is a girl, and the younger is a boy;
	\item \(BB\) is the event that both children are boys; and
	\item \(GG\) is the event that both children are girls.
\end{itemize}
Now, we have
\begin{enumerate}[(a)]
	\item \(\prob{BB \mid BB \cup BG \cup GB} = \frac{1}{3}\)
	\item \(\prob{BB \mid BB \cup BG} = \frac{1}{2}\)
	\item Let us define \(GT\) to be the event that the elder sibling is a girl, and the younger is a boy born on a Thursday, and define \(TN\) to be the event that the elder sibling is a boy born on a Thursday and the younger is a boy not born on a Thursday, and other events are defined similarly.
	      So
	      \begin{align*}
		      \prob{TT \cup TN \cup NT \mid GT \cup TG \cup TT \cup TN \cup NT} & = \frac{\prob{TT \cup TN \cup NT}}{\prob{GT \cup TG \cup TT \cup TN \cup NT}}                                                                                                                                                                                \\
		                                                                        & = \frac{\frac{1}{2}\frac{1}{7}\frac{1}{2}\frac{1}{7} + 2 \cdot \frac{1}{2}\frac{1}{7}\frac{1}{2}\frac{6}{7}}{2\cdot \frac{1}{2}\frac{1}{2}\frac{1}{7} + \frac{1}{2}\frac{1}{7}\frac{1}{2}\frac{1}{7} + 2 \cdot \frac{1}{2}\frac{1}{7}\frac{1}{2}\frac{6}{7}} \\
		                                                                        & = \frac{13}{27} \approx 48\%
	      \end{align*}
\end{enumerate}

\subsection{Simpson's paradox}
Consider admissions by men and women from state and independent schools to a university given by the tables
\medskip\begin{center}
	\begin{tabular}{c c c c}
		All applicants & Admitted & Rejected & \% Admitted \\ \midrule
		State          & 25       & 25       & 50\%        \\
		Independent    & 28       & 22       & 56\%        \\
	\end{tabular}
\end{center}
\medskip\begin{center}
	\begin{tabular}{c c c c}
		Men only    & Admitted & Rejected & \% Admitted \\ \midrule
		State       & 15       & 22       & 41\%        \\
		Independent & 5        & 8        & 38\%        \\
	\end{tabular}
\end{center}
\medskip\begin{center}
	\begin{tabular}{c c c c}
		Women only  & Admitted & Rejected & \% Admitted \\ \midrule
		State       & 10       & 3        & 77\%        \\
		Independent & 23       & 14       & 62\%        \\
	\end{tabular}
\end{center}
\medskip\noindent This is seemingly a paradox; both women and men are more likely to be admitted if they come from a state school, but when looking at all applicants, they are more likely to be admitted if they come from an independent school.
This is called Simpson's paradox; it arises when we aggregate data from disparate populations.
Let \(A\) be the event that an individual is admitted, \(B\) be the event that an individual is a man, and \(C\) be the event that an individual comes from a state school.
We see that
\begin{align*}
	\prob{A \mid B \cap C}          & > \prob{A \mid B \cap \stcomp{C}}          \\
	\prob{A \mid \stcomp{B} \cap C} & > \prob{A \mid \stcomp{B} \cap \stcomp{C}} \\
	\prob{A \mid C}                 & < \prob{A \mid \stcomp{C}}
\end{align*}
First, note that
\begin{align*}
	\prob{A \mid C} & = \prob{A \cap B \mid C} + \prob{A \cap \stcomp{B} \mid C}                                                                           \\
	                & = \frac{\prob{A \cap B \cap C}}{\prob{C}} + \frac{\prob{A \cap \stcomp{B} \cap C}}{\prob{C}}                                         \\
	                & = \frac{\prob{A \mid B \cap C} \prob{B \cap C}}{\prob{C}} + \frac{\prob{A \mid \stcomp{B} \cap C}\prob{\stcomp{B} \cap C}}{\prob{C}} \\
	                & = \prob{A \mid B \cap C} \prob{B \mid C} + \prob{A \mid \stcomp{B} \cap C} \prob{\stcomp{B} \mid C}                                  \\
	                & > \prob{A \mid B \cap \stcomp{C}} \prob{B \mid C} + \prob{A \mid \stcomp{B} \cap \stcomp{C}} \prob{\stcomp{B} \mid C}
\end{align*}
Let us also assume that \(\prob{B \mid C} = \prob{B \mid \stcomp{C}}\).
Then
\begin{align*}
	\prob{A \mid C} & > \prob{A \mid B \cap \stcomp{C}} \prob{B \mid \stcomp{C}} + \prob{A \mid \stcomp{B} \cap \stcomp{C}} \prob{\stcomp{B} \mid \stcomp{C}} \\
	                & = \prob{A \mid \stcomp{C}}
\end{align*}
So we needed to further assume that \(\prob{B \mid C} = \prob{B \mid \stcomp{C}}\) in order for the `intuitive' result to hold.
The assumption was not valid in the example, so the result did not hold.
