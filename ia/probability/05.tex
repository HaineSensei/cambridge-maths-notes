\subsection{Bayes' formula for medical tests}
Consider the probability of getting a false positive on a test for a rare condition.
Suppose 0.1\% of the population have condition \(A\), and we have a test which is positive for 98\% of the affected population, and 1\% of those unaffected by the disease.
Picking an individual at random that that they suffer from \(A\), given that they have a positive test?

We define \(A\) to be the set of individuals suffering from the condition, and \(P\) is the set of individuals testing positive.
Then by Bayes' formula,
\[
	\prob{A \mid P} = \frac{\prob{P \mid A}\prob{A}}{\prob{P \mid A}\prob{A} + \prob{P \mid \stcomp{A}}\prob{\stcomp{A}}} = \frac{0.98 \cdot 0.001}{0.98 \cdot 0.001 + 0.01 \cdot 0.999} \approx 0.09 = 9\%
\]
Why is this so low?
We can rewrite this instance of Bayes' formula as
\[
	\prob{A \mid P} = \frac{1}{1 + \frac{\prob{P \mid \stcomp{A}}\prob{\stcomp{A}}}{\prob{P \mid A}\prob{A}}}
\]
Here, \(\prob{\stcomp{A}} \approx 1, \prob{P \mid A} \approx 1\).
So
\[
	\prob{A \mid P} \approx \frac{1}{1 + \frac{\prob{P \mid \stcomp{A}}}{\prob{A}}}
\]
So this is low because the probability that \(\prob{P \mid \stcomp{A}} \gg \prob{A}\).
Suppose that there is a population of 1000 people and about 1 suffers from the disease.
Among the 999 not suffering from \(A\), about 10 will test positive.
So there will be about 11 people who test positive, and only 1 out of 11 (9\%) of those actually has the disease.

\subsection{Probability changes under extra knowledge}
Consider these three statements:
\begin{enumerate}[(a)]
	\item I have two children, (at least) one of whom is a boy.
	\item I have two children, and the eldest one is a boy.
	\item I have two children, one of whom is a boy born on a Thursday.
\end{enumerate}
What is the probability that I have two boys, given \(a\), \(b\) or \(c\)?
Since no further information is given, we will assume that all outcomes are equally likely.
We define:
\begin{itemize}
	\item \(BG\) is the event that the elder sibling is a boy, and the younger is a girl;
	\item \(GB\) is the event that the elder sibling is a girl, and the younger is a boy;
	\item \(BB\) is the event that both children are boys; and
	\item \(GG\) is the event that both children are girls.
\end{itemize}
Now, we have
\begin{enumerate}[(a)]
	\item \(\prob{BB \mid BB \cup BG \cup GB} = \frac{1}{3}\)
	\item \(\prob{BB \mid BB \cup BG} = \frac{1}{2}\)
	\item Let us define \(GT\) to be the event that the elder sibling is a girl, and the younger is a boy born on a Thursday, and define \(TN\) to be the event that the elder sibling is a boy born on a Thursday and the younger is a boy not born on a Thursday, and other events are defined similarly.
	      So
	      \begin{align*}
		      \prob{TT \cup TN \cup NT \mid GT \cup TG \cup TT \cup TN \cup NT} & = \frac{\prob{TT \cup TN \cup NT}}{\prob{GT \cup TG \cup TT \cup TN \cup NT}}                                                                                                                                                                                \\
		                                                                        & = \frac{\frac{1}{2}\frac{1}{7}\frac{1}{2}\frac{1}{7} + 2 \cdot \frac{1}{2}\frac{1}{7}\frac{1}{2}\frac{6}{7}}{2\cdot \frac{1}{2}\frac{1}{2}\frac{1}{7} + \frac{1}{2}\frac{1}{7}\frac{1}{2}\frac{1}{7} + 2 \cdot \frac{1}{2}\frac{1}{7}\frac{1}{2}\frac{6}{7}} \\
		                                                                        & = \frac{13}{27} \approx 48\%
	      \end{align*}
\end{enumerate}

\subsection{Simpson's paradox}
Consider admissions by men and women from state and independent schools to a university given by the tables
\medskip\begin{center}
	\begin{tabular}{c c c c}
		All applicants & Admitted & Rejected & \% Admitted \\ \midrule
		State          & 25       & 25       & 50\%        \\
		Independent    & 28       & 22       & 56\%        \\
	\end{tabular}
\end{center}
\medskip\begin{center}
	\begin{tabular}{c c c c}
		Men only    & Admitted & Rejected & \% Admitted \\ \midrule
		State       & 15       & 22       & 41\%        \\
		Independent & 5        & 8        & 38\%        \\
	\end{tabular}
\end{center}
\medskip\begin{center}
	\begin{tabular}{c c c c}
		Women only  & Admitted & Rejected & \% Admitted \\ \midrule
		State       & 10       & 3        & 77\%        \\
		Independent & 23       & 14       & 62\%        \\
	\end{tabular}
\end{center}
\medskip\noindent This is seemingly a paradox; both women and men are more likely to be admitted if they come from a state school, but when looking at all applicants, they are more likely to be admitted if they come from an independent school.
This is called Simpson's paradox; it arises when we aggregate data from disparate populations.
Let \(A\) be the event that an individual is admitted, \(B\) be the event that an individual is a man, and \(C\) be the event that an individual comes from a state school.
We see that
\begin{align*}
	\prob{A \mid B \cap C}          & > \prob{A \mid B \cap \stcomp{C}}          \\
	\prob{A \mid \stcomp{B} \cap C} & > \prob{A \mid \stcomp{B} \cap \stcomp{C}} \\
	\prob{A \mid C}                 & < \prob{A \mid \stcomp{C}}
\end{align*}
First, note that
\begin{align*}
	\prob{A \mid C} & = \prob{A \cap B \mid C} + \prob{A \cap \stcomp{B} \mid C}                                                                           \\
	                & = \frac{\prob{A \cap B \cap C}}{\prob{C}} + \frac{\prob{A \cap \stcomp{B} \cap C}}{\prob{C}}                                         \\
	                & = \frac{\prob{A \mid B \cap C} \prob{B \cap C}}{\prob{C}} + \frac{\prob{A \mid \stcomp{B} \cap C}\prob{\stcomp{B} \cap C}}{\prob{C}} \\
	                & = \prob{A \mid B \cap C} \prob{B \mid C} + \prob{A \mid \stcomp{B} \cap C} \prob{\stcomp{B} \mid C}                                  \\
	                & > \prob{A \mid B \cap \stcomp{C}} \prob{B \mid C} + \prob{A \mid \stcomp{B} \cap \stcomp{C}} \prob{\stcomp{B} \mid C}
\end{align*}
Let us also assume that \(\prob{B \mid C} = \prob{B \mid \stcomp{C}}\).
Then
\begin{align*}
	\prob{A \mid C} & > \prob{A \mid B \cap \stcomp{C}} \prob{B \mid \stcomp{C}} + \prob{A \mid \stcomp{B} \cap \stcomp{C}} \prob{\stcomp{B} \mid \stcomp{C}} \\
	                & = \prob{A \mid \stcomp{C}}
\end{align*}
So we needed to further assume that \(\prob{B \mid C} = \prob{B \mid \stcomp{C}}\) in order for the `intuitive' result to hold.
The assumption was not valid in the example, so the result did not hold.
