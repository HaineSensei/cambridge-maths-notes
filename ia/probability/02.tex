\subsection{Stirling's Formula}
Let \((a_n)\) and \((b_n)\) be sequences.
We will write \(a_n \sim b_n\) if \(\frac{a_n}{b_n} \to 1\) as \(n \to \infty\).
This is asymptotic equality.
\begin{theorem}[Stirling's Formula]
	\(n!
	\sim n^n\, \sqrt{2 \pi n}\, e^{-n}\) as \(n \to \infty\).
\end{theorem}
\noindent Let us first prove the weaker statement \(\log (n!) \sim n \log n\).
\begin{proof}
	Let us define \(l_n = \log (n!) = \log 2 + \log 3 + \dots + \log n\).
	For \(x \in \mathbb R\), we write \(\floor{x}\) for the integer part of \(x\).
	Note that
	\[
		\log \floor x \leq \log x \leq \log \floor{x+1}
	\]
	Let us integrate this from 1 to \(n\).
	\[
		\sum_{k=1}^{n-1} \log k \leq \int_1^n \log x\dd{x} \leq \sum_{k=2}^{n} \log k
	\]
	\[
		l_{n-1} \leq n \log n - n + 1 \leq l_n
	\]
	For all \(n\), therefore:
	\[
		n \log n - n + 1 \leq l_n \leq (n+1) \log (n+1) - (n+1) + 1
	\]
	Dividing through by \(n\log n\), we get
	\[
		\frac{l_n}{n \log n} \to 1
	\]
	as \(n \to \infty\).
\end{proof}
\noindent The following complete proof is non-examinable.
\begin{proof}
	For any twice-differentiable function \(f\), for any \(a < b\) we have
	\[
		\int_a^b f(x) \dd{x} = \frac{f(a) + f(b)}{2} (b - a) - \frac{1}{2}\int_a^b (x-a)(b-x)f''(x)\dd{x}
	\]
	Now let \(f(x) = \log x\), \(a=k\) and \(b=k+1\).
	Then
	\begin{align*}
		\int_k^{k+1} \log x \dd{x} & = \frac{\log k + \log(k+1)}{2} + \frac{1}{2}\int_k^{k+1} \frac{(x-k)(k+1-x)}{x^2}\dd{x}                            \\
		                           & = \frac{\log k + \log(k+1)}{2} + \frac{1}{2}\int_0^1 \frac{x(1-x)}{(x+k)^2}\dd{x}                                  \\
		\intertext{Let us take the sum for \(k=1, \dots, n-1\) of the equality.}
		\int_1^n \log x \dd{x}     & = \frac{\log ((n-1)!) + \log(n!)}{2} + \frac{1}{2}\sum_{k=1}^{n-1}\int_0^1 \frac{x(1-x)}{(x+k)^2}\dd{x}            \\
		n\log n - n + 1            & = \log (n!) - \frac{\log n}{2} + \sum_{k=1}^{n-1} a_k;\quad a_k = \frac{1}{2}\int_0^1 \frac{x(1-x)}{(x+k)^2}\dd{x} \\
		\log (n!)                  & = n \log n - n + \frac{\log n}{2} + 1 - \sum_{k=1}^{n-1} a_k                                                       \\
		n!
		                           & = n^n \, e^{-n} \, \sqrt n \, \exp\left( 1 - \sum_{k=1}^{n-1} a_k \right)
	\end{align*}
	Now, note that
	\[
		a_k \leq \frac{1}{2}\int_0^1 \frac{x(1-x)}{k^2}\dd{x} = \frac{1}{12k^2}
	\]
	So the sum of all \(a_k\) converges.
	We set
	\[
		A = \exp\left( 1 - \sum_{k=1}^\infty a_k \right)
	\]
	and then
	\[
		n!
		= n^n \, e^{-n} \, \sqrt n \, A \, \exp\left( \underbrace{\sum_{k=n}^\infty a_k}_{\text{converges to zero}} \right)
	\]
	Therefore,
	\[
		n!
		\sim n^n\, \sqrt{n}\, e^{-n}\, A
	\]
	To finish the proof, we must show that \(A = \sqrt{2 \pi}\).
	We can utilise the fact that \(n!
	\sim n^n\, \sqrt{n}\, e^{-n}\, A\).
	\begin{align*}
		2^{-2n} \binom{2n}{n} & = 2^{-2n} \cdot \frac{2n!}{(n!)^2}                                                                                           \\
		                      & \sim 2^{-2n} \frac{(2n)^{2n} \cdot \sqrt{2n} \cdot A \cdot e^{-2n}}{n^n\, e^{-n}\, \sqrt n\, A\, n^n\, e^{-n}\, \sqrt n\, A} \\
		                      & = \frac{\sqrt{2}}{A\sqrt{n}}
	\end{align*}
	Using a different method, we will prove that \(2^{-2n} \binom{2n}{n} \sim \frac{1}{\sqrt{\pi n}}\), which then forces \(A = \sqrt{2\pi}\).
	Consider
	\[
		I_n = \int_0^{\frac{\pi}{2}} (\cos \theta)^n \dd{\theta};\quad n \geq 0
	\]
	So \(I_0 = \frac{\pi}{2}\) and \(I_1 = 1\).
	By integrating by parts,
	\[
		I_n = \frac{n-1}{n}I_{n-2}
	\]
	Therefore,
	\[
		I_{2n} = \frac{2n-1}{2n}I_{2n-2} = \frac{(2n-1)(2n-3)\dots(3)(1)}{(2n)(2n-2)\dots(2)}I_0
	\]
	Multiplying the numerator and denominator by the denominator, we have
	\[
		I_{2n} = \frac{(2n)!}{(n!
			\cdot 2^n)^2} \cdot \frac{\pi}{2} = 2^{-2n} \frac{2n}{n} \cdot \frac{\pi}{2}
	\]
	In the same way, we can deduce that
	\[
		I_{2n+1} = \frac{(2n)(2n-2)\dots(2)}{(2n+1)(2n-1)\dots(3)(1)}I_1 = \frac{1}{2n+1} \left( 2^{-2n} \binom{2n}{n} \right)^{-1}
	\]
	From \(I_n = \frac{n-1}{n}I_{n-2}\), we get that
	\[
		\frac{I_n}{I_{n-2}} \to 1
	\]
	as \(n \to \infty\).
	We now want to show that \(\frac{I_{2n}}{I_{2n+1}} \to 1\).
	We see from the definition of \(I_n\) that \(I\) is a decreasing function of \(n\).
	Therefore,
	\[
		\frac{I_{2n}}{I_{2n+1}} \leq \frac{I_{2n-1}}{I_{2n+1}} \to 1
	\]
	and also
	\[
		\frac{I_{2n}}{I_{2n+1}} \geq \frac{I_{2n}}{I_{2n-2}} \to 1
	\]
	So
	\[
		\frac{I_{2n}}{I_{2n+1}} \to 1
	\]
	which means that
	\[
		\frac{2^{-2n} \binom{2n}{n} \frac{\pi}{2}}{\left( 2^{-2n} \binom{2n}{n} \right)^{-1} \frac{1}{2n+1}} \to 1 \implies \left( 2^{-2n} \binom{2n}{n} \right)^2 \frac{\pi}{2} (2n+1) \to 1
	\]
	Therefore,
	\[
		\left( 2^{-2n} \binom{2n}{n} \right)^2 \sim \frac{2}{\pi (2n+1)} \sim \frac{1}{\pi n}
	\]
	Finally,
	\[
		A = \sqrt{2 \pi}
	\]
	completes the proof.
\end{proof}

\subsection{Countable Subadditivity}
Let \((\Omega, \mathcal F, \mathbb P)\) be a probability space, and let \((A_n)_{n \geq 1}\) be a (not necessarily disjoint) sequence of events in \(\mathcal F\).
Then
\[
	\prob{\bigcup_{n=1}^\infty A_n} \leq \sum_{n=1}^\infty \prob{A_n}
\]
\begin{proof}
	Let us define a new sequence \(B_1 = A_1\) and \(B_n = A_n \setminus (A_1 \cup A_2 \cup \dots \cup A_{n-1})\).
	So by construction, the sequence \(B_n\) is a disjoint sequence of events in \(\mathcal F\).
	Note further that the union of all \(B_n\) is equal to the union of all \(A_n\).
	So
	\[
		\prob{\bigcup_{n=1}^\infty A_n} = \prob{\bigcup_{n=1}^\infty B_n}
	\]
	By the countable additivity axiom,
	\[
		\prob{\bigcup_{n=1}^\infty B_n} = \sum_{n=1}^\infty \prob{B_n}
	\]
	But \(B_n \subseteq A_n\).
	So \(\prob{B_n} \leq A_n\).
	Therefore,
	\[
		\prob{\bigcup_{n=1}^\infty A_n} \leq \sum_{n=1}^\infty \prob{A_n}
	\]
\end{proof}
