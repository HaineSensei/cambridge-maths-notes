\subsection{Multidimensional Gaussian random variables}
Recall that a random variable \(X\) with values in \(\mathbb R\) is called Gaussian (or normal) if
\[
	X = \mu + \sigma Z;\quad \mu \in \mathbb R, \sigma \geq 0, Z \sim \mathrm{N}(0, 1)
\]
The density function of \(X\) is
\[
	f(x) = \frac{1}{\sqrt{2\pi\sigma^2}} \exp(-\frac{(x-\mu)^2}{2\sigma^2})
\]
Now, let \(X = (X_1, \dots, X_n)^\transpose\) with values in \(\mathbb R^n\).
Then we define that \(X\) is a Gaussian vector (also called Gaussian) if
\[
	\forall u = \begin{pmatrix}
		u_1 \\ \vdots \\ u_n
	\end{pmatrix} \in \mathbb R^n,\, u^\transpose X = \sum_{i=1}^n u_i X_i = \mu + \sigma Z
\]
so any linear combination of the \(X_i\) is Gaussian.
This does not require that the \(X_i\) are independent, just that their sum is always Gaussian.

Let \(X\) be Gaussian in \(\mathbb R^n\).
Suppose that \(A\) is an \(m \times n\) matrix, and \(b \in \mathbb R^m\).
Then \(AX + b\) is also Gaussian.
Indeed, let \(u \in \mathbb R^m\), and let \(v = A^\transpose u\).
Then
\[
	u^\transpose(AX + b) = u^\transpose AX + u^\transpose b = v^\transpose X + u^\transpose b
\]
Since \(X\) is Gaussian, \(v^\transpose X\) is also Gaussian.
An additive constant preserves this property, so the entire expression is Gaussian.

\subsection{Expectation and variance}
We define the mean of a Gaussian vector \(X\) as
\[
	\mu = \expect{X} = \begin{pmatrix}
		\expect{X_1} \\ \vdots \\ \expect{X_n}
	\end{pmatrix};\quad \mu_i = \expect{X_i}
\]
We further define
\begin{align*}
	V & = \Var{X} = \expect{(X - \mu) (X - \mu)^\transpose} \\
	  & = \begin{pmatrix}
		\expect{(X_1 - \mu_1)^2}            & \expect{(X_1 - \mu_1)(X_2 - \mu_2)} & \cdots & \expect{(X_1 - \mu_1)(X_n - \mu_n)} \\
		\expect{(X_2 - \mu_2)(X_1 - \mu_1)} & \expect{(X_2 - \mu_2)^2}            & \cdots & \expect{(X_2 - \mu_2)(X_n - \mu_n)} \\
		\vdots                              & \vdots                              & \ddots & \vdots                              \\
		\expect{(X_n - \mu_n)(X_1 - \mu_1)} & \expect{(X_n - \mu_1)(X_n - \mu_2)} & \cdots & \expect{(X_n - \mu_n)^2}
	\end{pmatrix}
\end{align*}
Hence the components of \(V\) are
\[
	V_{ij} = \Cov{X_i, X_j}
\]
In particular, \(V\) is a symmetric matrix, and
\[
	\expect{u^\transpose X} = \expect{\sum_{i=1}^n u_i X_i} = \sum_{i=1}^n u_i \mu_i = u^\transpose \mu
\]
and
\[
	\Var{u^\transpose X} = \Var{\sum_{i=1}^n u_i X_i} = \sum_{i,j = 1}^n u_i \Cov{X_i, X_j} u_j = u^\transpose V u
\]
Hence \(u^\transpose X \sim \mathrm{N}(u^\transpose \mu, u^\transpose V u)\).
Further, \(V\) is a non-negative definite matrix.
Indeed, let \(u \in \mathbb R^n\).
Then \(\Var{u^\transpose X} = u^\transpose V u\).
Since \(\Var{u^\transpose X} \geq 0\), we have \(u^\transpose V u \geq 0\).

\subsection{Moment generating function}
We define the moment generating function of \(X\) by
\[
	m(\lambda) = \expect{e^{\lambda^\transpose X}}
\]
where \(\lambda \in \mathbb R^n\).
Then, we know that \(\lambda^\transpose X \sim \mathrm{N}(\lambda^\transpose \mu, \lambda^\transpose V \lambda)\).
Hence \(m(\lambda)\) is the moment generating function of a normal random variable with the above mean and variance, applied to the parameter \(\theta = 1\).
\[
	m(\lambda) = \exp(\lambda^\transpose \mu + \frac{\lambda^\transpose V\lambda}{2})
\]
Since the moment generating function uniquely characterises the distribution, it is clear that a Gaussian vector is uniquely characterised by its mean vector \(\mu\) and variance matrix \(V\).
In this case, we write \(X \sim \mathrm{N}(\mu, V)\).

\subsection{Constructing Gaussian vectors}
Given a \(\mu\) and a \(V\) matrix, we might like to create a Gaussian vector that has this mean and variance.
Let \(Z_1, \dots, Z_n\) be a list of independent and identically distributed standard normal random variables.
Let \(Z = (Z_1, \dots, Z_n)^\transpose\).
Then \(Z\) is a Gaussian vector.
\begin{proof}
	For any vector \(u \in \mathbb R^n\), we have
	\[
		u^\transpose Z = \sum_{i=1}^n u_i Z_i
	\]
	Because the \(Z_i\) are independent, it is easy to take the moment generating function to get
	\[
		\expect{\exp(\lambda \sum_{i=1}^n u_i z_i)} = \expect{\prod_{i=1}^n \exp(\lambda u_i Z_i)} = \prod_{i=1}^n \expect{\exp(\lambda u_i Z_i)} = \prod_{i=1}^n \exp(\frac{(\lambda u_i)^2}{2}) = \exp(\frac{\lambda^2 \abs{u}^2}{2})
	\]
	So \(u^\transpose Z \sim \mathrm{N}(0, \abs{u}^2)\), which is normal as required.
\end{proof}
Now, \(\expect{Z} = 0\), and \(\Var{Z} = I\), the identity matrix.
We then write \(Z \sim \mathrm{N}(0, I)\).
Now, let \(\mu \in \mathbb R^n\), and \(V\) be a non-negative definite matrix.
We want to construct a Gaussian vector \(X\) such that its mean is \(\mu\) and its expectation is \(V\), by using \(Z\).
In the one-dimensional case, this is easy, since \(\mu\) is a single value, and \(V\) contains only one element, \(\sigma^2\).
In this case therefore, \(Z \sim \mathrm{N}(0, 1)\) so \(\mu + \sigma Z \sim \mathrm{N}(\mu, \sigma^2)\).
In the general case, since \(V\) is non-negative definite, we can write
\[
	V = U^\transpose DU
\]
where \(U^{-1} = U^\transpose\), and \(D\) is a diagonal matrix with diagonal entries \(\lambda_i \geq 0\).
We define the square root of the matrix \(V\) to be
\[
	\sigma = U^\transpose \sqrt{D} U
\]
where \(\sqrt{D}\) is the diagonal matrix with diagonal entries \(\sqrt{\lambda_i}\).
Then clearly,
\[
	\sigma^2 = U^\transpose \sqrt{D} U U^\transpose \sqrt{D} U = U^\transpose \sqrt{D} \sqrt{D} U = U^\transpose DU = V
\]
Now, let \(X = \mu + \sigma Z\).
We now want to show that \(X \sim \mathrm{N}(\mu, V)\).
\begin{proof}
	\(X\) is certainly Gaussian, since it is generated by a linear multiple of the Gaussian vector \(Z\), with an added constant.
	By linearity,
	\[
		\expect{X} = \mu
	\]
	and
	\begin{align*}
		\Var{X} & = \expect{(X - \mu)(X - \mu)^\transpose}           \\
		        & = \expect{(\sigma Z)(\sigma Z)^\transpose}         \\
		        & = \expect{\sigma Z Z^\transpose \sigma^\transpose} \\
		        & = \sigma \expect{Z Z^\transpose} \sigma^\transpose \\
		        & = \sigma \sigma^\transpose                         \\
		        & = \sigma \sigma                                    \\
		        & = V
	\end{align*}
\end{proof}

\subsection{Density}
We can calculate the density of such a Gaussian vector \(X \sim \mathrm{N}(\mu, V)\).
First, consider the case where \(V\) is positive definite.
Recall that in the one-dimensional case,
\[
	f_X(x) = f_Z(z) \abs{J};\quad x = \mu + \sigma z
\]
In general, since \(V\) is positive definite, \(\sigma\) is invertible.
So \(x = \mu + \sigma z\) gives \(z = \sigma^{-1}(x-\mu)\).
Hence,
\begin{align*}
	f_X(x) & = f_Z(z) \abs{J}                                                                  \\
	       & = \prod_{i=1}^n \frac{\exp(-\frac{z_i^2}{2})}{\sqrt{2\pi}} \abs{\det \sigma^{-1}} \\
	       & = \frac{1}{(2\pi)^{n/2}}\exp(-\frac{\abs{z}^2}{2}) \cdot \frac{1}{\sqrt{\det V}}  \\
	       & = \frac{1}{\sqrt{(2\pi)^n \det V}}\exp(-\frac{z^\transpose z}{2})
\end{align*}
Now,
\begin{align*}
	z^\transpose z & = (\sigma^{-1} (x-\mu))^\transpose (\sigma^{-1} (x-\mu))          \\
	               & = (x-\mu)^\transpose (\sigma^{-1})^\transpose \sigma^{-1} (x-\mu) \\
	               & = (x-\mu)^\transpose \sigma^{-2} (x-\mu)                          \\
	               & = (x-\mu)^\transpose V^{-1} (x-\mu)
\end{align*}
Hence,
\[
	f_X(x) = \frac{1}{\sqrt{(2\pi)^n \det V}}\exp(-\frac{(x-\mu)^\transpose V^{-1} (x-\mu)}{2})
\]
In the case where \(V\) is just non-negative definite (so it could have some zero eigenvalues), we can make an orthogonal change of basis, and assume that
\[
	V = \begin{pmatrix}
		U & 0 \\
		0 & 0
	\end{pmatrix};\quad \mu = \begin{pmatrix}
		\lambda \\ \nu
	\end{pmatrix}
\]
where \(U\) is an \(m \times m\) positive definite matrix, where \(m < n\), and where \(\lambda \in \mathbb R^m\), \(\nu \in \mathbb R^{n-m}\).
For \(U\), we can then apply the result above.
We can write
\[
	X = \begin{pmatrix}
		Y \\ \nu
	\end{pmatrix}
\]
where \(Y\) has density
\[
	f_Y(y) = \frac{1}{\sqrt{(2\pi)^n \det U}}\exp(-\frac{(y-\lambda)^\transpose U^{-1} (y-\lambda)}{2})
\]

\subsection{Diagonal variance}
Note that if a Gaussian vector \(X = (X_1, \dots, X_n)\) is comprised of independent normal random variables, then \(V\) is a diagonal matrix.
Indeed, since the \(X_i\) are independent then \(\Cov{X_i, X_j} = 0\) for all \(i \neq j\), so \(V\) is diagonal.
\begin{lemma}
	If \(V\) is diagonal, then the \(X_i\) are independent.
\end{lemma}
Note that zero covariance does not in general imply independence, as we saw earlier in the course, but in this specific case with Gaussian variables, this is true.
\begin{proof}
	Since \(V\) is diagonal with diagonal entries \(\lambda_i\), we have
	\[
		(x-\mu)^\transpose V^{-1} (x-\mu) = \sum_{i=1}^n \frac{(x_i - \mu_i)^2}{\lambda_i}
	\]
	Hence,
	\[
		f_X(x) = \frac{1}{\sqrt{(2\pi)^n \det V}} \exp(-\sum_{i=1}^n \frac{(x_i - \mu_i)^2}{2\lambda_i})
	\]
	So \(f_X\) factorises into a product.
	Hence the \(X_i\) are independent.
\end{proof}
We can construct an alternative proof using moment generating functions.
\begin{proof}
	\begin{align*}
		m(\theta) & = \expect{e^{\theta^\transpose X}}                                                 \\
		          & = \exp(\theta^\transpose \mu + \frac{\theta^\transpose V \theta}{2})               \\
		          & = \exp(\sum_{i=1}^n \theta_i \mu_i + \frac{1}{2}\sum_{i=1}^n \theta_i^2 \lambda_i)
	\end{align*}
	Hence \(m(\theta)\) factorises into the moment generating functions of Gaussian random variables in \(\mathbb R\).
\end{proof}
In summary, for Gaussian vectors, we have \((X_1, \dots, X_n)\) independent if and only if \(V\) is diagonal.

\subsection{Bivariate Gaussian vectors}
A bivariate Gaussian is a Gaussian vector of two variables (\(n=2\)).
Let \(X = (X_1, X_2)\).
Let \(\mu_k = \expect{X_k}\) and \(\sigma_k^2 = \Var{X_k}\).
We further define the \textit{correlation}
\[
	\rho = \Corr{X_1, X_2} = \frac{\Cov{X_1, X_2}}{\sqrt{\Var{X_1}\Var{X_2}}}
\]
Note that due to the Cauchy-Schwarz inequality, we have \(\rho \in [-1, 1]\).
We can write the variance matrix as
\[
	V = \begin{pmatrix}
		\sigma_1^2             & \rho \sigma_1 \sigma_2 \\
		\rho \sigma_1 \sigma_2 & \sigma_2^2
	\end{pmatrix}
\]
This matrix \(V\) is non-negative definite.
Indeed, let \(u = \begin{pmatrix} u_1 \\ u_2 \end{pmatrix}\), then
\begin{align*}
	u^\transpose V u & = (1-\rho)(\sigma_1^2 u_1^2 + \sigma_2^2 u_2^2) + \rho(\sigma_1 u_1 + \sigma_2 u_2)^2 \\
	                 & = (1+\rho)(\sigma_1^2 u_1^2 + \sigma_2^2 u_2^2) - \rho(\sigma_1 u_1 - \sigma_2 u_2)^2
\end{align*}
Since \(\rho \in [-1, 1]\), this is non-negative for all choices of \(\rho\).

\subsection{Density of bivariate Gaussian}
When \(\rho = 0\) and \(\sigma_1, \sigma_2 > 0\), we have
\[
	f_{X_1, X_2}(x_1, x_2) = \prod_{i=1}^2 \frac{1}{\sqrt{2\pi \sigma_k^2}} \exp(-\frac{(x_k - \mu_k)^2}{2\sigma_k^2})
\]
So \(X_1\) and \(X_2\) are independent in this case.

\subsection{Conditional expectation}
Let \((X_1, X_2)\) be a bivariate Gaussian vector.
Then let \(a \in \mathbb R\), and consider \(X_2 - aX_1\).
We have
\[
	\Cov{X_2 - aX_1, X_1} = \Cov{X_2, X_1} - a\Cov{X_1, X_1} = \Cov{X_2, X_1} - a \Var{X_1} = \rho \sigma_1 \sigma_2 - a\sigma_1^2
\]
Now, let \(a = \frac{\rho \sigma_2}{\sigma_1}\), so \(\Cov{X_2 - aX_1, X_1} = 0\).
Since \(Y = X_2 - aX_1\) is Gaussian, \((X_1, Y)\) is a Gaussian vector, and so \(Y\) and \(X_1\) are independent.
Now, we can find
\begin{align*}
	\expect{X_2 \mid X_1} & = \expect{Y + aX_1 \mid X_1}          \\
	                      & = \expect{Y} + a\expect{X_1 \mid X_1} \\
	                      & = \expect{X_2 - aX_1} + aX_1
\end{align*}
In particular, since \(X_2 = (X_2 - aX_1) + aX_1\), we can say that given \(X_1\),
\[
	X_2 \sim \mathrm{N}(\mu_2 - a\mu_1 + aX_1, \Var{X_2 - aX_1})
\]
and
\[
	\Var{X_2 - aX_1} = \Var{X_2} + a^2 \Var{X_1} - 2a \Cov{X_1, X_2}
\]

\subsection{Multivariate central limit theorem}
This subsection is non-examinable, but included for completeness.
Let \(X\) be a random vector in \(\mathbb R^k\) with \(\mu = \expect{X}\) and covariance matrix \(\Sigma\).
Let \(X_1, X_2, \dots\) be independent and identically distributed with the same distribution as \(X\).
Then
\[
	S_n = \frac{1}{\sqrt{n}} \sum_{i=1}^n X_i - \expect{X_i} \convdist \mathrm{N}(\mu, \Sigma)
\]
Convergence in distribution here means that for all reasonable \(B \subseteq \mathbb R^k\), we have
\[
	\prob{S_n \in B} \to \prob{\mathrm{N}(\mu, \Sigma) \in B}
\]
