\subsection{Sampling from uniform distribution}
It is easy for a computer to generate a random number in the interval \([0, 1)\).

We can use this as a source of randomness to simulate a random variable with an arbitrary density.
Let \(U \sim \mathrm{U}[0, 1]\), then let \(X = -\log U\).
Then
\[
	\prob{X \leq x} = \prob{\log U \leq x} = \prob{U \geq e^{-x}} = 1-e^{-x}
\]
So \(X\) is exponentially distributed with parameter 1.
More generally, we have the following.
\begin{theorem}
	Let \(X\) be a continuous random variable with distribution function \(F\).
	Then, if \(U \sim \mathrm{U}[0, 1]\), then \(F^{-1}(U) \sim F\).
\end{theorem}
\begin{proof}
	Set \(Y = F^{-1}(U)\).
	Then
	\begin{align*}
		\prob{Y \leq x} & = \prob{F^{-1}(U) \leq x} \\
		                & = \prob{U \leq F(x)}      \\
		                & = F(x)
	\end{align*}
\end{proof}
\noindent One way of thinking of this function \(F^{-1}\) function is that it takes an input probability \(p\), and outputs the \(x\) value such that \(\prob{X \leq x} = p\).
Then, if \(U\) is uniformly distributed, we are essentially sampling a random \(p\).

\subsection{Rejection sampling}
In certain cases, finding such an \(F^{-1}\) function is difficult, if not impossible, especially where this function has jumps or has a higher dimension.
Here is an alternative sampling method.
Suppose \(A \subset [0, 1]^d\).
We then define
\[
	f(x) = \frac{1(x \in A)}{\abs{A}}
\]
where \(\abs{A}\) is the size or volume of this set \(A\).
Let \(X\) have density function \(f\).
How can we simulate \(X\)?
Let \((U_n)\) be an independent and identically distributed sequence of \(d\)-dimensional uniform random variables, i.e.
\[
	U_n = (U_{k, n} \colon k \in \{ 1, \dots, d \});\quad (U_{k, n}) \sim \mathrm{U}[0, 1] \text{ i.i.d.}
\]
Now, let
\[
	N = \min\qty{n \geq 1 \colon U_n \in A}
\]
So we keep generating random numbers until a \(U_n\) lies in \(A\), and reject all other possibilities.
We now show that \(U_N \sim f\).
In particular, we want to show that for all \(B \subseteq [0, 1]^d\),
\[
	\prob{U_n \in B} = \int_B f(x) \dd{x}
\]
We have
\begin{align*}
	\prob{U_n \in B} & = \sum_{n=1}^\infty \prob{U_N \in B, N = n}                                                    \\
	                 & = \sum_{n=1}^\infty \prob{U_n \in A \cap B, U_{n-1} \notin A, \dots, U_1 \notin A}             \\
	                 & = \sum_{n=1}^\infty \prob{U_n \in A \cap B} \prob{U_{n-1} \notin A} \cdots \prob{U_1 \notin A} \\
	                 & = \sum_{n=1}^\infty \abs{A \cap B} (1 -\abs{A})^{n-1}                                          \\
	                 & = \frac{\abs{A \cap B}}{\abs{A}}                                                               \\
	                 & = \int_A \frac{1(x \in B)}{\abs{A}} \dd{x}                                                     \\
	                 & = \int_B f(x) \dd{x}
\end{align*}
Now suppose that \(f\) is a density on \([0, 1]^{d-1}\) which is bounded by \(\lambda > 0\).
We can use rejection sampling to sample a random variable \(X\) with this density.
Consider the set
\[
	A = \qty{(x_1, \dots, x_d) \in [0, 1]^d \colon x_d \leq \frac{f(x_1, \dots, x_{d-1})}{\lambda}}
\]
From the above, we can generate a uniform random variable \(Y = (X_1, \dots, X_d)\) on \(A\).
Let \(X = (X_1, \dots, X_{d-1})\), then we will show that \(X \sim f\).
In particular, we want to show that for all \(B \subseteq [0, 1]^{d-1}\),
\[
	\prob{X \in B} = \int_B f(x) \dd{x}
\]
We find that
\begin{align*}
	\prob{X \in B}                 & = \prob{(X_1, \dots, X_{d-1}) \in B}                                                                                           \\
	                               & = \prob{(X_1, \dots, X_d) \in (B \times [0, 1]) \cap A}                                                                        \\
	                               & = \frac{\abs{(B \times [0, 1]) \cap A}}{\abs{A}}                                                                               \\
	\abs{(B \times [0, 1]) \cap A} & = \idotsint 1((X_1, \dots, X_d) \in (B \times [0, 1]) \cap A) \dd{x_1} \dots \dd{x_d}                                          \\
	                               & = \idotsint 1((X_1, \dots, X_{d-1}) \in B)\cdot 1\qty(x_d \leq \frac{f(x_1, \dots, x_{d-1})}{\lambda}) \dd{x_1} \dots \dd{x_d} \\
	                               & = \idotsint 1((X_1, \dots, X_{d-1}) \in B)\cdot \frac{f(x_1, \dots, x_{d-1})}{\lambda} \dd{x_1} \dots \dd{x_{d-1}}             \\
	                               & = \frac{1}{\lambda} \idotsint 1((X_1, \dots, X_{d-1}) \in B)\cdot f(x_1, \dots, x_{d-1}) \dd{x_1} \dots \dd{x_{d-1}}           \\
	                               & = \frac{1}{\lambda} \int_B f(x) \dd{x}                                                                                         \\
	\abs{A}                        & = \frac{1}{\lambda} \int_{[0,1]^{d-1}} f(x) \dd{x}                                                                             \\
	                               & = \frac{1}{\lambda}                                                                                                            \\
	\therefore \prob{X \in B}      & = \int_B f(x) \dd{x}
\end{align*}
