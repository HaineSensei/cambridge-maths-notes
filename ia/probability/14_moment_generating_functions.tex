\subsection{Moment generating functions}
Consider a continuous random variable \(X\) with density \(f\).
Then the moment generating function of \(X\) is defined as
\[
	m(\theta) = \expect{e^{\theta X}} = \int_{-\infty}^\infty e^{\theta x} f(x) \dd{x}
\]
whenever this integral is finite.
Note that \(m(0) = 1\).
\begin{theorem}
	The moment generating function uniquely determines the distribution of a continuous random variable, provided that it is defined on some open interval \((a, b)\) of values of \(\theta\).
\end{theorem}
\noindent No proof will be given.
\begin{theorem}
	Suppose the moment generating function is defined on an open interval of values of \(\theta\).
	Then
	\[
		\eval{\dv[r]{\theta}m(\theta)}_{\theta = 0} = \expect{X^r}
	\]
\end{theorem}
\begin{theorem}
	Suppose \(X_1, \dots, X_n\) are independent random variables.
	Then
	\[
		m(\theta) = \expect{e^{\theta(X_1 + \dots + X_n)}} = \prod_{i=1}^n \expect{e^{\theta X_i}}
	\]
\end{theorem}
\begin{proof}
	Since the \(X_i\) are independent, we can move the product outside of the expectation.
\end{proof}

\subsection{Gamma distribution}
Let \(X\) be a random variable with density
\[
	f(x) = e^{-\lambda x}\frac{\lambda^n x^{n-1}}{(n-1)!}
\]
where \(\lambda > 0\), \(n \in \mathbb N\), \(x \geq 0\).
We can say that \(X \sim \Gamma(n, \lambda)\).
First, we check that \(f\) is indeed a density.
\begin{align*}
	I_n & = \int_0^\infty f(x) \dd{x}                                                     \\
	    & = \int_0^\infty \lambda e^{-\lambda x} \frac{\lambda^n x^{n-1}}{(n-1)!} \dd{x}  \\
	    & = \int_0^\infty \frac{e^{-\lambda x} \lambda^{n-1} (n-1) x^{n-2}}{(n-1)!}\dd{x} \\
	    & = \int_0^\infty \frac{e^{-\lambda x} \lambda^{n-1} x^{n-2}}{(n-2)!}\dd{x}       \\
	    & = I_{n-1} = \dots = I_1
\end{align*}
Note that for \(n=1\), \(f(x) = \lambda e^{-\lambda x}\) which is the density of the exponential distribution.
Therefore, \(I_n = 1\) as required, so \(f\) really is a density.
Now,
\[
	m(\theta) = \int_0^\infty \frac{e^{\theta x} e^{-\lambda x} \lambda^n x^{n-1}}{(n-1)!} \dd{x}
\]
If \(\lambda > \theta\), then we have a finite integral.
If \(\lambda \leq \theta\), then the exponential term \(e^{\theta x}\) will dominate and we will have an infinite integral.
So, let \(\lambda > \theta\).
\begin{align*}
	m(\theta) & = \int_0^\infty \frac{e^{\theta x} e^{-\lambda x} \lambda^n x^{n-1}}{(n-1)!} \dd{x}                                                            \\
	          & = \left( \frac{\lambda}{\lambda - \theta} \right)^n \int_0^\infty \frac{e^{-(\lambda - \theta) x} (\lambda - \theta)^n x^{n-1}}{(n-1)!} \dd{x}
\end{align*}
The integral on the right hand side is the probability distribution function of a random variable \(Y \sim \Gamma(n, \lambda - \theta)\), which gives 1 since the integral is taken over the entire domain.
Hence,
\[
	m(\theta) = \left( \frac{\lambda}{\lambda - \theta} \right)^n
\]
Now, let \(X \sim \Gamma(n, \lambda)\) and \(Y \sim \Gamma(m, \lambda)\) be independent continuous random variables.
Then
\[
	m(\theta) = \expect{e^{\theta(X + Y)}} = \expect{e^{\theta X}} \expect{e^{\theta Y}} = \qty(\frac{\lambda}{\lambda - \theta})^{n+m}
\]
So by the uniqueness property we saw earlier, we get that \(X+Y \sim \Gamma(n+m, \lambda)\).
In particular, this implies that if \(X_1, \dots, X_n\) are independent and identically distributed with the distribution \(\mathrm{Exp}(\lambda) = \Gamma(1, \lambda)\), then
\[
	X_1 + \dots + X_n \sim \Gamma(n, \lambda)
\]
We could alternatively consider \(\Gamma(\alpha, \lambda)\) for \(\alpha > 0\) by replacing \((n-1)!
\) with
\[
	\Gamma(\alpha) = \int_0^\infty e^{-x} x^{\alpha - 1} \dd{x}
\]
which agrees with this factorial function for integer values of \(\alpha\).

\subsection{Moment generating function of the normal distribution}
Recall that
\[
	f(x) = \frac{1}{\sqrt{2\pi \sigma^2}} \exp(-\frac{(x-\mu)^2}{2\sigma^2})
\]
Now,
\[
	m(\theta) = \int_0^\infty e^{\theta x} \frac{1}{\sqrt{2\pi \sigma^2}} \exp(-\frac{(x-\mu)^2}{2\sigma^2}) \dd{x} = \int_0^\infty \frac{1}{\sqrt{2\pi \sigma^2}} \exp(\theta x-\frac{(x-\mu)^2}{2\sigma^2}) \dd{x}
\]
Note that
\[
	\theta x - \frac{(x-\mu)^2}{2\sigma^2} = \theta \mu + \frac{\theta^2 \sigma^2}{2} - \frac{\left( x - (\mu + \theta\sigma^2) \right)^2}{2\sigma^2}
\]
Hence,
\begin{align*}
	m(\theta) & = \int_0^\infty \frac{1}{\sqrt{2\pi \sigma^2}} \exp(\theta \mu + \frac{\theta^2 \sigma^2}{2} - \frac{\left( x - (\mu + \theta\sigma^2) \right)^2}{2\sigma^2}) \dd{x}       \\
	          & = \exp(\theta \mu + \frac{\theta^2 \sigma^2}{2}) \int_0^\infty \frac{1}{\sqrt{2\pi \sigma^2}} \exp(- \frac{\left( x - (\mu + \theta\sigma^2) \right)^2}{2\sigma^2}) \dd{x}
\end{align*}
Note that the integral on the right hand side has the form of the probability distribution function of a variable \(Y \sim \mathrm{N}(\mu + \theta \sigma^2, \sigma^2)\), hence it integrates to 1.
\[
	m(\theta) = \exp(\theta \mu + \frac{\theta^2 \sigma^2}{2})
\]
Recall that if \(X \sim \mathrm{N}(\mu, \sigma^2)\), then \(aX + b \sim \mathrm{N}(a\mu + b, a^2\sigma^2)\).
We can then deduce that
\[
	\expect{e^{\theta(aX + b)}} = \exp(\theta(a \mu + b) + \frac{\theta^2 a^2\sigma^2}{2})
\]
Now, suppose that \(X \sim \mathrm{N}(\mu, \sigma^2)\) and \(Y \sim \mathrm{N}(\nu, \tau^2)\) are independent.
Then
\[
	\expect{e^{\theta(X + Y)}} = \expect{e^{\theta X}} \expect{e^{\theta Y}} = \exp(\theta \mu + \frac{\theta^2 \sigma^2}{2}) \exp(\theta \nu + \frac{\theta^2 \tau^2}{2}) = \exp(\theta(\mu + \nu) + \frac{\theta^2 (\sigma^2 + \tau^2)}{2})
\]
Hence \(X + Y \sim \mathrm{N}(\mu + \nu, \sigma^2 + \tau^2)\).

\subsection{Cauchy distribution}
Suppose that a continuous random variable \(X\) has density
\[
	f(x) = \frac{1}{\pi(1 + x^2)}
\]
where \(x \in \mathbb R\).
Now,
\[
	m(\theta) = \expect{e^{\theta X}} = \int_{-\infty}^\infty \frac{e^{\theta x}}{\pi(1 + x^2)} = \begin{cases}
		\infty & \theta \neq 0 \\
		1      & \theta = 0
	\end{cases}
\]
Suppose \(X \sim f\).
Then \(X, 2X, 3X, \dots\) have the same moment generating function, but they do not have the same distribution.
This is because \(m(\theta)\) is not finite on an open interval.

\subsection{Multivariate moment generating functions}
Let \(X = (X_1, \dots, X_n)\) be a random variable with values in \(\mathbb R^n\).
Then the moment generating function of \(X\) is defined as
\[
	m(\theta) = \expect{e^{\theta^\transpose X}} = \expect{e^{\theta_1 X_1 + \dots + \theta_n X_n}};\quad \theta = \begin{pmatrix}
		\theta_1 \\ \vdots \\ \theta_n
	\end{pmatrix}
\]
\begin{theorem}
	If the moment generating function is finite for a range of values of \(\theta\), it uniquely determines the distribution of \(X\).
	Also,
	\[
		\eval{\pdv[r]{m}{\theta_i}}_{\theta = 0} = \expect{X_i^r}
	\]
	and
	\[
		\eval{\frac{\partial^{r+s}m}{\partial \theta_i^r \partial \theta_j^s}}_{\theta = 0} = \expect{X_i^r X_j^s}
	\]
	Further,
	\[
		m(\theta) = \prod_{i = 1}^n \expect{e^{\theta_i X_i}}
	\]
	if and only if \(X_1, \dots, X_n\) are independent.
\end{theorem}
\noindent No proof is provided.
