\subsection{Probability Spaces and \(\sigma\)-algebras}
\begin{definition}
	Suppose \(\Omega\) is a set, and \(\mathcal F\) is a collection of subsets of \(\Omega\).
	We call \(\mathcal F\) a \(\sigma\)-algebra if
	\begin{enumerate}
		\item \(\Omega \in \mathcal F\)
		\item if \(A \in \mathcal F\), then \(\stcomp{A} \in \mathcal F\)
		\item for any countable collection \((A_n)_{n \geq 1}\) with \(A_n \in \mathcal F\) for all \(n\), we must also have that \(\bigcup_n A_n \in \mathcal F\)
	\end{enumerate}
\end{definition}
\begin{definition}
	Suppose that \(\mathcal F\) is a \(\sigma\)-algebra on \(\Omega\).
	A function \(\mathbb P \colon \mathcal F \to [0, 1]\) is called a probability measure if
	\begin{enumerate}
		\item \(\prob{\Omega} = 1\)
		\item for any countable disjoint collection of sets \((A_n)_{n \geq 1}\) in \(\mathcal F\) (\(A_n \in \mathcal F\) for all \(n\)), then \(\prob{\bigcup_{n \geq 1}A_n} = \sum_{n \geq 1} \prob{A_n}\) (this is called `countable additivity')
	\end{enumerate}
	We say that \(\prob{A}\) is `the probability of \(A\)'.
	We call \((\Omega, \mathcal F, \mathbb P)\) a probability space, where \(\Omega\) is the sample space, \(\mathcal F\) is the \(\sigma\)-algebra, and \(\mathbb P\) is the probability measure.
\end{definition}

\begin{remark}
	When \(\Omega\) is countable, we take \(\mathcal F\) to be all subsets of \(\Omega\), i.e.\ \(\mathcal F = \mathcal P(\Omega)\), its power set.
\end{remark}
\begin{definition}
	The elements of \(\Omega\) are called outcomes, and the elements of \(\mathcal F\) are called events.
\end{definition}
Note that \(\mathbb P\) is dependent on \(\mathcal F\) but not on \(\Omega\).
We talk about probabilities of \textit{events}, not probabilities of \textit{outcomes}.
For example, if you pick a uniform number from the interval \([0, 1]\); then the probability of getting any specific outcome is zero --- but we can define useful events that have non-zero probabilities.

\subsection{Properties of the Probability Measure}
\begin{itemize}
	\item \(\prob{\stcomp{A}} = 1 - \prob{A}\), since \(A\) and \(\stcomp{A}\) are disjoint sets, whose union is \(\Omega\)
	\item \(\prob{\varnothing} = 0\), since it is the complement of \(\Omega\)
	\item if \(A \subseteq B\), then \(\prob{A} \leq \prob{B}\)
	\item \(\prob{A \cup B} = \prob{A} + \prob{B} - \prob{A \cap B}\) using the Inclusion-Exclusion theorem
\end{itemize}

\subsection{Examples of Probability Spaces}
\begin{itemize}
	\item Rolling a fair 6-sided die:
	      \begin{itemize}
		      \item \(\Omega = \{ 1, 2, 3, 4, 5, 6 \}\)
		      \item \(\mathcal F = \mathcal P(\Omega)\)
		      \item \(\forall \omega \in \Omega, \prob{\{ \omega \}} = \frac{1}{6}\), and if \(A \subseteq \Omega\) then \(\prob{A} = \frac{\abs{A}}{6}\)
	      \end{itemize}
	      
	\item Equally likely outcomes (more generally).
	      Suppose \(\Omega\) is some finite set, e.g.
	      \(\Omega = \{ \omega_1, \omega_2, \dots, \omega_n \}\).
	      Then we define \(\prob{A} = \frac{\abs{A}}{\abs{\Omega}}\).
	      In classical probability, this models picking a random element of \(\Omega\).
	      
	\item Picking balls from a bag.
	      Suppose we have \(n\) balls with \(n\) labels from the set \(\{1, \dots, n\}\), indistinguishable by touching.
	      Let us pick \(k \leq n\) balls at random from the bag, \textit{without replacement}.
	      Here, `at random' just means that all possible outcomes are equally likely, and their probability measures should be equal.
	      
	      We will take \(\Omega = \{ A \subseteq \{1, \dots, n\} : \abs{A} = k \}\).
	      Then \(\abs{\Omega} = \binom{n}{k}\).
	      Then of course \(\prob{\{ \omega \}} = \frac{1}{\abs*{\Omega}}\), since all outcomes (combinations, in this case) are equally likely.
	      
	\item Consider a well-shuffled deck of 52 cards, i.e.\ it is equally likely that each possible permutation of the 52 cards will appear.
	      \(\Omega = \{ \text{all permutations of 52 cards} \}\), and \(\abs*{\Omega} = 52!
	      \)
	      
	      The probability that the top two cards are aces is therefore \(\frac{4 \times 3 \times 50!}{52!} = \frac{1}{221}\), since there are \(4 \times 3 \times 50!
	      \) outcomes that produce such an event.
	      
	\item Consider a string of \(n\) random digits from \(\{0, \dots, 9\}\).
	      Then \(\Omega = \{ 0, \dots, 9 \}^n\), and \(\abs*{\Omega} = 10^n\).
	      We define \(A_k = \{ \text{no digit exceeds } k \}\), and \(B_k = \{ \text{largest digit is } k \}\).
	      Then \(\prob{B_k} = \frac{\abs*{B_k}}{\abs*{\Omega}}\).
	      Notice that \(B_k = A_k \setminus A_{k-1}\).
	      \(\abs*{A_k} = (k+1)^n\), so \(\abs*{B_k} = (k+1)^n - k^n\), so \(\prob{B_k} = \frac{(k+1)^n - k^n}{10^n}\).
	      
	\item The birthday problem.
	      There are \(n\) people; what is the probability that at least two of them share a birthday?
	      We assume that each year has exactly 365 days, i.e.\ nobody is born on 29\textsuperscript{th} of February, and that the probabilities of being born on any given day are equal.
	      
	      Let \(\Omega = \{1, \dots, 365\}^n\), and \(\mathcal F = \mathcal P(\Omega)\).
	      Since all outcomes are equally likely, we take \(\prob{\{\omega\}} = \frac{1}{365^n}\).
	      
	      Let \(A = \{ \text{at least two people share the same birthday} \}\).
	      \(\stcomp{A} = \{ \text{all } n \text{ birthdays are different} \}\).
	      Since \(\prob{A} = 1 - \prob{\stcomp{A}}\), it suffices to calculate \(\prob{\stcomp{A}}\), which is \(\frac{\abs*{\stcomp{A}}}{\abs*{\Omega}} = \frac{365!}{(365 - n)!365^n}\).
	      So the answer is \(\prob{A} = 1 - \frac{365!}{(365 - n)!365^n}\)
	      
	      Note that at \(n=22\), \(\prob{A} \approx 0.476\) and at \(n=23\), \(\prob{A} \approx 0.507\).
	      So when there are at least 23 people in a room, the probability that two of them share a birthday is around 50\%.
\end{itemize}

\subsection{Combinatorial Analysis}
Let \(\Omega\) be a finite set, and suppose \(\abs*{\Omega} = n\).
We want to partition \(\Omega\) into \(k\) disjoint subsets \(\Omega_1, \dots, \Omega_k\) with \(\abs*{\Omega_i} = n_i\) and \(\sum_{i=1}^k n_i = n\).
How many ways of doing such a partition are there?
The result is
\[
	\underbrace{\binom{n}{n_1}}_{\text{choose first set}}\underbrace{\binom{n-n_1}{n_2}}_{\text{choose second set}}\dots\underbrace{\binom{n-(n_1 + n_2 + \dots + n_{k-1})}{n_k}}_{\text{choose last set}} = \frac{n!}{n_1!n_2!\dots n_k!}
\]
So we will write
\[
	\binom{n}{n_1, \dots, n_k} = \frac{n!}{n_1!n_2!\dots n_k!}
\]

Now, let \(f\colon \{1, \dots, k\} \to \{1, \dots, n\}\).
\(f\) is strictly increasing if \(x < y \implies f(x) < f(y)\).
\(f\) is increasing if \(x < y \implies f(x) \leq f(y)\).
How many strictly increasing functions \(f\) exist?
Note that if we know the range of \(f\), the function is completely determined.
The range is a subset of \(\{1, \dots, n\}\) of size \(k\), i.e.\ a \(k\)-subset of an \(n\)-set, which yields \(\binom{n}{k}\) choices, and thus there are \(\binom{n}{k}\) strictly increasing functions.

How many increasing functions \(f\) exist?
Let us define a bijection from the set of increasing functions \(\{f\colon \{1, \dots, k\} \to \{1, \dots, n\}\}\) to the set of \textit{strictly} increasing functions \(\{g\colon \{1, \dots, k\} \to \{1, \dots, n+k-1\}\}\).
For any increasing function \(f\), we define \(g(i) = f(i) + i - 1\).
Then \(g\) is clearly strictly increasing, and takes values in the range \(\{1, \dots, n+k-1\}\).
By extension, we can define an increasing function \(f\) from any strictly increasing function \(g\).
So the total number of increasing functions \(f\colon \{1, \dots, k\} \to \{1, \dots, n\}\) is \(\binom{n+k-1}{k}\).
