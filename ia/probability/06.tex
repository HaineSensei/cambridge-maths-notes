\subsection{Discrete Distributions}
In a discrete probability distribution on a probability space \((\Omega, \mathcal F, \mathbb P)\), \(\Omega\) is either finite or countable, i.e.\ \(\Omega = \{ \omega_1, \omega_2, \dots \}\), and as stated before, \(\mathcal F\) is the power set of \(\Omega\).
If we know \(\prob{\{\omega_i\}}\), then this completely determines \(\mathbb P\).
Indeed, let \(A \subseteq \Omega\), then
\[
	\prob{A} = \prob{\bigcup_{i \colon \omega_i \in A} \{ \omega_i \}} = \sum_{i \colon \omega_i \in A}\prob{\{\omega_i\}}
\]
by countable additivity.
We will see later that this is not true if \(\Omega\) is uncountable.
We write \(p_i = \prob{\{\omega_i\}}\), and we then call this a discrete probability distribution.
It has the following key properties:
\begin{itemize}
	\item \(p_i \geq 0\)
	\item \(\sum_i p_i = 1\)
\end{itemize}

\subsection{Bernoulli Distribution}
We model the outcome of a test with two outcomes (e.g.
the toss of a coin) with the Bernoulli distribution.
Let \(\Omega = \{ 0, 1 \}\).
We will denote \(p = p_1\), then clearly \(p_0 = 1 - p\).

\subsection{Binomial Distribution}
The binomial distribution \(B\) has parameters \(N \in \mathbb Z^+, p \in [0, 1]\).
This distribution models a sequence of \(N\) independent Bernoulli distributions of parameter \(p\).
We then count the amount of `successes', i.e.\ trials in which the result was 1.
\(\Omega = \{ 0, 1, \dots, N \}\).
\[
	\prob{\{ k \}} = p_k = \binom{N}{k}p^k(1-p)^{N-k}
\]

\subsection{Multinomial Distribution}
The multinomial distribution is a generalisation of the binomial distribution.
\(M\) has parameters \(N \in \mathbb Z^+\) and \(p_1, p_2, \dots \in [0, 1]\) where \(\sum_{i=1}^k p_i = 1\).
This models a sequence of \(N\) independent trials in which a number from 1 to \(N\) is selected, where the probability of selecting \(i\) is \(p_i\).
\(\Omega = \{ (n_1, \dots, n_k) \in \mathbb N^k \colon \sum_{i=1}^k n_i = N \}\), in other words, ordered partitions of \(N\).
Therefore
\[
	\prob{n_1 \text{ outcomes had value 1}, \dots, n_k \text{ outcomes had value }k} = \prob{(n_1, \dots, n_k)} = \binom{N}{n_1,\dots,n_k}p_1^{n_1}\dots p_k^{n_k}
\]

\subsection{Geometric Distribution}
Consider a Bernoulli distribution of parameter \(p\).
The geometric distribution models running this trial many times independently until the first `success' (i.e.\ the first result of value 1).
Then \(\Omega = \{ 1, 2, \dots \} = \mathbb Z^+\).
Then
\[
	p_k = (1-p)^{k-1}p
\]
We can compute the infinite geometric series \(\sum p_k\) which gives 1.
We could alternatively model the distribution using \(\Omega' = \{ 0, 1, \dots \} = \mathbb N\) which records the amount of failures before the first success.
Then
\[
	p_k' = (1-p)^k p
\]
Again, the sum converges to 1.

\subsection{Poisson Distribution}
This is used to model the number of occurences of an event in a given interval of time.
\(\Omega = \{ 0, 1, 2, \dots \} = \mathbb N\).
This distribution has one parameter \(\lambda \in \mathbb R\).
We have
\[
	p_k = e^{-\lambda} \frac{\lambda^k}{k!}
\]
Then
\[
	\sum_{k=0}^\infty p_k = e^{-\lambda}  \sum_{k=0}^\infty \frac{\lambda^k}{k!} = e^{-\lambda} \cdot e^{\lambda} = 1
\]
Suppose customers arrive into a shop during the time interval \([0, 1]\).
We will subdivide \([0, 1]\) into \(N\) intervals \(\left[ \frac{i-1}{N}, \frac{i}{N} \right]\).
In each interval, a single customer arrives with probability \(p\), independent of other time intervals.
In this example,
\[
	\prob{k \text{ customers arrive}} = \binom{N}{k} p^k (1-p)^{N-k}
\]
Let \(p = \frac{\lambda}{N}\) for \(\lambda > 0\).
We will show that as \(N \to \infty\), this binomial distribution converges to the Poisson distribution.
\begin{align*}
	\binom{N}{k} p^k (1-p)^{N-k} & = \frac{N!}{k!(N-k)!} \left( \frac{\lambda}{n} \right)^k \cdot \left( 1 - \frac{\lambda}{n} \right)^{N-k} \\
	                             & = \frac{\lambda_k}{k!} \cdot \frac{N!}{N^k(N-k)!} \cdot \left( 1 - \frac{\lambda}{N} \right)^{N-k}        \\
	                             & \to \frac{\lambda_k}{k!} \cdot 1 \cdot e^{-\lambda}
\end{align*}
which matches the Poisson distribution.

\subsection{Random Variables}
\begin{definition}
	Consider the probability space \((\Omega, \mathcal F, \mathbb P)\).
	A random variable \(X\) is a function \(X \colon \Omega \to \mathbb R\) satisfying
	\[
		\{ \omega \in \Omega \colon X(\omega) \leq x \} \in \mathcal F
	\]
	for any given \(x\).
\end{definition}
\noindent Suppose \(A \subseteq \mathbb R\).
Then typically we write
\[
	\{ X \in A \} = \{ \omega \colon X(\omega) \in A \}
\]
as shorthand.
Given \(A \in \mathcal F\), we define the indicator of \(A\) to be
\[
	1_A(\omega) = 1(\omega \in A) = \begin{cases}
		1 & \text{if } \omega \in A \\
		0 & \text{otherwise}
	\end{cases}
\]
Because \(A \in \mathbb F\), \(1_A\) is a random variable.
Suppose \(X\) is a random variable.
We define probability distribution function of \(X\) to be
\[
	F_X \colon \mathbb R \to [0, 1];\quad F_X(x) = \prob{X \leq x}
\]
\begin{definition}
	\((X_1, \dots, X_n)\) is called a random variable in \(\mathbb R^n\) if \((X_1, \dots, X_n) \colon \Omega \to \mathbb R^n\), and for all \(x_1, \dots, x_n \in \mathbb R\) we have
	\[
		\{ X_1 \leq x_1, \dots, X_n \leq x_n \} = \{ \omega \colon X_1(\omega) \leq x_1, \dots, X_n(\omega) \leq x_n \} \in \mathcal F
	\]
\end{definition}
\noindent This definition is equivalent to saying that \(X_1, \dots, X_n\) are all random variables in \(\mathbb R\).
Indeed,
\[
	\{ X_1 \leq x_1, \dots, X_n \leq x_n \} = \{ X_1 \leq x_1 \} \cap \dots \cap \{ X_n \leq x_n \}
\]
which, since \(\mathcal F\) is a \(\sigma\)-algebra, is an element of \(\mathcal F\).
