\chapter[Probability \\ \textnormal{\emph{Lectured in Lent \oldstylenums{2021} by \textsc{Dr.\ P.\ Sousi}}}]{Probability}
\emph{\Large Lectured in Lent \oldstylenums{2021} by \textsc{Dr.\ P.\ Sousi}}

In this course, we establish the rules of probability spaces, which are the mathematical framework for dealing with randomness.
Potential states of a mathematical system are called outcomes, and we look at particular sets of outcomes called events.
For instance, rolling two six-sided dice produces 36 outcomes, and we might be interested in the event `rolled a double'.
Each event can be assigned a probability of occurring; in this case, one sixth.
By carefully reasoning about probabilities of events using the rules of probability spaces, we can avoid many apparent paradoxes of probability, such as Simpson's paradox.

When there are many different possible outcomes (or even infinitely many), it becomes helpful to think of certain events as tied to random variables.
For example, the amount of coin flips needed before getting a head is a random variable, and its value could be any integer at least 1.
The statement `at least three coin flips were needed' is an example of an event linked to this random variable.
The values that a random variable can be, as well as the probabilities that they occur, form the distribution of the random variable.
We study many different examples of distributions and their properties to gain a better understanding of random variables.

\subfile{../../ia/probability/main.tex}
