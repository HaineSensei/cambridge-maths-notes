\subsection{Random variables}
\begin{definition}
	Consider the probability space \((\Omega, \mathcal F, \mathbb P)\).
	A random variable \(X\) is a function \(X \colon \Omega \to \mathbb R\) satisfying
	\[
		\{ \omega \in \Omega \colon X(\omega) \leq x \} \in \mathcal F
	\]
	for any given \(x\).
\end{definition}
Suppose \(A \subseteq \mathbb R\).
Then typically we write
\[
	\{ X \in A \} = \{ \omega \colon X(\omega) \in A \}
\]
as shorthand.
Given \(A \in \mathcal F\), we define the indicator of \(A\) to be
\[
	1_A(\omega) = 1(\omega \in A) = \begin{cases}
		1 & \text{if } \omega \in A \\
		0 & \text{otherwise}
	\end{cases}
\]
Because \(A \in \mathbb F\), \(1_A\) is a random variable.
Suppose \(X\) is a random variable.
We define probability distribution function of \(X\) to be
\[
	F_X \colon \mathbb R \to [0, 1];\quad F_X(x) = \prob{X \leq x}
\]
\begin{definition}
	\((X_1, \dots, X_n)\) is called a random variable in \(\mathbb R^n\) if \((X_1, \dots, X_n) \colon \Omega \to \mathbb R^n\), and for all \(x_1, \dots, x_n \in \mathbb R\) we have
	\[
		\{ X_1 \leq x_1, \dots, X_n \leq x_n \} = \{ \omega \colon X_1(\omega) \leq x_1, \dots, X_n(\omega) \leq x_n \} \in \mathcal F
	\]
\end{definition}
This definition is equivalent to saying that \(X_1, \dots, X_n\) are all random variables in \(\mathbb R\).
Indeed,
\[
	\{ X_1 \leq x_1, \dots, X_n \leq x_n \} = \{ X_1 \leq x_1 \} \cap \dots \cap \{ X_n \leq x_n \}
\]
which, since \(\mathcal F\) is a \(\sigma\)-algebra, is an element of \(\mathcal F\).

\begin{definition}
	A random variable \(X\) is called discrete if it takes values in a countable set.
	Suppose \(X\) takes values in the countable set \(S\).
	For every \(x \in S\), we write
	\[
		p_x = \prob{X = x} = \prob{\{ \omega \colon X(\omega) = x \}}
	\]
	We call \((p_x)_{x \in S}\) the probability mass function of \(X\), or the distribution of \(X\).
	If \((p_x)\) is Bernoulli for example, then we say that \(X\) is a Bernoulli (or such) random variable, or that \(X\) has the Bernoulli distribution.
\end{definition}
\begin{definition}
	Suppose \(X_1, \dots, X_n\) are discrete random variables taking variables in \(S_1, \dots, S_n\).
	We say that the random variables \(X_1, \dots, X_n\) are independent if
	\[
		\prob{X_1 = x_1, \dots, X_n = x_n} = \prob{X_1 = x_1} \cdots \prob{X_n = x_n}\quad \forall x_1 \in S_1, \dots, x_n \in S_n
	\]
\end{definition}
As an example, suppose we toss a \(p\)-biased coin \(n\) times independently.
Let \(\Omega = \{ 0, 1 \}^n\).
For every \(\omega \in \Omega\),
\[
	p_\omega = \prod_{k=1}^n p^{\omega_k} (1-p)^{1-\omega_k};\quad \text{where we write } \omega = (\omega_1, \dots, \omega_n)
\]
We define a set of discrete random variables \(X_k(\omega) = \omega_k\).
Then \(X_k\) gives the output of the \(k\)th toss.
We have
\[
	\prob{X_k = 1} = \prob{\omega_k = 1} = p;\quad \prob{X_k = 0} = \prob{\omega_k = 0} = 1-p
\]
So \(X_k\) has the Bernoulli distribution with parameter \(p\).
We can also show that the \(X_i\) are independent.
Let \(x_1, \dots, x_n \in \{ 0, 1 \}\).
Then
\begin{align*}
	\prob{X_1 = x_1, \dots, X_n = x_n} & = \prob{\omega = (x_1, \dots, x_n)}   \\
	                                   & = p_{(x_1, \dots, x_n)}               \\
	                                   & = \prod_{k=1}^N p^{x_k} (1-p)^{1-x_k} \\
	                                   & = \prod_{k=1}^N \prob{X_k = x_k}
\end{align*}
as required.
Now, we define \(S_n(\omega) = X_1(\omega) + \dots + X_n(\omega)\).
This is the number of heads in \(N\) tosses.
So \(S_n \colon \Omega \to \{ 0, \dots, N \}\), and
\[
	\prob{S_n = k} = \binom{n}{k} p^k (1-p)^{n-k}
\]
So \(S_n\) has the binomial distribution with parameters \(n\) and \(p\).

\subsection{Expectation}
Let \((\Omega, \mathcal F, \mathbb P)\) be a probability space such that \(\Omega\) is countable.
Let \(X \colon \Omega \to \mathbb R\) be a random variable, which is necessarily discrete.
We say that \(X\) is non-negative if \(X \geq 0\).
We define the expectation of \(X\) to be
\[
	\expect{X} = \sum_\omega X(\omega) \cdot \prob{\{ \omega \}}
\]
We will write
\[
	\Omega_X = \{ X(\omega) \colon \omega \in \Omega \}
\]
So
\[
	\Omega = \bigcup_{x \in \Omega_X} \{ X = x \}
\]
So we have partitioned \(\Omega\) using \(X\).
Note that
\[
	\expect{X} = \sum_\omega X(\omega) \prob{\{\omega\}} = \sum_{x \in \Omega_X} \sum_{\omega \in \{ X = x\}} X(\omega) \prob{\{\omega\}} = \sum_{x \in \Omega_X} \sum_{\omega \in \{ X = x\}} x \prob{\{\omega\}} = \sum_{x \in \Omega_X} x\prob{\{X = x \}}
\]
which matches the more familiar definition of the expectation; the average of the values taken by \(X\), weighted by the probability of the event occcuring.
So
\[
	\expect{X} = \sum_{x \in \Omega_X} x p_x
\]

\subsection{Expectation of binomial distribution}
Let \(X \sim \mathrm{Bin}(N, p)\).
Then
\[
	\forall k = 0, \dots, N,\quad \prob{X = k} = \binom{N}{k} p^k (1-p)^{N-k}
\]
So using the second definition,
\begin{align*}
	\expect{X} & = \sum_{k=0}^N k \prob{X = k}                            \\
	           & = \sum_{k=0}^N k \binom{n}{k} p^k (1-p)^{N-k}            \\
	           & = \sum_{k=0}^N \frac{k \cdot N!}{k!
	\cdot (N-k)!} p^k (1-p)^{N-k}                                         \\
	           & = \sum_{k=1}^N \frac{(N-1)!
		\cdot N \cdot p}{(k-1)!
	\cdot (N-k)!} p^{k-1} (1-p)^{N-k}                                     \\
	           & = Np \sum_{k=1}^N \binom{N-1}{k-1} p^{k-1} (1-p)^{N-k}   \\
	           & = Np \sum_{k=0}^{N-1} \binom{N-1}{k} p^{k} (1-p)^{N-1-k} \\
	           & = Np (p + 1 - p)^{N-1}                                   \\
	           & = Np
\end{align*}

\subsection{Expectation of poisson distribution}
Let \(X \sim \text{Poi}(\lambda)\), so
\[
	\prob{X = k} = e^{-\lambda} \frac{\lambda^k}{k!}
\]
Hence
\begin{align*}
	\expect{X} & = \sum_{k=0}^\infty k e^{-\lambda} \frac{\lambda^k}{k!}              \\
	           & =\sum_{k=1}^\infty e^{-\lambda} \frac{\lambda^{k-1} \lambda}{(k-1)!} \\
	           & = e^{-\lambda} \cdot e^{\lambda} \cdot \lambda                       \\
	           & = \lambda
\end{align*}

\subsection{Expectation of a general random variable}
Let \(X\) be a general (not necessarily non-negative) discrete random variable.
Then we define
\[
	X^+ = \max(X, 0);\quad X^- = \max(-X, 0)
\]
Then \(X = X^+ - X^-\).
Note that \(X^+\) and \(X^-\) are non-negative random variables, which has a well-defined expectation.
So if at least one of \(\expect{X^+}, \expect{X^-}\) is finite, we define
\[
	\expect{X} = \expect{X^+} - \expect{X^-}
\]
If both are infinite, then we say that the expectation of \(X\) is not defined.
Whenever we write \(\expect{X}\), it is assumed to be well-defined.
If \(\expect{\abs{X}} < \infty\), we say that \(X\) is integrable.
When \(\expect{X}\) is well-defined, we have again that
\[
	\expect{X} = \sum_{x \in \Omega_x} x \cdot \prob{X = x}
\]

\subsection{Properties of the expectation}
The following properties follow immediately from the definition.
\begin{enumerate}
	\item If \(X \geq 0\), then \(\expect{X} \geq 0\).
	\item If \(X \geq 0\) and \(\expect{X} = 0\), then \(\prob{X = 0} = 1\).
	\item If \(c \in \mathbb R\), then \(\expect{cX} = c\expect{X}\), and \(\expect{c + X} = c + \expect{X}\).
	\item If \(X\), \(Y\) are two integrable random variables, then \(\expect{X + Y} = \expect{X} + \expect{Y}\).
	\item More generally, let \(c_1, \dots, c_n \in \mathbb R\) and \(X_1, \dots, X_n\) integrable random variables.
	      Then
	      \[
		      \expect{c_1X_1 + \dots + c_n X_n} = c_1 \expect{X_1} + \dots + c_n \expect{X_n}
	      \]
	      So the expectation is a linear operator over finitely many inputs.
\end{enumerate}

\subsection{Countable additivity for the expectation}
Suppose \(X_1, X_2, \dots\) are non-negative random variables.
Then
\[
	\expect{\sum_n X_n} = \sum_n \expect{X_n}
\]
The non-negativity constraint allows us to guarantee that the sums are well-defined; they could be infinite, but at least their values are well-defined.
We will construct a proof assuming that \(\Omega\) is countable, however the result holds regardless of the choice of \(\Omega\).
\begin{proof}
	\begin{align*}
		\expect{\sum_n X_n} & = \sum_\omega \sum_n X_n(\omega) \prob{\{ \omega \}} \\
		                    & = \sum_n \sum_\omega X_n(\omega) \prob{\{ \omega \}} \\
		                    & = \sum_n \expect{X_n}
	\end{align*}
\end{proof}
We are allowed to rearrange the sums since all relevant terms are non-negative.

\subsection{Expectation of indicator function}
If \(X = 1(A)\) where \(A \in \mathcal F\), then \(\expect{X} = \prob{A}\).
This is obvious from the second definition of the expectation.

\subsection{Expectation under function application}
If \(g \colon \mathbb R \to \mathbb R\), we can define \(g(X)\) to be the random variable given by
\[
	g(X)(\omega) = g(X(\omega))
\]
Then
\[
	\expect{g(X)} = \sum_{x \in \Omega_X} g(x) \cdot \prob{X = x}
\]
\begin{proof}
	Let \(Y = g(X)\).
	Then
	\[
		\expect{Y} = \sum_{y \in \Omega_Y} y \cdot \prob{Y = y}
	\]
	Note that
	\begin{align*}
		\{ Y = y \} & = \{ \omega \colon Y(\omega) = y \}              \\
		            & = \{ \omega \colon g(X(\omega)) = y \}           \\
		            & = \{ \omega \colon X(\omega) \in g^{-1}(\{y\})\} \\
		            & = \{ X \in g^{-1} (\{ y \}) \}
	\end{align*}
	where \(g^{-1}(\{ y \})\) is the set of all \(x\) such that \(g(x) \in \{ y \}\).
	So
	\begin{align*}
		\expect{Y} & = \sum_{y \in \Omega_y} y \cdot \prob{X \in g^{-1}(\{ y \})}              \\
		           & = \sum_{y \in \Omega_Y} y \cdot \sum_{x \in g^{-1}(\{ y \})} \prob{X = x} \\
		           & = \sum_{y \in \Omega_Y} \sum_{x \in g^{-1}(\{y\})} g(x) \prob{X = x}      \\
		           & = \sum_{x \in \Omega_X} g(x) \prob{X = x}
	\end{align*}
\end{proof}

\subsection{Calculating expectation with cumulative probabilities}
If \(X \geq 0\) and takes integer values, then
\[
	\expect{X} = \sum_{k=1}^\infty \prob{X \geq k} = \sum_{k=0}^\infty \prob{X > k}
\]
\begin{proof}
	Since \(X\) takes non-negative integer values,
	\[
		X = \sum_{k=1}^\infty 1(X \geq k) = \sum_{k=0}^\infty 1(X > k)
	\]
	This represents the fact that any integer is the sum of that many ones, e.g.
	\(4 = 1 + 1 + 1 + 1 + 0 + 0 + \dots\) to infinity.
	Taking the expectation of the above formula, using that \(\expect{1(A)} = \prob{A}\) and countable additivity, we have the result as claimed.
\end{proof}

\subsection{Inclusion-exclusion formula with indicators}
We can provide another proof of the inclusion-exclusion formula, using some basic properties of indicator functions.
\begin{itemize}
	\item \(1(\stcomp{A}) = 1 - 1(A)\)
	\item \(1(A \cap B) = 1(A) \cdot 1(B)\)
	\item Following from the above, \(1(A \cup B) = 1-(1-1(A))(1-1(B))\).
\end{itemize}
More generally,
\[
	1(A_1 \cup \dots \cup A_n) = 1-\prod_{i=1}^n(1-1(A_i))
\]
which gives the inclusion-exclusion formula.
Taking the expectation of both sides, we can see that
\[
	\prob{A_1 \cup \dots \cup A_n} = \sum_{i=1}^n \prob{A_i} - \sum_{i_1 < i_2} \prob{A_{i_1} \cap A_{i_2}} + \dots + (-1)^{n+1}\prob{A_1 \cap \dots \cap A_n}
\]
which is the result as previously found.
