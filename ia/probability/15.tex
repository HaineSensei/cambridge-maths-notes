\subsection{Probability Distribution Function}
Let \((\Omega, \mathcal F, \mathbb P)\) be a probability space.
Then, as defined before, \(X \colon \Omega \to \mathbb R\) is a random variable if
\[
	\forall x \in \mathbb R, \{ X \leq x \} = \{ \omega \colon X(\omega) \leq x \} \in \mathcal F
\]
We define the probability distribution function \(F \colon \mathbb R \to [0, 1]\) as
\[
	F(x) = \prob{X \leq x}
\]
\begin{theorem}
	The following properties hold.
	\begin{enumerate}[(i)]
		\item If \(x \leq y\), then \(F(x) \leq F(y)\).
		\item For all \(a < b\), \(\prob{a < x \leq b} = F(b) - F(a)\).
		\item \(F\) is a right continuous function, and left limits always exist.
		      In other words,
		      \[
			      F(x^+) = \lim_{y \to x^+} F(y) = F(x);\quad F(x^-) = \lim_{y \to x^-} F(y) \leq F(x)
		      \]
		\item For all \(x \in\mathbb R\), \(F(x^-) = \prob{X < x}\).
		\item We have \(\lim_{x \to \infty} F(x) = 1\) and \(\lim_{x \to -\infty} F(x) = 0\).
	\end{enumerate}
\end{theorem}
\begin{proof}
	\begin{enumerate}[(i)]
		\item The first statement is immediate from the definition of the probability measure.
		\item We can deduce
		      \begin{align*}
			      \prob{a < X \leq b} & = \prob{\{ a < X \} \cap \{ X \leq b \}}                  \\
			                          & = \prob{X \leq b} - \prob{\{X \leq b\} \cap \{X \leq a\}} \\
			                          & = \prob{X \leq b} - \prob{X \leq a}                       \\
			                          & = F(b) - F(a)
		      \end{align*}
		\item For right continuity, we want to prove \(\lim_{n \to \infty} F(x + \frac{1}{n}) = F(x)\).
		      We will define \(A_n = \{ x < X \leq x + \frac{1}{n} \}\).
		      Then the \(A_n\) are decreasing events, and the intersection of all \(A_n\) is the empty set \(\varnothing\).
		      Hence, by continuity of the probability measure, \(\prob{A_n} \to 0\) as \(n \to \infty\).
		      But \(\prob{A_n} = \prob{x < X \leq x + \frac{1}{n}} = F(x + \frac{1}{n}) - F(x)\), hence \(F(x + \frac{1}{n}) \to F(x)\) as required.
		      Now, we want to show that left limits always exist.
		      This is clear since \(F\) is an increasing function, and is always bounded above by 1.
		\item We know \(F(x^-) = \lim_{n \to \infty}F(x - \frac{1}{n})\).
		      Consider \(B_n = \{ X \leq x - \frac{1}{n} \}\).
		      Then the \(B_n\) is an increasing sequence of events, and their union is \(\{ X < x \}\).
		      Hence \(\prob{B_n}\) converges to \(\prob{X < x}\), so \(F(x^-) = \prob{X < x}\).
		\item This is evident from the properties of the probability measure.
	\end{enumerate}
\end{proof}

\subsection{Defining a Continuous Random Variable}
For a discrete random variable, \(F\) is a step function, which of course is right continuous with left limits.
\begin{definition}
	A random variable \(X\) is called \textit{continuous} if \(F\) is a continuous function.
	In this case, clearly left limits and right limits give the same value, and \(\prob{X = x} = 0\) for all \(x \in\mathbb R\).
\end{definition}
\noindent In this course, we will consider only \textit{absolutely} continuous random variables.
A continuous random variable is absolutely continuous if \(F\) is differentiable.
We will make the convention that \(F'(x) = f(x)\), where \(f(x)\) is called the probability density function of \(X\).
The following immediate properties hold.
\begin{enumerate}[(i)]
	\item \(f \geq 0\)
	\item \(\int_{-\infty}^{+\infty} f(x) \dd{x} = 1\)
	\item \(F(x) = \int_{-\infty}^x f(t) \dd{t}\)
	\item For \(S \subseteq \mathbb R\), \(\prob{X \in S} = \int_S f(x) \dd{x}\)
\end{enumerate}
Here is an intuitive explanation of the probability density function.
Suppose \(\Delta x\) is a small quantity.
Then
\[
	\prob{x < X \leq x + \Delta x} = \int_x^{x + \Delta x} f(y) \dd{y} \approx f(x) \cdot \Delta x
\]
So we can think of \(f(x)\) as the continuous analogy to \(\prob{X = x}\).
