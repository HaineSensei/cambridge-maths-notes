\subsection{Covariance}
\begin{definition}
	Let \(X\) and \(Y\) be random variables. Their covariance is defined
	\[ \Cov{X,Y} = \expect{(X - \expect{X})(Y - \expect{Y})} \]
	It is a measure of how dependent \(X\) and \(Y\) are.
\end{definition}
\noindent Immediately we can deduce the following properties.
\begin{itemize}
	\item \(\Cov{X,Y} = \Cov{Y,X}\)
	\item \(\Cov{X,X} = \Var{X}\)
	\item \(\Cov{X,Y} = \expect{XY} - \expect{X}\cdot\expect{Y}\). Indeed, \((X - \expect{X})(Y - \expect{Y}) = XY - X\expect{Y} - Y\expect{X} + \expect{X}\expect{Y}\) and the result follows.
	\item Let \(c \in \mathbb R\). Then \(\Cov{cX,Y} = c\Cov{X,Y}\), and \(\Cov{c + X,Y} = \Cov{X,Y}\).
	\item \(\Var{X + Y} = \Var{X} + \Var{Y} + 2\Cov{X,Y}\). Indeed, we have

	      \(\Var{X + Y} = \expect{(X - \expect{X} + Y - \expect{Y})^2}\) which gives

	      \(\expect{(X - \expect{X})^2} + \expect{(Y - \expect{Y})^2} + 2\expect{(X - \expect{X})(Y - \expect{Y})}\) as required.
	\item For all \(c \in \mathbb R\), \(\Cov{c, X} = 0\)
	\item If \(X\), \(Y\), \(Z\) are random variables, then \(\Cov{X + Y,Z} = \Cov{X,Z} + \Cov{Y,Z}\). More generally, for \(c_1, \dots, c_n, d_1, \dots, d_m\) real numbers, and for \(X_1, \dots, X_n, Y_1, \dots, Y_m\) random variables, we have
	      \[ \Cov{\sum_{i=1}^n c_i X_i,\sum_{j=1}^m d_j Y_j} = \sum_{i=1}^n \sum_{j=1}^m c_i d_j \Cov{X_i,Y_j} \]
	      In particular, if we apply this to \(X_i = Y_i\), and \(c_i = d_i = 1\), then we have
	      \[ \Var{\sum_{i=1}^n X_i} = \sum_{i=1}^n \Var{X_i} + \sum_{i \neq j} \Cov{X_i,X_j} \]
\end{itemize}

\subsection{Expectation of Functions of a Random Variable}
Recall that \(X\) and \(Y\) are independent if for all \(x\) and \(y\),
\[ \prob{X = x, Y = y} = \prob{X = x} \cdot \prob{Y = y} \]
We would like to prove that given positive functions \(f, g \colon \mathbb R \to \mathbb R_+\), if \(X\) and \(Y\) are independent we have
\[ \expect{f(X)g(Y)} = \expect{f(X)} \cdot \expect{g(Y)} \]
\begin{proof}
	\begin{align*}
		\expect{f(X)g(Y)} & = \sum_{(x, y)} f(x) g(y) \prob{X = x, Y = y}                 \\
		                  & = \sum_{(x, y)} f(x) g(y) \prob{X = x} \prob{Y = y}           \\
		                  & = \sum_{x} f(x) \prob{X = x} \cdot \sum_{y} g(y) \prob{Y = y} \\
		                  & = \expect{f(X)} \cdot \expect{g(Y)}
	\end{align*}
\end{proof}
\noindent The same result holds for general functions, provided the required expectations exist.

\subsection{Covariance of Independent Variables}
Suppose \(X\) and \(Y\) are independent. Then
\[ \Cov{X,Y} = 0 \]
This is because
\begin{align*}
	\Cov{X,Y} & = \expect{(X - \expect{X})(Y - \expect{Y})}             \\
	          & = \expect{X - \expect{X}} \cdot \expect{Y - \expect{Y}} \\
	          & = 0 \cdot 0                                             \\
	          & = 0
\end{align*}
\noindent In particular, we can deduce that
\[ \Var{X + Y} = \Var{X} + \Var{Y} \]
Note, however, that the covariance being equal to zero does not imply independence. For instance, let \(X_1, X_2, X_3\) be independent Bernoulli random variables with parameter \(\frac{1}{2}\). Let us now define \(Y_1 = 2X_1 - 1\), \(Y_2 = 2X_2 - 1\), and \(Z_1 = X_3 Y_1\), \(Z_2 = X_3 Y_2\). Now, we have
\[ \expect{Y_1} = \expect{Y_2} = \expect{Z_1} = \expect{Z_2} = 0 \]
We can find that
\[ \Cov{Z_1, Z_2} = \expect{Z_1 \cdot Z_2} = \expect{X_3^2 Y_1 Y_2} = \expect{X_3^2} \cdot 0 \cdot 0 = 0 \]
However, \(Z_1\) and \(Z_2\) are in fact not independent. Since \(Y_1, Y_2\) are never zero,
\[ \prob{Z_1 = 0, Z_2 = 0} = \prob{X_3 = 0} = \frac{1}{2} \]
But also
\[ \prob{Z_1 = 0} = \prob{Z_2 = 0} = \prob{X_3 = 0} = \frac{1}{2} \implies \prob{Z_1 = 0} \cdot \prob{Z_2 = 0} = 0 \]
So the events are not independent.

\subsection{Markov's Inequality}
The following useful inequality, and the others derived from it, hold in the discrete and the continuous case.
\begin{theorem}
	Let \(X \geq 0\) be a non-negative random variable. Then for all \(a > 0\),
	\[ \prob{X \geq a} \leq \frac{\expect{X}}{a} \]
\end{theorem}
\begin{proof}
	Observe that \(X \geq a \cdot 1(X \geq a)\). This can be seen to be true simply by checking both cases, \(X < a\) and \(X \geq a\). Taking expectations, we get
	\[ \expect{X} \geq \expect{a \cdot 1(X \geq a)} = \expect{a \cdot \prob{X \geq a}} = a \cdot \prob{X \geq a} \]
	and the result follows.
\end{proof}

\subsection{Chebyshev's Inequality}
\begin{theorem}
	Let \(X\) be a random variable with finite expectation. Then for all \(a > 0\),
	\[ \prob{\abs{X - \expect{X}} \geq a} \leq \frac{\Var{X}}{a^2} \]
\end{theorem}
\begin{proof}
	Note that \(\prob{\abs{X - \expect{X}} \geq a} = \prob{\abs{X - \expect{X}}^2 \geq a^2}\). Then we can apply Markov's inequality to this non-negative random variable to get
	\[ \prob{\abs{X - \expect{X}}^2 \geq a^2} \leq \frac{\expect{(X - \expect{X})^2}}{a^2} = \frac{\Var{X}}{a^2} \]
\end{proof}

\subsection{Cauchy-Schwarz Inequality}
\begin{theorem}
	If \(X\) and \(Y\) are random variables, then
	\[ \expect{\abs{XY}} \leq \sqrt{\expect{X^2}\cdot\expect{Y^2}} \]
\end{theorem}
\begin{proof}
	It suffices to prove this statement for \(X\) and \(Y\) which have finite second moments, i.e. \(\expect{X^2}\) and \(\expect{X^2}\) are finite. Clearly if they are infinite, then the upper bound is infinite which is trivially true. We need to show that \(\expect{\abs{XY}}\) is finite. Here we can apply the additional assumption that \(X\) and \(Y\) are non-negative, since we are taking the absolute value:
	\[ XY \leq \frac{1}{2}\left(X^2 + Y^2\right) \implies \expect{XY} \leq \frac{1}{2}\left( \expect{X^2} + \expect{Y^2}  \right) \]
	Now, we can assume \(\expect{X^2} > 0\) and \(\expect{Y^2} > 0\). If this were not the case, the result is trivial since if at least one of them were equal to zero, the corresponding random variable would be identically zero. Let \(t \in \mathbb R\) and consider
	\[ 0 \leq (X - tY)^2 = X^2 - 2tXY + t^2Y^2 \]
	Hence
	\[ \expect{X^2} - 2t\expect{XY} + t^2\expect{Y^2} \geq 0 \]
	We can view this left hand side as a function \(f(t)\). The minimum value of this function is achieved at \(t_\ast = \frac{\expect{XY}}{\expect{Y^2}}\). Then
	\[ f(t_\ast) \geq 0 \implies \expect{X^2} - \frac{2\expect{XY}}{\expect{Y^2}} + \frac{\expect{XY}^2}{\expect{Y^2}} \geq 0 \]
	Hence,
	\[ \expect{XY}^2 \leq \expect{X^2}\cdot \expect{Y^2} \]
	and the result follows.
\end{proof}
