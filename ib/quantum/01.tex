\subsection{Timeline}
\begin{itemize}
	\item (1801--3) Particles were shown to have wave-like properties using Young's double slit experiment.
	\item (1862--4) Electromagnetism was conceived by Maxwell.
	      Light was discovered to be an electromagnetic wave.
	\item (1897) Discovery of the electron by Thomson.
	\item (1900) The Planck law was discovered, which explains black-body radiation.
	\item (1905) The photoelectric effect was discovered by Einstein.
	\item (1909) Wave-light interference patterns were shown to exist with only one photon recorded at a time.
	\item (1911) Rutherford created his atomic model.
	\item (1913) Bohr created his atomic model.
	\item (1923) The Compton experiment showed x-ray scattering off electrons.
	\item (1923--4) De Broglie discovered the concept of wave-particle duality.
	\item (1925--30) The theory of quantum mechanics emerged at this time.
	\item (1927--8) The diffraction experiment was carried out with electrons.
\end{itemize}

\subsection{Particles and Waves in Classical Mechanics}
In classical mechanics, a point-particle is an object with energy and momentum in an infinitesimally small point of space.
Therefore, a particle is determined by the three-dimensional vectors \( \vb x, \vb v = \dot{\vb x} \).
The motion of a particle is governed by Newton's second law,
\[
	m \ddot{\vb x} = \vb F(\vb x, \dot{\vb x})
\]
Solving this equation involves determination of \( \vb x, \dot{\vb x} \) for all \( t > t_0 \), once initial conditions \( \vb x(t_0), \dot{\vb x}(t_0) \) are known.

Waves are classically defined as any real- or complex-valued function with periodicity in time and/or space.
For instance, consider a function \( f \) such that \( f(t + T) = f(t) \), which is a wave with period \( T \).
The frequency \( \nu \) is defined to be \( \frac{1}{T} \), and the angular frequency \( \omega \) is defined as \( 2 \pi \nu = \frac{2\pi}{T} \).
Suppose we have a function in one dimension obeying \( f(x+\lambda) = f(x) \).
This has wavelength \( \lambda \) and wave number \( k = \frac{2\pi}{\lambda} \).

Consider \( f(x) = \exp(\pm i k x) \).
In three dimensions, this becomes \( f(x) = \exp(\pm i \vb k \cdot \vb x) \).
This is called a `plane wave'; the one-dimensional wave number \( k \) has been transformed into a three-dimensional wave vector \( \vb k \).
\( \lambda \) is now defined as \( \frac{2\pi}{\abs{k}} \).

The wave equation in one dimension is
\[
	\pdv[2]{f(x,t)}{t} - c^2 \pdv[2]{f(x,t)}{x} = 0;\quad c \in \mathbb R
\]
The solutions to this equation are
\[
	f_\pm (x,t) = A_\pm \exp(\pm i k x - i \omega t)
\]
where \( \omega = c k; \lambda = \frac{c}{\nu} \).
The two conditions are known as the dispersion relations.
\( A_\pm \) is the amplitude of the waves.

In three dimensions,
\[
	\pdv[2]{f(\vb x,t)}{t} - c^2 \laplacian f(\vb x,t) = 0;\quad c \in \mathbb R
\]
The solution is
\[
	f (\vb x,t) = A \exp(\pm i \vb k \cdot \vb x - i \omega t)
\]
where \( \omega = c \abs{\vb k}; \lambda = \frac{c}{\nu} \).

\begin{note}
	Other kinds of waves are solutions to other governing equations, provided that another dispersion relation \( \omega(\vb k) \) is given.
	Also, for any governing equation linear in \( f \), the superposition principle holds: if \( f_1, f_2 \) are solutions then so is \( f_1 + f_2 \).
\end{note}

\subsection{Black-body Radiation}
Several experiments have shown that light behaves with some particle-like characteristics.
For example, consider a body heated at some temperature \( T \).
Any such body will emit radiation.
The simplest body to study is called a `black-body', which is a totally absorbing surface.
The intensity of light emitted by a black body was modelled as a function of the frequency.
The classical prediction for the spectrum of emitted radiation was that as the frequency increased, the intensity would also increase.
A curve with a clear maximum point was observed.
