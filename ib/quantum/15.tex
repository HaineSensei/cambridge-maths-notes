\subsection{???}
The set \( \qty{ \hat H, \hat L^2, \hat L_i } \) commutes pairwise.
By convention, we choose \( i = 3 \) to extract the \( z \) component of the angular momentum.
Hence,
\begin{enumerate}[(i)]
	\item We can find joint eigenstates of the three operators, and such eigenstates can be chosen to form a basis for the Hilbert space \( \mathcal H \).
	\item The corresponding eigenvalues \( \abs{L}, L_z, E \) can be measured simultaneously to an arbitrary precision.
	\item The set of operators is maximal; there exists no operator (other than a linear combination of the above) that commutes with all three.
\end{enumerate}

\subsection{Joint eigenfunctions of anglar momentum}
We search for joint eigenfunctions of \( \hat L_z \) and \( \hat L^2 \).
We will write \( \hat L \) in spherical coordinates.
In Cartesian coordinates,
\[
	\hat L = -i \hbar x \cdot \nabla
\]
Hence,
\[
	\hat L_3 = -i \hbar \pdv{\phi};\quad \hat L^2 = -\frac{\hbar^2}{\sin^2 \theta} \qty[\sin \theta \pdv{\theta} \qty(\sin \theta \pdv{\theta}) + \pdv[2]{\phi} ]
\]
Now we search for eigenfunctions of these operators.
\[
	\hat L^2 Y(\theta, \phi) = \lambda Y(\theta, \phi);\quad \hat L_3 Y(\theta, \phi) = \hbar m Y(\theta, \phi)
\]
Solving the equation in \( \hat L_3 \),
\[
	-i\hbar \pdv{\phi} Y(\theta, \phi) = \hbar m Y(\theta, \phi)
\]
We can find solutions of the form \( Y(\theta, \phi) = y(\theta)x(\phi) \).
We find
\[
	-i\hbar y(\theta) x'(\phi) = \hbar m y(\theta) x(\phi)
\]
Hence \( y(\theta) \) is arbitrary, and further
\[
	-i\hbar x'(\phi) = \hbar m x(\phi) \implies x(\phi) = e^{i m \phi}
\]
Given that the wavefunctions must be single-valued on \( \mathbb R^3 \), we must have \( x(\phi) \) invariant under the choice of \( \phi = \phi + 2\pi k \).
Hence \( m \) must be an integer.
Since this must also be an eigenfunction of \( \hat L^2 \), we have further
\[
	-\frac{\hbar^2}{\sin^2 \theta} \qty[\sin \theta \pdv{\theta} \qty(\sin \theta \pdv{\theta}) + \pdv[2]{\phi} ] [y(\theta) x(\phi)] = \lambda y(\theta) x(\phi)
\]
Hence, substituting \( x(\phi) = e^{i m \phi} \), we find
\[
	\frac{1}{\sin \theta} \pdv{\theta} \qty(\sin \theta y'(\theta)) - \frac{m^2}{\sin^2 \theta} y(\theta) = - \frac{\lambda}{\hbar^2} y(\theta)
\]
This is the associate Legendre equation.
The solutions of \( y(\theta) \) are the associate Legendre functions.
\[
	y(\theta) = P_{\ell,m}(\cos \theta) = (\sin \theta)^{\abs{m}} \dv[\abs{m}]{(\cos \theta)}  P_\ell(\cos \theta)
\]
where the \( P_\ell \) are the Legendre polynomials.
Since the ordinary Legendre polynomials are of degree \( \ell \), we must have \( \abs{m} \leq \ell \) to obtain a non-zero solution.
This corresponds to the classical notion that \( \abs{L_z} \leq \abs{L} \) for a physical solution.
The eigenvalues of \( \hat L^2 \) are
\[
	\lambda = \ell(\ell+1) \hbar^2
\]
with \( \ell \in \qty{0, 1, 2, \dots} \).
Thus,
\[
	Y_{\ell, m}(\theta, \phi) = P_{\ell,m}(\cos\theta) e^{im\phi}
\]
The \( Y \) functions are called the spherical harmonics.
The parameters \( \ell, m \) are known as the quantum numbers of the eigenfunction; \( \ell \) is the total angular momentum quantum number and \( m \) is the azimuthal quantum number.
Examples of normalised eigenfunctions are
\begin{align*}
	Y_{0,0} = \frac{1}{\sqrt{4 \pi}}             \\
	Y_{1,0} = \sqrt{\frac{3}{4 \pi}} \cos \theta \\
	Y_{1,\pm 1} = \mp \sqrt{\frac{3}{8 \pi}} \sin \theta e^{-i \phi}
\end{align*}
All spherical harmonics can be shown to be orthogonal.
