\subsection{Planck's Constant}
The Planck constant is \( h \approx \SI{6.61e-34}{\joule\second} \).
The reduced Planck constant is \( \hbar = \frac{h}{2\pi} \).
Quantum mechanics typically uses the reduced Planck constant over the normal Planck constant.
The dimensionality of \( h \) is energy multiplied by time, or position multiplied by momentum.

\subsection{Photoelectric Effect}
Consider a metal surface in a vacuum, which is hit by light with angular frequency \( \omega \).
When the radiation hits the surface of the metal, electrons were emitted.
Classically, we would expect that:
\begin{enumerate}[(i)]
\item Since the incident light carries energy proportional to its intensity, increasing the intensity we should have sufficient energy to break the bonds of the electrons with the atoms of the metal.
\item Since the intensity and frequency are independent, light of any \( \omega \) would eventually cause electrons to be emitted, given a high enough intensity.
\item The emission rate should be constant.
\end{enumerate}
In fact, the experiment showed that
\begin{enumerate}[(i)]
\item The maximum energy \( E_{\max} \) of emitted electrons depended on \( \omega \), and not on the intensity.
\item Below a given threshold \( \omega_{\min} \), there was no electron emission.
\item The emission rate increased with the intensity.
\end{enumerate}
Einstein's explanation for this phenomenon was that the light was quantised into small quanta, called photons.
Photons each carry an energy \( E = \hbar \omega \).
Each photon could liberate only one electron.
Thus,
\[ E_{\max} = \hbar \omega - \phi \]
where \( \phi \) is the binding energy of the electron with the metal.
The higher the intensity, the more photons hit the metal.
This implies that more electrons will be scattered.

\subsection{Compton Scattering}
X-rays were emitted towards a crystal, scattering free electrons.
The X-ray should then be deflected by some angle \( \theta \).
Classically, for a given \( \theta \) we would expect that the intensity as a function of \( \omega \) would have a maximum at \( \omega_0 \), the frequency of the incoming X-rays.
This is because we would not expect \( \omega \) to change much after scattering an electron.
However, there was another peak at \( \omega' \), which was dependent on the angle \( \theta \).
In fact, considering the photon and electron as a relativistic system of particles, we can derive (from IA Dynamics and Relativity),
\[ 2 \sin^2 \frac{\theta}{2} = \frac{mc}{\abs{\vb q}} - \frac{mc}{\abs{\vb p}} \]
where \( \vb p \) is the initial momentum and \( \vb q \) is the final momentum.
Assuming \( E = \hbar \omega \) and \( \vb p = \hbar \vb k \),
\[ \abs{\vb p} = \hbar \abs{\vb k} = \hbar \frac{\omega}{c};\quad \abs{\vb q} = \hbar \abs{\vb k'} = \hbar\frac{\omega'}{c} \]
Hence,
\[ \frac{1}{\omega} = \frac{1}{\omega'} + \frac{\hbar}{mc}(1-\cos\theta) \]
So the frequency of the outgoing X-ray should have an angular frequency which is shifted from the original.
The expected peak was actually caused by X-rays simply not interacting with the electrons.

\subsection{Atomic Spectra}
The Rutherford scattering experiment involved shooting \( \alpha \) particles at some thin gold foil.
Most particles travelled through the foil, some were slightly deflected, and some were deflected completely back.
This indicated that the gold foil was mostly comprised of vacuum and there was a high density of positive charge within the atom.
Electrons would orbit around the nucleus.
However, there were problems with this model:
\begin{enumerate}[(i)]
\item If the electrons in orbits moved, they would radiate and lose energy.
However if the electrons were static, they would simply collapse and fall into the nucleus.
\item This model did not explain the atomic spectra, the observed frequencies of light absorbed or emitted by an atom when electrons change energy levels.
\end{enumerate}
The spectra had frequency
\[ \omega_{mn} = 2 \pi c R_0 \qty(\frac{1}{n^2} - \frac{1}{mc}); \quad m, n \in \mathbb N, m > n \]
where \( R_0 \) is the Rydberg constant, approximately \( \SI{1e7}{\per\metre} \).
Bohr theorised that the electron orbits themselves are quantised, so \( L \) (the orbital angular momentum) is an integer multiple of \( \hbar \); \( L_n = n \hbar \).
First, the quantisation of \( L \) implies the quantisation of \( v \) and \( r \).
Indeed, given that \( L \equiv m_e v r \), we have that \( v \) is quantised: \( v_n = \frac{n\hbar}{m_e r} \).
Further, by the Coulomb force, \( F = \frac{e^2}{4 \pi \varepsilon^2} \frac{1}{r^2} \vb e_r = m_e a_r \vb e_r \) where \( a_r \) is the radial acceleration.
Then \( \frac{e^2}{4 \pi \varepsilon^2} \frac{1}{r^2} = m_e \frac{v^2}{r} \implies r = r_n = \frac{4 \pi \varepsilon_0 \hbar^2}{m_e e^2} n^2 \).
The coefficient on \( n^2 \) is known as the Bohr radius.
Immediately then the energy levels \( E \) of the atom can be shown to be quantised.
