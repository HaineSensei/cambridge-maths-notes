\subsection{Time-independent Schr\"odinger equation in spherical polar coordinates}
For a spherically symmetric potential in \( \mathbb R^3 \), the time-independent Schr\"odinger equation is
\[
	-\frac{\hbar^2}{2m} \laplacian{\chi(x)} + U(x) \chi(x) = E\chi(x)
\]
Recall that the Laplacian operator can be expanded in spherical polar coordinates as
\[
	-\frac{\hbar^2}{2m} \qty( \frac{1}{r} \pdv[2]{r} r + \frac{1}{r^2\sin^2\theta}\qty[sin \theta \pdv{\theta}\qty(\sin\theta \pdv{\theta}) + \pdv[2]{\phi}] )\chi(x) + U(x) \chi(x) = E\chi(x)
\]
where
\[
	x = r \cos \phi \sin \theta; \quad y = r \sin \phi \sin \theta;\quad z = r \cos \theta
\]
\begin{definition}
	A \textit{spherically symmetric potential} is a potential \( U \) which depends only on \( r \).
\end{definition}
We search for the particular solutions of the time-dependent Schr\"odinger equation with spherically symmetric potential that are radial eigenfunctions.
If \( \chi(r) \) is a function of \( r \) alone,
\[
	\laplacian{\chi(r)} = \frac{1}{r} \pdv[2]{r} \qty(r\chi(r))
\]
Hence,
\[
	-\frac{\hbar^2}{2mr} \pdv[2]{r} \qty(r \chi(r)) + U(r) \chi(r) = E \chi(r)
\]
This is equivalent to
\[
	-\frac{\hbar^2}{2m} \qty( \chi''(r) + \frac{2}{r} \chi'(r) ) + U(r) \chi(r) = E \chi(r)
\]
The normalisation condition is
\[
	\int_0^\infty \abs{\chi(r)}^2 r^2 \dd{r} < N
\]
The eigenfunctions \( \chi(r) \) must converge to zero sufficiently fast as \( r \to \infty \) in order to be normalisable.
To solve the time-independent Schr\"odinger equation, we will define
\[
	\sigma(r) = r \chi(r)
\]
Then,
\[
	-\frac{\hbar^2}{2m} \sigma''(r) + U(r) \sigma(r) = E \sigma(r)
\]
This is defined for \( r \geq 0 \).
The normalisation condition here is
\[
	\int_0^\infty \abs{\sigma(r)}^2 \dd{r} < N;\quad \sigma(0) = 0;\quad \sigma'(0) < \infty
\]
The conditions at zero force \( \chi \) to be defined and have finite derivative at zero.
To solve the equation for \( \sigma \), we solve on \( \mathbb R \) and search for odd solutions \( \sigma^{(-)} \), so
\[
	\sigma^{(-)}(-r) = -\sigma^{(-)}(r)
\]

\subsection{Spherically symmetric potential well}
Consider the potential well given by
\[
	U(r) = \begin{cases}
		0   & r \leq a \\
		U_0 & r > a
	\end{cases}
\]
where \( a, U_0 > 0 \).
The time-independent Schr\"odinger equation is
\[
	-\frac{\hbar^2}{2m} \sigma''(r) + U(r) \sigma(r) + E \sigma(r)
\]
We search for odd-parity bound states, so \( 0 < E < U_0 \).
Let
\[
	k = \sqrt{\frac{2mE}{\hbar^2}};\quad \overline k = \sqrt{\frac{2m(U_0 - E)}{\hbar^2}}
\]
The solution for \( \sigma \) is
\[
	\sigma(r) = \begin{cases}
		A \sin(kr)           & r \leq a \\
		B e^{-\overline k r} & r > a
	\end{cases}
\]
The continuity condition at \( r = a \) can be imposed to find \( A \sin ka = B e^{-\overline k a} \).
The continuity of the derivative gives \( kA \cos ka = -\overline k B e^{-\overline k a} \).
Therefore,
\[
	-k \cot(ka) = \overline k;\quad k^2 + \overline k^2 = \frac{2mU_0}{\hbar^2}
\]
Hence,
\[
	-\xi \cot \xi = \eta; \quad \xi^2 + \eta^2 = r_0^2
\]
where \( \xi = ka \) and \( \eta = \overline k a \), and \( r_0 = a\sqrt{2mU_0/\hbar} \).
If \( r_0 < \frac{\pi}{2} \), we have no solutions because \( \xi \geq 0 \).
Equivalently, there are no solutions if
\[
	U_0 < \frac{\pi^2 \hbar^2}{8ma^2}
\]

\subsection{Radial wavefunction of hydrogen atom}
The hydrogen atom is comprised of a nucleus and a single electron.
The nucleus has a positive charge and the electron has a negative charge.
We will model the proton to be stationary at the origin.
The Coulomb force experienced by the electron is given by
\[
	F = -\frac{e^2}{4\pi \varepsilon_0} \frac{1}{r^2} = -\pdv{U}{r} \implies U = -\frac{e^2}{4\pi \varepsilon_0} \frac{1}{r}
\]
Since zero potential is achieved only at infinity, we search for bound states with \( E < 0 \).
We will search for the radial symmetric eigenfunctions.
We have
\[
	-\frac{\hbar^2}{2m_e} \qty( \chi''(r) + \frac{2}{r} \chi'(r) ) - \frac{e^2}{4 \pi \varepsilon_0} \frac{1}{r} \chi(r) = E \chi(r)
\]
We define
\[
	\nu^2 = -\frac{2mE}{\hbar^2} > 0;\quad \beta = \frac{e^2 m_e}{2\pi \varepsilon_0 \hbar^2} > 0
\]
The Schr\"odinger equation becomes
\[
	\chi''(r) + \frac{2}{r} \chi'(r) + \qty( \frac{\beta}{r} - \nu^2 ) \chi(r) = 0
\]
Asymptotically as \( r \to \infty \), we can see that \( \chi'' \sim \nu^2 \chi \).
Since \( \nu^2 > 0 \), this yields solutions that asymptotically behave similarly to \( e^{-r \nu} \), where the positive exponential solution is not applicable due to the normalisation condition.
For \( r = 0 \), the eigenfunction should be finite.
We will consider an ansatz educated by the asymptotical behaviour.
Suppose
\[
	\chi(r) = f(r) e^{-\nu r}
\]
and we solve for \( f(r) \).
The Schr\"odinger equation is
\[
	f''(r) + \frac{2}{r} (1 - \nu r) f'(r) + \frac{1}{r} (\beta - 2 \nu) f(r) = 0
\]
This is a homogeneous linear ordinary differential equation with a regular point at \( r = 0 \).
Suppose there exist series solutions.
\[
	f(r) = r^c \sum_{n=0}^\infty a_n r^n
\]
We can differentiate and find
\[
	f'(r) = \sum_{n=0}^\infty a_n (c+n) r^{c+n-1};\quad f''(r) = \sum_{n=0}^\infty a_n (c+n)(c+n-1) r^{c+n-2}
\]
Hence,
\[
	\sum_{n=0}^\infty \qty[ a_n (c+n)(c+n-1) r^{c+n-2} + \frac{2}{r}(1-\nu r) a_n (c+n) r^{c+n-1} + (\beta - 2\nu)r^{c+n-1} ] = 0
\]
By comparing coefficients of the lowest power of \( r \),
\[
	a_0 c(c-1) + 2a_0 c = 0 \implies a_0 c (c+1) = 0 \implies c = -1, 0
\]
The solution \( c = -1 \) implies \( \chi(r) \sim \frac{1}{r} \) which is invalid at \( r = 0 \).
So we require \( c = 0 \).
Then the power series becomes
\[
	\sum_{n=0}^\infty a_n \qty[n(n-1) + 2n] r^{n-2} + \sum_{n=0}^\infty a_n \qty( -2\nu n + \beta - 2 \nu ) r^{n-1} = 0
\]
Comparing coefficients of equal powers of \( r \),
\[
	a_n n(n+1) + a_{n-1} (-2 \nu n + 2 \nu + \beta - 2 \nu) = 0
\]
Hence, we arrive at the recurrence relation
\[
	a_n = \frac{2\nu n - \beta}{n(n+1)} a_{n-1}
\]
Suppose this series were infinite.
Asymptotically, the behaviour of \( f(r) \) is determined by \( \frac{a_n}{a_{n-1}} \sim \frac{2\nu}{n} \).
We can compare this behaviour to the asymptotic behaviour of \( g(r) = e^{2\nu r} \).
In this case, the series expansion with coefficients \( b_n \) satisfies
\[
	b_n = \frac{(2\nu)^n}{n!} \implies \frac{b_n}{b_{n-1}} = \frac{2 \nu}{n}
\]
Hence, asymptotically \( f(r) \sim e^{2 \nu r} \) if the series does not terminate.
Since \( \chi(r) = f(r) e^{-\nu r} \), we have \( \chi(r) \sim e^{\nu r} \) which is not normalisable.
Hence the series is finite.
So there exists an integer \( N > 0 \) such that \( a_N = 0 \) and \( a_{N-1} \neq 0 \).
This implies \( 2 \nu N - \beta = 0 \) hence \( \nu = \frac{\beta}{2N} \).
Substituting \( \nu^2 \) and \( \beta \), we find
\[
	E = E_N = -\frac{e^4 m_e}{32 \pi^2 \varepsilon_0^2 \hbar^2 N^2}
\]
So the eigenvalues are equivalent to those found in the Bohr model.
We now wish to find the radial eigenfunctions.
Note, \( \frac{\beta}{2\nu} = N \) hence we can substitute and find
\[
	\frac{a_n}{a_{n-1}} = -2 \nu \frac{N - n}{n(n+1)}
\]
This recursion can be used to find the coefficients of the polynomial \( f_N(r) \).
\begin{align*}
	f_1(r) & = 1                                   \\
	f_2(r) & = 1 - \nu r                           \\
	f_3(r) & = 1 - 2 \nu r + \frac{2}{3} \nu^2 r^2
\end{align*}
These are called the Laguerre polynomials of order \( N-1 \) (for example, the first order Laguerre polynomial is \( f_2 \)).
We can then multiply the Laguerre polynomials by \( e^{-\nu r} \) and normalise over \( \mathbb R^3 \) to find the normalised eigenfunctions \( \chi_N(r) \).
For example,
\[
	\chi_1(r) = \frac{\nu^{3/2}}{\sqrt{\pi}} = \frac{1}{\sqrt{\pi}} \qty(\frac{e^2 m_e}{4 \pi \varepsilon_0 \hbar^2})^{3/2} e^{-\nu r}
\]
Recall that the Bohr model implied that the ground state has radius \( a_0 \), known as the Bohr radius, given in terms of \( \nu \) by \( a_0 = \frac{1}{\nu} \).
Using quantum mechanics, we instead find
\begin{align*}
	\inner{r}_{\chi_1} & = \int_{\mathbb R^3} \chi_1^\star(r) r \chi_1(r) \dd{V}                                                       \\
	                   & = \int_0^{2\pi} \dd{\phi} \int_{-1}^1 \dd{\cos \theta} \int_0^\infty \frac{\nu^3}{\pi} r^3 e^{-2\nu r} \dd{r} \\
	                   & = 4\pi \frac{\nu^3}{\pi} \int_0^\infty r^3 e^{-2\nu r} \dd{r}                                                 \\
	                   & = \frac{3}{2} a_0
\end{align*}
We have verified with physical experiments that this larger expected radius is physically accurate.
