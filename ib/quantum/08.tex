\subsection{Beam interpretation}
We can choose to ignore the normalisation problem and take the plane waves as the eigenfunctions of the Hamiltonian:
\[
	\chi_k(x) = Ae^{ikx};\quad \psi_k(x,t) = Ae^{ikx}e^{-\frac{i \hbar^2 k^2}{2m} t}
\]
Instead of \( \chi_k(x) \) describing a single particle, we can interpret it as a beam of particles with momentum \( p = \hbar k \) and \( E = \frac{\hbar^2 k^2}{2m} \) with probability density
\[
	\rho_k(x) = \abs{\chi_k(x) e^{-\frac{i\hbar^2 k^2}{2m}t}}^2 = \abs{A}^2
\]
which here is interpreted as a constant average density of particles.
The probability current is given by
\[
	J_k(x,t) = - \frac{i\hbar}{2m} \qty(\psi^\star_k \pdv{\psi_k}{x} - \psi_k \pdv{\psi_k^\star}{x}) = -\frac{i\hbar}{2m} \abs{A}^2 2ik = \abs{A}^2 \frac{\hbar k}{m} = \abs{A}^2 \underbrace{\frac{p}{m}}_{\mathclap{\text{velocity}}}
\]
This is interpreted as the average flux of particles.

\subsection{Scattering states}
We wish to investigate what happens when a particle, or beam of particles, is thrown onto a potential \( U(x) \).
In this case, suppose we have a step function
\[
	U(x) = \begin{cases} U_0 & \text{if } 0 \leq x < a \\
              0   & \text{otherwise}\end{cases}
\]
and a Gaussian wavepacket which is centred at \( x_0 \ll 0 \) moving in the \( +x \) direction, towards the spike in potential.
As \( t \gg 0 \), we end up with a probability density given by two wavepackets; one will be moving left from the spike and one will have cleared the spike and continues moving to the right.
\begin{definition}
	The reflection coefficient \( R \) is
	\[
		R = \lim_{t \to \infty} \int_{-\infty}^0 \abs{\psi_{\text{GP}}(x,t)}^2 \dd{x}
	\]
	which is the probability for the particle to be reflected.
	The transmission coefficient is
	\[
		T = \lim_{t \to \infty} \int_0^\infty \abs{\psi_{\text{GP}}(x,t)}^2 \dd{x}
	\]
	By definition, \( R + T = 1 \).
\end{definition}
In practice, working with Gaussian packets is mathematically challenging (but not impossible).
The beam interpretation, by allowing us to use non-normalisable stationary state wavefunctions, greatly simplifies the computation.

\subsection{Scattering off potential step}
Consider a potential
\[
	U(x) =
	\begin{cases}
		0 & \text{if } x \leq 0 \\
		U_0 \text{if } x > 0
	\end{cases}
\]
We want to solve
\[
	-\frac{\hbar}{2m} \chi''_k(x) + U(x) \chi_k(x) = E\chi_k(x)
\]
We split the problem into two regions: \( x \leq 0, x > 0 \).
For \( x \leq 0 \), the TISE becomes
\[
	\chi''_k(x) + k^2 \chi_k(x) = 0;\quad k = \sqrt{\frac{2mE}{\hbar^2}}
\]
The solution is
\[
	\chi(x) = Ae^{ikx} + Be^{-ikx}
\]
This is a superposition of two beams; the beam of incident particles \( Ae^{ikx} \) and the beam of reflected particles \( Be^{-ikx} \) which are travelling in the opposite direction.
In the region \( x > 0 \), we have
\[
	\chi''_{\overline k}(x) + \overline k^2 \chi_{\overline k}(x) = 0;\quad \overline k = \sqrt{\frac{2m(E-U_0)}{\hbar^2}}
\]
where \( \overline k \) is real if \( E > U_0 \), and \( \overline k \) is pure-imaginary if \( E < U_0 \).
Therefore, for \( E > U_0 \) we have
\[
	\chi_{\overline k}(x) = Ce^{i \overline k x} + De^{-i \overline k x}
\]
which is a beam of particles moving towards the right and an incident beam of particles from the right moving towards the left.
Since no such incident beam exists, we can set \( D = 0 \).
If \( E < U_0 \), the solution is
\[
	\overline k \equiv i \eta \implies \chi_{\overline k}(x) = Ce^{-\eta x} + De^{\eta x}
\]
\( D \neq 0 \) would give infinite values of \( \chi_{\overline k}(x) \) as \( x \to \infty \).
In either case, the eigenfunctions are
\[
	\chi_{k, \overline k}(x) =
	\begin{cases}
		Ae^{ikx} + Be^{-ikx} & x \leq 0 \\
		Ce^{i \overline k x} & x > 0
	\end{cases}
\]
By imposing the boundary conditions, specifically the continuity of \( \chi \), we can determine the constants.
\[
	A + B = C;\quad ikA - ikB = i\overline k C
\]
which gives
\[
	B = \frac{k - \overline k}{k + \overline k} A;\quad C = \frac{2k}{k + \overline k}A
\]
We can view these solutions in terms of particle flux.
\[
	J_k(x,t) = - \frac{i\hbar}{2m} \qty(\psi^\star_k \pdv{\psi_k}{x} - \psi_k \pdv{\psi_k^\star}{x})
\]
If \( E > U_0 \), we find
\[
	J(x,t) =
	\begin{cases}
		\frac{\hbar k}{m} (\abs{A}^2 - \abs{B}^2) & x < 0    \\
		\frac{\hbar \overline k}{m} \abs{C}^2     & x \geq 0
	\end{cases}
\]
The incident flux is \( \frac{\hbar k}{m} \abs{A}^2 \), the reflected flux is \( \frac{\hbar k}{m} \abs{B}^2 \), and the transmitted flux is \( \frac{\hbar k}{m} \abs{C}^2 \).
We can define
\[
	R = \frac{J_{\text{ref}}}{J_{\text{inc}}} = \frac{\abs{B}^2}{\abs{A}^2} = \qty(\frac{k - \overline k}{k + \overline k})^2
\]
We can also define
\[
	T = \frac{J_{\text{trans}}}{J_{\text{inc}}} = \frac{k}{\overline k} \frac{\abs{C}^2}{\abs{A}^2} = \frac{4k \overline k}{(k + \overline k)^2}
\]
We can check that our original interpretation makes sense; for example, \( R + T = 1 \), and \( E \to U_0, \overline k \to 0 \) implies \( T \to 0 \), \( R \to 1 \).
If \( E \to \infty \), \( T \to 1 \) and \( R \to 0 \).
If \( E < U_0 \),
\[
	J(x,t) =
	\begin{cases}
		\frac{\hbar k}{m} (\abs{A}^2 + \abs{B}^2) & x < 0    \\
		0                                         & x \geq 0
	\end{cases}
\]
since \( \chi_{\overline k} = \chi_{\overline k}^\star \).
Here, \( T = 0 \) but \( \chi_{\overline k}(x) \neq 0 \).
