\subsection{Time-independent Schr\"odinger equation in spherical polar coordinates}
For a spherically symmetric potential in \( \mathbb R^3 \), the time-independent Schr\"odinger equation is
\[
	-\frac{\hbar^2}{2m} \laplacian{\chi(x)} + U(x) \chi(x) = E\chi(x)
\]
Recall that the Laplacian operator can be expanded in spherical polar coordinates as
\[
	-\frac{\hbar^2}{2m} \qty( \frac{1}{r} \pdv[2]{r} r + \frac{1}{r^2\sin^2\theta}\qty[sin \theta \pdv{\theta}\qty(\sin\theta \pdv{\theta}) + \pdv[2]{\phi}] )\chi(x) + U(x) \chi(x) = E\chi(x)
\]
where
\[
	x = r \cos \phi \sin \theta; \quad y = r \sin \phi \sin \theta;\quad z = r \cos \theta
\]
\begin{definition}
	A \textit{spherically symmetric potential} is a potential \( U \) which depends only on \( r \).
\end{definition}
We search for the particular solutions of the time-dependent Schr\"odinger equation with spherically symmetric potential that are radial eigenfunctions.
If \( \chi(r) \) is a function of \( r \) alone,
\[
	\laplacian{\chi(r)} = \frac{1}{r} \pdv[2]{r} \qty(r\chi(r))
\]
Hence,
\[
	-\frac{\hbar^2}{2mr} \pdv[2]{r} \qty(r \chi(r)) + U(r) \chi(r) = E \chi(r)
\]
This is equivalent to
\[
	-\frac{\hbar^2}{2m} \qty( \chi''(r) + \frac{2}{r} \chi'(r) ) + U(r) \chi(r) = E \chi(r)
\]
The normalisation condition is
\[
	\int_0^\infty \abs{\chi(r)}^2 r^2 \dd{r} < N
\]
The eigenfunctions \( \chi(r) \) must converge to zero sufficiently fast as \( r \to \infty \) in order to be normalisable.
To solve the time-independent Schr\"odinger equation, we will define
\[
	\sigma(r) = r \chi(r)
\]
Then,
\[
	-\frac{\hbar^2}{2m} \sigma''(r) + U(r) \sigma(r) = E \sigma(r)
\]
This is defined for \( r \geq 0 \).
The normalisation condition here is
\[
	\int_0^\infty \abs{\sigma(r)}^2 \dd{r} < N;\quad \sigma(0) = 0;\quad \sigma'(0) < \infty
\]
The conditions at zero force \( \chi \) to be defined and have finite derivative at zero.
To solve the equation for \( \sigma \), we solve on \( \mathbb R \) and search for odd solutions \( \sigma^{(-)} \), so
\[
	\sigma^{(-)}(-r) = -\sigma^{(-)}(r)
\]

\subsection{Spherically symmetric potential well}
Consider the potential well given by
\[
	U(r) = \begin{cases}
		0   & r \leq a \\
		U_0 & r > a
	\end{cases}
\]
where \( a, U_0 > 0 \).
The time-independent Schr\"odinger equation is
\[
	-\frac{\hbar^2}{2m} \sigma''(r) + U(r) \sigma(r) + E \sigma(r)
\]
We search for odd-parity bound states, so \( 0 < E < U_0 \).
Let
\[
	k = \sqrt{\frac{2mE}{\hbar^2}};\quad \overline k = \sqrt{\frac{2m(U_0 - E)}{\hbar^2}}
\]
The solution for \( \sigma \) is
\[
	\sigma(r) = \begin{cases}
		A \sin(kr)           & r \leq a \\
		B e^{-\overline k r} & r > a
	\end{cases}
\]
The continuity condition at \( r = a \) can be imposed to find \( A \sin ka = B e^{-\overline k a} \).
The continuity of the derivative gives \( kA \cos ka = -\overline k B e^{-\overline k a} \).
Therefore,
\[
	-k \cot(ka) = \overline k;\quad k^2 + \overline k^2 = \frac{2mU_0}{\hbar^2}
\]
Hence,
\[
	-\xi \cot \xi = \eta; \quad \xi^2 + \eta^2 = r_0^2
\]
where \( \xi = ka \) and \( \eta = \overline k a \), and \( r_0 = a\sqrt{2mU_0/\hbar} \).
If \( r_0 < \frac{\pi}{2} \), we have no solutions because \( \xi \geq 0 \).
Equivalently, there are no solutions if
\[
	U_0 < \frac{\pi^2 \hbar^2}{8ma^2}
\]
