\subsection{Stationary states}
\begin{definition}
	With the ansatz \( \psi(x,t) = \chi(x) T(t) \), we have found a particular class of solutions of the time-independent Schr\"odinger equation:
	\[
		\psi(x,t) = \chi(x) e^{-\frac{i E t}{\hbar}}
	\]
	where \( \chi(x) \) are the eigenfunctions of \( \hat H \) with eigenvalue \( E \).
	Such solutions are called stationary states.
\end{definition}
\noindent Note,
\[
	\rho(x,t) = \abs{\psi(x,t)}^2 = \abs{\chi(x)}^2
\]
This explains the naming of the states as `stationary', as their probability density is independent of time.
Now, suppose \( E \) is quantised.
Then, the general solution to the system is
\[
	\psi(x,t) = \sum_{n=1}^N a_n \chi_n(x) e^{-\frac{iE_n t}{\hbar}}
\]
where \( N \) can be finite or infinite.
In principle, we can also have a continuous energy state \( E_\alpha, \alpha \in \mathbb R \).
We can still use the same idea:
\[
	\psi(x,t) = \int_{\Delta \alpha} A(\alpha) \chi_\alpha(x) e^{-\frac{iE_\alpha t}{\hbar}} \dd{\alpha}
\]
Note that \( \abs{a_n}^2 \) and \( A(\alpha) \dd{\alpha} \) give the probability of measuring the particle energy to be \( E_n \) or \( E_\alpha \).

\subsection{One-dimensional solutions to Schr\"odinger equation}
Recall the equation
\[
	\hat H \chi(x) = -\frac{\hbar^2}{2m} \chi''(x) + U(x) \chi(x) = E \chi(x)
\]
We will solve this equation for:
\begin{itemize}
	\item bound states, such as potential wells;
	\item unbound states, including free particles and scattering of the potential;
	\item harmonic oscillators.
\end{itemize}

\subsection{Infinite potential well}
We define
\[
	U(x) = \begin{cases}
		0      & \text{for } \abs{x} \leq a \\
		\infty & \text{for } \abs{x} < a
	\end{cases}
\]
For \( \abs{x} > a \), we must have \( \chi(a) = 0 \).
Otherwise, \( \chi \cdot U = \infty \).
This gives us a boundary condition, \( \chi(\pm a) = 0 \).
For \( \abs{x} \leq a \), we seek solutions of the form
\[
	-\frac{\hbar^2}{2m} \chi''(x) = E \chi(x);\quad \chi(\pm a) = 0
\]
Equivalently,
\[
	\chi''(x) + k^2 \chi(x) = 0;\quad k = \sqrt{\frac{2mE}{\hbar^2}}
\]
Since \( E > 0 \),
\[
	\chi(x) = A \sin kx + B \cos kx
\]
Imposing boundary conditions,
\[
	A \sin ka + B \cos ka = 0;\quad A \sin ka - B \cos ka = 0
\]
Suppose \( A = 0 \), giving \( \chi(x) = B \cos kx \).
Then, imposing boundary conditions, \( \chi_n(x) = B \cos k_n x \) where \( k_n = \frac{n \pi}{2a} \), and \( n \) are odd positive integers.
These are even solutions.

Alternatively, suppose \( B = 0 \).
In this case, \( \chi(x) = A \sin kx \).
Thus, \( \chi_n(x) = A \sin k_n x \) where \( k_n = \frac{n \pi}{2a} \), and \( n \) are even non-zero positive integers.
These provide odd solutions.

We can also determine the normalisation constants by defining that the eigenfunctions of the Hamiltonian are normalised to unity.
Thus,
\[
	\int_{-a}^a \abs{\chi_n(x)}^2 = 1 \implies A = B = \sqrt{\frac{1}{a}}
\]
Hence, the general solution is given by the eigenvalues
\[
	E_n = \frac{\hbar^2}{2n} k_n^2 = \frac{\hbar^2 \pi^2 n^2}{2ma^2}
\]
and eigenfunctions
\[
	\chi_n(x) = \sqrt{\frac{1}{a}} \begin{cases}
		\cos(\frac{n \pi x}{2a}) & \text{if } n \text{ odd}  \\
		\sin(\frac{n \pi x}{2a}) & \text{if } n \text{ even}
	\end{cases}
\]
\begin{remark}
	Note that unlike classical mechanics, the ground state energy is not zero.
	Note also that \( \chi_n \) have \( (n+1) \) nodes in which \( \rho(x) = 0 \).
	When \( n \to \infty \), \( \rho_n(x) \) tends to a constant, which is like in classical mechanics.
	Eigenfunctions of the Hamiltonian in this case were either odd or even; we can in fact prove that this is the case in general.
\end{remark}
\begin{proposition}
	If we have a system of non-degenerate eigenstates (\( E_i \neq E_j \)),  then if \( U(x) = U(-x) \) the eigenfunctions of \( \hat H \) must be either odd or even.
\end{proposition}
\begin{proof}
	The time-independent Schr\"odinger equation is invariant under \( x \mapsto -x \) if \( U \) is even.
	Hence, if \( \chi(x) \) is a solution with eigenvalue \( E \), then \( \chi(-x) \) is also a solution.
	Since we have a non-degenerate solution, \( \chi(-x) = \chi(x) \) hence the solutions must be the same up to a normalisation factor.
	For consistency, \( \chi(x) = \chi(-(-x)) = \alpha \chi(-x) = \alpha^2 \chi(x) \).
	Hence \( \alpha = \pm 1 \), so \( \chi \) is either odd or even.
\end{proof}

\subsection{Finite potential well}
We define
\[
	U(x) = \begin{cases}
		0   & \text{for } \abs{x} \leq a \\
		U_0 & \text{for } \abs{x} < a
	\end{cases}
\]
Classically, if \( E < U_0 \), the particle has insufficient energy to escape the well.
We will only consider eigenstates with \( E < U_0 \) here, but we will find that it is possible in quantum mechanics to escape the well with positive probability.
We will search for even functions only, odd functions can be solved independently.
If \( \abs{x} \leq a \),
\[
	-\frac{\hbar^2}{2m} \chi''(x) = E\chi(x)
\]
Equivalently,
\[
	\chi''(x) + k^2 \chi(x) = 0;\quad k = \sqrt{\frac{2mE}{\hbar^2}}
\]
The solution becomes
\[
	\chi(x) = A \sin kx + B \cos kx \implies \chi(x) = B \cos kx
\]
since we are only looking for even solutions.
In the region \( \abs{x} > a \),
\[
	-\frac{\hbar^2}{2m} \chi''(x) + U_0 \chi(x) = E \chi(x)
\]
giving
\[
	\chi''(x) - \overline k^2 \chi(x) = 0;\quad \overline k = \sqrt{\frac{2m(U_0 - E)}{\hbar^2}}
\]
This yields exponential solutions:
\[
	\chi(x) = C e^{\overline k x} + D e^{-\overline k x}
\]
Imposing the normalisability constraints, for \( x > a \) we have \( C = 0 \), and for \( x < -a \) we have \( D = 0 \).
Imposing even parity, \( C = D \) when nonzero.
Thus,
\[
	\chi(x) = \begin{cases}
		C e^{\overline k x}  & x < -a         \\
		B \cos(kx)           & \abs{x} \leq a \\
		C e^{-\overline k x} & x > a
	\end{cases}
\]
Now we must impose continuity of \( \chi(x) \) and its derivative at \( x = \pm a \).
First,
\[
	C e^{-\overline k a} = B \cos(k a)
\]
The other gives
\[
	-\overline k C e^{-\overline k a} = -k B \sin(k a)
\]
From the ratio of both constraints,
\[
	k \tan (ka) = \overline k
\]
From the definition of \( k, \overline k \),
\[
	k^2 + \overline k^2 = \frac{2mU_0}{\hbar^2}
\]
We will define some rescaled variables for convenience: \( \xi = ka \), \( \eta = \overline k a \).
Rewriting,
\[
	\xi \tan \xi = \eta;\quad \xi^2 + \eta^2 = r_0^2;\quad r_0 = \frac{2mU}{\hbar}
\]
This may be solved graphically.
The eigenvalues of the system correspond to the points of intersection between the two equations.
There are always a finite number of possible intersections, regardless of the value of \( r_0 \).
The eigenvalues are
\[
	E_n = \frac{\hbar^2}{2 n a^2} \xi_n^2;\quad \xi \in \qty{\xi_1, \dots, \xi_n};\quad n = 1, \dots, p
\]
When \( U_0 \to \infty \), \( r_0 \to \infty \).
At this point, there are an infinite amount of intersections, so the eigenvalues of the Hamiltonian become that of the infinite well.
Further \( \chi(x) \) tends to the eigenfunctions of the infinite well.
Note that the \( \chi_n(x) \) have some positive region outside the well.
We can use the unused condition above to write \( C \) in terms of \( B \), and then we can use the normalisation condition to find \( B \).
