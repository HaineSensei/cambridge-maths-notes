\subsection{Wave-like behaviour of particles}
De Broglie hypothesised that any particle of any mass is associated with a wave with
\[
	\omega = \frac{E}{\hbar}; \quad \vb k = \frac{\vb p}{\hbar}
\]
This hypothesis made sense of the quantisation of electron angular momentum; if the electron lies on a circular orbit then \( 2 \pi r = n \lambda \) where \( \lambda \) is the wavelength of the electron.
However,
\[
	p = \hbar k = \hbar \frac{2 \pi}{\lambda} \implies L = m_e v r = p r = \hbar \frac{2 \pi}{\lambda} \frac{n \lambda}{2 \pi} = n \hbar
\]
Hence the angular momentum must be quantised.
The electron diffraction experiment showed that this hypothesis was true, by showing that electrons behaved sufficiently like waves.
Interference patterns were observed with \( \lambda = \frac{2 \pi}{\abs{\vb k}} = \frac{2 \pi k}{\abs{\vb p}} \) compatible with the De Broglie hypothesis.

\subsection{Probabilistic interpretation of wavefunctions}
In classical mechanics, we can describe a particle with \( \vb x, \dot{\vb x} \) or \( \vb p = m \dot{\vb x} \).
In quantum mechanics, we need the state \( \psi \) described by \( \psi(\vb x, t) \) called the wavefunction.
\begin{remark}
	Note that the state is an abstract entity, while \( \psi(\vb x, t) \) is the representation of \( \psi \) in the space of \( \vb x \).
	In some sense, \( \psi(\vb x, t) \) is the complex coefficient of \( \psi \) in the continuous basis of \( \vb x \).
	In other words, \( \psi(\vb x, t) \) is \( \psi \) in the \( \vb x \) representation.
	In this course, we always work in the \( \vb x \) representation.
\end{remark}
\begin{definition}
	A wavefunction is a function \( \psi(\vb x, t) \colon \mathbb R^3 \to \mathbb C \) that satisfies certain mathematical properties (defined later) dictated by its physical interpretation.
	\( t \) is considered a fixed external parameter, so it is not included in the function's type.
\end{definition}
The physical interpretation of a wavefunction is called Born's rule.
The probability density for a particle to be at some point \( \vb x \) at \( t \) is given by \( \abs{\psi(\vb x, t)}^2 \).
We write the probability density as \( \rho \), hence \( \rho(\vb x, t) \dd{V} \) is the probability that the particle lies in some small volume \( V \) centred at \( \vb x \).
Now, since the particle must be somewhere, the wave function must be \textit{normalisable}, or \textit{square-integrable} in \( \mathbb R^3 \):
\[
	\int_{\mathbb R^3} \psi^\star(\vb x, t) \psi(\vb x, t) \dd{V} = \int_{\mathbb R^3} \abs{\psi(\vb x, t)}^2 \dd{V} = N \in (0, \infty)
\]
Since we want the total probability to be 1, we must normalise the wavefunction by defining
\[
	\overline{\psi}(\vb x, t) = \frac{1}{\sqrt{N}} \psi(\vb x, t) \iff \int_{\mathbb R^3} \abs{\overline{\psi}(\vb x, t)}^2 \dd{V} = 1
\]
Hence, \( \rho(\vb x,t) = \abs{\overline{\psi}(\vb x,t)}^2 \) really is a probability density.
From now, we will not use the bar for denoting normalisation, since normalisation is evident from context.

\subsection{Bases and equivalence classes}
In linear algebra, we consider vectors in some vector space such as \( \mathbb R^n \).
In quantum mechanics, we instead consider states in a space of wave functions.
The analogous concept to vector components is to represent a state \( \psi \) in an infinite-dimensional \( x \) axis basis \( \psi(x,t) \).
Note that if two wavefunctions differ by a constant phase, that is, \( \exists \alpha \in \mathbb R \) such that
\[
	\widetilde \psi(x,t) = e^{i \alpha} \psi(x,t)
\]
then the states are equivalent in terms of probability, since the probability density is given by the norm of \( \psi \), not its angle.
We can think of states as arrays in the vector space of wavefunctions.
We can then describe the equivalence class \( [\psi] \) as the set of all functions \( \phi \) such that \( \phi = \lambda \psi \), for some \( \lambda \in \mathbb C \setminus \qty{0} \), since we must retain the condition that \( \phi \) is normalisable.

\subsection{Hilbert spaces}
In quantum mechanics, we are interested in the functional space of square-integrable functions on \( \mathbb R^3 \), which is a type of \textit{Hilbert space} and denoted \( \mathcal H \).
\begin{remark}
	Since the set of wavefunctions form a vector space, \( \psi_1, \psi_2 \in \mathcal H \) implies that \( \psi = \lambda_1 \psi_1 + \lambda_2 \psi_2 \in \mathcal H \) for constants \( \lambda_1, \lambda_2 \in \mathbb C \) provided this \( \psi \) is nonzero.
	For waves, this is the well-known superposition principle.
	Note that this exact formuliation of linearity is unique to quantum mechanics; for example, in classical mechanics, two solutions to Newton's equations may not be combined into a new solution by taking their sum.
\end{remark}
\begin{proposition}
	If \( \psi_1(x,t), \psi_2(x,t) \) are normalisable, then \( \psi = \lambda_1 \psi_i(x,t) + \lambda_2 \psi_2(x,t) \) is also normalisable.
\end{proposition}
\begin{proof}
	Recall the inequality
	\[
		2 \abs{z_1} \abs{z_2} \leq \abs{z_1}^2 + \abs{z_2}^2
	\]
	Then we can show
	\begin{align*}
		\int_{\mathbb R^3}\abs{\lambda_1 \psi_1 + \lambda_2 \psi_2}^2 \dd{V} & = \int_{\mathbb R^3}\qty(\abs{\lambda_1 \psi_1} + \abs{\lambda_2 \psi_2})^2 \dd{V}                                                   \\
		                                                                     & = \int_{\mathbb R^3}\qty(\abs{\lambda_1 \psi_1}^2 + 2\abs{\lambda_1 \psi_1}\abs{\lambda_2 \psi_2} + \abs{\lambda_2 \psi_2}^2) \dd{V} \\
		                                                                     & = \int_{\mathbb R^3}\qty(2\abs{\lambda_1 \psi_1}^2 + 2\abs{\lambda_2 \psi_2}^2) \dd{V} < \infty                                      \\
	\end{align*}
	so the norm is non-infinite.
\end{proof}

\subsection{Inner product}
We define the inner product between two wavefunctions to be
\[
	\inner{\psi,\phi} = \int_{\mathbb R^3} \psi^\star \phi \dd{V}
\]
The following statements hold.
\begin{enumerate}
	\item \( \inner{\psi, \phi} \) exists for all wave functions \( \psi, \phi \in \mathcal H \);
	\item \( \inner{\psi, \phi}^\star = \inner{\phi, \psi} \);
	\item the inner product is antilinear in the first entry, and linear in the second entry; and
	\item for continuous \( \psi \), \( \inner{\psi, \psi} = 0 \) is true if and only if \( \psi \) is identically zero.
\end{enumerate}
We prove the first statement, since the others are obvious from the definition.
By the Cauchy--Schwarz inequality,
\begin{align*}
	\int_{\mathbb R^3} \abs{\psi}^2 \dd{V}               & \leq N_1;                                                                                                \\
	\int_{\mathbb R^3} \abs{\phi}^2 \dd{V}               & \leq N_2;                                                                                                \\
	\therefore\ \int_{\mathbb R^3} \abs{\psi \phi} \dd{V} & \leq \sqrt{\int_{\mathbb R^3} \abs{\psi}^2 \dd{V} \cdot \int_{\mathbb R^3} \abs{\phi}^2 \dd{V}} < \infty \\
\end{align*}

\subsection{Normalisation}
\begin{definition}
	We define the norm of a wavefunction to be \( \norm{\psi} \equiv \inner{\psi,\psi} \).
	A wavefunction \( \psi \) is \textit{normalised} if \( \norm{\psi} = 1 \).
\end{definition}
\begin{definition}
	A set of wavefunctions \( \qty{\psi_n} \) is \textit{orthonormal} if \( \inner{\psi_m, \psi_n} = \delta_{mn} \).
	A set of wavefunctions \( \qty{\psi_n} \) is \textit{complete} if for any \( \psi \in \mathcal H \), we can write
	\[
		\psi = \sum_n \lambda_n \psi_n
	\]
	for \( \lambda_n \in \mathbb C \).
\end{definition}
\begin{proposition}
	If \( \qty{\psi_n} \) is a complete and orthonormal basis of \( \mathcal H \), then
	\[
		\phi = \sum_{k=0}^n c_k \psi_k
	\]
	where
	\[
		c_k = \inner{\psi_k, \phi}
	\]
\end{proposition}
\begin{proof}
	Suppose we can write \( \phi \) in this form.
	Then,
	\begin{align*}
		\inner{\psi_n, \phi} & = \inner{\psi_n, \sum_m c_m \psi_m} \\
		                     & = \sum_m c_m \inner{\psi_n, \psi_m} \\
		                     & = \sum_m c_m \delta_{mn}             \\
		                     & = c_n
	\end{align*}
\end{proof}
\begin{remark}
	If \( \phi \) is the desired outcome of a measurement for a particle described by \( \psi \), then the probability of observing \( \phi \) given \( \psi \) at some time \( t \) is
	\[
		\abs{\inner{\psi, \phi}}^2 = \abs{\int_{\mathbb R^3} \psi^\star \phi \dd{V}}^2
	\]
\end{remark}

\subsection{Time-dependent Schr\"odinger equation}
\begin{definition}
	The evolution of the wavefunction over time is given by the \textit{time-dependent Schr\"odinger equation (TDSE)},
	\[
		i\hbar \pdv{\psi}{t} = -\frac{\hbar^2}{2m} \laplacian \psi + U \psi
	\]
	where \( U = U(x) \) is a real potential energy term.
\end{definition}
\begin{remark}
	This equation is a first-order differential equation in \( t \).
	Contrast this to Newton's second law, which is a second-order differential equation in \( t \).
	This implies that we only need a single initial condition \( \psi(x,t_0) \) to determine all future behaviour.
\end{remark}
\begin{remark}
	Note the asymmetry between the spatial and temporal components: there is only a first derivative in time but a second derivative in space.
	This implies that this equation is incompatible with relativity, where time and space must be treated equitably.
\end{remark}
One way to conceptualise the TDSE is by letting \( \psi \) be some wave defined by
\[
	\psi(x,t) = \exp[ i(k \cdot x - \omega t) ]
\]
Then, the De Broglie hypothesis (\( k = p/\hbar, \omega = E/m \)) implies that
\[
	\psi(x,t) = \exp\qty[\frac{i}{\hbar}\qty(p \cdot x - \frac{p^2}{2m}t)]
\]
which is a solution to the TDSE.\@

\subsection{Normalisation and time evolution}
Because of the TDSE, we can show that the norm \( N \) of a wavefunction \( \psi \) is independent of \( t \).
\[
	\dv{N}{t} = \int_{\mathbb R^3} \pdv{t} \abs{\psi(x,t)}^2 \dd{V}
\]
Now, note that
\[
	\pdv{t} \abs{\psi}^2 = \pdv{t} \inner{\psi^\star, \psi} = \psi^\star \pdv{\psi}{t} + \psi \pdv{\psi^\star}{t}
\]
The TDSE then gives
\begin{align*}
	\pdv{\psi}{t}                    & = \frac{i \hbar}{2m} \laplacian \psi^2 + \frac{i}{k} U \psi;                   \\
	\pdv{\psi^\star}{t}              & = - \frac{i \hbar}{2m} \laplacian \psi^2 - \frac{i}{k} U \psi^\star            \\
	\therefore\ \pdv{\abs{\psi}^2}{t} & = \div[\frac{i\hbar}{2m}\qty(\psi^\star \grad{\psi} - \psi \grad{\psi^\star})]
\end{align*}
Finally,
\[
	\int_{\mathbb R^3} \pdv{\abs{\psi}^2}{t} \dd{V} = \int_{\mathbb R^3} \div[ \frac{i\hbar}{2m}\qty(\psi^\star \grad{\psi} - \psi \grad{\psi^\star}) ] = 0
\]
since \( \psi, \psi^\star \) are such that \( \abs{\psi} \to 0 \) as \( \abs{x} \to \infty \).

\subsection{Conserved probability current}
We have proven that the normalisation of wavefunctions are constant in time.
Hence, we can derive the probability conservation law:
\[
	\pdv{\rho}{t}(x,t) + \div{J} = 0;\quad J(x,t) = \frac{-i\hbar}{2m} \qty(\psi^\star \grad{\psi} - \psi \grad{\psi^\star} )
\]
This is the conserved probability current.
