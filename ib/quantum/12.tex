\subsection{Consequences of uncertainty relation}
\begin{enumerate}[(i)]
	\item \( \qty[\hat A, \hat B] = 0 \) implies that there exists a joint set of eigenfunctions which is a complete basis of \( \mathcal H \).
	      In particular, \( \hat A \) and \( \hat B \) can be measured simulaneously with arbitrary precision.
	      For instance, we can measure \( E \), \( \abs{\overline L} \) and \( L z \) simultaneously for an electron on a hydrogen atom.
	\item We cannot simultaneously measure position and momentum of a particle with arbitrary precision.
	      In particular,
	      \[
		      \Delta_\psi x \Delta_\psi p \geq \frac{\hbar}{2}
	      \]
	      This is Heisenberg's uncertainty principle.
\end{enumerate}

\subsection{States of minimal uncertainty}
The Gaussian wavepacket was a state of minimal uncertainty:
\[
	\Delta_\psi x \Delta_\psi p = \frac{\hbar}{2}
\]
We would like to analyse the conditions for a state \( \psi \) to have minimal uncertainty.
\begin{lemma}
	\( \psi \) is a state of minimal uncertainty if and only if
	\[
		\hat x \psi = i a \hat p \psi
	\]
	for some \( a \in \mathbb R \).
	A non-example is the De Broglie plane waves.
\end{lemma}
\begin{lemma}
	The condition for the above lemma to hold is that
	\[
		\psi(x) = ce^{-bx^2};\quad b,c \in \mathbb R, b > 0, c \neq 0
	\]
	The Gaussian wavepacket is an example of this form.
\end{lemma}

\subsection{Ehrenfest theorem}
\begin{theorem}
	The time evolution of a Hermitian operator \( \hat A \) is governed by
	\[
		\dv{t} \inner{\hat A}_\psi = \frac{i}{\hbar} \inner{\qty[\hat H, \hat A]}_\psi + \inner{\pdv{\hat A}{t}}_\psi
	\]
	In this course, we will not see any operators with time dependence, so the last term will not be needed.
\end{theorem}
\begin{proof}
	\begin{align*}
		\dv{t} \inner{\hat A}_\psi & = \dv{t} \int_{-\infty}^\infty \psi^\star \hat A \psi \dd{x}                                                                                \\
		                           & = \int_{-\infty}^\infty \pdv{t} \qty(\psi^\star \hat A \psi) \dd{x}                                                                         \\
		                           & = \int_{-\infty}^\infty \qty[\pdv{\psi^\star}{t} \hat A \psi + \psi^\star \pdv{\hat A}{t} \psi + \psi^\star \hat A \pdv{\hat A}{t} ] \dd{x}
	\end{align*}
	The time-dependent Schr\"odinger equation gives
	\[
		\qty(i \hbar \pdv{\psi}{t})^\star = \qty(\hat H \psi)^\star \implies -i\hbar \pdv{\psi^\star}{t} = \psi^\star \hat H^\star = \psi^\star \hat H^\star
	\]
	Hence,
	\begin{align*}
		\dv{t} \inner{\hat A}_\psi & = \frac{i}{\hbar} \int_{-\infty}^\infty \qty[\psi^\star \hat H \hat A \psi - \psi^\star \hat A \hat H \psi] \dd{x} + \int_{-\infty}^\infty \psi^\star \pdv{\hat A}{t} \psi \dd{x} \\
		                           & = \frac{i}{\hbar} \inner{\qty[\hat H, \hat A]}_\psi + \inner{\pdv{\hat A}{t}}_\psi
	\end{align*}
\end{proof}
\begin{example}
	Let \( \hat A = \hat H \).
	Then,
	\[
		\dv{t} \inner{\hat H}_\psi = 0
	\]
	This corresponds to the classical notion of conservation of energy.
\end{example}
\begin{example}
	Let \( \hat A = \hat p \).
	First, note
	\begin{align*}
		\qty[\hat H, \hat p] \psi & = \qty[\frac{\hat p^2}{2m} + U(\hat x), \hat p] \psi               \\
		                          & = \qty[U(\hat x), \hat p] \psi                                     \\
		                          & = U(x) \qty(-i\hbar \pdv{x}) \psi - \qty(-i\hbar \pdv{x})U(x) \psi \\
		                          & = i \hbar \pdv{U(x)}{x} \psi
	\end{align*}
	Hence,
	\[
		\dv{t}\inner{\hat p}_\psi = \frac{i}{\hbar} \inner{\qty[\hat H, \hat p]}_\psi = -\inner{\pdv{U}{x}}_\psi
	\]
	This corresponds exactly to Newton's second law,
	\[
		\dot p = -\dv{U}{x}
	\]
\end{example}
\begin{example}
	Let \( \hat A = \hat x \).
	We have
	\begin{align*}
		\qty[\hat H, \hat x] \psi & = \qty[\frac{\hat p^2}{2m} + U(\hat x), \hat x] \psi                                 \\
		                          & = \frac{1}{2m} \qty[\hat p^2, \hat x] \psi                                           \\
		                          & = \frac{1}{2m} \qty( \hat p \qty[\hat p, \hat x] + \qty[\hat p, \hat x] \hat p) \psi \\
		                          & = \frac{-i\hbar}{m}
	\end{align*}
	Hence,
	\[
		\dv{t}\inner{\hat x}_\psi = \frac{i}{\hbar} \inner{\qty[\hat H, \hat x]}_\psi = \frac{\inner{\hat p}_\psi}{m}
	\]
	which aligns with the classical equation
	\[
		\dot x = \frac{p}{m}
	\]
\end{example}
