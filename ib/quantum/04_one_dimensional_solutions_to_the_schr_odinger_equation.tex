\subsection{Stationary states}
\begin{definition}
	With the ansatz \( \psi(x,t) = \chi(x) T(t) \), we have found a particular class of solutions of the time-independent Schr\"odinger equation:
	\[
		\psi(x,t) = \chi(x) e^{-\frac{i E t}{\hbar}}
	\]
	where \( \chi(x) \) are the eigenfunctions of \( \hat H \) with eigenvalue \( E \).
	Such solutions are called stationary states.
\end{definition}
\noindent Note,
\[
	\rho(x,t) = \abs{\psi(x,t)}^2 = \abs{\chi(x)}^2
\]
This explains the naming of the states as `stationary', as their probability density is independent of time.
Now, suppose \( E \) is quantised.
Then, the general solution to the system is
\[
	\psi(x,t) = \sum_{n=1}^N a_n \chi_n(x) e^{-\frac{iE_n t}{\hbar}}
\]
where \( N \) can be finite or infinite.
In principle, we can also have a continuous energy state \( E_\alpha, \alpha \in \mathbb R \).
We can still use the same idea:
\[
	\psi(x,t) = \int_{\Delta \alpha} A(\alpha) \chi_\alpha(x) e^{-\frac{iE_\alpha t}{\hbar}} \dd{\alpha}
\]
Note that \( \abs{a_n}^2 \) and \( A(\alpha) \dd{\alpha} \) give the probability of measuring the particle energy to be \( E_n \) or \( E_\alpha \).

\subsection{Infinite potential well}
We define
\[
	U(x) = \begin{cases}
		0      & \text{for } \abs{x} \leq a \\
		\infty & \text{for } \abs{x} > a
	\end{cases}
\]
For \( \abs{x} > a \), we must have \( \chi(a) = 0 \).
Otherwise, \( \chi \cdot U = \infty \).
This gives us a boundary condition, \( \chi(\pm a) = 0 \).
For \( \abs{x} \leq a \), we seek solutions of the form
\[
	-\frac{\hbar^2}{2m} \chi''(x) = E \chi(x);\quad \chi(\pm a) = 0
\]
Equivalently,
\[
	\chi''(x) + k^2 \chi(x) = 0;\quad k = \sqrt{\frac{2mE}{\hbar^2}}
\]
Since \( E > 0 \),
\[
	\chi(x) = A \sin kx + B \cos kx
\]
Imposing boundary conditions,
\[
	A \sin ka + B \cos ka = 0;\quad A \sin ka - B \cos ka = 0
\]
Suppose \( A = 0 \), giving \( \chi(x) = B \cos kx \).
Then, imposing boundary conditions, \( \chi_n(x) = B \cos k_n x \) where \( k_n = \frac{n \pi}{2a} \), and \( n \) are odd positive integers.
These are even solutions.

Alternatively, suppose \( B = 0 \).
In this case, \( \chi(x) = A \sin kx \).
Thus, \( \chi_n(x) = A \sin k_n x \) where \( k_n = \frac{n \pi}{2a} \), and \( n \) are even non-zero positive integers.
These provide odd solutions.

We can also determine the normalisation constants by defining that the eigenfunctions of the Hamiltonian are normalised to unity.
Thus,
\[
	\int_{-a}^a \abs{\chi_n(x)}^2 = 1 \implies A = B = \sqrt{\frac{1}{a}}
\]
Hence, the general solution is given by the eigenvalues
\[
	E_n = \frac{\hbar^2}{2n} k_n^2 = \frac{\hbar^2 \pi^2 n^2}{2ma^2}
\]
and eigenfunctions
\[
	\chi_n(x) = \sqrt{\frac{1}{a}} \begin{cases}
		\cos(\frac{n \pi x}{2a}) & \text{if } n \text{ odd}  \\
		\sin(\frac{n \pi x}{2a}) & \text{if } n \text{ even}
	\end{cases}
\]
\begin{remark}
	Note that unlike classical mechanics, the ground state energy is not zero.
	Note also that \( \chi_n \) have \( (n+1) \) nodes in which \( \rho(x) = 0 \).
	When \( n \to \infty \), \( \rho_n(x) \) tends to a constant, which is like in classical mechanics.
	Eigenfunctions of the Hamiltonian in this case were either odd or even; we can in fact prove that this is the case in general.
\end{remark}
\begin{proposition}
	If we have a system of non-degenerate eigenstates (\( E_i \neq E_j \)),  then if \( U(x) = U(-x) \) the eigenfunctions of \( \hat H \) must be either odd or even.
\end{proposition}
\begin{proof}
	The time-independent Schr\"odinger equation is invariant under \( x \mapsto -x \) if \( U \) is even.
	Hence, if \( \chi(x) \) is a solution with eigenvalue \( E \), then \( \chi(-x) \) is also a solution.
	Since we have a non-degenerate solution, \( \chi(-x) = \chi(x) \) hence the solutions must be the same up to a normalisation factor.
	For consistency, \( \chi(x) = \chi(-(-x)) = \alpha \chi(-x) = \alpha^2 \chi(x) \).
	Hence \( \alpha = \pm 1 \), so \( \chi \) is either odd or even.
\end{proof}

\subsection{Finite potential well}
We define
\[
	U(x) = \begin{cases}
		0   & \text{for } \abs{x} \leq a \\
		U_0 & \text{for } \abs{x} > a
	\end{cases}
\]
Classically, if \( E < U_0 \), the particle has insufficient energy to escape the well.
We will only consider eigenstates with \( E < U_0 \) here, but we will find that it is possible in quantum mechanics to escape the well with positive probability.
We will search for even functions only, odd functions can be solved independently.
If \( \abs{x} \leq a \),
\[
	-\frac{\hbar^2}{2m} \chi''(x) = E\chi(x)
\]
Equivalently,
\[
	\chi''(x) + k^2 \chi(x) = 0;\quad k = \sqrt{\frac{2mE}{\hbar^2}}
\]
The solution becomes
\[
	\chi(x) = A \sin kx + B \cos kx \implies \chi(x) = B \cos kx
\]
since we are only looking for even solutions.
In the region \( \abs{x} > a \),
\[
	-\frac{\hbar^2}{2m} \chi''(x) + U_0 \chi(x) = E \chi(x)
\]
giving
\[
	\chi''(x) - \overline k^2 \chi(x) = 0;\quad \overline k = \sqrt{\frac{2m(U_0 - E)}{\hbar^2}}
\]
This yields exponential solutions:
\[
	\chi(x) = C e^{\overline k x} + D e^{-\overline k x}
\]
Imposing the normalisability constraints, for \( x > a \) we have \( C = 0 \), and for \( x < -a \) we have \( D = 0 \).
Imposing even parity, \( C = D \) when nonzero.
Thus,
\[
	\chi(x) = \begin{cases}
		C e^{\overline k x}  & x < -a         \\
		B \cos(kx)           & \abs{x} \leq a \\
		C e^{-\overline k x} & x > a
	\end{cases}
\]
Now we must impose continuity of \( \chi(x) \) and its derivative at \( x = \pm a \).
First,
\[
	C e^{-\overline k a} = B \cos(k a)
\]
The other gives
\[
	-\overline k C e^{-\overline k a} = -k B \sin(k a)
\]
From the ratio of both constraints,
\[
	k \tan (ka) = \overline k
\]
From the definition of \( k, \overline k \),
\[
	k^2 + \overline k^2 = \frac{2mU_0}{\hbar^2}
\]
We will define some rescaled variables for convenience: \( \xi = ka \), \( \eta = \overline k a \).
Rewriting,
\[
	\xi \tan \xi = \eta;\quad \xi^2 + \eta^2 = r_0^2;\quad r_0 = \frac{2mU}{\hbar}
\]
This may be solved graphically.
The eigenvalues of the system correspond to the points of intersection between the two equations.
There are always a finite number of possible intersections, regardless of the value of \( r_0 \).
The eigenvalues are
\[
	E_n = \frac{\hbar^2}{2 n a^2} \xi_n^2;\quad \xi \in \qty{\xi_1, \dots, \xi_n};\quad n = 1, \dots, p
\]
When \( U_0 \to \infty \), \( r_0 \to \infty \).
At this point, there are an infinite amount of intersections, so the eigenvalues of the Hamiltonian become that of the infinite well.
Further \( \chi(x) \) tends to the eigenfunctions of the infinite well.
Note that the \( \chi_n(x) \) have some positive region outside the well.
We can use the unused condition above to write \( C \) in terms of \( B \), and then we can use the normalisation condition to find \( B \).

\subsection{Free particles}
A free particle is under no potential.
The time-independent Schr\"odinger equation is
\[
	-\frac{\hbar}{2m} \chi''(x) = E\chi(x)
\]
This has solutions
\[
	\chi_k(x) = A e^{i k x};\quad k = \sqrt{\frac{2mE}{\hbar^2}}
\]
The complete solution, adding \( T(t) \), is thus
\[
	\psi_k(x,t) = \chi_k(x) e^{-i E_k t / \hbar} = A e^{i\qty(kx - \frac{\hbar k^2}{2m} t)}
\]
which are called De Broglie plane waves.
This is not a solution since
\[
	\int_{-\infty}^\infty \abs{\phi_k(x,t)} \dd{x} = \abs{A}^2 \int_{-\infty}^\infty 1 \dd{x}
\]
which diverges.
In general, any non-bound solution is non-normalisable.
This is true since \( \int_{-\infty}^\infty \abs{\chi(x)}^2 \dd{x} < \infty \) requires \( \lim_{R \to \infty} \int_{\abs{x} > R} \abs{\chi(x)} \dd{x} = 0 \).
So, to solve the free particle system, we will build a linear combination of plane waves \( \chi \) to yield a normalisable solution.
This is called the Gaussian wavepacket.
Alternatively, we can simply ignore the problem of normalisability, and change the interpretation of \( \chi_n(x) \).

\subsection{Gaussian wavepacket}
Due to the superposition principle, we can take a continuous linear combination of the \( \psi_k \) functions.
\[
	\psi(x,t) = \int_0^\infty A(k) \psi_k(x,t) \dd{k}
\]
We can construct a suitable \( A(k) \) such that \( \psi \) is normalisable.
Choosing
\[
	A(k) = A_{\text{GP}}(k) = \exp[-\frac{\sigma}{2}(k-k_0)^2];\quad k_0 \in \mathbb R, \sigma \in \mathbb R^+
\]
produces a solution called the Gaussian wavepacket.
Substituting into the above,
\[
	\psi_{\text{GP}}(x,t) = \int_0^\infty \exp[-\frac{\sigma}{2}(k-k_0)^2] \psi_k(x,t) \dd{k} = \int_0^\infty \exp[F(k)] \dd{k};\quad F(k) = -\frac{\sigma}{2}(k-k_0)^2 + ikx - i \frac{\hbar k^2}{2m} t
\]
We can rewrite this as
\[
	F(k) = -\frac{1}{2}\qty(\sigma + \frac{i \hbar t}{m}) k^2 + (k_0 \sigma + ix)k - \frac{\sigma}{2} k_0^2
\]
We define further
\[
	\alpha \equiv \sigma + \frac{i \hbar t}{m};\quad \beta = k_0 \sigma + ix;\quad \delta = -\frac{\sigma}{2} k_0^2
\]
Completing the square,
\[
	F(k) = -\frac{\alpha}{2} \qty(k - \frac{\beta}{\alpha})^2 + \frac{\beta^2}{2\alpha} + \delta
\]
We arrive at the solution
\[
	\psi_{\text{GP}}(x,t) = \exp[\frac{\beta^2}{2\alpha} + \delta] \int_{-\infty}^\infty \exp[-\frac{\alpha}{2}\qty(k - \frac{\beta}{\alpha})^2 ] \dd{k}
\]
Under a change of variables \( \widetilde k = k - \frac{\beta}{\alpha}, u = \Im(\frac{\beta}{\alpha}) \),
\[
	\psi_{\text{GP}}(x,t) = \exp[\frac{\beta^2}{2\alpha} + \delta] \int_{\infty - iu}^{\infty - iu} \exp[-\frac{\alpha}{2} \widetilde k] \dd{\widetilde k}
\]
We arrive at the usual Gaussian integral:
\[
	I(a) = \int_{-\infty}^\infty \exp[-a x^2] \dd{x} = \sqrt{\frac{\pi}{2}}
\]
giving
\[
	\psi_{\text{GP}}(x,t) = \sqrt{\frac{2 \pi}{\alpha}} \exp[\frac{\beta^2}{2\alpha} + \delta] = \sqrt{\frac{2\pi}{\alpha}} \exp[-\frac{\sigma}{2} \frac{\qty(x - \frac{\hbar k_0}{m} t)^2}{\qty(\sigma^2 + \frac{\hbar^2 t^2}{m^2} )} ]
\]
We define \( \overline \psi_{\text{GP}} \) to be the normalised Gaussian wavefunction, so \( \overline \psi_{\text{GP}} = C \psi_{\text{GP}} \).
We can find that
\[
	\rho_{\text{GP}}(x,t) = \abs{\overline \psi_{\text{GP}}(x,t)}^2 = \sqrt{\frac{\sigma}{\pi \qty(\sigma^2 + \frac{\hbar^2 t^2}{m^2})}} \exp[ - \frac{\sigma\qty(x - \frac{\hbar k}{m} t)^2}{\sigma^2 + \frac{\hbar^2 t^2}{m^2}} ]
\]
This is a wavefunction whoes probability density distribution resembles a Gaussian \( e^{-x^2} \) term, with a maximum point at
\[
	\inner{x} = \int_{-\infty}^\infty \psi_{\text{GP}}^\star x \psi_{\text{GP}} \dd{x} = \int_{-\infty}^\infty x \rho_{\text{GP}} \dd{x} = \frac{\hbar k_0}{m} t
\]
and a width of
\[
	\Delta x = \sqrt{\inner{x^2} - \inner{x}^2} = \sqrt{\frac{1}{2} \qty(\sigma + \frac{\hbar^2 t^2}{m^2 \sigma})}
\]
The physical interpretation is that the uncertainty of the particle's position grows with time.
In this case, we can find
\[
	\inner{p} = \int_{-\infty}^\infty \psi_{\text{GP}}^\star i \hbar \pdv{x} \psi_{\text{GP}} \dd{x} = \hbar k_0
\]
which is constant.
The uncertainty in the momentum can be found to be
\[
	\Delta p = \sqrt{\inner{p^2} - \inner{p}^2} = \frac{\hbar}{\sqrt{\frac{1}{2} \qty(\sigma + \frac{\hbar^2 t^2}{m \sigma})}}
\]
Thus,
\[
	\Delta x \Delta p = \frac{\hbar}{2}
\]
We can find for a single plane wave that
\[
	\Delta x = \infty;\quad \Delta p = 0
\]

\subsection{Beam interpretation}
We can choose to ignore the normalisation problem and take the plane waves as the eigenfunctions of the Hamiltonian:
\[
	\chi_k(x) = Ae^{ikx};\quad \psi_k(x,t) = Ae^{ikx}e^{-\frac{i \hbar^2 k^2}{2m} t}
\]
Instead of \( \chi_k(x) \) describing a single particle, we can interpret it as a beam of particles with momentum \( p = \hbar k \) and \( E = \frac{\hbar^2 k^2}{2m} \) with probability density
\[
	\rho_k(x) = \abs{\chi_k(x) e^{-\frac{i\hbar^2 k^2}{2m}t}}^2 = \abs{A}^2
\]
which here is interpreted as a constant average density of particles.
The probability current is given by
\[
	J_k(x,t) = - \frac{i\hbar}{2m} \qty(\psi^\star_k \pdv{\psi_k}{x} - \psi_k \pdv{\psi_k^\star}{x}) = -\frac{i\hbar}{2m} \abs{A}^2 2ik = \abs{A}^2 \frac{\hbar k}{m} = \abs{A}^2 \underbrace{\frac{p}{m}}_{\mathclap{\text{velocity}}}
\]
This is interpreted as the average flux of particles.

\subsection{Scattering states}
We wish to investigate what happens when a particle, or beam of particles, is thrown onto a potential \( U(x) \).
In this case, suppose we have a step function
\[
	U(x) = \begin{cases} U_0 & \text{if } 0 \leq x < a \\
              0   & \text{otherwise}\end{cases}
\]
and a Gaussian wavepacket which is centred at \( x_0 \ll 0 \) moving in the \( +x \) direction, towards the spike in potential.
As \( t \gg 0 \), we end up with a probability density given by two wavepackets; one will be moving left from the spike and one will have cleared the spike and continues moving to the right.
\begin{definition}
	The reflection coefficient \( R \) is
	\[
		R = \lim_{t \to \infty} \int_{-\infty}^0 \abs{\psi_{\text{GP}}(x,t)}^2 \dd{x}
	\]
	which is the probability for the particle to be reflected.
	The transmission coefficient is
	\[
		T = \lim_{t \to \infty} \int_0^\infty \abs{\psi_{\text{GP}}(x,t)}^2 \dd{x}
	\]
	By definition, \( R + T = 1 \).
\end{definition}
In practice, working with Gaussian packets is mathematically challenging (but not impossible).
The beam interpretation, by allowing us to use non-normalisable stationary state wavefunctions, greatly simplifies the computation.

\subsection{Scattering off potential step}
Consider a potential
\[
	U(x) =
	\begin{cases}
		0   & \text{if } x \leq 0 \\
		U_0 & \text{if } x > 0
	\end{cases}
\]
We want to solve
\[
	-\frac{\hbar^2}{2m} \chi''_k(x) + U(x) \chi_k(x) = E\chi_k(x)
\]
We split the problem into two regions: \( x \leq 0, x > 0 \).
For \( x \leq 0 \), the TISE becomes
\[
	\chi''_k(x) + k^2 \chi_k(x) = 0;\quad k = \sqrt{\frac{2mE}{\hbar^2}}
\]
The solution is
\[
	\chi(x) = Ae^{ikx} + Be^{-ikx}
\]
This is a superposition of two beams; the beam of incident particles \( Ae^{ikx} \) and the beam of reflected particles \( Be^{-ikx} \) which are travelling in the opposite direction.
In the region \( x > 0 \), we have
\[
	\chi''_{\overline k}(x) + \overline k^2 \chi_{\overline k}(x) = 0;\quad \overline k = \sqrt{\frac{2m(E-U_0)}{\hbar^2}}
\]
where \( \overline k \) is real if \( E > U_0 \), and \( \overline k \) is pure-imaginary if \( E < U_0 \).
Therefore, for \( E > U_0 \) we have
\[
	\chi_{\overline k}(x) = Ce^{i \overline k x} + De^{-i \overline k x}
\]
which is a beam of particles moving towards the right and an incident beam of particles from the right moving towards the left.
Since no such incident beam exists, we can set \( D = 0 \).
If \( E < U_0 \), the solution is
\[
	\overline k \equiv i \eta \implies \chi_{\overline k}(x) = Ce^{-\eta x} + De^{\eta x}
\]
\( D \neq 0 \) would give infinite values of \( \chi_{\overline k}(x) \) as \( x \to \infty \).
In either case, the eigenfunctions are
\[
	\chi_{k, \overline k}(x) =
	\begin{cases}
		Ae^{ikx} + Be^{-ikx} & x \leq 0 \\
		Ce^{i \overline k x} & x > 0
	\end{cases}
\]
By imposing the boundary conditions, specifically the continuity of \( \chi \), we can determine the constants.
\[
	A + B = C;\quad ikA - ikB = i\overline k C
\]
which gives
\[
	B = \frac{k - \overline k}{k + \overline k} A;\quad C = \frac{2k}{k + \overline k}A
\]
We can view these solutions in terms of particle flux.
\[
	J_k(x,t) = - \frac{i\hbar}{2m} \qty(\psi^\star_k \pdv{\psi_k}{x} - \psi_k \pdv{\psi_k^\star}{x})
\]
If \( E > U_0 \), we find
\[
	J(x,t) =
	\begin{cases}
		\frac{\hbar k}{m} (\abs{A}^2 - \abs{B}^2) & x < 0    \\
		\frac{\hbar \overline k}{m} \abs{C}^2     & x \geq 0
	\end{cases}
\]
The incident flux is \( \frac{\hbar k}{m} \abs{A}^2 \), the reflected flux is \( \frac{\hbar k}{m} \abs{B}^2 \), and the transmitted flux is \( \frac{\hbar k}{m} \abs{C}^2 \).
We can define
\[
	R = \frac{J_{\text{ref}}}{J_{\text{inc}}} = \frac{\abs{B}^2}{\abs{A}^2} = \qty(\frac{k - \overline k}{k + \overline k})^2
\]
We can also define
\[
	T = \frac{J_{\text{trans}}}{J_{\text{inc}}} = \frac{k\abs{C}^2}{\overline k\abs{A}^2} = \frac{4k \overline k}{(k + \overline k)^2}
\]
We can check that our original interpretation makes sense; for example, \( R + T = 1 \), and \( E \to U_0, \overline k \to 0 \) implies \( T \to 0 \), \( R \to 1 \).
If \( E \to \infty \), \( T \to 1 \) and \( R \to 0 \).
If \( E < U_0 \),
\[
	J(x,t) =
	\begin{cases}
		\frac{\hbar k}{m} (\abs{A}^2 + \abs{B}^2) & x < 0    \\
		0                                         & x \geq 0
	\end{cases}
\]
since \( \chi_{\overline k} = \chi_{\overline k}^\star \).
Here, \( T = 0 \) but \( \chi_{\overline k}(x) \neq 0 \).

\subsection{Scattering off a potential barrier}
Consider the potential
\[
	U(x) = \begin{cases}
		0   & x \leq 0, x \geq a \\
		U_0 & 0 < x < a
	\end{cases}
\]
When \( E < U_0 \), we define
\[
	k = \sqrt{\frac{2mE}{\hbar^2}} > 0;\quad \eta = \sqrt{\frac{2m(U_0 - E)}{\hbar^2}} > 0
\]
The solution is then
\[
	\chi(x) = \begin{cases}
		e^{ikx} + Ae^{-ikx}        & x \leq 0  \\
		Be^{-\eta x} + Ce^{\eta x} & 0 < x < a \\
		De^{ikx}                   & x \geq a
	\end{cases}
\]
since we can normalise the incoming flux to one.
The boundary conditions are that \( \chi(x) = \chi'(x) \) are both continuous at \( x = 0, x = a \).
This gives four conditions, which are enough to solve the problem.
\( \chi(x) \) and its derivative at zero give
\[
	1 + A = B + C;\quad ik - ikA = -\eta B + \eta C
\]
and the continuity at \( a \) gives
\[
	B e^{-\eta a} + C e^{\eta a} = D e^{ika};\quad -\eta B e^{-\eta a} + \eta C e^{\eta a} = ikD e^{ika}
\]
Solving the system gives
\[
	D = \frac{-4 i \eta k}{(\eta-ik)^2 \exp[(\eta+ik)a] - (\eta+ik)^2\exp[-(\eta-ik)a]}
\]
The transmitted flux is \( j_{\text{tr}} = \frac{\hbar k}{m} \abs{D}^2 \) and the incident flux is \( j_{\textit{inc}} = \frac{\hbar k}{m} \).
Hence, the transmission coefficient is \( T = \abs{D}^2 \).
This is
\[
	T = \frac{4 k^2 \eta^2}{(k^2+\eta^2)^2 \sinh^2(\eta a) + 4 k^2 \eta^2}
\]
If we take the limit as \( U_0 \gg E \), we have \( \eta a \gg 1 \).
Then
\[
	T \to \frac{16k^2 \eta^2}{(\eta^2 + k^2)^2} \exp[-2\eta a] \propto \exp[-\frac{2a}{k} \sqrt{2m(U_0 - E)}]
\]
So the probability decreases exponentially with the width of the barrier.

\subsection{Harmonic oscillator}
Consider a parabolic potential
\[
	U(x) = \frac{1}{2} kx^2 = \frac{1}{2} m \omega^2 x^2
\]
where \( k \) is an elastic constant and \( \omega = \sqrt{\frac{k}{m}} \) is the angular frequency of the harmonic oscillator.
Classically, we find the solution \( x = A \cos \omega t + B \sin \omega t \).
This gives a continuous energy spectrum.
The TDSE gives
\[
	-\frac{\hbar^2}{2m} \chi''(x) + \frac{1}{2} m\omega^2 x^2 \chi(x) = E \chi(X)
\]
Since this is a bound system, we will have a discrete set of eigenvalues.
The potential is symmetric so the eigenfunctions are odd or even.
We will make the change of variables
\[
	\xi^2 = \frac{m\omega}{\hbar} x^2;\quad \varepsilon = \frac{2E}{\hbar \omega}
\]
which reformulates the TDSE as
\[
	-dv[2]{\chi}{\xi} + \xi^2 \chi = \varepsilon \chi
\]
We will start by considering the solution for \( \varepsilon = 1 \).
In this case, \( E = \frac{\hbar \omega}{2} \).
The solution in this case is
\[
	\chi_0(\xi) = \exp[-\frac{\xi^2}{2}]
\]
So the first eigenfunction, \( \chi_0 \), is known in terms of \( x \), given by
\[
	\chi_0(x) = A \exp[-\frac{m\omega}{2\hbar}x^2];\quad E_0 = \frac{\hbar \omega}{2}
\]
To find the other eigenfunctions, we will take the general form
\[
	\chi(\xi) = f(\xi) \exp[-\frac{\xi^2}{2}]
\]
This works because we know we have a bound solution and \( \chi \) must tend to zero quickly as \( \xi \) tends to infinity, due to the differential equation in terms of \( \xi, \varepsilon \).
Using the above ansatz for \( \chi \) in the Schr\"odinger equation,
\[
	-\dv[2]{f}{\xi} + 2\xi \dv{f}{\xi} + (1-\varepsilon)f = 0
\]
Note that if \( \varepsilon = 1 \), a solution is \( f = 1 \).
We can find a power series solution to this differential equation, with \( \xi = 0 \) as a regular point.
\[
	f(\xi) = \sum_{n=0}^\infty a_n \xi^n
\]
We find
\[
	\xi \dv{f}{\xi} = \sum_{n=0}^\infty n a_n \xi^n;\quad \dv[2]{f}{\xi} = \sum_{n=0}^\infty n(n-1)a_n \xi^{n-2} = \sum_{n=0}^\infty (n+1)(n+2)a_{n+2}\xi^n
\]
Comparing coefficients of \( \xi^n \),
\[
	(n+1)(n+2) a_{n+2} - 2n a_n + (\varepsilon - 1) a_n = 0
\]
Hence,
\[
	a_{n+2} = \frac{2n - \varepsilon + 1}{(n+1)(n+2)} a_n
\]
Since the function must be either even or odd, exactly one of \( a_0 \) and \( a_1 \) must be zero.
\begin{proposition}
	If the series for \( f \) does not terminate, \( \chi \) is not normalisable.
\end{proposition}
\begin{proof}
	Suppose the series does not terminate.
	We will consider the asymptotic behaviour as \( n \to \infty \).
	\[
		\frac{a_{n+2}}{a_n} \to \frac{2}{n}
	\]
	But this is the same asymptotic behaviour as the function \( g(\xi) \) given by
	\[
		g(\xi) = \exp[\xi^2] = \sum_{m=0}^\infty \frac{\xi^{2m}}{m!} = \sum_{n=0}^\infty b_n \xi^n
	\]
	with
	\[
		b_n = \begin{cases}
			\frac{1}{m!} & n = 2m     \\
			0            & n = 2m + 1
		\end{cases}
	\]
	So asymptotically,
	\[
		\frac{b_{n+2}}{b_n} = \frac{\qty(\frac{n}{2})!}{\qty(\frac{n}{2} + 1)!} = \frac{2}{n+2} \to \frac{2}{n}
	\]
	Hence \( \chi \) would have a form asymptotically equal to
	\[
		\chi(\xi) \sim \exp[\frac{\xi^2}{2}]
	\]
	Hence \( \chi(\xi) \) would be not normalisable.
\end{proof}
Hence \( f \) must be a polynomial.
So there exists \( N \) such that \( a_{N+2} = 0 \) and \( a_N \neq 0 \).
So for this value,
\[
	2N - \varepsilon + 1 = 0 \implies \varepsilon = 2N + 1
\]
By the definition of \( \varepsilon \),
\[
	E_N = \qty(N + \frac{1}{2}) \hbar \omega
\]
In particular, \( E_{N+1} - E_N = \hbar \omega \).
The eigenfunctions are
\[
	\chi_N(\xi) = f_N(\xi) \exp[-\frac{\xi^2}{2}]
\]
with the property that
\[
	\chi_N(-\xi) = (-1)^N \chi_N(\xi)
\]
\begin{align*}
	f_0(\xi) & = 1                      \\
	f_1(\xi) & = \xi                    \\
	f_2(\xi) & = 1 - 2 \xi^2            \\
	f_3(\xi) & = \xi - \frac{2}{3}\xi^3 \\
	         & \vdots
\end{align*}
