\subsection{Probability of visits}
Recall that
\[
	V_i = \sum_{n=0}^\infty 1(X_n = i)
\]
and
\[
	f_i = \psub{i}{T_i < \infty}
\]
\begin{lemma}
	For all \( r \in \mathbb N, i \in I \), \( \psub{i}{V_i > r} = f_i^r \).
\end{lemma}
\begin{proof}
	For \( r = 0 \), this is trivially true.
	Now, suppose that the statement is true for \( r \), and we will show that it is true for \( r + 1 \).
	\begin{align*}
		\psub{i}{V_i > r+1} & = \psub{i}{T_i^{(r+1)} < \infty}                                                      \\
		                    & = \psub{i}{T_i^{(r+1)} < \infty, T_i^{(r)} < \infty}                                  \\
		                    & = \psub{i}{T_i^{(r+1)} < \infty \mid T_i^{(r)} < \infty} \psub{i}{T_i^{(r)} < \infty} \\
		                    & = \psub{i}{T_i^{(r+1)} < \infty \mid T_i^{(r)} < \infty} \psub{i}{V_i > r}            \\
		                    & = \psub{i}{T_i^{(r+1)} < \infty \mid T_i^{(r)} < \infty} f_i^r                        \\
		\intertext{By the strong Markov property applied to the stopping time \( T_i^{(r)} \),}
		                    & = \psub{i}{T_i < \infty} f_i^r                                                        \\
		                    & = f_i f_i^r                                                                           \\
		                    & = f_i^{r+1}
	\end{align*}
\end{proof}

\subsection{Duality of transience and recurrence}
\begin{theorem}
	Let \( X \) be a Markov chain with transition matrix \( P \), and let \( i \in I \).
	Then, exactly one of the following is true.
	\begin{enumerate}[(i)]
		\item If \( \psub{i}{T_i < \infty} = 1 \), then \( i \) is recurrent, and
		      \[
			      \sum_{n=0}^\infty p_{ii}(n) = \infty
		      \]
		\item If \( \psub{i}{T_i < \infty} < 1 \), then \( i \) is transient, and
		      \[
			      \sum_{n=0}^\infty p_{ii}(n) < \infty
		      \]
	\end{enumerate}
\end{theorem}
\begin{proof}
	\begin{align*}
		\esub{i}{V_i} & = \esub{i}{\sum_{n=0}^\infty 1(X_n = i)} \\
		              & = \sum_{n=0}^\infty \esub{i}{1(X_n = i)} \\
		              & = \sum_{n=0}^\infty \psub{i}{X_n = i}    \\
		              & = \sum_{n=0}^\infty p_{ii}(n)
	\end{align*}
	First, suppose \( \psub{i}{T_i < \infty} = 1 \).
	Then, for all \( r \), \( \psub{i}{V_i > r} = 1 \), so \( \psub{i}{V_i = \infty} = 1 \).
	Hence, \( i \) is recurrent.
	Further, \( \esub{i}{V_i} = \infty \) so \( \sum_{n=0}^\infty p_{ii}(n) = \infty \).

	Now, if \( f_i < 1 \), by the previous lemma we see that \( \esub{i}{V_i} = \frac{1}{1-f_i} < \infty \) hence \( \psub{i}{V_i < \infty} = 1 \).
	Thus, \( i \) is transient.
	Further, \( \esub{i}{V_i} < \infty \) so \( \sum_{n=0}^\infty p_{ii}(n) < \infty \).
\end{proof}
\begin{theorem}
	Let \( x, y \) be states that communicate.
	Then, either \( x \) and \( y \) are both recurrent, or they are both transient.
\end{theorem}
\begin{proof}
	Suppose \( x \) is recurrent.
	Then, since \( x \) and \( y \) communicate, \( \exists m, \ell \in \mathbb N \) such that
	\[
		p_{xy}(m) > 0;\quad p_{yx}(\ell) > 0
	\]
	Note, \( \sum_n p_{xx}(n) = \infty \).
	Then,
	\[
		p_{yy}(n) \geq \sum_n p_{yy}(n+m+\ell) \geq \sum_n p_{yx}(\ell) p_{xx}(n) p_{xy}(m) \geq p_{yx}(\ell) p_{xy}(m) p_{xx}(n) = \infty
	\]
\end{proof}
\begin{corollary}
	Either all states in a communicating class are recurrent or they are all transient.
\end{corollary}

\subsection{Recurrent communicating classes}
\begin{theorem}
	Any recurrent communicating class is closed.
\end{theorem}
\begin{proof}
	Suppose a communicating class \( C \) is not closed.
	Then there exists \( x \in C \) and \( y \not\in C \) such that \( x \to y \).
	Let \( m \) be such that \( p_{xy}(m) > 0 \).
	If, starting from \( x \), we hit \( y \) which is outside the communicating class, then we can never return to the communicating class (including \( x \)) again.
	In particular,
	\[
		\psub{x}{V_x < \infty} \geq \psub{x}{X_m = y} = p_{xy}(m) > 0
	\]
	Hence \( x \) is not recurrent, which is a contradiction.
\end{proof}
\begin{theorem}
	Any finite closed communicating class is recurrent.
\end{theorem}
\begin{proof}
	Let \( C \) be a finite closed communicating class.
	Let \( x \in C \).
	Then, by the pigeonhole principle, there must exist \( y \in C \) such that
	\[
		\psubx{X_n = y \text{ for infinitely many } n} > 0
	\]
	Since \( x \) and \( y \) communicate, there exists \( m \in \mathbb N \) such that \( p_{yx}(m) > 0 \).
	Now,
	\begin{align*}
		\psub{y}{X_m = y \text{ for infinitely many } n} & \geq \psubx{X_m = x, X_n = y \text{ for infinitely many } n \geq m}                   \\
		                                                 & =\psubx{X_n = y \text{ for infinitely many } n \geq m \mid X_m = x} \psub{y}{X_m = x} \\
		                                                 & =\psubx{X_n = y \text{ for infinitely many } n} \psub{y}{X_m = x} > 0
	\end{align*}
	Thus \( y \) is recurrent.
	Since recurrence is a class property, \( C \) is recurrent.
\end{proof}
\begin{theorem}
	Let \( P \) be irreducible and recurrent.
	Then, for all \( x, y \),
	\[
		\psub{x}{T_y < \infty} = 1
	\]
\end{theorem}
\begin{proof}
	Since \( y \) is recurrent,
	\[
		1 = \psub{y}{X_n = y \text{ for infinitely many } n}
	\]
	Let \( m \) such that \( p_{yx}(m) > 0 \).
	Now,
	\begin{align*}
		1 & = \psub{y}{X_n = y \text{ infinitely often}}                                                       \\
		  & = \sum_z \psub{y}{X_m = z, X_n = y \text{ for infinitely many } n \geq m}                          \\
		  & = \sum_z \psub{y}{X_n = y \text{ for infinitely many } n \geq m \mid X_m = z} \psub{y}{X_m = z}    \\
		  & = \sum_z \psub{z}{X_n = y \text{ for infinitely many } n} \psub{y}{X_m = z}                        \\
		\intertext{By the strong Markov property,}
		  & = \sum_z \psub{z}{T_y < \infty} \psub{y}{X_n = y \text{ for infinitely many } n} \psub{y}{X_m = z} \\
		\intertext{Since \( y \) is recurrent,}
		  & = \sum_z \psub{z}{T_y < \infty} \psub{y}{X_m = z}                                                  \\
		  & = \sum_z \psub{z}{T_y < \infty} p_{yz}(m)
	\end{align*}
	Since \( p_{yz}(m) > 0 \) and \( \sum_z p_{yz}(m) = 1 \), \( \psub{x}{T_y < \infty} = 1 \).
\end{proof}
