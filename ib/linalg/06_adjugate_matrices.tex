\subsection{Column and row expansions}
Let \( A \in M_n(F) \) with column vectors \( A^{(i)} \).
We know that
\[
	\det(A^{(1)}, \dots, A^{(j)}, \dots, A^{(k)}, \dots, A^{(n)}) = -\det(A^{(1)}, \dots, A^{(k)}, \dots, A^{(j)}, \dots, A^{(n)})
\]
Using the fact that \( \det A = \det A^\transpose \) we can similarly see that swapping two rows will invert the sign of the determinant.
\begin{remark}
	We could have proven all of the properties of the determinant above by using the decomposition of \( A \) into elementary matrices.
\end{remark}
\begin{definition}
	Let \( A \in M_n(F) \).
	Let \( i, j \in \qty{1, \dots, n} \).
	We define the \textit{minor} \( A_{\widehat{ij}} \in M_{n-1}(F) \) to be the matrix obtained by removing the \( i \)th row and the \( j \)th column.
\end{definition}
\begin{lemma}
	Let \( A \in M_n(F) \).
	\begin{enumerate}[(i)]
		\item Let \( j \in \qty{1, \dots, n} \).
		      The determinant of \( A \) is given by the \textit{column expansion with respect to the \( j \)th column}:
		      \[
			      \det A = \sum_{i=1}^n (-1)^{i+j} a_{ij} \det A_{\widehat{ij}}
		      \]
		\item Let \( i \in \qty{1, \dots, n} \).
		      The same determinant is also given by the \textit{row expansion with respect to the \( i \)th row}:
		      \[
			      \det A = \sum_{j=1}^n (-1)^{i+j} a_{ij} \det A_{\widehat{ij}}
		      \]
	\end{enumerate}
	This is a process of reducing the computation of \( n \times n \) determinants to \( (n-1) \times (n-1) \) determinants.
\end{lemma}
\begin{proof}
	We will prove case (i), the column expansion with respect to the \( j \)th column.
	Then (ii) will follow from the transpose of the matrix.
	Let \( j \in \qty{1, \dots, n} \).
	We can write \( A^{(j)} = \sum_{i=1}^n a_{ij} e_i \) where the \( e_i \) are the canonical basis.
	Then, by swapping rows and columns,
	\begin{align*}
		\det A & = \det\qty(A^{(1)}, \dots, \sum_{i=1}^n a_{ij} e_i, \dots, A^{(n)})                                    \\
		       & = \sum_{i=1}^n a_{ij} \det\qty(A^{(1)}, \dots, e_i, \dots, A^{(n)})                                    \\
		       & = \sum_{i=1}^n a_{ij} (-1)^{j-1} \det\qty(e_i, A^{(1)}, \dots, A^{(n)})                                \\
		       & = \sum_{i=1}^n a_{ij} (-1)^{j-1} (-1)^{i-1} \det\qty(e_1, \overline A^{(1)}, \dots, \overline A^{(n)})
	\end{align*}
	This has brought the matrix into block form, where there is an element of value 1 in the top left, and the matrix \( A_{\widehat{ij}} \) in the bottom right.
	The bottom left block is entirely zeroes.
	Hence,
	\[
		\det A = \sum_{i=1}^n (-1)^{i+j} a_{ij} \det A_{\widehat{ij}}
	\]
	as required.
\end{proof}
\begin{remark}
	We have proven that
	\[
		\det (A^{(1)}, \dots, e_i, \dots, A^{(n)}) = (-1)^{i+j} \det A_{\widehat{ij}}
	\]
\end{remark}

\subsection{Adjugates}
\begin{definition}
	Let \( A \in M_n(F) \).
	The \textit{adjugate matrix} of \( A \), denoted \( \adj A \), is the \( n \times n \) matrix given by
	\[
		(\adj A)_{ij} = (-1)^{i+j} \det A_{\widehat{ji}}
	\]
	Hence,
	\[
		\det (A^{(1)}, \dots, e_i, \dots, A^{(n)}) = (\adj A)_{ji}
	\]
\end{definition}
\begin{theorem}
	Let \( A \in M_n(F) \).
	Then
	\[
		(\adj A) A = (\det A) I
	\]
	In particular, when \( A \) is invertible,
	\[
		A^{-1} = \frac{\adj A}{\det A}
	\]
\end{theorem}
\begin{proof}
	We have
	\[
		\det A = \sum_{i=1}^n (-1)^{i+j} a_{ij} \det A_{\widehat{ij}}
	\]
	Hence,
	\[
		\det A = \sum_{i=1}^n (\adj A)_{ji} a_{ij} = ((\adj A) A)_{jj}
	\]
	So the diagonal terms match.
	Off the diagonal,
	\[
		0 = \det(A^{(1)}, \dots, \underbrace{A^{(k)}}_{\mathclap{j\text{th position}}}, \dots, A^{(k)}, \dots, A^{(n)})
	\]
	By linearity,
	\begin{align*}
		0 & = \det\qty(A^{(1)}, \dots, \underbrace{\sum_{i=1}^n a_{ik} e_i}_{\mathclap{j\text{th position}}}, \dots, A^{(k)}, \dots, A^{(n)}) \\
		  & = \sum_{i=1}^n a_{ik} \det\qty(A^{(1)}, \dots, \underbrace{e_i}_{\mathclap{j\text{th position}}}, \dots, A^{(k)}, \dots, A^{(n)}) \\
		  & = \sum_{i=1}^n a_{ik} (\adj A)_{ji}                                                                                               \\
		  & = ((\adj A) A)_{jk}
	\end{align*}
\end{proof}

\subsection{Cramer's rule}
\begin{proposition}
	Let \( A \) be an invertible square matrix of dimension \( n \).
	Let \( b \in F^n \).
	Then the unique solution to \( Ax = b \) is given by
	\[
		x_i = \frac{1}{\det A} \det (A_{\widehat{ib}})
	\]
	where \( A_{\widehat{ib}} \) is obtained by replacing the \( i\)th column of \( A \) by \( b \).
	This is an algorithm to compute \( x \), avoiding the computation of \( A^{-1} \).
\end{proposition}
\begin{proof}
	Let \( A \) be invertible.
	Then there exists a unique \( x \in F^n \) such that \( Ax = b \).
	Then, since the determinant is alternating,
	\begin{align*}
		\det (A_{\widehat{ib}}) & = \det(A^{(1)}, \dots, A^{(i-1)}, b, A^{(i+1)}, \dots, A^{(n)})                            \\
		                        & = \det\qty(A^{(1)}, \dots, A^{(i-1)}, \sum_{j=1}^n x_j A^{(j)}, A^{(i+1)}, \dots, A^{(n)}) \\
		                        & = \det\qty(A^{(1)}, \dots, A^{(i-1)}, x_i A^{(i)}, A^{(i+1)}, \dots, A^{(n)})              \\
		                        & = x_i \det A
	\end{align*}
	So the formula works.
\end{proof}
