\subsection{Definition}
\begin{definition}
	Let \( V \) be a vector space over \( \mathbb R \) or \( \mathbb C \).
	A \textit{scalar product} or \textit{inner product} is a positive-definite symmetric (respectively Hermitian) bilinear form \( \phi \) on \( V \).
	We write
	\[
		\phi(u,v) = \inner{u,v}
	\]
	\( V \), when equipped with this inner product, is called a real (respectively complex) \textit{inner product space}.
\end{definition}
\begin{example}
	In \( \mathbb C^n \), we define
	\[
		\inner{x,y} = \sum_{i=1}^n x_i \overline y_i
	\]
\end{example}
\begin{example}
	Let \( V = C^0([0,1], \mathbb C) \).
	Then we can define
	\[
		\inner{f,g} = \int_0^1 f(t) \overline g(t) \dd{t}
	\]
	This is the \( L^2 \) scalar product.
\end{example}
\begin{example}
	Let \( \omega \colon [0,1] \colon \mathbb R^\star_+ \) where \( \mathbb R^\star_+ = \mathbb R_+ \setminus \qty{0} \) and define
	\[
		\inner{f,g} = \int_0^1 f(t) \overline g(t) w(t) \dd{t}
	\]
\end{example}
\begin{remark}
	Typically it suffices to check \( \inner{u,u} = 0 \implies u = 0 \) since linearity and positivity are usually trivial.
\end{remark}
\begin{definition}
	Let \( V \) be an inner product space.
	Then for \( v \in V \), the \textit{norm} of \( v \) induced by the inner product is defined by
	\[
		\norm{v} = \qty(\inner{v,v})^{1/2}
	\]
	This is real, and positive if \( v \neq 0 \).
\end{definition}

\subsection{Cauchy-Schwarz inequality}
\begin{lemma}
	For an inner product space,
	\[
		\abs{\inner{u,v}} \leq \norm{a} \cdot \norm{b}
	\]
\end{lemma}
\begin{proof}
	Let \( t \in F \).
	Then,
	\[
		0 \leq \norm{t u - v} = \inner{tu - v, tu - v} = t \overline t \inner{u,u} - u \inner{u,v} - \overline t \inner{v,u} + \norm{v}^2
	\]
	Since the inner product is Hermitian,
	\[
		0 \leq \abs{t}^2 \norm{u}^2 + \norm{v}^2 - 2 \Re(t \inner{u,v})
	\]
	By choosing
	\[
		t = \frac{\overline{\inner{u,v}}}{\norm{u}^2}
	\]
	we have
	\[
		0 \leq \frac{\abs{\inner{u,v}}^2}{\norm{u}^2} + \norm{v}^2 - 2 \Re(\frac{\abs{\inner{u,v}}^2}{\norm{u}^2})
	\]
	Since the term under the real part operator is real, the result holds.
\end{proof}
\noindent Note that equality implies collinearity in the Cauchy-Schwarz inequality.
\begin{corollary}[triangle inequality]
	In an inner product space,
	\[
		\norm{u+v} \leq \norm{u} + \norm{v}
	\]
\end{corollary}
\begin{proof}
	We have
	\[
		\norm{u+v}^2 = \inner{u+v, u+v} = \norm{u^2} + 2 \Re(\inner{u,v}) + \norm{v}^2 \leq \norm{u^2} + \norm{v}^2 + 2 \norm{u} \cdot \norm{v} = (\norm{u} + \norm{v})^2
	\]
\end{proof}
\begin{remark}
	Any inner product induces a norm, but not all norms derive from scalar products.
\end{remark}

\subsection{Orthogonal and orthonormal sets}
\begin{definition}
	A set \( (e_1, \dots, e_k) \) of vectors of \( V \) is said to be \textit{orthogonal} if \( \inner{e_i, e_j} = 0 \) for all \( i \neq j \).
	The set is said to be \textit{orthonormal} if it is orthogonal and \( \norm{e_i} = 1 \) for all \( i \).
	In this case, \( \inner{e_i, e_j} = \delta_{ij} \).
\end{definition}
\begin{lemma}
	If \( (e_1, \dots, e_k) \) are orthogonal and non-zero, then they are linearly independent.
	Further, let \( v \in \genset{\qty{e_i}} \).
	Then,
	\[
		v = \sum_{j=1}^k \lambda_j e_j \implies \lambda_j = \frac{\inner{v, e_j}}{\norm{e_j}^2}
	\]
\end{lemma}
\begin{proof}
	Suppose
	\[
		\sum_{i=1}^k \lambda_i e_i = 0
	\]
	Then,
	\[
		0 = \inner{\sum_{i=1}^k \lambda_i, e_j} \implies \sum_{i=1}^k \lambda_i \inner{e_i, e_j}
	\]
	Thus \( \lambda_j = 0 \) for all \( j \).
	Further, for \( v \) in the span of these vectors,
	\[
		\inner{v, e_j} = \sum_{i=1}^k \lambda_i \inner{e_i, e_j} = \lambda_j \norm{e_j}^2
	\]
\end{proof}

\subsection{Parseval's identity}
\begin{corollary}
	Let \( V \) be a finite-dimensional inner product space over \( F \).
	Let \( (e_1, \dots, e_n) \) be an orthonormal basis.
	Then, for any vectors \( u, v \in V \), we have
	\[
		\inner{u, v} = \sum_{i=1}^n \inner{u, e_i} \overline{\inner{v, e_i}}
	\]
	Hence,
	\[
		\norm{u}^2 = \sum_{i=1}^n \abs{\inner{u,e_i}}^2
	\]
\end{corollary}
\begin{proof}
	By orthonormality,
	\[
		u = \sum_{i=1}^n \inner{u, e_i} e_i;\quad v = \sum_{i=1}^n \inner{v, e_i} e_i
	\]
	Hence, by sesquilinearity,
	\[
		\inner{u,v} = \sum_{i=1}^n \inner{u, e_i} \overline{\inner{v, e_i}}
	\]
	By taking \( u = v \) we find
	\[
		\norm{u}^2 = \inner{u,u} = \sum_{i=1}^n \abs{\inner{u,e_i}}^2
	\]
\end{proof}

\subsection{Gram-Schmidt orthogonalisation process}
\begin{theorem}
	Let \( V \) be an inner product space.
	Let \( (v_i)_{i \in I} \) be a linearly independent family of vectors such that \( I \) is countable.
	Then there exists a family \( (e_i)_{i \in I} \) of orthonormal vectors such that for all \( k \geq 1 \),
	\[
		\genset{v_1, \dots, v_k} = \genset{e_1, \dots, e_k}
	\]
\end{theorem}
\begin{proof}
	This proof is an explicit algorithm to compute the family \( (e_i) \), which will be computed by induction on \( k \).
	For \( k = 1 \), take \( e_1 = \frac{v_1}{\norm{v_1}} \).
	Inductively, suppose \( (e_1, \dots, e_k) \) satisfy the conditions as above.
	Then we will find a valid \( e_{k+1} \).
	We define
	\[
		e_{k+1}' = v_{k+1} - \sum_{i=1}^k \inner{v_{k+1}, e_i} e_i
	\]
	This ensures that the inner product between \( e_{k+1}' \) and any basis vector \( e_j \) is zero, while maintaining the same span.
	Suppose \( e_{k+1}' = 0 \).
	Then, \( v_{k+1} \in \genset{e_1, \dots, e_k} = \genset{v_1, \dots, v_k} \) which contradicts the fact that the family is free.
	Thus,
	\[
		e_{k+1} = \frac{e_{k+1}'}{\norm{e_{k+1}'}}
	\]
	satisfies the requirements.
\end{proof}
\begin{corollary}
	In finite-dimensional inner product spaces, there always exists an orthonormal basis.
	In particular, any orthonormal set of vectors can be extended into an orthonormal basis.
\end{corollary}
\begin{remark}
	Let \( A \in M_n(\mathbb R) \) be a real-valued (or complex-valued) matrix.
	Then, the column vectors of \( A \) are orthogonal if \( A^\transpose A = I \) (or \( A^\transpose \overline A = I \) in the complex-valued case).
\end{remark}

\subsection{Orthogonality of matrices}
\begin{definition}
	A matrix \( A \in M_n(\mathbb R) \) is \textit{orthogonal} if \( A^\transpose A = I \), hence \( A^\transpose = A^{-1} \).
	A matrix \( A \in M_n(\mathbb C) \) is \textit{unitary} if \( A^\transpose \overline A = I \), hence \( A^\dagger = A^{-1} \).
\end{definition}
\begin{proposition}
	Let \( A \) be a square, non-singular, real-valued (or complex-valued) matrix.
	Then \( A \) can be written as \( A = RT \) where \( T \) is upper triangular and \( R \) is orthogonal (or respectively unitary).
\end{proposition}
\begin{proof}
	We apply the Gram-Schmidt process to the column vectors of the matrix.
	This gives us an orthonormal set of vectors, which gives an upper triangular matrix in this new basis.
\end{proof}

\subsection{Orthogonal complement and projection}
\begin{definition}
	Let \( V \) be an inner product space.
	Let \( V_1, V_2 \leq V \).
	Then we say that \( V \) is the \textit{orthogonal direct sum} of \( V_1 \) and \( V_2 \) if \( V = V_1 \oplus V_2 \) and for all vectors \( v_1\in V_1, v_2\in V_2 \) we have \( \inner{v_1, v_2} = 0 \).
	When this holds, we write \( V = V_1 \overset{\perp}{\oplus} V_2 \).
\end{definition}
\begin{remark}
	If for all vectors \( v_1, v_2 \) we have \( \inner{v_1, v_2} = 0 \), then \( v \in V_1 \cap V_2 \implies \norm{v}^2 = 0 \implies v = 0 \).
	Hence the sum is always direct if the subspaces are orthogonal.
\end{remark}
\begin{definition}
	Let \( V \) be an inner product space and let \( W \leq V \).
	We define the \textit{orthogonal} of \( W \) to be
	\[
		W^\perp = \qty{v \in V \colon \forall w \in W, \inner{v,w} = 0}
	\]
\end{definition}
\begin{lemma}
	For any inner product space \( V \) and any subspace \( W \leq V \), we have \( V = W \overset{\perp}{\oplus} W^\perp \).
\end{lemma}
\begin{proof}
	First note that \( W^\perp \leq V \).
	Then, if \( w \in W \), \( w \in W^\perp \), we have
	\[
		\norm{w}^2 = \inner{w, w} = 0
	\]
	since they are orthogonal, so the vector subspaces intersect only in the zero vector.
	Now, we need to show \( V = W + W^\perp \).
	Let \( (e_1, \dots, e_k) \) be an orthonormal basis of \( W \) and extend it into \( (e_1, \dots, e_k, e_{k+1}, \dots, e_n) \) which can be made orthonormal.
	Then, \( (e_{k+1}, \dots, e_n) \) are elements of \( W^\perp \) and form a basis.
\end{proof}

\subsection{Projection maps}
\begin{definition}
	Suppose \( V = U \oplus W \), so \( U \) is a complement of \( W \) in \( V \).
	Then, we define \( \pi \colon V \to W \) which maps \( v = u + w \) to \( w \).
	This is well defined, since the sum is direct.
	\( \pi \) is linear, and \( \pi^2 = \pi \).
	We say that \( \pi \) is the \textit{projection} operator onto \( W \).
\end{definition}
\begin{remark}
	The map \( \iota - \pi \) is the projection onto \( U \), where \( \iota \) is the identity map.
\end{remark}
\begin{lemma}
	Let \( V \) be an inner product space.
	Let \( W \leq V \) be a finite-dimensional subspace.
	Let \( (e_1, \dots, e_k) \) be an orthonormal basis for \( W \).
	Then,
	\begin{enumerate}[(i)]
		\item \( \pi(v) = \sum_{i=1}^k \inner{v, e_i} e_i \); and
		\item for all \( v \in V, w in W \), \( \norm{v - \pi(v)} \leq \norm{v - w} \) with equality if and only if \( w = \pi(v) \), hence \( \pi(v) \) is the point in \( W \) closest to \( v \).
	\end{enumerate}
\end{lemma}
\begin{proof}
	We define \( \pi(v) = \sum_{i=1}^k \inner{v, e_i} e_i \).
	Since \( W = \genset{\qty{e_k}} \), \( \pi(v) \in W \) for all \( v \in V \).
	Then, \( v = (v - \pi(v)) + \pi(v) \) has a term in \( W \).
	We claim that the remaining term is in the orthogonal; \( v - \pi(v) \in W^\perp \).
	Indeed, we must show \( \inner{v - \pi(v), w} = 0 \) for all \( w \in W \).
	Equivalently, \( \inner{v - \pi(v), e_i} = 0 \) for all basis vectors \( e_i \) of \( W \).
	We can explicitly compute
	\[
		\inner{v - \pi(v), e_j} = \inner{v, e_j} - \inner{\sum_{i=1}^k \inner{v, e_i} e_i, e_j} = \inner{v, e_j} - \sum_{i=1}^k \inner{v, e_i} \inner{e_i, e_j} = \inner{v, e_j} - \inner{v, e_j} = 0
	\]
	Hence, \( v = (v - \pi(v)) + \pi(v) \) is a decomposition into \( W \) and \( W^\perp \).
	Since \( W \cap W^\perp = \qty{0} \), we have \( V = W \overset{\perp}{\oplus} W^\perp \).
	For the second part, let \( v \in V \), \( w \in W \), and we compute
	\[
		\norm{v - w}^2 = \norm{\underbrace{v - \pi(v)}_{\in W^\perp} + \underbrace{\pi(v) - w}_{\in W}}^2 = \norm{v - \pi(v)}^2 + \norm{\pi(v) - w}^2 \geq \norm{v - \pi(v)}^2
	\]
	with equality if and only if \( w = \pi(v) \).
\end{proof}

\subsection{Adjoint maps}
\begin{definition}
	Let \( V, W \) be finite-dimensional inner product spaces.
	Let \( \alpha \in L(V, W) \).
	Then there exists a unique linear map \( \alpha^\star \colon W \to V \) such that for all \( v, w \in V, W \),
	\[
		\inner{\alpha(v), w} = \inner{v, \alpha^\star(w)}
	\]
	Moreover, if \( B \) is an orthonormal basis of \( V \), and \( C \) is an orthonormal basis of \( W \), then
	\[
		[\alpha^\star]_{C, B} = \qty(\overline{[\alpha]_{B, C}})^\transpose
	\]
\end{definition}
\begin{proof}
	Let \( B = (v_1, \dots, v_n) \) and \( C = (w_1, \dots, w_m) \) and \( A = [\alpha]_{B, C} = (a_{ij}) \).
	To check existence, we define \( [\alpha^\star]_{C, B} = \overline{A}^\transpose = (c_{ij}) \) and explicitly check the definition.
	By orthogonality,
	\[
		\inner{\alpha\qty(\sum \lambda_i v_i), \sum \mu_j w_j} = \inner{\sum_{i,k} \lambda_i a_{ki} w_k, \sum_j \mu_j w_j} = \sum_{i,j} \lambda_i a_{ji} \overline{\mu_j}
	\]
	Then,
	\[
		\inner{\sum \lambda_i v_i, \alpha^\star\qty(\sum \mu_j w_j)} = \inner{\sum_i \lambda_i v_i, \sum_{j,k} \mu_j c_{kj} v_k} = \sum_{i,j} \lambda_i \overline{c_{ij}} \overline{\mu_j}
	\]
	So equality requires \( \overline{c_{ij}} = a_{ji} \).
	Uniqueness follows from the above; the expansions are equivalent for any vector if and only if \( \overline{c_{ij}} = a_{ji} \).
\end{proof}
\begin{remark}
	The same notation, \( \alpha^\star \), is used for the adjoint as just defined, and the dual map as defined before.
	If \( V, W \) are real product inner spaces and \( \alpha \in L(V,W) \), we define \( \psi \colon V \to V^\star \) such that \( \psi(v)(x) = \inner{x,v} \) and similarly for \( W \).
	Then we can check that the adjoint for \( \alpha \) is given by the composition of \( \psi \) from \( V \to V^\star \), then applying the dual, then applying the inverse of \( \psi \) for \( W \).
\end{remark}

\subsection{Self-adjoint and isometric maps}
\begin{definition}
	Let \( V \) be a finite-dimensional inner product space, and \( \alpha \) be an endomorphism of \( V \).
	Let \( \alpha^\star \in L(V) \) be the adjoint map.
	Then,
	\begin{enumerate}[(i)]
		\item the condition \( \inner{\alpha v, w} = \inner{v, \alpha w} \) is equivalent to the condition \( \alpha = \alpha^\star \), and such an \( \alpha \) is called \textit{self-adjoint} (for \( \mathbb R \) we call such endomoprhisms \textit{symmetric}, and for \( \mathbb C \) we call such endomorphisms Hermitian);
		\item the condition \( \inner{\alpha v, \alpha w} = \inner{v, w} \) is equivalent to the condition \( \alpha^\star = \alpha^{-1} \), and such an \( \alpha \) is called an \textit{isometry} (for \( \mathbb R \) it is called \textit{orthogonal}, and for \( \mathbb C \) it is called \textit{unitary}).
	\end{enumerate}
\end{definition}
\begin{proposition}
	The conditions for isometries defined as above are equivalent.
\end{proposition}
\begin{proof}
	Suppose \( \inner{\alpha v, \alpha w} = \inner{v,w} \).
	Then for \( v = w \), we find \( \norm{\alpha v}^2 = \norm{v}^2 \), so \( \alpha \) preserves the norm.
	In particular, this implies \( \ker \alpha = \qty{0} \).
	Since \( \alpha \) is an endomorphism and \( V \) is finite-dimensional, \( \alpha \) is bijective.
	Then for all \( v, w \in V \),
	\[
		\inner{v, \alpha^\star(w)} = \inner{\alpha v, w} = \inner{\alpha v, \alpha\qty(\alpha^{-1}(w))} = \inner{v, \alpha^{-1}(w)}
	\]
	Hence \( \alpha^\star = \alpha^{-1} \).
	Conversely, if \( \alpha^\star = \alpha^{-1} \) we have
	\[
		\inner{\alpha v, \alpha w} = \inner{v, \alpha^\star(\alpha w)} = \inner{v, w}
	\]
	as required.
\end{proof}
\begin{remark}
	Using the polarisation identity, we can show that \( \alpha \) is isometric if and only if for all \( v \in V \), \( \norm{\alpha(v)} = \norm{v} \).
\end{remark}
\begin{lemma}
	Let \( V \) be a finite-dimensional real (or complex) inner product space.
	Then for \( \alpha \in L(V) \),
	\begin{enumerate}[(i)]
		\item \( \alpha \) is self-adjoint if and only if for all orthonormal bases \( B \) of \( V \), we have \( [\alpha]_B \) is symmetric (or Hermitian);
		\item \( \alpha \) is an isometry if and only if for all orthonormal bases \( B \) of \( V \), we have \( [\alpha]_B \) is orthogonal (or unitary).
	\end{enumerate}
\end{lemma}
\begin{proof}
	Let \( B \) be an orthonormal basis for \( V \).
	Then we know \( [\alpha^\star]_B = [\alpha]_B^\dagger \).
	We can then check that \( [\alpha]_B^\dagger = [\alpha]_B \) and \( [\alpha]_B^\dagger = [\alpha]_B^{-1} \) respectively.
\end{proof}
\begin{definition}
	For \( F = \mathbb R \), we define the \textit{orthogonal group} of \( V \) by
	\[
		O(V) = \qty{ v \in L(V) \colon \alpha \text{ is an isometry} }
	\]
	Note that \( O(V) \) is bijective with the set of orthogonal bases of \( V \).
	For \( F = \mathbb C \), we define the \textit{unitary group} of \( V \) by
	\[
		U(V) = \qty{ v \in L(V) \colon \alpha \text{ is an isometry} }
	\]
	Again, note that \( U(V) \) is bijective with the set of orthogonal bases of \( V \).
\end{definition}

\subsection{Spectral theory for self-adjoint maps}
Spectral theory is the study of the spectrum of operators.
Recall that in finite-dimensional inner product spaces \( V, W \), \( \alpha \in L(V, W) \) yields the adjoint \( \alpha^\star \in L(W, V) \) such that for all \( v \in V, w \in W \), we have \( \inner{\alpha(v), w} = \inner{v, \alpha^\star(w)} \).
\begin{lemma}
	Let \( V \) be a finite-dimensional inner product space.
	Let \( \alpha \in L(V) \) be a self-adjoint endomorphism.
	Then \( \alpha \) has real eigenvalues, and eigenvectors of \( \alpha \) with respect to different eigenvalues are orthogonal.
\end{lemma}
\begin{proof}
	Suppose \( \lambda \in \mathbb C \), \( v \in V \) non-zero such that \( \alpha(v) = \lambda v \).
	Then, \( \inner{\lambda v, v} = \lambda \norm{v}^2 \) and also
	\[
		\inner{\alpha v, v} = \inner{v, \alpha v} = \inner{v, \lambda v} = \overline{\lambda} \norm{v}^2
	\]
	Hence \( \lambda = \overline{\lambda} \) since \( v \neq 0 \).
	Now, suppose \( \mu \neq \lambda \) and \( w \in V \) non-zero such that \( \alpha(w) = \mu w \).
	Then,
	\[
		\lambda \inner{v, w} = \inner{\alpha v, w} = \inner{v, \alpha w} = \overline{\mu} \inner{v, w} = \mu \inner{v,w}
	\]
	So if \( \lambda \neq \mu \) we must have \( \inner{v,w} = 0 \).
\end{proof}
\begin{theorem}[spectral theorem for self-adjoint maps]
	Let \( V \) be a finite-dimensional inner product space.
	Let \( \alpha \in L(V) \) be self-adjoint.
	Then \( V \) has an orthonormal basis of eigenvectors of \( \alpha \).
	Hence \( \alpha \) is diagonalisable in an orthonormal basis.
\end{theorem}
\begin{proof}
	We will consider induction on the dimension of \( V \).
	Suppose \( A = [\alpha]_B \) with respect to the fundamental basis \( B \).
	By the fundamental theorem of algebra, we know that \( \chi_A(\lambda) \) has a (complex) root.
	But since \( \lambda \) is an eigenvalue of \( \alpha \) and \( \alpha \) is self-adjoint, \( \lambda \in \mathbb R \).
	Now, we choose an eigenvector \( v_1 = V \setminus \qty{0} \) such that \( \alpha(v_1) = \lambda v_1 \).
	We can set \( \norm{v_1} = 1 \) by linearity.
	Let \( U = \genset{v_1}^\perp \leq V \).
	We then observe that \( U \) is stable by \( \alpha \); \( \alpha(U) \leq U \).
	Indeed, let \( u \in U \).
	Then \( \inner{\alpha(u), v_1} = \inner{u, \alpha(v_1)} = \lambda \inner{u, v_1} = 0 \) by orthogonality.
	Hence \( \alpha(u) \in U \).
	We can then restrict \( \alpha \) to the domain \( U \), and by induction we can then choose an orthonormal basis of eigenvectors for \( U \).
	Since \( V = \genset{v_1} \overset{\perp}{\oplus} U \) we have an orthonormal basis of eigenvectors for \( V \) when including \( v_1 \).
\end{proof}
\begin{corollary}
	Let \( V \) be a finite-dimensional inner product space.
	Let \( \alpha \in L(V) \) be self-adjoint.
	Then \( V \) is the orthogonal direct sum of the eigenspaces of \( \alpha \).
\end{corollary}

\subsection{Spectral theory for unitary maps}
\begin{lemma}
	Let \( V \) be a complex inner product space.
	Let \( \alpha \) be unitary, so \( \alpha^\star = \alpha^{-1} \).
	Then all eigenvalues of \( \alpha \) have unit norm.
	Eigenvectors corresponding to different eigenvalues are orthogonal.
\end{lemma}
\begin{proof}
	Let \( \lambda \in \mathbb C \), \( v \in V \setminus \qty{0} \) such that \( \alpha(v) = \lambda v \).
	First, \( \lambda \neq 0 \) since \( \alpha \) is invertible, and in particular \( \ker \alpha = \qty{0} \).
	Since \( v = \lambda \alpha^{-1}(v) \), we can compute
	\[
		\lambda \inner{v,v} = \inner{\lambda v, v} = \inner{\alpha v, v} = \inner{v, \alpha^{-1} v} = \inner{v, \frac{1}{\lambda} v} = \frac{1}{\overline \lambda} \inner{v, v}
	\]
	Hence \( (\lambda \overline \lambda - 1) \norm{v}^2 = 0 \) giving \( \abs{\lambda} = 1 \).
	Further, suppose \( \mu \in \mathbb C \) and \( w \in V \setminus \qty{0} \) such that \( \alpha(w) = \mu w, \lambda \neq \mu \).
	Then
	\[
		\lambda \inner{v,w} = \inner{\lambda v, w} = \inner{\alpha v, w} = \inner{v, \alpha^{-1} w} = \inner{v, \frac{1}{\mu} w} = \frac{1}{\overline \mu} \inner{v,w} = \mu \inner{v,w}
	\]
	since \( \mu \overline \mu = 1 \).
\end{proof}
\begin{theorem}[spectral theorem for unitary maps]
	Let \( V \) be a finite-dimensional complex inner product space.
	Let \( \alpha \in L(V) \) be unitary.
	Then \( V \) has an orthonormal basis of eigenvectors of \( \alpha \).
	Hence \( \alpha \) is diagonalisable in an orthonormal basis.
\end{theorem}
\begin{proof}
	Let \( A = [\alpha]_B \) where \( B \) is an orthonormal basis.
	Then \( \chi_A(\lambda) \) has a complex root \( \lambda \).
	As before, let \( v_1 \neq 0 \) such that \( \alpha(v_1) = \lambda v_1 \) and \( \norm{v_1} = 1 \).
	Let \( U = \genset{v_1}^\perp \), and we claim that \( \alpha(U) = U \).
	Indeed, let \( u \in U \), and we find
	\[
		\inner{\alpha(u), v_1} = \inner{u, \alpha^{-1}(v_1)} = \inner{u, \frac{1}{\lambda} v_1} = \frac{1}{\overline \lambda} \inner{u,v_1}
	\]
	Since \( \inner{u, v_1} = 0 \), we have \( \alpha(u) \in U \).
	Hence, \( \alpha \) restricted to \( U \) is a unitary endomorphism of \( U \).
	By induction we have an orthonormal basis of eigenvectors of \( \alpha \) for \( U \) and hence for \( V \).
\end{proof}
\begin{remark}
	We used the fact that the field is complex to find an eigenvalue.
	In general, a real-valued orthonormal matrix \( A \) giving \( A A^\transpose = I \) cannot be diagonalised over \( \mathbb R \).
	For example, consider
	\[
		A = \begin{pmatrix}
			\cos\theta & -\sin\theta \\
			\sin\theta & \cos\theta
		\end{pmatrix}
	\]
	This is orthogonal and normalised.
	However, \( \chi_A(\lambda) = 1 + 2\lambda \cos\theta + \lambda^2 \) hence \( \lambda = e^{\pm i \theta} \) which are complex in the general case.
\end{remark}

\subsection{Application to bilinear forms}
We wish to extend the previous statements about spectral theory into statements about bilinear forms.
\begin{corollary}
	Let \( A \in M_n(\mathbb R) \) (or \( M_n(\mathbb C) \)) be a symmetric (or respectively Hermitian) matrix.
	Then there exists an orthonormal (respectively unitary) matrix \( P \) such that \( P^\transpose A P \) (or \( P^\dagger A P \)) is diagonal with real-valued entries.
\end{corollary}
\begin{proof}
	Using the standard inner product, \( A \in L(F^n) \) is self-adjoint and hence there exists an orthonormal basis \( B \) of \( F^n \) such that \( A \) is diagonal in this basis.
	Let \( P = (v_1, \dots, v_n) \) be the matrix of this basis.
	Since \( B \) is orthonormal, \( P \) is orthogonal (or unitary).
	The result follows from the fact that \( P^{-1} A P \) is diagonal.
	The eigenvalues are real, hence the diagonal matrix is real.
\end{proof}
\begin{corollary}
	Let \( V \) be a finite-dimensional real (or complex) inner product space.
	Let \( \phi \colon V \times V \to F \) be a symmetric (or Hermitian) bilinear form.
	Then, there exists an orthonormal basis \( B \) of \( V \) such that \( [\phi]_B \) is diagonal.
\end{corollary}
\begin{proof}
	\( A^\transpose = A \) (or respectively \( A^\dagger = A \)), hence there exists an orthogonal (respectively unitary) matrix \( P \) such that \( P^{-1} A P \) is diagonal.
	Let \( (v_i) \) be the \( i \)th row of \( P^{-1} = P^\transpose \) (or \( P^\dagger \)).
	Then \( (v_1, \dots, v_n) \) is an orthonormal basis \( B \) of \( V \) such that \( [\phi]_V \) is this diagonal matrix.
\end{proof}
\begin{remark}
	The diagonal entries of \( P^{-1} A P \) are the eigenvalues of \( A \).
	Moreover, we can define the signature \( s(\phi) \) to be the difference between the number of positive eigenvalues of \( A \) and the number of negative eigenvalues of \( A \).
\end{remark}

\subsection{Simultaneous diagonalisation}
\begin{corollary}
	Let \( V \) be a finite-dimensional real (or complex) vector space.
	Let \( \phi, \psi \) be symmetric (or Hermitian) bilinear forms on \( V \).
	Let \( \phi \) be positive definite.
	Then there exists a basis \( (v_1, \dots, v_n) \) of \( V \) with respect to which \( \phi \) and \( \psi \) are represented with a diagonal matrix.
\end{corollary}
\begin{proof}
	Since \( \phi \) is positive definite, \( V \) equipped with \( \phi \) is a finite-dimensional inner product space where \( \inner{u,v} = \phi(u,v) \).
	Hence, there exists a basis of \( V \) in which \( \psi \) is represented by a diagonal matrix, which is orthonormal with respect to the inner product defined by \( \phi \).
	Then, \( \phi \) in this basis is represented by the identity matrix given by \( \phi(v_i, v_j) = \inner{v_i, v_j} = \delta_{ij} \), which is diagonal.
\end{proof}
\begin{corollary}
	Let \( A, B \in M_n(\mathbb R) \) (or \( \mathbb C \)) which are symmetric (or Hermitian).
	Suppose for all \( x \neq 0 \) we have \( x^\dagger A x > 0 \), so \( A \) is positive definite.
	Then there exists an invertible matrix \( Q \in M_n(\mathbb R) \) (or \( \mathbb C \)) such that \( Q^\transpose A Q \) (or \( Q^\transpose A \overline{Q} \)) and \( Q^\transpose B Q \) (or \( Q^\transpose B \overline{Q} \)) are diagonal.
\end{corollary}
\begin{proof}
	\( A \) induces a quadratic form \( Q(x) = x^\dagger A x \) which is positive definite by assumption.
	Similarly, \( \widetilde Q(x) = x^\dagger B x \) is induced by \( B \).
	Then we can apply the previous corollary and change basis.
\end{proof}
