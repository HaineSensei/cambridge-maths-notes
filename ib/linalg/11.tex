\subsection{Trace}
\begin{definition}
	The \textit{trace} of a square matrix \( A \in M_{n,n}(F) \equiv M_n(F) \) is defined by
	\[
		\tr A = \sum_{i=1}^n a_{ii}
	\]
	The trace is a linear form.
\end{definition}
\begin{lemma}
	\( \tr (AB) = \tr (BA) \) for any matrices \( A, B \in M_n(F) \).
\end{lemma}
\begin{proof}
	We have
	\[
		\tr (AB) = \sum_{i=1}^n \sum_{j=1}^n a_{ij} b_{ji} = \sum_{j=1}^n \sum_{i=1}^n b_{ji} a_{ij} = \tr (BA)
	\]
\end{proof}
\begin{corollary}
	Similar matrices have the same trace.
\end{corollary}
\begin{proof}
	\[ \tr(P^{-1}AP) = \tr (A P^{-1} P) = \tr A \]
\end{proof}
\begin{definition}
	If \( \alpha \colon V \to V \) is linear, we can define the trace of \( \alpha \) as
	\[ \tr \alpha = \tr [\alpha]_B \]
	for any basis \( B \).
	This is well-defined by the corollary above.
\end{definition}
\begin{lemma}
	If \( \alpha \colon V \to V \) is linear, \( \alpha^\star \colon V^\star \to V^\star \) satisfies
	\[ \tr \alpha = \tr \alpha^\star \]
\end{lemma}
\begin{proof}
	\[ \tr \alpha = \tr [\alpha]_B = \tr [\alpha]_B^\transpose = \tr [\alpha^\star]_{B^\star} = \tr \alpha^\star \]
\end{proof}

\subsection{Permutations and transpositions}
Recall the following facts about permutations and transpositions.
\( S_n \) is the group of permutations of the set \( \qty{1, \dots, n} \); the group of bijections \( \sigma \colon \qty{1, \dots, n} \to \qty{1, \dots, n} \).
A transposition \( \tau_{k \ell} = (k, \ell) \) is defined by \( k \mapsto \ell, \ell \mapsto k, x \mapsto x \) for \( x \neq k, \ell \).
Any permutation \( \sigma \) can be decomposed as a product of transpositions.
This decomposition is not necessarily unique, but the parity of the number of transpositions is well-defined.
We say that the signature of a permutation, denoted \( \varepsilon \colon S_n \to \qty{-1, 1} \), is \( 1 \) if the decomposition has even parity and \( -1 \) if it has odd parity.
We can then show that \( \varepsilon \) is a homomorphism.

\subsection{Determinant}
\begin{definition}
	Let \( A \in M_n(F) \).
	We define
	\[ \det A = \sum_{\sigma \in S_n} \varepsilon(\sigma) A_{\sigma(1) 1} \dots A_{\sigma(n) n} \]
\end{definition}
\begin{example}
	Let \( n = 2 \).
	Then,
	\[ A = \begin{pmatrix} a_{11} & a_{12} \\ a_{21} & a_{22} \end{pmatrix} \implies \det A = a_{11} a_{22} - a_{12} a_{21} \]
\end{example}
\begin{lemma}
	If \( A = (a_{ij}) \) is an upper (or lower) triangular matrix (with zeroes on the diagonal), then \( \det A = 0 \).
\end{lemma}
\begin{proof}
	Let \( (a_{ij}) = 0 \) for \( i > j \).
	Then
	\[ \det A = \sum_{\sigma \in S_n} \varepsilon(\sigma) a_{\sigma(1) 1} \dots a_{\sigma(n) n} \]
	For the summand to be non-zero, \( \sigma(j) \leq j \) for all \( j \).
	Thus,
	\[ \det A = a_{1 1} \dots a_{n n} = 0 \]
\end{proof}
\begin{lemma}
	Let \( A \in M_n(F) \).
	Then, \( \det A = \det A^\transpose \).
\end{lemma}
\begin{proof}
	\begin{align*}
		\det A &= \sum_{\sigma \in S_n} \varepsilon(\sigma) a_{\sigma(1) 1} \dots a_{\sigma(n) n} \\
		&= \sum_{\sigma^{-1} \in S_n} \varepsilon(\sigma) a_{\sigma(1) 1} \dots a_{\sigma(n) n} \\
		&= \sum_{\sigma \in S_n} \varepsilon(\sigma^{-1}) a_{1 \sigma(1)} \dots a_{n \sigma(n)} \\
		&= \sum_{\sigma \in S_n} \varepsilon(\sigma) a_{1 \sigma(1)} \dots a_{n \sigma(n)} \\
		&= \det A^\transpose
	\end{align*}
\end{proof}

\subsection{Volume forms}
\begin{definition}
	A volume form \( d \) on \( F^n \) is a function \( d \colon \underbrace{F^n \times \dots \times F^n}_{n \text{ times}} \to F \) satisfying
	\begin{enumerate}[(i)]
		\item \( d \) is multilinear: for all \( i \in \qty{1, \dots, n} \) and for all \( v_1, \dots, v_{i-1}, v_{i+1}, \dots, v_n \in F^n \), the map from \( F^n \) to \( F \) defined by
			\[ v \mapsto (v_1, \dots, v_{i-1}, v, v_{i+1}, \dots, v_n) \]
			is linear.
			In other wods, this map is an element of \( (F^n)^\star \).
		\item \( d \) is alternating: for \( v_i = v_j \) for some \( i \neq j \), \( d = 0 \).
	\end{enumerate}
	So an alternating multilinear form is a volume form.
	We want to show that, up to multiplication by a scalar, the determinant is the only volume form.
\end{definition}
\begin{lemma}
	The map \( (F^n)^n \to F \) defined by \( (A^{(1)}, \dots, A^{(n)}) \mapsto \det A \) is a volume form.
	This map is the determinant of \( A \), but thought of as acting on the column vectors of \( A \).
\end{lemma}
\begin{proof}
	We first show that this map is multilinear.
	Fix \( \sigma \in S_n \), and consider \( \prod_{i=1}^n a_{\sigma(i) i} \).
	This product contains exactly one term in each column of \( A \).
	Thus, the map \( (A^{(1)}, \dots, A^{(n)}) \mapsto \prod_{i=1}^n a_{\sigma(i) i} \) is multilinear.
	This then clearly implies that the determinant, a sum of such multilinear maps, is itself multilinear.

	Now, we show that the determinant is alternating.
	Let \( k \neq \ell \), and \( A^{(k)} = A^{(\ell}) \).
	Let \( \tau = ( k \ell ) \) be the transposition exchanging \( k \) and \( \ell \).
	Then, for all \( i, j \in \qty{1, \dots, n} \), \( a_{ij} = a_{i \tau(j)} \).
	We can decompose permutations into two disjoint sets: \( S_n = A_n \cup \tau A_n \), where \( A_n \) is the alternating group of order \( n \).
	Now, note that \( \prod_{i=1}^n a_{\sigma(i) i} + \prod_{i=1}^n a_{(\tau \circ \sigma)(i) i} = 0 \).
	So the sum over all \( \sigma \in A_n \) gives zero.
	So the determinant is alternating, and hence a volume form.
\end{proof}
\begin{lemma}
	Let \( d \) be a volume form.
	Then, swapping two entries changes the sign.
\end{lemma}
\begin{proof}
	Take the sum of these two results:
	\begin{align*}
		d(v_1, \dots, v_i, \dots, v_j, \dots, v_n) \\
		&+ d(v_1, \dots, v_j, \dots, v_i, \dots, v_n) \\
		&= d(v_1, \dots, v_i, \dots, v_j, \dots, v_n) \\
		&+ d(v_1, \dots, v_j, \dots, v_i, \dots, v_n) \\
		&+ d(v_1, \dots, v_i, \dots, v_i, \dots, v_n) \\
		&+ d(v_1, \dots, v_j, \dots, v_j, v_n) \\
		&= 2 d(v_1, \dots, v_i + v_j, \dots, v_i + v_j, \dots, v_n) \\
		&= 0
	\end{align*}
	as required.
\end{proof}
\begin{corollary}
	If \( \sigma \in S_n \) and \( d \) is a volume form, \( d(v_{\sigma(1)}, \dots, v_{\sigma(n)}) = \varepsilon(\sigma) d(v_1, \dots, v_n) \).
\end{corollary}
