\subsection{Consequences of Steinitz Exchange Lemma}
\begin{corollary}
	Let \( V \) be a finite-dimensional \( F \)-vector space.
	Then, any two bases of \( V \) have the same number of vectors.
	This number is called the dimension of \( V \), \( \dim_F V \).
\end{corollary}
\begin{proof}
	Suppose the two bases are \( (v_1, \dots, v_n) \) and \( (w_1, \dots, w_m) \).
	Then, \( (v_1, \dots, v_n) \) is free and \( (w_1, \dots, w_m) \) is generating, so the Steinitz exchange lemma shows that \( n \leq m \).
	Vice versa, \( m \leq n \).
	Hence \( m = n \).
\end{proof}
\begin{corollary}
	Let \( V \) be an \( F \)-vector space with finite dimension \( n \).
	Then,
	\begin{enumerate}[(i)]
		\item Any independent set of vectors has at most \( n \) elements, with equality if and only if it is a basis.
		\item Any spanning set of vectors has at least \( n \) elements, with equality if and only if it is a basis.
	\end{enumerate}
\end{corollary}
\begin{proof}
	Exercise.
\end{proof}

\subsection{???}
\begin{proposition}
	Let \( V \) be an \( F \)-vector space.
	Let \( U, W \) be subspaces of \( V \).
	If \( U, W \) are finite-dimensional, then so is \( U + W \), with
	\[
		\dim_F (U + W) = \dim_F U + \dim_F W - \dim_F (U \cap W)
	\]
\end{proposition}
\begin{proof}
	Consider a basis \( (v_1, \dots, v_n) \) of the intersection.
	Extend this basis to a basis \( (v_1, \dots, v_n, u_1, \dots, u_m) \) of \( U \) and \( (v_1, \dots, v_n, w_1, \dots, w_k) \) of \( W \).
	Then, we will show that \( (v_1, \dots, v_n, u_1, \dots, u_m, w_1, \dots, w_k) \) is a basis of \( \dim_F (U + W) \), which will conclude the proof.
	Indeed, since any component of \( U + W \) can be decomposed as a sum of some element of \( U \) and some element of \( W \), we can add their decompositions together.
	Now we must show that this new basis is free.
	\begin{align*}
		\sum_{i=1}^n \alpha_i v_i + \sum_{i=1}^m \beta_i u_i + \sum_{i=1}^k \gamma_i w_i & = 0                                              \\
		\underbrace{\sum_{i=1}^n \alpha_i v_i + \sum_{i=1}^m \beta_i u_i}_{\in U}        & = \underbrace{\sum_{i=1}^k \gamma_i w_i}_{\in W} \\
		\sum_{i=1}^k \gamma_i w_i                                                        & \in U \cap W                                     \\
		\sum_{i=1}^k \gamma_i w_i                                                        & = \sum_{i=1}^n \delta_i v_i                      \\
		\sum_{i=1}^n (\alpha_i + \delta_i) v_i + \sum_{i=1}^m \beta_i u_i                & = 0                                              \\
		\beta_i = 0, \alpha_i                                                            & = -\delta_i                                      \\
		\sum_{i=1}^n \alpha_i v_i + \sum_{i=1}^k \gamma_i w_i                            & = 0                                              \\
		\alpha_i = 0, \gamma_i                                                           & = 0
	\end{align*}
\end{proof}
\begin{proposition}
	If \( V \) is a finite-dimensional \( F \)-vector space, and \( U \leq V \), then \( U \) and \( V / U \) are also finite-dimensional.
	In particular, \( \dim_F V = \dim_F U + \dim_F (V / U) \).
\end{proposition}
\begin{proof}
	Let \( (u_1, \dots, u_\ell) \) be a basis of \( U \).
	We extend this basis to a basis of \( V \): \( (u_1, \dots, u_\ell, w_{\ell + 1}, \dots, w_n) \).
	We claim that \( (w_{\ell + 1} + U, \dots, w_n + U) \) is a basis of the vector space \( V / U \).
	% exercise.
\end{proof}

\begin{remark}
	If \( V \) is an \( F \)-vector space, and \( U \leq V \), then we say \( U \) is a proper subspace if \( U \neq V \).
	Then if \( U \) is proper, then \( \dim_F U < \dim_F V \) and \( \dim_F ( V / U ) > 0 \) because \( (V/U) \neq \varnothing \).
\end{remark}

\begin{definition}
	If \( V \) is an \( F \)-vector space and \( U, W \) be subspaces of \( V \).
	We say that \( V = U \oplus V \), read as the direct sum of \( U \) and \( V \), if \( \forall v \in V, \exists!
	u \in U, \exists!
	w \in W, u + w = v \).
	We say that \( W \) is \textit{a} direct complement of \( U \) in \( V \); there is no uniqueness of such a complement.
\end{definition}
\begin{lemma}
	Let \( V \) be an \( F \)-vector space, and \( U, W \leq V \).
	Then the following statements are equivalent.
	\begin{enumerate}[(i)]
		\item \( V = U \oplus W \)
		\item \( V = U + W \) and \( U \cap W = \{0\} \)
		\item For any basis \( B_1 \) of \( U \) and \( B_2 \) of \( W \), \( B_1 \cup B_2 \) is a basis of \( V \)
	\end{enumerate}
\end{lemma}
\begin{proof}
	First, we show that (ii) implies (i).
	If \( V = U + W \), then certainly \( \forall v \in V, \exists u \in U, \exists w \in W, v = u + w \), so it suffices to show uniqueness.
	Note, \( u_1 + w_1 = u_2 + w_2 \implies u_1 - u_2 = w_2 - w_1 \).
	The left hand side is an element of \( U \) and the right hand side is an element of \( W \), so they must be the zero vector; \( u_1 = u_2, w_1 = w_2 \).

	Now, we show (i) implies (iii).
	Suppose \( B_1 \) is a basis of \( U \) and \( B_2 \) is a basis of \( W \).
	Let \( B = B_1 \cup B_2 \).
	First, note that \( B \) is a generating family of \( U + W \).
	Now we must show that \( B \) is free.
	\[
		\underbrace{\sum_{u \in B_1} \lambda_u u}_{\in U} + \underbrace{\sum_{w \in B_2} \lambda_w w}_{\in W} = 0
	\]
	Hence both sums must be zero.
	Since \( B_1, B_2 \) are bases, all \( \lambda \) are zero, so \( B \) is free and hence a basis.

	Now it remains to show that (iii) implies (ii).
	We must show that \( V = U + W \) and \( U \cap W = \{0\} \).
	Now, suppose \( v \in V \).
	Then, \( v = \sum_{u \in B_1} \lambda_u u + \sum{w \in B_2} \lambda_w w \).
	In particular, \( V = U + W \), since the \( \lambda_u, \lambda_w \) are arbitrary.
	Now, let \( v \in U \cap W \).
	Then
	\[
		v = \sum_{u \in B_1} \lambda_u u = \sum_{w \in B_2} \lambda_w w \implies \lambda_u = \lambda_w = 0
	\]
\end{proof}

\begin{definition}
	Let \( V \) be an \( F \)-vector space, with subspaces \( V_1, \dots, V_p \leq V \).
	Then
	\[
		\sum_{i=1}^p V_i = \qty{ v_1, \dots, v_\ell, v_i \in V_i, 1 \leq i \leq \ell}
	\]
	We say the sum is direct, written
	\[
		\bigoplus_{i=1}^p V_i
	\]
	if the decomposition is unique.
	Equivalently,
	\[
		V = \bigoplus_{i=1}^p V_i \iff \exists!
		v_1 \in V_1, \dots, v_n \in V_n, v = \sum_{i=1}^n v_i
	\]
\end{definition}
\begin{lemma}
	The following are equivalent:
	\begin{enumerate}[(i)]
		\item \( \sum_{i=1}^p V_i = \bigoplus_{i=1}^p V_i \)
		\item \( \forall 1 \leq i \leq l \), \( V_i \cap \qty( \sum_{j \neq i} V_j ) = \{0\} \)
		\item For any basis \( B_i \) of \( V_i \), \( B = \bigcup_{i=1}^n B_i \) is a basis of \( \sum_{i=1}^n V_i \).
	\end{enumerate}
\end{lemma}
\begin{proof}
	Exercise.
\end{proof}
