\subsection{Spectral theory for self-adjoint maps}
Spectral theory is the study of the spectrum of operators.
Recall that in finite-dimensional inner product spaces \( V, W \), \( \alpha \in L(V, W) \) yields the adjoint \( \alpha^\star \in L(W, V) \) such that for all \( v \in V, w \in W \), we have \( \inner{\alpha(v), w} = \inner{v, \alpha^\star(w)} \).
\begin{lemma}
	Let \( V \) be a finite-dimensional inner product space.
	Let \( \alpha \in L(V) \) be a self-adjoint endomorphism.
	Then \( \alpha \) has real eigenvalues, and eigenvectors of \( \alpha \) with respect to different eigenvalues are orthogonal.
\end{lemma}
\begin{proof}
	Suppose \( \lambda \in \mathbb C \), \( v \in V \) non-zero such that \( \alpha(v) = \lambda v \).
	Then, \( \inner{\lambda v, v} = \lambda \norm{v}^2 \) and also
	\[
		\inner{\alpha v, v} = \inner{v, \alpha v} = \inner{v, \lambda v} = \overline{\lambda} \norm{v}^2
	\]
	Hence \( \lambda = \overline{\lambda} \) since \( v \neq 0 \).
	Now, suppose \( \mu \neq \lambda \) and \( w \in V \) non-zero such that \( \alpha(w) = \mu w \).
	Then,
	\[
		\lambda \inner{v, w} = \inner{\alpha v, w} = \inner{v, \alpha w} = \overline{\mu} \inner{v, w} = \mu \inner{v,w}
	\]
	So if \( \lambda \neq \mu \) we must have \( \inner{v,w} = 0 \).
\end{proof}
\begin{theorem}[spectral theorem for self-adjoint maps]
	Let \( V \) be a finite-dimensional inner product space.
	Let \( \alpha \in L(V) \) be self-adjoint.
	Then \( V \) has an orthonormal basis of eigenvectors of \( \alpha \).
	Hence \( \alpha \) is diagonalisable in an orthonormal basis.
\end{theorem}
\begin{proof}
	We will consider induction on the dimension of \( V \).
	Suppose \( A = [\alpha]_B \) with respect to the fundamental basis \( B \).
	By the fundamental theorem of algebra, we know that \( \chi_A(\lambda) \) has a (complex) root.
	But since \( \lambda \) is an eigenvalue of \( \alpha \) and \( \alpha \) is self-adjoint, \( \lambda \in \mathbb R \).
	Now, we choose an eigenvector \( v_1 = V \setminus \qty{0} \) such that \( \alpha(v_1) = \lambda v_1 \).
	We can set \( \norm{v_1} = 1 \) by linearity.
	Let \( U = \genset{v_1}^\perp \leq V \).
	We then observe that \( U \) is stable by \( \alpha \); \( \alpha(U) \leq U \).
	Indeed, let \( u \in U \).
	Then \( \inner{\alpha(u), v_1} = \inner{u, \alpha(v_1)} = \lambda \inner{u, v_1} = 0 \) by orthogonality.
	Hence \( \alpha(u) \in U \).
	We can then restrict \( \alpha \) to the domain \( U \), and by induction we can then choose an orthonormal basis of eigenvectors for \( U \).
	Since \( V = \genset{v_1} \overset{\perp}{\oplus} U \) we have an orthonormal basis of eigenvectors for \( V \) when including \( v_1 \).
\end{proof}
\begin{corollary}
	Let \( V \) be a finite-dimensional inner product space.
	Let \( \alpha \in L(V) \) be self-adjoint.
	Then \( V \) is the orthogonal direct sum of the eigenspaces of \( \alpha \).
\end{corollary}

\subsection{Spectral theory for unitary maps}
\begin{lemma}
	Let \( V \) be a complex inner product space.
	Let \( \alpha \) be unitary, so \( \alpha^\star = \alpha^{-1} \).
	Then all eigenvalues of \( \alpha \) have unit norm.
	Eigenvectors corresponding to different eigenvalues are orthogonal.
\end{lemma}
\begin{proof}
	Let \( \lambda \in \mathbb C \), \( v \in V \setminus \qty{0} \) such that \( \alpha(v) = \lambda v \).
	First, \( \lambda \neq 0 \) since \( \alpha \) is invertible, and in particular \( \ker \alpha = \qty{0} \).
	Since \( v = \lambda \alpha^{-1}(v) \), we can compute
	\[
		\lambda \inner{v,v} = \inner{\lambda v, v} = \inner{\alpha v, v} = \inner{v, \alpha^{-1} v} = \inner{v, \frac{1}{\lambda} v} = \frac{1}{\overline \lambda} \inner{v, v}
	\]
	Hence \( (\lambda \overline \lambda - 1) \norm{v}^2 = 0 \) giving \( \abs{\lambda} = 1 \).
	Further, suppose \( \mu \in \mathbb C \) and \( w \in V \setminus \qty{0} \) such that \( \alpha(w) = \mu w, \lambda \neq \mu \).
	Then
	\[
		\lambda \inner{v,w} = \inner{\lambda v, w} = \inner{\alpha v, w} = \inner{v, \alpha^{-1} w} = \inner{v, \frac{1}{\mu} w} = \frac{1}{\overline \mu} \inner{v,w} = \mu \inner{v,w}
	\]
	since \( \mu \overline \mu = 1 \).
\end{proof}
\begin{theorem}[spectral theorem for unitary maps]
	Let \( V \) be a finite-dimensional complex inner product space.
	Let \( \alpha \in L(V) \) be unitary.
	Then \( V \) has an orthonormal basis of eigenvectors of \( \alpha \).
	Hence \( \alpha \) is diagonalisable in an orthonormal basis.
\end{theorem}
\begin{proof}
	Let \( A = [\alpha]_B \) where \( B \) is an orthonormal basis.
	Then \( \chi_A(\lambda) \) has a complex root \( \lambda \).
	As before, let \( v_1 \neq 0 \) such that \( \alpha(v_1) = \lambda v_1 \) and \( \norm{v_1} = 1 \).
	Let \( U = \genset{v_1}^\perp \), and we claim that \( \alpha(U) = U \).
	Indeed, let \( u \in U \), and we find
	\[
		\inner{\alpha(u), v_1} = \inner{u, \alpha^{-1}(v_1)} = \inner{u, \frac{1}{\lambda} v_1} = \frac{1}{\overline \lambda} \inner{u,v_1}
	\]
	Since \( \inner{u, v_1} = 0 \), we have \( \alpha(u) \in U \).
	Hence, \( \alpha \) restricted to \( U \) is a unitary endomorphism of \( U \).
	By induction we have an orthonormal basis of eigenvectors of \( \alpha \) for \( U \) and hence for \( V \).
\end{proof}
\begin{remark}
	We used the fact that the field is complex to find an eigenvalue.
	In general, a real-valued orthonormal matrix \( A \) giving \( A A^\transpose = I \) cannot be diagonalised over \( \mathbb R \).
	For example, consider
	\[
		A = \begin{pmatrix}
			\cos\theta & -\sin\theta \\
			\sin\theta & \cos\theta
		\end{pmatrix}
	\]
	This is orthogonal and normalised.
	However, \( \chi_A(\lambda) = 1 + 2\lambda \cos\theta + \lambda^2 \) hence \( \lambda = e^{\pm i \theta} \) which are complex in the general case.
\end{remark}
