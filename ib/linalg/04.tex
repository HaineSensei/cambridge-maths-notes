\subsection{Linear Maps}
\begin{definition}
    If \( V, W \) are \( F \)-vector spaces, a map \( \alpha \colon V \to W \) is \textit{linear} if
    \[ \forall \lambda_1, \lambda_2 \in F, \forall v_1, v_2 \in V, \alpha(\lambda_1 v_1 + \lambda_2 v_2) = \lambda_1 \alpha(v_1) + \lambda_2 \alpha(v_2) \]
\end{definition}
\begin{example}
    Let \( M \) be a matrix with \( n \) rows and \( m \) columns.
    Then the map \( \alpha \colon \mathbb R^m \to \mahtbb R^n \) defined by \( x \mapsto M x \) is a linear map.
\end{example}
\begin{example}
    Let \( \alpha \colon \mathcal C([0,1], \mathbb R) \to \mathcal C([0,1], \mathbb R) \) defined by \( f \mapsto a(f)(x) = \int_0^x f(t) \dd{t} \).
    This is linear.
\end{example}
\begin{example}
    Let \( x \in [a,b] \).
    Then \( \alpha \colon \mathcal C([a,b], \mathbb R) \to \mathbb R \) defined by \( f \mapsto f(x) \) is a linear map.
\end{example}
\begin{remark}
    Let \( U, V, W \) be \( F \)-vector spaces.
    Then,
    \begin{enumerate}[(i)]
        \item The identity function \( i_V \colon V \to V \) defined by \( x \mapsto x \) is linear.
        \item If \( alpha \colon U \to V \) and \( \beta \colon V \to W \) are linear, then \( \beta \circ \alpha \) is linear.
    \end{enumerate}
\end{remark}
\begin{lemma}
    Let \( V, W \) be \( F \)-vector spaces.
    Let \( B \) be a basis for \( V \).
    If \( \alpha_0 \colon B \to V \) is \textit{any} map (not necessarily linear), then there exists a unique linear map \( \alpha \colon V \to W \) extending \( \alpha_0 \): \( \forall v \in B, \alpha_0(v) = \alpha(v) \).
\end{lemma}
\begin{proof}
    Let \( v \in V \).
    Then, given \( B = (v_1, \dots, v_n) \).
    \[ v = \sum_{i=1}^n \lambda_i v_i \]
    By linearity,
    \[ \alpha(v) = \alpha\qty(\sum_{i=1}^n \lambda_i v_i) = \sum_{i=1}^n \alpha(\lambda_i v_i) = \sum_{i=1}^n \alpha_0(\lambda_i v_i) \]
\end{proof}
\begin{remark}
    This lemma is also true in infinite-dimensional vector spaces.
    Often, to define a linear map, we instead define its action on the basis vectors, and then we `extend by linearity' to construct the entire map.
\end{remark}
\begin{remark}
    If \( \alpha_1, \alpha_2 \colon V \to W \) are linear maps, then if they agree on any basis of \( V \) then they are equal.
\end{remark}

\subsection{Isomorphism}
\begin{definition}
    Let \( V, W \) be \( F \)-vector spaces.
    A map \( \alpha \colon V \to W \) is an \textit{isomorphism} if and only if
    \begin{enumerate}[(i)]
        \item \( \alpha \) is linear
        \item \( \alpha \) is bijective
    \end{enumerate}
    If such an \( \alpha \) exists, we say that \( V \) and \( W \) are isomorphic, written \( V \simeq W \).
\end{definition}
\begin{remark}
    If \( \alpha \) in the above definition is an isomorphism, then \( \alpha^{-1} \colon W \to V \) is linear.
    Indeed, if \( w_1, w_2 \in W \) with \( w_1 = \alpha(v_1) \) and \( w_2 = \alpha(v_2) \),
    \[ \alpha^{-1} (w_1 + w_2) = \alpha^{-1} (\alpha(v_1) + \alpha(v_2)) = \alpha^{-1} \alpha (v_1 + v_2) = v_1 + v_2 = \alpha^{-1}(w_1) + \alpha^{-1}(w_2) \]
    Similarly, for \( \lambda \in F, w \in W \),
    \[ \alpha^{-1}(\lambda w) = \lambda \alpha^{-1}(w) \]
\end{remark}
\begin{lemma}
    Isomorphism is an equivalence relation on the class of all vector spaces over \( F \).
\end{lemma}
\begin{proof}
    \begin{enumerate}[(i)]
        \item \( i_V \colon V \to V \) is an isomorphism
        \item If \( \alpha \colon V \to W \) is an isomorphism, \( \alpha^{-1} \colon W \to V \) is an isomorphism.
        \item If \( \beta \colon U \to V, \alpha \colon V \to W \) are isomorphisms, then \( \alpha \circ \beta \colon \U \to W \) is an isomorphism.
    \end{enumerate}
    The proofs of each part are left as an exercise.
\end{proof}
\begin{theorem}
    If \( V \) is an \( F \)-vector space of dimension \( n \), then \( V \simeq F^n \).
\end{theorem}
\begin{proof}
    Let \( B = (v_1, \dots, v_n) \) be a basis for \( V \).
    Then, consider \( \alpha \colon V \to F^n \) defined by
    \[ v = \sum_{i=1}^n \lambda_i v_i \mapsto \begin{pmatrix}\lambda_1 \\ \vdots \\ \lambda_n \end{pmatrix} \]
    We claim that this is an isomorphism.
    This is left as an exercise.
\end{proof}
\begin{remark}
    Choosing a basis for \( V \) is analogous to choosing an isomorphism from \( V \) to \( F^n \).
\end{remark}
\begin{theorem}
    Let \( V, W \) be \( F \)-vector spaces with finite dimensions \( n, m \).
    Then,
    \[ V \simeq W \iff n = m \]
\end{theorem}
\begin{proof}
    If \( \dim V = \dim W = n \), then there exist isomorphisms from both \( V \) and \( W \) to \( F^n \).
    By transitivity, therefore, there exists an isomorphism between \( V \) and \( W \).

    Conversely, if \( V \simeq W \) then let \( \alpha \colon V \to W \) be an isomorphism.
    Let \( B \) be a basis of \( V \), then we claim that \( \alpha(B) \) is a basis of \( W \).
    Indeed, \( \alpha(B) \) spans \( W \) from the surjectivity of \( \alpha \), and \( \alpha(B) \) is free due to injectivity.
\end{proof}

\subsection{Kernel and Image}
\begin{definition}
    Let \( V, W \) be \( F \)-vector spaces.
    Let \( \alpha \colon V \to W \) be a linear map.
    We define the kernel and image as follows.
    \[ N(\alpha) = \ker\alpha = \qty{v \in V \colon \alpha(v) = 0} \]
    \[ \Im(\alpha) = \qty{w \in W \colon \exists v \in V, w = \alpha(v)} \]
\end{definition}
\begin{lemma}
    \( \ker \alpha \) is a subspace of \( V \), and \( \Im \alpha \) is a subspace of \( W \).
\end{lemma}
\begin{proof}
    Let \( \lambda_1, \lambda_2 \in F \) and \( v_1, v_2 \in \ker \alpha \).
    Then
    \[ \alpha(\lambda_1 v_1 + \lambda_2 v_2) = \lambda_1 \alpha(v_1) + \lambda_2 \alpha(v_2) = 0 \]
    Hence \( \lambda_1 v_1 + \lambda_2 v_2 \in \ker \alpha \).

    Now, let \( \lambda_1, \lambda_2 \in F \), \( v_1, v_2 \in V \), and \( w_1 = \alpha(v_1), w_2 = \alpha(v_2) \).
    Then
    \[ \lambda_1 w_1 + \lambda_2 w_2 = \lambda_1 \alpha(v_1) + \lambda_2 \alpha(v_2) = \alpha(\lambda_1 v_1 + \lambda_2 v_2) \in \Im \alpha \]
\end{proof}
\begin{remark}
    \( \alpha \colon V \to W \) is injective if and only if \( \ker \alpha = \{ 0 \} \).
    Further, \( \alpha \colon V \to W \) is surjective if and only if \( \Im \alpha = W \).
\end{remark}
\begin{theorem}
    Let \( V, W \) be \( F \)-vector spaces.
    Let \( \alpha \colon V \to W \) be a linear map.
    Then \( \overline \alpha \colon V / \ker \alpha \to \Im \alpha \) defined by
    \[ \overline \alpha (v + \ker \alpha) = \alpha(v) \]
    is an isomorphism.
    \textit{This is the isomorphism theorem from IA Groups.}
\end{theorem}
\begin{proof}
    First, note that \( \overline\alpha \) is well defined.
    Suppose \( v + \ker \alpha = v' + \ker \alpha \).
    Then \( v - v' \in \ker \alpha \), hence
    \[ \alpha(v - v') = 0 \implies \alpha(v) - \alpha(v') = 0 \]
    so \( \overline\alpha \) is indeed well defined.

    Now, we show \( \overline\alpha \) is injective.
    \[ \overline\alpha(v + \ker \alpha) = 0 \implies \alpha(v) = 0 \implies v \in \ker \alpha \]
    Hence, \( v + \ker \alpha = 0 + \ker \alpha \).

    Further, \( \overline\alpha \) is surjective.
    This follows from the definition the image.
\end{proof}

\subsection{Rank and Nullity}
\begin{definition}
    The \textit{rank} of \( \alpha \) is
    \[ r(\alpha) = \dim\Im \alpha \]
    The \textit{nullity} of \( \alpha \) is
    \[ n(\alpha) = \dim\ker \alpha \]
\end{definition}
\begin{theorem}[Rank-nullity theorem]
    Let \( U, V \) be \( F \)-vector spaces such that the dimension of \( U \) is finite.
    Let \( \alpha \colon U \to V \) be a linear map.
    Then,
    \[ \dim U = r(\alpha) + n(\alpha) \]
\end{theorem}
\begin{proof}
    We have proven that \( U / \ker \alpha \simeq \Im \alpha \).
    Hence, the dimensions on the left and right match: \( \dim (U/\ker\alpha) = \dim \Im \alpha \).
    \[ \dim U - \dim \ker \alpha = \dim \Im \alpha \]
    and the result follows.
\end{proof}
\begin{lemma}[Characterisation of isomorphisms]
    Let \( V, W \) be \( F \)-vector spaces with equal, finite dimension.
    Let \( \alpha \colon V \to W \) be a linear map.
    Then, the following are equivalent.
    \begin{enumerate}[(i)]
        \item \( \alpha \) is injective.
        \item \( \alpha \) is surjective.
        \item \( \alpha \) is an isomorphism.
    \end{enumerate}
\end{lemma}
\begin{proof}
    Clearly, (iii) follows from (i) and (ii) and vice versa.
    The rest of the proof is left as an exercise, which follows from the rank-nullity theorem.
\end{proof}
