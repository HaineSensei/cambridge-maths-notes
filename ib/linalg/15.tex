\subsection{Notation}
Let \( p(t) \) be a polynomial over \( F \).
We will write
\[
	p(t) = a_n t^n + \dots + a_0
\]
For a matrix \( A \in M_n(F) \), we write
\[
	p(A) = a_n A^n + \dots + a_0 \in M_n(F)
\]
For an endomorphism \( \alpha \in L(V) \),
\[
	p(\alpha) = a_n \alpha^n + \dots + a_0 I \in L(V);\quad \alpha^k \equiv \underbrace{\alpha \circ \dots \circ \alpha}_{k \text{ times}}
\]

\subsection{Sharp criterion of diagonalisability}
\begin{theorem}
	Let \( V \) be a vector space over \( F \) of finite dimension \( n \).
	Let \( \alpha \) be an endomorphism of \( V \).
	Then \( \alpha \) is diagonalisable if and only if there exists a polynomial \( p \) which is a product of \textit{distinct} linear factors, such that \( p(\alpha) = 0 \).
	In other words, there exist distinct \( \lambda_1, \dots, \lambda_k \) such that
	\[
		p(t) = \prod_{i=1}^n (t - \lambda_i) \implies p(\alpha) = 0
	\]
\end{theorem}
\begin{proof}
	Suppose \( \alpha \) is diagonalisable in a basis \( B \).
	Let \( \lambda_1, \dots, \lambda_k \) be the \( k \leq n \) \textit{distinct} eigenvalues.
	Let
	\[
		p(t) = \prod_{i=1}^k (t-\lambda_i)
	\]
	Let \( v \in B \).
	Then \( \alpha(v) = \lambda_i v \) for some \( i \).
	Then, since the terms in the following product commute,
	\[
		(\alpha - \lambda_i I)(v) = 0 \implies p(\alpha)(v) = \qty[\prod_{i=1}^k (\alpha - \lambda_i I)](v) = 0
	\]
	So for all basis vectors, \( p(\alpha)(v) \).
	By linearity, \( p(\alpha) = 0 \).

	Conversely, suppose that \( p(\alpha) = 0 \) for some polynomial \( p(t) = \prod_{i=1}^k (t-\lambda_i) \) with distinct \( \lambda_i \).
	Let \( V_{\lambda_i} = \ker(\alpha - \lambda_i I) \).
	We claim that
	\[
		V = \bigoplus_{i=1}^k V_{\lambda_i}
	\]
	Consider the polynomials
	\[
		q_j(t) = \prod_{i=1,i \neq j}^k \frac{t-\lambda_i}{\lambda_j - \lambda_i}
	\]
	These polynomials evaluate to one at \( \lambda_j \) and zero at \( \lambda_i \) for \( i \neq j \).
	Hence \( q_j(\lambda_i) = \delta_{ij} \).
	We now define the polynomial
	\[
		q = q_1 + \dots + q_k
	\]
	The degree of \( q \) is at most \( (k-1) \).
	Note, \( q(\lambda_i) = 1 \) for all \( i \in \qty{1, \dots, k} \).
	The only polynomial that evaluates to one at \( k \) points with degree at most \( (k-1) \) is exactly given by \( q(t) = 1 \).
	Consider the endomorphism
	\[
		\pi_j = q_j(\alpha) \in L(V)
	\]
	These are called the `projection operators'.
	By construction,
	\[
		\sum_{j=1}^k \pi_j = \sum_{j=1}^k q_j(\alpha) = I
	\]
	So the sum of the \( \pi_j \) is the identity.
	Hence, for all \( v \in V \),
	\[
		I(v) = v = \sum_{j=1}^k \pi_j(v) = \sum_{j=1}^k q_j(\alpha)(v)
	\]
	So we can decompose any vector as a sum of its projections \( \pi_j(v) \).
	Now, by definition of \( q_j \) and \( p \),
	\begin{align*}
		(\alpha - \lambda_j I) q_j(\alpha)(v) & = \frac{1}{\prod_{i \neq j} (\lambda_j - \lambda_i)} (\alpha - \lambda_j I) \qty[\prod_{i \neq j} (t - \lambda_i)] (\alpha) \\
		                                      & = \frac{1}{\prod_{i \neq j} (\lambda_j - \lambda_i)} \prod_{i=1}^k (\alpha - \lambda_i I)(v)                                \\
		                                      & = \frac{1}{\prod_{i \neq j} (\lambda_j - \lambda_i)} p(\alpha)(v)
	\end{align*}
	By assumption, this is zero.
	For all \( v \), we have \( (\alpha - \lambda_j I) q_j(\alpha)(v) \).
	Hence,
	\[
		(\alpha - \lambda_j I) \pi_j(v) = 0 \implies \pi_j(v) \in \ker(\alpha - \lambda_j I) = v_j
	\]
	We have then proven that, for all \( v \in V \),
	\[
		v = \sum_{j=1}^k \underbrace{\pi_j(v)}_{\in V_j}
	\]
	Hence,
	\[
		V = \sum_{j=1}^k V_j
	\]
	It remains to show that the sum is direct.
	Indeed, let
	\[
		v \in V_{\lambda_j} \cap \qty(\sum_{i \neq j} V_{\lambda_i})
	\]
	We must show \( v = 0 \).
	Applying \( \pi_j \),
	\[
		\pi_j(v) = q_j(\alpha)(v) = \prod_{i \neq j} \frac{(\alpha - \lambda_i I)(v)}{\lambda_j - \lambda_i}
	\]
	Since \( \alpha(v) = \lambda_j v \),
	\[
		\pi_j(v) = \prod_{i \neq j} \frac{(\lambda_j - \lambda_i)v}{\lambda_j - \lambda_i} = v
	\]
	Hence \( \pi_j \) really projects onto \( V_{\lambda_j} \).
	However, we also know \( v \in \sum_{i \neq j} V_{\lambda_i} \).
	So we can write \( v = \sum_{i \neq j} w_i \) for \( w \in V_{\lambda_i} \).
	Thus,
	\[
		\pi_j(w_i) = \prod_{m \neq j} \frac{(\alpha - \lambda_m I)(v)}{\lambda_m - \lambda_j}
	\]
	Since \( \alpha(w_i) = \lambda_i w_i \), one of the factors will vanish, hence
	\[
		\pi_j(w_i) = 0
	\]
	So
	\[
		v = \sum_{i \neq j} w_i \implies \pi_j(v) = \sum_{i \neq j} \pi_j(w_i) = 0
	\]
	But \( v = \pi_j(v) \) hence \( v = 0 \).
	So the sum is direct.
	Hence, \( B = (B_1, \dots, B_k) \) is a basis of \( V \), where the \( B_i \) are bases of \( V_{\lambda_i} \).
	Then \( [\alpha]_B \) is diagonal.
\end{proof}
\begin{remark}
	We have shown further that if \( \lambda_1, \dots, \lambda_k \) are distinct eigenvalues of \( \alpha \), then
	\[
		\sum_{i=1}^k V_{\lambda_i} = \bigoplus_{i=1}^k V_{\lambda_i}
	\]
	Therefore, the only way that diagonalisation fails is when this sum is not direct, so
	\[
		\sum_{i=1}^k V_{\lambda_i} < V
	\]
\end{remark}
\begin{example}
	Let \( F = \mathbb C \).
	Let \( A \in M_n(F) \) such that \( A \) has finite order; there exists \( m \in \mathbb N \) such that \( A^m = I \).
	Then \( A \) is diagonalisable.
	This is because
	\[
		t^m - 1 = p(t) = \prod_{j=1}^m (t - \xi_m^j);\quad \xi_m = e^{2 \pi i/m}
	\]
	and \( p(A) = 0 \).
\end{example}

\subsection{Simultaneous diagonalisation}
\begin{theorem}
	Let \( \alpha, \beta \) be endomorphisms of \( V \) which are diagonalisable.
	Then \( \alpha, \beta \) are \textit{simultaneously diagonalisable} (there exists a basis \( B \) of \( V \) such that \( [\alpha]_B, [\beta]_B \) are diagonal) if and only if \( \alpha \) and \( \beta \) commute.
\end{theorem}
\begin{proof}
	Two diagonal matrices commute.
	If such a basis exists, \( \alpha \beta = \beta \alpha \) in this basis.
	So this holds in any basis.
	Conversely, suppose \( \alpha \beta = \beta \alpha \).
	We have
	\[
		V = \bigoplus_{i=1}^k V_{\lambda_i}
	\]
	where \( \lambda_i, \dots, \lambda_k \) are the \( k \) distinct eigenvalues of \( \alpha \).
	We claim that \( \beta \qty(V_{\lambda_j}) \leq V_{\lambda_j} \).
	Indeed, for \( v \in V_{\lambda_j} \),
	\[
		\alpha \beta(v) = \beta \alpha(v) = \beta(\lambda_j v) = \lambda_j \beta(v) \implies \alpha(\beta(v)) = \lambda_j \beta(v)
	\]
	Hence, \( \beta(v) \in V_{\lambda_j} \).
	By assumption, \( \beta \) is diagonalisable.
	Hence, there exists a polynomial \( p \) with distinct linear factors such that \( p(\beta) = 0 \).
	Now, \( \beta\qty(V_{\lambda_j}) \leq V_{\lambda_j} \) so we can consider \( \eval{\beta}_{V_{\lambda_j}} \).
	This is an endomorphism of \( V_{\lambda_j} \).
	We can compute
	\[
		p\qty(\eval{\beta}_{V_{\lambda_j}}) = 0
	\]
	Hence, \( \eval{\beta}_{V_{\lambda_j}} \) is diagonalisable.
	Let \( B_i \) be the basis of \( V_{\lambda_i} \) in which \( \eval{\beta}_{V_{\lambda_j}} \) is diagonal.
	Since \( V = \bigoplus V_{\lambda_i} \), \( B = (B_1, \dots, B_k) \) is a basis of \( V \).
	Then the matrices of \( \alpha \) and \( \beta \) in \( V \) are diagonal.
\end{proof}

\subsection{Minimal polynomials}
Recall from IB Groups, Rings and Modules the Euclidean algorithm for dividing polynomials.
Given \( a, b \) polynomials over \( F \) with \( b \) non-zero, there exist polynomials \( q, r \) over \( F \) with \( \deg r < \deg b \) and \( a = qb + r \).
\begin{definition}
	Let \( V \) be a finite dimensional \( F \)-vector space.
	Let \( \alpha \) be an endomorphism on \( V \).
	The \textit{minimal polynomial} \( m_\alpha \) of \( \alpha \) is the non-zero polynomial with smallest degree such that \( m_\alpha(\alpha) = 0 \).
\end{definition}
\begin{remark}
	If \( \dim V = n < \infty \), then \( \dim L(V) = n^2 \).
	In particular, the family \( \qty{I, \alpha, \dots, \alpha^{n^2}} \) cannot be free since it has \( n^2+1 \) entries.
	This generates a polynomial in \( \alpha \) which evaluates to zero.
	Hence, a minimal polynomial always exists.
\end{remark}
\begin{lemma}
	Let \( \alpha \in L(V) \) and \( p \in F[t] \) be a polynomial.
	Then \( p(\alpha) = 0 \) if and only if \( m_\alpha \) is a factor of \( p \).
	In particular, \( m_\alpha \) is well-defined and unique up to a constant multiple.
\end{lemma}
\begin{proof}
	Let \( p \in F[t] \) such that \( p(\alpha) = 0 \).
	If \( m_\alpha(\alpha) = 0 \) and \( \deg m_\alpha < \deg p \), we can perform the division \( p = m_\alpha q + r \) for \( \deg r < \deg m_\alpha \).
	Then \( p(\alpha) = m_\alpha(\alpha) q(\alpha) + r(\alpha) \).
	But \( m_\alpha(\alpha) = 0 \).
	But \( \deg r < \deg m_\alpha \) and \( m_\alpha \) is the smallest degree polynomial which evaluates to zero for \( \alpha \), so \( r \equiv 0 \) so \( p = m_\alpha q \).
	In particular, if \( m_1, m_2 \) are both minimal polynomials that evaluate to zero for \( \alpha \), we have \( m_1 \) divides \( m_2 \) and \( m_2 \) divides \( m_1 \).
	Hence they are equivalent up to a constant.
\end{proof}
\begin{example}
	Let \( V = F^2 \) and
	\[
		A= \begin{pmatrix}
			1 & 0 \\
			0 & 1
		\end{pmatrix};\quad B = \begin{pmatrix}
			1 & 1 \\
			0 & 1
		\end{pmatrix}
	\]
	We can check \( p(t) = (t-1)^2 \) gives \( p(A) = p(B) = 0 \).
	So the minimal polynomial of \( A \) or \( B \) must be either \( (t-1) \) or \( (t-1)^2 \).
	For \( A \), we can find the minimal polynomial is \( (t-1) \), and for \( B \) we require \( (t-1)^2 \).
	So \( B \) is not diagonalisable, since its minimal polynomial is not a product of distinct linear factors.
\end{example}
