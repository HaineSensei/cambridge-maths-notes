\subsection{Statement and proof}
Consider a functional
\[
	F[\vb y] = \int_\alpha^\beta f(y_i, y_i', x) \dd{x};\quad i = 1,\dots,n
\]
Suppose there exists a one-parameter family of transformations
\[
	y_i(x) \mapsto Y_i(x,s);\quad Y_i(x,0) = y_i(x)
\]
This can be thought of as a change of variables parametrised by \( s \in \mathbb R \), where \( s = 0 \) implies no change of variables.
This family is called a \textit{continuous symmetry} of the Lagrangian \( f \) if
\[
	\dv{s} f(Y_i(x,s), Y_i'(x,s), x) = 0
\]
In this course, we only consider continuous symmetries, so they may be abbreviated as just `symmetries'.
\begin{theorem}[Noether's Theorem]
	Given a continuous symmetry \( Y_i(x,s) \) of \( f \),
	\[
		\eval{\pdv{f}{y_i'}\pdv{Y_i}{s}}_{s=0}
	\]
	is a first integral of the Euler-Lagrange equation (where the summation convention applies).
\end{theorem}
\begin{proof}
	\begin{align*}
		0                          & = \eval{\dv{s} f}_{s=0}                                                                           \\
		                           & = \eval{\pdv{f}{y_i} \dv{Y_i}{s}}_{s=0} + \eval{\pdv{f}{y_i'}\pdv{Y_i'}{s}}_{s=0}                 \\
		                           & = \eval{\qty[\dv{x} \qty(\pdv{f}{y_i'})\dv{Y_i}{s} + \pdv{f}{y_i'}\dv{x}\qty(\dv{Y_i}{s})]}_{s=0} \\
		                           & = \dv{x}\eval{\qty[\pdv{f}{y_i'}\pdv{Y_i}{s}]}_{s=0}                                              \\
		\therefore \text{constant} & = \pdv{f}{y_i'}\pdv{Y_i}{s}
	\end{align*}
\end{proof}

\subsection{Conservation of momentum}
\begin{example}
	Consider a vector \( \vb y = (y,z) \) and the function
	\[
		f = \frac{1}{2}y'^2 + \frac{1}{2}z'^2 - V(y-z)
	\]
	Consider the symmetry
	\begin{align*}
		Y = y + s                  & \implies Y' = y'      \\
		Z = z + s                  & \implies Z = z'       \\
		\therefore V(Y-Z) = V(y-z) & \implies \dv{s} f = 0
	\end{align*}
	Then from Noether's theorem,
	\[
		\text{constant} = \eval{\qty[\pdv{f}{y'}\dv{Y}{s} + \pdv{f}{z'}\dv{Z}{s}]}_{s=0} = y' + z'
	\]
	This can be thought of as a conserved momentum in the \(y+z\) direction.
\end{example}

\subsection{Conservation of angular momentum under central force}
\begin{example}
	Suppose \( \Theta = \theta + s, R = r \).
	Our space is isotropic, so \( \dv{L}{s} = 0 \), hence
	\[
		\eval{\qty[\pdv{L}{\dot\theta}\pdv{\Theta}{s} + \pdv{L}{\dot r}\pdv{R}{s}]}_{s=0} = mr^2\dot\theta
	\]
	which shows that angular momentum is conserved.
\end{example}
