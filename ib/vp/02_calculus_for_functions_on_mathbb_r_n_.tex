\subsection{Introduction}
Let \(f \in C^2(\mathbb R^n)\), so \(f \colon \mathbb R^n \to \mathbb R\) with all continuous second partial derivatives.
We say that the point \( \vb a \in \mathbb R^n \) is stationary if
\[
	\grad{f(\vb a)} = \vb 0
\]
Consider a Taylor series expansion near a stationary point.
\[
	f(\vb x) = f(\vb a) + \frac{1}{2} (x_i - a_i)(x_j - a_j) \eval{\partial^2_{ij} f}_{\vb a} + O(\norm{\vb x - \vb a}^2)
\]
The Hessian matrix is defined as \(H_{ij} = \partial_i \partial_j f = H_{ji}\), where \(\partial_i \equiv \pdv{}{x_i}\).
For convenience, we will shift the origin to let \(\vb a = \vb 0\).
The Hessian, evaluated at \(\vb 0\), written \(H(\vb 0)\), is a real symmetric matrix and hence can be diagonalised using an orthogonal transformation.
\[
	H' = R^\transpose H(\vb 0) R = \begin{pmatrix}
		\lambda_1 & 0         & \cdots & 0         \\
		0         & \lambda_2 & \cdots & 0         \\
		\vdots    & \vdots    & \ddots & \vdots    \\
		0         & 0         & \cdots & \lambda_n
	\end{pmatrix}
\]
Then
\[
	f(\vb x') - f(\vb 0) = \frac{1}{2} \sum \lambda_i (x_i')^2 + O(\norm{\vb x}^2)
\]
We can characterise the stationary point using the eigenvalues of the Hessian.
\begin{enumerate}[(i)]
	\item If all \(\lambda_i > 0\), then \(f(\vb x') > f(\vb 0)\) so \(f(\vb x')\) is a local minimum.
	\item If all \(\lambda_i < 0\), then \(f(\vb x') < f(\vb 0)\) so \(f(\vb x')\) is a local maxmimum.
	\item If the eigenvalues have mixed signs, this is a saddle point.
	      \(f(\vb x')\) increases in some directions, but decreases in other directions.
	\item If some eigenvalues are zero, we must consider higher-order terms of the Taylor expansion.
\end{enumerate}
When \(n = 2\), this is a special case.
We can compute properties of the eigenvalues using the trace and determinant of the matrix.
\[
	\det H = \lambda_1 \lambda_2;\;\tr H = \lambda_1 + \lambda_2
\]
\begin{enumerate}[(i)]
	\item If \(\det H > 0,\,\tr H > 0\) then we have a local minimum.
	\item If \(\det H > 0,\,\tr H < 0\) then we have a local maximum.
	\item If \(\det H < 0\) then we have a saddle point.
	\item If \(\det H = 0\) we need to consider higher-order terms.
\end{enumerate}

Note that if \(f \colon D \to \mathbb R\) where \(D \subset \mathbb R^n\), it is possible that we have a local maximum which is not the global maximum, if such a global maximum actually lies on the boundary and is not a stationary point.

Now, let us suppose that \(f\) is harmonic, i.e.\ \(\laplacian{f(\vb x)} = 0\) on \(D \subset \mathbb R^2\).
Hence, \(\tr H = 0\) which implies that if there exists a turning point it is a saddle point.
The minimum or maximum of a harmonic function must therefore occur on the boundary.

\begin{example}
	Let
	\[
		f(x, y) = x^3 + y^3 - 3xy
	\]
	\[
		\grad{f(\vb x)} = \begin{pmatrix}
			3x^2 - 3y \\ 3y^2 - 3x
		\end{pmatrix} = \begin{pmatrix}
			0 \\ 0
		\end{pmatrix} \implies \begin{pmatrix}
			x \\ y
		\end{pmatrix} = \begin{pmatrix}
			0 \\ 0
		\end{pmatrix} \text{ or } \begin{pmatrix}
			1 \\ 1
		\end{pmatrix}
	\]
	The Hessian is
	\[
		H = \begin{pmatrix}
			6x & -3 \\ -3 & 6y
		\end{pmatrix} \implies H(\vb 0) = \begin{pmatrix}
			0 & -3 \\ -3 & 0
		\end{pmatrix};\quad H\begin{pmatrix}
			1 \\ 1
		\end{pmatrix} = \begin{pmatrix}
			6 & -3 \\ -3 & 6
		\end{pmatrix}
	\]
	The determinant is negative at zero, giving us a saddle point.
	At the other point, the determinant is positive and the trace is positive, giving a local minimum.
\end{example}

\subsection{Constraints and Lagrange multipliers}
\begin{example}
	Find the circle centered at \((0, 0)\) with smallest radius that intersects the parabola \(y = x^2 - 1\).
	There are essentially two approaches.
	\begin{itemize}
		\item First, we consider the `direct' method.
		      We solve the constraints directly, which in this case means solving the equations
		      \begin{align*}
			      f & = x^2 + y^2 \\
			      y & = x^2 - 1
		      \end{align*}
		      for minimal \(f\).
		      This gives
		      \[
			      f = x^2 + (x^2 - 1)^2 = x^4 - x^2 + 1
		      \]
		      Then by setting \(\partial_x f = 0\) we have
		      \[
			      4x^3 - 2x = 0 \implies x \in \qty{0, \frac{1}{\sqrt 2}, \frac{-1}{\sqrt 2}}
		      \]
		      which gives
		      \[
			      x = \frac{\pm 1}{\sqrt{2}} \implies y = \frac{-1}{2};\; r = \frac{\sqrt{3}}{2}
		      \]
		      The other solution for \(x\) yields a larger radius.
		      This method works fine for simple problems like this where the constraints are solvable.
		      Therefore, we present an alternative method that works in the more general case.
		\item This method uses `Lagrange multipliers'.
		      We define a new function
		      \[
			      h(x, y, \lambda) = f(x, y) - \lambda g(x, y)
		      \]
		      where \(g(x, y)\) is defined such that \(g = 0\) is the constraint.
		      \(\lambda\) is called the Lagrange multiplier.
		      In this example,
		      \[
			      h(x, y, \lambda) = x^2 + y^2 - \lambda (y - x^2 + 1)
		      \]
		      We now extremise \(h\) over all free variables without constraints.
		      \[
			      \grad{h} = \begin{pmatrix}
				      \pdv*{h}{x} \\ \pdv*{h}{y} \\ \pdv*{h}{\lambda}
			      \end{pmatrix} = \begin{pmatrix}
				      2x + 2\lambda x \\
				      2y - \lambda    \\
				      y - x^2 + 1
			      \end{pmatrix}
		      \]
		      Solving \(\grad{h} = 0\), we have
		      \[
			      2x + 4xy = 0 \implies x = 0 \text{ or } y = \frac{-1}{2}
		      \]
		      and the same results follow as before by substitution.
	\end{itemize}
\end{example}

\subsection{Geometric justification of Lagrange multipliers}
Consider a curve given by \(g = 0\).
At each point on this curve, there is a normal to the curve of gradient \(\grad{g}\).
In particular, \(\grad{g}\) is perpendicular to \(g = 0\).
The function \(f\) has gradient perpendicular to the function \(f = c\) for some constant \(c\).
So at the extremum, \(\grad{f} \propto \grad{g}\), so \(\grad{f - \lambda g} = 0\) for some \(\lambda\).
This guides the creation of the new function \(h\), for which we can optimise without constraints.
This same reasoning generalises to functions in higher dimensions and with multiple constraints.
