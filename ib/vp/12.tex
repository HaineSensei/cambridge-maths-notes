\subsection{Jacobi Accessory Condition}
Legendre tried to prove that \( P > 0 \) implied local minimality; obviously this was impossible due to the counterexample shown above.
However, the method he used is still useful to analyse, since we can find an actual sufficient condition using the same idea.
Let \( \phi(x) \) be any differentiable function of \( x \) on \( [\alpha, \beta] \).
Then note that
\[ \int_\alpha^\beta \dv{x} \qty( \phi \eta^2 ) \dd{x} = 0 \]
since \( \eta(\alpha) = \eta(\beta) = 0 \).
We can expand the integrand to give
\[ \int_\alpha^\beta \qty( \phi' \eta^2 + 2 \eta \eta' \phi ) \dd{x} = 0 \]
We can add this new zero to both sides of the second variation equation.
\[ \delta^2 F[y] = \frac{1}{2} \int_\alpha^\beta \qty( P(\eta')^2 + 2 \eta \eta' \phi + \qty(Q + \phi') \eta^2 ) \dd{x} \]
Now, suppose that \( P > 0 \) at a particular \( y \).
Then, we can complete the square on the integrand, giving
\[ \delta^2 F[y] = \frac{1}{2} \int_\alpha^\beta \qty( P\qty(\eta' + \frac{\phi}{P} \eta)^2 + \qty(Q + \phi' - \frac{\phi^2}{P}) \eta^2 ) \dd{x} \]
If we could choose a \( \phi \) such that the second bracket vanishes, then the integrand would be \( P\qty(\eta' + \frac{\phi}{P} \eta)^2 \).
The only way the integral can be zero is if \( \eta' + \frac{\phi}{P} \eta \equiv 0 \). Since \( \eta = 0 \) at \( \alpha \), we have \( \eta'(\alpha) = 0 \).
Hence, \( \eta \equiv 0 \) by the uniqueness of solutions to first order differential equations.
Therefore, by contradiction, the integrand is not identically zero, and the second variation is positive.
Now, such a \( \phi \) function is given by
\[ \phi^2 = P\qty(Q + \phi') \]
If a solution to this differential equation exists, then \( \delta^2 F[y] > 0 \).
We can transform this non-linear equation into a second order equation by the substitution \( \phi = -P \frac{u'}{u} \) for some function \( u \neq 0 \).
We have
\[ P \qty(\frac{u'}{u})^2 = Q - \qty(\frac{Pu'}{u})' = Q - \frac{(Pu')'}{u} + P \qty(\frac{u'}{u})^2 \]
Hence,
\[ -(Pu')' + Qu = 0 \]
This is known as the Jacobi accessory condition.
Note that ther left hand side is just \( \mathcal L(u) \), where \( \mathcal L \) is the Sturm-Liouville operator.

\subsection{Solving the Jacobi Condition}
We need to find a solution to \( \mathcal L(u) = 0 \), where \( u \neq 0 \) on \( [\alpha, \beta] \).
The solution we find may not be non-zero on a large enough interval, in which case we would not have a local minimum.

\begin{example}
    Consider
    \[ F[y] = \frac{1}{2} \int_\alpha^\beta \qty( (y')^2 - y^2 ) \dd{x} \]
    The second variation is
    \[ \delta^2 F[y] = \frac{1}{2} \int_\alpha^\beta \qty( (\eta')^2 - \eta^2 ) \dd{x} \]
    In this case, \( P = 1, Q = -1 \).
    The Jacobi accessory equation is
    \[ u'' + u = 0 \]
    We can solve this to find
    \[ u = A \sin x - B \cos x;\quad A,B \in \mathbb R \]
    We want this to be non-zero on the interval \( [\alpha, \beta] \).
    In particular,
    \[ \tan x \neq \frac{B}{A};\quad \forall x \in [\alpha, \beta] \]
    Note that \( \tan x \) repeats every \( \pi \), so if \( \abs{\beta - \alpha} < \pi \) we have a positive second variation for any stationary \( y \).
\end{example}

\begin{example}
    Consider again the geodesic on a sphere.
    \[ F[\theta] = \int \sqrt{\dd{\theta}^2 + \sin^2\theta \dd{\phi}^2} = \int \sqrt{(\theta')^2 + \sin^2\theta}\dd{\phi} \]
    We have already proven that critical points of this functional are segments of great circles.
    Considering an equatorial great circle (since all great circles are equatorial under a change of perspective),
    \[ \theta = \frac{\pi}{2} \]
    Consider \( \phi_1, \phi_2 \) on this great circle.
    The minor arc is clearly the shortest path, but the major arc is also a stationary point and must still be analysed.
    \[ P = 1;\quad Q = -1 \]
    Thus,
    \[ \delta^2 F\qty[\theta_0 = \frac{\pi}{2}] = \frac{1}{2} \int_{\phi_1}^{\phi_2} \qty((\eta')^2 - \eta^2) \dd{\phi} \]
    which is exactly the example from above.
    This is a minimiser if \( \abs{\phi_2 - \phi_1} < \pi \), which is exactly the condition of being a minor arc.
    If \( \phi_2 - \phi_1 = \pi \), we have an infinite amount of geodesics, since these represent antipodal points.
    The set of geodesics exhibit rotational symmetry.
\end{example}
