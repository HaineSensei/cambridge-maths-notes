\subsection{Conditions for local minimisers}
The Euler-Lagrange equation gives a necessary condition for a stationary point.
We cannot tell whether this leads to a minimum, a maximum, or a saddle point, just from the Euler-Lagrange equation.
We can analyse the nature of the stationary points by considering the second variation.
Consider the functional
\[
	F[y] = \int_\alpha^\beta f(x,y,y') \dd{x}
\]
where \( y \) is perturbed by a perturbation \( \varepsilon\eta \).
Let us assume that \( y \) is a solution to the Euler-Lagrange equation, so has no first variation.
We will then expand \( F[y+\varepsilon\eta] \) to second order.
\begin{align*}
	F[y+\varepsilon\eta]        & = \int_\alpha^\beta \qty[f(x,y+\varepsilon\eta,y'+\varepsilon\eta')] \dd{x}                                                                                  \\
	F[y+\varepsilon\eta] - F[y] & = \int_\alpha^\beta \qty[f(x,y+\varepsilon\eta,y'+\varepsilon\eta') - f(x,y,y')] \dd{x}                                                                      \\
	                            & = 0 + \varepsilon \underbrace{\int_\alpha^\beta \eta \qty( \pdv{f}{y} - \dv{x} \pdv{f}{y'} ) \dd{x}}_{\text{zero by Euler-Lagrange equation}}                \\
	                            & + \frac{1}{2} \varepsilon^2 \int_\alpha^\beta  \qty( \eta^2 \pdv[2]{f}{y} + \eta'^2 \pdv[2]{f}{(y')} + 2\eta\eta' \pdv{f}{y}{y'} ) \dd{x} + O(\varepsilon^3)
\end{align*}
The last term (excluding the \( \varepsilon^2 \) component) is called the second variation.
We write
\[
	\delta^2 F[y] \equiv \frac{1}{2}\int_\alpha^\beta \qty( \eta^2 \pdv[2]{f}{y} + \eta'^2 \pdv[2]{f}{(y')} + \dv{x} (\eta^2) \pdv{f}{y}{y'} ) \dd{x}
\]
Integrating the last term by parts, using \( \eta = 0 \) at \( \alpha, \beta \), we have
\[
	\delta^2 F[y] = \frac{1}{2}\int_\alpha^\beta \qty( Q\eta^2 + P(\eta')^2 ) \dd{x}
\]
where
\[
	P = \pdv[2]{f}{(y')};\quad Q = \pdv[2]{f}{y} - \dv{x} \qty(\pdv{f}{y}{y'})
\]
Thus, if \( y \) is a solution to the Euler-Lagrange equation, and also \( Q\eta^2 + P(\eta')^2 > 0 \) for all \( \eta \) vanishing at \( \alpha, \beta \), then \( y \) is a local minimiser of \( F \).

\begin{example}
	We will prove that the geodesic on a plane is a local minimiser of path length.
	The functional we will analyse is given by
	\[
		f = \sqrt{1 + (y')^2}
	\]
	Hence,
	\[
		P = \pdv[2]{f}{(y')} = \pdv{y'} \qty(\frac{y'}{\sqrt{1+(y')^2}}) = \frac{1}{(1+(y')^2)^{\frac{3}{2}}} > 0
	\]
	\[
		Q = 0
	\]
	Therefore the second variation is positive, so any \( y \) that satisfies the Euler-Lagrange equation minimises path length.
	In particular, straight lines minimise path length on the plane.
\end{example}

\subsection{Legendre condition for minimisers}
\begin{proposition}[Legendre condition]
	If \( y_0(x) \) is a local minimiser, then \( \eval{P}_{y=y_0} \geq 0 \).
\end{proposition}
\noindent We can say that the Legendre condition is a necessary condition for a minimiser.
In less formal terms, \( P \) is `more important' than \( Q \) when determining if a stationary point is a minimiser.
\begin{proof}
	This condition is not proven rigorously.
	However, the general idea of the proof is to construct a function \( \eta \) which is small everywhere (giving a small \( Q \) contribution), but oscillates very rapidly near some point \( x_0 \), at which \( P < 0 \).
	This gives a large \( P \) contribution which can overpower the \( Q \) contribution.
	Then this gives \( Q\eta^2 + P(\eta')^2 < 0 \) if there exists some \( x_0 \) where \( \eval{P}_{y=y_0} < 0 \).
\end{proof}

\noindent Note that the Legendre condition is not a sufficient condition for local minima, but \( P > 0 \) and \( Q \geq 0 \) is sufficient.

\begin{example}
	Consider again the brachistochrone problem.
	\[
		f = \sqrt{\frac{1 + (y')^2}{-y}}
	\]
	We have
	\[
		\pdv{f}{y} = -\frac{1}{2y}f
	\]
	\[
		\pdv{f}{y'} = \frac{y'}{\sqrt{1+(y')^2}\sqrt{-y}}
	\]
	Hence
	\[
		P = \frac{1}{(1+(y')^2)^\frac{3}{2} \sqrt{-y}} > 0
	\]
	\[
		Q = \frac{1}{2\sqrt{1 + (y^2)^2}y^2 \sqrt{-y}} > 0
	\]
	Hence the cycloid is a local minimiser of the time taken to travel between the two points.
\end{example}

\subsection{Associated eigenvalue problem}
When deriving the minimiser condition, we had the integrand
\[
	Q\eta^2 + P(\eta')^2
\]
We can integrate this by parts:
\[
	Q\eta^2 + \dv{x} (P \eta \eta') - \eta \dv{x} (P \eta')
\]
giving
\[
	\delta^2 F[y] = \frac{1}{2}\int_\alpha^\beta \eta \qty[-(P\eta')' + Q\eta]\dd{x}
\]
The bracketed term \( -(P\eta')' + Q\eta \) is known as the Sturm-Liouville operator acting on \( \eta \), denoted \( \mathcal L(\eta) \).
If there exists \( \eta \) such that \( \mathcal L(\eta) = -\omega^2\eta \), \( \omega \in \mathbb R\), and \( \eta(\alpha) = \eta(\beta) = 0 \), then \( y \) is not a minimiser, since the integrand will be \( -\omega^2\eta^2 < 0 \).

\begin{example}
	Consider
	\[
		F[y] = \int_0^\beta \qty((y')^2 - y^2)\dd{x}
	\]
	such that
	\[
		y(0) = y(\beta) = 0;\quad \beta \neq k\pi, k \in \mathbb N
	\]
	The Euler-Lagrange equation gives
	\[
		y'' + y = 0
	\]
	Thus, constrained to the boundary conditions, the only stationary point of \( F \) is
	\[
		y \equiv 0
	\]
	Analysing the second variation,
	\[
		\delta^2 F[0] = \frac{1}{2} \int_0^\beta \qty[\eta'^2 - \eta^2] \dd{x}
	\]
	giving
	\[
		P = 1 > 0;\quad Q < 0
	\]
	Let us now examine the eigenvalue problem, since we cannot find whether \( y \equiv 0 \) is a minimiser from what we know already.
	Consider the eigenvalue problem
	\[
		-\eta'' - \eta = -\omega^2 \eta;\quad \eta(0) = \eta(\beta) = 0
	\]
	Let us take
	\[
		\eta = A \sin(\frac{\pi x}{\beta})
	\]
	to give
	\[
		\qty(\frac{\pi}{\beta})^2 = 1 - \omega^2
	\]
	So this has a solution \( \omega > 0 \) if and only if \( \beta > \pi \).
	If \( P > 0 \), a problem may arise if the interval of integration is `too large' (in this case \( \beta > \pi \)).
	Next lecture we will make this notion precise.
\end{example}

\subsection{Jacobi accessory condition}
Legendre tried to prove that \( P > 0 \) implied local minimality; obviously this was impossible due to the counterexample shown above.
However, the method he used is still useful to analyse, since we can find an actual sufficient condition using the same idea.
Let \( \phi(x) \) be any differentiable function of \( x \) on \( [\alpha, \beta] \).
Then note that
\[
	\int_\alpha^\beta \dv{x} \qty( \phi \eta^2 ) \dd{x} = 0
\]
since \( \eta(\alpha) = \eta(\beta) = 0 \).
We can expand the integrand to give
\[
	\int_\alpha^\beta \qty( \phi' \eta^2 + 2 \eta \eta' \phi ) \dd{x} = 0
\]
We can add this new zero to both sides of the second variation equation.
\[
	\delta^2 F[y] = \frac{1}{2} \int_\alpha^\beta \qty( P(\eta')^2 + 2 \eta \eta' \phi + \qty(Q + \phi') \eta^2 ) \dd{x}
\]
Now, suppose that \( P > 0 \) at a particular \( y \).
Then, we can complete the square on the integrand, giving
\[
	\delta^2 F[y] = \frac{1}{2} \int_\alpha^\beta \qty( P\qty(\eta' + \frac{\phi}{P} \eta)^2 + \qty(Q + \phi' - \frac{\phi^2}{P}) \eta^2 ) \dd{x}
\]
If we could choose a \( \phi \) such that the second bracket vanishes, then the integrand would be \( P\qty(\eta' + \frac{\phi}{P} \eta)^2 \).
The only way the integral can be zero is if \( \eta' + \frac{\phi}{P} \eta \equiv 0 \).
Since \( \eta = 0 \) at \( \alpha \), we have \( \eta'(\alpha) = 0 \).
Hence, \( \eta \equiv 0 \) by the uniqueness of solutions to first order differential equations.
Therefore, by contradiction, the integrand is not identically zero, and the second variation is positive.
Now, such a \( \phi \) function is given by
\[
	\phi^2 = P\qty(Q + \phi')
\]
If a solution to this differential equation exists, then \( \delta^2 F[y] > 0 \).
We can transform this non-linear equation into a second order equation by the substitution \( \phi = -P \frac{u'}{u} \) for some function \( u \neq 0 \).
We have
\[
	P \qty(\frac{u'}{u})^2 = Q - \qty(\frac{Pu'}{u})' = Q - \frac{(Pu')'}{u} + P \qty(\frac{u'}{u})^2
\]
Hence,
\[
	-(Pu')' + Qu = 0
\]
This is known as the Jacobi accessory condition.
Note that ther left hand side is just \( \mathcal L(u) \), where \( \mathcal L \) is the Sturm-Liouville operator.

\subsection{Solving the Jacobi condition}
We need to find a solution to \( \mathcal L(u) = 0 \), where \( u \neq 0 \) on \( [\alpha, \beta] \).
The solution we find may not be non-zero on a large enough interval, in which case we would not have a local minimum.

\begin{example}
	Consider
	\[
		F[y] = \frac{1}{2} \int_\alpha^\beta \qty( (y')^2 - y^2 ) \dd{x}
	\]
	The second variation is
	\[
		\delta^2 F[y] = \frac{1}{2} \int_\alpha^\beta \qty( (\eta')^2 - \eta^2 ) \dd{x}
	\]
	In this case, \( P = 1, Q = -1 \).
	The Jacobi accessory equation is
	\[
		u'' + u = 0
	\]
	We can solve this to find
	\[
		u = A \sin x - B \cos x;\quad A,B \in \mathbb R
	\]
	We want this to be non-zero on the interval \( [\alpha, \beta] \).
	In particular,
	\[
		\tan x \neq \frac{B}{A};\quad \forall x \in [\alpha, \beta]
	\]
	Note that \( \tan x \) repeats every \( \pi \), so if \( \abs{\beta - \alpha} < \pi \) we have a positive second variation for any stationary \( y \).
\end{example}

\begin{example}
	Consider again the geodesic on a sphere.
	\[
		F[\theta] = \int \sqrt{\dd{\theta}^2 + \sin^2\theta \dd{\phi}^2} = \int \sqrt{(\theta')^2 + \sin^2\theta}\dd{\phi}
	\]
	We have already proven that critical points of this functional are segments of great circles.
	Considering an equatorial great circle (since all great circles are equatorial under a change of perspective),
	\[
		\theta = \frac{\pi}{2}
	\]
	Consider \( \phi_1, \phi_2 \) on this great circle.
	The minor arc is clearly the shortest path, but the major arc is also a stationary point and must still be analysed.
	\[
		P = 1;\quad Q = -1
	\]
	Thus,
	\[
		\delta^2 F\qty[\theta_0 = \frac{\pi}{2}] = \frac{1}{2} \int_{\phi_1}^{\phi_2} \qty((\eta')^2 - \eta^2) \dd{\phi}
	\]
	which is exactly the example from above.
	This is a minimiser if \( \abs{\phi_2 - \phi_1} < \pi \), which is exactly the condition of being a minor arc.
	If \( \phi_2 - \phi_1 = \pi \), we have an infinite amount of geodesics, since these represent antipodal points.
	The set of geodesics exhibit rotational symmetry.
\end{example}
