\subsection{Fundamental Lemma of Calculus of Variations}
Consider again the functional
\[
	F[y] = \int_\alpha^\beta f(x, y, y') \dd{x}
\]
where \(f\) is given, and \(f(\alpha, \wildcard, \wildcard)\) and \(f(\beta, \wildcard, \wildcard)\) are fixed.
Consider a small perturbation
\[
	y \mapsto y + \varepsilon \eta(x);\quad \eta(\alpha) = \eta(\beta) = 0
\]
In order to compute the functional for this new function, we first need an additional lemma.
\begin{lemma}[Fundamental Lemma of Calculus of Variations]
	If \( g \colon [\alpha, \beta] \to \mathbb R \) is continuous on this interval, and is such that
	\[
		\forall \eta \text{ continuous}, \eta(\alpha) = \eta(\beta) = 0,\; \int_\alpha^\beta g(x) \eta(x) \dd{x} = 0
	\]
	Then
	\[
		\forall x \in (\alpha, \beta),\; g(x) \equiv 0
	\]
\end{lemma}
\begin{proof}
	Suppose that there exists a value \(\overline{x} \in (\alpha, \beta)\) such that \(g(\overline{x}) \neq 0\).
	Without loss of generality suppose that this value is positive.
	Then, by continuity, there exists a sub-interval \([x_1, x_2] \subset (\alpha, \beta)\) where \(g(x) > c\) for some positive real \(c\) in this sub-interval.
	So we will construct an \(\eta\) such that \(\eta > 0\) in \([x_1, x_2]\) and \(\eta = 0\) outside this interval, for example
	\[
		\eta(x) = \begin{cases}
			(x-x_1)(x_2-x) & x \in [x_1, x_2] \\
			0              & \text{otherwise}
		\end{cases}
	\]
	Then the integrand is non-negative everywhere, and is lower bounded by a positive number:
	\[
		\int_\alpha^\beta g(x) \eta(x) > c\int_{x_1}^{x_2} (x-x_1)(x_2-x)\dd{x} > 0
	\]
	So this leads to a contradiction.
\end{proof}
\begin{remark}
	We call such an \(\eta\) function a `bump function'.
	In general it is possible to construct a \(C^k\) bump function, e.g.
	\[
		\eta = \begin{cases}
			[(x-x_1)(x_2-x)]^{k+1} & x \in [x_1, x_2] \\
			0                      & \text{otherwise}
		\end{cases}
	\]
\end{remark}

\subsection{Euler-Lagrange Equation}
Now, we can evaluate the original functional.
Using a Taylor expansion,
\begin{align*}
	F[y + \varepsilon \eta] & = \int_\alpha^\beta f(x, y+\varepsilon\eta, y'+\varepsilon\eta')                                        \\
	                        & = F[y] + \varepsilon\int_\alpha^\beta \qty(\pdv{f}{y}\eta + \pdv{f}{y'}\eta') \dd{x} + O(\varepsilon^2)
\end{align*}
For an extremum,
\[
	\eval{\dv{F}{\varepsilon}}_{\varepsilon = 0} = 0
\]
So we want the first order term to vanish, so
\[
	\varepsilon\int_\alpha^\beta \qty(\pdv{f}{y}\eta + \pdv{f}{y'}\eta') \dd{x} = 0
\]
Integrating by parts, we have
\begin{align*}
	0 & = \int_\alpha^\beta \qty(\pdv{f}{y}\eta - \dv{x} \qty(\pdv{f}{y'}\eta)) \dd{x} + \qty[\pdv{f}{y'}\eta]_\alpha^\beta \\
	  & = \int_\alpha^\beta \qty(\pdv{f}{y}\eta - \dv{x} \qty(\pdv{f}{y'}\eta)) \dd{x}                                      \\
	  & = \int_\alpha^\beta \underbrace{\qty(\pdv{f}{y} - \dv{x} \qty(\pdv{f}{y'}))}_{g(x)} \eta \dd{x}
\end{align*}
We can apply the lemma above, showing that a necessary condition for the optimum is
\[
	\dv{x} \qty(\pdv{f}{y'}) - \pdv{f}{y} = 0
\]
This is the Euler-Lagrange equation.
\begin{remark}
	Note that
	\begin{itemize}
		\item This can be seen as a second-order differential equation for \(y(x)\) with boundary conditions at \(\alpha\) and \(\beta\).
		\item The left hand side of the Euler-Lagrange equation is called a `functional derivative' of \(y\), and is written
		      \[
			      \fdv{F[y]}{y(x)}
		      \]
		      Sometimes, the notation
		      \[
			      \delta y = \varepsilon\eta(x)
		      \]
		      is used, but is not used in this course.
		      Note that in this notation,
		      \[
			      F[y + \delta y] = F[y] + \delta F[y];\quad \delta F[y] = \int_\alpha^\beta \qty[\fdv{F[y]}{y(x)} \delta y(x)]\dd{x}
		      \]
		\item Other boundary conditions, such as \(\eval{\pdv{f}{y'}}_{\alpha, \beta}\) can be used.
		\item Note that when computing the derivatives, we regard \(x, y, y'\) as independent;
		      \[
			      \pdv{f}{y} = \eval{\pdv{f}{y}}_{x, y'}
		      \]
		      We can also compute a total derivative, for instance
		      \[
			      \dv{x} = \pdv{x} + \pdv{y}y' + \pdv{y'}y''
		      \]
		      Note that these give different results.
		      As an example, let \(f(x, y, y') = x[(y')^2 - y^2]\).
		      Then
		      \[
			      \pdv{f}{x} = (y')^2 - y^2;\quad \pdv{f}{y} = -2xy;\quad \pdv{f}{y'} = 2xy'
		      \]
		      Hence
		      \[
			      \dv{f}{x} = (y')^2 - y^2 - 2xyy' + 2xy'y''
		      \]
	\end{itemize}
\end{remark}

\subsection{First Integral of Euler-Lagrange Equation (eliminating \(y\))}
In some cases, we can integrate the Euler-Lagrange equation to give a first-order ordinary differential equation.
Suppose \(f\) does not explicitly depend on \(y\).
Then
\[
	\pdv{f}{y} = 0 \implies \dv{x}\qty(\pdv{f}{y'}) = 0
\]
Hence,
\[
	\pdv{f}{y'} = c;\quad c \in \mathbb R
\]
\begin{example}
	Consider geodesics on \(\mathbb R^2\); we want to find curves on which the length is minimised.
	\[
		F[y] = \int_\alpha^\beta \sqrt{\dd{x}^2 + \dd{y}^2} = \int_\alpha^\beta \underbrace{\sqrt{1 + \dv{y}{x}^2}}_{f(y')}\dd{x}
	\]
	We can apply this `first integral' form of the Euler-Lagrange equation to get
	\[
		\frac{y'}{\sqrt{1 + (y')^2}} = c
	\]
	Hence \(y'\) is a constant, so let \(y' = m\) for \(m \in \mathbb R\).
	Hence \(y = mx + c\).
\end{example}
