\subsection{Euler-Lagrange Equation with Constraints}
Given a functional \( F[y] = \int_\alpha^\beta f(x,y,y') \dd{x} \), we would like to extremise \( F \) subject to \( G[y] = \int_\alpha^\beta g(x,y,y') \dd{x} = k \) for some constant \( k \).
We can use the method of Lagrange multipliers.
Instead of extremising \( F \), we will extremise
\[
	\Phi[y;\lambda] = F[y] - \lambda G[y]
\]
Thus, we replace \( f \) in the Euler-Lagrange equation with \( f - \lambda g \), giving
\[
	\dv{x}\qty(\pdv{y'}(f-\lambda g)) - \pdv{y}(f-\lambda g) = 0
\]

\subsection{Dido's Isoparametric Problem}
Given a fixed perimeter, we wish to find the simple and closed plane curve which maximises the enclosed area.
We can restrict ourselves to convex curves.
This is because any concave curve can be transformed into a convex curve with greater area and equal perimeter, by reflecting the non-convex region.
We will parametrise the curve in \( \mathbb R^2 \) by letting the minimal and maximal values of \( x \) be \( \alpha, \beta \).
Then, as we trace out the curve, \( x \) monotonically increases from \( \alpha \) to \( \beta \), and then monotonically decreases as we return from \( \beta \) to \( \alpha \).
This induces two functions \( y_1, y_2 \) on \( (\alpha, \beta) \) where \( y_2 > y_1 \).
The infinitesimal area is given by \( \dd{A} = (y_2 - y_1) \dd{x} \).
Thus, the area functional is given by
\[
	A[y] = \int_\alpha^\beta (y_2(x) - y_1(x))\dd{x} = \oint_C y(x) \dd{y}
\]
The constraint functional is
\[
	L[y] = \oint_C \dd{\ell} = \oint_C \sqrt{1 + (y')^2} \dd{x} = L
\]
where \( L \) is the fixed perimeter.
Using Lagrange multipliers, we can define
\[
	h = y - \lambda \sqrt{1 + (y')^2}
\]
Note that we do not need to consider a boundary term in the derivation of the Euler-Lagrange equation, since the curve has no boundary.
Using a first integral form of the Euler-Lagrange equation on \( h \), we have
\[
	k = h - y'\dv{h}{y'} = y - \lambda \sqrt{1 + (y')^2} + y' \lambda \frac{y'}{\sqrt{1 + (y')^2}} = y - \frac{\lambda}{\sqrt{1 + (y')^2}}
\]
for some constant \( k \).
Hence,
\[
	(y')^2 = \frac{\lambda^2}{(y - k)^2} - 1
\]
A solution here is the circle of radius \( \lambda \):
\[
	(x - x_0)^2 + (y - y_0)^2 = \lambda^2
\]
Here, \( L = 2 \pi \lambda \) so we can write the solution in terms of \( L \) instead, giving
\[
	(x - x_0)^2 + (y - y_0)^2 = \frac{L^2}{4\pi^2}
\]

\subsection{The Sturm-Liouville Problem}
Let \( \rho(x), \sigma(x) \) be defined for \( x \in [\alpha, \beta] \), and let \( \rho(x) > 0 \) on this interval.
Consider the functional
\[
	F[y] = \int_\alpha^\beta \qty[ \rho (y')^2 + \sigma y^2 ]\dd{x}
\]
Let us extremise \( F \) subject to the constraint
\[
	G[y] = \int_\alpha^\beta y^2 \dd{x} = 1
\]
We have
\[
	\Phi[y;\lambda]  F[y] - \lambda (G[y] - 1)
\]
This induces the integrand
\[
	h = \rho (y')^2 + \sigma y^2 - \lambda (y^2 - \frac{1}{\beta - \alpha})
\]
We consider the derivatives for the Euler-Lagrange equation:
\[
	\pdv{h}{y'} = 2 \rho y';\quad \pdv{h}{y} = 2 \sigma y - 2 \lambda y
\]
Hence,
\[
	-\dv{x}\qty(\rho y') + \sigma y = \lambda y
\]
We can write this as \( \mathcal L(y) = \lambda y \), where the \( \mathcal L \) is known as the Sturm-Liouville operator.
This is essentially an eigenvalue problem, since \( \mathcal L \) is a linear operator.
For example, if \( \rho = 1 \), this eigenvalue problem is exactly the time-independent Schr\"odinger equation where \( \sigma \) is the quantum-mechanical potential.

Suppose \( \sigma > 0 \).
Then the functional \( F[y] \) is also greater than zero.
Then, the positive minimum of \( F \) (if it exists) is the lowest eigenvalue.
\begin{proof}
	Using the result from the Euler-Lagrange equation, we can multiply by \( y \) and integrate by parts giving
	\begin{align*}
		-y\dv{x}\qty(\rho y') + \sigma y^2                         & = \lambda y^2                            \\
		F[y] - \underbrace{[y y' \rho]_\alpha^\beta}_{\text{zero}} & = \lambda \underbrace{G[y]}_{\text{one}}
	\end{align*}
	Thus, the lowest eigenvalue is the minimum of \( F[y] / G[y] \).
\end{proof}

\subsection{Multiple Dependent Variables}
Suppose we have some vector
\[
	\vb y(x) = (y_1(x), y_2(x), \dots, y_n(x))
\]
Suppose we want to extremise the functional
\[
	F[\vb y] = \int_\alpha^\beta f(x, y_1, \dots, y_n, y'_1, \dots, y'_n) \dd{x}
\]
If there is some critical point \( \vb y \), we perturb by a small amount \( \varepsilon\vb \eta = \varepsilon(\eta_1(x), \dots, \eta_n(x)) \), where \( \vb\eta(\alpha) = \vb\eta(\beta) = \vb 0 \).
Following the derivation of the one-dimensional Euler-Lagrange equation, we can deduce that
\[
	F[\vb y + \varepsilon \vb \eta] - F[\vb y] = \int_\alpha^\beta \sum_{i=1}^n \eta_i \qty(\dv{x}\pdv{f}{y_i'} - \pdv{f}{y_i})\dd{x} + \text{boundary term} + O(\varepsilon^2)
\]
We can apply the fundamental lemma, choosing \( \eta_i \) in a useful way, we can show that a necessary condition for a critical point is
\[
	\dv{x}\pdv{f}{y_i'} - \pdv{f}{y_i} = 0
\]
for all \( i \).
This is a second-order \textit{system} of \( n \) ODEs that we can solve.
If \( f \) does not depend on one of the \( y_i \), then we have a first integral form for this particular equation.
In particular, if \( \pdv{f}{y_j} \equiv 0 \) then \( \pdv{f}{y_j'} = \text{constant} \).
If \( f \) does not depend on \( x \), then we have \( f - \sum_i y_i' \pdv{f}{y_i'} = \text{constant} \).
