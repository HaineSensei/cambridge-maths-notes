\subsection{Zero-Sum Games}
\begin{definition}
	A \textit{zero-sum two-person game} is a scenario in which two players (denoted P1 and P2) have different actions they can take:
	\begin{itemize}
		\item P1 has \( m \) possible actions \( \qty{1, 2, \dots, m } \), and
		\item P2 has \( n \) possible actions \( \qty{1, 2, \dots, n } \); such that
	\end{itemize}
	if P1 plays move \( i \) and P2 plays move \( j \), then we say P1 `wins' an amount \( a_{ij} \) and P2 `loses' the same amount \( a_{ij} \).
	The matrix of results \( A \) is called the \textit{payoff matrix}.
	P1 chooses a row of the matrix, and P2 chooses a column, and the intersection is the outcome of the game.
\end{definition}
Suppose P1 plays first, and chooses row \( i \).
P1 knows that P2 will choose the row \( j \) such that \( a_{ij} \) is minimised, since that will maximise P2's winnings.
In particular, if P1 picks row \( i \) then they can expect to win \( \min_{j \in \qty{1,\dots,m}} a_{ij} \).
So P1 will try to solve the problem
\begin{alignat*}{2}
	 & \text{maximise}   & \qquad & \min_{j \in \qty{1,\dots,m}} a_{ij} \\
	 & \text{subject to} &        & i \in \qty{1,\dots,n}
\end{alignat*}
If P2 plays first, they will try to solve the problem
\begin{alignat*}{2}
	 & \text{minimise}   & \qquad & \max_{i \in \qty{1,\dots,n}} a_{ij} \\
	 & \text{subject to} &        & j \in \qty{1,\dots,m}
\end{alignat*}
\begin{example}
	Suppose the payoff matrix is
	\[
		A = \begin{pmatrix}
			1 & 2 \\ 3 & 4
		\end{pmatrix}
	\]
	P1 chooses a row, and P2 chooses a column.
	If P1 plays first, they choose row 2, then P2 chooses row 1, and the payoff is 3.
	If P2 plays first, they choose column 1, then P1 chooses row 2, and the payoff is again 3.
	Since the solution is the same for both problems, this point \( (2,1) \) is called a \textit{saddle point}.
	The value \( a_{21} = 3 \) is called the \textit{value} of the game.
\end{example}
\begin{example}
	Consider the payoff matrix
	\[
		A = \begin{pmatrix}
			4 & 2 \\ 1 & 3
		\end{pmatrix}
	\]
	If P1 plays first, they choose row 1, then P2 chooses column 2, and the payoff is 2.
	If P2 plays first, they choose column 2, then P1 chooses row 2, and the payoff is 3.
	Here, both players cannot play optimally simultaneously since different outcomes will occur depending on what they think their opponent will do.
\end{example}

\subsection{Mixed Strategies}
In a mixed strategy, the players are allowed to choose their action randomly.
Such mixed strategies are employed when we do not know what our opponent will pick; for example, when both players choose their option at the same time.
P1 picks action \( i \) with probability \( p_i \), and P2 picks action \( j \) with probability \( q_j \), such that \( \sum p_i = \sum q_j = 1 \).
Now, a player's strategy is encoded as a probability vector.
If P1 picks the mixed strategy \( (p_1, \dots, p_m) \), then the expected reward of P1 (if P2 picks a pure strategy \( j \)) is
\[
	\sum_i a_{ij} p_i
\]
The optimisation problem for P1 is
\begin{alignat*}{2}
	 & \text{maximise}   & \qquad & \min_{j \in \qty{1,\dots,n}} \sum_i a_{ij} p_i \\
	 & \text{subject to} &        & \sum p_i = 1                                   \\
	 &                   &        & \vb p \geq 0
\end{alignat*}
Equivalently, where \( \vb e = (1,1,\dots,1)^\transpose \),
\begin{alignat*}{2}
	 & \text{maximise}   & \qquad & v                               \\
	 & \text{subject to} &        & A^\transpose \vb p \geq v \vb e \\
	 &                   &        & \vb e^\transpose \vb p = 1      \\
	 &                   &        & \vb p \geq 0
\end{alignat*}
This \( v \) is the minimum value of \( A^\transpose \vb p \).
P2's optimisation problem is
\begin{alignat*}{2}
	 & \text{minimise}   & \qquad & \max_{i \in \qty{1,\dots,m}} \sum_i a_{ij} q_j \\
	 & \text{subject to} &        & \sum q_j = 1                                   \\
	 &                   &        & \vb q \geq 0
\end{alignat*}
or equivalently,
\begin{alignat*}{2}
	 & \text{minimise}   & \qquad & w                          \\
	 & \text{subject to} &        & A \vb q \leq w \vb e       \\
	 &                   &        & \vb e^\transpose \vb q = 1 \\
	 &                   &        & \vb q \geq 0
\end{alignat*}

\subsection{Duality of Mixed Strategy Problems}
The two problems above are duals of each other.
Adding slack variables, P2's problem is
\begin{alignat*}{2}
	 & \text{minimise}   & \qquad & w                          \\
	 & \text{subject to} &        & A \vb q + \vb s = w \vb e  \\
	 &                   &        & \vb e^\transpose \vb q = 1 \\
	 &                   &        & \vb q \geq 0               \\
	 &                   &        & \vb s \geq 0
\end{alignat*}
The Lagrangian of this problem is
\begin{align*}
	L(w, \vb q, \vb s, \vb\lambda_1, \lambda_2) & = w + \vb\lambda_1^\transpose (A\vb q + \vb s - w\vb e) - \lambda_2 (\vb e^\transpose \vb q - 1)                                                    \\
	                                            & = w(1 - \vb\lambda_1^\transpose \vb e) + (\vb\lambda_1^\transpose A - \lambda_2 \vb e^\transpose) \vb q + \vb \lambda_1^\transpose\vb s + \lambda_2
\end{align*}
Thus,
\[
	\vb\Lambda = \qty{\vb\lambda \colon \vb\lambda_1^\transpose \vb e = 1, \vb\lambda_1^\transpose A - \lambda_2\vb e^\transpose \geq 0, \vb \lambda_1 \geq 0}
\]
When \( \vb\lambda \in \vb\Lambda \),
\[
	\inf L = \lambda_2
\]
Hence the dual is
\begin{alignat*}{2}
	 & \text{maximise}   & \qquad & \lambda_2                                                 \\
	 & \text{subject to} &        & \vb\lambda_1^\transpose \vb e = 1                         \\
	 &                   &        & \vb\lambda_1^\transpose A \geq \lambda_2 \vb e^\transpose \\
	 &                   &        & \vb \lambda_1 \geq 0
\end{alignat*}
Note that \( \vb \lambda_1 = \vb p \) and \( \lambda_2 = v \) in the above formulation of P1's problem.
\begin{theorem}
	A strategy \( \vb p \) is optimal for P1 if there exist \( \vb q, v \) such that
	\begin{itemize}
		\item (primal feasibility) \( A^\transpose \vb p \geq v \vb e, \vb e^\transpose \vb p = 1, \vb p \geq 0 \);
		\item (dual feasibility) \( Aq \leq v\vb e, \vb e^\transpose \vb q = 1, \vb q \geq 0 \); and
		\item (complementary slackness) \( v = \vb p^\transpose A \vb q \)
	\end{itemize}
\end{theorem}
\begin{proof}
	\( (\vb p, v) \) and \( (\vb q, w) \) are optimal if
	\[
		(A \vb q - w \vb e)^\transpose \vb p = 0; \vb q^\transpose(A^\transpose \vb p - v \vb e) = 0
	\]
	which gives
	\[
		v = w = \vb p^\transpose A \vb q
	\]
\end{proof}

\subsection{Finding Optimal Strategies}
There are a number of strategies for finding optimal strategies.
\begin{enumerate}
	\item We can search for saddle points in the payoff matrix.
	      If such a saddle point is found, a pure strategy aiming for this saddle point is optimal for both players.
	\item We can search for \textit{dominating actions}.
	      Suppose there exist \( i, i' \) such that \( a_{ij} \geq a_{i'j} \) for all \( j \).
	      Then \( i \) dominates \( i' \), so P1 will never play \( i' \) and we can simply drop this row in the matrix.
	      A similar technique can be used to drop columns.
	\item If these simplification techniques are not sufficient, we can simply solve the linear program using (for instance) the simplex method.
\end{enumerate}
\begin{example}
	Suppose we have a payoff matrix
	\[
		A = \begin{pmatrix}
			2 & 3 & 4           \\
			3 & 1 & \frac{1}{2} \\
			1 & 3 & 2
		\end{pmatrix}
	\]
	First, observe that there is no saddle point.
	Note that the first row dominates the last row, so we can simplify the payoff matrix to
	\[
		\widetilde{A} = \begin{pmatrix}
			2 & 3 & 4           \\
			3 & 1 & \frac{1}{2} \\
		\end{pmatrix}
	\]
	P1's strategy is \( \vb p = (p, 1-p, 0) \), and the optimisation problem is
	\begin{alignat*}{2}
		 & \text{maximise}   & \qquad & v                                           \\
		 & \text{subject to} &        & \widetilde{A}^\transpose \vb p \geq v \vb e \\
		 &                   &        & \vb e^\transpose \vb p = 1                  \\
		 &                   &        & \vb p \geq 0
	\end{alignat*}
	which is
	\begin{alignat*}{2}
		 & \text{maximise}   & \qquad & v                            \\
		 & \text{subject to} &        & 2p + 3(1-p) \geq v           \\
		 &                   &        & 3p + (1-p) \geq v            \\
		 &                   &        & 4p + \frac{1}{2}(1-p) \geq v \\
		 &                   &        & 0 \leq p \leq 1
	\end{alignat*}
	and by simplifying,
	\begin{alignat*}{2}
		 & \text{maximise}   & \qquad & v                                 \\
		 & \text{subject to} &        & v \leq 3 - p                      \\
		 &                   &        & v \leq 1 + 2p                     \\
		 &                   &        & v \leq \frac{1}{2} + \frac{7}{2}p \\
		 &                   &        & 0 \leq p \leq 1
	\end{alignat*}
	We can solve this graphically since it is a one-dimensional problem, or use the simplex method.
	We arrive at the solution \( \vb p = \qty(\frac{2}{3}, \frac{1}{3}, 0) \), i.e.
	\( p = \frac{2}{3} \).
	The payoff is \( \frac{7}{3} \).
	Player 2 has the dual optimisation problem, so we can use complementary slackness to compute P2's strategy.
	The first two constraints are tight, but the final constraint may not be (since it is zero in P1's strategy).
	Therefore \( q_3 = 0 \), and P2's strategy is \( \vb q = (q, 1-q, 0) \).
	Since the value of the game is \( \frac{7}{3} \), we have
	\[
		\frac{7}{3} = \vb p^\transpose A \vb q
	\]
	which lets us find \( q \).
	Alternatively, we can use complementary slackness.
	Since \( p_1, p_2 > 0 \), the first two constraints in the dual problem must be tight.
	\[
		2q + 3(1-q) = \frac{7}{3} \implies q = \frac{2}{3}
	\]
\end{example}
