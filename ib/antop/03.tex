\subsection{General principle of uniform convergence}
\begin{definition}
	A sequence \( (f_n) \) of scalar functions on a set \( S \) is called \textit{uniformly Cauchy} if
	\[
		\forall \varepsilon > 0, \exists N \in \mathbb N, \forall m, n \geq N, \forall x \in S, \abs{f_m(x) - f_n(x)} < \varepsilon
	\]
\end{definition}
\begin{theorem}
	A uniformly Cauchy sequence of functions is uniformly convergent.
\end{theorem}
\begin{proof}
	Let \( x \in S \) and we will show that \( (f_n(x))_{n=1}^\infty \) converges.
	Given \( \varepsilon > 0, \exists N \in \mathbb N, \forall m, n \geq N, \forall t \in S, \abs{f_m(t) - f_n(t)} < \varepsilon \).
	In particular, \( \forall m,n \geq N, \abs{f_m(x) - f_n(x)} < \varepsilon \).
	So certainly \( (f_n(x))_{n=1}^\infty \) is Cauchy and hence convergent by the general principle of convergence.
	Therefore \( f_n \) converges pointwise.
	Now, let \( f(x) \) be the limit \( f(x) = \lim_{n\to\infty} f_n(x) \).
	Then \( f_n \to f \) pointwise on \( S \).
	Now we must extend this to show \( f_n \to f \) uniformly on \( S \).
	Given \( \varepsilon > 0 \), we know that \( \exists N \in \mathbb N, \forall m, n \geq N, \forall x \in S, \abs{f_m(x) - f_n(x)} < \varepsilon \).
	Now, we must show \( \forall n \geq N, \forall x \in S, \abs{f_n(x) - f(x)} < 2 \varepsilon \), then we are done.
	We will fix \( x \in S, n \geq N \).
	Since \( f_n(x) \to f(x) \), we can choose \( m \in \mathbb N \) such that \( \abs{f_m(x) - f(x)} < \varepsilon \), and \( m \geq N \).
	Note however that \( m \) depends on \( x \) in this statement, but this doesn't matter --- we have shown that
	\[
		\abs{f_n(x) - f(x)} \leq \abs{f_n(x) - f_m(x)} + \abs{f_m(x) - f(x)} \leq \varepsilon + \varepsilon = 2 \varepsilon
	\]
	which is a result that, in itself, does \textit{not} depend on \( x \).
\end{proof}
\begin{note}
	Alternatively, we could end the proof as the following.
	Fix \( x \in S, n \geq N \).
	Then
	\[
		\forall m \geq N, \abs{f_n(x) - f_m(x)} < \varepsilon
	\]
	Then let \( m \to \infty \), and
	\[
		\abs{f_n(x) - f(x)} \leq \varepsilon
	\]
\end{note}

\subsection{Weierstrass M-test}
\begin{theorem}
	Let \( (f_n) \) be a sequence of scalar functions on \( S \).
	Assume that \( \forall n \in \mathbb N, \exists M_n \in \mathbb R^+, \forall x \in S, \abs{f_n(x)} \leq M_n \).
	In other words, \( (f_n) \) is a sequence of bounded scalar functions.
	Then,
	\[
		\sum_{n = 1}^\infty M_n < \infty \implies \sum_{n=1}^\infty f_n(x) \text{ is uniformly convergent on } S
	\]
\end{theorem}
\begin{proof}
	Let \( F_n(x) = \sum_{k=1}^\infty f_k(x) \) for \( x \in S, n \in \mathbb N \).
	Then
	\[
		\abs{F_n(x) - F_m(x)} \leq \sum_{k=m+1}^n \abs{f_k(x)} \leq \sum_{k=m+1}^n M_k
	\]
	Hence, given \( \varepsilon > 0 \), we can choose \( N \in \mathbb N \) such that \( \sum_{k=N+1}^n M_k < \varepsilon \).
	Thus, \( \forall x \in S, \forall n \geq m \geq N \), we have
	\[
		\abs{F_n(x) - F_m(x)} \leq \sum_{k=m+1}^n M_k < \varepsilon
	\]
	We have shown \( (F_n) \) is uniformly Cauchy on \( S \) and hence uniformly convergent on \( S \).
\end{proof}

\subsection{Power series}
Consider the power series
\[
	\sum_{n=0}^\infty c_n (z-a)^n
\]
where \( c_n \in \mathbb C, a \in \mathbb C \) are constants, and \( z \in \mathbb C \).
Let \( R \in [0, \infty] \) be the radius of convergence.
Recall that
\begin{align*}
	\abs{z-a} < R & \implies \sum_{n = 0}^\infty c_n (z-a)^n \text{ converges absolutely}; \\
	\abs{z-a} > R & \implies \sum_{n = 0}^\infty c_n (z-a)^n \text{ diverges}
\end{align*}
Let \( D(a, R) := \qty{z \in \mathbb C \mid \abs{z-a} < R} \) be the open disc centred on \( a \) with radius \( R \).
Then we can create \( f \colon D(a, \mathbb R) \to \mathbb C \) to be defined by the power series, which is well-defined.
\( f \) is the pointwise limit of the power series on \( D \).
In general, the convergence of the power series is not uniformly convergent.
\begin{example}
	\( \sum_{n=1}^\infty \frac{z^n}{n^2} \) has \( R = 1 \).
	Let \( f_n \colon D(0,1) \to \mathcal C \) be defined by \( f_n(z) = \frac{z^n}{n^2} \).
	Then for every \( z \in D(0,1), \abs{z} \leq \frac{1}{n^2} \).
	Since \( \sum_{n=1}^\infty \frac{1}{n^2} = \frac{\pi^2}{6} < \infty \), by the Weierstrass M-test, the power series converges uniformly on the disc.
\end{example}
\begin{example}
	Consider \( \sum_{n=0}^\infty z^n = \frac{1}{1-z} \) with \( R = 1 \).
	Now,
	\[
		\forall z \in D(0,1), \abs{\sum_{n=0}^\infty z^n} \leq N + 1
	\]
	Therefore, the series does not converge uniformly on the disc since \( \frac{1}{1-z} \) is unbounded on the disc.
	Alternatively, consider
	\[
		\sup_{\abs{z} < 1} \abs{\frac{1}{1-z} - \sum_{k=0}^n z_k} = \sup_{\abs{z} < 1} \abs{\frac{z^{n+1}}{1-z}} = \infty
	\]
	In some sense, the problem with uniform convergence here is that we are allowed to go too close too the boundary.
\end{example}
\begin{theorem}
	Suppose the power series \( \sum_{n=0}^\infty c_n (z-a)^n \) has radius of convergence \( R \).
	Then for all \( 0 < r < R \), the power series converges uniformly on \( D(a,r) \).
\end{theorem}
\begin{proof}
	Let \( w \in \mathbb C \) such that \( r < \abs{w - a} < R \), for instance \( w = a + \frac{r + R}{2} \).
	Now, let \( \rho = \frac{r}{\abs{w-a}} \in (0,1) \).
	Since \( \sum_{n=0}^\infty c_n (w-a)^n \) converges, we have that \( c_n (w-a)^n \to 0 \) as \( n \to \infty \).
	Therefore, \( \exists M \in \mathbb R+ \) such that \( \abs{c_n (w-a)^n} \leq M \) for all \( n \in \mathbb N \), since convergence implies boundedness.
	Now, for \( z \in D(a,r), n \in \mathbb N \) we have
	\[
		\abs{c_n(z-a)^n} = \abs{c_n(w-a)^n} \qty(\frac{\abs{z-a}}{\abs{w-a}})^n \leq M \qty(\frac{r}{\abs{w-a}})^n = M\rho^n
	\]
	Since the sum \( \sum_{n=0}^\infty M \rho^n \) converges, the Weierstrass M-test shows us that \( \sum_{n=0}^\infty c_n (z-a)^n \) converges uniformly on \( D(a,r) \).
\end{proof}
\begin{remark}
	\( f \colon D(a,R) \to \mathbb C \) defined by \( f(z) = \sum_{n=0}^\infty c_n (z-a)^n \) is the uniform limit on \( D(a,r) \) of polynomials for any \( r \) such that \( 0 < r < R \).
	Hence \( f \) is continuous on \( D(a,r) \).
	Since \( D(a,R) = \bigcup_{0 < r < R} D(a,r) \), it follows that \( f \) is continuous everywhere inside the radius of convergence.

	Recall that the termwise derivative \( \sum_{n=1}^\infty c_n n(z-a)^{n-1} \) has the same radius of convergence.
	This sequence therefore also converges uniformly on \( D(a,r) \) if \( 0 < r < R \).
	Analogously to the previous result about interchanging derivatives and sums, we can show that \( \sum c_n (z-a)^n \) is complex differentiable on \( D(a,R) \) with derivative \( \sum_{n=1}^\infty c_n n(z-a)^{n-1} \).
	This is seen in the IB Complex Analysis course.

	Now, fix \( w \in D(a,R) \).
	Then fix \( r \) such that \( \abs{w-a} < r < R \), and fix \( \delta > 0 \) such that \( \abs{w-a} + \delta < r \).
	If \( \abs{z-w} < \delta \), then \( \abs{z-a} \leq \abs{z-w} + \abs{w-a} < \delta + \abs{w-a} < r \).
	Therefore, geometrically, \( D(w,\delta) \subset D(a,r) \).
	Hence, \( \sum_{n=0}^\infty c_n (z-a)^n \) converges uniformly on \( D(w,\delta) \).
	This is known as local uniform convergence.
\end{remark}
\begin{definition}
	\( U \subset \mathbb C \) is called open if \( \forall w \in W, \exists \delta > 0, D(w,\delta) \subset U \).
\end{definition}
\begin{definition}
	Let \( U \) be an open subset of \( \mathbb C \), and \( f_n \) be a sequence of scalar functions on \( U \).
	Then \( f_n \) converges locally uniformly on \( U \) if
	\[
		\forall w \in U, \exists \delta > 0, f_n \text{ converges uniformly on } D(w,\delta) \subset U
	\]
\end{definition}
\begin{remark}
	Above, we showed that power series always converge locally uniformly inside the radius of convergence, or equivalently inside the disc \( D(a,R) \).
	We will return to this point about local uniform convergence when discussing compactness.
\end{remark}

\subsection{Uniform continuity}
Let \( U \subset \mathbb R, \mathbb C \).
Let \( f \) be a scalar function on \( U \).
Then for \( x \in U \), we say \( f \) is continuous at \( x \) if
\[
	\forall \varepsilon > 0, \exists \delta > 0, \forall y \in U, \abs{y-x} < \delta \implies \abs{f(y)-f(x)} < \varepsilon
\]
We say \( f \) is continuous on \( U \) if \( f \) is continuous at \( x \) for all \( x \in U \):
\[
	\forall x \in U, \forall \varepsilon > 0, \exists \delta > 0, \forall y \in U, \abs{y-x} < \delta \implies \abs{f(y)-f(x)} < \varepsilon
\]
Note here that \( \delta \) depends on \( \varepsilon \) and \( x \).
