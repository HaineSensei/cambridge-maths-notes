\subsection{Uniform Continuity}
\begin{definition}
Let \( U, f \) be as in the previous definition.
We say \( f \) is \textit{uniformly continuous} if
\[ \forall \varepsilon > 0, \exists \delta > 0, \forall x, y \in U, \abs{y-x} < \delta \implies \abs{f(y) - f(x)} < \varepsilon \]
Now, \( \delta \) works for all \( x \in U \) simultaneously; \( \delta \) depends on \( \varepsilon \) only.
Certainly, uniform continuity implies continuity.
\end{definition}
\begin{example}
Let \( f \colon \mathbb R \to \mathbb R \) such that \( f(x) = 2x + 17 \).
Then \( f \) is uniformly continuous; given \( \varepsilon > 0 \), we can find \( \delta = \frac{1}{2} \varepsilon \).
Then \( \forall x, y \in \mathbb R, \abs{y-x} < \delta \implies \abs{f(y)-f(x)} = \abs{2y-2x} = 2{y-x} < 2 \delta = \varepsilon \).
\end{example}
\begin{example}
Let \( f \colon \mathbb R \to \mathbb R \), defined by \( f(x) = x^2 \).
This is not uniformly continuous, since no \( \delta \) works for all \( x \) given some `bad` \( \varepsilon \).
Let us take \( \varepsilon = 1 \), and we wish to show that no \( \delta \) exists.
Suppose some \( \delta \) does exist.
Then, let \( x > 0 \) and \( y = x + \frac{\delta}{2} \).
We should have \( \abs{f(y) - f(x)} < 1 \).
\[ \qty(x + \frac{\delta}{2})^2 - x^2 = \delta x + \frac{\delta^2}{4} \]
So for \( x = \frac{1}{\delta} \), this condition \( \abs{f(y) - f(x)} < 1 \) is not satisfied.
Hence \( f \) is not uniformly continuous.
\end{example}
\begin{note}
For \( U, f \) as in the above definition, \( f \) is not uniformly continuous on \( U \) if
\[ \exists \varepsilon > 0, \forall \delta > 0, \exists x, y \in U, \abs{y-x} < \delta, \abs{f(y) - f(x)} \geq \varepsilon \]
So there are points arbitrarily close together whose difference of function values exceed some fixed \( \varepsilon \).
\end{note}
\begin{theorem}
Let \( f \) be a scalar function on a closed bounded interval \( [a,b] \).
If \( f \) is continuous on \( [a,b] \), then \( f \) is uniformly continuous on \( [a,b] \).
\end{theorem}
\begin{proof}
Suppose there exists \( \varepsilon > 0 \) such that \( \forall \delta > 0, \exists x,y \in [a,b], \abs{y-x} < \delta, \abs{f(y)-f(x)} < \varepsilon \).
In particular, we can construct a sequence \( (\delta_n) \) defined by \( \delta_n = \frac{1}{n} \), and we can construct sequences \( x_n, y_n \in [a,b] \) such that \( \abs{y_n-x_n}, \abs{f(y_n) - f(x_n)} \).
By the Bolzano-Weierstrass theorem, there exists a subsequence \( (x_{k_n}) \) that converges.
Now, let \( x \) be the limit of the subsequence, \( \lim_{n \to \infty} x_{k_n} \).
Then \( x \in [a,b] \) since the interval is closed.
Then, \( \abs{y_{k_n} - x} \leq \abs{y_{k_n} - x_{k_n}} + \abs{x_{k_n} - x} < \frac{1}{n} + \abs{x_{k_n} - x} \to 0 \).
Hence \( y_{k_n} \to x \).
Now, since \( f \) is continuous \( f(x_{k_n}), f(y_{k_n}) \to f(x) \).
Now, \( \varepsilon \leq \abs{f(x_{k_n}) - f(y_{k_n})} \to \abs{f(x) - f(x)} = 0 \), which is a contradiction.
\end{proof}
\begin{corollary}
A continuous function \( f \colon [a,b] \to \mathbb R \) is Riemann integrable.
\end{corollary}
\begin{proof}
Since a continuous function on a closed bounded interval is bounded, we have that \( f \) is bounded.
Now, fix \( \varepsilon > 0 \), and we want to find a dissection \( \mathcal D \) such that the difference between upper and lower sums is less than \( \varepsilon \).
By the above theorem, \( f \) is uniformly continuous.
Hence,
\[ \exists \delta > 0, \forall x, y \in [a,b], \abs{y-x} < \delta \implies \abs{f(y) - f(x)} < \varepsilon \]
So we must simply choose a dissection such that all intervals have size smaller than \( \delta \).
For instance, choose some \( n \in \mathbb N \) such that \( \frac{b-a}{N} < \delta \), and then divide the interval equally into \( n \) subintervals.
If \( I \) is an interval in this dissection, then \( \forall x,y \in I \) we have \( \abs{y-x} < \delta \) and hence \( \abs{f(y) - f(x)} < \varepsilon \).
Hence,
\[ \sup_{x,y \in I} \abs{f(y) - f(x)} \leq \varepsilon \]
Multiplying by the length of \( I \) and summing over all subintervals \( I \),
\[ U_{\mathcal D}(f) - L_{\mathcal D}(f) \leq (b-a) \varepsilon \]
Hence \( f \) is Riemann integrable.
\end{proof}

\subsection{Metric Spaces}
\begin{definition}
Let \( M \) be a set.
Then a \textit{metric} on \( M \) is a function \( d \colon M \times M \to \mathbb R \) such that
\begin{enumerate}[(i)]
\item (positivity) \( \forall x,y \in M, d(x,y) > 0 \), and in particular, \( x = y \iff d(x,x) = 0 \)
\item (symmetric) \( \forall x,y \in M, d(x,y) = d(y,x) \)
\item (triangle inequality) \( \forall x,y,z \in M, d(x,z) \leq d(x,y) + d(y,z) \).
\end{enumerate}
A metric space is a set \( M \) together with a metric \( d \) on \( M \), written as the pair \( (M, d) \).
\end{definition}
\begin{example}
Let \( M = \mathbb R, \mathbb C \) and \( d(x,y) = \abs{x-y} \).
This is known as the `standard metric' on \( M \).
If a metric is not specified, the standard metric is taken as implied.
\end{example}
\begin{example}
Let \( M = \mathbb R^n, \mathbb C^n \), and we define the Euclidean norm (or Euclidean length) to be
\[ \norm{x} = \norm{x}_2 = \qty(\sum_{k=1}^n \abs{x_k}^2)^{\frac{1}{2}} \]
This satisfies
\[ \norm{x+y} \leq \norm{x} + \norm{y} \]
and it then follows that we can define the metric as
\[ d_2(x,y) = \norm{x-y}_2 \]
called the Euclidean metric.
We can check that this is indeed a metric easily.
This is the standard metric on \( \mathbb R^n, \mathbb C^n \).
The metric space \( (M, d) \) in this case is called \( n \)-dimensional real (or complex) Euclidean space, sometimes denoted \( \ell_2^n \).
The Euclidean norm is sometimes called the \( \ell_2 \) norm, and the Euclidean metric is the \( \ell_2 \) metric.
\end{example}
\begin{example}
Let \( M = \mathbb R^n, \mathbb C^n \), and we define the \( \ell_1 \) norm to be
\[ \abs{x}_1 = \sum_{k=1}^n \abs{x_k} \]
which defines the \( \ell_1 \) metric given by
\[ d_1(x,y) = \norm{x-y}_1 \]
\( (M, d_1) \) is denoted \( \ell_1^n \).
We can generalise and form the metric space \( \ell_p^n \) for all \( p \in [1, \infty] \).
\end{example}
\begin{example}
Again, let \( M = \mathbb R^n, \mathbb C^n \).
We can define the \( \ell_\infty \) norm by
\[ \norm{x}_\infty = \max_{1 \leq k \leq n} \abs{x_k} \]
This defines the \( \ell_\infty \) metric:
\[ d_\infty(x,y) = \norm{x-y}_\infty = \max_{1 \leq k \leq n} \abs{x_k - y_k} \]
We denote \( (M, d) \) by \( \ell_\infty^n \).
\end{example}
\noindent \textit{In this course, we will only work with \( p = 1, 2, \infty \), although the calculations can be made to work for other \( p \).}
\begin{example}
Let \( S \) be a set.
Let \( \ell_\infty(S) \) be the set of all bounded scalar functions on \( S \).
We then define the \( \ell_\infty \) norm of \( f \in \ell_\infty(S) \) by
\[ \norm{f} = \norm{f}_\infty = \sup_{x \in S} \abs{f(x)} \]
The supremum exists since the function is always bounded.
This is also known as the `sup norm' or the `uniform norm'.
Note that, for \( f,g \in \ell_\infty(S) \), and \( x \in S \),
\[ \norm{f+g} \leq \sup_{x \in S} \abs{f(x) + g(x)} \leq \abs{f(x) + g(x)} \leq \abs{f(x)} + \abs{g(x)} \leq \norm{f} + \norm{g} \]
Hence \( d(f,g) =\norm{f-g} \) defines a metric on \( \ell_\infty(S) \).
This is the standard metric on this space \( \ell_\infty(S) \), also called the `uniform metric'.
For example, \( \ell_\infty(\qty{1, \dots, n}) = \mathbb R^n \) with the metric \( \ell_\infty \).
Also, for \( \ell_\infty(\mathbb N) \), we typically omit the \( \mathbb N \) and instead write \( \ell_\infty \) for the space of scalar sequences with the uniform metric.
\end{example}
\begin{example}
Consider \( C[a,b] \), the set of all continuous functions on \( [a,b] \).
For \( p = 1,2 \), we define the \( L_p \) norm of \( f \in C[a,b] \) by
\[ \norm{f}_p = \qty( \int_a^b \abs{f(x)}^p \dd{x} )^{\frac{1}{p}} \]
which induces the \( L_p \) metric on \( C[a,b] \).
\end{example}
