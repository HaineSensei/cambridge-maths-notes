\subsection{Lindel\"of-Picard theorem}
\begin{theorem}
	Let \( n \in \mathbb N \), \( y_0 \in \mathbb R^n \), and \( a,b,R \in \mathbb R \), such that \( a < b \) and \( R > 0 \).
	Let \( \phi \colon [a,b] \times \mathcal B_R(y_0) \to \mathbb R^n \) be a continuous function.
	Given that there exists \( K > 0 \) such that \( \forall t \in [a,b], \forall x,y \in \mathcal B_R(y_0) \), such that
	\[
		\norm{\phi(t,x) - \phi(t,y)} \leq K \norm{x-y}
	\]
	Then, \( \exists \varepsilon > 0 \) such that \( \forall t, t_0 \in [a,b] \), the initial value problem
	\[
		f'(t) = \phi(t, f(t));\quad f(t_0) = y_0
	\]
	has a unique solution on \( [c,d] = [t_0 - \varepsilon, t_0 + \varepsilon] \cap [a,b] \).
\end{theorem}
\begin{remark}
	If \( f \) is a solution of the initial value problem, implicitly this includes the assumption that \( f(t) \in B_r(y_0) \) for all \( t \in [c,d] \).
	Note that if \( f \colon [c,d] \to \mathbb R^n \), we let \( f_k \colon [c,d] \to \mathbb R \) be the \( k \)th component of \( f \), and \( f_k = q_k \circ f \) where \( q_k \) is the \( k \)th coordinate projection.
	Then, \( f(t) = (f_1(t), \dots, f_n(t)) \) and we define \( f \) to be differentiable if and only if all of the components are differentiable, with \( f'(t) = (f_1'(t), \dots, f_n'(t)) \).
	Note further, if \( f \) is continuous, then so are \( f_k \), hence \( f_k \) are integrable.
	So we define
	\[
		\int_c^d f(t) \dd{t} = v = \qty(\int_c^d f_1(t) \dd{t}, \dots, \int_c^d f_n(t) \dd{t})
	\]
	Note that we can use the Cauchy-Schwarz inequality to give
	\begin{align*}
		\norm{v}^2 & = \sum_{k=1}^n v_k^2                            \\
		           & = \sum_{k=1}^n v_k \int_c^d f_k(t) \dd{t}       \\
		           & = \int_c^d \sum_{k=1}^n v_k f_k(t) \dd{t}       \\
		           & = \int_c^d v \cdot f(t) \dd{t}                  \\
		           & \leq \int_c^d \norm{v} \cdot \norm{f(t)} \dd{t} \\
		           & = \norm{v} \int_c^d \norm{f(t)} \dd{t}
	\end{align*}
	Hence,
	\[
		\norm{\int_c^d f(t) \dd{t}} \leq \int_c^d \norm{f(t)} \dd{t} \leq (d-c) \sup_{t \in [c,d]} \norm{f(t)}
	\]
\end{remark}
\begin{proof}
	Recall that closed balls are closed, hence \( \mathcal B_R(y_0) \) is a closed subset of \( \mathbb R^n \).
	So \( \phi \) is a continuous function on the closed and bounded set \( [a,b] \times \mathcal B_R(y_0) \).
	It follows that \( \phi \) is bounded.
	Now, let \( c = \sup \qty{ \norm{\phi(t,x)} \colon t \in [a,b], x \in \mathcal B_R(y_0) } \).
	Let \( \varepsilon = \min(\frac{R}{c}, \frac{1}{2K}) \).
	Let \( t_0 \in [a,b] \) and let \( [c,d] = [t_0 - \varepsilon, t_0 + \varepsilon] \cap [a,b] \).
	We need to show that there exists a unique differentiable function \( f \colon [c,d] \to \mathbb R^n \) such that \( f(t_0) = y_0 \) and \( f'(t) = \phi(t,f(t)) \) for all \( t \in [c,d] \).
	Since \( \mathcal B_R(y_0) \) is closed in \( \mathbb R^n \), and since \( \mathbb R^n \) is complete, \( \mathcal B_R(y_0) \) is complete.
	Then, \( M = C([c,d], \mathcal B_R(y_0)) \) is complete in the uniform metric \( D \).
	This is certainly non-empty; consider the constant function yielding \( y_0 \).
	\( f \) is a solution to the initial value problem if \( f \in M \) and \( f'(t) = y_0 + \int_{t_0}^t \phi(s, f(s)) \dd{s} \), from the fundamental theorem of calculus applied coordinatewise.
	We define \( T \colon M \to M \) mapping \( g \mapsto Tg \) where \( Tg \) is given by
	\[
		(Tg)(t) = y_0 + \int_{t_0}^t \phi(s, g(s)) \dd{s}
	\]
	We must show \( T \) is well defined.
	First, note that the integral is well defined; \( s \mapsto \phi(s,g(s)) \) is continuous so integrable.
	By the fundamental theorem of calculus, \( Tg \) is differentiable and the derivative is \( (Tg)'(t) = \phi(t,g(t)) \).
	In particular, \( Tg \colon [c,d] \to \mathbb R^n \) is continuous.
	Finally, for \( t \in [c,d] \),
	\[
		\norm{(Tg)(t) - y_0} = \norm{\int_{t_0}^t \phi(s, g(s)) \dd{s}} \leq \abs{t - t_0} \sup_{s \in [c,d]} \norm{\phi(s,g(s))} \leq \varepsilon c \leq R
	\]
	So \( Tg \in M \).
	Recall that \( f \) is a solution of the initial value problem if and only if \( f \in M \) and \( Tf = f \).
	Now we must show that \( T \) has a unique fixed point, so we will show that \( T \) is a contraction.
	Let \( t \in [c,d] \) and \( g,h \in M \).
	\[
		\norm{(Tg)(t) - (Th)(t)} = \norm{\int_{t_0}^t \qty[ \phi(s,g(s)) - \phi(s,h(s)) ] \dd{s}}
	\]
	Note that \( \norm{\phi(s,g(s)) - \phi(s,h(s))} \leq K \norm{g(s) - h(s)} \leq K \cdot D(g,h) \).
	\[
		\norm{(Tg)(t) - (Th)(t)} = \abs{t-t_0} \cdot K \cdot K(g,h) \leq \varepsilon K D(g,h)
	\]
	Taking the supremum over \( t \in (c,d) \),
	\[
		D(Tg, Th) \leq \varepsilon K D(g,h) \leq \frac{1}{2} D(g,h)
	\]
	So \( T \) is a contraction.
	By the contraction mapping theorem, \( T \) has a unique fixed point in \( M \).
\end{proof}
\begin{remark}
	For any \( \delta \in (0,1) \), taking \( \varepsilon = \min(\frac{R}{c}, \frac{\delta}{K}) \) works.
	But by the uniqueness of the solution, the choice does not matter for constructing the solution.
	So we can construct the solution for \( \varepsilon = \min(\frac{R}{c}, \frac{1}{K}) \), on \( (t_0 - \varepsilon, t_0 + \varepsilon) \cap [a,b] \).
	In general, there is no solution on \( [a,b] \).
	Finally, note that the above theorem can handle any \( n \)th order ODE for any \( n \in \mathbb N \).
\end{remark}

\subsection{Topology}
\begin{definition}
	Let \( X \) be a set.
	A \textit{topology} on \( X \) is a family \( \tau \) of subsets of \( X \) (so \( \tau \subset \mathcal P(X) \)) such that
	\begin{enumerate}[(i)]
		\item \( \varnothing, X \in \tau \);
		\item if \( U_i \in \tau \) for all \( i \in I \) where \( I \) is some index set, then \( \bigcup_{i \in I} U_i \in \tau \); and
		\item if \( U, V \in \tau \) then \( U \cap V \in \tau \).
	\end{enumerate}
	A \textit{topological space} is a pair \( (X, \tau) \) where \( X \) is a set and \( \tau \) is a topology on \( X \).
	Members of \( \tau \) are called \textit{open sets} in the topology.
	So we say that \( U \subset X \) is \textit{open in} \( X \), or \( U \) is \( \tau \)\textit{-open}, if \( U \in \tau \).
\end{definition}
\begin{remark}
	If \( U_i \in \tau \) for \( i = 1, \dots, n \), then \( \bigcap_{i=1}^n U_i \in \tau \).
\end{remark}
\begin{example}
	Let \( (M, d) \) be a metric space.
	Recall that \( U \subset M \) is open in the metric sense if \( \forall x \in U, \exists r > 0, \mathcal B_r(x) \subset U \).
	We may say that \( U \) is \( d \)\textit{-open}.
	We have already proven that the family of \( d \)-open sets is a topology on \( M \).
	This is a metric topology.
\end{example}
\begin{definition}
	Let \( (X, \tau) \) be a topological space.
	Then we say that \( X \) is \textit{metrisable} (or sometimes we say \( \tau \) is metrisable) if there exists a metric \( d \) on \( X \) such that \( \tau \) is the metric topology on \( X \) induced by \( d \).
	In other words, \( U \subset X \) is \( \tau \)-open if and only if \( U \) is \( d \)-open.
	If \( d' \sim d \), then \( d' \) also induces the same topology \( \tau \) on \( X \).
\end{definition}
\begin{example}
	The indiscrete topology on a set \( X \) is a topology \( \tau = \qty{ \varnothing, X } \).
	If \( \abs{X} \geq 2 \), then this is not metrisable.
	Let \( d \) be a metric on \( X \).
	Then let \( x \neq Y \in X \), let \( r = d(x,y) \), and finally let \( U = \mathcal D_r(x) \).
	We know that \( U \) is \( d \)-open.
	But since \( x \in U, y \not\in U \), \( U \not\in \tau \).
\end{example}
\begin{definition}
	If \( \tau_1, \tau_2 \) are topologies on \( X \), we say that \( \tau_1 \) is \textit{coarser} than \( \tau_2 \), or that \( \tau_2 \) is finer than \( \tau_1 \), if \( \tau_1 \subset \tau_2 \).
	For example, the indiscrete topology on \( X \) is the coarsest topology on \( X \).
\end{definition}
