\subsection{Linear maps}
Let \( m, n \in \mathbb N \).
Recall that \( L(\mathbb R^m, \mathbb R^n) \) is the vector space of linear maps from \( \mathbb R^m \) to \( \mathbb R^n \).
This is isomorphic to \( M_{n,m} \), the space of \( n \times m \) real matrices.
There is also an isomorphism to \( \mathbb R^{mn} \).
Let \( e_1, \dots, e_m \) be the standard basis of \( \mathbb R^m \), and similarly let \( e_1', \dots, e_n' \) be the standard basis of \( \mathbb R^n \).
Then \( T \in L(\mathbb R^m, \mathbb R^n) \) is identified with the \( n \times m \) matrix \( (T_{ji}) \) where \( 1 \leq j \leq n \) and \( 1 \leq i \leq m \), such that \( T_{ji} = \inner{T e_i, e_j'} \).
We can therefore view \( L(\mathbb R^m, \mathbb R^n) \) as the \( mn \)-dimensional vector space \( \mathbb R^{mn} \) with the Euclidean norm.
So the norm of a linear map \( T \) is given by
\[
	\norm{T} = \sqrt{\sum_{i=1}^m \sum_{j=1}^n T_{ji}^2} = \sqrt{\sum_{i=1}^m \norm{Te_i}^2}
\]
where \( T e_i \) is the \( i \)th column of \( T \).
Thus, \( L(\mathbb R^m, \mathbb R^n) \) becomes a metric space together with the Euclidean distance \( d(S,T) = \norm{S-T} \).
\begin{lemma}
	For \( T \in L(\mathbb R^m, \mathbb R^n) \) and \( x \in \mathbb R^m \),
	\[
		\norm{Tx} \leq \norm{T} \cdot \norm{x}
	\]
	So \( T \) is a Lipschitz map and hence continuous.
	Further, if \( S \in L(\mathbb R^n, \mathbb R^p) \) then
	\[
		\norm{ST} \leq \norm{S} \cdot \norm{T}
	\]
\end{lemma}
\begin{proof}
	We can write
	\[
		x = \sum_{i=1}^m x_i e_i
	\]
	Hence,
	\[
		Tx = \sum_{i=1}^m x_i T e_i
	\]
	Thus,
	\[
		\norm{Tx} \leq \sum_{i=1}^m \abs{x_i} \norm{T e_i} \leq \qty(\sum_{i=1}^m x_i^2)^{1/2} \cdot \qty(\sum_{i=1}^m \norm{Te_i}^2)^{1/2} = \norm{T} \cdot \norm{x}
	\]
	Further, for \( x,y \in \mathbb R^m \) we have
	\[
		d(Tx, Ty) = \norm{Tx - Ty} = \norm{T(x-y)} \leq \norm{T} \cdot \norm{x-y} = \norm{T} d(x,y)
	\]
	So \( T \) is Lipschitz, and any Lipschitz function is continuous.
	Now,
	\[
		\norm{ST} = \qty(\sum_{i=1}^m \norm{STe_i}^2)^{1/2} \leq \qty(\sum_{i=1}^m \norm{S} \norm{Te_i}^2)^{1/2} = \norm{S} \qty(\sum_{i=1}^m \norm{Te_i}^2)^{1/2} = \norm{S} \cdot \norm{T}
	\]
\end{proof}

\subsection{Differentiation}
Recall from IA Analysis that a function \( f \colon \mathbb R \to \mathbb R \) is \textit{differentiable} at a point \( a \in \mathbb R \) if
\[
	\lim_{h \to 0} \frac{f(a+h) - f(a)}{h}
\]
exists.
The value of this limit is called the \textit{derivative} of \( f \) at \( a \), and denoted \( f'(a) \).
Note that \( f \) is differentiable at \( a \) if and only if there exists \( \lambda \in \mathbb R \) and \( \varepsilon \colon \mathbb R \to \mathbb R \) such that \( \varepsilon(0) = 0 \) and \( \varepsilon \) is continuous at \( 0 \), and
\[
	f(a+h) = f(a) + \lambda h + h \varepsilon(h)
\]
This is because we can define
\[
	\varepsilon(h) = \begin{cases}
		0                                 & h = 0    \\
		\frac{f(a+h) - f(a)}{h} - \lambda & h \neq 0
	\end{cases}
\]
Informally, this \( \varepsilon \) definition states that \( f \) is approximated very well (the error \( h\varepsilon(h) \) shrinks rapidly since \( \varepsilon \to 0 \)) by a linear function in a small neighbourhood of \( a \).
Recall that if \( f \) is \( n \) times differentiable at \( a \), then
\[
	f(a+h) = f(a) + \sum_{k=1}^n \frac{f^{(k)}(a)}{k!}h^k + o(h^n)
\]
\begin{definition}
	Let \( m, n \in \mathbb N \).
	Then \( f \colon \mathbb R^m \to \mathbb R^n \) and \( a \in \mathbb R^m \).
	We say that \( f \) is \textit{differentiable} at \( a \) if there exists a linear map \( T \in L(\mathbb R^m, \mathbb R^n) \) and a function \( \varepsilon \colon \mathbb R^m \to \mathbb R^n \) such that \( \varepsilon(0) = 0 \) and \( \varepsilon \) is continuous at \( 0 \), and
	\[
		f(a+h) = f(a) + T(h) + \norm{h} \varepsilon(h)
	\]
	Note that
	\[
		\varepsilon(h) = \begin{cases}
			0                                     & h = 0    \\
			\frac{f(a+h) - f(a) - T(h)}{\norm{h}} & h \neq 0
		\end{cases}
	\]
	So \( f \) is differentiable at \( a \) if and only if there exists \( T \in L(\mathbb R^m, \mathbb R^n) \) such that
	\[
		\frac{f(a+h) - f(a) - T(h)}{\norm{h}} \to 0
	\]
	as \( h \to 0 \).
	Such a \( T \) is unique.
	Indeed, suppose \( S, T \) satisfy the above limit.
	Then, by subtracting,
	\[
		\frac{S(h) - T(h)}{\norm{h}} \to 0
	\]
	For a fixed \( x \in \mathbb R^m \), \( x \neq 0 \), we have \( \frac{x}{k} \to 0 \) as \( k \to \infty \) so
	\[
		\frac{S\qty(\frac{x}{k}) - T\qty(\frac{x}{k})}{\norm{\frac{x}{k}}} \to 0 \implies \frac{S(x) - T(x)}{\norm{x}} = 0
	\]
	So \( Sx = Tx \).
	It follows that \( S = T \).
	We say that if a function \( f \) is differentiable at a point \( a \), \( T \) is the unique \textit{derivative} of \( f \) at \( a \).
	This is denoted \( f'(a) = Df(a) = \eval{Df}_a \).
	If \( f \colon \mathbb R^m \to \mathbb R^n \) is differentiable at \( a \in \mathbb R^m \) for every \( a \), we say that \( f \) is \textit{differentiable on} \( \mathbb R^m \).
	The function \( f' = D \colon \mathbb R^m \to L(\mathbb R^m, \mathbb R^n) \) mapping \( a \mapsto f'(a) \) is the derivative of \( f \).
\end{definition}
\begin{example}
	Constant functions are differentiable.
	Let \( f \colon \mathbb R^m \to \mathbb R^n \) such that \( f(x) = b \) for \( b \in \mathbb R^n \).
	Then for all \( a \in \mathbb R^m \), we have
	\[
		f(a+h) = f(a) + 0h + 0
	\]
	so \( f \) is differentiable at \( a \) and the derivative is zero.
\end{example}
\begin{example}
	Linear maps are differentiable.
	Let \( f \colon \mathbb R^m \to \mathbb R^n \) be defined by \( f(x) = Tx \) for a linear map \( T \in L(\mathbb R^m, \mathbb R^n) \).
	Then
	\[
		f(a+h) = f(a) + f(h) + 0
	\]
	so \( f \) is differentiable at \( a \) with derivative \( T = f \).
	So \( f' \) is a constant function.
\end{example}
\begin{example}
	Consider
	\[
		f(x) = \norm{x}^2
	\]
	For \( a \in \mathbb R^m \), we can find
	\[
		f(a+h) = \norm{a+h}^2 = \norm{a}^2 + 2\inner{a,h} + \norm{h}^2 = f(a) + 2\inner{a,h} + \norm{h} \varepsilon(h)
	\]
	Hence, \( f \) is differentiable with derivative
	\[
		f'(a)(h) = 2\inner{a,h}
	\]
	Note that \( f' \colon \mathbb R^m \to L(\mathbb R^m \to \mathbb R) \) is linear.
\end{example}
\begin{example}
	Note \( M_n \simeq \mathbb R^{n^2} \).
	The function \( f \colon M_n \to M_n \) given by \( f(A) = A^2 \).
	For a fixed \( A \in M_n \),
	\[
		f(A+H) = (A+H)^2 = A^2 + AH + HA + H^2
	\]
	It suffices to show \( H^2 \) is \( o(\norm{H}) \).
	We have \( \norm{H^2} \leq \norm{H}^2 \), hence
	\[
		\frac{\norm{H^2}}{\norm{H}} \leq \norm{H} \to 0
	\]
	So \( f \) is differentiable at \( A \) and the derivative is given by
	\[
		f'(A)(H) = AH + HA
	\]
\end{example}
\begin{example}
	Suppose \( f \colon \mathbb R^m \times \mathbb R^n \to \mathbb R^p \) is bilinear.
	Let \( (a, b) \in \mathbb R^m \times \mathbb R^n \).
	Then,
	\[
		f((a,b) + (h,k)) = f((a+h, b+k)) = f(a,b) + f(a,k) + f(h,b) + f(h,k)
	\]
	The map \( \mathbb R^m \times \mathbb R^n \to \mathbb R^p \) given by \( (h,k) \mapsto f(a,k) + f(h,b) \) is linear as the sum of two linear maps.
	So it suffices to show \( f(h,k) \) is \( o(\norm{(h,k)}) \).
	\[
		h = \sum_{i=1}^m h_i e_i;\quad k = \sum_{j=1}^n k_j e_j'
	\]
	Hence,
	\[
		f(h,k) = \sum_{i=1}^m \sum_{j=1}^n h_i k_j f(e_i, e_j') \implies \norm{f(h,k)} \leq \sum_{i=1}^m \sum_{j=1}^n \abs{h_i} \cdot \abs{k_j} \cdot \norm{f(e_i, e_j')} \leq C \norm{(h,k)}^2
	\]
	for some constant \( C \), since \( \abs{h_i} \leq \norm{(h,k)}^2 \) and similarly for \( \abs{k_j} \).
	So
	\[
		\frac{\norm{f(h,k)}}{\norm{(h,k)}} \leq C \norm{(h,k)} \to 0
	\]
	Hence \( f \) is differentiable with
	\[
		f'(a,b)(h,k) = f(a,k) + f(h,b)
	\]
\end{example}
