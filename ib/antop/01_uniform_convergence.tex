\subsection{Definition}
Recall that \( x_n \to x \) as \( n \to \infty \) (for \( x \in \mathbb R\) or \(\mathbb C\)) if
\[
	\forall \varepsilon > 0, \exists N \in \mathbb N, \forall n \geq N, \abs{x_n-x} < \varepsilon
\]
This is essentially considering the \( \varepsilon \)-neighbourhood of \( x \).
We aim to define the same notion of convergence for functions, by defining an analogous concept of an \( \varepsilon \)-neighbourhood.
In particular, each value on the domain should converge in its own \( \varepsilon \)-neighbourhood.
\begin{definition}
	Let \( S \) be a set, and \( f, f_n \colon S \to \mathbb R \), be functions.
	We say that \( (f_n) \) converges to \( f \) uniformly on \( S \) if
	\[
		\forall \varepsilon > 0, \exists N \in \mathbb N, \forall n \geq N, \forall x \in S, \abs{f_n(x) - f(x)} < \varepsilon
	\]
\end{definition}
\begin{note}
	\( N \) depends only on \( \varepsilon \), \textit{not} on any \( x \).
	Each \( x \) converges therefore at a `similar speed', hence the name `uniform convergence'.
\end{note}
\noindent Equivalently, we can write
\[
	\forall \varepsilon > 0, \exists N \in \mathbb N, \forall n \geq N, \sup_{x \in S} \abs{f_n(x) - f(x)} < \varepsilon
\]
The supremum condition is equivalent overall because the inequality on the right is weakened to a possible equality, but we can always decrease \( \varepsilon \) to retain the inequality.
Alternatively, we could write
\[
	\lim_{n \to \infty} \sup_{x \in S} \abs{f_n - f} = 0
\]
For each \( x \in S \), \( (f_n(x))_{n=1}^\infty \to f(x) \).
Hence, \( f \) is unique given \( (f_n) \), since limits are unique.
We call \( f \) the \textit{uniform limit} of \( (f_n) \) on \( S \).

\subsection{Pointwise convergence}
\begin{definition}
	\( (f_n) \) converges \textit{pointwise} to \( f \) on \( S \) if \( (f_n(x))_{n=1}^\infty \) converges to \( f(x) \) for every \( x \in S \).
	In other words,
	\[
		\underbrace{\forall x \in S}_{\mathclap{\text{order rearranged}}}, \forall \varepsilon > 0, \exists N \in \mathbb N, \forall n \geq N, \abs{f_n(x) - f(x)} < \varepsilon
	\]
	Now, \( N \) depends both on \( \varepsilon \) and on \( x \).
	Note that the pointwise limit of \( (f_n) \) on \( S \) is also unique since limits are unique.
\end{definition}
\begin{remark}
	Uniform convergence implies pointwise convergence, and the uniform limit is the pointwise limit.
\end{remark}

\begin{example}
	Let \( f_n(x) = x^2 e^{-nx} \) on \( [0, \infty), n \in \mathbb N \).
	Does \( (f_n) \) converge uniformly on the domain?
	First let us check pointwise convergence.
	We have \( x^2 e^{-nx} \to 0 \) hence pointwise convergence to \( f(x) = 0 \) is satisfied.
	Now, we need only check uniform convergence to the function \( f(x) = 0 \).
	\[
		\sup_{x \in [0, \infty)} \abs{f_n(x) - 0} = \sup_{x \in [0, \infty)} f_n(x)
	\]
	We could differentiate \( f_n \) and find the maximum if it exists, but we might not find the maximum if it is (for example) on the endpoints.
	A much better method is to find an upper bound on \( \abs{f_n(x)-f(x)} \) (which, in this example, is \( f_n(x) \)) that does not depend on \( x \).
	In this case, we can expand \( e^{nx} \) on the denominator and isolate a single term to get
	\[
		x^2 e^{-nx} = \frac{x^2}{e^{nx}} \leq \frac{2}{n^2};\quad \forall x
	\]
	Hence,
	\[
		\sup_{x \in [0, \infty)} \abs{f_n(x) - 0} \to 0
	\]
	and uniform convergence is satisfied.
\end{example}
\begin{example}
	Consider \( f_n(x) = x^n \) on \( [0,1], n \in \mathbb N \).
	A pointwise limit is reached by
	\[
		f(x) = \begin{cases}
			1 & x = 1            \\
			0 & \text{otherwise}
		\end{cases}
	\]
	Consider \( \sup \abs{f_n(x) - f(x)} \) excluding 1 (since at 1 the supremum is zero).
	Note \( f_n(x) \to 1 \) as \( x \to 1 \) from below, for all \( n \).
	Hence the supremum is always 1 by choosing an \( x \) sufficiently close to 1.
	So \( f_n \not\to f \) uniformly on \( [0,1] \), hence \( (f_n) \) does not converge at all uniformly on this domain.
	Or,
	\[
		\sup f_n(x) \geq f_n\qty(\qty(\frac{1}{2})^{1/n}) = \frac{1}{2}
	\]
\end{example}

\begin{remark}
	If \(f_n \not\to f\) uniformly on S,
	\[
		\exists \varepsilon > 0, \forall N \in \mathbb N, \exists n \geq N, \exists x \in S, \abs{f_n(x) - f(x)} \geq \varepsilon
	\]
	In the above example, we proved something stronger:
	\[
		\forall n, \exists x \in S, f_n(x) \geq \frac{1}{2}
	\]
	We could have alternatively stated, for example, \( f_n(x) \) is continuous so there exists some subset of \( [0, 1] \) greater than \( \frac{1}{2} \) always.
\end{remark}

\begin{theorem}
	Let \( S \subseteq \mathbb R, \mathbb C \).
	Let \( (f_n), f \colon S \to \mathbb R \text{(or } \mathbb{C} \text{)} \), where \( f_n \) is continuous and \( (f_n) \to f \) uniformly on \( S \).
	Then \( f \) is continuous.
\end{theorem}
\noindent Informally, the uniform limit of continuous functions is continuous.
\begin{proof}
	Fix some point \( a \in S \), \( \varepsilon > 0 \).
	We seek \( \delta > 0 \) such that \( \forall x \in S, \abs{x - a} < \delta \implies \abs{f(x) - f(a)} < \varepsilon \).
	We fix an \( n \in \mathbb N \) such that \( \forall x \in S, \abs{f_n(x) - f(x)} < \varepsilon \).
	Since \( f_n \) is continuous, there exists \( \delta > 0 \) such that \( \forall x \in S, \abs{x - a} < \delta \implies \abs{f_n(x) - f_n(a)} < \varepsilon \).
	So, \( \forall x \in S \),
	\[
		\abs{x - a} < \delta  \implies \abs{f(x) - f(a)} \leq \abs{f(x) - f_n(x)} + \abs{f_n(x) - f_n(a)} + \abs{f_n(a) - f(a)} < 3\varepsilon
	\]
\end{proof}
\begin{remark}
	The above proof is often called a \( 3\varepsilon \)-proof.
	Note, the proof is not true for pointwise convergence; if \( f_n \to f \) pointwise and \( f_n \) continuous, \( f \) is not necessarily continuous.
	Further, it is not true for differentiability; \( f_n \) differentiable does not imply \( f \) differentiable (see example sheet).
	Another way to interpret the result of the above theorem is to swap limits:
	\[
		\lim_{x \to a} \lim_{n \to \infty} f_n(x) = \lim_{x \to a} f(x) = f(a) = \lim_{n \to \infty} f_n(a) = \lim_{n \to \infty} \lim_{x \to a} f_n(x)
	\]
\end{remark}

\subsection{Uniform limit of bounded functions}
\begin{lemma}
	Let \( f_n \to f \) uniformly on \( S \).
	If \( f_n \) is bounded for every \( n \), then so is \( f \).
\end{lemma}
\noindent In other words, the uniform limit of bounded functions is bounded.
\begin{proof}
	Fix some \( n \in \mathbb N \) such that \( \forall x \in S, \abs{f_n(x) - f(x)} < 1 \).
	Since \( f_n \) is bounded, \( \exists M \in \mathbb R \) such that \( \forall x \in S, \abs{f_n(x)} < M \).
	Hence, \( \forall x \in S, \abs{f(x)} \leq \abs{f(x) - f_n(x)} + \abs{f_n(x)} \leq 1 + M \).
	So \( f \) is bounded.
\end{proof}

\subsection{Integrability}
Let \( f \colon [a, b] \to \mathbb R \) be a bounded function.
Recall that for a dissection \( \mathcal D \) of \( [a, b] \), we define the upper and lower sums of \( f \) with respect to \( \mathcal D \) by
\[
	U_{\mathcal D}(f) = \sum_{k=1}^n (x_k - x_{k-1}) \sup_{[x_{k-1}, x_k]} f(x)
\]
\[
	L_{\mathcal D}(f) = \sum_{k=1}^n (x_k - x_{k-1}) \inf_{[x_{k-1}, x_k]} f(x)
\]
Riemann's integrability criterion states that \( f \) is integrable if and only if
\[
	\forall \varepsilon, \exists \mathcal D, U_{\mathcal D}(f) - L_{\mathcal D}(f) < \varepsilon
\]
Equivalently, % exercise
for any \( I \subset [a, b] \), we have
\[
	\sup_I f - \inf_I f = \sup_{x,y \in I} \qty(f(x) - f(y)) = \sup_{x,y \in I} \abs{f(x) - f(y)}
\]
This is called the oscillation of \( f \) on \( I \).
So an integrable function `doesn't oscillate too much'.

\begin{theorem}
	Let \( f_n \colon [a,b] \to \mathbb R \) be integrable for all \( n \).
	If \( f_n \to f \) uniformly on \( [a,b] \), then \( f \) is integrable and
	\[
		\int_a^b f_n \to \int_a^b f
	\]
\end{theorem}
\begin{proof}
	First, we prove \( f \) to be bounded, then we will check Riemann's criterion.
	We know \( f \) is bounded because each \( f_n \) is bounded, hence by the lemma above \( f \) is bounded.
	Now fix \( \varepsilon > 0 \), and choose \( n \in \mathbb N \) such that \( \forall x \in [a,b], \abs{f_n(x) - f(x)} < \varepsilon \).
	Since \( f_n \) is integrable, \( \exists \mathcal D \colon a = x_0 < x_1 < \dots < x_N = b \) of \( [a,b] \) such that \( U_{\mathcal D} - L_{\mathcal D} < \varepsilon \).
	Now, we fix \( k \in \qty{1,\dots,N} \) and then for any \( x,y \in [x_{k-1}, x_k] \) we have
	\[
		\abs{f(x) - f(y)} \leq \abs{f(x) - f_n(x)} + \abs{f_n(x) - f_n(y)} + \abs{f_n(y) - f(y)} < 2\varepsilon + \abs{f_n(x) - f_n(y)}
	\]
	Taking the supremum,
	\[
		\sup_{x,y \in [x_{k-1},x_k]} \qty(f(x) - f(y)) \leq \sup_{x,y \in [x_{k-1},x_k]} \abs{f_n(x) - f_n(y)} + 2\varepsilon
	\]
	Multiplying by \( (x_k - x_{k-1}) \) and taking the sum over all \( k \),
	\[
		U(f) - L(f) \leq U(f_n) - L(f_n) + 2\varepsilon (b-a) \leq \varepsilon (2(b-a) + 1)
	\]
	Hence \( f \) is integrable.
	We can now show that
	\[
		\abs{\int_a^b f_n - \int_a^b f} \leq \int_a^b \abs{f_n - f} \leq (b-a) \sup_{[a,b]} \abs{f_n - f} \to 0
	\]
\end{proof}
\begin{remark}
	We can interpret this as
	\[
		\int_a^b \lim_{n \to \infty} f_n(x) \dd{x} = \lim_{n \to \infty} \int_a^b f_n(x) \dd{x}
	\]
	This is another `allowed' way to swap limits.
\end{remark}

\begin{corollary}
	Let \( f_n \colon [a,b] \to \mathbb R \) be integrable for all \( n \).
	If \( \sum_{n=1}^\infty f_n(x) \) converges uniformly on \( [a,b] \), then
	\[
		F(x) = \sum_{n=1}^\infty f_n(x)
	\]
	is integrable, and
	\[
		\int_a^b \sum_{n=1}^\infty f_n(x) \dd{x} = \sum_{n=1}^\infty \int_a^b f_n(x) \dd{x}
	\]
\end{corollary}
\begin{proof}
	Let \( F_n(x) = \sum_{k=1}^n f_k(x) \).
	By assumption, \( F_n \to F \) uniformly on \( [a,b] \).
	\( F_n \) is integrable where the integral of \( F_n \) is the sum of the integrals:
	\[
		\int_a^b F_n = \sum_{k=1}^n \int_a^b f_k
	\]
	Then the result follows from the theorem above.
\end{proof}

\subsection{Differentiability}
\begin{theorem}
	Let \( f_n \colon [a,b] \to \mathbb R \) be continuously differentiable for all \( n \).
	Suppse \( \sum_{k=1}^\infty f_k'(x) \) converges uniformly on \( [a,b] \), and that \( \exists c \in [a,b], \sum_{n-1}^\infty f_n(c) \) converges.
	Then, \( \sum_{k=1}^\infty f_k(x) \) converges uniformly on \( [a,b] \) to a continuously differentiable function \( f \), and
	\[
		\dv{x} \qty(\sum_{k=1}^\infty f_k) = \sum_{k=1}^\infty \dv{x} f_k(x)
	\]
\end{theorem}
\begin{proof}
	Let \( g(x) = \sum_{k=1}^\infty f_k'(x) \), for \( x \in [a,b] \).
	The general idea is that we want to solve the differential equation \( f' = g \) subject to the initial conditino \( f(c) = \sum_{n=1}^\infty f_n(c) \).
	Let \( \lambda = \sum_{n=1}^\infty f_n(c) \) and define \( f \colon [a,b] \to \mathbb R \) by
	\[
		f(x) = \lambda + \int_c^x g(t) \dd{t}
	\]
	Note that \( g \) is integrable; \( \sum_{k=1}^\infty f_k'(x) \to g \) uniformly implies that \( g \) is continuous and hence integrable.
	By the fundamental theorem of calculus, \( f' = g \) and \( f(c) = \lambda \).
	So we have found such an \( f \) that satisfies the conditions set out.
	All that remains is to prove uniform convergence of \( \sum_{k=1}^\infty f_k \to f \).
	Also by the fundamental theorem, \( f_k(x) = f_k(c) + \int_c^x f_k'(t) \dd{t} \).
	Let \( \varepsilon > 0 \).
	There exists \( N \in \mathbb N \) such that \( \abs{\lambda - \sum_{k=1}^N f_k(c)} < \varepsilon \) and \( \abs{g(t) - \sum_{k=1}^N f_k'(t)} < \varepsilon \).
	Now, for \( n \geq N \) we have
	\begin{align*}
		\abs{f(x) - \sum_{k=1}^n f_k(x)} & = \abs{\lambda + \int_c^x g(t) \dd{t} - \sum_{k=1}^n \qty(f_k(c) + \int_c^x f_k'(t) \dd{t})}       \\
		                                 & \leq \abs{\lambda - \sum_{k=1}^n f_k(c)} + \abs{\int_c^x \qty(g(t) - \sum_{k-1}^n f_k'(t)) \dd{t}} \\
		                                 & \leq \varepsilon + \abs{x-c} \varepsilon                                                           \\
		                                 & \leq \varepsilon (b-a + 1)
	\end{align*}
\end{proof}

\subsection{Conditions for uniform convergence}
Recall that a scalar sequence \( x_n \) is Cauchy if
\[
	\forall \varepsilon > 0, \exists N \in \mathbb N, \forall m,n \geq N, \abs{x_m - x_n} < \varepsilon
\]
and that the general principle of convergence shows that any Cauchy sequence converges.

\subsection{General principle of uniform convergence}
\begin{definition}
	A sequence \( (f_n) \) of scalar functions on a set \( S \) is called \textit{uniformly Cauchy} if
	\[
		\forall \varepsilon > 0, \exists N \in \mathbb N, \forall m, n \geq N, \forall x \in S, \abs{f_m(x) - f_n(x)} < \varepsilon
	\]
\end{definition}
\begin{theorem}
	A uniformly Cauchy sequence of functions is uniformly convergent.
\end{theorem}
\begin{proof}
	Let \( x \in S \) and we will show that \( (f_n(x))_{n=1}^\infty \) converges.
	Given \( \varepsilon > 0, \exists N \in \mathbb N, \forall m, n \geq N, \forall t \in S, \abs{f_m(t) - f_n(t)} < \varepsilon \).
	In particular, \( \forall m,n \geq N, \abs{f_m(x) - f_n(x)} < \varepsilon \).
	So certainly \( (f_n(x))_{n=1}^\infty \) is Cauchy and hence convergent by the general principle of convergence.
	Therefore \( f_n \) converges pointwise.
	Now, let \( f(x) \) be the limit \( f(x) = \lim_{n\to\infty} f_n(x) \).
	Then \( f_n \to f \) pointwise on \( S \).
	Now we must extend this to show \( f_n \to f \) uniformly on \( S \).
	Given \( \varepsilon > 0 \), we know that \( \exists N \in \mathbb N, \forall m, n \geq N, \forall x \in S, \abs{f_m(x) - f_n(x)} < \varepsilon \).
	Now, we must show \( \forall n \geq N, \forall x \in S, \abs{f_n(x) - f(x)} < 2 \varepsilon \), then we are done.
	We will fix \( x \in S, n \geq N \).
	Since \( f_n(x) \to f(x) \), we can choose \( m \in \mathbb N \) such that \( \abs{f_m(x) - f(x)} < \varepsilon \), and \( m \geq N \).
	Note however that \( m \) depends on \( x \) in this statement, but this doesn't matter --- we have shown that
	\[
		\abs{f_n(x) - f(x)} \leq \abs{f_n(x) - f_m(x)} + \abs{f_m(x) - f(x)} \leq \varepsilon + \varepsilon = 2 \varepsilon
	\]
	which is a result that, in itself, does \textit{not} depend on \( x \).
\end{proof}
\begin{note}
	Alternatively, we could end the proof as the following.
	Fix \( x \in S, n \geq N \).
	Then
	\[
		\forall m \geq N, \abs{f_n(x) - f_m(x)} < \varepsilon
	\]
	Then let \( m \to \infty \), and
	\[
		\abs{f_n(x) - f(x)} \leq \varepsilon
	\]
\end{note}

\subsection{Weierstrass M-test}
\begin{theorem}
	Let \( (f_n) \) be a sequence of scalar functions on \( S \).
	Assume that \( \forall n \in \mathbb N, \exists M_n \in \mathbb R^+, \forall x \in S, \abs{f_n(x)} \leq M_n \).
	In other words, \( (f_n) \) is a sequence of bounded scalar functions.
	Then,
	\[
		\sum_{n = 1}^\infty M_n < \infty \implies \sum_{n=1}^\infty f_n(x) \text{ is uniformly convergent on } S
	\]
\end{theorem}
\begin{proof}
	Let \( F_n(x) = \sum_{k=1}^\infty f_k(x) \) for \( x \in S, n \in \mathbb N \).
	Then
	\[
		\abs{F_n(x) - F_m(x)} \leq \sum_{k=m+1}^n \abs{f_k(x)} \leq \sum_{k=m+1}^n M_k
	\]
	Hence, given \( \varepsilon > 0 \), we can choose \( N \in \mathbb N \) such that \( \sum_{k=N+1}^n M_k < \varepsilon \).
	Thus, \( \forall x \in S, \forall n \geq m \geq N \), we have
	\[
		\abs{F_n(x) - F_m(x)} \leq \sum_{k=m+1}^n M_k < \varepsilon
	\]
	We have shown \( (F_n) \) is uniformly Cauchy on \( S \) and hence uniformly convergent on \( S \).
\end{proof}

\subsection{Power series}
Consider the power series
\[
	\sum_{n=0}^\infty c_n (z-a)^n
\]
where \( c_n \in \mathbb C, a \in \mathbb C \) are constants, and \( z \in \mathbb C \).
Let \( R \in [0, \infty] \) be the radius of convergence.
Recall that
\begin{align*}
	\abs{z-a} < R & \implies \sum_{n = 0}^\infty c_n (z-a)^n \text{ converges absolutely}; \\
	\abs{z-a} > R & \implies \sum_{n = 0}^\infty c_n (z-a)^n \text{ diverges}
\end{align*}
Let \( D(a, R) := \qty{z \in \mathbb C \mid \abs{z-a} < R} \) be the open disc centred on \( a \) with radius \( R \).
Then we can create \( f \colon D(a, \mathbb R) \to \mathbb C \) to be defined by the power series, which is well-defined.
\( f \) is the pointwise limit of the power series on \( D \).
In general, the convergence of the power series is not uniformly convergent.
\begin{example}
	\( \sum_{n=1}^\infty \frac{z^n}{n^2} \) has \( R = 1 \).
	Let \( f_n \colon D(0,1) \to \mathcal C \) be defined by \( f_n(z) = \frac{z^n}{n^2} \).
	Then for every \( z \in D(0,1), \abs{z} \leq \frac{1}{n^2} \).
	Since \( \sum_{n=1}^\infty \frac{1}{n^2} = \frac{\pi^2}{6} < \infty \), by the Weierstrass M-test, the power series converges uniformly on the disc.
\end{example}
\begin{example}
	Consider \( \sum_{n=0}^\infty z^n = \frac{1}{1-z} \) with \( R = 1 \).
	Now,
	\[
		\forall z \in D(0,1), \abs{\sum_{n=0}^\infty z^n} \leq N + 1
	\]
	Therefore, the series does not converge uniformly on the disc since \( \frac{1}{1-z} \) is unbounded on the disc.
	Alternatively, consider
	\[
		\sup_{\abs{z} < 1} \abs{\frac{1}{1-z} - \sum_{k=0}^n z_k} = \sup_{\abs{z} < 1} \abs{\frac{z^{n+1}}{1-z}} = \infty
	\]
	In some sense, the problem with uniform convergence here is that we are allowed to go too close too the boundary.
\end{example}
\begin{theorem}
	Suppose the power series \( \sum_{n=0}^\infty c_n (z-a)^n \) has radius of convergence \( R \).
	Then for all \( 0 < r < R \), the power series converges uniformly on \( D(a,r) \).
\end{theorem}
\begin{proof}
	Let \( w \in \mathbb C \) such that \( r < \abs{w - a} < R \), for instance \( w = a + \frac{r + R}{2} \).
	Now, let \( \rho = \frac{r}{\abs{w-a}} \in (0,1) \).
	Since \( \sum_{n=0}^\infty c_n (w-a)^n \) converges, we have that \( c_n (w-a)^n \to 0 \) as \( n \to \infty \).
	Therefore, \( \exists M \in \mathbb R+ \) such that \( \abs{c_n (w-a)^n} \leq M \) for all \( n \in \mathbb N \), since convergence implies boundedness.
	Now, for \( z \in D(a,r), n \in \mathbb N \) we have
	\[
		\abs{c_n(z-a)^n} = \abs{c_n(w-a)^n} \qty(\frac{\abs{z-a}}{\abs{w-a}})^n \leq M \qty(\frac{r}{\abs{w-a}})^n = M\rho^n
	\]
	Since the sum \( \sum_{n=0}^\infty M \rho^n \) converges, the Weierstrass M-test shows us that \( \sum_{n=0}^\infty c_n (z-a)^n \) converges uniformly on \( D(a,r) \).
\end{proof}
\begin{remark}
	\( f \colon D(a,R) \to \mathbb C \) defined by \( f(z) = \sum_{n=0}^\infty c_n (z-a)^n \) is the uniform limit on \( D(a,r) \) of polynomials for any \( r \) such that \( 0 < r < R \).
	Hence \( f \) is continuous on \( D(a,r) \).
	Since \( D(a,R) = \bigcup_{0 < r < R} D(a,r) \), it follows that \( f \) is continuous everywhere inside the radius of convergence.

	Recall that the termwise derivative \( \sum_{n=1}^\infty c_n n(z-a)^{n-1} \) has the same radius of convergence.
	This sequence therefore also converges uniformly on \( D(a,r) \) if \( 0 < r < R \).
	Analogously to the previous result about interchanging derivatives and sums, we can show that \( \sum c_n (z-a)^n \) is complex differentiable on \( D(a,R) \) with derivative \( \sum_{n=1}^\infty c_n n(z-a)^{n-1} \).
	This is seen in the IB Complex Analysis course.

	Now, fix \( w \in D(a,R) \).
	Then fix \( r \) such that \( \abs{w-a} < r < R \), and fix \( \delta > 0 \) such that \( \abs{w-a} + \delta < r \).
	If \( \abs{z-w} < \delta \), then \( \abs{z-a} \leq \abs{z-w} + \abs{w-a} < \delta + \abs{w-a} < r \).
	Therefore, geometrically, \( D(w,\delta) \subset D(a,r) \).
	Hence, \( \sum_{n=0}^\infty c_n (z-a)^n \) converges uniformly on \( D(w,\delta) \).
	This is known as local uniform convergence.
\end{remark}
\begin{definition}
	\( U \subset \mathbb C \) is called open if \( \forall w \in W, \exists \delta > 0, D(w,\delta) \subset U \).
\end{definition}
\begin{definition}
	Let \( U \) be an open subset of \( \mathbb C \), and \( f_n \) be a sequence of scalar functions on \( U \).
	Then \( f_n \) converges locally uniformly on \( U \) if
	\[
		\forall w \in U, \exists \delta > 0, f_n \text{ converges uniformly on } D(w,\delta) \subset U
	\]
\end{definition}
\begin{remark}
	Above, we showed that power series always converge locally uniformly inside the radius of convergence, or equivalently inside the disc \( D(a,R) \).
	We will return to this point about local uniform convergence when discussing compactness.
\end{remark}
