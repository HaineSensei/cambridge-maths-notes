\subsection{Continuity}
\begin{definition}
	Let \( f \colon M \to M' \) be a function between metric spaces \( (M, d), (M', d') \).
	Then for \( a \in M \), we say \( f \) is continuous at \( a \) if
	\[
		\forall \varepsilon > 0, \exists \delta > 0, \forall x \in M, d(x,a) < \delta \implies d'(f(x) - f(a)) < \varepsilon
	\]
	We say \( f \) is continuous if \( f \) is continuous at \( a \) for all \( a \in M \).
	In other words,
	\[
		\forall a \in M, \forall \varepsilon > 0, \exists \delta > 0, \forall x \in M, d(x,a) < \delta \implies d'(f(x) - f(a)) < \varepsilon
	\]
	Note that \( \delta \) depends both on \( \varepsilon \) and \( a \).
\end{definition}
\begin{proposition}
	Let \( f \colon M \to M' \) be as above.
	Let \( a \in M \).
	Then the following are equivalent:
	\begin{enumerate}[(i)]
		\item \( f \) is continuous at \( a \);
		\item \( x_n \to a \) in \( M \) implies \( f(x_n) \to f(a) \) in \( M \)
	\end{enumerate}
\end{proposition}
\begin{proof}
	First we show (i) implies (ii).
	Suppose \( x_n \to a \) in \( M \).
	Then fix \( \varepsilon > 0 \), and seek \( N \in \mathbb N \) such that \( \forall n \geq N, d'(f(x_n), f(a)) < \varepsilon \).
	By continuity, there exists \( \delta > 0 \) such that \( \forall x \in M, d(x,a) < \delta \implies d'(f(x_n), f(a)) < \varepsilon \) as required.
	So we want \( N \) such that \( \forall n \geq N, d(x,a) < \delta \), which must exist since \( x_n \to a \).

	Now, we show (ii) implies (i).
	Suppose that \( f \) is not continuous at \( a \).
	Then,
	\[
		\exists \varepsilon > 0, \forall \delta > 0, \exists x \in M, d(x,a) < \delta, d'(f(x), f(a)) \geq \varepsilon
	\]
	So fix such an \( \varepsilon \) for which no suitable \( \delta \) exists.
	Choose the sequence \( \delta_n = \frac{1}{n} \), so
	\[
		d(x_n,a) < \frac{1}{n};\quad d'(f(x_n), f(a)) \geq \varepsilon
	\]
	Then \( x_n \to a \) in \( M \) but \( f(x_n) \not\to f(a) \) in \( M \), which is a contradiction.
\end{proof}
\begin{proposition}
	Let \( f,g \) be scalar functions on a metric space \( M \).
	Let \( a \in M \).
	Then if \( f,g \) are continuous at \( a \), so are \( f+g \) and \( f \cdot g \).
	Moreover, letting \( N = \qty{x \in M \colon g(x) \neq 0} \) and assuming \( a \in N \), \( \frac{f}{g} \) is continuous at \( a \).
	Hence if \( f,g \) are continuous, then so are \( f+g, f \cdot g, \frac{f}{g} \) where they are defined.
\end{proposition}
\begin{proof}
	Suppose \( x_n \to a \).
	Then by the above proposition, \( (f\cdot g)(x_n) = f(x_n) \cdot g(x_n) \to f(a) \cdot g(a) = (f \cdot g)(a) \), and similar results hold for the other operators.
\end{proof}
\begin{remark}
	If \( f \colon M \to M' \) is continuous everywhere,
	\[
		\lim_{n \to \infty} f(x_n) = f\qty(\lim_{n \to \infty} x_n)
	\]
	by the second proposition.
\end{remark}
\begin{proposition}
	Let \( f \colon M \to M', g \colon M' \to M'' \) be functions between metric spaces.
	If \( f \) is continuous at \( a \) and \( g \) is continuous at \( f(a) \), then \( g \circ f \) is continuous at \( a \).
	If \( f,g \) are continuous, \( g \circ f \) is continuous.
\end{proposition}
\begin{proof}
	Let \( \varepsilon > 0 \).
	We want to find \( \delta > 0 \) such that \( \forall x \in M \), \( d(x,a) < \delta \) implies \( d''(g(f(x)), g(f(a))) < \varepsilon \).
	Since \( g \) is continuous at \( f(a) \), there exists \( \eta > 0 \) such that \( \forall y \in M' \), \( d'(y,f(a)) < \eta \implies d''(g(y), g(f(a))) < \varepsilon \).
	Now, since \( f \) is continuous at \( a \), for this \( \eta \) there exists \( \delta \) such that for all \( x \in M \), \( d(x,a) < \delta \implies d'(f(x) - f(a)) < \eta \).
	Then \( d(x,a) < \delta \implies d''(g(f(x)), g(f(a))) < \varepsilon \) as required.
\end{proof}

\begin{example}
	Constant functions are continuous.
	For instance, let \( b \in M \) and let \( f(x) = b \).
	Then this is continuous since \( d'(f(x) - f(a)) = d'(b,b) = 0 \) so any \( \delta > 0 \) will satisfy the condition.
\end{example}
\begin{example}
	The identity function \( f \colon M \to M \) defined by \( x \mapsto x \) is continuous.
	Consider \( d(f(x) - f(a)) = d(x-a) \).
	So \( \delta = \varepsilon \) will suffice.
\end{example}
\begin{example}
	All real and complex polynomials and rational functions are continuous wherever they are defined by the propositions and examples above.
	In fact, using uniform convergence, the uniform limits of such functions are also continuous.
	For example, exponential and trigonometric functions are continuous.
\end{example}
\begin{example}
	Let \( (M, d) \) be a metric space.
	Then \( d \colon M \oplus_p M \to \mathbb R \), which can be viewed as a function between metric spaces \( M \oplus_p M \) and \( \mathbb R \).
	Then, given \( v = (x,x'), w = (y,y') \in M \oplus_p M \),
	\[
		\abs{d(v) - d(w)} = \abs{d(x,x') - d(y,y')} \leq d(x,y) + d(x',y') = d_1(v,w) \leq 2 d_p(v,w)
	\]
	Hence \( \delta = \frac{\varepsilon}{2} \) will suffice.
\end{example}

\subsection{Isometric, Lipschitz, and Uniformly Continuous Functions}
\begin{definition}
	Let \( f \colon M \to M' \) be a function between metric spaces.
	Then, \( f \) is
	\begin{enumerate}[(i)]
		\item \textit{isometric}, if \( \forall x,y \in M, d'(f(x),f(y)) = d(x,y) \)
		\item \textit{Lipschitz}, or \( c \)-Lipschitz, if \( \exists c \in \mathbb R^+, \forall x,y \in M, d'(f(x),f(y)) \leq c\cdot d(x,y) \)
		\item \textit{uniformly continuous}, if \( \forall \varepsilon > 0, \exists \delta > 0, \forall x,y \in M, d(x,y) < \delta \implies d'(f(x), f(y)) < \varepsilon \)
	\end{enumerate}
\end{definition}
\begin{remark}
	Any isometric function is 1-Lipschitz.
	Any Lipschitz function is uniformly continuous.
	Any uniformly continuous function is continuous.
\end{remark}
\begin{remark}
	If a function is isometric, it is injective, since \( f(x) = f(y) \implies x = y \).
	For example, if \( N \subset M \), the inclusion map \( i \colon N \to M \) defined by \( i(x) = x \) is isometric but not surjective.
	An isometric and surjective map is called an \textit{isometry}.
	If there exists an isometry \( M \to M' \), we say that \( M \) and \( M' \) are isometric metric spaces, or \( M' \) is an isometric copy of \( M \).
\end{remark}
\begin{example}
	Suppose \( (M, d), (M', d') \) be metric spaces.
	Let \( y \in M' \).
	We define \( f \colon M \to M \oplus_p M' \) by \( x \mapsto (x,y) \).
	Then \( d_p(f(x),f(z)) = d_p((x,y), (z,y)) = d(x,z) \).
	So the function \( f \) is isometric.
	Therefore, \( M \times \qty{ y } \) is an isometric copy of \( M \) in \( M \oplus_p M' \).
\end{example}
\begin{example}
	Consider the projections \( q \colon M \oplus_p M' \to M \) defined by \( q(x,y) = x \) and \( q' \colon M \oplus_p M' \to M' \) defined by \( q'(x,y) = y \).
	These projections are both 1-Lipschitz.
	Indeed,
	\[
		d(q(x,y), q(x',y')) = d(x,x') \leq d_p((x,y), (x',y'))
	\]
	In particular, polynomials in any finite number of variables are continuous since we can multiply continuous functions together.
\end{example}
