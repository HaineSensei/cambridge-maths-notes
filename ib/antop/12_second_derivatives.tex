\subsection{Definition}
\begin{definition}
	Let \( U \subset \mathbb R^m \) be an open set, and \( f \colon U \to \mathbb R^n \).
	Let \( a \in U \).
	Suppose that there exists an open neighbourhood \( V \) of \( a \) contained within \( U \), and \( f \) is differentiable on \( V \).
	We say that \( f \) is \textit{twice differentiable} at \( a \) if \( f' \colon V \to L(\mathbb R^m \to \mathbb R^n) \) is differentiable at \( a \).
	We write \( f''(a) \) for the derivative of \( f' \) at \( a \), called the \textit{second derivative} of \( f \) at \( a \).
	Note that \( f''(a) \in L(\mathbb R^m, L(\mathbb R^m, \mathbb R^n)) \).
\end{definition}
\begin{remark}
	We can visualise the second derivative as a bilinear map instead of a nested sequence of linear maps.
	Note,
	\[
		L(\mathbb R^m, L(\mathbb R^m, \mathbb R^n)) \sim \mathrm{Bil}(\mathbb R^m \times \mathbb R^m, \mathbb R^n)
	\]
	where \( \mathrm{Bil}(X \times Y, Z) \) is the vector space of bilinear maps from \( X \times Y \) to \( Z \).
	For \( h, k \in \mathbb R^m \), and \( T \) is the second derivative, we can say \( T(h)(k) = \widetilde T(h,k) \) where \( \widetilde T \) is a bilinear map.
	From now on, this bilinear map notation will be used, and \( T \) and \( \widetilde T \) will be identified as the same.
\end{remark}
\begin{proposition}
	Let \( U \subset \mathbb R^m \) be open, \( f \colon U \to \mathbb R^n \) be a function, and \( a \in U \).
	Let \( f \) be differentiable on an open neighbourhood \( V \) of \( A \) contained in \( U \).
	Then \( f \) is twice differentiable at \( a \) if and only if there exists a bilinear map \( T \in \mathrm{Bil}(\mathbb R^m \times \mathbb R^m, \mathbb R^n) \) such that for every \( k \in \mathbb R^m \), we have
	\[
		f'(a+h)(k) = f'(a)(k) + T(h,k) + o(\norm{h})
	\]
	Then \( T = f''(a) \).
\end{proposition}
\begin{proof}
	Suppose \( f \) is twice differentiable at \( a \).
	Then \( f' \) is differentiable at \( a \).
	So,
	\[
		f'(a+h) = f'(a) + f''(a)(h) + \norm{h} \cdot \varepsilon(h)
	\]
	All terms are linear maps \( L(\mathbb R^m, \mathbb R^n) \).
	In particular, \( \varepsilon \) is defined on \( V - a \to L(\mathbb R^m, \mathbb R^n) \) such that \( \varepsilon(0) = 0 \) and \( \varepsilon \) is continuous at zero.
	If we evaluate this equation at a fixed \( k \in \mathbb R^m \),
	\[
		f'(a+h)(k) = f'(a)(k) + f''(a)(h,k) + \norm{h} \cdot \varepsilon(h)(k)
	\]
	Here, \( f''(a) \) is a bilinear map.
	Further,
	\[
		\norm{varepsilon(h)(k)} \leq \norm{\varepsilon(h)} \cdot \norm{k} \to 0
	\]
	Hence, \( \norm{h} \cdot \varepsilon(h)(k) = o\qty(\norm{h}) \).
	Conversely, suppose \( T \) is a bilinear map and
	\[
		\frac{f'(a+h)(k) - f'(a)(k) - T(h,k)}{\norm{h}} \to 0
	\]
	for any fixed \( k \), as \( h \to 0 \).
	We need to show that
	\[
		\varepsilon(h) = \frac{f'(a+h) - f'(a) - T(h)}{\norm{h}} \to 0
	\]
	in the space \( L(\mathbb R^m, \mathbb R^n) \).
	We know that for a fixed \( k \in \mathbb R^m \), \( \varepsilon(h)(k) \to 0 \) in \( \mathbb R^n \) as \( h \to 0 \).
	It then follows that
	\[
		\norm{\varepsilon(h)} = \sqrt{\sum_{i=1}^m \norm{\varepsilon(h)(e_i)}^2} \to 0
	\]
	since we are in a finite-dimensional vector space.
\end{proof}
\begin{example}
	Let \( f \colon \mathbb R^m \to \mathbb R^n \) be linear.
	Then \( f \) is differentiable on \( \mathbb R^m \) with \( f'(a) = f \) for all \( a \).
	Hence \( f' \colon \mathbb R^m \to L(\mathbb R^m, \mathbb R^n) \) sends \( a \) to \( f \) for all \( a \).
	So this is a constant function, so has derivative \( f''(a) = 0 \).
\end{example}
\begin{example}
	Let \( f \colon \mathbb R^m \times \mathbb R^n \to \mathbb R^p \) be bilinear.
	Then \( f \) is differentiable on \( \mathbb R^m \times \mathbb R^n \) and for all \( (a,b) \in \mathbb R^m \times \mathbb R^n \), we have
	\[
		f'(a,b)(h,k) = f(a,k) + f(h,b)
	\]
	Note that this is linear in \( (a,b) \) for a fixed \( (h,k) \).
	Hence, \( f' \colon \mathbb R^m \times \mathbb R^n \to L(\mathbb R^m, \mathbb R^n, \mathbb R^p) \) is linear.
	Hence this is differentiable, and its derivative is
	\[
		f''(a,b) = f' \in L(\mathbb R^m, \mathbb R^n, L(\mathbb R^m \times \mathbb R^n, \mathbb R^p)) \simeq \mathrm{Bil}((\mathbb R^m \times \mathbb R^n) \times (\mathbb R^m \times \mathbb R^n), \mathbb R^p)
	\]
\end{example}
\begin{example}
	Let \( f \colon M_n \to M_n \) be defined by \( f(A) = A^3 \).
	Let \( A \) be fixed.
	Then,
	\begin{align*}
		f(A+H) &= (A+H)^3 = A^3 + A^2 H + AHA + H A^2 + A H^2 + HAH + H^2 A + H^3 \\
		&= f(A) + (A^2 H + AHA + H A^2) + o\qty(\norm{H})
	\end{align*}
	Hence \( f \) is differentiable at \( A \) and
	\[
		f'(A)(H) = A^2 H + AHA + H A^2
	\]
	Thus, if \( n = 1 \), we have commutativity and hence \( f'(A) = 3A^2 \).
	So \( f \) is differentiable on \( M_n \).
	For a fixed \( A \) and fixed \( K \), the second derivative is given by
	\begin{align*}
		f'(A+H)(K) &= (A+H)^2 K + (A+H)K(A+H) + K (A+H)^2 \\
		&= \underbrace{(A^2 K + AKA + K A^2)}_{f'(A)(K)} \\
		&+ (AHK + HAK + AKH + HKA + KAH + KHA) + (H^2 K + HKH + KH^2)
	\end{align*}
	The term \( T(H,K) = (AHK + HAK + AKH + HKA + KAH + KHA) \) is bilinear in \( H \) and \( K \) as required.
	So the second derivative is \( T \).
	In one dimension, this is equivalent to saying \( f''(A) = 6A \).
\end{example}

\subsection{Second derivatives and partial derivatives}
Let \( U \) be open in \( \mathbb R^n \), let \( f \colon U \to \mathbb R^n \), and let \( a \in U \).
Let \( f \) be twice differentiable at \( a \), so \( f \) is differentiable on some open neighbourhood \( V \) of \( a \) contained within \( U \), and \( f' \colon V \to L(\mathbb R^m, \mathbb R^n) \) is differentiable at \( a \).
Recall that
\[
	f'(a+h) = f'(a) + f''(a)(h) + o\qty(\norm{h})
\]
Evaluating at a fixed \( k \),
\[
	f'(a+h)(k) = f'(a)(k) + f''(a)(h,k) + o\qty(\norm{h})
\]
Let \( u, v \in \mathbb R^m \setminus \qty{0} \) be directions.
Let \( k = v \).
Then,
\[
	f'(a+h)(v) = D_v f(a+h) = D_v f(a) + f''(a)(h,v) + o\qty(\norm{h})
\]
Hence, the map \( D_v f \colon V \to \mathbb R^n \) maps \( x \mapsto D_v f(x) = f'(x)(v) \).
Then this map is differentiable at \( a \) and
\[
	(D_v f)'(a)(h) = f''(a)(h,v)
\]
Hence there exist directional derivatives.
\[
	D_u D_v f(a) \overset{\mathrm{def}}{=} D_u (D_v f)(a) = (D_v f)'(a)(u) = f''(a)(u,v)
\]
In particular, we have
\[
	D_i D_j f(a) = f''(a)(e_i, e_j)
\]
for \( 1 \leq i, j \leq m \).

\subsection{Symmetry of mixed directional derivatives}
\begin{theorem}
	Let \( U \) be open in \( \mathbb R^n \), let \( f \colon U \to \mathbb R^n \), and let \( a \in U \).
	Let \( f \) be twice differentiable on an open set \( V \) with \( a \in V \subset U \).
	Let \( f'' \colon V \to \mathrm{Bil}(\mathbb R^m \times \mathbb R^m, \mathbb R^n) \) be continuous at \( a \).
	Then, for all directions \( u,v \in \mathbb R^m \setminus \qty{0} \), we have
	\[
		D_u D_v f(a) = D_v D_u f(a)
	\]
	Equivalently,
	\[
		f''(a)(u,v) = f''(a)(v,u)
	\]
	In other words, \( f'' \) is a symmetric bilinear map.
\end{theorem}
\begin{proof}
	Without loss of generality we can let \( n = 1 \).
	Indeed, we have
	\[
		(D_u f)_j(x) = [D_u f(x)]_j = [f'(x)(u)]_j = f_j'(x)(u) = D_u f_j(x)
	\]
	Hence, \( (D_u f)_j = D_u f_j \).
	For \( v \):
	\[
		(D_v D_u f)_j = D_v (D_u f)_j = D_v D_u f_j
	\]
	So it is sufficient to show that \( D_v D_u f_j(a) = D_u D_v f_j(a) \).
	Now, consider
	\[
		\phi(s,t) = f(a+su+tv) - f(a+tv) - f(a+su) + f(a)
	\]
	for \( s, t \in \mathbb R \).
	Let \( s, t \) be fixed, and consider
	\[
		\psi(y) = f(a+yu+tv) - f(a+yu)
	\]
	Note that \( \phi(s,t) \) can be written as
	\[
		\phi(s,t) = \psi(s) - \psi(0)
	\]
	The term \( \psi(s) - \psi(0) \) can be interpreted as \( (f(a+su+tv) - f(a+tv)) - (f(a+su) - f(a)) \), which is the second difference given by the function when traversing the parallelogram with sides \( su, tv \).
	By the mean value theorem, there exists \( \alpha(s,t) \in (0,1) \) such that
	\[
		\phi(s,t) = \psi(s) - \psi(0) = s \psi'(\alpha s) = s \qty[ D_u f(a+\alpha s u+ tv) - D_u f(a + \alpha s u) ]
	\]
	Now, applying the mean value theorem to the function \( y \mapsto D_u f(a+\alpha s u + y v) \), we have
	\[
		\phi(s,t) = s t D_v D_u f (a+\alpha s u + \beta t v)
	\]
	for \( \beta(s,t) \in (0,1) \).
	Now,
	\[
		\frac{\phi(s,t)}{st} = D_v D_u f(a+\alpha su + \beta tv) = f''(a+\alpha su + \beta tv)(u,v)
	\]
	Since \( f'' \) is continuous at \( a \), we can let \( s, t \to 0 \) and find
	\[
		\frac{\phi(s,t)}{st} \to f''(a)(u,v)
	\]
	Now, we can repeat the above using
	\[
		\psi(y) = f(a+su+yv) - f(a+yv)
	\]
	This calculates the second difference from above, but using the other path.
	We can find
	\[
		\frac{\phi(s,t)}{st} \to f''(a)(v,u)
	\]
	as required.
\end{proof}
