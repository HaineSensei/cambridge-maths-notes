\subsection{Second derivatives and partial derivatives}
Let \( U \) be open in \( \mathbb R^n \), let \( f \colon U \to \mathbb R^n \), and let \( a \in U \).
Let \( f \) be twice differentiable at \( a \), so \( f \) is differentiable on some open neighbourhood \( V \) of \( a \) contained within \( U \), and \( f' \colon V \to L(\mathbb R^m, \mathbb R^n) \) is differentiable at \( a \).
Recall that
\[
	f'(a+h) = f'(a) + f''(a)(h) + o\qty(\norm{h})
\]
Evaluating at a fixed \( k \),
\[
	f'(a+h)(k) = f'(a)(k) + f''(a)(h,k) + o\qty(\norm{h})
\]
Let \( u, v \in \mathbb R^m \setminus \qty{0} \) be directions.
Let \( k = v \).
Then,
\[
	f'(a+h)(v) = D_v f(a+h) = D_v f(a) + f''(a)(h,v) + o\qty(\norm{h})
\]
Hence, the map \( D_v f \colon V \to \mathbb R^n \) maps \( x \mapsto D_v f(x) = f'(x)(v) \).
Then this map is differentiable at \( a \) and
\[
	(D_v f)'(a)(h) = f''(a)(h,v)
\]
Hence there exist directional derivatives.
\[
	D_u D_v f(a) \overset{\mathrm{def}}{=} D_u (D_v f)(a) = (D_v f)'(a)(u) = f''(a)(u,v)
\]
In particular, we have
\[
	D_i D_j f(a) = f''(a)(e_i, e_j)
\]
for \( 1 \leq i, j \leq m \).

\subsection{Symmetry of mixed directional derivatives}
\begin{theorem}
	Let \( U \) be open in \( \mathbb R^n \), let \( f \colon U \to \mathbb R^n \), and let \( a \in U \).
	Let \( f \) be twice differentiable on an open set \( V \) with \( a \in V \subset U \).
	Let \( f'' \colon V \to \mathrm{Bil}(\mathbb R^m \times \mathbb R^m, \mathbb R^n) \) be continuous at \( a \).
	Then, for all directions \( u,v \in \mathbb R^m \setminus \qty{0} \), we have
	\[
		D_u D_v f(a) = D_v D_u f(a)
	\]
	Equivalently,
	\[
		f''(a)(u,v) = f''(a)(v,u)
	\]
	In other words, \( f'' \) is a symmetric bilinear map.
\end{theorem}
\begin{proof}
	Without loss of generality we can let \( n = 1 \).
	Indeed, we have
	\[
		(D_u f)_j(x) = [D_u f(x)]_j = [f'(x)(u)]_j = f_j'(x)(u) = D_u f_j(x)
	\]
	Hence, \( (D_u f)_j = D_u f_j \).
	For \( v \):
	\[
		(D_v D_u f)_j = D_v (D_u f)_j = D_v D_u f_j
	\]
	So it is sufficient to show that \( D_v D_u f_j(a) = D_u D_v f_j(a) \).
	Now, consider
	\[
		\phi(s,t) = f(a+su+tv) - f(a+tv) - f(a+su) + f(a)
	\]
	for \( s, t \in \mathbb R \).
	Let \( s, t \) be fixed, and consider
	\[
		\psi(y) = f(a+yu+tv) - f(a+yu)
	\]
	Note that \( \phi(s,t) \) can be written as
	\[
		\phi(s,t) = \psi(s) - \psi(0)
	\]
	The term \( \psi(s) - \psi(0) \) can be interpreted as \( (f(a+su+tv) - f(a+tv)) - (f(a+su) - f(a)) \), which is the second difference given by the function when traversing the parallelogram with sides \( su, tv \).
	By the mean value theorem, there exists \( \alpha(s,t) \in (0,1) \) such that
	\[
		\phi(s,t) = \psi(s) - \psi(0) = s \psi'(\alpha s) = s \qty[ D_u f(a+\alpha s u+ tv) - D_u f(a + \alpha s u) ]
	\]
	Now, applying the mean value theorem to the function \( y \mapsto D_u f(a+\alpha s u + y v) \), we have
	\[
		\phi(s,t) = s t D_v D_u f (a+\alpha s u + \beta t v)
	\]
	for \( \beta(s,t) \in (0,1) \).
	Now,
	\[
		\frac{\phi(s,t)}{st} = D_v D_u f(a+\alpha su + \beta tv) = f''(a+\alpha su + \beta tv)(u,v)
	\]
	Since \( f'' \) is continuous at \( a \), we can let \( s, t \to 0 \) and find
	\[
		\frac{\phi(s,t)}{st} \to f''(a)(u,v)
	\]
	Now, we can repeat the above using
	\[
		\psi(y) = f(a+su) + yv) - f(a+yv)
	\]
	This calculates the second difference from above, but using the other path.
	We can find
	\[
		\frac{\phi(s,t)}{st} \to f''(a)(v,u)
	\]
	as required.
\end{proof}
