\subsection{Continuity in product topology}
\begin{example}
	Let \( (M, d), (M', d') \) be metric spaces.
	Then, the metric \( d_\infty \) on \( M \times M' \) is
	\[
		d_\infty((x,x'), (y,y')) = \max(d(x,y), d'(x',y'))
	\]
	This metric is chosen since all \( d_p \) metrics induce the same metric topology, but this is easier to work with.
	Also, \( M, M' \) are topological spaces with their metric topologies, which induce the product topology on the product space \( M \times M' \).
	These two constructions create the same topology.
	For a point \( z = (x,x') \in M \times M' \) and \( r > 0 \), the open ball \( \mathcal D_r(z) \) is exactly\
	\begin{align*}
		\mathcal D_r(z) & = \qty{(y,y') \in M \times M' \colon d_\infty((y,y'), (x,x')) < r} \\
		                & = \qty{(y,y') \in M \times M' \colon d(x,y) < r, d(x',y') < r}     \\
		                & = \mathcal D_r(x) \times \mathcal D_r(x')
	\end{align*}
	Now, let \( W \subset M \times M' \).
	Then \( W \) is open in the product topology if and only if for all \( z = (x,x') \in W \), there exist open sets \( U \) in \( M \) and \( U' \) in \( M' \) such that \( (x,x') \in U \times U' \subset W \).
	Equivalently, for all \( z = (x,x') \in W \), there exists \( r > 0 \) such that \( \mathcal D_r(x) \times \mathcal D_r(x') \subset W \).
	But \( \mathcal D_r(x) \times \mathcal D_r(x') = \mathcal D_r(z) \), so \( W \) is \( d_\infty \)-open, as required.
	For instance, the product topology on \( \mathbb R \times \mathbb R \) is the Euclidean topology on \( \mathbb R^2 \).
\end{example}
\begin{proposition}
	Let \( X, Y \) be topological spaces.
	Let \( X \times Y \) be given the product topology.
	Then, the coordinate projections \( q_X \colon X \times Y \to X \) and \( q_Y \colon X \to Y \to Y \) satisfy
	\begin{enumerate}[(i)]
		\item \( q_X, q_Y \) are continuous;
		\item if \( Z \) is any topological space, and \( g \colon Z \to X \times Y \) is a function, then \( g \) is continuous if and only if \( q_X \circ g, q_Y \circ g \) are continuous.
	\end{enumerate}
\end{proposition}
\begin{proof}
	If \( U \) is open in \( X \), then \( q_X^{-1}(U) = U \times Y \), which is the product of an open set in \( X \) and an open set in \( Y \), so is open in \( X \times Y \).
	Hence \( q_X \) is continuous.
	Similarly, \( q_Y \) is continuous.

	If \( g \) is continuous then certainly \( q_X \circ g, q_Y \circ g \) are continuous since the composition of continuous functions are continuous.
	Conversely, let \( h \colon Z \to X \) and \( k \colon Z \to Y \) be continuous functions with \( h = q_X \circ g \) and \( k = q_Y \circ g \).
	Then \( g(x) = (h(x), k(x)) \) for \( x \in Z \).
	Now, for open sets \( U \) in \( X \) and \( V \) in \( Y \), we have
	\[
		z \in g^{-1}(U \times V) \iff g(z) \in U \times V \iff h(z) \in U, k(z) \in V \iff z \in h^{-1}(U) \cap k^{-1}(V)
	\]
	So \( g^{-1}(U \times V) = h^{-1}(U) \cap k^{-1}(V) \) which is open in \( Z \) as \( h, k \) are continuous.
	Given an arbitrary open set \( W \) in \( X \times Y \), we can write \( W = \bigcup_{i\in I} U_i \times V_i \), where \( U_i \) are open in \( X \) and \( V_i \) are open in \( Y \).
	Thus, \( g^{-1}(W) = \bigcup_{i \in I} g^{-1}(U_i \times V_i) \) which is open.
\end{proof}
\begin{remark}
	The product topology may be extended to a finite product \( X_1 \times \dots \times X_n \), consisting of all unions of sets of the form \( U_1 \times \dots \times U_n \) where \( U_j \) is open in \( X_j \).
	Properties of the product topology hold in this more general case.
	For example, if \( X_j \) is metrisable with metric \( e_j \) for all \( j \), then the product topology is metrisable with, for instance, the \( d_\infty \) metric.
\end{remark}

\subsection{Quotients}
Let \( X \) be a set and \( R \) an equivalence relation on \( X \).
So \( R \subset X \times X \), but we will write \( x \sim y \) to mean \( (x,y) \in R \).
For \( x \in X \), we define \( q(x) = \qty{y \in X \colon y \sim x} \) to be the equivalence class of \( x \), the set of which partition \( X \).
Let \( X / R \) denote the set of all equivalence classes.
The map \( q \colon X \to X/R \) is called the quotient map.
\begin{definition}
	Let \( X \) be a topological space, and \( R \) an equivalence relation on \( X \).
	The \textit{quotient topology} on \( X/R \) is given by
	\[
		\tau = \qty{V \subset X/R \colon q^{-1}(V) \text{ open in } X }
	\]
	This is a topology:
	\begin{enumerate}[(i)]
		\item \( q^{-1}(\varnothing) = \varnothing \) which is open, and \( q^{-1}(X/R) = X \) which is open.
		\item If \( V_i \) are open, then \( q^{-1}\qty(\bigcup_{i \in I} V_i) = \bigcup_{i \in I} q^{-1}(V_i) \) which is a union of open sets which is open.
		\item If \( U, V \) are open, then \( q^{-1}(U \cap V) = q^{-1}(U) \cap q^{-1}(V) \) which is open.
	\end{enumerate}
\end{definition}
\begin{remark}
	The quotient map \( q \colon X \to X/R \) is continuous.
	In particular, it is the largest possible topology on \( X \) such that \( q \) is continuous.

	Let \( x \in X, t \in X/R \).
	Then \( x \in t \) if and only if \( t = q(x) \).
	For \( V \subset X/R \),
	\[
		q^{-1}(V) = \qty{x \in X \colon q(x) \in V} = \qty{x \in X \colon \exists t \in V, t = q(x)} = \qty{x \in X \colon \exists t \in V, x \in t} = \bigcup_{t \in V} t
	\]
\end{remark}
\begin{example}
	Consider \( \mathbb R \), an abelian group under addition, and the subgroup \( \mathbb Z \).
	We can form the quotient group \( \mathbb R / \mathbb Z \), which is the set of equivalence classes where \( x \sim y \iff x - y \in \mathbb Z \).
	For all \( x \in \mathbb R \), there exists \( y \in [0,1] \) such that \( x \sim y \), and for all \( x, y \in [0,1] \) we have \( x \sim y \) if and only if \( x = y \) or \( \qty{x,y} = \qty{0,1} \).
	So we can think of the quotient topology of \( \mathbb R / \mathbb Z \) as a circle.
	We can say that \( \mathbb R / \mathbb Z \) is homeomorphic to \( S_1 = \qty{(x,y) \in \mathbb R^2 \colon \norm{(x,y)} = 1} \), which we will prove later.
\end{example}
\begin{example}
	Consider the subgroup \( \mathbb Q \) of \( \mathbb R \).
	Let \( V \subset \mathbb R / \mathbb Q \), such that \( V \neq \varnothing \) and \( V \) is open.
	Then \( q^{-1}(V) \) is open and not empty.
	Therefore, there exist \( a < b \in \mathbb R \) such that \( (a,b) \subset q^{-1}(V) \).
	Given \( x \in \mathbb R \), we can choose a rational \( r \) in the interval \( (a-x, b-x) \).
	Then \( r + x \in (a,b) \subset q^{-1}(V) \), so \( q(x) = q(r+x) \in V \).
	So \( V = \mathbb R / \mathbb Q \).
	This is the indiscrete topology, which is not metrisable or Hausdorff.
	So we cannot (in general) take quotients of metric spaces.
\end{example}
\begin{example}
	Let \( Q = [0,1] \times [0,1] \subset \mathbb R^2 \).
	We define the equivalence relation \( R \) given by
	\[
		(x_1, x_2) \sim (y_1, y_2) \iff
		\begin{cases}
			(x_1, x_2) = (y_1, y_2)               & \text{or} \\
			x_1 = y_1, \qty{x_2, y_2} = \qty{0,1} & \text{or} \\
			x_2 = y_2, \qty{x_1, y_1} = \qty{0,1} & \text{or} \\
			x_1, x_2, y_1, y_2 \in \qty{0,1}
		\end{cases}
	\]
	The space \( Q / R \) is homeomorphic to \( \mathbb R^2 / \mathbb Z^2 \).
	This is a square where the top and bottom edges are identified as the same, and the left and right edges are also identified as the same.
	This is homeomorphic to the surface of a torus with the Euclidean topology embedded in Euclidean three-dimensional space.
\end{example}
\begin{proposition}
	Let \( X \) be a set, and let \( R \) be an equivalence relation on \( X \).
	Let \( q \colon X \to X/R \) be the quotient map.
	Let \( Y \) be a set, and \( f \colon X \to Y \) be a function.
	Suppose that \( f \) `respects' \( R \); that is, \( x \sim y \implies f(x) = f(y) \).
	Then there exists a unique map \( \widetilde f \colon X/R \to Y \) such that \( f = \widetilde f \circ q \).
	For \( z \in X/R \), we write \( z = q(x) \) for some \( x \in X \), and then define \( \widetilde f(z) = f(x) \).
\end{proposition}
\begin{remark}
	Note that \( \Im f = \Im \widetilde f \) since \( q \) is surjective.
	\( \widetilde f \) is injective if for all \( x, y \in X \), \( \widetilde f(q(x)) = \widetilde f(q(y)) \) implies \( q(x) = q(y) \).
	In other words, for all \( x,y \in X \), \( f(x) = f(y) \implies x \sim y \).
	We say that \( f \) \textit{fully respects} \( R \) if, for all \( x,y \in X \),
	\[
		x \sim y \iff f(x) = f(y)
	\]
	In this case, \( \widetilde f \) is injective.
\end{remark}
