\subsection{Definitions}
\begin{definition}
	Let \( X \) be a set.
	A \textit{topology} on \( X \) is a family \( \tau \) of subsets of \( X \) (so \( \tau \subset \mathcal P(X) \)) such that
	\begin{enumerate}
		\item \( \varnothing, X \in \tau \);
		\item if \( U_i \in \tau \) for all \( i \in I \) where \( I \) is some index set, then \( \bigcup_{i \in I} U_i \in \tau \); and
		\item if \( U, V \in \tau \) then \( U \cap V \in \tau \).
	\end{enumerate}
	A \textit{topological space} is a pair \( (X, \tau) \) where \( X \) is a set and \( \tau \) is a topology on \( X \).
	Members of \( \tau \) are called \textit{open sets} in the topology.
	So we say that \( U \subset X \) is \textit{open in} \( X \), or \( U \) is \( \tau \)\textit{-open}, if \( U \in \tau \).
\end{definition}
\begin{remark}
	If \( U_i \in \tau \) for \( i = 1, \dots, n \), then \( \bigcap_{i=1}^n U_i \in \tau \).
\end{remark}
\begin{example}
	Let \( (M, d) \) be a metric space.
	Recall that \( U \subset M \) is open in the metric sense if \( \forall x \in U, \exists r > 0, \mathcal B_r(x) \subset U \).
	We may say that \( U \) is \( d \)\textit{-open}.
	We have already proven that the family of \( d \)-open sets is a topology on \( M \).
	This is a metric topology.
\end{example}
\begin{definition}
	Let \( (X, \tau) \) be a topological space.
	Then we say that \( X \) is \textit{metrisable} (or sometimes we say \( \tau \) is metrisable) if there exists a metric \( d \) on \( X \) such that \( \tau \) is the metric topology on \( X \) induced by \( d \).
	In other words, \( U \subset X \) is \( \tau \)-open if and only if \( U \) is \( d \)-open.
	If \( d' \sim d \), then \( d' \) also induces the same topology \( \tau \) on \( X \).
\end{definition}
\begin{example}
	The indiscrete topology on a set \( X \) is a topology \( \tau = \qty{ \varnothing, X } \).
	If \( \abs{X} \geq 2 \), then this is not metrisable.
	Let \( d \) be a metric on \( X \).
	Then let \( x \neq y \in X \), let \( r = d(x,y) \), and finally let \( U = \mathcal D_r(x) \).
	We know that \( U \) is \( d \)-open.
	But since \( x \in U, y \not\in U \), \( U \not\in \tau \).
\end{example}
\begin{definition}
	If \( \tau_1, \tau_2 \) are topologies on \( X \), we say that \( \tau_1 \) is \textit{coarser} than \( \tau_2 \), or that \( \tau_2 \) is finer than \( \tau_1 \), if \( \tau_1 \subset \tau_2 \).
	For example, the indiscrete topology on \( X \) is the coarsest topology on \( X \).
\end{definition}
\begin{example}
	The discrete topology on a set \( X \) is \( \tau = \mathcal P(X) \).
	This is the finest topology on \( X \).
	This is metrisable by the discrete metric.
\end{example}
\begin{definition}
	A topological space \( X \) is \textit{Hausdorff} if \( \forall x \neq y \) in \( X \), there exist open sets \( U, V \) in \( X \) such that \( x \in U, y \in V, U \cap V = \varnothing \).
	Informally, \( x, y \) are `separated by open sets'.
\end{definition}
\begin{proposition}
	Metric spaces are Hausdorff.
\end{proposition}
\begin{proof}
	Let \( x \neq y \) be points in a metric space \( (M, d) \).
	Let \( r > 0 \) such that \( 2r < d(x,y) \).
	Then let \( U = \mathcal D_r(x) \), let \( V = \mathcal D_r(y) \).
	Certainly \( U, V \) are open since they are open balls, and they have no intersection by the triangle inequality, so the metric space is Hausdorff as required.
\end{proof}
\begin{example}
	The cofinite topology on a set \( X \) is
	\[
		\tau = \qty{\varnothing} \cup \qty{U \in X \colon U \text{ is cofinite in } X}
	\]
	where \( U \) is cofinite in \( X \) if \( X \setminus U \) is finite.
	When \( X \) is finite, this topology \( \tau \) is simply \( \mathcal P(X) \).
	When \( X \) is infinite, \( \tau \) is not metrisable.
	Let \( x \neq y \) in \( X \), and let \( x \in U, y \in V \) where \( U, V \) are open in \( X \).
	Then \( U \) and \( V \) are cofinite, and hence \( U \cap V \neq \varnothing \).
	So this topology on an infinite set is not Hausdorff and hence not metrisable.
\end{example}

\subsection{Closed subsets}
\begin{definition}
	A subset \( A \) of a topological space \( (X, \tau) \) is said to be \textit{closed} in \( X \) if \( X \setminus A \) is open in \( X \).
\end{definition}
\begin{remark}
	In a metric space, this agrees with the earlier definition of a closed subset, as proven before.
\end{remark}
\begin{proposition}
	The collection of closed sets in a topological space \( X \) satisfy
	\begin{enumerate}
		\item \( \varnothing, X \) are closed;
		\item If \( A_i \) are closed in \( X \) for \( i \) in some non-empty index set \( I \), then \( \bigcap_{i \in I} A_i \) is closed;
		\item If \( A_1, A_2 \) are closed in \( X \) then \( A_1 \cup A_2 \) is closed.
	\end{enumerate}
\end{proposition}
\begin{example}
	In a discrete topological space, every set is closed.
\end{example}
\begin{example}
	In the cofinite topology, a subset is closed if and only if it is finite or the full set.
\end{example}

\subsection{Neighbourhoods}
\begin{definition}
	Let \( X \) be a topological space, and let \( U \subset X \) and \( x \in X \).
	We say that \( U \) is a \textit{neighbourhood} of \( x \) in \( X \) if there exists an open set \( V \) in \( X \) such that \( X \in V \subset U \).
\end{definition}
\begin{remark}
	In a metric space, we defined this in terms of open balls not open sets.
	However, we have already proven that the definitions agree.
\end{remark}
\begin{proposition}
	Let \( U \) be a subset of a topological space \( X \).
	Then \( U \) is open if and only if \( U \) is a neighbourhood of \( x \) for every \( x \in U \).
\end{proposition}
\begin{proof}
	If \( U \) is open, and \( x \in U \), then by letting \( V = U \), \( V \) is open and \( x \in V \subset U \).
	Conversely, if \( x \in U \), there exists \( V_x \) in \( X \) such that \( x \in V_x \subset U \).
	Then, \( U = \bigcup_{x \in U} x = \bigcup_{x \in U} V_x \) is open, since each \( V_x \) is open.
\end{proof}

\subsection{Convergence}
\begin{definition}
	Let \( (x_n) \) be a sequence in a topological space \( X \).
	Let \( x \in X \).
	We say that \( (x_n) \) \text{converges to} \( x \) if for all neighbourhoods \( U \) of \( x \) in \( X \), there exists \( N \in \mathbb N \) such that \( \forall n \geq N, x_n \in U \).
	Equivalently, for all open sets \( U \) which contain \( x \), there exists \( N \in \mathbb N \) such that \( \forall n \geq N, x_n \in U \).
\end{definition}
\begin{remark}
	Again, the definition in a metric space agrees with this definition.
\end{remark}
\begin{example}
	Eventually constant sequences converge.
	If \( \exists z \in X, \exists N \in \mathbb N, \forall n \geq N, x_n = z \), then \( x_n \to z \).
\end{example}
\begin{example}
	In an indiscrete topological space, every sequence converges to every point.
\end{example}
\begin{example}
	In the cofinite topology on a set \( X \), let \( x_n \to X \).
	Suppose that \( x_n \to x \) in \( X \).
	Then if \( y \neq x \), \( X \setminus \qty{y} \) is a neighbourhood of \( x \).
	Then \( N_y = \qty{n \in N \colon x_n = y} \) is finite.

	Conversely, suppose \( (x_n) \) is a sequence such that for some \( x \in X \) and for all \( y \neq x \), \( N_y \) is finite.
	Then \( x_n \to x \).

	In particular, if \( N_y \) is finite for all \( y \in X \), the sequence converges to every point.
\end{example}
\begin{proposition}
	If \( x_n \to x \) and \( x_n \to y \) in a Hausdorff space, then \( x = y \).
\end{proposition}
\begin{proof}
	Suppose \( x \neq y \), then we can choose open sets \( U, V \) such that \( x \in U, y \in V, U \cap V = \varnothing \).
	Since \( x_n \to x \), there exists \( N_1 \in \mathbb N \) such that \( \forall n \geq N_1, x_n \in U \).
	Similarly there exists an analogous \( N_2 \).
	Hence \( \forall n \geq \max(N_1, N_2), x_n \in U, x_n \in V \) which is a contradiction since \( U \cap V = \varnothing \).
\end{proof}
\begin{remark}
	If \( x_n \to x \) in a Hausdorff space, we write \( x = \lim_{n \to \infty} x_n \) since the limit is unique.
\end{remark}
\begin{remark}
	In a metric space, for a subset \( A \), we say that \( A \) is closed if and only if \( x_n \to x \) in \( A \) implies \( x \in A \).
	In a general topological space, any closed set is closed under limits, but not every subset that is closed under limits is closed.
\end{remark}

\subsection{Interiors and closures}
\begin{definition}
	Let \( X \) be a topological space, and \( A \subset X \).
	We define the \textit{interior} of \( A \) in \( X \), denoted \( A^\circ \) or \( \mathrm{int}(A) \), by
	\[
		A^\circ = \bigcup \qty{ U \subset X \colon U \text{ is open in } X, U \subset A }
	\]
	Similarly we define the \textit{closure} of \( A \) in \( X \), denoted \( \overline A \) or \( \mathrm{cl}(A) \), by
	\[
		\overline A = \bigcap \qty{ F \subset X \colon F \text{ is closed in } X, F \supset A }
	\]
\end{definition}
\begin{remark}
	Note that \( A^\circ \) is open in \( X \), and \( A^\circ \subset A \).
	In particular, if \( U \) is open in \( X \) and \( U \subset A \), then \( U \subset A^\circ \).
	Hence, \( A^\circ \) is the largest open subset of \( A \).

	Similarly, \( \overline A \) is closed in \( X \), and \( \overline A \supset A \).
	The intersection is not empty since \( X \) is closed and \( X \supset A \), so it is well-defined.
	We have that \( \overline A \) is the smallest closed superset of \( A \).
\end{remark}
\begin{proposition}
	Let \( X \) be a topological space and let \( A \subset X \).
	Then the interior is exactly those \( x \in X \) for which \( A \) is a neighbourhood of \( x \).
	Similarly, the closure is those \( x \in X \) such that for all neighbourhoods \( U \) of \( x \), \( U \cap A \neq \varnothing \).
\end{proposition}
\begin{proof}
	If \( A \) is a neighbourhood of \( X \), then by definition there exists an open set \( U \) such that \( x \in U \subset A \), which is true if and only if \( x \in A^\circ \).

	For the other part, suppose \( x \not\in \overline A \).
	Then there exists a closed set \( F \supset A \) such that \( x \not\in F \).
	Let \( U = X \setminus F \).
	Then \( U \) is open and \( x \in U \).
	So \( U \) is a neighbourhood of \( x \), and \( U \cap A = \varnothing \).

	Conversely, suppose there exists a neighbourhood \( U \) of \( x \) such that \( U \cap A = \varnothing \).
	Then there exists an open set \( V \) such that \( x \in V \subset U \).
	Since \( V \subset U \), \( V \cap A = \varnothing \).
	Let \( F = X \setminus V \).
	Then \( F \) is closed, and \( A \subset F \).
	Hence \( \overline A \subset F \).
	So \( x \not\in \overline A \).
\end{proof}
\begin{example}
	In \( \mathbb R \), let \( A = [0,1) \cup \qty{2} \).
	Then \( A^\circ = (0,1) \), and \( \overline A = [0,1] \cup \qty{2} \).
	Further, \( \mathbb Q^\circ = \varnothing \) and \( \overline {\mathbb Q} = \mathbb R \).
	Finally, \( \mathbb Z^\circ = \varnothing \) and \( \overline {\mathbb Z} = \mathbb Z \).
\end{example}
\begin{remark}
	In a metric space, for a subset \( A \) we have that \( x \in \overline A \) if and only if there exists a sequence \( (x_n) \) in \( A \) such that \( x_n \to x \).
	In a general topological space, the existence of a sequence implies \( x \in \overline A \) but the converse is not true.
\end{remark}

\subsection{Dense subsets}
\begin{definition}
	A subset \( A \) of a topological space \( X \) is said to be \textit{dense} in \( X \) if \( \overline A = X \).
	\( X \) is \textit{separable} if there exists a countable subset \( A \subset X \) such that \( A \) is dense in \( X \).
\end{definition}
\begin{example}
	\( \mathbb R \) is separable as \( \mathbb Q \) is dense in \( \mathbb R \).
	\( \mathbb R^n \) is separable in the same way as \( \mathbb Q^n \) is dense in \( \mathbb R^n \).
\end{example}
\begin{example}
	An uncountable discrete topological space is not separable, since the closure of any set is itself.
\end{example}

\subsection{Subspaces}
\begin{definition}
	Let \( (X, \tau) \) be a topological space.
	Let \( Y \subset X \).
	Then the \textit{subspace topology}, or \textit{relative topology} on \( Y \) induced by \( \tau \) is the topology
	\[
		\qty{ V \cap Y \colon V \in \tau }
	\]
	on \( Y \).
	This is the intersection of \( Y \) with all open sets in \( X \).
	We can denote this \( \eval{\tau}_Y \).
	So, for \( U \subset Y \), \( U \) is open in \( Y \) if and only if there exists an open set \( V \) in \( X \) with \( U = V \cap Y \).
\end{definition}
\begin{example}
	Let \( X = \mathbb R, Y = [0,2] \), and \( U = (1,2] \).
	Then certainly \( U \subset Y \subset X \).
	\( U \) is open in \( Y \), since \( V = (1,3) \) is open in \( X \) and \( U = V \cap Y \).
	However, \( U \) is not open in \( X \), since no neighbourhood (or ball) around \( 2 \) can be constructed in \( X \) that is contained within \( U \).
\end{example}
\begin{remark}
	On a subset of a topological space, this is considered the standard topology.
	Suppose that \( (X, \tau) \) is a topological space, and \( Z \subset Y \subset X \).
	There are two natural topologies on \( Z \): \( \eval{\tau}_Z \) and \( \eval{\eval{\tau}_Y}_Z \).
	One can easily check that these two topologies are equal.

	Let \( (M,d) \) be a metric space, and \( N \subset M \).
	Again, there are two natural topologies on \( N \): \( \eval{\tau(d)}_N \) and \( \tau\qty(\eval{d}_N) \), where \( \tau(e) \) is the metric topology induced by the metric \( e \).
	This is because, for any \( x \in N, r > 0 \),
	\[
		\qty{y \in N \colon d(y,x) < r} = \qty{y \in M \colon d(y,x) < r} \cap N
	\]
\end{remark}
\begin{proposition}
	Let \( X \) be a topological space, and let \( A \subset Y \subset X \).
	\( A \) is closed in \( Y \) if and only if there exists a closed subset \( B \subset X \) such that \( A = B \cap Y \).
	Further,
	\[
		\mathrm{cl}_Y(A) = \mathrm{cl}_X(A) \cap Y
	\]
	This is not true for the interior of a subset in general.
	For instance, consider \( X = \mathbb R, A = Y = \qty{0} \).
	In this case, \( \mathrm{int}_Y(A) = A, \mathrm{int}_X(A) = \varnothing \).
\end{proposition}
\begin{proof}
	The first part is true by taking complements: \( Y \setminus A \) is open in \( Y \).
	By definition, \( Y \setminus A = V \cap Y \) for some open \( V \) in \( X \).
	So \( B = X \setminus V \) is closed in \( X \) and \( A = B \cap Y \).
	If \( A = B \cap Y \), \( B \) is closed in \( X \), then \( X \setminus B \) is open in \( X \), and hence \( Y \setminus A = (X \setminus B) \cap Y \) is open in \( Y \).

	For the second part, we know \( \mathrm{cl}_X(A) \) is closed in \( X \), so by the first part, \( \mathrm{cl}_X(A) \cap Y \) is closed in \( Y \).
	Then \( A \subset \mathrm{cl}_X(A) \cap Y \).
	So by definition, \( \mathrm{cl}_Y(A) \subset \mathrm{cl}_X(A) \cap Y \).
	Similarly, since \( \mathrm{cl}_Y(A) \) is closed in \( Y \), we can write \( \mathrm{cl}_Y(A) = B \cap Y \) for some closed set \( B \) in \( X \).
	But \( A \subset B \), and \( B \) is closed in \( X \), so \( \mathrm{cl}_X(A) \subset B \) and hence \( \mathrm{cl}_Y(A) = B \cap Y \supset \mathrm{cl}_X(A) \cap Y \).
\end{proof}
\begin{remark}
	If \( U \subset Y \subset X \), and \( Y \) is open in \( X \), then \( U \) is open in \( Y \) if and only if \( U \) is open in \( X \).
\end{remark}

\subsection{Continuity}
\begin{definition}
	A function \( f \colon X \to Y \) between topological spaces is said to be continuous if for all open sets \( V \) in \( Y \), the preimage \( f^{-1}(V) \) is open in \( X \).
\end{definition}
\begin{remark}
	We have already proven that this agrees with the definition of continuity of functions between metric spaces.
\end{remark}
\begin{example}
	Constant functions are always continuous.
	Consider \( f \colon X \to Y \) defined by \( f(x) = y_0 \) for a fixed \( y_0 \in Y \).
	For any \( V \subset Y \), \( f^{-1}(V) = \varnothing \) if \( y_0 \not\in V \), and \( f^{-1}(V) = X \) if \( y_0 \in V \).
	So \( f \) is continuous.
\end{example}
\begin{example}
	The identity map is always continuous.
	If \( f \colon X \to X \) is defined by \( x \mapsto x \), \( f^{-1}(V) = V \) so if \( V \) is open, \( f^{-1}(V) \) is trivially open.
\end{example}
\begin{example}
	Let \( Y \subset X \).
	Let \( i \colon Y \to X \) be the inclusion map.
	Then for an open set \( V \) in \( X \), \( i^{-1}(V) = V \cap Y \) which by definition is open in \( Y \).
	Hence, if \( g \colon X \to Z \) is continuous, then \( \eval{g}_Y = g \circ i \colon X \to Y \) is continuous, as we will see below.
\end{example}
\begin{proposition}
	Let \( f \colon X \to Y \) be a function between topological spaces.
	Then,
	\begin{enumerate}
		\item \( f \) is continuous if and only if for all closed sets \( B \) in \( Y \), \( f^{-1}(B) \) is closed in \( X \);
		\item if \( f \) is continuous and \( g \colon Y \to Z \) is continuous, then \( g \circ f \) is continuous.
	\end{enumerate}
\end{proposition}
\begin{proof}
	To prove (i), note that for any subset \( D \subset Y \), \( f^{-1}(Y \setminus D) = X \setminus f^{-1}(D) \).
	We can now use the fact that \( A \subset X \) is open in \( X \) if and only if \( X \setminus A \) is closed in \( X \), and vice versa for \( Y \).

	To prove (ii), note that if \( W \) is an open subset of \( Z \), then \( g^{-1}(W) \) is open in \( Y \) since \( g \) is continuous.
	Hence \( f^{-1}g^{-1}(W) \) is open in \( X \) since \( f \) is continuous.
	But then \( f^{-1}g^{-1} = (g \circ f)^{-1} \), so \( g \circ f \) is continuous.
\end{proof}
\begin{remark}
	There exists a notion of `continuity at a point' for topological spaces, but it is not as useful in this course as the global continuity definition.
\end{remark}

\subsection{Homeomorphisms and topological invariance}
\begin{definition}
	A function \( f \colon X \to Y \) between topological spaces is a homeomorphism if \( f \) is a bijection, and both \( f, f^{-1} \) are continuous.
	If such an \( f \) exists, we say that \( X \) and \( Y \) are homeomorphic.
	This is exactly the definition from metric spaces.
\end{definition}
\begin{definition}
	A property \( \mathcal P \) of topological spaces is said to be a \textit{topological property} or \textit{topological invariant} if, for all pairs \( X, Y \) of homeomorphic spaces, \( X \) satisfies \( \mathcal P \) if and only if \( Y \) satisfies \( \mathcal P \).
\end{definition}
\begin{example}
	Metrisability is a topological invariant.
	Being Hausdorff is a topological invariant.
	Being completely metrisable (metrisable into a complete metric space) is \textit{not} a topological invariant.
	For example, consider metrics \( d, d' \) on \( \mathbb R \) such that \( d \sim d' \) but \( d \) is complete and \( d' \) is not.
\end{example}
\begin{remark}
	If \( f \colon X \to Y \) is a homeomorphism, for an open set \( U \) in \( X \), \( f(U) = (f^{-1})^{-1}(U) \) is open in \( Y \) since \( f^{-1} \colon Y \to X \) is continuous.
\end{remark}
\begin{definition}
	A function \( f \colon X \to Y \) between topological spaces is an \textit{open map} if for all open sets \( U \) in \( X \), \( f(U) \) is open in \( Y \).
\end{definition}
\begin{remark}
	\( f \colon X \to Y \) is a homeomorphism if and only if \( f \) is a continuous and open bijection.
\end{remark}

\subsection{Products}
Let \( X, Y \) be topological spaces.
We want to define the topology on \( X \times Y \).
If \( U \) is open in \( X \) and \( V \) is open in \( Y \), then we would like \( U \times V \) to be open in \( X \times Y \).
Certainly \( \varnothing = \varnothing \times \varnothing \) and \( X \times Y \) should be open.
Further \( (U \times V) \cap (U' \times V') = (U \cap U') \times (V \cap V') \), so intersections work.
\( \bigcup_{i \in I} U_i \times V_i \) must be open for open sets \( U_i, V_i \), but this is not obvious from what we have shown so far, so we must include this in our definition.
\begin{definition}
	The \textit{product topology} on \( X \times Y \) is the topology such that a subset \( U \) of \( X \times Y \) is open if there exists a set \( I \) and open sets \( U_i, V_i \) in \( X, Y \) for all \( i \in I \) such that
	\[
		U = \bigcup_{i \in I} U_i \times V_i
	\]
\end{definition}
\begin{remark}
	For \( W \subset X \times Y \), we know that \( W \) is open if and only if for all \( z \in W \), there exist open sets \( U \subset X, V \subset Y \), such that \( z \in U \times V \subset W \).
	So, thinking of the product as a product of real lines, we might say that \( W \) is open if for every point \( z \in W \), we can construct a `box set' (the Cartesian product of open intervals) contained in \( W \) that has \( z \) as an element.
	More formally, \( W \) is a neighbourhood of \( z \) if and only if there exist neighbourhoods \( U \) of \( x \) in \( X \) and \( V \) of \( y \) in \( Y \) such that \( U \times V \subset W \).
\end{remark}

\subsection{Continuity in product topology}
\begin{example}
	Let \( (M, d), (M', d') \) be metric spaces.
	Then, the metric \( d_\infty \) on \( M \times M' \) is
	\[
		d_\infty((x,x'), (y,y')) = \max(d(x,y), d'(x',y'))
	\]
	This metric is chosen since all \( d_p \) metrics induce the same metric topology, but this is easier to work with.
	Also, \( M, M' \) are topological spaces with their metric topologies, which induce the product topology on the product space \( M \times M' \).
	These two constructions create the same topology.
	For a point \( z = (x,x') \in M \times M' \) and \( r > 0 \), the open ball \( \mathcal D_r(z) \) is exactly\
	\begin{align*}
		\mathcal D_r(z) & = \qty{(y,y') \in M \times M' \colon d_\infty((y,y'), (x,x')) < r} \\
		                & = \qty{(y,y') \in M \times M' \colon d(x,y) < r, d(x',y') < r}     \\
		                & = \mathcal D_r(x) \times \mathcal D_r(x')
	\end{align*}
	Now, let \( W \subset M \times M' \).
	Then \( W \) is open in the product topology if and only if for all \( z = (x,x') \in W \), there exist open sets \( U \) in \( M \) and \( U' \) in \( M' \) such that \( (x,x') \in U \times U' \subset W \).
	Equivalently, for all \( z = (x,x') \in W \), there exists \( r > 0 \) such that \( \mathcal D_r(x) \times \mathcal D_r(x') \subset W \).
	But \( \mathcal D_r(x) \times \mathcal D_r(x') = \mathcal D_r(z) \), so \( W \) is \( d_\infty \)-open, as required.
	For instance, the product topology on \( \mathbb R \times \mathbb R \) is the Euclidean topology on \( \mathbb R^2 \).
\end{example}
\begin{proposition}
	Let \( X, Y \) be topological spaces.
	Let \( X \times Y \) be given the product topology.
	Then, the coordinate projections \( q_X \colon X \times Y \to X \) and \( q_Y \colon X \times Y \to Y \) satisfy
	\begin{enumerate}
		\item \( q_X, q_Y \) are continuous;
		\item if \( Z \) is any topological space, and \( g \colon Z \to X \times Y \) is a function, then \( g \) is continuous if and only if \( q_X \circ g, q_Y \circ g \) are continuous.
	\end{enumerate}
\end{proposition}
\begin{proof}
	If \( U \) is open in \( X \), then \( q_X^{-1}(U) = U \times Y \), which is the product of an open set in \( X \) and an open set in \( Y \), so is open in \( X \times Y \).
	Hence \( q_X \) is continuous.
	Similarly, \( q_Y \) is continuous.

	If \( g \) is continuous then certainly \( q_X \circ g, q_Y \circ g \) are continuous since the composition of continuous functions are continuous.
	Conversely, let \( h \colon Z \to X \) and \( k \colon Z \to Y \) be continuous functions with \( h = q_X \circ g \) and \( k = q_Y \circ g \).
	Then \( g(x) = (h(x), k(x)) \) for \( x \in Z \).
	Now, for open sets \( U \) in \( X \) and \( V \) in \( Y \), we have
	\[
		z \in g^{-1}(U \times V) \iff g(z) \in U \times V \iff h(z) \in U, k(z) \in V \iff z \in h^{-1}(U) \cap k^{-1}(V)
	\]
	So \( g^{-1}(U \times V) = h^{-1}(U) \cap k^{-1}(V) \) which is open in \( Z \) as \( h, k \) are continuous.
	Given an arbitrary open set \( W \) in \( X \times Y \), we can write \( W = \bigcup_{i\in I} U_i \times V_i \), where \( U_i \) are open in \( X \) and \( V_i \) are open in \( Y \).
	Thus, \( g^{-1}(W) = \bigcup_{i \in I} g^{-1}(U_i \times V_i) \) which is open.
\end{proof}
\begin{remark}
	The product topology may be extended to a finite product \( X_1 \times \dots \times X_n \), consisting of all unions of sets of the form \( U_1 \times \dots \times U_n \) where \( U_j \) is open in \( X_j \).
	Properties of the product topology hold in this more general case.
	For example, if \( X_j \) is metrisable with metric \( e_j \) for all \( j \), then the product topology is metrisable with, for instance, the \( d_\infty \) metric.
\end{remark}

\subsection{Quotients}
Let \( X \) be a set and \( R \) an equivalence relation on \( X \).
So \( R \subset X \times X \), but we will write \( x \sim y \) to mean \( (x,y) \in R \).
For \( x \in X \), we define \( q(x) = \qty{y \in X \colon y \sim x} \) to be the equivalence class of \( x \), the set of which partition \( X \).
Let \( X / R \) denote the set of all equivalence classes.
The map \( q \colon X \to X/R \) is called the quotient map.
\begin{definition}
	Let \( X \) be a topological space, and \( R \) an equivalence relation on \( X \).
	The \textit{quotient topology} on \( X/R \) is given by
	\[
		\tau = \qty{V \subset X/R \colon q^{-1}(V) \text{ open in } X }
	\]
	This is a topology:
	\begin{enumerate}
		\item \( q^{-1}(\varnothing) = \varnothing \) which is open, and \( q^{-1}(X/R) = X \) which is open.
		\item If \( V_i \) are open, then \( q^{-1}\qty(\bigcup_{i \in I} V_i) = \bigcup_{i \in I} q^{-1}(V_i) \) which is a union of open sets which is open.
		\item If \( U, V \) are open, then \( q^{-1}(U \cap V) = q^{-1}(U) \cap q^{-1}(V) \) which is open.
	\end{enumerate}
\end{definition}
\begin{remark}
	The quotient map \( q \colon X \to X/R \) is continuous.
	In particular, it is the largest possible topology on \( X \) such that \( q \) is continuous.

	Let \( x \in X, t \in X/R \).
	Then \( x \in t \) if and only if \( t = q(x) \).
	For \( V \subset X/R \),
	\[
		q^{-1}(V) = \qty{x \in X \colon q(x) \in V} = \qty{x \in X \colon \exists t \in V, t = q(x)} = \qty{x \in X \colon \exists t \in V, x \in t} = \bigcup_{t \in V} t
	\]
\end{remark}
\begin{example}
	Consider \( \mathbb R \), an abelian group under addition, and the subgroup \( \mathbb Z \).
	We can form the quotient group \( \mathbb R / \mathbb Z \), which is the set of equivalence classes where \( x \sim y \iff x - y \in \mathbb Z \).
	For all \( x \in \mathbb R \), there exists \( y \in [0,1] \) such that \( x \sim y \), and for all \( x, y \in [0,1] \) we have \( x \sim y \) if and only if \( x = y \) or \( \qty{x,y} = \qty{0,1} \).
	So we can think of the quotient topology of \( \mathbb R / \mathbb Z \) as a circle.
	We can say that \( \mathbb R / \mathbb Z \) is homeomorphic to \( S^1 = \qty{(x,y) \in \mathbb R^2 \colon \norm{(x,y)} = 1} \), which we will prove later.
\end{example}
\begin{example}
	Consider the subgroup \( \mathbb Q \) of \( \mathbb R \).
	Let \( V \subset \mathbb R / \mathbb Q \), such that \( V \neq \varnothing \) and \( V \) is open.
	Then \( q^{-1}(V) \) is open and not empty.
	Therefore, there exist \( a < b \in \mathbb R \) such that \( (a,b) \subset q^{-1}(V) \).
	Given \( x \in \mathbb R \), we can choose a rational \( r \) in the interval \( (a-x, b-x) \).
	Then \( r + x \in (a,b) \subset q^{-1}(V) \), so \( q(x) = q(r+x) \in V \).
	So \( V = \mathbb R / \mathbb Q \).
	This is the indiscrete topology, which is not metrisable or Hausdorff.
	So we cannot (in general) take quotients of metric spaces.
\end{example}
\begin{example}
	Let \( Q = [0,1] \times [0,1] \subset \mathbb R^2 \).
	We define the equivalence relation \( R \) given by
	\[
		(x_1, x_2) \sim (y_1, y_2) \iff
		\begin{cases}
			(x_1, x_2) = (y_1, y_2)               & \text{or} \\
			x_1 = y_1, \qty{x_2, y_2} = \qty{0,1} & \text{or} \\
			x_2 = y_2, \qty{x_1, y_1} = \qty{0,1} & \text{or} \\
			x_1, x_2, y_1, y_2 \in \qty{0,1}
		\end{cases}
	\]
	The space \( Q / R \) is homeomorphic to \( \mathbb R^2 / \mathbb Z^2 \).
	This is a square where the top and bottom edges are identified as the same, and the left and right edges are also identified as the same.
	This is homeomorphic to the surface of a torus with the Euclidean topology embedded in Euclidean three-dimensional space.
\end{example}
\begin{proposition}
	Let \( X \) be a set, and let \( R \) be an equivalence relation on \( X \).
	Let \( q \colon X \to X/R \) be the quotient map.
	Let \( Y \) be a set, and \( f \colon X \to Y \) be a function.
	Suppose that \( f \) `respects' \( R \); that is, \( x \sim y \implies f(x) = f(y) \).
	Then there exists a unique map \( \widetilde f \colon X/R \to Y \) such that \( f = \widetilde f \circ q \).
	For \( z \in X/R \), we write \( z = q(x) \) for some \( x \in X \), and then define \( \widetilde f(z) = f(x) \).
\end{proposition}
\begin{remark}
	Note that \( \Im f = \Im \widetilde f \) since \( q \) is surjective.
	\( \widetilde f \) is injective if for all \( x, y \in X \), \( \widetilde f(q(x)) = \widetilde f(q(y)) \) implies \( q(x) = q(y) \).
	In other words, for all \( x,y \in X \), \( f(x) = f(y) \implies x \sim y \).
	We say that \( f \) \textit{fully respects} \( R \) if, for all \( x,y \in X \),
	\[
		x \sim y \iff f(x) = f(y)
	\]
	In this case, \( \widetilde f \) is injective.
\end{remark}

\subsection{Continuity of functions in quotient spaces}
\begin{proposition}
	Let \( X \) be a topological space and let \( R \) be an equivalence relation on \( X \).
	Let \( q \colon X \to X/R \) be a quotient map, where \( X/R \) has the quotient topology.
	Let \( Y \) be another topological space and \( f \colon X \to Y \) be a function that respects \( R \).
	Let \( \widetilde f \colon X/R \to Y \) be the unique map such that \( f = \widetilde f \circ q \).
	Then
	\begin{enumerate}
		\item if \( f \) is continuous then \( \widetilde f \) is continuous; and
		\item if \( f \) is an open map (the image of an open set is open) then \( \widetilde f \) is an open map.
	\end{enumerate}
	In particular, if \( f \) is a continuous surjective map that fully respects \( R \), then \( \widetilde f \) is a continuous bijection.
	If in addition \( f \) is an open map, then \( \widetilde f \) is a continuous bijective open map, so is a homeomorphism.
\end{proposition}
\begin{proof}
	We prove part (i).
	Let \( V \) be an open set in \( Y \).
	\[
		q^{-1}\qty(\widetilde f^{-1}(V)) = (\widetilde f \circ q)^{-1}(V) = f^{-1}(V) \text{ is open}
	\]
	So by definition, \( \widetilde f^{-1}(V) \) is open in \( X/R \).
	Hence \( \widetilde f \) is continuous.
	Now, we prove part (ii).
	Let \( V \) be an open set in \( X/R \).
	Let \( U = q^{-1}(V) \).
	Then \( U \) is open in \( X \) by definition of the quotient topology.
	Since \( q \) is surjective, \( q(U) = q\qty(q^{-1}(V)) = V \).
	Hence,
	\[
		\widetilde f(V) = \widetilde f(q(U)) = (\widetilde f \circ q)(U) = f(U) \text{ is open}
	\]
	since \( f \) is an open map.
\end{proof}
\begin{example}
	\( \mathbb R / \mathbb Z \) is homeomorphic to a circle \( S^1 = \qty{x \in \mathbb R^2 \colon \norm{x} = 1} \).
	We define
	\[
		f(t) = (\cos 2 \pi t, \sin 2 \pi t)
	\]
	Then, \( s - t \in \mathbb Z \) if and only if \( f(s) = f(t) \) so \( f \) fully respects the relation, and \( f \) is surjective.
	\( f \) is also continuous since each component is continuous.
	Hence, there exists \( \widetilde f \colon \mathbb R / \mathbb Z \to S^1 \) such that \( f = \widetilde f \circ q \) and \( \widetilde f \) is a continuous bijection.
	Now we must show \( f \) is an open map, and then \( \widetilde f \) will be a homeomorphism.
	Suppose \( f \) is not an open map, so there exists an open set \( U \) in \( \mathbb R \) such that \( f(U) \) is not open in \( S^1 \).
	So \( S^1 \setminus f(U) \) is not closed, so there exists a sequence \( (z_n) \) in this complement and \( z \in f(U) \) such that \( z_n \to z \).
	\( f \) is surjective so for all \( n \in N \) we can choose \( x_n \in [0,1] \) such that \( f(x_n) = z_n \).
	This is a bounded sequence, so by the Bolzano-Weierstrass theorem, without loss of generality we can let \( x_n \to x \in [0,1] \).
	Since \( f \) is continuous, \( f(x_n) \to f(x) \), so \( z_n \to z \).
	But since \( z_n \not\in f(U) \), we have \( x_n \in \mathbb R \setminus U \).
	Since the complement is closed and \( x_n \to x \), we have \( x \in \mathbb R \setminus U \) so \( x \not\in U \).
	Since \( z \in f(U) \), there exists \( y \in U \) such that \( z = f(y) \).
	Hence \( k = y - x \in \mathbb Z \).
	Now, \( f(x_n + k) = f(x_n) = z_n \to z \), but also \( x_n + k \to x + k = y \in U \).
	Since \( z_n \not\in f(U) \), we have \( x_n + k \not\in U \).
	Since \( \mathbb R \setminus U \) is closed and \( x_n + k \to y \), we have \( y \in \mathbb R \setminus U \) which is a contradiction.
\end{example}
\begin{proposition}
	Let \( X \) be a topological space, and \( R \) an equivalence relation on \( X \).
	Then,
	\begin{enumerate}[(a)]
		\item If \( X / R \) is Hausdorff, then \( R \) is closed in \( X \times X \).
		\item If \( R \) is closed in \( X \times X \) and the quotient map \( q \colon X \to X/R \) is an open map, then \( X / R \) is Hausdorff.
	\end{enumerate}
\end{proposition}
\begin{proof}
	Let \( W = X \times X \setminus R \).
	For part (a), we want to show \( W \) is open, so is a neighbourhood of all of its points.
	Given \( (x,y) \in W \), we have \( x \not\sim y \), so \( q(x) \neq q(y) \).
	Since the quotient is Hausdorff, there exist open sets \( S, T \) in \( X/R \) such that \( S \cap T = \varnothing \) and \( q(x) \in S, q(y) \in T \).
	Let \( U = q^{-1}(S), V = q^{-1}(T) \) which are open in \( X \), and \( x \in U, y \in V \).
	For all \( (a,b) \in U \times V \), we have \( q(a) \in S, q(b) \in T \) hence \( a \not\sim b \).
	So \( (x,y) \in U \times V \subset W \).
	Hence \( R \) is closed.

	For part (b), let \( z \neq w \) be elements of \( X/R \), and we want to separate these points by open sets.
	Let \( x,y \in X \) such that \( q(x) = z, q(y) = w \).
	Then \( (x,y) \in W \) since \( x \not\sim w \).
	Since \( R \) is closed, \( W \) is open, so there exist open sets \( U, V \) in \( X \) such that \( (x,y) \in U \times V \subset W \).
	Since \( q \) is an open map, \( q(U) \) and \( q(V) \) are open in \( X/R \), and \( z = q(x) \in q(U), w = q(y) \in q(V) \).
	Now it suffices to show \( q(U) \cap q(V) = \varnothing \).
	For \( (a,b) \in U \times V \subset W \), \( (a,b) \not\in R \) hence \( q(a) \neq q(b) \) so \( q(U) \cap q(V) = \varnothing \).
\end{proof}
