\subsection{Properties of topology of metric space}
\begin{definition}
	A subspace \( A \) of a metric space \( M \) is \textit{closed in} \( M \) if for every sequence \( (x_n) \in A \) that is convergent in \( M \),
	\[
		\lim_{n \to \infty} x_n \in A
	\]
\end{definition}
\begin{lemma}
	Closed balls are closed.
\end{lemma}
\begin{proof}
	Consider \( \mathcal B_r(x) \) in \( M \).
	Consider further \( (x_n) \in \mathcal B_r(x) \) such that \( x_n \to z \) in \( M \).
	\[
		d(z,x) \leq d(z,x_n) + d(x_n,x) \leq d(z,x_n) + r \to r
	\]
	Hence \( d(z,x) \leq r \), so \( z \in \mathcal B_r(x) \).
\end{proof}
\begin{example}
	\( [0,1] = \mathcal B_{1/2}(1/2) \) is closed in \( \mathbb R \).
	This is not open, for instance consider \( D_r(0) \not\subset [0,1] \).
\end{example}
\begin{example}
	\( (0,1) = \mathcal D_{1/2}(1/2) \) is open in \( \mathbb R \).
	This is not closed, for instance the sequence \( \frac{1}{n+1} \) tends to zero in \( \mathbb R \).
\end{example}
\begin{example}
	\( \mathbb R \) and \( \varnothing \) are open and closed in \( \mathbb R \).
\end{example}
\begin{example}
	\( (0,1] \) in \( \mathbb R \) is neither open nor closed.
	Consider \( \mathcal D_r(1) \not\subset (0,1] \) and \( \frac{1}{n} \to 0 \not\in (0,1] \).
\end{example}
\begin{lemma}
	Let \( A \subset M \).
	Then \( A \) is closed in \( M \) if and only if \( M \setminus A \) is open in \( M \).
\end{lemma}
\begin{proof}
	Let \( A \) be closed.
	Suppose \( M \setminus A \) is not open.
	Then \( \exists x \in M \setminus A, \forall r > 0, \mathcal D_r(x) \not\subset M \setminus A \), so \( \mathcal D_r(x) \cap A \neq \varnothing \).
	In particular, for every \( n \) we can choose a point in \( \mathcal D_{1/n}(x) \cap A \).
	Then, \( d(x_n,x) < \frac{1}{n} \to 0 \) and \( x_n \in A \) which contradicts the fact that \( A \) is closed.

	Conversely, let us assume \( M \setminus A \) is open, but suppose \( A \) is not closed.
	Then there exists a sequence \( (x_n) \in A \) such that \( x_n \to x \) in \( M \) but \( x \not\in A \).
	Since \( x \in M \setminus A \) and \( M \setminus A \) is open, there exists \( \varepsilon > 0, \mathcal D_\varepsilon(x) \subset M \setminus A \).
	Since \( x_n \to x \), we must have \( \exists N \in \mathbb N, \forall n \geq N, x_n \in \mathcal D_\varepsilon(x) \) and hence \( x_n \in M \setminus A \), which is a contradiction.
\end{proof}
\begin{example}
	Let \( M \) be a discrete metric space.
	Let \( A \subset M \).
	Then for all \( x \in A \), \( \mathcal D_1(x) = \qty{x} \subset A \).
	Hence \( A \) is open.
	So in a discrete metric space, all subsets are open.
	Hence every subset is closed.
\end{example}

\subsection{Homeomorphisms}
\begin{definition}
	A map \( f \colon M \to M' \) between metric spaces is called a \textit{homeomorphism} if \( f \) is a bijection and \( f, f^{-1} \) are continuous.
	Equivalently, \( f \) is a bijection, and for all open sets \( V \) in \( M' \), \( f^{-1}(V) \) is open in \( M \), and for all open sets \( U \) in \( M \), \( f(U) \) is open in \( M' \).
	If there exists a homeomorphism between \( M, M' \), we say that \( M, M' \) are homeomorphic.
\end{definition}
\begin{example}
	Consider \( (0,\infty) \) and \( (0,1) \).
	Consider the map \( x \mapsto \frac{1}{x+1} \) with inverse \( x \mapsto \frac{1}{x} - 1 \).
	These are continuous, so the metric spaces are homeomorphic.
\end{example}
\begin{remark}
	Every isometry is a homeomorphism.
	It is not true that every homeomorphism is an isometry.

	Consider the identity on \( \mathbb R \) with the discrete metric to \( \mathbb R \) with the Euclidean metric.
	This is a continuous bijection whose inverse is not continuous.
	So it is not true that a continuous bijection always has a continuous inverse.
\end{remark}

\subsection{Equivalence of metrics}
\begin{definition}
	Let \( d, d' \) be metrics on a set \( M \).
	We say that \( d, d' \) are \textit{equivalent}, written \( d \sim d' \), if they define the same topology.
	In particular, \( U \subset M \) is open in \( (M,d) \) if and only if \( U \) is open in \( (M,d') \).
	So \( d \sim d' \) if and only if \( \mathrm{id} \colon (M,d) \to (M,d') \) is a homeomorphism.
\end{definition}
\begin{remark}
	If \( d \sim d' \), then \( (M,d) \) and \( (M, d') \) have the same convergent sequences and continuous maps.
\end{remark}
\begin{definition}
	Let \( d, d' \) be metrics on \( M \).
	Then we say \( d, d' \) are \textit{uniformly equivalent}, written \( d \sim_u d' \) if
	\[
		\mathrm{id} \colon (M, d) \to (M, d');\quad \mathrm{id} \colon (M, d') \to (M, d)
	\]
	are uniformly continuous.
	We say \( d, d' \) are \textit{Lipschitz equivalent}, written \( d \sim_\mathrm{Lip} d' \), if the identity maps above are Lipschitz.
	Equivalently, \( d \sim_\mathrm{Lip} d' \) if \( \exists a > 0, b > 0, ad(x,y) \leq d'(x,y) \leq bd(x,y) \).
	Note, \( d \sim_\mathrm{Lip} d' \implies d \sim_u d' \implies d \sim d' \).
\end{definition}
\begin{example}
	Given a metric space \( (M,d) \), we define \( d'(x,y) = \min(1,d(x,y)) \).
	This defines a metric on \( M \), and \( d' \sim_u d \).
\end{example}
\begin{example}
	On \( M \times M' \), \( d_1, d_2, d_\infty \) are pairwise Lipschitz equivalent.
\end{example}
\begin{example}
	Consider \( C[0,1] \).
	The \( L_1 \) metric and the uniform metric are not equivalent.
	Consider \( f_n(x) = x^n \).
	This is convergent to zero in the \( L_1 \) metric but is not convergent in the uniform metric.
\end{example}
\begin{example}
	The discrete metric and Euclidean metric on \( \mathbb R \) are not equivalent.
	This is becaue in the discrete metric all sets are open, but in the Euclidean metric there are some non-open sets.
\end{example}

\subsection{Completeness}
In \( \mathbb R, \mathbb C \), every Cauchy sequence is convergent.
We wish to generalise this notion to an arbitrary metric space.
Recall that a sequence \( (x_n) \) in \( \mathbb R \) or \( \mathbb C \) is bounded if there exists \( c \in \mathbb R^+ \) such that \( \forall n \in \mathbb N, \abs{x_n} \leq c \).
\begin{definition}
	A sequence \( (x_n) \) in a metric space \( M \) is said to be \textit{Cauchy} if
	\[
		\exists \varepsilon > 0, \exists N \in \mathbb N, \forall m,n \geq N, d(x_m,x_n) < \varepsilon
	\]
	The sequence is bounded if
	\[
		\exists z \in M, \exists r > 0, \forall n \in \mathbb N, x_n \in \mathcal B_r(z)
	\]
	This is equivalent to
	\[
		\forall z \in M, \exists r > 0, \forall n \in \mathbb N, x_n \in \mathcal B_r(z)
	\]
	by considering the triangle inequality around the given \( z \) point.
	In particular, if the metric arises from a norm, \( (x_n) \) is bounded if and only if \( \norm{x_n} \) is bounded.
\end{definition}
\begin{lemma}
	If a sequence is convergent, it is Cauchy.
	If a sequence is Cauchy, it is bounded.
\end{lemma}
\begin{proof}
	Let \( (x_n) \) be a sequence in \( M \).
	First, we assume that \( (x_n) \) is convergent in \( M \), so let \( x \) be the limit.
	Given \( \varepsilon > 0 \), there exists \( N \in \mathbb N \) such that \( \forall n \geq N \), \( d(x_n, x) < \varepsilon \).
	Then, for all \( m, n \geq N \), we have \( d(x_m, x_n) \leq d(x_m, x) + d(x, x_n) \leq 2\varepsilon \) as required.
	So \( (x_n) \) is Cauchy.

	Now conversely, we assume \( (x_n) \) is Cauchy.
	There exists \( n \in \mathbb N \) such that \( \forall m, n \geq N \), we have \( d(x_m, x_n) < 1 \).
	In particular, \( d(x_n, x_N) < 1 \) for \( n \geq N \).
	In other words, \( x_n \in \mathcal B_1(x_N) \).
	Now, let \( r = \max{d(x_1, x_N), \dots, d(x_{N-1}, x_N), 1} \).
	This \( r \) is a bound for all elements of the sequence; for all \( n \in \mathbb N, x_n \in \mathcal B_r(x_N) \).
\end{proof}
\begin{remark}
	Boundedness does not imply the sequence is Cauchy.
	For instance, consider \( 0,1,0,1,\dots \) in \( \mathbb R \).
	If a sequence is Cauchy, it is not necessary convergent in an arbitrary metric space (not \( \mathbb R, \mathbb C \)).
	For instance, consider \( x_n = \frac{1}{n} \) in \( (0, \infty) \).
	This is certainly not convergent, since the limit cannot be zero.
\end{remark}
