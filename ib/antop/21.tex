\subsection{Restricted definition of derivative}
We may define the derivative on a subset of \( \mathbb R^m \).
We will use the notion of open subsets since we are typically interested in neigbourhoods of points.
\begin{definition}
	Let \( U \) be an open subset of \( \mathbb R^m \).
	Let \( f \colon U \to \mathbb R^n \) be a function, and \( a \in U \).
	Then we say \( f \) is \textit{differentiable} at \( a \) if there exists a linear map \( T \in L(\mathbb R^m, \mathbb R^n) \) such that
	\[
		f(a+h) = f(a) + T(h) + \norm{h} \varepsilon(h)
	\]
	where \( \epsilon(0) = 0 \) and \( \varepsilon \) is continuous at zero.
	Note that \( \varepsilon \) need only be defined on the set of \( h \) such that \( a + h \in U \), or more precisely the open set \( U - a \).
	Hence there exists \( r > 0 \) such that \( \mathcal D_r(0) \subset U_a \).
	Then
	\[
		\varepsilon(h) = \begin{cases}
			0                                     & h = 0                 \\
			\frac{f(a+h) - f(a) - T(h)}{\norm{h}} & h \neq 0, a + h \in U
		\end{cases}
	\]
	So \( f \) is differentiable at \( a \) if and only if there exists a linear map \( T \in L(\mathbb R^m, \mathbb R^n) \) such that
	\[
		\frac{f(a+h) - f(a) - T(h)}{\norm{h}} \to 0
	\]
\end{definition}
\begin{remark}
	The linear map \( T \) is unique, and is called the \textit{derivative} of \( f \) at \( a \), denoted \( f'(a) \).
	In particular,
	\[
		f(a+h) = f(a) + f'(a)(h) + o(\norm{h})
	\]
\end{remark}
\begin{remark}
	If \( m = 1 \), the space \( L(\mathbb R, \mathbb R^n) \) is isomorphic to \( \mathbb R^n \).
	The linear map is defined uniquely by a vector in \( \mathbb R^n \) which multiplies by the scalar \( h \).
	Hence, if \( U \subset \mathbb R \) is open and \( f \colon U \to \mathbb R \) be a function and \( a \in U \), then \( f \) is differentiable at \( a \) if there exists a vector \( v \in \mathbb R^n \) such that
	\[
		\frac{f(a+h) - f(a) - hv}{\abs{v}} \to 0
	\]
	Equivalently, there exists \( v \in \mathbb R^n \) such that
	\[
		\frac{f(a+h) - f(a)}{h} \to v
	\]
\end{remark}

\subsection{Properties of derivative}
\begin{proposition}[chain rule]
	Let \( U \subset \mathbb R^m \) be open, \( f \colon U \to \mathbb R^n \) be a function, and \( a \in U \).
	If \( f \) is differentiable at \( a \), \( f \) is continuous at \( a \).
\end{proposition}
\begin{proof}
	We have
	\[
		f(a+h) = f(a) + f'(a)(h) + \norm{h} \varepsilon(h)
	\]
	Hence,
	\[
		f(x) = f(a) + f'(a)(x-a) + \norm{x-a} \varepsilon(x-a)
	\]
	The functions \( x \mapsto f(a) \), \( x \mapsto f'(a)(x-a) \) and \( x \mapsto \norm{x-a}\varepsilon(x-a) \) are all continuous at \( a \).
	Hence their sum is continuous.
\end{proof}
\begin{proposition}
	Let \( U \subset \mathbb R^m \) and \( V \subset \mathbb R^n \) be open, \( f \colon U \to \mathbb R^n \) and \( g \colon V \to \mathbb R^p \) be functions, and \( a \in U, b \equiv f(a) \in V \).
	Suppose \( f \) is differentiable at \( a \), and \( g \) is differentiable at \( b \).
	Then \( g \circ f \) is differentiable at \( a \) and
	\[
		(g \circ f)'(a) = g'(b) \circ f'(a)
	\]
\end{proposition}
\begin{proof}
	Let \( S = f'(a) \) and \( T = g'(b) \).
	Then by assumption
	\[
		f(a+h) = f(a) + S(h) + \norm{h} \varepsilon(h);\quad g(b+k) + g(b) + T(k) + \norm{k} \zeta(k)
	\]
	for suitable \( \varepsilon, \zeta \).
	Then,
	\begin{align*}
		(g \circ f)(a+h) & = g\qty(f(a) + S(h) + \norm{h} \varepsilon(h))                                                                               \\
		                 & = g\qty(b + \underbrace{S(h) + \norm{h} \varepsilon(h)}_{k})                                                                 \\
		                 & = g(b) + T\qty(S(h) + \norm{h} \varepsilon(h)) + \norm{S(h) + \norm{h} \varepsilon(h)} \zeta(S(h) + \norm{h} \varepsilon(h)) \\
		                 & = (g \circ f)(a) + (T \circ S)(h) + \norm{h} T(\varepsilon(h)) + \norm{k} \zeta(k)
	\end{align*}
	It suffices to show that
	\[
		\eta(h) \equiv \norm{h} T(\varepsilon(h)) + \norm{k} \zeta(k)
	\]
	satisfies \( \frac{\eta}{\norm{h}} \to 0 \).
	Then the result follows.
	First,
	\[
		\frac{\norm{h}T(\varepsilon(h))}{\norm{h}} = T(\varepsilon(h)) \to 0
	\]
	as \( \norm{T(\varepsilon(h))} \leq \norm{T} \cdot \norm{\varepsilon(h)} \to 0 \).
	Then,
	\[
		\frac{\norm{k}}{\norm{h}} = \frac{\norm{S(h)} + \norm{h} \cdot \norm{\varepsilon(h)}}{\norm{h}} \leq \norm{S} + \norm{\varepsilon(h)}
	\]
	Hence, \( k = S(h) + \norm{h} \cdot \varepsilon(h) \to 0 \) as \( h \to 0 \).
	Thus \( \zeta(k) \to 0 \) as \( k \to 0 \).
	So
	\[
		\frac{\eta(h)}{\norm{h}} = T(\varepsilon(h)) + \frac{\norm{k}}{\norm{h}} \zeta(k) \to 0
	\]
	as required.
\end{proof}
\begin{proposition}
	Let \( U \subset \mathbb R^m \) be open, \( f \colon U \to \mathbb R^n \) be a function, and \( a \in U \).
	Let \( f_j \) be the \( j \)th component of \( f \), so \( f_j = \pi_j \circ f \).
	Then \( f \) is differentiable at \( a \) if and only if each \( f_j \) is differentiable at \( a \).
	If this holds,
	\[
		f'(a)(h) = \sum_{j=1}^n f_j'(a)(h) e_j'
	\]
	Equivalently,
	\[
		\pi_j \qty[ f'(a)(h) ] = f_j'(a)(h)
	\]
\end{proposition}
\begin{proof}
	If \( f \) is differentiable at \( a \), by the chain rule the composite \( \pi_j \circ f \) is differentiable at \( a \).
	Since the derivative of a linear map is itself, the derivative is given by
	\[
		f_j'(a) = \pi_j'(f(a)) \circ f'(a) = \pi_j \circ f'(a)
	\]
	Hence
	\[
		f'(a)(h) = \sum_{j=1}^n \pi_j\qty[ f'(a)(h) e_j' ] = \sum_{j=1}^n f_j'(a)(h) e_j'
	\]
	Conversely suppose each \( f_j \) is differentiable.
	Then
	\[
		f_j(a+h) = f_j(a) + f_j'(a)(h) + \norm{h} \varepsilon_j(h)
	\]
	for suitable \( \varepsilon(j) \).
	Now,
	\begin{align*}
		f(a+h) & = \sum_{j=1}^n f_j(a+h)e_j'                                                                             \\
		       & = \sum_{j=1}^n \qty[f_j(a) + f_j'(a)(h) + \norm{h} \varepsilon_j(h)] e_j'                               \\
		       & = \sum_{j=1}^n f_j(a) e_j' + \sum_{j=1}^n f_j'(a)(h) e_j' + \norm{h} \sum_{j=1}^n \varepsilon_j(h) e_j'
	\end{align*}
	Since each \( \varepsilon_j \) tends to zero as \( h \to 0 \), so does their sum.
\end{proof}
\begin{remark}
	This proposition shows that we can prove things for an image \( \mathbb R^n = \mathbb R \) without loss of generality.
\end{remark}

\subsection{Linearity and product rule}
\begin{proposition}
	Let \( U \subset \mathbb R^m \) be open and functions \( f,g \colon U \to \mathbb R^n \), \( \phi \colon U \to \mathbb R \) which are differentiable at \( a \).
	Then the functions \( f + g \) and \( \phi \cdot f \) are also differentiable and their derivatives are
	\[
		(f+g)'(a) = f'(a) + g'(a);\quad (\phi f)'(a)(h) = \phi(a)\qty[f'(a)(h)] + \qty[\phi'(a)(h)] f(a)
	\]
	For \( m = n = 1 \) this is the usual product rule.
\end{proposition}
\begin{proof}
	We have
	\begin{align*}
		f(a+h)    & = f(a) + f'(a)(h) + \norm{h}\varepsilon(h) \\
		g(a+h)    & = g(a) + g'(a)(h) + \norm{h} \zeta(h)      \\
		\phi(a+h) & = \phi(a) + \phi'(a)(h) + \norm{h}\eta(h)
	\end{align*}
	for suitable \( \varepsilon, \zeta, \eta \).
	The sum gives
	\[
		(f+g)(a+h) = (f+g)(a+h) + (f'(a) + g'(a))(h) + \norm{h}(\varepsilon(h) + \zeta(h))
	\]
	It follows that \( f+g \) is differentiable at \( a \) and its derivative is the sum of the derivatives of its components.
	\begin{align*}
		(\phi \cdot f)(a+h) & = \phi(a+h) f(a+h)                                                                                                                                                \\
		                    & = (\phi \cdot f)(a) + \qty[\phi(a) f'(a)(h) + \phi'(a)(h) f(a)] + f'(a)(h) \phi'(a)(h)                                                                            \\
		                    & + \norm{h} \underbrace{\qty(f'(a)(h) \eta(h) + \phi'(a)(h) \varepsilon(h) + \eta(h) f(a) + \phi(a) \varepsilon(h) + \norm{h} \eta(h) \varepsilon(h))}_{\delta(h)}
	\end{align*}
	Now,
	\[
		\frac{\norm{\phi'(a)(h) \cdot f'(a)(h)}}{\norm{h}} = \frac{\abs{\phi'(a)(h)} \cdot \norm{f'(a)(h)}}{\norm{h}} \leq \frac{\norm{\phi'(a)} \cdot \norm{h} \cdot \norm{f'(a)} \cdot \norm{h}}{\norm{h}} \to 0
	\]
	Clearly \( \delta \to 0 \) since the same is true for all of its components.
\end{proof}
