\subsection{???}
\begin{example}
	Any indiscrete topological space is connected.
	Any cofinite topological space on an infinite set is connected.
	The discrete topological space on a set of size at least two is disconnected.
\end{example}
\begin{lemma}
	Let \( Y \) be a subspace of a topological space \( X \).
	Then, \( Y \) is disconnected if and only if there exist open subsets \( U, V \) of \( X \) such that \( U \cap V \cap Y = \varnothing \) and \( U \cup V \supset Y \), and \( U \cap Y \neq \varnothing, V \cap Y \neq \varnothing \).
\end{lemma}
\begin{proof}
	Suppose \( Y \) is disconnected.
	Then there exist open subsets \( U', V' \) of \( Y \) that disconnect \( Y \).
	Then there exist open sets \( U, V \) in \( X \) such that \( U' = U \cap Y \) and \( V' = V \cap Y \).
	Then \( U, V \) satisfy the requirements from the lemma.

	Conversely, suppose \( U, V \) are as given.
	Then, let \( U' = U \cap Y, V' = V \cap Y \).
	They are open in \( Y \) by the definition of the subspace topology, and they disconnect \( Y \).
\end{proof}
\begin{remark}
	In the above lemma, we say subsets \( U, V \) of \( X \) disconnect \( Y \).
\end{remark}
\begin{proposition}
	Let \( Y \) be a subspace of a topological space \( X \).
	If \( Y \) is connected, then so is \( \overline Y \).
\end{proposition}
\begin{proof}
	Suppose \( \overline Y \) is disconnected.
	Then there exist open sets \( U, V \) in \( X \) which disconnect \( \overline Y \).
	Then \( U \cap V \cap Y \subset U \cap V \cap \overline Y = \varnothing \) by definition.
	Hence \( U \cap V \cap Y = \varnothing \).
	Also, \( U \cup V \supset \overline Y \supset Y \).
	So \( U, V \) disconnect \( Y \) unless \( U \cap Y = \varnothing \) or \( V \cap Y = \varnothing \).
	But \( Y \) is connected, so without loss of generality let \( V \cap Y = \varnothing \).
	Then \( Y \subset X \setminus V \) and \( X \setminus V \) is closed, so \( \overline Y \subset X \setminus V \).
	Hence \( V \cap \overline Y = \varnothing \).
	This is a contradiction since \( U, V \) disconnect \( \overline Y \).
\end{proof}
\begin{remark}
	More generally, if \( Y \subset Z \subset \overline Y \), and \( Y \) is connected, then \( Z \) is connected.
	This is since \( \mathrm{cl}_Z(Y) = \mathrm{cl}_X(Y) \cap Z = Z \).
\end{remark}
\begin{theorem}
	Let \( f \colon X \to Y \) be a continuous function between topological spaces.
	If \( X \) is connected, then so is \( f(X) \).
\end{theorem}
\begin{proof}
	Let \( U, V \) be open subsets of \( Y \) which disconnect \( f(X) \).
	For \( x \in X \), \( f(x) \in f(X) \subset U \cup V \).
	Hence, \( f^{-1}(U) \cup f^{-1}(V) = X \).
	Also, if \( x \in f^{-1}(U) \cap f^{-1}(V) \) then \( f(x) \in U \cap V \cap f(X) = \varnothing \).
	This is a contradiction, so \( f^{-1}(U) \cap f^{-1}(V) = \varnothing \).
	Since \( f \) is continuous, \( f^{-1}(U), f^{-1}(V) \) are open in \( X \).
	Since \( U \cap f(X) \neq \varnothing \) and \( V \cap f(X) \neq \varnothing \), \( f^{-1}(U) \neq \varnothing \) and \( f^{-1}(V) \neq \varnothing \)
	So \( f^{-1}(U), f^{-1}(V) \) disconnect \( X \).
\end{proof}
\begin{remark}
	This shows that connectedness is a topological property.
	If \( X, Y \) are homeomorphic spaces, then \( X \) is connected if and only if \( Y \) is connected.
	Further, note that if \( f \colon X \to Y \) is continuous and \( A \subset X \) and \( A \) is connected, then \( f(A) \) is connected.
	This can be shown by restricting \( f \) to the domain \( A \).
\end{remark}
\begin{corollary}
	Any quotient of a connected topological space is connected.
\end{corollary}
\begin{example}
	Let \( Y = \qty{\qty(x, \sin\frac{1}{x}) \colon x > 0} \subset \mathbb R^2 \).
	This space is connected; the function \( f \colon (0, \infty) \to \mathbb R^2 \) defined by \( f(x) = \qty(x,\sin \frac{1}{x}) \) is continuous.
	So we have that \( Y = \Im f \) is connected.
	Hence, \( \overline Y \) is connected.
	We claim that \( Z \equiv Y \cup \qty{(0,y) \colon y \in [-1,1]} = \overline Y \).
	Indeed, given \( y \in [-1,1] \), for all \( n \in \mathbb N \) we have that \( (0, \frac{1}{n}) \) is mapped to \( (n,\infty) \) by \( x \to \frac{1}{x} \), so by the intermediate value theorem there exists \( x_n \in \qty(0, \frac{1}{n}) \) such that \( \sin \frac{1}{x_n} = y \).
	Hence, \( \qty(x_n, \sin \frac{1}{x_n}) = (x_n, y) \to (0,y) \in \overline Y \).
	So \( Y \subset Z \subset \overline Y \).
	If we can show \( Z \) is closed, \( Z = \overline Y \) since \( \overline Y \) is the smallest closed superset of \( Y \).
	Suppose \( (x_n, y_n) \in Z \) for all \( n \in \mathbb N \), and \( (x_n, y_n) \to (x,y) \) in \( \mathbb R^2 \).
	Since \( y_n \in [-1,1] \) and \( y_n \to y \), we have \( y \in [-1,1] \).
	If \( x = 0 \), we have \( (x,y) \in Z \).
	If \( x \neq 0 \), then \( x_n \to x \) implies \( x_n \neq 0 \) for all sufficiently large \( n \).
	Hence \( y_n = \sin \frac{1}{x_n} \) for all sufficiently large \( n \).
	Hence \( (x_n, y_n) \to \qty(x, \sin \frac{1}{x}) \in Z \).
\end{example}
\begin{lemma}
	Let \( X \) be a topological space and \( \mathcal A \) be a family of connected subsets of \( X \).
	Suppose that \( A \cap B \neq \varnothing \) for all \( A, B \in \mathcal A \).
	Then \( \bigcup_{A \in \mathcal A} A \) is connected.
\end{lemma}
\begin{proof}
	Let \( Y = \bigcup_{A \in \mathcal A} A \), and let \( f \colon Y \to \mathbb Z \) be a continuous function.
	We must show that \( f \) is constant.
	For all \( A \in \mathcal A \), \( \eval{f}_A \colon A \to \mathbb Z \) is continuous and hence constant, since \( A \) is connected.
	For all \( A, B \in \mathcal A \), \( A \cap B \neq \varnothing \) hence \( \eval{f}_A \) and \( \eval{f}_B \) are both constant and have the same value.
	So \( f \) must be constant, and hence \( Y \) is connected.
\end{proof}
\begin{theorem}
	Let \( X, Y \) be connected topological spaces.
	Then \( X \times Y \) is connected (in the product topology).
\end{theorem}
\begin{proof}
	Without loss of generality, let \( X \neq \varnothing, Y \neq \varnothing \).
	Let \( x_0 \in X \).
	Consider the function \( f \colon Y \to X \times Y \) defined by \( f(y) = (x_0, y) \).
	The components of \( f \) are the functions \( y \mapsto x_0 \) which is continuous as it is constant, and \( y \mapsto y \) which is continuous as it is the identity.
	So \( f \) is continuous.
	Then, the image of \( f \), which is \( \qty{x_0} \times Y \), is connected.
	Similarly, for all \( y \in Y \), \( X \times \qty{y} \) is connected.
	For \( y \in Y \), \( \qty{x_0} \times Y \cap X \times \qty{y} = \qty{(x_0, y)} \neq \varnothing \).
	Hence, \( A_y = \qty{x_0} \times Y \cup X \times \qty{y} \) is connected.
	For all \( y,z \in Y \), \( A_y \cap A_z \supset \qty{x_0} \times Y \) hence \( A_y \cap A_z \neq \varnothing \).
	Hence, \( \bigcup_{y \in Y} A_y = X \times Y \) is connected.
\end{proof}
\begin{example}
	\( \mathbb R^n \) is connected for all \( n \in \mathbb N \).
\end{example}

\subsection{Partitioning into connected components}
\begin{definition}
	Let \( X \) be a topological space.
	We define a relation \( \sim \) on \( X \) by \( x \sim y \) if and only if there exists a connected subset \( A \) of \( X \) such that \( x, y \in A \).
	For all \( x \in X \), \( x \sim x \) since \( \qty{x} \) is connected.
	Symmetry is clear from the definition.
	If \( x \sim y \) and \( y \sim z \) then by definition there exist connected subsets \( A, B \) in \( X \) such that \( x, y \in A \) and \( y, z \in B \).
	In particular, \( A \cap B \) is not empty since \( y \in A \cap B \).
	Hence \( A \cup B \) is connected.
	Since \( A \cup B \) contains \( x, z \), we have \( x \sim z \) as required for transitivity.
	Hence \( \sim \) is an equivalence relation.
	For \( x \in X \), we write \( C_x \) for the equivalence class containing \( x \), called the \textit{connected component} of \( x \).
	The equivalence classes are called \textit{connected components} of \( X \).
\end{definition}
\begin{proposition}
	The connected components of a topological space \( X \) are non-empty, maximal connected subsets of \( X \), they are closed, and they partition \( X \).
\end{proposition}
\begin{proof}
	Let \( C \) be a connected component of \( X \).
	So \( C = C_x \) for some \( x \in X \).
	Then \( x \in C \) hence \( C \neq \varnothing \).
	Suppose \( C \subset A \subset X \) and \( A \) is connected.
	Then for all \( y \in A \), since \( x, y \in A \) we must have \( x \sim y \).
	So \( y \in C \).
	Hence \( A \subset C \), giving \( A = C \).
	For all \( y \in C \), we have \( y \sim x \), so there exists a connected subset \( A_y \subset X \) such that \( x, y \in A_y \).
	Let \( A = \bigcup_{y \in C} A_y \).
	\( A \) is connected since the union of pairwise intersecting connected sets are connected.
	Further \( A \supset C \) so \( A = C \) and \( C \) is connected.
	Since the closure of a connected set is connected, \( \overline C \) is connected.
	But \( \overline C \supset C \), so \( C = \overline C \) is closed.
\end{proof}
