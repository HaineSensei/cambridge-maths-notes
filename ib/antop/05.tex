\subsection{More Metrics}
\begin{example}
	Let \( M \) be a set.
	Then
	\[
		d(x,y) = \begin{cases}
			0 & \text{if } x = y \\
			1 & \text{otherwise}
		\end{cases}
	\]
	is a metric, called the discrete metric on \( M \).
	In particular, \( (M, d) \) is called a discrete metric space.
\end{example}
\begin{example}
	Let \( G \) be a group generated by \( S \subset G \).
	We assume \( e \not\in S \) and \( x \in S \implies x^{-1} \in S \).
	Then
	\[
		d(x,y) = \min \qty{ n \geq 0 \colon \exists s_1, \dots, s_n, y = x s_1 \dots s_n }
	\]
	defines a metric called the word metric.
\end{example}
\begin{example}
	Let \( p \) be prime.
	Then
	\[
		d(x,y) = \begin{cases}
			0      & \text{if } x = y                                                        \\
			p^{-n} & \text{otherwise, where } x - y = p^n m, n \geq 0, m \in Z, p \not\mid m
		\end{cases}
	\]
	defines a metric on \( \mathbb Z \).
	This is known as the \( p \)-adic metric.
\end{example}

\subsection{Subspaces}
Let \( (M, d) \) be a metric space, and \( N \subset M \).
Then naturally we can restrict \( d \) to \( N \times N \), giving a metric on \( N \).
\( (N, d) \) is called a subspace of \( M \).
\begin{example}
	Consider \( \mathbb Q \) with the metric \( d(x,y) = \abs{x-y} \).
	This is clearly a subspace of \( \mathbb R \) (implicitly with the standard metric on \( \mathbb R \)).
\end{example}
\begin{example}
	Since every continuous function on a closed bounded interval is bounded, \( C[a,b] \) is a subset of \( \ell_\infty[a,b] \).
	Hence \( C[a,b] \) ith the uniform metric is a subspace of \( \ell_\infty[a,b] \).
\end{example}

\subsection{Product Spaces}
Let \( (M, d), (M', d') \) be metric spaces.
Then any of the following defines a metric on the Cartesian product \( M \times M' \).
\begin{enumerate}[(i)]
	\item \( d_1 ((x,x'), (y,y')) = d(x,y) + d(x',y') \)
	\item \( d_2 ((x,x'), (y,y')) = \qty(d(x,y)^2 + d(x',y')^2)^{\frac{1}{2}} \)
	\item \( d_\infty ((x,x'), (y,y')) = \max\qty{ d(x,y), d(x',y') } \)
\end{enumerate}
We commonly write \( (M \times M', p) \) as \( M \oplus_p M' \).
Note that we always have
\[
	d_\infty \leq d_2 \leq d_1 \leq 2 d_\infty
\]
We can generalise for \( n \in \mathbb N \) and metric spaces \( (M_k, \rho_k) \) for \( k \in \qty{1,\dots,n} \), by defining
\[
	\qty( \bigoplus_{k=1}^n M_k )_p = M_1 \oplus_p \dots \oplus_p M_n = \qty( M_1 \times \dots \times M_n, d_p )
\]
\begin{example}
	\( \mathbb R \oplus_1 \mathbb R = \ell_1^2 \).
	Further, \( \mathbb R \oplus_2 \mathbb R \oplus_2 R = \ell_2^3 \), and other analogous results hold.
\end{example}
\begin{remark}
	\( \mathbb R \oplus_1 \mathbb R \oplus_2 \mathbb R \) does not make sense since we have not defined the associativity of the \( \oplus \) operator.
	The two choices yield different metric spaces.
\end{remark}

\subsection{Convergence}
Let \( M \) be a metric space, and \( (x_n) \) a sequence in \( M \).
Given \( x \in M \), we say that \( (x_n) \) converges to \( x \) in \( M \) if
\[
	\forall \varepsilon > 0, \exists N \in \mathbb N, \forall n \geq N, d(x_n,x)<\varepsilon
\]
We say that \( (x_n) \) is convergent in \( M \) if \( \exists x \in M \) such that \( x_n \to x \).
Otherwise, we say that \( (x_n) \) is divergent.
Note that \( x_n \to x \) in \( M \) if and only if \( d(x_n, x) \to 0 \) in \( \mathbb R \).
\begin{lemma}
	Suppose we have a sequence \( x_n \to x \) and \( x_n \to y \) in a metric space \( M \).
	Then \( x = y \).
\end{lemma}
\begin{proof}
	Suppose \( x \neq y \).
	Then let \( \varepsilon = \frac{d(x,y)}{3} > 0 \).
	So, by the definition of convergence,
	\[
		\exists N_1 \in \mathbb N, \forall n \geq N_1, d(x_n, x) < \varepsilon;
	\]
	\[
		\exists N_2 \in \mathbb N, \forall n \geq N_2, d(x_n, y) < \varepsilon
	\]
	Now, fix \( N \in \mathbb N \) such that \( n \geq N_1, n \geq N_2 \), for instance \( N = \max\qty{N_1, N_2} \).
	Then
	\[
		d(x,y) \leq d(x, x_n) + d(x_n, y) < 2\varepsilon = \frac{2}{3} d(x,y)
	\]
	which is a contradiction.
\end{proof}
\begin{definition}
	Given a convergent sequence \( (x_n) \) in a metric space \( M \), we say the \textit{limit} of \( (x_n) \) is the unique \( x \in M \) such that \( x_n \to x \) as \( n \to \infty \).
	This is denoted
	\[
		\lim_{n \to \infty} x_n
	\]
\end{definition}
\begin{example}
	This definition has the usual meaning when \( M = \mathbb R, \mathbb C \).
\end{example}
\begin{example}
	The constant sequence defined by \( x_n = x \) converges to \( x \).
	In particular, `eventually constant' sequences converge; let \( (x_n) \) be a sequence in \( M \) such that \( \exists x \in M, \exists N \in \mathbb N, \forall n \geq N, x_n = x \), then \( x_n \to x \).
	It is not necessarily true that sequences only converge if they are eventually constant.
	However, in a discrete metric space, the converse is true, since we can choose \( \varepsilon \) smaller than all distances.
\end{example}
\begin{example}
	Consider the \( 3 \)-adic metric.
	Then, \( 3^n \to 0 \) as \( n \to \infty \) since \( d(3^n, 0) = 3^{-n} \to 0 \).
\end{example}
\begin{example}
	Let \( S \) be a set.
	Then, \( f_n \to f \) in \( \ell_\infty(S) \) in the uniform metric if and only if \( d(f_n, f) = \norm{f_n - f}_\infty = \sup_S \abs{f_n - f} \to 0 \), which is precisely the condition that \( f_n \to f \) uniformly on \( S \).
	Note, however, that \( f_n(x) = x + \frac{1}{n} \) for \( x \in \mathbb R, n \in \mathbb N \) and \( f(x) = x \), then certainly \( f_n \to x \) uniformly on \( \mathbb R \).
	However, \( f_n, f \not\in \ell_\infty(\mathbb R) \), so the uniform metric is not defined on these functions.
	So the notion of uniform convergence visited before is slightly more general than the idea of convergence in this metric space.
\end{example}
\begin{example}
	Consider Euclidean space \( M = \mathbb R^n, \mathbb C^n \) with the \( \ell_2 \) metric.
	Then, consider
	\[
		x^{(k)} = \qty(x^{(k)}_1, \dots, x^{(k)}_n) \in M
	\]
	for \( k \in \mathbb N \), and \( x = (x_1, \dots, x_n) \in M \).
	Then,
	\[
		\abs{x^{(k)}_i - x_i} \leq \norm{x^{(k)} - x}_2 \leq \sum_{i=1}^n \abs{x^{(k)}_i - x_i}
	\]
	So \( x^{(k)} \to x \) if and only if all \( i \) satisfy \( x^{(k)}_i \to x_i \).
	This can be thought of as convergence being equivalent to coordinate-wise (or pointwise) convergence.
\end{example}
\begin{example}
	Consider \( f_n(x) = x^n \) for \( x \in [0,1] \), and \( n \in \mathbb N \).
	Then \( (f_n) \) is a sequence in \( C[0,1] \), which converges pointwise but not uniformly.
	So \( (f_n) \) is not convergent in the uniform metric.
	However, using the \( L_1 \) metric, we have
	\[
		d_1(f_n, 0) = \norm{f_n}_1 = \int_0^1 f_n = \frac{1}{n+1} \to 0
	\]
	So, \( f_n \to 0 \) in \( (C[0,1], L_1) \).
\end{example}
\begin{example}
	Let \( N \) be a subspace of a metric space \( M \), and \( (x_n) \) be a convergent sequence in \( N \).
	Then \( (x_n) \) converges in \( M \).
	The converse is not necessarily true; consider \( M = \mathbb R \) and \( N = (0, \infty) \) with \( (x_n) = \frac{1}{n} \).
	This is divergent in \( N \) but convergent in \( M \).
\end{example}
\begin{example}
	Let \( (M, d), (M', d') \) be metric spaces.
	Let \( N = M \oplus_p M' \).
	Let \( a_n = (x_n, y_n) \in N \) for all \( n \in \mathbb N \), and \( a = (x, y) \in N \).
	Then
	\[
		a_n \to a \text{ in } N \iff x_n \to x \text{ in } M, y_n \to y \text{ in } M'
	\]
	Indeed,
	\[
		\max\qty{ d(x_n, x), d'(y_n, y) } = d_\infty(a_n, a) \leq d_p(a_n, a) \leq 2 d_1(a_n, a) = 2d(x_n, x) + 2d'(y_n, y)
	\]
\end{example}
