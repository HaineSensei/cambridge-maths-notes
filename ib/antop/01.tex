\subsection{Definition}
Recall that \( x_n \to x \) as \( n \to \infty \) (for \( x \in \mathbb R\) or \(\mathbb C\)) if
\[
	\forall \varepsilon > 0, \exists N \in \mathbb N, \forall n \geq N, \abs{x_n-x} < \varepsilon
\]
This is essentially considering the \( \varepsilon \)-neighbourhood of \( x \).
We aim to define the same notion of convergence for functions, by defining an analogous concept of an \( \varepsilon \)-neighbourhood.
In particular, each value on the domain should converge in its own \( \varepsilon \)-neighbourhood.
\begin{definition}
	Let \( S \) be a set, and \( f, f_n \colon S \to \mathbb R \), be functions.
	We say that \( (f_n) \) converges to \( f \) uniformly on \( S \) if
	\[
		\forall \varepsilon > 0, \exists N \in \mathbb N, \forall n \geq N, \forall x \in S, \abs{f_n(x) - f(x)} < \varepsilon
	\]
\end{definition}
\begin{note}
	\( N \) depends only on \( \varepsilon \), \textit{not} on any \( x \).
	Each \( x \) converges therefore at a `similar speed', hence the name `uniform convergence'.
\end{note}
\noindent Equivalently, we can write
\[
	\forall \varepsilon > 0, \exists N \in \mathbb N, \forall n \geq N, \sup_{x \in S} \abs{f_n(x) - f(x)} < \varepsilon
\]
The supremum condition is equivalent overall because the inequality on the right is weakened to a possible equality, but we can always decrease \( \varepsilon \) to retain the inequality.
Alternatively, we could write
\[
	\lim_{n \to \infty} \sup_{x \in S} \abs{f_n - f} = 0
\]
For each \( x \in S \), \( (f_n(x))_{n=1}^\infty \to f(x) \).
Hence, \( f \) is unique given \( (f_n) \), since limits are unique.
We call \( f \) the \textit{uniform limit} of \( (f_n) \) on \( S \).

\subsection{Pointwise convergence}
\begin{definition}
	\( (f_n) \) converges \textit{pointwise} to \( f \) on \( S \) if \( (f_n(x))_{n=1}^\infty \) converges to \( f(x) \) for every \( x \in S \).
	In other words,
	\[
		\underbrace{\forall x \in S}_{\mathclap{\text{order rearranged}}}, \forall \varepsilon > 0, \exists N \in \mathbb N, \forall n \geq N, \abs{f_n(x) - f(x)} < \varepsilon
	\]
	Now, \( N \) depends both on \( \varepsilon \) and on \( x \).
	Note that the pointwise limit of \( (f_n) \) on \( S \) is also unique since limits are unique.
\end{definition}
\begin{remark}
	Uniform convergence implies pointwise convergence, and the uniform limit is the pointwise limit.
\end{remark}

\begin{example}
	Let \( f_n(x) = x^2 e^{-nx} \) on \( [0, \infty), n \in \mathbb N \).
	Does \( (f_n) \) converge uniformly on the domain?
	First let us check pointwise convergence.
	We have \( x^2 e^{-nx} \to 0 \) hence pointwise convergence to \( f(x) = 0 \) is satisfied.
	Now, we need only check uniform convergence to the function \( f(x) = 0 \).
	\[
		\sup_{x \in [0, \infty)} \abs{f_n(x) - 0} = \sup_{x \in [0, \infty)} f_n(x)
	\]
	We could differentiate \( f_n \) and find the maximum if it exists, but we might not find the maximum if it is (for example) on the endpoints.
	A much better method is to find an upper bound on \( \abs{f_n(x)-f(x)} \) (which, in this example, is \( f_n(x) \)) that does not depend on \( x \).
	In this case, we can expand \( e^{nx} \) on the denominator and isolate a single term to get
	\[
		x^2 e^{-nx} = \frac{x^2}{e^{nx}} \leq \frac{2}{n^2};\quad \forall x
	\]
	Hence,
	\[
		\sup_{x \in [0, \infty)} \abs{f_n(x) - 0} \to 0
	\]
	and uniform convergence is satisfied.
\end{example}
\begin{example}
	Consider \( f_n(x) = x^n \) on \( [0,1], n \in \mathbb N \).
	A pointwise limit is reached by
	\[
		f(x) = \begin{cases}
			1 & x = 1            \\
			0 & \text{otherwise}
		\end{cases}
	\]
	Consider \( \sup \abs{f_n(x) - f(x)} \) excluding 1 (since at 1 the supremum is zero).
	Note \( f_n(x) \to 1 \) as \( x \to 1 \) from below, for all \( n \).
	Hence the supremum is always 1 by choosing an \( x \) sufficiently close to 1.
	So \( f_n \not\to f \) uniformly on \( [0,1] \), hence \( (f_n) \) does not converge at all uniformly on this domain.
	Or,
	\[
		\sup f_n(x) \geq f_n\qty(\qty(\frac{1}{2})^{1/n}) = \frac{1}{2}
	\]
\end{example}

\begin{remark}
	If \(f_n \not\to f\) uniformly on S,
	\[
		\exists \varepsilon > 0, \forall N \in \mathbb N, \exists n \geq N, \exists x \in S, \abs{f_n(x) - f(x)} \geq \varepsilon
	\]
	In the above example, we proved something stronger:
	\[
		\forall n, \exists x \in S, f_n(x) \geq \frac{1}{2}
	\]
	We could have alternatively stated, for example, \( f_n(x) \) is continuous so there exists some subset of \( [0, 1] \) greater than \( \frac{1}{2} \) always.
\end{remark}

\begin{theorem}
	Let \( S \subseteq \mathbb R, \mathbb C \).
	Let \( (f_n), f \colon S \to \mathbb R \text{(or } \mathbb{C} \text{)} \), where \( f_n \) is continuous and \( (f_n) \to f \) uniformly on \( S \).
	Then \( f \) is continuous.
\end{theorem}
\noindent Informally, the uniform limit of continuous functions is continuous.
\begin{proof}
	Fix some point \( a \in S \), \( \varepsilon > 0 \).
	We seek \( \delta > 0 \) such that \( \forall x \in S, \abs{x - a} < \delta \implies \abs{f(x) - f(a)} < \varepsilon \).
	We fix an \( n \in \mathbb N \) such that \( \forall x \in S, \abs{f_n(x) - f(x)} < \varepsilon \).
	Since \( f_n \) is continuous, there exists \( \delta > 0 \) such that \( \forall x \in S, \abs{x - a} < \delta \implies \abs{f_n(x) - f_n(a)} < \varepsilon \).
	So, \( \forall x \in S \),
	\[
		\abs{x - a} < d  \implies \abs{f(x) - f(a)} \leq \abs{f(x) - f_n(x)} + \abs{f_n(x) - f_n(a)} + \abs{f_n(a) - f(a)} < 3\varepsilon
	\]
\end{proof}
\begin{remark}
	The above proof is often called a \( 3\varepsilon \)-proof.
	Note, the proof is not true for pointwise convergence; if \( f_n \to f \) pointwise and \( f_n \) continuous, \( f \) is not necessarily continuous.
	Further, it is not true for differentiability; \( f_n \) differentiable does not imply \( f \) differentiable (see example sheet).
	Another way to interpret the result of the above theorem is to swap limits:
	\[
		\lim_{x \to a} \lim_{n \to \infty} f_n(x) = \lim_{x \to a} f(x) = f(a) = \lim_{n \to \infty} f_n(a) = \lim_{n \to \infty} \lim_{x \to a} f_n(x)
	\]
\end{remark}
