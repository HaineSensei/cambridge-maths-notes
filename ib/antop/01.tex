\subsection{Uniform Convergence}
Recall that \( x_n \to x \) as \( n \to \infty \) (for \( x \in \mathbb R\) or \(\mathbb C\)) if
\[ \forall \varepsilon > 0, \exists N in \mathbb N, \forall n \geq N, \abs{x_n-x} < \varepsilon \]
This is essentially considering the \( \varepsilon \)-neighbourhood of \( x \).
We aim to define the same notion of convergence for functions, by defining an analogous concept of an \( \varepsilon \)-neighbourhood.
In particular, each value on the domain should converge in its own \( \varepsilon \)-neighbourhood.
\begin{definition}
Let \( S \) be a set, and \( f, f_n \colon S \to \mathbb R \), be functions.
We say that \( (f_n) \) converges to \( f \) uniformly on \( S \) if
\[ \forall \varepsilon > 0, \exists N \in \mathbb N, \forall n \geq N, \forall x \in S, \abs{f_n(x) - f(x)} < \varepsilon \]
\begin{note}
\( N \) depends only on \( \varepsilon \), FIT(not) on any \( x \).
Each \( x \) converges therefore at a `similar speed', hence the name `uniform convergence'.
\end{note}
Equivalently, we can write
\[ \forall \varepsilon > 0, \exists N \in \mathbb N, \forall n \geq N, \sup_{x \in S} \abs{f_n(x) - f(x)} < \varepsilon \]
The supremum condition is equivalent overall because the inequality on the right is weakened to a possible equality, but we can always decrease \( \varepsilon \) to retain the inequality.
alternatively \sup_{x \in S} \abs{f_n - f} \to 0
For each \( x \in S \), \( (f_n(x))_{n=1}^\infty \to f(x) \).
Hence, \( f \) is unique given \( (f_n) \), since limits are unique.
We call \( f \) the FIT(uniform limit) of \( (f_n) \) on \( S \).

\subsection{Pointwise Convergence}
\begin{definition}
\( (f_n) \) converges \textit{pointwise} to \( f \) on \( S \) if \( (f_n(x))_{n=1}^\infty \) converges to \( f(x) \) for every \( x \in S \).
In other words,
\[ \underbrace{\forall x \in S}_{\mathclap{\text{order rearranged}}}, \forall \varepsilon > 0, \exists N \in \mathbb N, \forall n \geq N, \abs{f_n(x) - f(x)} < \varepsilon \]
Now, \( N \) depends both on \( \varepsilon \) and on \( x \).
Note that the pointwise limit of \( (f_n) \) on \( S \) is also unique since limits are unique.
\end{definition}
Now, uniform convergence implies pointwise convergence, and the uniform limit is the pointwise limit.

\begin{example}
Let \( f_n(x) = x^2 e^{-nx} \) on \( [0, \infty), n \in \mathbb N \).
Does \( (f_n) \) converge uniformly on the domain?
First let us check pointwise convergence.
We have \( x^2 e^{-nx} \to 0 \) hence pointwise convergence to \( f(x) = 0 \) is satisfied.
Now, we need only check uniform convergence to the function \( f(x) = 0 \).
\[ \sup_{x \in [0, \infty)} \abs{f_n(x) - 0} = \sup_{x \in [0, \infty)} f_n(x) \]
We could differentiate \( f_n \) and find the maximum if it exists, but we might not find the maximum if it is (for example) on the endpoints.
A much better method is to find an upper bound on \( \abs{f_n(x)-f(x)} \) (which, in this example, is \( f_n(x) \)) that does not depend on \( x \).
In this case, we can expand \( e^{nx} \) on the denominator and isolate a single term to get
\[ x^2 e^{-nx} = \frac{x^2}{e^{nx}} \leq \frac{2}{n^2} \forall x \]
Hence,
\[ \sup_{x \in [0, \infty)} \abs{f_n(x) - 0} \to 0 \]
and uniform convergence is satisfied.
\end{example}
\begin{example}
Consider \( f_n(x) = x^n \) on \( [0,1], n \in \mathbb N \).
A pointwise limit is reached by
\[ f(x) = cases x=1 -> 1, otherwise -> 0 \]
Consider \( \sup \abs{f_n(x) - f(x)} \) excluding 1 (since at 1 the supremum is zero).
Note \( f_n(x) \to 1 \) as \( x \to 1 \) from below, for all \( n \).
Hence the supremum is always 1 by choosing an \( x \) sufficiently close to 1.
So \( f_n \not\to f \) uniformly on \( [0,1] \), hence \( (f_n) \) does not converge at all uniformly on this domain.
Or,
\[ \sup f_n(x) \geq f_n((1/2)^{1/n}) = 1/2 \]
\end{example}

\remark{
If f_n \not\to f uniformly on S,
\[ \exists \varepsilon > 0, \forall N \in \mathbb N, \exists n \geq N, \exists x \in S, abs{f_n(x) - f(x)} \geq \varepsilon \]
In the above example, we proved something stronger:
\[ \forall n, \exists x \in S, f_n(x) \geq 1/2 \]
We could have alternatively stated, for example, \( f_n(x) \) is continuous so there exists some subset of \( [0, 1] \) greater than \( 1/2 \) always.
}

\begin{theorem}
Let \( S \subseteq \mathbb R, \mathbb C \).
Let \( (f_n), f \colon S \to \mathbb R \text{(or } \mathbb C\text{)} \), where \( f_n \) is continuous and \( (f_n) \to f \) uniformly on \( S \).
Then \( f \) is continuous.
\end{theorem}
Informally, the uniform limit of continuous functions is continuous.
\begin{proof}

\end{proof}
