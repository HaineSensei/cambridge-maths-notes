\subsection{Motivation and definition}
Recall from IA Analysis that a continuous function on a closed bounded interval is bounded and attains its bounds.
We wish to generalise this result to more general topological spaces.
\begin{example}
	\begin{enumerate}[(i)]
		\item If \( X \) is finite, any function \( X \to \mathbb R \) is finite.
		\item If, for all continuous functions \( f \colon X \to \mathbb R \) there exists \( n \in \mathbb N \) and subsets \( A_1, \dots, A_n \) of \( X \) such that \( X = \bigcup_{j=1}^n A_j \) and \( f \) is bounded on \( A_j \) for all \( j \), then the property holds.
		\item Note that continuous functions are `locally bounded'; if \( f \colon X \to \mathbb R \) is continuous, then for all \( x \in X \) we have \( U_x = f^{-1}((f(x)-1, f(x)+1)) \) is an open set containing \( X \), and \( f \) is bounded on \( U_x \).
		      So each point has an open neighbourhood on which \( f \) is bounded.
		      Further, \( X = \bigcup_{x \in X} U_x \).
		      If there exists a finite subset \( F \subset X \) such that \( \bigcup_{x \in F} U_x = X \), then \( f \) is bounded on \( X \).
		      This is exactly the definition we will use for compactness.
	\end{enumerate}
\end{example}
\begin{definition}
	Let \( X \) be a topological space.
	An \textit{open cover} for \( X \) is a family \( \mathcal U \) of open subsets of \( X \) that cover \( X \); that is, \( \bigcup_{U \in \mathcal U} U = X \).
	A \textit{subcover} of \( \mathcal U \) is a subset \( \mathcal V \subset \mathcal U \) that covers \( U \).
	This is called a \textit{finite subcover} if \( \mathcal V \) is finite.
	We say that \( X \) is \textit{compact} if every open cover has a finite subcover.
\end{definition}
\begin{remark}
	Compactness can be thought of as the next best thing to finiteness.
\end{remark}
\begin{theorem}
	Let \( X \) be a compact topological space and \( f \colon X \to \mathbb R \) be continuous.
	Then \( f \) is bounded, and if \( X \) is not empty \( f \) attains its bounds.
\end{theorem}
\begin{proof}
	For \( n \in \mathbb N \), let \( U_n = \qty{ x \in X \colon \abs{f(x)} < n} \).
	\( U_n \) is open since \( x \mapsto \abs{f(x)} \) is continuous and \( (-n,n) \) is open.
	It is clear that \( X = \bigcup_{n \in \mathbb N} U_n \).
	This is an open cover of \( X \).
	Hence there exists a finite subcover \( F \subset \mathbb N \) such that \( X = \bigcup_{n \in F} U_n = U_N \) where \( N = \max F \).
	Hence, for all \( x \in X \), we have \( \abs{f(x)} < N \) so \( f \) is bounded.

	Let \( \alpha = \inf_X f \); this exists since \( f \) is bounded.
	Suppose there exists no \( x \in X \) such that \( f(x) = \alpha \).
	Then, for all \( x \in X \), \( f(x) > \alpha \).
	Then there exists \( n \in \mathbb N \) such that \( f(x) > \alpha + \frac{1}{n} \).
	So let
	\[
		V_n = \qty{ x \in X \colon f(x) > \alpha + \frac{1}{n} } = f^{-1}\qty(\qty(\alpha + \frac{1}{n}, \infty))
	\]
	We can see that \( V_n \) is open.
	Now, since \( \bigcup_{n \in \mathbb N} V_n = X \), there exists a finite subcover \( F \subset \mathbb N \) such that \( \bigcup_{n \in F} V_n = X = V_N \) where \( N \) is the maximal \( F \).
	Then for all \( x \in X \), we have \( f(x) > \alpha + \frac{1}{N} \).
	Hence \( \inf_X f \geq \alpha + \frac{1}{N} \), which is a contradiction.
	The same argument applies for the supremum.
\end{proof}
\begin{lemma}
	Let \( Y \) be a subspace of a topological space \( X \).
	Then \( Y \) is compact if and only if whenever \( \mathcal U \) is a family of open sets in \( X \) such that \( \bigcup_{U \in \mathcal U} \supset Y \), there is a finite subfamily \( \mathcal V \subset \mathcal U \) with \( \bigcup_{U \in \mathcal V} U \supset Y \).
\end{lemma}
\begin{theorem}
	\( [0,1] \) is compact.
\end{theorem}
\begin{proof}
	Let \( \mathcal U \) be a family of open sets in \( \mathbb R \) that cover \( [0,1] \).
	For a subset \( A \subset [0,1] \), we say that \( \mathcal U \) \textit{finitely covers} \( A \) if there exists a finite subcover \( \mathcal V \subset \mathcal U \) of \( A \).
	Note that if \( A = B \cup C \) and \( A,B,C \subset [0,1] \) and \( \mathcal U \) finitely covers \( B \) and \( C \), we can take the union of the finite subcovers to find a finite subcover of \( A \), so \( U \) finitely covers \( A \).
	Suppose that \( \mathcal U \) does not finitely cover \( [0,1] \).
	Then one of the intervals \( \qty[0, \frac{1}{2}] \) and \( \qty[\frac{1}{2}, 1] \) is not finitely coverable by \( \mathcal U \).
	Let this interval be \( [a_1, b_1] \).
	Let \( c = \frac{1}{2} (a_1 + b_1) \).
	Then one of the intervals \( [a_1, c], [c,b_1] \) is not finitely coverable by \( \mathcal U \).
	Inductively, we obtain a nested sequence of intervals \( \qty[a_1, b_1] \supset \dots \supset \qty[a_n, b_n] \supset \cdots \) which are not finitely covered by \( \mathcal U \) and \( b_n - a_n = 2^{-n} \).
	Now, \( a_n \to x \) for some \( x \in [0,1] \) and \( b_n = a^n + 2^{-n} \to x \).
	But since \( \mathcal U \) covers \( [0,1] \), there exists \( U \in \mathcal U \) such that \( x \in U \).
	\( U \) is open in \( \mathbb R \), so for all \( \varepsilon > 0 \), we have \( (x-\varepsilon, x+\varepsilon) \subset U \).
	Since \( a_n, b_n \to x \), we can choose \( n \) such that \( a_n, b_n \in (x-\varepsilon, x+\varepsilon) \).
	This is covered by one open set \( U \) in \( \mathcal U \), so this is a finite subcover.
	This is a contradiction.
\end{proof}
\begin{example}
	Other examples of compact spaces include the following.
	\begin{enumerate}[(i)]
		\item Any finite set is compact.
		\item On any set \( X \), the cofinite topology is compact.
		      Suppose without loss of generality that \( X \) is not empty, and let \( \mathcal U \) be an open cover for \( X \).
		      Let \( U \in \mathcal U \) such that \( U \neq \varnothing \).
		      Then \( F = X \setminus U \) is finite.
		      For all \( x \in F \), let \( U_x \in \mathcal U \) such that \( x \in U_x \).
		      Then \( \bigcup_{x \in F} U_x \cup U \) is a finite subcover.
		\item Let \( x_n \to x \) in a topological space \( X \).
		      Let \( Y = \qty{x_n \colon n \in \mathbb N} \cup \qty{x} \).
		      Then \( Y \) is compact.
		      Indeed, let \( \mathcal U \) be a family of open sets in \( X \) such that \( \bigcup_{U \in \mathcal U} U \supset Y \).
		      In particular, let \( U \in \mathcal U \) such that \( x \in U \).
		      Since \( U \) is open and \( x_n \to x \), there exists \( N \in \mathbb N \) such that for all \( n \geq N \) we have \( x_n \in U \).
		      So we can cover the remaining finitely many elements analogously to the previous example, and this yields a finite subcover.
		\item The indiscrete topology on any set is compact, since there are only two open sets.
	\end{enumerate}
	Counterexamples include the following.
	\begin{enumerate}[(i)]
		\item An infinite set \( X \) in the discrete topology is not compact.
		      Let
		      \[
			      \mathcal U = \qty{\qty{x} \colon x \in X}
		      \]
		      This has no finite subcover.
		\item \( \mathbb R \) is not compact.
		      Consider the intervals \( (-n, n) \) for all \( n \in \mathbb N \).
		      This is an open cover with no finite subcover.
	\end{enumerate}
\end{example}

\subsection{Subspaces}
\begin{theorem}
	Let \( Y \) be a subspace of a topological space \( X \).
	Then,
	\begin{enumerate}[(i)]
		\item Let \( X \) be compact and \( Y \) be closed in \( X \).
		      Then \( Y \) is compact.
		\item Let \( X \) be Hausdorff and \( Y \) be compact.
		      Then \( Y \) is closed in \( X \).
	\end{enumerate}
\end{theorem}
\begin{proof}
	Let \( \mathcal U \) be a family of open sets in \( X \) such that their union covers \( Y \).
	Then \( \mathcal U \cup (X \setminus Y) \) is an open cover for \( X \) since \( Y \) is closed.
	This has a finite subcover \( \mathcal V \subset \mathcal U \) such that \( \bigcup_{U \in \mathcal V} U \cup (X \setminus Y) = X \).
	Then \( \bigcup_{U \in \mathcal V} U \supset Y \).

	For part (ii), let \( x \in X \setminus Y \).
	For \( y \in Y \), since \( x \neq y \) there exist open sets \( U_y, V_y \) in \( X \) such that \( x \in U_y, y \in V_y, U_y \cap V_y = \varnothing \).
	Now, \( \qty{V_y \colon y \in Y} \) is an open cover of \( Y \).
	Hence there exists \( F \subset Y \) finite such that \( \bigcup_{y \in F} V_y \supset Y \).
	Now, \( U = \bigcap_{y \in F} U_y \) is open, further \( x \in U \) and
	\[
		U \cap Y \subset \qty(\bigcap_{y \in F} U_y) \cap \qty(\bigcup_{y \in F} V_y) = \varnothing
	\]
	Hence \( X \setminus Y \) is a neighbourhood of all of its points, so it is open and \( Y \) is closed.
\end{proof}

\subsection{Continuous images of compact spaces}
\begin{theorem}
	Let \( f \colon X \to Y \) be a continuous function between topological spaces such that \( X \) is compact.
	Then \( f(X) \) is compact.
\end{theorem}
\begin{proof}
	Let \( \mathcal U \) be a family of open sets in \( Y \) such that \( \bigcup_{U \in \mathcal U} U \supset f(X) \).
	Then \( \bigcup_{U \in \mathcal U} f^{-1}(U) = X \) and \( f^{-1}(U) \) is open in \( X \) for all \( U \in \mathcal U \) since \( f \) is continuous.
	Since \( X \) is compact, we have a finite subcover \( \mathcal V \subset \mathcal U \) such that \( X = \bigcup_{U \in \mathcal V} f^{-1}(V) \).
	Hence \( f(X) \subset \bigcup_{U \in \mathcal V} U \).
\end{proof}
\begin{remark}
	Compactness is a topological property.
	If \( f \colon X \to Y \) is continuous and \( A \subset X \) is compact, then \( f(A) \) is compact.
\end{remark}
\begin{corollary}
	Any quotient of a compact space is compact.
\end{corollary}
\begin{example}
	Let \( a < b \in \mathbb R \).
	Then \( [a,b] \simeq [0,1] \) so is compact.
\end{example}

\subsection{Topological inverse function theorem}
\begin{theorem}
	Let \( f \colon X \to Y \) be a continuous bijection from a compact space \( X \) to a Hausdorff space \( Y \).
	Then \( f^{-1} \) is continuous, so \( f \) is an open map.
	Hence \( f \) is a homeomorphism.
\end{theorem}
\begin{proof}
	Let \( U \) be an open subset of \( X \).
	Then \( K = X \setminus U \) is closed.
	Since \( X \) is compact, \( K \) is compact.
	Further, \( f(K) \) is compact.
	Hence \( f(K) \) is closed in \( Y \).
	So \( f(U) = Y \setminus f(K) \) is open in \( Y \).
\end{proof}
\begin{example}
	\( \mathbb R / \mathbb Z \) is homeomorphic to \( S^1 = \qty{ x \in \mathbb R^2 \colon \norm{x} = 1 } \).
	Indeed, let \( f \colon \mathbb R \to S_1 \) by \( f(t) = (\cos(2\pi t), \sin(2 \pi t)) \).
	For all \( s,t \), we have \( f(s) = f(t) \) if and only if \( s \sim t \) so \( f \) fully respects \( \sim \).
	\( f \) is continuous and surjective.
	Let \( \widetilde f \colon \mathbb R / \mathbb Z \to S^1 \) be the unique map such that \( \widetilde f \circ q = f \).
	So \( \widetilde f \) is a continuous bijection.
	\( S^1 \) is Hausdorff, and \( \mathbb R / \mathbb Z \) is the image of \( [0,1] \) under a continuous map, hence is compact.
	Hence \( \widetilde f \) is a homeomorphism.
\end{example}

\subsection{Tychonov's theorem}
\begin{theorem}
	Let \( X, Y \) be compact topological spaces.
	Then \( X \times Y \) is compact in the product topology.
\end{theorem}
\begin{proof}
	Let \( \mathcal U \) be an open cover for \( X \times Y \).
	We want to show that there exists a finite subcover.
	Without loss of generality, every member of \( \mathcal U \) can be of the form \( U \times V \) where \( U \) is open in \( X \) and \( V \) is open in \( Y \).
\end{proof}
