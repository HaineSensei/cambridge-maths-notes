\subsection{Motivation and definition}
Recall from IA Analysis that a continuous function on a closed bounded interval is bounded and attains its bounds.
We wish to generalise this result to more general topological spaces.
\begin{example}
	\begin{enumerate}[(i)]
		\item If \( X \) is finite, any function \( X \to \mathbb R \) is finite.
		\item If, for all continuous functions \( f \colon X \to \mathbb R \) there exists \( n \in \mathbb N \) and subsets \( A_1, \dots, A_n \) of \( X \) such that \( X = \bigcup_{j=1}^n A_j \) and \( f \) is bounded on \( A_j \) for all \( j \), then the property holds.
		\item Note that continuous functions are `locally bounded'; if \( f \colon X \to \mathbb R \) is continuous, then for all \( x \in X \) we have \( U_x = f^{-1}((f(x)-1, f(x)+1)) \) is an open set containing \( X \), and \( f \) is bounded on \( U_x \).
		      So each point has an open neighbourhood on which \( f \) is bounded.
		      Further, \( X = \bigcup_{x \in X} U_x \).
		      If there exists a finite subset \( F \subset X \) such that \( \bigcup_{x \in F} U_x = X \), then \( f \) is bounded on \( X \).
		      This is exactly the definition we will use for compactness.
	\end{enumerate}
\end{example}
\begin{definition}
	Let \( X \) be a topological space.
	An \textit{open cover} for \( X \) is a family \( \mathcal U \) of open subsets of \( X \) that cover \( X \); that is, \( \bigcup_{U \in \mathcal U} U = X \).
	A \textit{subcover} of \( \mathcal U \) is a subset \( \mathcal V \subset \mathcal U \) that covers \( U \).
	This is called a \textit{finite subcover} if \( \mathcal V \) is finite.
	We say that \( X \) is \textit{compact} if every open cover has a finite subcover.
\end{definition}
