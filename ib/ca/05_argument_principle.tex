\subsection{The argument principle}
\begin{proposition}
	If \( f \) has a zero (or pole) of order \( k \geq 1 \) at \( z = a \), then \( \frac{f'}{f} \) has a simple pole at \( z = a \) with residue \( k \) (or \( -k \), respectively).
\end{proposition}
\begin{proof}
	If \( z = a \) is a zero of order \( k \), there is a disc \( D(a,r) \) such that \( f(z) = (z-a)^k g(z) \) for \( z \in D(a,r) \) where \( g \colon D(a,r) \to \mathbb C \) is holomorphic with \( g(z) \neq 0 \) for all \( z \in D(a,r) \).
	Hence,
	\[
		f'(z) = k(z-a)^{k-1} g(z) + (z-a)^k g'(z)
	\]
	and
	\[
		\frac{f'(z)}{f(z)} = \frac{k}{z-a} + \frac{g'(z)}{g(z)}
	\]
	for all \( z \in D(a,r) \setminus \qty{a} \).
	Since \( \frac{g'}{g} \) is holomorphic in \( D(a,R) \), the claim follows.
	A similar argument holds for poles.
\end{proof}
\begin{definition}
	The order of a zero or pole \( a \) of a holomorphic function \( f \) is denoted \( \ord_f(a) \).
\end{definition}
\begin{theorem}[the argument principle]
	Let \( f \) be a meromorphic function on a domain \( U \) with finitely many zeroes \( a_1, \dots, a_k \) and finitely many poles \( b_1, \dots, b_\ell \).
	If \( \gamma \) is a closed curve in \( U \) homologous to zero in \( U \), and if \( a_i, b_j \not\in \Im \gamma \) for all \( i,j \), then
	\[
		\frac{1}{2\pi i} \int_\gamma \frac{f'(z)}{f(z)} \dd{z} = \sum_{i=1}^k I(\gamma;a_i) \ord_f(a_i) - \sum_{j=1}^\ell I(\gamma;b_j) \ord_f(b_j)
	\]
\end{theorem}
\begin{proof}
	Apply the residue theorem to \( g = \frac{f'}{f} \).
	If \( z_0 \in U \) is not a pole of \( f \), then \( f \) and hence \( f' \) are holomorphic near \( z_0 \).
	If additionally \( z_0 \) is not a zero of \( f \), \( g \) is holomorphic near \( z_0 \).
	So the set of singularities of \( g \) is precisely \( \qty{a_1, \dots, a_k} \cup \qty{b_1, \dots, b_\ell} \).
	By the previous proposition, their residues are known, and the result follows.
\end{proof}
\begin{remark}
	Let \( f, \gamma \) be as in the theorem, and let \( \Gamma(t) = f(\gamma(t)) \).
	Then \( \Gamma(t) \) is a closed curve with image \( \Im \Gamma \subset \mathbb C \setminus \qty{0} \), since no zeroes or poles of \( f \) are in \( \Im \gamma \).
	Moreover, if \( [a,b] \) is the domain of \( \gamma \), we have
	\[
		I(\gamma;0) = \frac{1}{2\pi i} \int_\Gamma \frac{\dd{z}}{z} = \frac{1}{2\pi i} \int_a^b \frac{\Gamma'(t)}{\Gamma(t)} \dd{t} = \frac{1}{2\pi i} \int_a^b \frac{f'(\gamma(t)) \gamma'(t)}{f(\gamma(t))} \dd{t} = \frac{1}{2 \pi i} \int_\gamma \frac{f'(z)}{f(z)} \dd{z}
	\]
	Thus, \( \frac{1}{2 \pi i} \int_\gamma \frac{f'(z)}{f(z)} \) is the number of times the image curve \( f \circ \gamma \) winds around zero as we move along \( \gamma \).
\end{remark}
\begin{definition}
	Let \( \Omega \) be a domain, and let \( \gamma \) be a closed curve in \( \mathbb C \).
	We say that \( \gamma \) \textit{bounds} \( \Omega \) if \( I(\gamma;w) = 1 \) for all \( w \in \Omega \), and \( I(\gamma;w) = 0 \) for all \( w \in \mathbb C \setminus \qty(\Omega \cup \Im \gamma) \).
\end{definition}
\begin{example}
	\( \partial D(0,1) \) bounds \( D(0,1) \), but does not bound \( D(0,1) \setminus \qty{0} \).
\end{example}
\begin{remark}
	If \( \gamma \) bounds \( \Omega \), then
	\begin{enumerate}
		\item \( \Omega \) is bounded.
		      Indeed, let \( D(a,R) \) such that \( \Im \gamma \subseteq D(a,R) \).
		      Then \( I(\gamma;w) = 0 \) for \( w \in \mathbb C \setminus D(a,R) \).
		      Since \( I(\gamma;w) = 1 \) for all \( w \in \Omega \), we must have \( \Omega \subset D(a,R) \).
		\item the topological boundary \( \partial \Omega \) is contained within \( \Im \gamma \), but it need not be the case that \( \partial \Omega = \Im \gamma \).
	\end{enumerate}
	There is a large class of closed curves that bound domains, namely, \textit{simple closed curves}, which are curves \( \gamma \colon [a,b] \to \mathbb C \) with \( \gamma(a) = \gamma(b) \), and such that \( \gamma(t_1) = \gamma(t_2) \) implies \( t_1 = t_2 \) or \( t_1, t_2 \in \qty{a,b} \).
	That a simple closed curve bounds a domain is a highly non-trivial fact guaranteed by the Jordan curve theorem: if \( \gamma \) is a simple closed curve, then \( \mathbb C \setminus \Im \gamma \) consists precisely of two connected components, one of which is bounded and the other unbounded, and moreover, \( \gamma \) (or \( -\gamma \)) bounds the bounded component, and \( \Im \gamma \) is the boundary of each of the two components.
	Thus, if \( \Omega_1 \) is the bounded component and \( \Omega_2 \) is the unbounded component, then after possibly changing the orientation of \( \gamma \), we have \( I(\gamma;w) = 1 \) for \( w \in \Omega_1 \), and \( I(\gamma;w) = 0 \) for \( w \in \Omega_2 \).
	This last assertion is simply that for any disc \( D(a,R) \supset \Im \gamma \), we have \( I(\gamma;w) = 0 \) for all \( w \in \mathbb C \setminus D(a,R) \).
\end{remark}
For a domain bounded by a closed curve, the argument principle gives the following.
\begin{corollary}
	Let \( \gamma \) be a closed curve bounding a domain \( \Omega \), and let \( f \) be meromorphic in an open set \( U \) with \( \Omega \cup \Im \gamma \subseteq U \).
	Suppose that \( f \) has no zeroes or poles on \( \Im \gamma \).
	Then \( f \) has finitely many zeroes and finitely many poles in \( \Omega \).

	Let the number of zeroes in \( \Omega \) be \( N \), and the number of poles in \( \Omega \) be \( P \), both counted with multiplicity.
	Then in addition we have that
	\[
		N - P = \frac{1}{2 \pi i} \int_\gamma \frac{f'(z)}{f(z)} \dd{z} = I(\Gamma;0)
	\]
	where \( \Gamma = f \circ \gamma \).
\end{corollary}
\begin{proof}
	Since \( f \) is meromorphic in \( U \), its singularities form a discrete set \( S \subset U \) consisting of poles or removable singularities.
	Since \( \gamma \) bounds \( \Omega \), we have that \( \Omega \) is bounded and hence \( \overline \Omega \) is compact.
	Also, \( \overline \Omega \subseteq \Omega \cup \Im \gamma \subseteq U \).
	If \( \overline \Omega \cap S \) is infinite, then by compactness of \( \overline \Omega \), there exists a point \( w \in \overline \Omega \) and distinct points \( w_j \in \overline \Omega \cap S \) such that \( w_j \to w \).
	If \( w \not\in S \), then \( f \) is defined and holomorphic near \( w \) which is impossible since \( w_j \in S \) and \( w_j \to w \).
	So \( w \in S \), but this is impossible since \( S \) is a discrete set.
	So \( \overline \Omega \cap S \) is finite, and in particular \( P \) is finite.

	If \( f \) has infinitely many zeroes in \( \Omega \), then by compactness there exists \( z \in \overline \Omega \subset U \) and distinct zeroes \( z_j \in \Omega \) such that \( z_j \to z \).
	Then either \( z \in U \setminus S \), or (if \( z \in S \)) \( z \) is a removable singularity, since otherwise \( z \) would be a pole and hence \( \abs{f(\zeta)} \to \infty \) as \( \zeta \to z \) which is impossible since \( z_j \to z \) and \( f(z_j) = 0 \).
	In either case, by the principle of isolated zeroes, \( f \) must be identically zero in \( D(z,\rho) \setminus \qty{z} \) for some \( \rho > 0 \).
	Since \( f \) is holomorphic in \( \Omega \setminus S \) which is connected (since \( \Omega \cap S \) is finite and \( \Omega \)) is connected, it follows from the unique continuation principle that \( f \equiv 0 \) in \( \Omega \).
	This is impossible since \( f \) has no zeroes in \( \Im \gamma \), so \( N \) must be finite.

	By the definition of \( \gamma \) bounding \( \Omega \), we have that \( I(\gamma;w) = 1 \) for all \( w \in \Omega \), and \( I(\gamma;w) = 0 \) for all \( w \in \mathbb C \setminus \qty(\Omega \cup \Im \gamma) \).
	In particular, \( \gamma \) is homologous to zero in \( U \).
	The final conclusion then follows from the fact that \( \Gamma \) is a closed curve in \( \mathbb C \setminus \qty{0} \) and \( I(\gamma;0) = \frac{1}{2\pi i} \int_\gamma \frac{f'(z)}{f(z)} \dd{z} \) as proven above.
\end{proof}

\subsection{Local degree theorem}
\begin{definition}
	Let \( f \) be a holomorphic function on a disc \( D(a,R) \) that is not constant.
	Then the \textit{local degree of \( f \) at \( a \)}, denoted \( \deg_f(a) \), is the order of the zero of \( f(z) - f(a) \) at \( z = a \).
	This is a finite positive integer.
\end{definition}
\begin{example}
	If \( f(z) = (z-1)^4 + 1 \) has \( \deg_f(1) = 4 \).
\end{example}
\begin{theorem}
	Let \( f \colon D(a,R) \to \mathbb C \) be holomorphic and non-constant, with \( \deg_f(a) = d \).
	Then there exists \( r_0 > 0 \) such that for any \( r \in (0,r_0] \), there exists \( \varepsilon > 0 \) such that for all \( w \) with \( 0 < \abs{f(a) - w} < \varepsilon \), the equation \( f(z) = w \) has precisely \( d \) distinct roots in \( D(a,r) \setminus \qty{a} \).
\end{theorem}
\begin{proof}
	Let \( g(z) = f(z) - f(a) \).
	Since \( g \) is non-constant, \( g' \not\equiv 0 \) in \( D(a,R) \).
	Applying the principle of isolated zeroes to \( g \) and \( g' \), there exists \( r_0 \in (0,R) \) such that \( g(z) \neq 0 \) and \( g'(z) \neq 0 \) for \( z \in \overline{D(a,r_0)} \setminus \qty{a} \).

	We will show that the conclusion holds for this choice of \( r_0 \).
	Let \( r \in (0,r_0] \), and for \( t \in [0,1] \), let \( \gamma(t) = a + re^{2\pi i t} \) and \( \Gamma(t) = g(\gamma(t)) \).
	Note that \( \Im \Gamma \) is compact and hence closed in \( \mathbb C \), and \( 0 \not\in \Im \Gamma \) since \( g \neq 0 \) on \( \partial D(a,r) \).
	Hence there exists \( \varepsilon > 0 \) such that \( D(0,\varepsilon) \subseteq \mathbb C \setminus \Im \Gamma \).

	We now show that this \( \varepsilon \) satisfies the conditions in the theorem for this \( r \).
	Let \( w \) such that \( 0 < \abs{w-f(a)} < \varepsilon \).
	Then \( w-f(a) \in D(0,\varepsilon) \subseteq \mathbb C \setminus \Im \Gamma \).
	Since \( z \mapsto I(\Gamma;z) \) is locally constant, it is constant on \( D(0,\varepsilon) \), so in particular \( I(\Gamma;w-f(a)) = I(\Gamma;0) \).

	By direct calculation,
	\[
		I(\Gamma;w-f(a)) = \frac{1}{2\pi i} \int_0^1 \frac{g'(\gamma(t)) \gamma'(t)}{g(\gamma(t)) - (w-f(a))} \dd{t} = \frac{1}{2\pi i} \int_{\partial D(a,r)} \frac{f'(z)}{f(z) - w} \dd{z}
	\]
	By the argument principle, \( I(\Gamma;0) = d \), since \( I(\Gamma;0) \) is the number of zeroes of \( g \) in \( D(a,r) \) counted with multiplicity; the zero of \( g \) at \( z = a \) has order \( d \), and it is the only zero in \( D(a,r) \).
	Hence,
	\[
		\frac{1}{2\pi i} \int_{\partial D(a,r)} \frac{f'(z)}{f(z) - w} \dd{z} = d
	\]
	Again, the argument principle shows that the number of zeroes of \( f(z) - w \) in \( D(a,r) \) is \( d \), counted with multiplicity.
	None of these zeroes is equal to \( a \) since \( w \neq f(a) \).
	Since \( f'(z) = g'(z) \neq 0 \) in \( D(a,r) \setminus \qty{a} \), it follows from the Taylor series that these zeroes are simple.
	Thus \( f(z) - w \) has \( d \) distinct zeroes in \( D(a,r) \setminus \qty{a} \).
\end{proof}

\subsection{Open mapping theorem}
\begin{corollary}
	A non-constant holomorphic function maps open sets to open sets.
	That is, non-constant holomorphic functions are open maps.
\end{corollary}
\begin{proof}
	Let \( f \colon U \to \mathbb C \) be holomorphic and non-constant, and let \( V \subseteq U \) be an open set.
	Let \( b \in f(V) \).
	Then \( b = f(a) \) for some \( a \in V \).
	Since \( V \) is open, there exists \( r > 0 \) such that \( D(a,r) \subseteq V \).
	By the local degree theorem, if \( r \) is sufficiently small, there exists \( \varepsilon > 0 \) such that \( w \in D(f(a), \varepsilon) \setminus \qty{f(a)} \implies w = f(z) \) for some \( z \in D(a,r) \setminus \qty{a} \), hence \( D(f(a),\varepsilon) \setminus \qty{f(a)} \subseteq f(D(a,r) \setminus \qty{a}) \).
	Hence \( D(b,\varepsilon) = D(f(a), \varepsilon) \subseteq f(D(a,r)) \subseteq f(V) \).
	Thus, for all \( b \in f(V) \), there exists a disc \( D(b,\varepsilon) \subseteq f(V) \), so \( f(V) \) is open.
\end{proof}

\subsection{Rouch\'e's theorem}
\begin{theorem}
	Let \( \gamma \) be a closed curve bounding a domain \( \Omega \), and let \( f,g \) be holomorphic functions on an open set \( U \) containing \( \Omega \cup \Im \gamma \).
	If \( \abs{f(z) - g(z)} < \abs{g(z)} \) for all \( z \in \Im \gamma \), then \( f \) and \( g \) have the same number of zeroes in \( \Omega \), counted with multiplicity.
\end{theorem}
\begin{proof}
	The strict inequality \( \abs{f-g} < \abs{g} \) on \( \Im \gamma \) implies that \( f,g \) are never zero on \( \Im \gamma \) and hence never zero on some open set \( V \) containing \( \Im \gamma \).
	So \( h = \frac{f}{g} \) is holomorphic and nonzero in \( V \).
	In particular, \( g \) is not identically zero in \( \Omega \), and hence the zeroes of \( g \) in \( \Omega \cup V \) are isolated.
	Hence \( h \) is meromorphic in \( \Omega \cup V \), and \( h \) has no zeroes or poles on \( \Im \gamma \).
	Also, \( f, g \) have finitely many zeroes in \( \Omega \).

	Now, \( \abs{h(z) - 1} < 1 \) for all \( z \in \Im \gamma \).
	Hence, the curve \( \Gamma = h \circ \gamma \) has image contained within \( D(1,1) \).
	Since zero is outside this disc, \( I(\Gamma;0) = 0 \), and so by the argument principle,
	\[
		\sum_{w \in \mathcal P} \ord_h(w) = \sum_{w \in \mathcal Z} \ord_h(w)
	\]
	where \( \mathcal P \) and \( \mathcal Z \) denote the sets of distinct poles and zeroes of \( h \) respectively, and the sums are finite.
	Now, \( \mathcal P = \mathcal P_1 + \mathcal P_2 \) and \( \mathcal Z = \mathcal Z_1 \cup \mathcal Z_2 \), where
	\begin{align*}
		\mathcal P_1 & = \qty{w \in \Omega \colon g(w) = 0; f(w) \neq 0};                  \\
		\mathcal P_2 & = \qty{w \in \Omega \colon g(w) = f(w) = 0; \ord_g(w) > \ord_f(w)}; \\
		\mathcal Z_1 & = \qty{w \in \Omega \colon f(w) = 0; g(w) \neq 0};                  \\
		\mathcal Z_2 & = \qty{w \in \Omega \colon f(w) = g(w) = 0; \ord_f(w) > \ord_g(w)}
	\end{align*}
	Hence,
	\[
		\sum_{w \in \mathcal P_1} \ord_g(w) + \sum_{w \in \mathcal P_2} \qty(\ord_g(w) - \ord_f(w)) = \sum_{w \in \mathcal Z_1} \ord_f(w) + \sum_{w \in \mathcal Z_2} \qty(\ord_f(w) - \ord_g(w))
	\]
	Equivalently,
	\[
		\sum_{w \in \mathcal P_1} \ord_g(w) + \sum_{w \in \mathcal P_2} \ord_g(w) + \sum_{w \in \mathcal Z_2} \ord_g(w) =
		\sum_{w \in \mathcal Z_1} \ord_f(w) + \sum_{w \in \mathcal Z_2} \ord_f(w) + \sum_{w \in \mathcal P_2} \ord_f(w)
	\]
	Adding \( \sum_{w \in \mathcal R} \ord_g(w) \) to the left hand side and the equal number \( \sum_{w \in \mathcal R} \ord_f(w) \) to the right hand side, where
	\[
		\mathcal R = \qty{w \in \Omega \colon f(w) = g(w) = 0; \ord_f(w) = \ord_g(w)}
	\]
	we have
	\[
		\sum_{w \in \Omega \colon g(w) = 0} \ord_g(w) = \sum_{w \in \Omega \colon f(w) = 0} \ord_f(w)
	\]
	as required.
\end{proof}
\begin{example}
	\( z^4 + 6z + 3 \) has three roots counted with multiplicity in \( \qty{1 < \abs{z} < 2} \).
	Let \( f(z) = z^4 + 6z + 3 \).

	On \( \abs{z} = 2 \) we have \( \abs{z^4} = 16 \) and \( \abs{6z+3} \leq 6\abs{z} + 3 = 15 \), so \( \abs{z}^4 > \abs{6z+3} \).
	By Rouch\'e's theorem, \( f \) has the same number of roots inside \( \qty{\abs{z} < 2} \) as \( z^4 \), counting with multiplicity.
	Thus, all roots of \( z^4 + 6z + 3 \) lie inside \( \qty{\abs{z}<2} \); this is all of the roots since \( f \) is a polynomial with degree 4.

	On \( \abs{z} = 1 \), we have \( \abs{6z} = 6 \) and \( \abs{z^4 + 3} \leq \abs{z}^4 + 3 \leq 4 \).
	Again by Rouch\'e's theorem, \( f \) has one root inside \( \qty{\abs{z} < 1} \), as \( 6z \) has one root in this region.
	From the strict inequalities, no roots lie on \( \qty{\abs{z} = 2} \) or \( \qty{\abs{z} = 1} \).
	Hence three roots of \( f \) lie in \( \abs{z \in \mathbb C \colon 1 < \abs{z} < 2} \).
\end{example}
