\subsection{Introduction}
\begin{definition}
	If \( f \colon [a,b] \subset \mathbb R \to \mathbb C \) is a complex function, and the real and imaginary parts of \( f \) are Riemann integrable, then we define
	\[ \int_a^b f(t) \dd{t} = \int_a^b \Re(f(t)) \dd{t} + i \int_a^b \Im(f(t)) \dd{t} \]
	In particular, for \( g \colon [a,b] \to \mathbb R \), we have
	\[ \int_a^b ig(t) \dd{t} = i\int_a^b g(t) \dd{t} \]
	Thus, for a complex constant \( w \in \mathbb C \), we can find
	\[ \int_a^b wf(t) \dd{t} = w \int_a^b f(t) \dd{t} \]
\end{definition}
\begin{proposition}[basic estimate]
	If \( f \colon [a,b] \to \mathbb C \) is continuous, then
	\[ \abs{\int_a^b f(t) \dd{t}} \leq \int_a^b \abs{f(t)} \dd{t} \leq (b-a) \sup_{t \in [a,b]} \abs{f(t)} \]
	Equality holds if and only if \( f \) is constant.
\end{proposition}
\begin{proof}
	If \( \int_a^b f(t) \dd{t} = 0 \) then the proof is complete.
	Otherwise, we can write the value of the integral as \( re^{i\theta} \) for \( \theta \in [0, 2\pi) \).
	Let \( M = \sup_{t \in [a,b]} \abs{f(t)} \).
	Then we have
	\begin{align*}
		\abs{\int_a^b f(t) \dd{t}} &= r \\
		&= e^{-i\theta} \int_a^b f(t) \dd{t} \\
		&= \int_a^b e^{-i\theta} f(t) \dd{t} \\
		&= \int_a^b \Re(e^{-i\theta} f(t)) \dd{t} + i \int_a^b \Im(e^{-\theta f(t)}) \dd{t}
	\end{align*}
	Since the left hand side is real, the imaginary integral vanishes.
	\begin{align*}
		\abs{\int_a^b f(t) \dd{t}} &= \int_a^b \Re(e^{-i\theta} f(t)) \dd{t} \\
		&\leq \int_a^b \abs{e^{-i\theta} f(t)} \dd{t} = \int_a^b \abs{f(t)} \dd{t} \\
		&\leq (b-a)M
	\end{align*}
	Equality holds if and only if \( \abs{f(t)} = M \) and \( \Re(e^{-i\theta} f(t)) = M \) for all \( t \in [a,b] \), which is true only if \( \abs{f(t)} = M \) and \( \arg(f(t)) = \theta \) hence \( f = Me^{i\theta} \) for all \( t \).
\end{proof}

\subsection{Integrating along curves}
\begin{definition}
	Let \( U \subset \mathbb C \) be an open set and let \( f \colon U \to \mathbb C \) be continuous.
	Let \( \gamma \colon [a,b] \to U \) be a \( C^1 \) curve.
	Then the \textit{integral of \( f \) along \( \gamma \)} is
	\[ \int_\gamma f(z) \dd{z} = \int_a^b f(\gamma(t)) \gamma'(t) \dd{t} \]
\end{definition}
This definition is consistent with the previous definition of the integral of a function \( f \) along the interval \( [a,b] \).
The integral along a curve has various convenient properties.
\begin{enumerate}[(i)]
	\item It is invariant under the choice of parametrisation.
		Let \( \varphi \colon [a_1, b_1] \to [a,b] \) be \( C^1 \) and injective with \( \varphi(a_1) = a \) and \( \varphi(b_1) = b \).
		Let \( \delta = \gamma \circ \varphi \colon [a_1, b_1] \to U \).
		Then
		\[ \int_\delta f(z) \dd{z} = \int_\gamma f(z) \dd{z} \]
		Indeed,
		\begin{align*}
			\int_\delta f(z) \dd{z} &= \int_{a_1}^{b_1} f(\gamma (\varphi(t))) \gamma'(\varphi(t)) \varphi'(t) \dd{t} \\
			&= \int_a^b f(\gamma(s)) \gamma'(s) \dd{s} \\
			&= \int_\gamma f(z) \dd{z}
		\end{align*}
	\item The integral is linear.
		It is easy to check that
		\[ \int_\gamma (\lambda f(z) + \mu g(z)) \dd{z} = \lambda \int_\gamma f(z) \dd{z} + \mu \int_\gamma g(z) \dd{z} \]
		for complex constants \( \lambda, \mu \in \mathbb C \).
	\item The additivity property states that if \( \gamma \colon[a,b] \to U \) is \( C^1 \) and \( a < c < b \), then
		\[ \int_\gamma f(z) \dd{z} = \int_{\eval{\gamma}_{[a,c]}} f(z) \dd{z} + \int_{\eval{\gamma}_{c,b}} f(z) \dd{z} \]
	\item We define the \textit{inverse path} \( (-\gamma) \colon [-b, -a] \to U \) by \( (-\gamma)(t) = \gamma(-t) \).
		Then
		\[ \int_{(-\gamma)} f(z) \dd{z} = \int_\gamma f(z) \dd{z} \]
\end{enumerate}
\begin{definition}
	Let \( \gamma \colon [a,b] \to \mathbb C \) be a \( C^1 \) curve.
	Then the \textit{length} of \( \gamma \) is
	\[ \mathrm{length}(\gamma) = \int_a^b \abs{\gamma'(t)} \dd{t} \]
\end{definition}
\begin{definition}
	A \textit{piecewise \( C^1 \) curve} is a continuous map \( \gamma \colon [a,b] \to \mathbb C \) such that there exists a finite subdivision
	\[ a < a_0 < a_1 < \dots < a_n = b \]
	such that each \( \gamma_j = \eval{\gamma}_{[a_{j-1}, a_j]} \) is \( C^1 \) for \( 1 \leq j \leq n \).
	Then, for such a piecewise \( C^1 \) curve, we define
	\[ \int_\gamma f(z) \dd{z} = \sum{j=1}^n \int_{\gamma_j} f(z) \dd{z} \]
	and
	\[ \mathrm{length}(\gamma) = \sum_{j=1}^n \mathrm{length}(\gamma_j) = \sum_{j=1}^n \int_{a_{j-1}}^{a_j} \abs{\gamma'(t)} \dd{t} \]
	By the additivity property, both definitions are invariant under changing the subdivision.
	From here, we will use `curve' to refer to `piecewise \( C^1 \) curve', unless stated otherwise.
\end{definition}
\begin{definition}
	If \( \gamma_1 \colon [a,b] \to \mathbb C \) and \( \gamma_2 \colon [c,d] \) are curves with \( \gamma_1(b) = \gamma_2(c) \), we define the \textit{sum} of \( \gamma_1 \) and \( \gamma_2 \) to be the curve
	\[ (\gamma_1 + \gamma_2) \colon [a,b+d-c] \to \mathbb C;\quad (\gamma_1 + \gamma_2)(t) = \begin{cases}
		\gamma_1(t) & a \leq t \leq b \\
		\gamma_2(t - b + c) & b \leq t \leq b + d - c
	\end{cases} \]
\end{definition}
\begin{proposition}
	Let \( f \colon U \to \mathbb C \) be continuous and \( \gamma \colon [a,b] \to \mathbb C \), we have
	\[ \abs{\int_\gamma f(z) \dd{z}} \leq \mathrm{length}(\gamma) \sup_\gamma \abs{f} \]
	where \( \sup_\gamma g \equiv \sup_{t \in [a,b]} g(\gamma(t)) \).
\end{proposition}
\begin{proof}
	If \( \gamma \) is \( C^1 \), then
	\[ \abs{\int_\gamma f(z) \dd{z}} = \abs{\int_a^b f(\gamma(t)) \gamma'(t) \dd{t}} \leq \int_a^b \abs{f(\gamma(t))} \cdot \abs{\gamma'(t)} \dd{t} \leq \sup_{t \in [a,b]} \abs{f(\gamma(t))} \mathrm{length}(\gamma) \]
	If \( \gamma \) is piecewise \( C^1 \), then the result follows from the definition of a piecewise \( C^1 \) function and the above.
\end{proof}

\subsection{Fundamental theorem of calculus}
\begin{theorem}[fundamental theorem of calculus]
	Let \( f \colon U \to \mathbb C \) be continuous on an open set \( U \subset \mathbb C \).
	Let \( F \colon U \to \mathbb C \) be a function such that \( F'(z) = f(z) \) for all \( z \in U \).
	Then, for any curve \( \gamma \colon [a,b] \to U \), we have
	\[ \int_\gamma f(z) \dd{z} = F(\gamma(b)) - F(\gamma(a)) \]
	If \( \gamma \) is a closed curve, then \( \int_\gamma f(z) = 0 \).
	Such a function \( F \) is known as an \textit{antiderivative} of \( f \).
\end{theorem}
\begin{proof}
	\[ \int_\gamma f(z) \dd{z} = \int_a^b f(\gamma(t)) \gamma'(t) \dd{t} = \int_a^b \dv{t} F(\gamma(t)) \dd{t} = F(\gamma(b)) - F(\gamma(a)) \]
\end{proof}
\begin{remark}
	Note that we assume that \( f \) exists such that \( F'(z) = f(z) \); such an \( f \) is not provided for by the theorem.
	A function \( F \) satisfying these criteria is called an \textit{antiderivative} of \( f \).
\end{remark}
\begin{example}
	For an integer \( n \) and the curve \( \gamma(t) = Re^{2\pi it} \) for \( t = [0,1] \), consider the integral \( \int_\gamma z^n \dd{z} \).
	For \( n \neq -1 \), the function \( \frac{z^{n+1}}{n+1} \) is an antiderivative of \( z^n \).
	Hence, \( \int_\gamma z^n \dd{z} = 0 \) since \( \gamma \) is a closed curve.
	If \( n = -1 \), we can use the definition of the integral to find
	\[ \int_\gamma \frac{1}{z} \dd{z} = \int_0^1 \frac{1}{\gamma(t)} \gamma'(t) \dd{t} = \int_0^1 \frac{1}{Re^{2\pi i t}}2\pi i R e^{2 \pi i t} \dd{t} = 2 \pi i \]
	This is not zero, hence for all \( R > 0 \), \( \frac{1}{z} \) has no antiderivative in any open set containing the circle \( \abs{\abs{Z} = R} \).
	In particular, for any branch of logarithm \( \lambda \), it has derivative \( \frac{1}{z} \), hence there exists no branch of logarithm on \( \mathbb C^\star = \mathbb C \setminus \qty{0} \).
\end{example}
\begin{theorem}[converse to fundamental theorem of calculus]
	Let \( U \subset \mathbb C \) be a domain.
	If \( f \colon U \to \mathbb C \) is continuous and if \( \int_\gamma f(z) \dd{z} \) for every closed curve \( \gamma \) in \( U \), then \( f \) has an antiderivative.
	In other words, there exists a holomorphic function \( F \colon U \to \mathbb C \) such that \( F' = f \) in \( U \).
\end{theorem}
\begin{proof}
	Let \( a_0 \in U \).
	Then for \( w \in U \), we can define
	\[ F(w) = \int_{\gamma_w} f(z) \dd{z} \]
	where \( \gamma_w \colon [0,1] \to \mathbb C \) is a curve with \( \gamma_w(0) = a_0 \) and \( \gamma_w(1) = w \).

	The definition of \( F \) is independent of the choice of \( \gamma_w \) is independent of the choice of \( \gamma_w \).
	Indeed, suppose two paths \( \gamma_w, \gamma_w' \) exist.
	Then the curve \( \gamma_w + (-\gamma_w') \) is a closed path, and by assumption the integral along this curve is zero.
	Thus \( F \) is independent of the choice of path as claimed.
	So \( F \) is a well-defined function.

	Now, let \( w \in U \).
	Since \( U \) is an open set, there exists \( r > 0 \) such that \( D(w,r) \subset U \).
	For \( h \in \mathbb C \) with \( 0 < \abs{h} < r \), let \( \delta_h \) be the radial path \( t \mapsto w + th \) for \( t \in [0,1] \).
	Now we define
	\[ \gamma = \gamma_w + \delta_h + (-\gamma_{w+h}) \]
	This is a closed curve contained within \( U \), hence \( \int_\gamma f(z) \dd{z} = 0 \).
	Thus
	\[ \int_{\gamma_{w+h}} f(z) \dd{z} = \int_{\gamma_w} f(z) \dd{z} + \int_{\delta_h} f(z) \dd{z} \]
	Informally, the integral has an additivity property which is independent of the path taken.
	Rewriting this using \( F \),
	\begin{align*}
		F(w+h) &= F(w) + \int_{\delta_h} f(z) \dd{z} \\
		&= F(w) + \int_{\delta_h} (f(z) + f(w) - f(w)) \dd{z} \\
		&= F(w) + hf(w) + \int_{\delta_h} (f(z) - f(w)) \dd{z}
	\end{align*}
	Hence, by continuity of \( f \),
	\begin{align*}
		\abs{\frac{F(w+h) - F(w)}{h} - f(w)} &= \frac{1}{\abs{h}} \abs{\int_{\delta_h} (f(z) - f(w)) \dd{z}} \\
		&\leq \frac{1}{\abs{h}} \mathrm{length}(\delta_h) \sup_{z \in \Im \delta_h} \abs{f(z) - f(w)} \\
		&= \sup_{z \in \Im \delta_h} \abs{f(z) - f(w)} \\
		\therefore \lim_{h \to 0} \abs{\frac{F(w+h) - F(w)}{h} - f(w)} &= \lim_{h \to 0} \sup_{z \in \Im \delta_h} \abs{f(z) - f(w)} = 0
	\end{align*}
	Thus, \( F \) is differentiable at \( w \) with \( F'(w) = f(w) \).
\end{proof}

\subsection{Star-shaped domains}
\begin{definition}
	A domain \( U \) is \textit{star-shaped}, or a \textit{star domain}, if there exists a (not necessarily unique) centre \( a_0 \in U \) such that for all \( w \in U \), the straight line segment \( [a_0, w] \) is contained within \( U \).
\end{definition}
\begin{remark}
	Any disk is convex; any convex domain is star-shaped; any star-shaped domain is path-connected.
	The reverse implications are not true in general.
\end{remark}
\begin{definition}
	A \textit{triangle} in \( \mathbb C \) is the \textit{convex hull} of three points in \( \mathbb C \).
	The (closed) convex hull of a set \( S \) is the smallest (closed) convex set \( C \) such that \( S \subseteq C \).
	In this case, if \( z_1, z_2, z_3 \in \mathbb C \), we have
	\[ T = \qty{az_1 + bz_2 + cz_3 \colon 0 \leq a,b,c \leq 1, a+b+c=1} \]
	When used as a curve, the boundary \( \partial T \) represents the piecewise affine closed curve \( \gamma = \gamma_1 + \gamma_2 + \gamma_3 \) where \( \gamma_i \) are affine functions parametrising the three line segments on the boundary of \( T \).
\end{definition}
\begin{corollary}
	Let \( U \) be a star-shaped domain.
	Let \( f \colon U \to \mathbb C \) be continuous and \( \int_{\partial T} f(z) \dd{z} = 0 \) for any triangle \( T \subset U \).
	Then \( f \) has an antiderivative in \( U \).
\end{corollary}
\begin{remark}
	This is a relaxation of the conditions from the previous theorem.
\end{remark}
\begin{proof}
	Let \( a_0 \) be a centre for the domain \( U \).
	Let \( w \) be an arbitrary point in \( U \).
	Then let \( \gamma_w \) be the affine function parametrising the directed line segment \( [a_0,w] \), and let \( F(w) = \int_{\gamma_w} f(z) \dd{z} \).
	Using \( h \) and \( \delta_h \) as above, by letting \( \gamma = \gamma_w + \delta_h + (-\gamma_{w+h}) \) we then have \( \int f(z) \dd{z} = \pm \int_{\partial T} f(z) \dd{z} \) for a triangle \( T \subset U \).
	Since the integral around a triangle is zero by hypothesis, \( \int_\gamma f(z) \dd{z} = 0 \).
	We then complete the proof in analogous way to the general case.
\end{proof}
\begin{theorem}[Cauchy's theorem for triangles]
	Let \( U \subset \mathbb C \) be an open set and \( f \colon U \to \mathbb C \) be a holomorphic function.
	Then \( \int_{\partial T} f(z) \dd{z} = 0 \) for all triangles \( T \subset U \).
\end{theorem}
\begin{proof}
	Let \( \eta(t) = \int_{\partial T} f(z) \dd{z} \).
	We will subdivide the triangle \( T \) into four smaller triangles \( T^{(1)}, T^{(2)}, T^{(3)}, T^{(4)} \).
	The vertices of the inner triangle are the midpoints of the sides of \( T \), and the three other triangles are constructed to fill the remaining area of \( T \).
	Thus,
	\[ \eta(T) = \int_{\partial T^{(1)}} f(z) \dd{z} + \int_{\partial T^{(2)}} f(z) \dd{z} + \int_{\partial T^{(3)}} f(z) \dd{z} + \int_{\partial T^{(4)}} f(z) \dd{z} \]
	Then, by the triangle inequality, there exists a triangle \( T^{(j)} \) such that
	\[ \abs{\int_{\partial T^{(j)}} f(z) \dd{z}} \geq \frac{\abs{\eta(T)}}{4} \]
	Let \( T_0 = T \), and \( T_1 = T^{(j)} \), so \( \abs{\eta(T_1)} \geq \frac{1}{4}\abs{\eta(T_0)} \).
	We can show geometrically that for any choice of \( T_i \), \( \mathrm{length}(\partial T_1) = \frac{1}{2}\mathrm{length}(\partial T_0) \).
	Inductively, we can subdivide \( T_i \) and produce \( T_{i+1} \), such that
	\[ T_0 \supset T_1 \supset \cdots;\quad \abs{\eta(T_n)} \geq \frac{1}{4}\abs{\eta(T_{n-1})};\quad \mathrm{length}(\partial T_n) = \frac{1}{2} \mathrm{length}(\partial T_{n-1}) \]
	Hence,
	\[ \abs{\eta(T_n)} \geq \frac{1}{4^n}\abs{\eta(T_0)};\quad \mathrm{length}(\partial T_n) = \frac{1}{2^n} \mathrm{length}(\partial T_0) \]
	Since \( T_n \) are non-empty, nested closed subsets with diameter converging to zero, we can show that \( \bigcap_{n=1}^\infty T_n = \qty{z_0} \) for some \( z_0 \in \mathbb C \).
	Let \( \varepsilon > 0 \).
	Since \( f \) is differentiable at \( z_0 \), there exists \( \delta > 0 \) such that
	\begin{align*}
		z \in U, \abs{z - z_0} < \delta &\implies \abs{\frac{f(z) - f(z_0)}{z - z_0} - f'(z_0)} \leq \varepsilon \\
		&\implies \abs{f(z) - f(z_0) - f'(z_0)(z - z_0)} \leq \varepsilon \abs{z - z_0}
	\end{align*}
	Now, observe that for all \( n \),
	\[ \int_{\partial T_n} f(z) \dd{z} = \int_{\partial T_n} (f(z) - f(z_0) - f'(z_0)(z - z_0)) \dd{z} \]
	since \( \int_{\partial T_n} \dd{z} = \int_{\partial T_n} z \dd{z} = 0 \) by the fundamental theorem of calculus.
	Let \( n \) such that \( T_n \subset D(z_0, \delta) \).
	Hence,
	\begin{align*}
		\frac{1}{4^n} \abs{\eta(T_0)} &\leq \abs{\eta(T_n)} \\
		&= \abs{\int_{\partial T_n} f(z) \dd{z}} \\
		&= \abs{\int_{\partial T_n} (f(z) - f(z_0) - f'(z_0)(z - z_0)) \dd{z}} \\
		&\leq \qty(\sup_{z \in \partial T_n} \abs{f(z) - f(z_0) - f'(z_0)(z-z_0)}) \mathrm{length}(\partial T_n) \\
		&\leq \varepsilon \qty(\sup_{z \in \partial T_n} \abs{z - z_0}) \mathrm{length}(\partial T_n) \\
		&\leq \varepsilon \mathrm{length}(\partial T_n)^2 \\
		&\leq \frac{\varepsilon}{4^n} \mathrm{length}(\partial T_0)^2 \\
		\therefore \abs{\eta(T_0)} &\leq \varepsilon \mathrm{length}(\partial T_0)^2
	\end{align*}
	\( \varepsilon \) was arbitrary, hence \( \eta(T_0) \) must be zero.
\end{proof}
