\subsection{Basic notions}
We use the following definitions.
\begin{itemize}
	\item The complex plane is denoted \( \mathbb C \).
	\item The complex conjugate of a complex number \( z \) is denoted \( \overline z \).
	\item The modulus is denoted \( \abs{z} \).
	\item The function \( d(z,w) = \abs{z - w} \) is a metric on \( \mathbb C \).
		All topological notions will be with respect to this metric.
	\item We define the disk \( D(a,r) = \qty{z \in \mathbb C \colon \abs{z-a}<r} \) to be the open ball with centre \( a \) and radius \( r \).
	\item A subset \( U \subset \mathbb C \) is said to be open if it is open with respect to the above metric.
		In particular, by identifying \( \mathbb C \) with \( \mathbb R^2 \), we can see that \( U \subset C \) is open if and only if \( U \subset \mathbb R^2 \) is open with respect to the Euclidean metric.
\end{itemize}
The course concerns itself with complex-valued functions of a single complex variable.
Identifying \( \mathbb C \) with \( \mathbb R^2 \) allows us to construct \( f(z) = u(x,y) + i v(x,y) \), where \( u,v \) are real-valued functions.
We can denote these parts by \( u = \Re(f) \) and \( v = \Im(f) \).

\subsection{Continuity and differentiability}
The definition of continuity is carried over from metric spaces.
That is, \( f \colon A \to \mathbb C \) is continuous at a point \( w \in A \) if
\[ \forall \varepsilon > 0, \exists \delta > 0, \forall z \in A, \abs{z-w} < \delta \implies \abs{f(z) - f(w)} < \varepsilon \]
Equivalently, the limit \( \lim_{z \to w} f(z) \) exists and takes the value \( f(w) \).
We can easily check that \( f \) is continuous at \( w = c + id \in A \) if and only if \( u, v \) are continuous at \( (c,d) \) with respect to the Euclidean metric on \( A \subset \mathbb R^2 \).

\begin{definition}
	Let \( f \colon U \to \mathbb C \), where \( U \) is open in \( \mathbb C \).
	\begin{enumerate}[(i)]
		\item \( f \) is \textit{differentiable} at \( w \in U \) if the limit
			\[ f'(w) = \lim_{z \to w} \frac{f(z) - f(w)}{z-w} \]
			exists, and its value is complex.
			We say that \( f'(w) \) is the derivative of \( f \) at \( w \).
		\item \( f \) is \textit{holomorphic} at \( w \in U \) if there exists \( \varepsilon > 0 \) such that \( D(w, \varepsilon) \subset U \) and \( f \) is differentiable at every point in \( D(w, \varepsilon) \).
		\item \( f \) is holomorphic in \( U \) if \( f \) is holomorphic at every point in \( U \), or equivalently, \( f \) is differentiable everywhere.
	\end{enumerate}
\end{definition}

Differentiation of composite functions, sums, products and quotients can be computed in the complex case exactly as they are in the real case.

\begin{example}
	Polynomials \( p(z) \sum_{j=0}^n a_j z^j \) for complex coefficients \( a_j \) are holomorphic on \( \mathbb C \).
	Further, if \( p, q \) are polynomials, \( \frac{p}{q} \) is holomorphic on \( \mathbb C \setminus \qty{z \colon q(z) = 0} \).
\end{example}

\begin{remark}
	The differentiability of \( f \) at a point \( c + id \) is not equivalent to the differentiability of \( u, v \) at \( (c,d) \).
	\( u \colon U \to \mathbb R \) is differentiable at \( (c,d) \in U \) if there is a `good' affine approximation of \( u \) at \( (c,d) \); there exists a linear transformation \( L \colon \mathbb R^2 \to \mathbb R \) such that
	\[ \lim_{(x,y) \to (c,d)} \frac{u(x,y) - (u(c,d)-L(x-c,y-d))}{\sqrt{(x-c)^2+(y-d)^2}} = 0 \]
	If \( u \) is differentiable at \( (c,d) \), then \( L \) is uniquely defined, and can be denoted \( L = D u(c,d) \).
	\( L \) is given by the partial derivatives of \( u \), which are
	\[ L(x,y) = \qty(\pdv{u}{x}\qty(c,d))x+\qty(\pdv{u}{y}\qty(c,d))y \]
	This seems to imply that the differentiability of \( f \) requires more than the differentiability of \( u,v \).
\end{remark}

\subsection{Cauchy-Riemann equations}
\begin{theorem}
	\( f = u + iv \colon U \to \mathbb C \) is differentiable at \( w = c + id \in U \) if and only if \( u,v \colon U \to \mathbb R \) are differentiable at \( (c,d) \in U \) and \( u,v \) satisfy the Cauchy-Riemann equations at \( (c,d) \), which are
	\[ \pdv{u}{x} = \pdv{v}{y};\quad \pdv{u}{y} = -\pdv{v}{x} \]
	If \( f \) is differentiable at \( w = c + id \), then
	\[ f'(w) = \pdv{u}{x}\qty(c,d) + i \pdv{v}{x}\qty(c,d) \]
	and other expressions, which follow directly from the Cauchy-Riemann equations.
\end{theorem}
\begin{proof}
	All of the following statements will be bi-implications.
	Suppose \( f \) is differentiable at \( w \) with \( f'(w) = p+iq \), so
	\begin{align*}
		\lim_{z \to w} \frac{f(z) - f(w)}{z-w} &= p+iq \\
		\lim_{z \to w} \frac{f(z) - f(w) - (z-w)(p+iq)}{\abs{z-w}} &= 0
	\end{align*}
	By separating real and imaginary parts, writing \( w = c + id \) we have
	\[ \lim_{(x,y) \to (c,d)} \frac{u(x,y) - u(c,d) - p(x-c) + q(y-d)}{\sqrt{(x-c)^2 + (y-d)^2}} = 0;\quad \lim_{(x,y) \to (c,d)} \frac{v(x,y) - v(c,d) - q(x-c) - p(y-d)}{\sqrt{(x-c)^2 + (y-d)^2}} = 0 \]
	Thus, \( u \) is differentiable at \( (c,d) \) with \( Du(c,d)(x,y) = px - qy \) and \( v \) is differentiable at \( (c,d) \) with \( Dv(c,d)(x,y) = qx + py \).
	\[ u_x(c,d) = v_y(c,d) = p;\quad -u_y(c,d) = v_x(c,d) = q \]
	Hence the Cauchy-Riemann equations hold at \( (c,d) \).
	We also find that if \( f \) is differentiable at \( w \), we have \( f'(w) = u_x(c,d) + i v_x(c,d) \).
\end{proof}
\begin{remark}
	If \( u,v \) simply satisfy the Cauchy-Riemann equations alone, that does not imply differentiability of \( f \).
	\( u,v \) must also be differentiable.
\end{remark}
\begin{remark}
	If we simply want to show that the differentiability of \( f \) implies that the Cauchy-Riemann equations hold, we can proceed in a simpler way.
	For \( t \in \mathbb R \),
	\[ f'(w) = \lim_{t \to 0} \qty(\frac{u(c+t,d) - u(c,d)}{t} + i \frac{v(c+t,d) - v(c,d)}{t}) \]
	Hence the real part and the complex part both exist, so \( u_x(c,d) \) and \( v_x(c,d) \) exist, and \( f'(w) = u_x(c,d) + i v_x(c,d) \).
	If we instead considered a perturbation along the imaginary axis, we find \( f'(w) = v_y(c,d) - iu_y(c,d) \), giving the Cauchy-Riemann equations.
\end{remark}
\begin{example}
	The complex conjugate function \( z \mapsto \overline z \) is not differentiable.
	Here, \( u(x,y) = x \), and \( v(x,y) = -y \), so the Cauchy-Riemann equations do not hold.
\end{example}
\begin{corollary}
	If \( u, v \) have continuous partial derivatives at \( (c,d) \) and satisfy the Cauchy-Riemann equations at \( (c,d) \), then \( f \) is differentiable at \( c + id \).
	In particular, if \( u,v \) are \( C^1 \) functions on \( U \) (i.e.\ have continuous partial derivatives in \( U \)) satisfying the Cauchy-Riemann equations everywhere, then \( f \) is holomorphic (in \( U \)).
\end{corollary}
\begin{proof}
	If \( u,v \) have continuous partial derivatives then \( u, v \) are differentiable at \( (c,d) \) by Analysis and Topology.
\end{proof}

\subsection{Curves and path-connectedness}
\begin{definition}
	A \textit{curve} is a continuous function \( \gamma \colon [a,b] \to \mathbb C \), where \( a,b \in \mathbb R \).
	\( \gamma \) is a \( C^1 \) curve if \( \gamma' \) exists and is continuous on \( [a,b] \).
	An open set \( U \subset \mathbb C \) is \textit{path connected} if for any two points \( z,w \in U \), there exists \( \gamma \colon [0,1] \to U \) such that \( \gamma(0) = z \) and \( \gamma(1) = w \).
	A \textit{domain} is a non-empty, open, path connected subset of \( \mathbb C \).
\end{definition}
\begin{corollary}
	Let \( U \) be a domain.
	Let \( f \colon U \to \mathbb C \) be a holomorphic function with derivative zero everywhere.
	Then \( f \) is constant on \( U \).
\end{corollary}
\begin{proof}
	By the Cauchy-Riemann equations, \( f' = 0 \) implies that \( Du = Dv = 0 \) in \( U \).
	By Analysis and Topology, the path-connectedness of \( U \) implies that \( u \) and \( v \) are constant functions.
\end{proof}

\subsection{Power series}
Recall the following theorem from IA Analysis.
\begin{theorem}
	Let \( (c_n)_{n=0}^\infty \) be a sequence of complex numbers.
	Then, the power series
	\[ \sum_{n=0}^\infty c_n (z-a)^n \]
	has a unique \textit{radius of convergence} \( R \in [0,\infty] \) such that the power series converges absolutely for \( \abs{z - a} < R \) and diverges if \( \abs{z - a} > R \).
	Further, if \( 0 < r < R \), the series converges uniformly with respect to \( z \) on the compact disk \( D(a,r) \).
\end{theorem}
Note that
\[ R = \sup\qty{r \geq 0 \colon \lim_{n \to \infty} \abs{c_n}r^n = 0};\quad \frac{1}{R} = \limsup_{n \to \infty} \abs{c_n}^{\frac{1}{n}} \]
\begin{theorem}
	Let the sequence \( (c_n) \) define a power series \( f \) centred around \( a \) with positive radius of convergence \( R \).
	Then, the function \( f \colon D(a,R) \to \mathbb C \) satisfies
	\begin{enumerate}[(i)]
		\item \( f \) is holomorphic on \( D(a,R) \);
		\item the term-by-term differentiated series \( \sum_{n=1}^\infty nc_n(z-a)^{n-1} \) also has radius of convergence equal to \( R \), and this series is exactly the value of \( f' \);
		\item \( f \) has derivatives of all orders on \( D(a,R) \) and \( c_n = \frac{f^{(n)}(a)}{n!} \);
		\item if \( f \) vanishes on \( D(a, \varepsilon) \) for any \( \varepsilon > 0 \), then \( f \equiv 0 \) on \( D(a,R) \).
	\end{enumerate}
\end{theorem}
\begin{proof}
	\begin{enumerate}[(i)]
		\item Without loss of generality, let \( a = 0 \).
			\( \sum_{n=1}^\infty nc_n(z-a)^{n-1} \) has some radius of convergence \( R_1 \).

			Let \( z \in D(0,R) \) and choose \( \rho \) such that \( \abs{z} < \rho < R \).
			Then,
			\[ n \abs{c_n}\abs{z}^{n-1} = n \abs{c_n} \abs{\frac{z}{\rho}}^{n-1} \rho^{n-1} \leq \abs{c_n} \rho^{n-1} \]
			for sufficiently large \( n \), since \( n \abs{\frac{z}{\rho}}^{n-1} \to 0 \) as \( n \to \infty \).
			Since \( \sum \abs{c_n} \rho^n \) converges, we must have that \( n \abs{c_n}\abs{z}^{n-1} \) converges.
			Hence \( R_1 \geq R \).

			Now, since
			\[ \abs{c_n}\abs{z}^n \leq n \abs{c_n}\abs{z}^n = \abs{z}\qty(n \abs{c_n}\abs{z}^{n-1}) \]
			If \( \sum n\abs{c_n}{z}^{n-1} \) converges then so does \( \sum \abs{c_n}\abs{z}^n \).
			Hence \( R_1 \leq R \).
			This leads us to conclude \( R_1 = R \).
		\item Let \( z \in D(0,R) \).
			The statement that \( f' \) is the above differentiated power series at \( z \) is equivalent to continuity at \( z \) of the function
			\[ g \colon D(0,R) \to \mathbb C;\quad g(w) = \begin{cases}
				\frac{f(w) - f(z)}{w - z} & w \neq z \\
				\sum_{n=1}^\infty n c_n z^{n-1} & w = z
			\end{cases} \]
			Substituting for \( f \), we have \( g(w) = \sum_{n=1}^\infty h_n(w) \) for \( w \in D(0,R) \) where
			\[ h_n(w) = \begin{cases}
				\frac{c_n(w^n - z^n)}{w - z} & w \neq z \\
				n c_n z^{n-1} & w = z
			\end{cases} \]
			Note that \( h_n \) is continuous on \( D(0,R) \).
			Further, note that
			\[ \frac{w^n - z^n}{w - z} = \sum_{j=0}^{n-1} z^j w^{n-1-j} \]
			We have that for all \( r \) with \( \abs{z} < r < R \) and all \( w \in D(0,r) \), \( \abs{h_n}(w) \leq n \abs{c_n} r^{n-1} \equiv M_n \).
			Since \( \sum M_n < \infty \), the Weierstrass \( M \) test shows that \( \sum h_n \) converges uniformly on \( D(0,r) \).
			A uniform limit of continuous functions is continuous, hence \( g = \sum h_n \) is continuous in \( D(0,r) \) and in particular at \( z \).
		\item Part (ii) can be applied inductively.
			The equation \( c_n = \frac{f^{(n)}(a)}{n!} \) can be found by differentiating the series \( n \) times.
		\item If \( f \equiv 0 \) in some disk \( D(a, \varepsilon) \), then \( f^{(n)}(a) = 0 \) for all \( n \).
			Thus the powr series is identically zero.
	\end{enumerate}
\end{proof}
