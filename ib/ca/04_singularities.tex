\subsection{Motivation}
Let \( U \) be open, and \( \gamma \) be a closed curve in \( U \) homologous to zero in \( U \).
Then, if \( f \colon U \to \mathbb C \) is holomorphic, we have Cauchy's integral formula
\[ \int_\gamma \underbrace{\frac{f(z) \dd{z}}{z-a}}_{g(z) \dd{z}} = 2 \pi i \cdot I(\gamma;a) f(a) \]
for all \( a \in U \setminus \Im \gamma \).
This allows us to compute \( \int_\gamma g(z) \dd{z} \) for a holomorphic function \( g \colon U \setminus \qty{a} \to \mathbb C \) where \( \gamma \) does not pass through the point \( a \), provided that \( g \) satisfies a particular condition: \( (z-a) g(z) \) is the restriction to \( U \setminus \qty{a} \) of a holomorphic function \( f \colon U \to \mathbb C \).
We wish to drop this restriction and observe the consequences; that is, we wish to compute \( \int_\gamma g(z) \dd{z} \) for arbitrary holomorphic functions \( g \colon U \setminus \qty{a} \to \mathbb C \) for \( a \in U \) and \( a \not\in \Im \gamma \).
For example, consider \( g(z) = e^{z^{-1}} \) for \( U = \mathbb C \) and \( a = 0 \), \( \gamma = \partial D(0,1) \).
Note that \( zg(z) = ze^{z^{-1}} \) is not continuous at \( z = 0 \), so it is certainly not holomorphic.
This leads us to the study of singularities, and to eventually prove the residue theorem.

\subsection{Isolated singularities}
\begin{definition}
	Let \( U \subseteq \mathbb C \) be open.
	If \( a \in U \) and \( f \colon U \setminus \qty{a} \to \mathbb C \) is holomorphic, we say that \( f \) has an \textit{isolated singularity} at \( a \).
\end{definition}
\begin{definition}
	An isolated singularity \( a \) of \( f \) is a \textit{removable singularity} if \( f \) can be defined at \( a \) such that the extended function is holomorphic on \( U \).
\end{definition}
\begin{proposition}
	Let \( U \) be open, \( a \in U \), and \( f \colon U \setminus \qty{a} \to \mathbb C \) be holomorphic.
	Then, the following are equivalent.
	\begin{enumerate}[(i)]
		\item \( f \) has a removable singularity at \( a \);
		\item \( \lim_{z \to a} f(z) \) exists in \( \mathbb C \);
		\item there exists \( D(a,\varepsilon) \subseteq U \) such that \( \abs{f(z)} \) is bounded in \( D(a,\varepsilon) \setminus \qty{a} \);
		\item \( \lim_{z \to a} (z-a) f(z) = 0 \).
	\end{enumerate}
\end{proposition}
