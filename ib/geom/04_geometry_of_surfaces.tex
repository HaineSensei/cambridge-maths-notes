\subsection{Length}
Let \( \gamma \colon (a,b) \to \mathbb R^3 \) be smooth.
The \textit{length} of \( \gamma \) is
\[ L(\gamma) = \int_a^b \norm{\gamma'(t)} \dd{t} \]
This result is independent of the choice of parametrisation.
Let \( s \colon (A,B) \to (a,b) \) be a monotonically increasing function, and let \( \tau(t) = \gamma(s(t)) \).
Then
\[ L(\tau) = \int_A^B \norm{\tau'(t)} \dd{t} = \int_A^B \norm{\gamma(s(t))} \abs{s'(t)} \dd{t} = \int_a^b \norm{\gamma(t')} \dd{t'} = L(\gamma) \]
\begin{lemma}
	If \( \gamma \colon (a,b) \to \mathbb R^3 \) is continuously differentiable and \( \gamma'(t) \neq 0 \), then \( \gamma \) can be parametrised by arc length.
\end{lemma}
The proof is left as an exercise.
Let \( \Sigma \) be a smooth surface in \( \mathbb R^3 \), and let \( \sigma \colon V \to U \subseteq \Sigma \) be an allowable parametrisation.
If \( \gamma \colon (a,b) \to \mathbb R^3 \) is smooth and its image is contained within \( U \), then there exist functions \( (u(t), v(t)) \colon (a,b) \to V \) such that \( \gamma(t) = \sigma(u(t), v(t)) \).
Hence \( \gamma'(t) = \sigma_u u'(t) + \sigma_v v'(t) \), giving
\[ \norm{\gamma'(t)}^2 = E u'(t)^2 + 2F u'(t) v'(t) + G v'(t)^2 \]
for functions
\[ E = \inner{\sigma_u, \sigma_u};\quad F = \inner{\sigma_u, \sigma_v} = \inner{\sigma_v, \sigma_u};\quad G = \inner{\sigma_v, \sigma_v} \]
where \( \inner{\wildcard,\wildcard} \) represents the usual Euclidean inner product.
Note that \( E, F, G \) depend only on \( \sigma \) and not on \( \gamma \).
\begin{definition}
	The \textit{first fundamental form} of \( \Sigma \) in the parametrisation \( \sigma \) is the expression
	\[ E \dd{u}^2 + 2F \dd{u} \dd{v} + G \dd{v}^2 \]
	This notation is illustrative of the fact that if \( \gamma \) has image in the image of \( \sigma(v) \), we find
	\[ L(\gamma) = \int_a^b \sqrt{E (u')^2 + 2F u'v' + G (v')^2} \dd{t} \]
	where \( \gamma(t) = \sigma(u(t),v(t)) \).
\end{definition}
\begin{remark}
	The Euclidean inner product on \( \mathbb R^3 \) provides an inner product on the subspace \( T_p \Sigma \).
	Choosing a parametrisation \( \sigma \), we can say \( T_p \Sigma = \Im D \eval{\sigma}_0 = \vecspan{\qty{\sigma_u, \sigma_v}} \) where \( \sigma(0) = p \).
	The first fundamental form is a symmetric bilinear form on the tangent spaces \( T_p \Sigma \), varying smoothly in \( p \).
	However, we choose to express this in a basis coming from the parametrisation \( \sigma \).
	In particular, we can think about the matrix expression
	\[ \begin{pmatrix}
		E & F \\
		F & G
	\end{pmatrix} \]
\end{remark}
\begin{example}
	The plane \( \mathbb R^2_{xy} \subset \mathbb R^3 \) has the parametrisation \( (u,v) \mapsto (u,v,0) \).
	Hence, \( \sigma_u = e_1 \) and \( \sigma_v = e_2 \), hence the first fundamental form is \( \dd{u}^2 + \dd{v}^2 \).
	We could also use polar coordinates, using \( \sigma(r,\theta) = (r\cos\theta,r\sin\theta,0) \).
	This parametrises the plane without the origin.
	This gives \( \sigma_r = (\cos\theta,\sin\theta,0) \) and \( \sigma_\theta = (-r\sin\theta, r\cos\theta,0) \).
	The first fundamental form is \( \dd{r}^2 + r^2 \dd{\theta}^2 \).
\end{example}
\begin{definition}
	Let \( \Sigma, \Sigma' \) be smooth surfaces in \( \mathbb R^3 \).
	We say that they are \textit{isometric} if there exists a diffeomorphism \( f\colon \Sigma \to \Sigma' \) that preserves the lengths of all curves.
	More formally, for every smooth curve \( \gamma \colon (a,b) \to \Sigma \), the length of \( \gamma \) is the same as the length of \( f \circ \gamma \).
\end{definition}
\begin{example}
	Let \( \Sigma' = f(\Sigma) \) where \( f \colon \mathbb R^3 \to \mathbb R^3 \) is a global isometry, or rigid motion, of \( \mathbb R^3 \); that is, \( v \mapsto Av+b \) for an orthogonal matrix \( A \).
	These isometries preserve the Eucliean inner product on \( \mathbb R^3 \), hence \( f \) preserves length.
	However, in the definition, we need not map all of \( \mathbb R^3 \) to itself, just \( \Sigma \to \Sigma' \).
\end{example}
\begin{definition}
	We say that \( \Sigma \) and \( \Sigma' \) are \textit{locally isometric} near points \( p \in \Sigma \) and \( q \in \Sigma' \) if there exist open neighbourhoods \( U \) of \( p \) and \( V \) of \( q \) such that \( U \) and \( V \) are isometric.
	We can also say that \( \Sigma \) and \( \Sigma' \) are locally isometric if they are locally isometric at all points.
	% TODO: does this mean "locally isometric for p arbitrary and q fixed given p" or "p arbitrary, q arbitrary"?
\end{definition}
\begin{lemma}
	Smooth surfaces \( \Sigma, \Sigma' \) in \( \mathbb R^3 \) are locally isometric near \( p \in \Sigma \) and \( q \in \Sigma' \) if and only if there exist allowable parametrisations \( \sigma \colon V \to U \subseteq \Sigma \) and \( \sigma' \colon V \to U' \subseteq \Sigma' \) such that the first fundamental forms are equivalent.
\end{lemma}
\begin{proof}
	By definition, the first fundamental form of \( \Sigma \) determines the lengths of all curves on \( \Sigma \) that lie in \( U \).
	We will now show that lengths of curves determine the first fundamental form of a parametrisation.
	Given \( \sigma \colon V \to U \), without loss of generality let \( V = B(0,\delta) \) for some \( \delta > 0 \), where \( \sigma(0) = p \).
	Consider, for all \( \varepsilon < \delta \), the curve
	\[ \gamma_\varepsilon \colon [0,\varepsilon] \to U;\quad t \mapsto \sigma(t,0) \]
	Then,
	\[ \dv{\varepsilon} L(\gamma_\varepsilon) = \dv{\varepsilon} \int_0^\varepsilon \sqrt{E(t,0)} \dd{t} = \sqrt{E(\varepsilon,0)} \]
	Hence,
	\[ \eval{\dv{\varepsilon}}_{\varepsilon = 0} L(\gamma_\varepsilon) = \sqrt{E(0,0)} \]
	So we can determine \( E \) at \( p \) by looking at lengths of curves.
	We can similarly consider
	\[ \chi_\varepsilon \colon [0,\varepsilon] \to U;\quad t \mapsto \sigma(0,t) \]
	which determines \( G \).
	Finally, consider
	\[ \lambda_\varepsilon \colon [0,\varepsilon] \to U;\quad t \mapsto \sigma(t,t) \]
	which determines \( \sqrt{(E+2F+G)(0,0)} \) which gives \( F \) implicitly.
\end{proof}
\begin{example}
	The sphere of radius \( a \), given by \( \qty{x^2 + y^2 + z^2 = a^2} \), has an open set with allowable parametrisation
	\[ \sigma(u,v) = (a\cos u \cos v, a \cos u \sin v, a \sin u) \]
	where \( u \in (-\pi, \pi) \) and \( v \in (0,2\pi) \).
	This parametrises the complement of a half great circle.
	Here,
	\[ \sigma_u = (-a \sin u \cos v, -a \sin u \sin v, a \cos u);\quad \sigma_v = (-a \cos u \sin v, a \cos u \cos v, 0) \]
	Hence,
	\[ E = a^2; \quad F = 0;\quad G = a^2 \cos^2 u \]
	which gives the first fundamental form as
	\[ a^2 \dd{u}^2 + a^2 \cos^2 u \dd{v}^2 \]
\end{example}
\begin{example}
	Consider the surface of revolution given by a curve
	\[ \eta(t) = (f(t),0,g(t)) \]
	rotated about the \( z \) axis.
	The resulting surface has parametrisation
	\[ \sigma(u,v) = (f(u) \cos v, f(u) \sin v, g(u)) \]
	Hence,
	\[ \sigma_u = (f_u \cos v, f_u \sin v, g_u);\quad \sigma_v = (-f \sin v, f \cos v, 0) \]
	which gives
	\[ (f_u^2 + g_u^2) \dd{u}^2 + f^2 \dd{v}^2 \]
\end{example}
\begin{example}
	Consider the cone with angle \( \arctan a \) to the vertical.
	For \( u > 0 \) and \( v \in (0,2\pi) \), we define
	\[ \sigma(u,v) = (au\cos v, au\sin v, u) \]
	The first fundamental form is
	\[ (1+a^2)\dd{u}^2 + a^2 u^2 \dd{v}^2 \]
	Consider cutting the cone along the line \( v = 0 \) and flattening it into a plane sector.
	The circumference of the sector is \( 2 \pi a \) and the radius is \( \sqrt{1+a^2} \), hence the angle traced out by the sector is \( \theta_0 = \frac{2 \pi a}{\sqrt{1+a^2}} \).
	We can parametrise this subset of the plane by
	\[ \sigma(r,\theta) = \qty(\sqrt{1+a^2} r\cos\qty(\frac{a\theta}{\sqrt{1+a^2}}), \sqrt{1+a^2} r\sin\qty(\frac{a\theta}{\sqrt{1+a^2}}), 0) \]
	for \( r > 0 \) and \( \theta \in (0,\theta_0) \).
	We can then check that the first fundamental form here is
	\[ (1+a^2) \dd{r}^2 + r^2 a^2 \dd{\theta}^2 \]
	which matches the first fundamental form for the cone itself.
	Hence the cone and the plane are locally isometric.
	However, the cone and plane are not globally isometric, since the two topological spaces are not homeomorphic, so no diffeomorphism that preserves lengths can be constructed.
\end{example}
\begin{lemma}
	Let \( \Sigma \) be a smooth surface in \( \mathbb R^3 \), and let \( p \in \Sigma \).
	Suppose we have two allowable parametrisations \( \sigma \colon V \to U \) and \( \sigma' \colon V' \to U \) for the same open neighbourhood of \( p \).
	The two parametrisations differ by a transition map \( F = {\sigma'}^{-1} \circ \sigma \) which is a diffeomorphism of open subsets of \( \mathbb R^2 \).
	There exist first fundamental forms for both parametrisations.
	Then,
	\[ \begin{pmatrix}
		E & F \\
		F & G
	\end{pmatrix} = (DF)^\transpose \begin{pmatrix}
		E' & F' \\
		F' & G'
	\end{pmatrix} (DF) \]
\end{lemma}
\begin{proof}
	By definition,
	\[ \begin{pmatrix}
		E & F \\
		F & G
	\end{pmatrix} = \begin{pmatrix}
		\sigma_u \cdot \sigma_u & \sigma_u \cdot \sigma_v \\
		\sigma_v \cdot \sigma_u & \sigma_v \cdot \sigma_v
	\end{pmatrix} = (D\sigma)^\transpose (D\sigma) \]
	Now, \( \sigma = \sigma' \circ F \) hence the result follows.
\end{proof}
If \( v,w \in \mathbb R^3 \), we have \( v \cdot w = \abs{v} \cdot \abs{w} \cdot \cos\theta \).
This allows us to deduce the angle \( \theta \) between two vectors given their dot product and lengths.
This can also be done when \( v,w \) are in the tangent plane \( T_p \Sigma \), and then we can express the angle in terms of the first fundamental form.
Let \( \sigma \) be an allowable parametrisation for \( \Sigma \) near \( p \), such that \( D\eval{\sigma}_0 \) evaluates to \( v \) at \( v_0 \) and \( w \) at \( w_0 \).
\[ \cos \theta = \frac{v \cdot w}{\abs{v} \cdot \abs{w}} = \frac{I(v_0, w_0)}{\sqrt{I(v_0,v_0)} \sqrt{I(w_0,w_0)}} \]
where \( I \) denotes the first fundamental form of \( \sigma \) at zero.
\begin{lemma}
	Let \( \Sigma \) be a smooth surface in \( \mathbb R^3 \), and let \( \sigma \colon V \to U \) be an allowable parametrisation of \( \Sigma \) near \( p \).
	Then \( \Sigma \) is \textit{conformal} if \( E = G \) and \( F = 0 \) in the first fundamental form.
\end{lemma}
