\subsection{Basic definitions}
\begin{definition}
	A \textit{topological surface} is a topological space \( \Sigma \) such that
	\begin{enumerate}
		\item for all points \( p \in \Sigma \), there exists an open neighbourhood \( p \in U \subset \Sigma \) such that \( U \) is homeomorphic to \( \mathbb R^2 \), or a disk \( D^2 \subset \mathbb R^2 \), with its usual Euclidean topology;
		\item \( \Sigma \) is Hausdorff and second countable.
	\end{enumerate}
\end{definition}
\begin{remark}
	\( \mathbb R^2 \) is homeomorphic to the open disk \( D(0,1) = \qty{ x \in \mathbb R^2 \colon \norm{x} < 1 } \).
	Recall that a space \( X \) is Hausdorff if two points \( p \neq q \in X \) have open neighbourhoods \( U, V \) such that \( U \cap V = \varnothing \).
	A space \( X \) is \textit{second countable} if it has a countable base; there exists a countable family of open sets \( U_i \), such that every open set is a union of some of the \( U_i \).

	Note that subspaces of Hausdorff and second countable spaces are also Hausdorff and second countable.
	In particular, Euclidean space \( \mathbb R^n \) is Hausdorff (as \( \mathbb R^n \) is a metric space) and second countable (consider the set of balls \( D(p,q) \) for points \( p \) with rational coordinates, and rational radii \( q \)).
	Hence, any subspace of \( \mathbb R^n \) is implicitly Hausdorff and second countable.
	These topological requirements are typically not the purpose of considering topological spaces, but they are occasionally technical requirements to prove interesting theorems.
\end{remark}
\begin{example}
	\( \mathbb R^2 \) is a topological surface.
	Any open subset of \( \mathbb R^2 \) is also a topological surface.
	For example, \( \mathbb R^2 \setminus \qty{0} \) and \( \mathbb R^2 \setminus \qty{(0,0)} \cup \qty{\qty(0, \frac{1}{n}) \colon n = 1, 2, \dots} \) are topological surfaces.
\end{example}
\begin{example}
	Let \( f \colon \mathbb R^2 \to \mathbb R \) be a continuous function.
	The graph of \( f \), denoted \( \Gamma_f \), is defined by
	\[
		\Gamma_f = \qty{(x,y,f(x,y)) \colon (x,y) \in \mathbb R^2}
	\]
	with the subspace topology when embedded in \( \mathbb R^3 \).
	Recall that a product topology \( X \times Y \) has the feature that \( f \colon Z \to X \times Y \) is continuous if and only if \( \pi_x \circ f \colon Z \to X \) and \( \pi_y \circ f \colon Z \to Y \) are continuous.
	Hence, any graph \( \Gamma \subseteq X \times Y \) is homeomorphic to \( X \) if \( f \) is continuous.
	Indeed, the projection \( \pi_x \) projects each point in the graph onto the domain.
	The function \( s \colon x \mapsto (x,f(x)) \) is continuous by the above.
	In particular, in our case, the graph \( \Gamma_f \) is homeomorphic to \( \mathbb R^2 \), which we know is a surface.
\end{example}
\begin{remark}
	As a topological surface, \( \Gamma_f \) is independent of the function \( f \).
	However, we will later introduce more ways to describe topological spaces that will ascribe new properties to \( \Gamma_f \) which do depend on \( f \).
\end{remark}
\begin{example}
	The sphere:
	\[
		S^2 = \qty{(x,y,z) \in \mathbb R^3 \colon x^2 + y^2 + z^2 = 1}
	\]
	is a topological surface, when using the subspace topology in \( \mathbb R^3 \).
	Consider the stereographic projection of \( S^2 \) onto \( \mathbb R^2 \) from the north pole \( (0,0,1) \).
	The projection satisfies \( \pi_+ \colon S^2 \setminus \qty{(0,0,1)} \) and
	\[
		(x,y,z) \mapsto \qty(\frac{x}{1-z}, \frac{y}{1-z})
	\]
	Certainly, \( \pi_+ \) is continuous, since we do not consider the point \( (0,0,1) \) to be in its domain.
	The inverse map is given by
	\[
		(u,v) \mapsto \qty(\frac{2u}{u^2+v^2+1}, \frac{2v}{u^2+v^2+1}, \frac{u^2+v^2-1}{u^2+v^2+1})
	\]
	This is also a continuous function.
	Hence \( \pi_+ \) is a homeomorphism.
	Similarly, we can construct the stereographic projection from the south pole, \( \pi_- \).
	This is a homeomorphism.
	Hence, every point in \( S^2 \) lies either in the domain of \( \pi_+ \) or \( \pi_- \), and hence sits in an open set \( S^2 \setminus \qty{(0,0,1)} \) or \( S^2 \setminus \qty{(0,0,-1)} \) which is homeomorphic to \( \mathbb R^2 \).
\end{example}
\begin{remark}
	\( S^2 \) is compact by the Heine-Borel theorem; it is a closed bounded set in \( \mathbb R^3 \).
\end{remark}
\begin{example}
	The real projective plane is a topological surface.
	The group \( \mathbb Z / 2 \) acts on \( S^2 \) by homeomorphisms via the \textit{antipodal map} \( a \colon S^2 \to S^2 \), mapping \( x \mapsto -x \).
	There exists a homeomorphism \( \mathbb Z / 2 \) to the group \( \mathrm{Homeo}(\mathbb S^2) \) of homeomorphisms of \( S^2 \), by mapping \( 1 + \mathbb Z \mapsto a \).
	We now define the real projective plane to be the quotient of \( S^2 \) given by identifying every point \( x \) with its image \( -x \) under \( a \).
	\[
		\mathbb R \mathbb P^2 = \faktor{S^2}{\mathbb Z/2} = \faktor{S^2}{\sim};\quad x \sim a(x)
	\]
	\begin{lemma}
		\( \mathbb R \mathbb P^2 \) bijects with the set of straight lines in \( \mathbb R^3 \) through the origin.
	\end{lemma}
	\begin{proof}
		Any line through the origin intersects \( S^2 \) exactly in a pair of antipodal points \( x, -x \).
		Similarly, pairs of antipodal points uniquely define a line through the origin.
	\end{proof}
	\begin{lemma}
		\( \mathbb R \mathbb P^2 \) is a topological surface.
	\end{lemma}
	\begin{proof}
		We must check that \( \mathbb R \mathbb P^2 \) is Hausdorff since it is constructed by a quotient, not a subspace.
		If \( X \) is a space and \( q \colon X \to Y \) is a quotient map, then by definition \( V \subset Y \) is open if and only if \( q^{-1}(V) \subset X \) is open.
		If \( [p] \neq [q] \in \mathbb R \mathbb P^2 \), then \( \pm p, \pm q \in S^2 \) are distinct antipodal pairs.
		We can therefore construct distinct open disks around \( p, q \) in \( S^2 \), and their antipodal images.
		These uniquely define open neighbourhoods of \( [p], [q] \), which are disjoint.

		Similarly, we can check that \( \mathbb R \mathbb P^2 \) is second countable.
		We know that \( S^2 \) is second countable, so let \( \mathcal U \) be a countable base for the topology on \( S^2 \).
		Without loss of generality, we can assert that for all sets \( U \in \mathcal U \), we have \( -U \in \mathcal U \).
		Let \( \overline{\mathcal U} \) be the family of open sets in \( \mathbb R \mathbb P^2 \) of the form \( q(U) \cup q(-U) \) for \( U \in \mathcal U \), where \( q \) is the quotient map.
		Now, if \( V \subseteq \mathbb R \mathbb P^2 \) is open, then by definition \( q^{-1}(V) \) is open in \( S^2 \) hence \( q^{-1}(V) \) contains some \( U \in \mathcal U \) and hence contains \( U \cup (-U) \).
		Hence \( \overline{\mathcal U} \) is a countable base for the quotient topology on \( \mathbb R \mathbb P^2 \).

		Finally, let \( p \in S^2 \) and \( [p] \in \mathbb R \mathbb P^2 \) its image.
		Let \( \overline D \) be a small (contained in an open hemisphere) closed disk, which is a neighbourhood of \( p \in S^2 \).
		The quotient map restricted to \( \overline D \), written \( \eval{q}_{\overline D} \colon \overline D \to q(\overline D) \subset \mathbb R \mathbb P^2 \), is a continuous function from a compact space to a Hausdorff space.
		Further, \( q \) is injective on \( \overline D \) since the disk was contained entirely in a single hemisphere.
		The topological inverse function theorem states that a continuous bijection from a compact space to a Hausdorff space is a homeomorphism.
		So \( \eval{q}_{\overline D} \) is a homeomorphism from \( \overline D \) to \( q(\overline D) \).
		This then induces the homeomorphism \( \eval{q}_{D} \) from the open disk \( D = {\overline D}^\circ \) to \( q(D) \).
		So by construction, \( [p] \in q(D) \); it has an open neighbourhood in \( \mathbb R \mathbb P^2 \) which is homeomorphic to an open disk, concluding the proof.
	\end{proof}
\end{example}
\begin{example}
	Let \( S^1 \) be the unit circle in \( \mathbb C \), and then we define the torus to be the product space \( S^1 \times S^1 \), with the subspace topology from \( \mathbb C^2 \) (which is identical to the product topology).
	\begin{lemma}
		The torus is a topological surface.
	\end{lemma}
	\begin{proof}
		Consider the map \( e \colon \mathbb R^2 \to S^1 \times S^1 \) defined by
		\[
			(s,t) \mapsto \qty(e^{2\pi i s}, e^{2 \pi i t})
		\]
		Note that this induces a map \( \hat e \) from \( \faktor{\mathbb R^2}{\mathbb Z^2} \), since \( e \) is constant under translations by \( \mathbb Z^2 \).

		% https://tikzcd.yichuanshen.de/#N4Igdg9gJgpgziAXAbVABwnAlgFyxMJZABgBpiBdUkANwEMAbAVxiRAB12BbOnACwBGAgAQAlAHoAmEAF9S6TLnyEUARnJVajFmwDK41cM54u8YftWz5IDNjwEiZVZvrNWiDuwBmAJzoBjYE4efiExKRkg7l5BEQAtCNlNGCgAc3giUF8ILiQyEBwIJElqVx0PAEcQagY6ARgGAAVFexUQHyxUvhwrLJ8cvOpCpHUtNzZWOT6BxBKCosRRsvdPPl5hSYoZIA
		\begin{center}
			\begin{tikzcd}
				\mathbb R^2 \arrow[d, "q"'] \arrow[r, "e"]             & S^1 \times S^1 \\
				\faktor{\mathbb R^2}{\mathbb Z^2} \arrow[ru, "\hat e"] &
			\end{tikzcd}
		\end{center}

		Under the quotient topology given by the quotient map \( q \), \( \faktor{\mathbb R^2}{\mathbb Z^2} \) is a topological space.
		The map \( [0,1]^2 \to \mathbb R^2 \to \faktor{\mathbb R^2}{\mathbb Z^2} \) is surjective, so \( \faktor{\mathbb R^2}{\mathbb Z^2} \) is compact.
		So \( \hat e \) is a continuous map from a compact space to a Hausdorff space, and \( \hat e \) is bijective, so \( \hat e \) is a homeomorphism.
		We already have that \( S^1 \times S^1 \) is compact and Hausdorff (as a closed and bounded set in \( \mathbb C^2 \)), so it suffices to show it is locally homeomorphic to \( \mathbb R^2 \).
		Let \( [p] = q(p) \in S^1 \times S^1 \), then we can choose a small disk \( \overline D(p) \) such that \( \overline D(p) \cap \qty(\overline D(p) + (n,m)) = \varnothing \) for non-zero \( (n,m) \in \mathbb Z^2 \).
		Hence \( \eval{e}_{\overline D(p)} \) is injective and \( \eval{q}_{\overline D(p)} \) is injective.
		Now, restricting to the open disk as before, we can find an open disk neighbourhood of \( [p] \).
		Since \( [p] \) was chosen arbitrarily, \( S^1 \times S^1 \) is a topological surface.
	\end{proof}
\end{example}
\begin{example}
	Let \( P \) be a planar Euclidean polygon, with oriented edges.
	We will pair the edges, and without loss of generality we will assume that paired edges have the same Euclidean length.
	\begin{center}
		\tikzfig{torus_polygon}
		\quad
		\tikzfig{other_square_polygon}
		\quad
		\tikzfig{hexagon}
	\end{center}
	We can assign letter names to each edge pair, and denote a polygon by the sequence of edges found when traversing in a clockwise orientation.
	The edge pair name is inverted if the edge is traversed in the reverse direction.
	Note the difference between the annotations on the first two shapes above, due to the reversed direction of the edge.
	If two edges \( e, \hat e \) are paired, this defines a unique Euclidean isometry from \( e \) to \( \hat e \) respecting the orientation, which will be written \( f_{e\hat e} \colon e \to \hat e \).
	The set of all such functions generate an equivalence relation on the polygon, identifying paired edges with each other.
	\begin{lemma}
		\( \faktor{P}{\sim} \), with the quotient topology, is a topological surface.
	\end{lemma}
	\begin{example}
		Consider the torus, defined here as \( T^2 = \faktor{[0,1]^2}{\sim} \).
		Let \( P \) be the polygon \( [0,1]^2 \).
		If \( p \) is in the interior of \( P \), then construct a sufficiently small disk that lies entirely within the interior.
		The quotient map is injective on the closure of the disk and is a homeomorphism on its interior.

		Let \( p \) be on an edge, but not a vertex.
		Let us say without loss of generality that \( p = (0,y_0) \sim (1,y_0) \).
		Let \( \delta \) be sufficiently small that the closed half-disks \( U, V \) centred on \( p \) with radius \( \delta \) do not intersect any vertices.
		Then we define a map from the union of the two half-disks to the disk \( B(0,\delta) \subseteq \mathbb R^2 \) via \( (x,y) \mapsto (x,y-y_0) \) or \( (x,y) \mapsto (x-1,y-y_0) \), which will be a bijective map.
		Recall the gluing lemma from Analysis and Topology: that if \( X = A \cup B \) is a union of closed subspaces, and \( f \colon A \to Y \), \( g \colon B \to Y \) are continuous and \( \eval{f}_{A \cap B} = \eval{g}_{A \cap B} \), they define a continuous map on \( X \).
		Let \( f_U, f_V \) be the maps on the half-disks \( U, V \).
		By the definition of the quotient topology, \( q \circ f_U \) and \( q \circ f_V \) are also continuous.
		On the overlapping area, the functions \( q \circ f_U \) and \( q \circ f_V \) agree.
		Hence, by the gluing lemma, we can construct a function \( f \colon U \cup V \to B(0, \delta) \).
		We can show that this is a homeomorphism using the usual process: pass to the closed disk, apply the topological inverse function theorem, then apply the result to the interior.
		If \( [p] \in T^2 \) lies on the image of an edge in \( [0,1]^2 \), it has indeed a neighbourhood homeomorphic to a disk.

		Now it suffices to consider points \( p \) on a vertex.
		All four vertices of the square are identified to the same point in the torus.
		A neighbourhood of each vertex can be identified with a quarter-disk in \( \mathbb R^2 \).
		We can repeatedly apply the gluing lemma to construct the whole disk \( B(0,\delta) \subseteq \mathbb R^2 \) and complete the argument as before.

		Thus, \( \faktor{[0,1]^2}{\sim} \) is a topological surface.
	\end{example}
	We can generalise this proof to an arbitrary planar Euclidean polygon \( P \), such as the hexagon above.
	The equivalence relation \( x \sim f_{e \hat e}(x) \) induces an equivalence relation on the vertices of \( P \), by considering the images of the vertices under all \( f_{e\hat e} \).
	However, it is not necessarily the case that an equivalence class of vertices contains exactly four vertices, so quarter-disks are not necessarily applicable.
	Again, there are three types of point:
	\begin{itemize}
		\item interior points, for which a neighbourhood not intersecting the boundary is chosen;
		\item points on edges, for which a corresponding point exists and two half-disks can be glued to form the neighbourhood; and
		\item points on vertices.
		      For this case, all vertices of the polygon have a neighbourhood which is a sector of a circle.
		      Let there be \( r \) vertices in a given equivalence class.
		      Let \( \alpha \) be the sum of the angles of the sectors in a given class.
		      Any sector can be identified with a given sector in the disk \( B(0,\delta) \subseteq \mathbb R^2 \), which we will choose to have angle \( \alpha / r \).
		      Then, we can glue each sector together in \( \mathbb R^2 \), compatibly with the orientations of the edges and arrows, inducing a neighbourhood which is locally homeomorphic to a disk.
		      If \( r = 1 \), we have an equivalence class comprising a single vertex, which gives a single sector.
		      For \( r \) to be one, the two edges attached to this vertex must be paired and have the same direction (either both inwards or outwards from the vertex).
		      This quotient space is simply a cone, which is homeomorphic to a disk as required.
	\end{itemize}
	We can also show that the quotient space is Hausdorff and second countable.
	By construction, two distinct points in the quotient space can be separated by open neighbourhoods by selecting a sufficiently small radius such that the disks considered in the derivation above are disjoint.
	For second countability, consider
	\begin{itemize}
		\item disks in the interior of \( P \) with rational centres and radii;
		\item for each edge of \( P \), consider an isometry \( e \to [0, \ell] \) where \( \ell \) is the length of \( e \), taking disks on \( e \) which are centred at rational values in \( [0,\ell] \); and
		\item for each vertex, consider disks centred at these vertices with rational radii.
	\end{itemize}
\end{example}
\begin{example}
	Given topological surfaces \( \Sigma_1, \Sigma_2 \) we can remove an open disk from each and glue the resulting circles.
	Explicitly, we form a quotient relation on the disjoint union of the surfaces with the disks removed.
	This process is known as forming the \textit{connect sum} of the surfaces, written \( \Sigma_1 \connect \Sigma_2 \).
	Typically, the information about where the disks were removed from is discarded when considering the connect sum.
	The connect sum of two topological surfaces is a topological surface.
	\begin{example}
		Consider the following octagon.
		\ctikzfig{double_torus_polygon}
		The associated quotient space \( \faktor{P}{\sim} \) can be seen to be homeomorphic to a surface with two holes, known as a double torus.
		All vertices are identified as the same vertex in the quotient space.
		We can cut the octagon along a diagonal, leaving two topological surfaces which are homeomorphic to a torus.
		\begin{center}
			\tikzfig{double_torus_polygon_expanded} \( \mapsto \) \tikzfig{torus_polygon_with_loop}
		\end{center}
		Thus, the connect sum of the two half-octagons are the connect sum of two toruses.
	\end{example}
	\begin{example}
		Consider the following square.
		\ctikzfig{rp2_polygon}
		This is homeomorphic to the real projective plane \( \mathbb R \mathbb P^2 \).
		This is because we identify points on the boundary with their antipodes, when interpreting the square as the closed disk \( B(0,1) \).
		The real projective plane was constructed by identifying points on the unit sphere with their antipodes.
		Thus, we can construct a homeomorphism by considering only points in the upper hemisphere (taking antipodes as required), and then orthographically projecting onto the \( xy \) plane.
		Under this transformation, points on the boundary are identified with their antipodes as required.
	\end{example}
\end{example}

\subsection{Subdivisions}
\begin{definition}
	A \textit{subdivision} of a compact topological surface \( \Sigma \) comprises
	\begin{enumerate}
		\item a finite subset \( V \subseteq \Sigma \) of vertices;
		\item a finite subset \( E = \qty{e_i \colon [0,1] \to \Sigma} \) which are continuous injections and pairwise disjoint except perhaps at the endpoints;
		\item such that each connected component of the complement of \( V \cup E \) in \( \Sigma \) is homeomorphic to an open disk, and each such component will be called a face.
		      In particular, the boundary of each face has boundary inside the union of the edges and the vertices.
	\end{enumerate}
	We say that a subdivision is a \textit{triangulation} if each closed face (closure of a face) contains exactly three edges, and two closed faces meet either at exactly one edge or not at any edges.
\end{definition}
\begin{example}
	A cube displays a subdivision of \( S^2 \).
	A tetrahedron displays a triangulation of \( S^2 \).
\end{example}
\begin{example}
	We can display subdivisions of surfaces constructed from polygons.
	\ctikzfig{torus_polygon}
	This is a subdivision of a torus with one edge, two edges, and one face.
	We can construct additional subdivisions of a torus, for example:
	\begin{center}
		\tikzfig{torus_polygon_subdivided} \quad \tikzfig{torus_polygon_triangulated}
	\end{center}
	The first of these examples is not a triangulation, since the two faces meet in more than one edge.
	The second is a triangulation.
\end{example}
\begin{remark}
	The following is a very degenerate subdivision of \( S_2 \).
	\ctikzfig{s2_degenerate}
	This has one vertex, no edges, and one face.
\end{remark}

\subsection{Euler classification}
\begin{definition}
	The \textit{Euler characteristic} of a subdivision is
	\[
		\# V - \# E + \# F
	\]
\end{definition}
\begin{theorem}
	\begin{enumerate}
		\item Every compact topological surface has a subdivision (and indeed triangulations).
		\item The Euler characteristic is invariant under choice of subdivision, and is topologically invariant.
	\end{enumerate}
	Hence, we might say that a surface has a particular Euler characteristic, without referring to subdivisions.
	We write this \( \chi(\Sigma) \).
\end{theorem}
No proof will be given.
\begin{example}
	The Euler characteristic of \( S^2 \) is \( \chi(S^2) = 2 \).
	For the torus, \( \chi(T^2) = 0 \).
	If \( \Sigma_1, \Sigma_2 \) are compact surfaces, then the connect sum \( \Sigma_1 \connect \Sigma_2 \) can be constructed by removing a face of a triangulation, then gluing together the boundary circles (three edges) in a way that matches the edges.
	Then the connect sum inherits a subdivision, and we can find that it has Euler characteristic \( \chi(\Sigma_1 \connect \Sigma_2) = \chi(\Sigma_1) + \chi(\Sigma_2) - 2 \), where the remaining term corresponds to the two faces that were removed; the changes of three vertices and three edges cancel each other.
	In particular, a surface \( \Sigma_g \) with \( g \) holes can be written \( \bigconnect_{i=1}^g T_2 \), so \( \chi(\Sigma_g) = 2 - 2g \).
	We call \( g \) the \textit{genus} of \( \Sigma \).
\end{example}
\begin{remark}
	It is not trivial to prove part (i).
	For part (ii), note that subdivisions can be converted into triangulation by constructing triangle fans.
	\ctikzfig{triangle_fan}
	Triangulations can be related by local moves, such as
	\ctikzfig{triangulation_local_move}
	It is easy to check that both of these moves do not change the Euler characteristic.
	However, it is hard to make this argument rigorous, and it does not give much explanation for why the result is true.
	In Part II Algebraic Topology, a more advanced definition of the Euler characteristic is given, which admits a more elegant proof.
\end{remark}
