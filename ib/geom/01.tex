\subsection{Surfaces}
\begin{definition}
	A \textit{topological surface} is a topological space \( \Sigma \) such that
	\begin{enumerate}[(i)]
		\item for all points \( p \in \Sigma \), there exists an open neighbourhood \( p \in U \subset \Sigma \) such that \( U \) is homeomorphic to \( \mathbb R^2 \), or a disk \( D^2 \subset \mathbb R^2 \), with its usual Euclidean topology;
		\item \( \Sigma \) is Hausdorff and second countable.
	\end{enumerate}
\end{definition}
\begin{remark}
	\( \mathbb R^2 \) is homeomorphic to the open disk \( D(0,1) = \qty{ x \in \mathbb R^2 \colon \norm{x} < 1 } \).
	Recall that a space \( X \) is Hausdorff if two points \( p \neq q \in X \) have open neighbourhoods \( U, V \) such that \( U \cap V = \varnothing \).
	A space \( X \) is \textit{second countable} if it has a countable base; there exists a countable family of open sets \( U_i \), such that every open set is a union of some of the \( U_i \).
	
	Note that subspaces of Hausdorff and second countable spaces are also Hausdorff and second countable.
	In particular, Euclidean space \( \mathbb R^n \) is Hausdorff (as \( \mathbb R^n \) is a metic space) and second countable (consider the set of balls \( D(p,q) \) for points \( p \) with rational coordinates, and rational radii \( q \)).
	Hence, any subspace of \( \mathbb R^n \) is implicitly Hausdorff and second countable.
	These topological requirements are typically not the purpose of considering topological spaces, but they are occasionally technical requirements to prove interesting theorems.
\end{remark}
\begin{example}
	\( \mathbb R^2 \) is a topological surface.
	Any open subset of \( \mathbb R^2 \) is also a topological surface.
	For example, \( \mathbb R^2 \setminus \qty{0} \) and \( \mathbb R^2 \setminus \qty{(0,0)} \cup \qty{\qty(0, \frac{1}{n}) \colon n = 1, 2, \dots} \) are topological surfaces.
\end{example}
\begin{example}
	Let \( f \colon \mathbb R^2 \to \mathbb R \) be a continuous function.
	The graph of \( f \), denoted \( \Gamma_f \), is defined by
	\[ \Gamma_f = \qty{(x,y,f(x,y)) \colon (x,y) \in \mathbb R^2} \]
	with the subspace topology when embedded in \( \mathbb R^3 \).
	Recall that a product topology \( X \times Y \) has the feature that \( f \colon Z \to X \times Y \) is continuous if and only if \( \pi_x \circ f \colon Z \to X \) and \( \pi_y \circ f \colon Z \to Y \) are continuous.
	Hence, any graph \( \Gamma \subseteq X \times Y \) is homeomorphic to \( X \) if \( f \) is continuous.
	Indeed, the projection \( \pi_x \) projects each point in the graph onto the domain.
	The function \( s \colon x \mapsto (x,f(x)) \) is continuous by the above.
	In particular, in our case, the graph \( \Gamma_f \) is homeomorphic to \( \mathbb R^2 \), which we know is a surface.
\end{example}
\begin{remark}
	As a topological surface, \( \Gamma_f \) is independent of the function \( f \).
	However, we will later introduce more ways to describe topological spaces that will ascribe new properties to \( \Gamma_f \) which do depend on \( f \).
\end{remark}
\begin{example}
	The sphere:
	\[ S^2 = \qty{(x,y,z) \in \mathbb R^3 \colon x^2 + y^2 + z^2 = 1} \]
	is a topological surface, when using the subspace topology in \( \mathbb R^3 \).
	Consider the stereographic projection of \( S^2 \) onto \( \mathbb R^2 \) from the north pole \( (0,0,1) \).
	The projection satisfies \( \pi_+ \colon S^2 \setminus \qty{(0,0,1)} \) and
	\[ (x,y,z) \mapsto \qty(\frac{x}{1-z}, \frac{y}{1-z}) \]
	Certainly, \( \pi_+ \) is continuous, since we do not consider the point \( (0,0,1) \) to be in its domain.
	The inverse map is given by
	\[ (u,v) \mapsto \qty(\frac{2u}{u^2+v^2+1}, \frac{2v}{u^2+v^2+1}, \frac{u^2+v^2-1}{u^2+v^2+1}) \]
	This is also a continuous function.
	Hence \( \pi_+ \) is a homeomorphism.
	Similarly, we can construct the stereographic projection from the south pole, \( \pi_- \).
	This is a homeomorphism.
	Hence, every point in \( S^2 \) lies either in the domain of \( \pi_+ \) or \( \pi_- \), and hence sits in an open set \( S^2 \setminus \qty{(0,0,1)} \) or \( S^2 \setminus \qty{(0,0,-1)} \) which is homeomorphic to \( \mathbb R^2 \).
\end{example}
\begin{remark}
	\( S^2 \) is compact by the Heine-Borel theorem; it is a closed bounded set in \( \mathbb R^3 \).
\end{remark}
\begin{example}
	The real projective plane is a topological surface.
	The group \( \mathbb Z / 2 \) acts on \( S^2 \) by homeomorphisms via the \textit{antipodal map} \( a \colon S^2 \to S^2 \), mapping \( x \mapsto -x \).
	There exists a homeomorphism \( \mathbb Z / 2 \) to the group \( \mathrm{Homeo}(\mathbb S^2) \) of homeomorphisms of \( S^2 \), by mapping \( 1 + \mathbb Z \mapsto a \).
	We now define the real projective plane to be the quotient of \( S^2 \) given by identifying every point \( x \) with its image \( -x \) under \( a \).
	\[ \mathbb R \mathbb P^2 = \frac{S^2}{\mathbb Z/2} = \frac{S^2}{\sim};\quad x \sim a(x) \]
	\begin{lemma}
		\( \mathbb R \mathbb P^2 \) bijects with the set of straight lines in \( \mathbb R^3 \) through the origin.
	\end{lemma}
	\begin{proof}
		Any line through the origin intersects \( S^2 \) exactly in a pair of antipodal points \( x, -x \).
		Similarly, pairs of antipodal points uniquely define a line through the origin.
	\end{proof}
	\begin{lemma}
		\( \mathbb R \mathbb P^2 \) is a topological surface.
	\end{lemma}
	\begin{proof}
		We must check that \( \mathbb R \mathbb P^2 \) is Hausdorff since it is constructed by a quotient, not a subspace.
		If \( X \) is a space and \( q \colon X \to Y \) is a quotient map, then by definition \( V \subset Y \) is open if and only if \( q^{-1}(V) \subset X \) is open.
		If \( [p] \neq [q] \in \mathbb R \mathbb P^2 \), then \( \pm p, \pm q \in S^2 \) are distinct antipodal pairs.
		We can therefore construct distinct open disks around \( p, q \) in \( S^2 \), and their antipodal images.
		These uniquely define open neighbourhoods of \( [p], [q] \), which are disjoint.
	\end{proof}
\end{example}
