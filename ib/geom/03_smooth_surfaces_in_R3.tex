\subsection{???}
Recall that if \( V \subseteq \mathbb R^n \) and \( V' \subseteq \mathbb R^m \), then \( f \colon V \to V' \) is smooth if it is infinitely differentiable.
\begin{definition}
	If \( Z \) is an arbitrary subset of \( \mathbb R^n \), we say that a continuous function \( f \colon Z \to \mathbb R^m \) is smooth at \( p \in Z \) if there exists an open ball \( p \in B \subseteq \mathbb R^n \) and a smooth map \( F \colon B \to \mathbb R^m \) which extends \( f \) such that they agree on \( B \cap Z \).
	In other words, \( f \) is locally the restriction of a smooth map defined on an open set.
\end{definition}
\begin{definition}
	Let \( X \subseteq \mathbb R^n \) and \( Y \subseteq \mathbb R^m \).
	We say that \( X \) and \( Y \) are \textit{diffeomorphic} if there exists a continuous function \( f \colon X \to Y \) such that \( f \) is a smooth homeomorphism with smooth inverse.
\end{definition}
\begin{definition}
	A \textit{smooth surface in \( \mathbb R^3 \)} is a subspace of \( \mathbb R^3 \) such that for all points \( p \in \Sigma \), there exists an open subset \( p \in U \subseteq \Sigma \) that is diffeomorphic to an open set in \( \mathbb R^2 \).
	In other words, for all \( p \in \Sigma \), there exists an open ball \( p \in B \subseteq \mathbb R^3 \) such that if \( U = B \cap \Sigma \) and there exists a map \( f \colon B \to V \subseteq \mathbb R^2 \) such that \( \eval{f}_U \colon U \to V \) is a homeomorphism, and the inverse map \( V \to U \subseteq \Sigma \subseteq \mathbb R^3 \) is smooth.
	% TODO: rephrase this
\end{definition}
\begin{theorem}
	For a subset \( \Sigma \subseteq \mathbb R^3 \), the following are equivalent.
	\begin{enumerate}[(a)]
		\item \( \Sigma \) is a smooth surface in \( \mathbb R^3 \);
		\item \( \Sigma \) is locally the graph of a smooth function, over one of the three co-ordinate planes: for all \( p \in \Sigma \) there exists an open ball \( p \in B \subseteq \mathbb R^3 \) and an open set \( V \subseteq \mathbb R^2 \) such that
			\[ \Sigma \cap B = \qty{(x, y, g(x,y)) \colon g \colon V \to \mathbb R \text{ smooth}} \]
			or one of the other co-ordinate planes;
		\item \( \Sigma \) is locally cut out by a smooth function: for all \( p \in \Sigma \) there exists an open ball \( p \in B \subseteq \mathbb R^3 \) and a smooth function \( f \colon B \to \mathbb R \) such that
			\[ \Sigma \cap B = f^{-1}(a);\quad D \eval{f}_x \neq 0 \]
			for all \( x \in B \);
		\item \( \Sigma \) is locally the image of an \textit{allowable} parametrisation: if \( p \in \Sigma \) there exists an open set \( p \in U \subseteq \mathbb R^3 \) and \( \sigma \colon V \to U \) where \( V \subseteq \mathbb R^2 \) is open and \( U \subseteq \Sigma \subseteq \mathbb R^3 \) is open in \( \Sigma \) such that \( \sigma \) is a homeomorphism and \( D \eval{\sigma}_x \) has rank 2 for all \( x \in V \).
	\end{enumerate}
\end{theorem}
\begin{remark}
	Part (b) implies that if \( \Sigma \) is a smooth surface in \( \mathbb R^3 \), each \( p \in \Sigma \) belongs to a chart \( (U, \varphi) \) where \( \varphi \) is (the restriction of) one of the three co-ordinate plane projections \( \pi_{xy}, \pi_{yz}, \pi_{xz} \) from \( \mathbb R^3 \) to \( \mathbb R^2 \).
	Consider the transition map between two such charts.
	If the two charts are based on the same projection such as \( \pi_{xy} \), then the transition map is the identity.
	If they are based on different projections \( \pi_{xy} \) and \( \pi_{yz} \), then the transition map is
	\[ (x,y) \mapsto (x,y,g(x,y)) \mapsto (y,g(x,y)) \]
	which has inverse
	\[ (y,z) \mapsto (h(y,z),y,z) \mapsto (h(y,z),y) \]
	Hence all of the transition maps between such charts involve projection maps and the smooth maps involved in defining \( \Sigma \) as a graph.
	This gives \( \Sigma \) the structure of an abstract smooth surface.
\end{remark}
Some of the relations given in the above theorem are easy to prove, but others come as a result of the inverse function theorem.

\subsection{Inverse function theorem}
\begin{theorem}
	Let \( U \subseteq \mathbb R^n \) be open, and \( f \colon U \to \mathbb R^n \) be continuously differentiable.
	Let \( p \in U \) and \( f(p) = q \).
	Suppose \( D\eval{f}_p \) is invertible.
	Then there is an open neighbourhood \( V \) of \( q \) and a differentiable map \( g \colon V \to \mathbb R^n \) and \( g(q) = p \) with image an open neighbourhood \( U' \subseteq U \) of \( p \) such that \( f \circ g = \id_V \).
	If \( f \) is smooth, then \( g \) is also.
\end{theorem}
