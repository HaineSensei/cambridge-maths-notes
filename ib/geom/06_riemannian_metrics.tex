\subsection{???}
\begin{definition}
	Let \( V \subseteq \mathbb R^2 \) be an open set.
	An \textit{(abstract) Riemannian metric} is a smooth map from \( V \) to the set of positive definite symmetric bilinear forms, given by
	\[ v \mapsto \begin{pmatrix}
		E(v) & F(v) \\
		F(v) & G(v)
	\end{pmatrix} \]
	such that \( E > 0 \), \( G > 0 \), \( EG - F^2 > 0 \).
	The image of this map can be viewed as an open subset of \( \mathbb R^4 \).
\end{definition}
If \( v \) is a vector at \( p \in V \), we can compute its infinitesimal length by
\[ \norm{v}^2 = v^\transpose \begin{pmatrix}
	E(v) & F(v) \\
	F(v) & G(v)
\end{pmatrix} v \]
Thus, if \( \gamma \colon [a,b] \to V \) is smooth,
\[ \mathrm{length}(\gamma) = \int_a^b \qty( E \dot u^2 + 2F \dot u \dot v + G \dot v^2 )^{\frac{1}{2}} \dd{t} \]
where \( \gamma(t) = (u(t),v(t)) \).
\begin{definition}
	Let \( \Sigma \) be an abstract smooth surface, so \( \Sigma = \bigcup_{i \in I} U_i \) for open sets \( U_i \), with charts \( \varphi_i \colon U_i \to V_i \subseteq \mathbb R^2 \) which are homeomorphisms, and with smooth transition maps \( \varphi_i \varphi_j^{-1} \colon \varphi_j(U_i \cap U_j) \to \varphi_i(U_i \cap U_j) \).
	A \textit{Riemannian metric} on \( \Sigma \), usually called \( g \) or \( \dd{s}^2 \), is a choice of Riemannian metric in the above sense on each \( V_i \), which are compatible in the following sense.
	Let \( \sigma = \varphi_i^{-1} \) and \( \widetilde \sigma = \varphi_j^{-1} \) for some \( i,j \), and define \( f = \widetilde \sigma^{-1} \circ \sigma \).
	Then we require
	\[ (Df)^\transpose \begin{pmatrix}
		\widetilde E & \widetilde F \\
		\widetilde F & \widetilde G
	\end{pmatrix} (Df) = \begin{pmatrix}
		E & F \\
		F & G
	\end{pmatrix} \]
	So \( Df \) defines an isometry from an open set in the chart \( (U, \varphi(U) = V) \) to one in \( \qty(\widetilde U, \widetilde \varphi\qty(\widetilde U) = \widetilde V) \).
\end{definition}
This compatibility condition is the transition law for first fundamental forms for smooth surfaces in \( \mathbb R^3 \).
\begin{example}
	Recall the torus \( T^2 = \faktor{\mathbb R^2}{\mathbb Z^2} \).
	\begin{center}
		\tikzfig{torus_polygon}
	\end{center}
	We have an atlas of charts for which the transition maps are the restrictions of translations of open subsets of \( \mathbb R^2 \).
	For each \( V_i \subseteq \mathbb R^2 \), we associate the natural Euclidean metric \( \dd{u}^2 + \dd{v}^2 \).
	If \( f \) is a translation, \( Df \) is the identity, and so
	\[ (Df)^\transpose I (Df) = I \]
	holds trivially.
	So this gives a global Riemannian metric on \( T^2 \).
	This metric is flat, since it is locally isometric to \( \mathbb R^2 \) at all points.

	Conversely, consider the torus of revolution embedded in \( \mathbb R^3 \).
	As a compact smooth surface in \( \mathbb R^3 \), it must contain an elliptic point.
	Hence, the flat Riemannian metric described above is not the same (up to isometry) as the metric obtained by any possible embedding of the torus in \( \mathbb R^3 \).

	The real projective plane \( \mathbb R \mathbb P^2 \) admits a Riemannian metric with constant curvature \( +1 \).
	We have constructed a smooth atlas for \( \mathbb R \mathbb P^2 \) where the charts were of the form \( (U, \varphi) \), with \( U = q \hat U \) and \( q \colon S^2 \to \mathbb R \mathbb P^2 \) the quotient map, \( \hat U \subseteq S^2 \) open and contained within an open hemisphere, and \( \varphi \colon U \colon U \to V \subseteq \mathbb R^2 \) is given by \( \hat \varphi \circ \eval{q^{-1}}_U \) and \( \hat \varphi \colon \hat U \to V \) a chart on \( S^2 \).
	The transition maps for this atlas were found to be locally the identity, or induced from the antipodal map.
	The antipodal map from \( S^2 \) to \( S^2 \) is an isometry, so both types of transition maps preserve the usual round metric on \( S^2 \).

	In the first example sheet, we consider the Klein bottle.
	This has an atlas such that all transition maps are either translations or translations composed with a reflection.
	These preserve the flat metric in \( \mathbb R^2 \), so the Klein bottle inherits a flat Riemannian metric.
	The Klein bottle and \( \mathbb R \mathbb P^2 \) are not embedded in \( \mathbb R^3 \), so we could not construct a `non-abstract' Riemannian metric.
\end{example}
\begin{definition}
	Let \( (\Sigma_1, g_1), (\Sigma_2, g_2) \) be abstract smooth surfaces with abstract Riemannian metrics.
	A diffeomorphism \( f \colon \Sigma_1 \to \Sigma_2 \) is an \textit{isometry} if it preserves the lengths of all curves, where lengths are taken with respect to these abstract Riemannian metrics.
\end{definition}
\begin{example}
	If \( (\Sigma_2, g_2) \) is given, and \( f \colon \Sigma_1 \to \Sigma_2 \) is a diffeomorphism, we can equip \( \Sigma_1 \) with a metric known as the \textit{pullback} metric \( g_1 = f^\star g_2 \) that gives that \( f \) is an isometry.
\end{example}

\subsection{The length metric}
\begin{definition}
	Let \( (\Sigma, g) \) be a connected abstract smooth surface with an abstract Riemannian metric.
	The \textit{length metric} is defined by
	\[ d_g(p,q) = \inf_\gamma L(\gamma) \]
	where \( \gamma \) varies over piecewise smooth paths in \( \Sigma \) from \( p \) to \( q \), and \( L \) is length computed using \( g \).
\end{definition}
\begin{proposition}
	Let \( (\Sigma, g) \) be a connected abstract smooth surface with an abstract Riemannian metric.
	Then \( d_g \) is indeed a metric, and \( d_g \) induces a topology on \( \Sigma \) that agrees with the given topology.
\end{proposition}
\begin{proof}
	Let \( p, q \in \Sigma \).
	We will show that there exists some piecewise smooth path \( \gamma \) from \( p \) to \( q \), so \( d_g(p,q) \) is well-defined and finite.
	Connected surfaces are path-connected.
	There exists a continuous path \( \gamma \) and a finite set of charts \( (U_i, \varphi_i) \) with associated parametrisations \( \sigma_i = \varphi_i^{-1} \colon V_i \to U_i \subset \Sigma \) such that \( \Im \gamma \subseteq \bigcup_{i=1}^N U_i \).
	Consider points
	\[ p = x_0 \in U_1, x_1 \in U_1 \cap U_2, x_2 \in U_2 \in U_3, \dots, q = x_N \in U_N \]
	Smooth paths in \( V_i \) from \( \varphi_i(x_i) \) to \( \varphi_{i+1}(x_{i+1}) \) exist, since smooth paths between two points in \( \mathbb R^2 \) exist.
	Since the atlas is smooth, being a smooth path in some \( U_i \) is the same as being smooth in \( U_{i+1} \) whenever \( U_i \) and \( U_{i+1} \) intersect, since the transition maps are smooth.
	So \( p,q \in \Sigma \) are joined by some piecewise smooth path.

	For any piecewise smooth path from \( p \) to \( q \) there exists the inverse path parametrised in the opposite direction, which has the same length.
	We can also concatenate paths from \( p \) to \( q \) and from \( q \) to \( r \), with length equal to the sum of the lengths.
	In both cases, the new paths are piecewise smooth.
	This then implies that \( d_g \) is symmetric, and satisfies the triangle inequality.

	To show \( d_g \) is a metric, it now suffices to show that \( d_g(p,q) = 0 \) implies \( p = q \), since the converse is trivial.
	Let \( p \in \Sigma \) and fix a chart \( (U,\varphi) \) at \( p \).
	Without loss of generality let \( V = B(0,1) \), and \( \varphi(p) = 0 \).
	If \( q \neq p \in \Sigma \), there exists \( \varepsilon > 0 \) such that \( q \not\in \varphi^{-1} \qty(\overline{B(0,\varepsilon)}) \).
	Suppose \( \gamma \colon [0,1] \to \Sigma \) is a piecewise smooth path from \( p \) to \( q \).
	Certainly, \( \gamma \) must escape the disk \( \varphi^{-1} \qty(\overline{B(0,\varepsilon)}) \), since it must reach \( q \).
	Length along paths is additive, so by the triangle inequality, it suffices to show that there exists \( \delta > 0 \) such that \( d_g(p,r) > \delta \) for all \( r \in \partial \varphi^{-1} \qty(\overline{B(0,\varepsilon)}) = \varphi^{-1} \qty{ \text{circle of radius } \varepsilon } \).
	The data on the Riemannian metric \( g \) includes the non-degenerate symmetric bilinear form \( \begin{pmatrix}
		E_z & F_z \\
		F_z & G_z
	\end{pmatrix} \) for all \( z \in \overline{B(0,\varepsilon)} \subseteq V \).
	We also have the usual Euclidean inner product on the disk, \( \begin{pmatrix}
		1 & 0 \\
		0 & 1
	\end{pmatrix} \).
	For all \( z \in \overline{B(0,\varepsilon)} \), these matrices are positive definite.
	Since \( \overline{B(0,\varepsilon)} \) is compact, there exists \( \delta > 0 \) such that
	\( \begin{pmatrix}
		E_z - \delta & F_z \\
		F_z & G_z - \delta
	\end{pmatrix} \)
	is still positive definite for all \( z \in \overline{B(0,\varepsilon)} \).
	In other words, the determinant \( EG-F^2 > 0 \) for all \( z \in \overline{B(0,\varepsilon)} \), which is compact, so it is bounded below by some positive number.
	Hence, \( \mathrm{length}_g(\hat\gamma) \geq \mathrm{length}_{\delta \cdot \mathrm{Euclidean}}(\hat \gamma) \) for any \( \hat \gamma \) contained withing \( \overline{B(o,\varepsilon)} \).
	Taking \( \hat\gamma = \varphi \qty[ \gamma \cap \varphi^{-1}\qty(\overline{B(o,\varepsilon)}) ] \), which is the part of \( \gamma \) in \( \overline{B(0,\varepsilon)} \) with respect to the chart, we have that \( \mathrm{length}_{\delta \cdot \mathrm{Euclidean}}(\hat \gamma) \geq \delta \varepsilon \), so \( d_g(p,q) \geq \delta\varepsilon \).
\end{proof}
\begin{remark}
	The last step of the argument for the proof above, comparing the inner products \( \begin{pmatrix}
		E_z & F_z \\
		F_z & G_z
	\end{pmatrix} \) and \( \begin{pmatrix}
		1 & 0 \\
		0 & 1
	\end{pmatrix} \) can be modified to show that \( d_g \) induces a topology on \( \Sigma \) that agrees with the given topology, which is given by local homeomorphisms to \( \mathbb R^2 \) everywhere.
\end{remark}

\subsection{The hyperbolic metric}
\begin{definition}
	Let
	\[ D = B(0,1) = \qty{z \in \mathbb C \colon \abs{z} < 1} \]
	The abstract Riemannian metric \( g_{\mathrm{hyp}} \) on \( D \) is given by
	\[ \frac{4(\dd{u}^2 + \dd{v}^2)}{(1-u^2-v^2)^2} = \frac{4 \abs{\dd{z}}^2}{\qty(1 - \abs{z}^2)^2} \]
	Since there is only one chart, this holds for all of \( D \).
	In particular, if \( \gamma \colon [0,1] \to D \) is smooth, then
	\[ L_{g_{\mathrm{hyp}}}(\gamma) = 2 \int_0^1 \frac{\abs{\dot\gamma(t)}}{1 - \abs{\gamma(t)}^2} \dd{t} \]
	If \( \gamma(t) = (u(t), v(t)) \), we can write
	\[ L(\gamma) = 2 \int_0^1 \frac{\qty(\dot u^2 + \dot v^2)^{\frac{1}{2}}}{1 - u^2 - v^2} \dd{t} \]
\end{definition}
This is very similar to a first fundamental form with \( E = G = \frac{4}{(1-u^2-v^2)^2} \) and \( F = 0 \), but we do not claim that this fundamental form arises from an embedding in \( \mathbb R^3 \).

Note that the flat metric on \( \mathbb R^2 \) and the usual round metric on \( S^2 \) have large and transitive isometry groups.
We will show that this metric also induces a large symmetry group, which is induced by the M\"obius group.
Recall that
\[ \Mob = \qty{z \mapsto \frac{az+b}{cz+d} \colon \begin{pmatrix}
	a & b \\
	c & d
\end{pmatrix} \in GL(2,\mathbb C)} \acts \mathbb C \cup \qty{\infty} \]
\begin{lemma}
	The subgroup of the M\"obius group that preserves \( D \),
	\[ \Mob(D) = \qty{T \in \Mob \colon T(D) = D} \]
	is also given by
	\[ \Mob(D) = \qty{z \mapsto e^{i\theta} \frac{z-a}{1-\overline a z} \colon \abs{a} < 1} = \qty{\begin{pmatrix}
		a & b \\
		-\overline b & \overline a
	\end{pmatrix} \in \Mob \colon \abs{a}^2 + \abs{b}^2 = 1} \]
\end{lemma}
\begin{proof}
	Note that
	\begin{align*}
		\abs{\frac{z-a}{1-\overline a z}} = 1 &\iff (z-a)(\overline z - \overline a) = (1 - \overline a z)(1 - a \overline z) \\
		&\iff z\overline z - a\overline z - \overline a z + a \overline a = 1 - \overline a z - a \overline z + a \overline a z \overline z \\
		&\iff \abs{z}^2\qty(1 - \abs{a}^2) = 1 - \abs{a}^2 \\
		&\iff \abs{z} = 1
	\end{align*}
	So these maps of the form
	\[ z \mapsto e^{i\theta} \frac{z-a}{1-\overline a z} \]
	do indeed preserve the unit circle, and \( 0 \in D \) is mapped to \( a \in D \).
	Hence, it preserves the entire disk.
\end{proof}
