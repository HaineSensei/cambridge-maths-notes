\subsection{Definitions}
\begin{definition}
	Let \( R \) be a ring.
	A \textit{module over \( R \)} is a triple \( (M, +, \cdot) \) consisting of a set \( M \) and two operations \( + \colon M \times M \to M \) and \( \cdot \colon R \times M \to M \), that satisfy
	\begin{enumerate}[(i)]
		\item \( (M, +) \) is an abelian group with identity \( 0 = 0_M \);
		\item \( (r_1 + r_2) \cdot m = r_1 \cdot m + r_2 \cdot m \);
		\item \( r \cdot (m_1 + m_2) = r \cdot m_1 + r \cdot m_2 \);
		\item \( r_1 \cdot (r_2 \cdot m) = (r_1 \cdot r_2) \cdot m \);
		\item \( 1_R \cdot m = m \);
	\end{enumerate}
\end{definition}
\begin{remark}
	Closure is implicitly required by the types of the \( + \) and \( \cdot \) operations.
\end{remark}
\begin{example}
	A module over a field is precisely a vector space.

	A \( \mathbb Z \)-module is precisely the same as an abelian group, since
	\[ \cdot \colon \mathbb Z \times A \to A;\quad n \cdot a = \begin{cases}
		\underbrace{a + \dots + a}_{n \text{ times}} & \text{if } n > 0 \\
		0 & n = \text{if } 0 \\
		-\qty(\underbrace{a + \dots + a}_{-n \text{ times}}) & \text{if } n < 0
	\end{cases} \]

	Let \( F \) be a field, and \( V \) be a vector space over \( F \).
	Let \( \alpha \colon V \to V \) be an endomorphism.
	We can turn \( V \) into an \( F[X] \)-module by
	\[ \cdot \colon F[X] \times V \to V;\quad f \cdot v = (f(\alpha))(v) \]
	Note that the structure of the \( F[X] \)-module depends on the choice of \( \alpha \).
	We can write \( V = V_\alpha \) to disambiguate.

	For any ring \( R \), we can consider \( R^n \) as an \( R \)-module via
	\[ r \cdot (r_1, \dots, r_n) = (r \cdot r_1, \dots, r \cdot r_n) \]
	In particular, the case \( n = 1 \) shows that any ring \( R \) can be considered an \( R \)-module where the scalar multiplication in the ring and the module agree.

	For an ideal \( I \triangleleft R \), we can regard \( I \) as an \( R \)-module, since \( I \) is preserved under multiplication by elements in \( R \).
	The quotient ring \( \faktor{R}{I} \) is also an \( R \)-module, defining multiplication as \( r \cdot (s+I) = rs + I \).

	Let \( \varphi \colon R \to S \) be a ring homomorphism.
	Then any \( S \)-module can be regarded as an \( R \)-module.
	We define \( r \cdot m = \varphi(r) \cdot m \).
	In particular, this applies when \( R \) is a subring of \( S \), and \( \varphi \) is the inclusion map.
	So any module over a ring can be viewed as a module over any subring.
\end{example}
\begin{definition}
	Let \( M \) be an \( R \)-module.
	Then \( N \subseteq M \) is an \textit{\( R \)-submodule of \( M \)}, written \( N \leq M \), if \( (N, +) \leq (M, +) \), and for all \( rn \in N \) for all \( r \in R \) and \( n \in N \).
\end{definition}
\begin{example}
	By considering \( R \) as an \( R \)-module, a subset of \( R \) is an \( R \)-submodule if and only if it is an ideal.
	If \( R = F \) is a field, this definition corresponds to the definition of a vector subspace.
\end{example}
\begin{definition}
	Let \( N \leq M \) be \( R \)-modules.
	Then, the \textit{quotient} \( \faktor{M}{N} \) is defined as the quotient of groups under addition, and with scalar multiplication defined as \( r \cdot (m + N) = rm + N \).
	This is well-defined, since \( N \) is preserved under scalar multiplication.
	This makes \( \faktor{M}{N} \) an \( R \)-module.
\end{definition}
\begin{remark}
	Submodules are analogous both to subrings and to ideals.
\end{remark}
\begin{definition}
	Let \( M, N \) be \( R \)-modules.
	Then \( f \colon M \to N \) is a \textit{\( R \)-module homomorphism} if it is a homomorphism of \( (M, +) \) and \( (N, +) \), and scalar multiplication is preserved: \( f(r \cdot m) = r \cdot f(m) \).
	An \textit{\( R \)-module isomorphism} is an \( R \)-module homomorphism that is a bijection.
\end{definition}
\begin{example}
	If \( R = F \) is a field, \( F \)-module homomorphisms are exactly linear maps.
\end{example}
\begin{theorem}
	Let \( f \colon M \to N \) be an \( R \)-module homomorphism.
	Then
	\begin{enumerate}[(i)]
		\item \( \ker f = \qty{m \in M \colon f(m) = 0} \leq M \);
		\item \( \Im f = \qty{f(m) \in N \colon m \in M} \leq N \);
		\item \( \faktor{M}{\ker f} \cong \Im f \).
	\end{enumerate}
\end{theorem}
\begin{theorem}
	Let \( A, B \leq M \) be \( R \)-submodules.
	Then
	\begin{enumerate}[(i)]
		\item \( A + B = \qty{a + b \colon a \in A, b \in B} \leq M \);
		\item \( A \cap B \leq M \);
		\item \( \faktor{A}{A \cap B} \cong \faktor{A + B}{B} \).
	\end{enumerate}
\end{theorem}
\begin{theorem}
	For \( N \leq L \leq M \) are \( R \)-submodules, then
	\[ \faktor{M/N}{L/N} \cong \faktor{M}{L} \]
\end{theorem}
For \( N \leq M \), there is a correspondence between submodules of \( \faktor{M}{N} \) and submodules of \( M \) containing \( N \).
These isomorphism theorems can be proved exactly as before.
Note that these results apply to vector spaces; for example, the first isomorphism theorem immediately gives the rank-nullity theorem.

\subsection{Finitely generated modules}
\begin{definition}
	Let \( M \) be an \( R \)-module.
	If \( m \in M \), then we write \( Rm = \qty{rm \colon r \in R} \).
	This is an \( R \)-submodule of \( M \), known as the submodule \textit{generated by \( m \)}.

	If \( A, B \leq M \), we can define \( A + B = \qty{a + b \colon a \in A, b \in B} \), known as the \textit{sum of submodules}.
	In particular, this sum is commutative.
\end{definition}
\begin{definition}
	A module \( M \) is \textit{finitely generated} if it is the sum of finitely many submodules generated by a single element.
	In other words, \( M = Rm_1 + \dots + Rm_n \).
\end{definition}
This is the analogue of finite dimensionality in linear algebra.
\begin{lemma}
	An \( R \)-module \( M \) is finitely generated if and only if there exists a surjective \( R \)-module homomorphism \( f \colon R^n \to M \) for some \( n \).
\end{lemma}
\begin{proof}
	If \( M \) is finitely generated, we have \( M = Rm_1 + \dots + Rm_n \).
	We define \( f \colon R^n \to M \) by \( (r_1, \dots, r_n) \mapsto = r_1 m_1 + \dots + r_n m_n \).
	This is surjective.

	Conversely, suppose such a surjective homomorphism \( f \) exists.
	Let \( e_i = (0, \dots, 1, \dots, 0) \) be the element of \( R^n \) with all entries zero except for 1 in the \( i \)th place.
	Let \( m_i = f(e_i) \).
	Then, since \( f \) is surjective, any element \( m \in M \) is contained in the image of \( f \), so is of the form \( f(r_1, \dots, r_n) = r_1 m_1 + \dots + r_n m_n \).
\end{proof}
\begin{corollary}
	Any quotient by a submodule of a finitely generated module is finitely generated.
\end{corollary}
\begin{proof}
	Let \( N \leq M \), where \( M \) is finitely generated.
	Then there exists a surjective \( R \)-module homomorphism \( f \colon R^n \to M \).
	Then \( f \circ q \), where \( q \) is the quotient map, is also a surjective homomorphism.
	So \( \faktor{M}{N} \) is finitely generated.
\end{proof}
\begin{example}
	It is not always the case that a submodule of a finitely generated module is finitely generated.
	Let \( R \) be a non-Noetherian ring, and \( I \) an ideal in \( R \) that is not finitely generated (in the ring sense).
	\( R \) is a finitely generated \( R \)-module, since \( R1 = R \).
	\( I \) is a submodule of \( R \), which is not finitely generated (in the module sense).
\end{example}
\begin{remark}
	If \( R \) is Noetherian, it is always the case that submodules of finitely generated \( R \)-modules are finitely generated.
	This will be shown on the example sheets.
\end{remark}

\subsection{Torsion}
\begin{definition}
	Let \( M \) be an \( R \)-module.
	\begin{enumerate}[(i)]
		\item \( m \in M \) is \textit{torsion} if there exists \( 0 \neq r \in R \) such that \( rm = 0 \);
		\item \( M \) is a \textit{torsion module} if every element is torsion;
		\item \( M \) is a \textit{torsion-free module} if 0 is the only torsion element.
	\end{enumerate}
\end{definition}
\begin{example}
	The torsion elements in a \( \mathbb Z \)-module (which is an abelian group) are precisely the elements of finite order.
	If \( F \) is a field, any \( F \)-module is torsion-free.
\end{example}

\subsection{Direct sums}
\begin{definition}
	Let \( M_1, \dots, M_n \) be \( R \)-modules.
	Then the \textit{direct sum} of \( M_1, \dots, M_n \), written \( M_1 \oplus \dots \oplus M_n \), is the set \( M_1 \times \dots \times M_n \), with the operations of addition and scalar multiplication defined componentwise.
	We can show that the direct sum of (finitely many) \( R \)-modules is an \( R \)-module.
\end{definition}
\begin{example}
	\( R^n = R \oplus \dots \oplus R \), where we take the direct sum of \( n \) copies of \( R \).
\end{example}
\begin{lemma}
	Let \( M = \bigoplus_{i=1}^n M_i \), and for each \( M_i \), let \( N_i \leq M_i \).
	Then \( N = \bigoplus_{i=1}^n N_i \) is a submodule of \( M \).
	Further,
	\[ \faktor{M}{N} = \faktor{\bigoplus_{i=1}^n M_i}{\bigoplus_{i=1}^n N_i} \cong \bigoplus_{i=1}^n \faktor{M_i}{N_i} \]
\end{lemma}
\begin{proof}
	First, we can see that this \( N \) is a submodule.
	Applying the first isomorphism theorem to the surjective \( R \)-module homomorphism \( M \to \bigoplus_{i=1}^n \faktor{M_i}{N_i} \) given by \( (m_1, \dots, m_n) \mapsto (m_1 + N_1, \dots, m_n + N_n) \), the result follows as required, since the kernel is \( N \).
\end{proof}

\subsection{Free modules}
\begin{definition}
	Let \( m_1, \dots, m_n \in M \).
	The set \( \qty{m_1, \dots, m_n} \) is \textit{independent} if \( \sum_{i=1}^n r_i m_i = 0 \) implies that the \( r_i \) are all zero.
\end{definition}
\begin{definition}
	A subset \( S \subseteq M \) \textit{generates \( M \) freely} if:
	\begin{enumerate}[(i)]
		\item \( S \) generates \( M \), so for all \( m \in M \), we can find finitely many entries \( s_i \) and coefficients \( r_i \) such that \( m = \sum_{i=1}^k r_i s_i \);
		\item any function \( \psi \colon S \to N \), where \( N \) is an \( R \)-module, extends to an \( R \)-module homomorphism from \( \theta \colon M \to N \).
	\end{enumerate}
\end{definition}
\begin{remark}
	In (ii), such an extension \( \theta \) is always unique if it exists, by (i).
\end{remark}
\begin{definition}
	An \( R \)-module \( M \) freely generated by some subset \( S \subseteq M \) is called \textit{free}.
	We say that \( S \) is a \textit{free basis} for \( M \).
\end{definition}
\begin{remark}
	Free bases in the study of modules are analogous to bases in linear algebra.
	All vector spaces are free modules, but not all modules are free.
\end{remark}
\begin{proposition}
	For a finite subset \( S = \qty{m_1, \dots, m_n} \subseteq M \), the following are equivalent.
	\begin{enumerate}[(i)]
		\item \( S \) generates \( M \) freely;
		\item \( S \) generates \( M \), and \( S \) is independent;
		\item every element of \( M \) can be written uniquely as \( r_1 m_1 + \dots + r_n m_n \) for some \( r_i \in R \);
		\item the \( R \)-module homomorphism \( R^n \to M \) given by \( (r_1, \dots, r_n) \mapsto r_1 m_1 + \dots + r_n m_n \) is bijective, so is an isomorphism.
	\end{enumerate}
\end{proposition}
\begin{proof}
	Not all implications are shown, but they are similar to arguments found in Part IB Linear Algebra.
	We show (i) implies (ii).
	Let \( S \) generate \( M \) freely.
	Suppose \( S \) is not independent.
	Then there exist \( r_i \) such that \( \sum_{i=1}^n r_i m_i = 0 \) but not all \( r_i \) are zero.
	Let \( r_j \neq 0 \).
	Since \( S \) generates \( M \) freely, consider the module homomorphism \( \psi \colon S \to R \) given by
	\[ \psi(m_i) = \begin{cases}
		1 & \text{if } i = j \\
		0 & \text{otherwise}
	\end{cases} \]
	Then
	\[ 0 = \theta(0) = \theta\qty(\sum_{i=1}^n r_i m_i) = \sum_{i=1}^n r_i \theta(m_i) = r_j \neq 0 \]
	This is a contradiction, so \( S \) is independent.

	To show (ii) implies (iii), it suffices to show uniqueness.
	If there exist two ways to write an element as a linear combination, consider their difference to find a contradiction from (ii).

	We can show (iii) implies (i).
	Then it remains to show (iii) and (iv) are equivalent.
\end{proof}
\begin{example}
	A non-trivial finite abelian group is not a free \( \mathbb Z \)-module.

	The set \( \qty{2,3} \) generates \( \mathbb Z \) as a \( \mathbb Z \)-module.
	This is not a free basis, since they are not independent: \( 2 \cdot 3 - 3 \cdot 2 = 0 \).
	However, it contains no subset that is a free basis.
	This is different to vector spaces, where we can always construct a basis from a subset of a spanning set.
\end{example}
\begin{proposition}[invariance of dimension]
	Let \( R \) be a nonzero ring.
	If \( R^m \cong R^n \) as \( R \)-modules, then \( m = n \).
\end{proposition}
\begin{proof}
	Let \( I \triangleleft R \), and \( M \) an \( R \)-module.
	We define \( IM = \qty{\sum a_i m_i \colon a_i \in I, m_i \in M} \).
	Since \( I \) is an ideal, we can show that \( IM \) is a submodule of \( M \).
	The quotient module \( \faktor{M}{IM} \) is an \( R \)-module, but we can also show that it is an \( \faktor{R}{I} \)-module, by defining scalar multiplication as
	\[ (r+I) \cdot (m+IM) = (r \cdot m + IM) \]
	We can check that this is well-defined; this follows from the fact that for \( b \in I \), \( b \cdot (m + IM) = bm + IM \), but \( b \in I \) so \( bm \in IM \).

	Now, suppose that \( R^m \cong R^n \).
	Then let \( I \triangleleft R \) be a maximal ideal in \( R \).
	We can prove the existence of such an ideal under the assumption of the axiom of choice, and in particular using Zorn's lemma.
	By the above discussion, we find an isomorphism of \( \faktor{R}{I} \)-modules
	\[ \qty(\faktor{R}{I})^m \cong \faktor{R^M}{IR^m} \cong \faktor{R^n}{IR^n} \cong \qty(\faktor{R}{I})^n \]
	This is an isomorphism of vector spaces over \( \faktor{R}{I} \) which is a field, since \( I \) is maximal.
	Hence, using the corresponding result from linear algebra, \( n = m \).
\end{proof}

\subsection{Row and column operations}
We will assume that \( R \) is a Euclidean domain in this subsection, and let \( \varphi \) be a Euclidean function for \( R \).
We will consider an \( m \times n \) matrix with entries in \( R \).
\begin{definition}
	The \textit{elementary row operations} on a matrix are
	\begin{enumerate}[(i)]
		\item add \( \lambda \in R \) multiplied by the \( j \)th row to the \( i \)th row, where \( i \neq j \);
		\item swap the \( i \)th row and the \( j \)th row;
		\item multiply the \( i \)th row by \( u \in R^\times \).
	\end{enumerate}
	Each of these operations can be realised by left-multiplication by some \( m \times m \) matrix.
	These operations are all invertible, so their matrices are all invertible.
\end{definition}
We can define elementary column operations in an analogous way, using right-multiplication by an \( n \times n \) matrix instead.
\begin{definition}
	Two \( m \times n \) matrices \( A, B \) are \textit{equivalent} if there exists a sequence of elementary row and column operations that transforms one matrix into the other.
	If they are equivalent, then there exist invertible matrices \( P, Q \) such that \( B = QAP \).
\end{definition}
\begin{definition}
	A \( k \times k \) \textit{minor} of an \( m \times n \) matrix \( A \) is the determinant of a \( k \times k \) submatrix of \( A \), which is a matrix of \( A \) produced by removing \( m-k \) rows and \( n-k \) columns.

	The \( k \)th Fitting ideal \( \mathrm{Fit}_k(A) \triangleleft R \) is the ideal generated by the \( k \times k \) minors of \( A \).
\end{definition}
\begin{lemma}
	The \( k \)th Fitting ideal of a matrix is invariant under elementary row and column operations.
\end{lemma}
\begin{proof}
	It suffices by symmetry to show that the elementary row operations do not change the Fitting ideal.
	For the first elementary row operation on a matrix \( A \), suppose we add \( \lambda \in R \) multiplied by the \( j \)th row to the \( i \)th row, yielding a matrix \( A' \).
	In particular, \( a_{ik} \mapsto a_{ik} + \lambda a_{jk} \) for all \( k \).
	Let \( C \) be a \( k \times k \) submatrix of \( A \) and \( C' \) the corresponding submatrix of \( A' \).

	If row \( i \) was not chosen in \( C \), then \( C \) and \( C' \) are the same matrix.
	Hence the corresponding minors are equal.
	If row \( i \) and row \( j \) were both chosen in \( C \), we have that \( C, C' \) differ by a row operation.
	Since the determinant is invariant under this elementary row operations, the corresponding minors are equal.

	If row \( i \) was chosen but row \( j \) was not chosen, by expanding the determinant along the \( i \)th row, we find
	\[ \det C' = \det C + \lambda \det D \]
	where we can show that \( D \) is a \( k \times k \) submatrix of \( A \) that includes row \( j \) but not row \( i \).
	By definition, \( \det D \in \mathrm{Fit}_k(A) \) and \( \det C \in \mathrm{Fit}_k(A) \), so certainly \( \det C' \in \mathrm{Fit}_k(A) \).
	Hence \( \mathrm{Fit}_k(A') \subseteq \mathrm{Fit}_k(A) \).
	By the invertibility of the elementary row operations, \( \mathrm{Fit}_k(A') \supseteq \mathrm{Fit}_k(A) \).

	The proofs for the other elementary row operations are left as an exercise.
\end{proof}

\subsection{Smith normal form}
\begin{theorem}
	An \( m \times n \) matrix \( A = (a_{ij}) \) over a Euclidean domain \( R \) is equivalent to a matrix of the form
	\[ \begin{pmatrix}
		d_1 \\
		& \ddots \\
		& & d_t \\
		& & & 0 \\
		& & & & \ddots \\
		& & & & &
	\end{pmatrix};\quad d_1 \mid d_2 \mid \dots \mid d_t \]
	The \( d_i \) are known as \textit{invariant factors}, and they are unique up to associates.
\end{theorem}
\begin{proof}
	If \( A = 0 \), the matrix is already in Smith normal form.
	Otherwise, we can swap columns and rows such that \( a_{11} \neq 0 \).
	We will reduce \( \varphi(a_{11}) \) as much as possible until it divides every other element in the matrix, using the following algorithm.

	If \( a_{11} \nmid a_{1j} \) for some \( j \geq 2 \), then \( a_{1j} = q a_{11} + r \) where \( q, r \in R \) and \( \varphi(r) < \varphi(a_{11}) \).
	We can subtract \( q \) multiplied by column 1 from column \( j \).
	Swapping such columns leaves \( a_{11} = r \).
	If \( a_{11} \nmid a_{i1} \) for some \( i \geq 2 \), then repeat the above process using row operations.
	Now, \( a_{11} \mid a_{ij} \) for all \( i,j \).
	These steps are repeated until \( a_{11} \) divides all entries of the first row and first column.
	This algorithm will always terminate, for example because the Euclidean function takes values in \( \mathbb Z_{\geq 0} \) and \( \varphi(a_{11}) \) strictly decreases in each iteration.

	Now, we can subtract multiples of the first row and column from the others to give
	\[ A = \begin{pmatrix}
		a_{11} & 0 & \cdots & 0 \\
		0 \\
		\vdots & & A' \\
		0
	\end{pmatrix} \]
	If \( a_{11} \nmid a_{ij} \) for \( i,j \geq 2 \), then add the \( i \)th row to the first row.
	There is now an element in the first row that does \( a_{11} \) not divide.
	We can then perform column operations as above to decrease \( \varphi(a_{11}) \).
	We will then restart the algorithm.
	After finitely many steps, this algorithm will terminate and \( a_{11} \) will divide all elements \( a_{ij} \) of the matrix.
	\[ A = \begin{pmatrix}
		a_{11} & 0 & \cdots & 0 \\
		0 \\
		\vdots & & A' \\
		0
	\end{pmatrix};\quad a_{11} \equiv d_1 \mid a_{ij} \]
	We can now apply the algorithm to \( A' \), since column and row operations not including the first row or column do not change whether \( a_{11} \mid a_{ij} \).

	We now demonstrate uniqueness of the invariant factors.
	Suppose \( A \) has Smith normal form with invarant factors \( d_i \) where \( d_1 \mid \dots \mid d_t \).
	Then, for all \( k \), \( \mathrm{Fit}_k(A) \) can be evaluated in Smith normal form by invariance of the Fitting ideal under row and column operations.
	Hence \( \mathrm{Fit}_k(A) = (d_1 d_2 \cdots d_k) \triangleleft R \).
	Thus, the product \( d_1 \cdots d_k \) depends only on \( A \), and is unique up to associates.
	Cancelling, we can see that each \( d_i \) depends only on \( A \), up to associates.
\end{proof}
\begin{example}
	Consider the matrix over \( \mathbb Z \) given by
	\[ A = \begin{pmatrix}
		2 & -1 \\
		1 & 2
	\end{pmatrix} \]
	Using elementary row and column operations,
	\[ \begin{pmatrix}
		2 & -1 \\
		1 & 2
	\end{pmatrix} \xrightarrow{c_1 \mapsto c_1 + c_2} \begin{pmatrix}
		1 & -1 \\
		3 & 2
	\end{pmatrix} \xrightarrow{c_2 \mapsto c_1 + c_2} \begin{pmatrix}
		1 & 0 \\
		3 & 5
	\end{pmatrix} \xrightarrow{r_2 \mapsto -3r_1 + r_2} \begin{pmatrix}
		1 & 0 \\
		0 & 5
	\end{pmatrix} \]
	This is in Smith normal form as \( 1 \mid 5 \).

	Alternatively, \( (d_1) = (2, -1, 1, 2) = (1) \).
	So \( d_1 = \pm 1 \).
	Further, \( (d_1 d_2) = (\det A) = (5) \).
	So \( d_1 d_2 = \pm 5 \) and hence \( d_2 = \pm 5 \).
\end{example}

\subsection{The structure theorem}
\begin{lemma}
	Let \( R \) be a Euclidean domain with Euclidean function \( \varphi \) (or, indeed, a principal ideal domain).
	Any submodule of the free module \( R^m \) is generated by at most \( m \) elements.
\end{lemma}
\begin{proof}
	Let \( N \leq R^m \).
	Consider
	\[ I = \qty{r \in R \colon \exists r_2, \dots, r_m \in R,\, (r,r_2, \dots, r_m) \in N} \]
	Since \( N \) is a submodule, this is an ideal.
	Since \( R \) is a principal ideal domain, \( I = (a) \) for some \( a \in R \).
	Let \( n = (a, a_2, \dots, a_m) \in N \).
	For \( (r_1, \dots, r_m) \in N \), we have \( r_1 = ra \) for some \( r \).
	Hence \( (r_1, \dots, r_m) - rn = \)
	
	TODO: finish this proof
\end{proof}
