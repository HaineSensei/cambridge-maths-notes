\subsection{Definitions}
\begin{definition}
	Let \( R \) be a ring.
	A \textit{module over \( R \)} is a triple \( (M, +, \cdot) \) consisting of a set \( M \) and two operations \( + \colon M \times M \to M \) and \( \cdot \colon R \times M \to M \), that satisfy
	\begin{enumerate}[(i)]
		\item \( (M, +) \) is an abelian group with identity \( 0 = 0_M \);
		\item \( (r_1 + r_2) \cdot m = r_1 \cdot m + r_2 \cdot m \);
		\item \( r \cdot (m_1 + m_2) = r \cdot m_1 + r \cdot m_2 \);
		\item \( r_1 \cdot (r_2 \cdot m) = (r_1 \cdot r_2) \cdot m \);
		\item \( 1_R \cdot m = m \);
	\end{enumerate}
\end{definition}
\begin{remark}
	Closure is implicitly required by the types of the \( + \) and \( \cdot \) operations.
\end{remark}
\begin{example}
	A module over a field is precisely a vector space.

	A \( \mathbb Z \)-module is precisely the same as an abelian group, since
	\[ \cdot \colon \mathbb Z \times A \to A;\quad n \cdot a = \begin{cases}
		\underbrace{a + \dots + a}_{n \text{ times}} & \text{if } n > 0 \\
		0 & n = \text{if } 0 \\
		-\qty(\underbrace{a + \dots + a}_{-n \text{ times}}) & \text{if } n < 0
	\end{cases} \]

	Let \( F \) be a field, and \( V \) be a vector space over \( F \).
	Let \( \alpha \colon V \to V \) be an endomorphism.
	We can turn \( V \) into an \( F[X] \)-module by
	\[ \cdot \colon F[X] \times V \to V;\quad f \cdot v = (f(\alpha))(v) \]
	Note that the structure of the \( F[X] \)-module depends on the choice of \( \alpha \).
	We can write \( V = V_\alpha \) to disambiguate.

	For any ring \( R \), we can consider \( R^n \) as an \( R \)-module via
	\[ r \cdot (r_1, \dots, r_n) = (r \cdot r_1, \dots, r \cdot r_n) \]
	In particular, the case \( n = 1 \) shows that any ring \( R \) can be considered an \( R \)-module where the scalar multiplication in the ring and the module agree.

	For an ideal \( I \triangleleft R \), we can regard \( I \) as an \( R \)-module, since \( I \) is preserved under multiplication by elements in \( R \).
	The quotient ring \( \faktor{R}{I} \) is also an \( R \)-module, defining multiplication as \( r \cdot (s+I) = rs + I \).

	Let \( \varphi \colon R \to S \) be a ring homomorphism.
	Then any \( S \)-module can be regarded as an \( R \)-module.
	We define \( r \cdot m = \varphi(r) \cdot m \).
	In particular, this applies when \( R \) is a subring of \( S \), and \( \varphi \) is the inclusion map.
	So any module over a ring can be viewed as a module over any subring.
\end{example}
\begin{definition}
	Let \( M \) be an \( R \)-module.
	Then \( N \subseteq M \) is an \textit{\( R \)-submodule of \( M \)}, written \( N \leq M \), if \( (N, +) \leq (M, +) \), and for all \( rn \in N \) for all \( r \in R \) and \( n \in N \).
\end{definition}
\begin{example}
	By considering \( R \) as an \( R \)-module, a subset of \( R \) is an \( R \)-submodule if and only if it is an ideal.
	If \( R = F \) is a field, this definition corresponds to the definition of a vector subspace.
\end{example}
\begin{definition}
	Let \( N \leq M \) be \( R \)-modules.
	Then, the \textit{quotient} \( \faktor{M}{N} \) is defined as the quotient of groups under addition, and with scalar multiplication defined as \( r \cdot (m + N) = rm + N \).
	This is well-defined, since \( N \) is preserved under scalar multiplication.
	This makes \( \faktor{M}{N} \) an \( R \)-module.
\end{definition}
\begin{remark}
	Submodules are analogous both to subrings and to ideals.
\end{remark}
\begin{definition}
	Let \( M, N \) be \( R \)-modules.
	Then \( f \colon M \to N \) is a \textit{\( R \)-module homomorphism} if it is a homomorphism of \( (M, +) \) and \( (N, +) \), and scalar multiplication is preserved: \( f(r \cdot m) = r \cdot f(m) \).
	An \textit{\( R \)-module isomorphism} is an \( R \)-module homomorphism that is a bijection.
\end{definition}
\begin{example}
	If \( R = F \) is a field, \( F \)-module homomorphisms are exactly linear maps.
\end{example}
\begin{theorem}
	Let \( f \colon M \to N \) be an \( R \)-module homomorphism.
	Then
	\begin{enumerate}[(i)]
		\item \( \ker f = \qty{m \in M \colon f(m) = 0} \leq M \);
		\item \( \Im f = \qty{f(m) \in N \colon m \in M} \leq N \);
		\item \( \faktor{M}{\ker f} \cong \Im f \).
	\end{enumerate}
\end{theorem}
\begin{theorem}
	Let \( A, B \leq M \) be \( R \)-submodules.
	Then
	\begin{enumerate}[(i)]
		\item \( A + B = \qty{a + b \colon a \in A, b \in B} \leq M \);
		\item \( A \cap B \leq M \);
		\item \( \faktor{A}{A \cap B} \cong \faktor{A + B}{B} \).
	\end{enumerate}
\end{theorem}
\begin{theorem}
	For \( N \leq L \leq M \) are \( R \)-submodules, then
	\[ \faktor{M/N}{L/N} \cong \faktor{M}{L} \]
\end{theorem}
For \( N \leq M \), there is a correspondence between submodules of \( \faktor{M}{N} \) and submodules of \( M \) containing \( N \).
These isomorphism theorems can be proved exactly as before.
Note that these results apply to vector spaces; for example, the first isomorphism theorem immediately gives the rank-nullity theorem.

\subsection{Finitely generated modules}
\begin{definition}
	Let \( M \) be an \( R \)-module.
	If \( m \in M \), then we write \( Rm = \qty{rm \colon r \in R} \).
	This is an \( R \)-submodule of \( M \), known as the submodule \textit{generated by \( m \)}.

	If \( A, B \leq M \), we can define \( A + B = \qty{a + b \colon a \in A, b \in B} \), known as the \textit{sum of submodules}.
	In particular, this sum is commutative.
\end{definition}
\begin{definition}
	A module \( M \) is \textit{finitely generated} if it is the sum of finitely many submodules generated by a single element.
	In other words, \( M = Rm_1 + \dots + Rm_n \).
\end{definition}
This is the analogue of finite dimensionality in linear algebra.
\begin{lemma}
	An \( R \)-module \( M \) is finitely generated if and only if there exists a surjective \( R \)-module homomorphism \( f \colon R^n \to M \) for some \( n \).
\end{lemma}
\begin{proof}
	If \( M \) is finitely generated, we have \( M = Rm_1 + \dots + Rm_n \).
	We define \( f \colon R^n \to M \) by \( (r_1, \dots, r_n) \mapsto = r_1 m_1 + \dots + r_n m_n \).
	This is surjective.

	Conversely, suppose such a surjective homomorphism \( f \) exists.
	Let \( e_i = (0, \dots, 1, \dots, 0) \) be the element of \( R^n \) with all entries zero except for 1 in the \( i \)th place.
	Let \( m_i = f(e_i) \).
	Then, since \( f \) is surjective, any element \( m \in M \) is contained in the image of \( f \), so is of the form \( f(r_1, \dots, r_n) = r_1 m_1 + \dots + r_n m_n \).
\end{proof}
\begin{corollary}
	Any quotient by a submodule of a finitely generated module is finitely generated.
\end{corollary}
\begin{proof}
	Let \( N \leq M \), where \( M \) is finitely generated.
	Then there exists a surjective \( R \)-module homomorphism \( f \colon R^n \to M \).
	Then \( f \circ q \), where \( q \) is the quotient map, is also a surjective homomorphism.
	So \( \faktor{M}{N} \) is finitely generated.
\end{proof}
\begin{example}
	It is not always the case that a submodule of a finitely generated module is finitely generated.
	Let \( R \) be a non-Noetherian ring, and \( I \) an ideal in \( R \) that is not finitely generated (in the ring sense).
	\( R \) is a finitely generated \( R \)-module, since \( R1 = R \).
	\( I \) is a submodule of \( R \), which is not finitely generated (in the module sense).
\end{example}
\begin{remark}
	If \( R \) is Noetherian, it is always the case that submodules of finitely generated \( R \)-modules are finitely generated.
	This will be shown on the example sheets.
\end{remark}

\subsection{Torsion}
\begin{definition}
	Let \( M \) be an \( R \)-module.
	\begin{enumerate}[(i)]
		\item \( m \in M \) is \textit{torsion} if there exists \( 0 \neq r \in R \) such that \( rm = 0 \);
		\item \( M \) is a \textit{torsion module} if every element is torsion;
		\item \( M \) is a \textit{torsion-free module} if 0 is the only torsion element.
	\end{enumerate}
\end{definition}
\begin{example}
	The torsion elements in a \( \mathbb Z \)-module (which is an abelian group) are precisely the elements of finite order.
	If \( F \) is a field, any \( F \)-module is torsion-free.
\end{example}
