\subsection{Introduction}
If \( K \trianglelefteq G \), then studying the groups \( K \) and \( \faktor{G}{K} \) give information about \( G \) itself.
This approach is available only if \( G \) has nontrivial normal subgroups.
It therefore makes sense to study groups with no normal subgroups, since they cannot be decomposed into simpler structures in this way.
\begin{definition}
	A group \( G \) is \textit{simple} if \( \qty{1} \) and \( G \) are its only normal subgroups.
\end{definition}
By convention, we do not consider the trivial group to be a simple group.
This is analogous to the fact that we do not consider one to be a prime.
\begin{lemma}
	Let \( G \) be an abelian group.
	\( G \) is simple if and only if \( G \cong C_p \) for some prime \( p \).
\end{lemma}
\begin{proof}
	Certainly \( C_p \) is simple by Lagrange's theorem.
	Conversely, since \( G \) is abelian, all subgroups are normal.
	Let \( 1 \neq g \in G \).
	Then \( \genset g \trianglelefteq G \).
	Hence \( \genset g = G \) by simplicity.
	If \( G \) is infinite, then \( G \cong \mathbb Z \), which is not a simple group; \( 2\mathbb Z \vartriangleleft \mathbb Z \).
	Hence \( G \) is finite, so \( G \cong C_{o(g)} \).
	If \( o(g) = mn \) for \( m, n \neq 1, p \), then \( \genset{g^m} \leq G \), contradicting simplicity.
\end{proof}
\begin{lemma}
	If \( G \) is a finite group, then \( G \) has a \textit{composition series}
	\[
		1 \cong G_0 \vartriangleleft G_1 \vartriangleleft \dots \vartriangleleft G_n = G
	\]
	where each quotient \( \faktor{G_{i+1}}{G_i} \) is simple.
\end{lemma}
\begin{remark}
	It is not the case that necessarily \( G_i \) be normal in \( G_{i+k} \) for \( k \geq 2 \).
\end{remark}
\begin{proof}
	We will consider an inductive step on \( \abs{G} \).
	If \( \abs{G} = 1 \), then trivially \( G = 1 \).
	Conversely, if \( \abs{G} > 1 \), let \( G_{n-1} \) be a normal subgroup of largest possible order not equal to \( \abs{G} \).
	Then, \( \faktor{G}{G_{n-1}} \) exists, and is simple by the correspondence theorem.
\end{proof}
