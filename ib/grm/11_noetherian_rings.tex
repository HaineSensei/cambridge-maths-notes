\subsection{???}
Recall the definition of a Noetherian ring.
\begin{definition}
	A ring \( R \) is \textit{Noetherian} if, for all sequences of nested ideals \( I_1 \subseteq I_2 \subseteq \cdots \), there exists \( N \in \mathbb N \) such that for all \( n > N \), \( I_n = I_{n+1} \).
\end{definition}
\begin{lemma}
	Let \( R \) be a ring.
	Then \( R \) satisfies the ascending chain condition (so \( R \) is Noetherian) if and only if all ideals in \( R \) are finitely generated.
\end{lemma}
We have already shown that principal ideal domains are Noetherian, since they satisfy this `ascending chain' condition.
This now will immediately follow from the lemma.
\begin{proof}
	First, suppose that all ideals in \( R \) are finitely generated.
	Let \( I_1 \subseteq I_2 \subseteq \cdots \) be an ascending chain of ideals.
	Consider \( I = \bigcup_{i=1}^\infty I_i \), which is an ideal.
	\( I \) is finitely generated, so \( I = (a_1, \dots, a_n) \).
	These elements belong to a nested union of ideals.
	In particular, we can choose \( N \in \mathbb N \) such that all \( a_i \) are contained within \( I_N \).
	Then, for \( n \geq N \), we find
	\[ (a_1, \dots, a_n) \subseteq I_N \subseteq I_n \subseteq I = (a_1, \dots, a_n) \]
	So the inclusions are all equalities, so \( I_N = I_n \).

	Conversely, suppose that \( R \) is Noetherian.
	Suppose that there exists an ideal \( J \triangleleft R \) which is not finitely generated.
	Let \( a_1 \in J \).
	Then since \( J \) is not finitely generated, \( (a_1) \subset J \).
	We can therefore choose \( a_2 \in J \setminus (a_1) \), and then \( (a_1) \subset (a_1, a_2) \subset J \).
	Continuing inductively, we contradict the ascending chain condition.
\end{proof}
