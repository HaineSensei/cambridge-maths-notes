\subsection{\( p \)-groups}
\begin{definition}
	Let \( p \) be a prime.
	A finite group \( G \) is a \textit{\( p \)-group} if \( \abs{G} = p^n \) for \( n \geq 1 \).
\end{definition}
\begin{theorem}
	If \( G \) is a \( p \)-group, the centre \( Z(G) \) is non-trivial.
\end{theorem}
\begin{proof}
	For \( g \in G \), due to the orbit-stabiliser theorem, \( \abs{\ccl(g)} \abs{C(g)} = p^n \).
	In particular, \( \abs{\ccl(g)} \) divides \( p^n \), and they partition \( G \).
	Since \( G \) is a disjoint union of conjugacy classes, modulo \( p \) we have
	\[
		\abs{G} \equiv \text{number of conjugacy classes of size } 1 \equiv 0 \implies \abs{Z(G)} \equiv 0
	\]
	Hence \( Z(G) \) has order zero modulo \( p \) so it cannot be trivial.
	We can check this by noting that \( g \in Z(G) \iff x^{-1} g x = g \) for all \( x \), which is true if and only if \( \ccl_G(g) = \qty{g} \).
\end{proof}
\begin{corollary}
	The only simple \( p \)-groups are the cyclic groups of order \( p \).
\end{corollary}
\begin{proof}
	Let \( G \) be a simple \( p \)-group.
	Since \( Z(G) \) is a normal subgroup of \( G \), we have \( Z(G) = 1 \) or \( Z(G) = G \).
	But \( Z(G) \) may not be trivial, so \( Z(G) = G \).
	This implies \( G \) is abelian.
	The only abelian simple groups are cyclic of prime order, hence \( G \cong C_p \).
\end{proof}
\begin{corollary}
	Let \( G \) be a \( p \)-group of order \( p^n \).
	Then \( G \) has a subgroup of order \( p^r \) for all \( r \in \qty{0, \dots, n} \).
\end{corollary}
\begin{proof}
	Recall that any group \( G \) has a composition series \( 1 = G_1 \vartriangleleft \dots \vartriangleleft G_N = G \) where each quotient \( \faktor{G_{i+1}}{G_i} \) is simple.
	Since \( G \) is a \( p \)-group, \( \faktor{G_{i+1}}{G_i} \) is also a \( p \)-group.
	Each successive quotient is an order \( p \) group by the previous corollary, so we have a composition series of nested subgroups of order \( p^r \) for all \( r \in \qty{0, \dots, n} \).
\end{proof}
\begin{lemma}
	Let \( G \) be a group.
	If \( \faktor{G}{Z(G)} \) is cyclic, then \( G \) is abelian.
	This then implies that \( Z(G) = G \), so in particular \( \faktor{G}{Z(G)} = 1 \).
\end{lemma}
\begin{proof}
	Let \( g Z(G) \) be a generator for \( \faktor{G}{Z(G)} \).
	Then, each coset of \( Z(G) \) in \( G \) is of the form \( g^r Z(G) \) for some \( r \in \mathbb Z \).
	Thus, \( G = \qty{g^r z \colon r \in \mathbb Z, z \in Z(G)} \).
	Now, we multiply two elements of this group and find
	\[
		g^{r_1} z_1 g^{r_2} z_2 = g^{r_1 + r_2} z_1 z_2 = g^{r_1 + r_2} z_2 z_1 = z_2 z_1 g^{r_1 + r_2} = g^{r_2} z_2 g^{r_1} z_1
	\]
	So any two elements in \( G \) commute.
\end{proof}
\begin{corollary}
	Any group of order \( p^2 \) is abelian.
\end{corollary}
\begin{proof}
	Let \( G \) be a group of order \( p^2 \).
	Then \( \abs{Z(G)} \in \qty{1, p, p^2} \).
	The centre cannot be trivial as proven above, since \( G \) is a \( p \)-group.
	If \( \abs{Z(G)} = p \), we have that \( \faktor{G}{Z(G)} \) is cyclic as it has order \( p \).
	Applying the previous lemma, \( G \) is abelian.
	However, this is a contradiction since the centre of an abelian group is the group itself.
	If \( \abs{Z(G)} = p^2 \) then \( Z(G) = G \) and then \( G \) is clearly abelian.
\end{proof}

\subsection{Sylow theorems}
\begin{theorem}
	Let \( G \) be a finite group of order \( p^a m \) where \( p \) is a prime and \( p \) does not divide \( m \).
	Then:
	\begin{enumerate}
		\item The set \( \mathrm{Syl}_p(G) = \qty{P \leq G \colon \abs{P} = p^a} \) of Sylow \( p \)-subgroups is non-empty.
		\item All Sylow \( p \)-subgroups are conjugate.
		\item The amount of Sylow \( p \)-subgroups \( n_p = \abs{\mathrm{Syl}_p(G)} \) satisfies
		      \[
			      n_p \equiv 1 \mod p;\quad n_p \mid \abs{G} \implies n_p \mid m
		      \]
	\end{enumerate}
\end{theorem}
\begin{proof}
	\begin{enumerate}
		\item Let \( \Omega \) be the set of all subsets of \( G \) of order \( p^a \).
		      We can directly find
		      \[
			      \abs{\Omega} = \binom{p^a m}{p^a} = \frac{p^a m}{p^a} \cdot \frac{p^a m - 1}{p^a - 1} \cdots \frac{p^a m - p^a + 1}{1}
		      \]
		      Note that for \( 0 \leq k < p^a \), the numbers \( p^a m - k \) and \( p^a - k \) are divisible by the same power of \( p \).
		      In particular, \( \abs{\Omega} \) is coprime to \( p \).

		      Let \( G \acts \Omega \) by left-multiplication, so \( g \ast X = \qty{gx \colon x \in X} \).
		      For any \( X \in \Omega \), the orbit-stabiliser theorem can be applied to show that
		      \[
			      \abs{G_X} \abs{\mathrm{orb}_G(X)} = \abs{G} = p^a m
		      \]
		      By the above, there must exist an orbit with size coprime to \( p \), since orbits partition \( \Omega \).
		      For such an \( X \), \( p^a \mid \abs{G_X} \).

		      Conversely, note that if \( g \in G \) and \( x \in X \), then \( g \in (gx^{-1}) \ast X \).
		      Hence, we can consider
		      \[
			      G = \bigcup_{g \in G} g \ast X = \bigcup_{Y \in \mathrm{orb}_G(X)} Y
		      \]
		      Thus \( \abs{G} \leq \abs{\mathrm{orb}_G(X)} \cdot \abs{X} \), giving \( \abs{G_X} = \frac{\abs{G}}{\abs{\mathrm{orb}_G(X)}} \leq \abs{X} = p^a \).

		      Combining with the above, we must have \( \abs{G_X} = p^a \).
		      In other words, the stabiliser \( G_X \) is a Sylow \( p \)-subgroup of \( G \).
		\item We will prove a stronger result for this part of the proof.
		      We claim that if \( P \) is a Sylow \( p \)-subgroup and \( Q \leq G \) is a \( p \)-subgroup, then \( Q \leq g P g^{-1} \) for some \( g \in G \).
		      Indeed, let \( Q \) act on the set of left cosets of \( P \) in \( G \) by left multiplication.
		      By the orbit-stabiliser theorem, each orbit has size which divides \( \abs{Q} = p^k \) for some \( k \).
		      Hence each orbit has size \( p^r \) for some \( r \).

		      Since \( \faktor{G}{P} \) has size \( m \), which is coprime to \( p \), there must exist an orbit of size 1.
		      Therefore there exists \( g \in G \) such that \( q \ast gP = gP \) for all \( q \in Q \).
		      Equivalently, \( g^{-1} q g \in P \) for all \( q \in Q \).
		      This implies that \( Q \leq gPg^{-1} \) as required.
		      This then weakens to the second part of the Sylow theorems.
		\item Let \( G \) act on \( \mathrm{Syl}_p(G) \) by conjugation.
		      Part (ii) of the Sylow theorems implies that this action is transitive.
		      By the orbit-stabiliser theorem, \( n_p = \abs{\mathrm{Syl}_p(G)} \mid \abs{G} \).

		      Let \( P \in \mathrm{Syl}_p(G) \).
		      Then let \( P \) act on \( \mathrm{Syl}_p(G) \) by conjugation.
		      Since \( P \) is a Sylow \( p \)-subgroup, the orbits of this action have size dividing \( \abs{P} = p^a \), so the size is some power of \( p \).
		      To show \( n_p \equiv 1 \mod p \), it suffices to show that \( \qty{P} \) is the unique orbit of size 1.
		      Suppose \( \qty{Q} \) is another orbit of size 1, so \( Q \) is a Sylow \( p \)-subgroup which is preserved under conjugation by \( P \).
		      \( P \) normalises \( Q \), so \( P \leq N_G(Q) \).
		      Notice that \( P \) and \( Q \) are both Sylow \( p \)-subgroups of \( N_G(Q) \).
		      By (ii), \( P \) and \( Q \) are conjugate inside \( N_G(Q) \).
		      Hence \( P = Q \) since \( Q \trianglelefteq N_G(Q) \).
		      Thus, \( \abs{P} \) is the unique orbit of size 1, so \( n_p \equiv 1 \mod p \) as required.
	\end{enumerate}
\end{proof}
\begin{corollary}
	If \( n_p = 1 \), then there is only one Sylow \( p \)-subgroup, and it is normal.
\end{corollary}
\begin{proof}
	Let \( g \in G \) and \( P \in \mathrm{Syl}_p(G) \).
	Then \( g P g^{-1} \) is a Sylow \( p \)-subgroup, hence \( g P g^{-1} = P \).
	\( P \) is normal in \( G \).
\end{proof}
\begin{example}
	Let \( G \) be a group with \( \abs{G} = 1000 = 2^3 \cdot 5^3 \).
	Here, \( n_5 \equiv 1 \mod 5 \), and \( n_5 \mid 8 \), hence \( n_5 = 1 \).
	Thus the unique Sylow 5-subgroup is normal.
	Hence no group of order 1000 is simple.
\end{example}
\begin{example}
	Let \( G \) be a group with \( \abs{G} = 132 = 2^2 \cdot 3 \cdot 11 \).
	\( n_{11} \) satisfies \( n_{11} \equiv 1 \mod 11 \) and \( n_{11} \mid 12 \), thus \( n_{11} \in \qty{1, 12} \).
	Suppose \( G \) is simple.
	Then \( n_{11} = 12 \).
	The amount of Sylow 3-subgroups satisfies \( n_3 \equiv 1 \mod 3 \) and \( n_3 \mid 44 \) so \( n_3 \in \qty{1, 4, 22} \).
	Since \( G \) is simple, \( n_3 \in \qty{4, 22} \).

	Suppose \( n_3 = 4 \).
	Then \( G \acts \mathrm{Syl}_3(G) \) by conjugation, and this generates a group homomorphism \( \varphi \colon G \to S_4 \).
	But the kernel of this homomorphism is a normal subgroup of \( G \), so \( \ker \varphi \) is trivial or \( G \) itself.
	If \( \ker \varphi = G \), then \( \Im \varphi \) is trivial, contradicting Sylow's second theorem.
	If \( \ker \varphi = 1 \), then \( \Im \varphi \) has order 132, which is impossible.

	Thus \( n_3 = 22 \).
	This means that \( G \) has \( 22 \cdot (3-1) = 44 \) elements of order 3, and further \( G \) has \( 12 \cdot (11 - 1) = 120 \) elements of order 11.
	However, the sum of these two totals is more than the total of 132 elements, so this is a contradiction.
	Hence \( G \) is not simple.
\end{example}
