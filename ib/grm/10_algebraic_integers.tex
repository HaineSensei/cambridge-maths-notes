\subsection{Gaussian integers}
Recall the ring of Gaussian integers \( \mathbb Z[i] = \qty{a+bi \colon a, b \in \mathbb Z} \leq \mathbb C \).
There is a norm function \( N \colon \mathbb Z[i] \to \mathbb Z_{\geq 0} \) given by \( a + bi \mapsto a^2 + b^2 \), and \( N(xy) = N(x) N(y) \).
This norm is a Euclidean function, giving the Gaussian integers the structure of a Euclidean domain and hence a principal ideal domain and a unique factorisation domain.
In particular, the primes are the irreducibles.
The units in \( \mathbb Z[i] \) are \( \pm 1, \pm i \), since they are the only elements of unit norm.
\begin{example}
    2 is not irreducible in \( \mathbb Z[i] \), since it factors as \( (1+i)(1-i) \).
    5 is not irreducible, since it factors as \( (2+i)(2-i) \).
    These are nontrivial factorisations since the norms of the factors are not unit length.

    3 is a prime, since it is irreducible.
    Indeed, \( N(3) = 9 \), so if 3 were reducible it would factor as \( ab \) where \( N(a) = N(b) = 3 \).
    But \( \mathbb Z[i] \) has no elements of norm 3.
    Similarly, 7 is a prime.
\end{example}
\begin{proposition}
    Let \( p \in \mathbb Z \) be a prime.
    Then, the following are equivalent.
    \begin{enumerate}
        \item \( p \) is not prime in \( \mathbb Z[i] \);
        \item \( p = a^2 + b^2 \) for \( a, b \in \mathbb Z \);
        \item \( p = 2 \) or \( p \equiv 1 \mod 4 \).
    \end{enumerate}
\end{proposition}
\begin{proof}
    Suppose \( p \) is not prime in \( \mathbb Z[i] \).
    So let \( p = xy \) for \( x, y \in \mathbb Z[i] \) not units.
    Then, \( p^2 = N(p) = N(x)N(y) \).
    Since \( x, y \) are not units, \( N(x), N(y) > 1 \) and in particular \( N(x) = N(y) = p \).
    Writing \( x = a+bi \) for \( a, b \in \mathbb Z \), we have \( p = N(x) = a^2 + b^2 \), which is the condition in (ii).

    Now, suppose \( p = a^2 + b^2 \).
    The only squares modulo 4 are 0 and 1.
    Since \( p \equiv a^2 + b^2 \mod 4 \), we have that \( p \) cannot be congruent to 3, modulo 4.

    Finally, suppose \( p = 2 \) or \( p \equiv 1 \mod 4 \).
    We have already observed above that 2 is not prime.
    It hence suffices to consider the case where \( p \equiv 1 \mod 4 \).
    We have that \( \qty(\faktor{Z}{pZ})^\times \) is cyclic of order \( p-1 \) by a previous theorem.
    Hence, if \( p \equiv 1 \mod 4 \), we have that \( 4 \mid p-1 \), and hence \( \qty(\faktor{Z}{pZ})^\times \) contains an element of order 4.
    In particular, there exists \( x \in \mathbb Z \) with \( x^4 \equiv 1 \mod p \), but \( x^2 \not\equiv 1 \mod p \).
    Then \( x^2 \equiv -1 \mod p \), or in other words, \( p \mid (x^2 + 1) \).
    But this factorises as \( p \mid (x+i)(x-i) \).
    We can see that \( p \nmid x+i \), \( p \nmid x-i \), so \( p \) cannot be prime.
\end{proof}
\begin{remark}
    The proof that (iii) implies (ii) is entirely nontrivial.
    It required lots of theory in order to reach the result, even though its statement did not require even the notion of a complex number.
\end{remark}
\begin{theorem}
    The primes in \( \mathbb Z[i] \) are, up to associates,
    \begin{enumerate}
        \item \( a + bi \), where \( a, b \in mathbb Z \) and \( a^2 + b^2 = p \) is a prime in \( \mathbb Z \) with \( p = 2 \) or \( p \equiv 1 \mod 4 \); and
        \item the primes \( p \) in \( \mathbb Z \) satisfying \( p \equiv 3 \mod 4 \).
    \end{enumerate}
\end{theorem}
\begin{proof}
    First, we must check that all such elements are prime.
    For (i), note that \( N(a+bi) = p \) is prime, so \( a+bi \) is irreducible.
    We can use the above proof to deduce that primes in \( \mathbb Z \) of form (ii) are primes in \( \mathbb Z[i] \).

    It now suffices to show that any prime in the Gaussian integers satisfies one of the two above conditions.
    Let \( z \) be prime in \( \mathbb Z[i] \).
    We note that \( \overline z \) is also irreducible.
    Now, \( N(z) = z\overline z \), which is a factorisation of the norm into irreducibles.
    % TODO: finish this proof
\end{proof}
