\subsection{Definitions}
\begin{definition}
	Let \( X \) be a set.
	Then \( \Sym(X) \) is the group of permutations of \( X \); that is, the group of all bijections of \( X \) to itself under composition.
	The identity can be written \( \id \) or \( \id_X \).
\end{definition}
\begin{definition}
	A group \( G \) is a permutation group of degree \( n \) if \( G \leq \Sym(X) \) where \( \abs{X} = n \).
\end{definition}
\begin{example}
	The symmetric group \( S_n \) is exactly equal to \( \Sym(\qty{1, \dots, n}) \), so is a permutation group of order \( n \).
	\( A_n \) is also a permutation group of order \( n \), as it is a subgroup of \( S_n \).
	\( D_{2n} \) is a permutation group of order \( n \).
\end{example}
\begin{definition}
	A \textit{group action} of a group \( G \) on a set \( X \) is a function \( \alpha \colon G \times X \to X \) satisfying
	\[ \alpha(e, x) = x;\quad \alpha(g_1 \cdot g_2, x) = \alpha(g_1, \alpha(g_2, x)) \]
	for all \( g_1, g_2 \in G, x \in X \).
	The group action may be written \( \ast \), defined by \( g \ast x \equiv \alpha(g,x) \).
\end{definition}
\begin{proposition}
	An action of a group \( G \) on a set \( X \) is uniquely characterised by a group homomorphism \( \varphi \colon G \to \Sym(X) \).
\end{proposition}
\begin{proof}
	For all \( g \in G \), we can define \( \varphi_g \colon X \to X \) by \( x \mapsto g \ast x \).
	Then, for all \( x \in X \),
	\[ \varphi_{g_1 g_2} (x) = (g_1 g_2) \ast x = g_1 \ast (g_2 \ast x) = \varphi_{g_1}(\varphi_{g_2}(x)) \]
	Thus \( \varphi_{g_1 g_2} = \varphi_{g_1} \circ \varphi_{g_2} \).
	In particular, \( \varphi_g \circ \varphi_{g^{-1}} = \varphi_e \).
	We now define
	\[ \varphi \colon G \to \Sym(X);\quad \varphi(g) = \varphi_g \implies \varphi(g)(x) = g \ast x \]
	This is a homomorphism.

	Conversely, any group homomorphism \( \varphi \colon G \to \Sym(X) \) induces a group action \( \ast \) by \( g \ast x = \varphi(g) \).
	This yields \( e \ast x = \varphi(e)(x) = \id x = x \) and \( (g_1 g_2) \ast x = \varphi(g_1 g_2) x = \varphi(g_1) \varphi(g_2) x = g_1 \ast (g_2 \ast x) \) as required.
\end{proof}
\begin{definition}
	The homomorphism \( \varphi \colon G \to \Sym(X) \) defined in the above proof is called a \textit{permutation representation} of \( G \).
\end{definition}
\begin{definition}
	Let \( G \acts X \).
	Then,
	% TODO: make this the default for enumerations
	\begin{enumerate}[(i)]
		\item the orbit of \( x \in X \) is \( \Orb_G(x) = \qty{g \ast x \colon g \in G} \subseteq X \);
		\item the stabiliser of \( x \in X \) is \( G_x = \qty{g \in G \colon g \ast x = x} \leq G \).
	\end{enumerate}
\end{definition}
\begin{theorem}[Orbit-stabiliser theorem]
	The orbit \( \Orb_G(x) \) bijects with the set \( \faktor{G}{G_x} \) of left cosets of \( G_x \) in \( G \) (which may not be a quotient group).
	In particular, if \( G \) is finite, we have
	\[ \abs{G} = \abs{\Orb(x)} \cdot \abs{G_x} \]
\end{theorem}
\begin{example}
	If \( G \) is the group of symmetries of a cube and we let \( X \) be the set of vertices in the cube, \( G \acts X \).
	Here, for all \( x \in X \), \( \abs{\Orb(x)} = 8 \) and \( \abs{G_x} = 6 \) (including reflections), hence \( \abs{G} = 48 \).
\end{example}
\begin{remark}
	Note that \( \ker \varphi = \bigcap_{x \in X} G_x \).
	The kernel of the permutation representation \( \varphi \) is also referred to as the kernel of the group action itself.
	If the kernel is trivial the action is saif to be \textit{faithful}.

	The orbits partition \( X \).
	In particular, if there is exactly one orbit, the group action is said to be \textit{transitive}.

	Note that \( G_{g \ast x} = g G_x g^{-1} \).
	Hence, if \( x, y \) lie in the same orbit, their stabilisers are conjugate.
\end{remark}

\subsection{Examples}
\begin{example}
	\( G \) acts on itself by left multiplication.
	This is known as the \textit{left regular action}.
	The kernel is trivial, hence the action is faithful.
	The action is transitive, since for all \( g_1, g_2 \in G \), the element \( g_2 g_1^{-1} \) maps \( g_1 \) to \( g_2 \).
\end{example}
\begin{theorem}[Cayley's theorem]
	Any finite group \( G \) is a permutation group of order \( \abs{G} \); it is isomorphic to a subgroup of \( S_\abs{G} \).
\end{theorem}
\begin{example}
	Let \( H \leq G \).
	Then \( G \acts \faktor{G}{H} \) by left multiplication, where \( \faktor{G}{H} \) is the set of left cosets of \( H \) in \( G \).
	This is known as the \textit{left coset action}.
	This action is transitive using the construction above for the left regular action.
	We have \( \ker\varphi = \bigcap_{x \in G} xHx^{-1} \), which is the largest normal subgroup of \( G \) contained within \( H \).
\end{example}
\begin{theorem}
	Let \( G \) be a non-abelian simple group, and \( H \leq G \) with index \( n > 1 \).
	Then \( n \geq 5 \) and \( G \) is isomorphic to a subgroup of \( A_n \).
\end{theorem}
\begin{proof}
	Let \( G \acts X = \faktor{G}{H} \) by left multiplication.
	Let \( \varphi \colon G \to \Sym(X) \) be the permutation representation associated to this group action.
	Since \( G \) is simple, \( \ker \varphi = 1 \) or \( \ker \varphi = G \).
	If \( \ker \varphi = G \), then \( \Im\varphi = \id \), which is a contradiction since \( G \) acts transitively on \( X \), which has index greater than one.
	Thus \( \ker \varphi = 1 \), and \( G \cong \Im\varphi \leq S_n \).
	Since \( G \leq S_n \) and \( A_n \triangleleft S_n \), the second isomorphism theorem shows that \( G \cap A_n \triangleleft G \), and
	\[ \faktor{G}{G \cap A_n} \cong \faktor{GA_n}{A_n} \leq \faktor{S_n}{A_n} \cong C_2 \]
	Since \( G \) is simple, \( G \cap A_n = 1 \) or \( G \cap A_n = G \).
	If \( G \cap A_n = 1 \), then \( G \) is isomorphic to a subgroup of \( C_2 \), but this is false, since \( G \) is non-abelian.
	Hence \( G \cap A_n = G \) so \( G \leq A_n \).
	Finally, if \( n \leq 4 \) we can check manually that \( A_n \) is not simple; \( A_n \) has no non-abelian simple subgroups.
\end{proof}
