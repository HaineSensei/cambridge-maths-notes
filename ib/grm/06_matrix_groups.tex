\subsection{Definitions}
\begin{definition}
	Let \( F \) be a field, such as \( \mathbb C \) or \( \faktor{\mathbb Z}{p \mathbb Z} \).
	Let \( GL_n(F) \) be set of \( n \times n \) invertible matrices over \( F \), which is called the \textit{general linear group}.
	Let \( SL_n(F) \) be set of \( n \times n \) matrices with determinant one over \( F \), which is called the \textit{special linear group}.
	\( SL_n(F) \) is the kernel of the determinant homomorphism on \( GL_n(F) \), so \( SL_n(F) \vartriangleleft GL_n(F) \).

	Let \( Z \vartriangleleft GL_n(F) \) denote the subgroup of \textit{scalar matrices}, the group of nonzero multiples of the identity.
	The group \( PGL_n(F) = \faktor{GL_n(F)}{Z} \) is called the \textit{projective general linear group}.
	Let \( PSL_n(F) = \faktor{SL_n(F)}{Z \cap SL_n(F)} \).
	By the second isomorphism theorem, \( PSL_n(F) \) is isomorphic to \( \faktor{Z \cdot SL_n(F)}{Z} \), which is a subgroup of \( PGL_n(F) \).
\end{definition}
\begin{example}
	Consider the finite group \( G = GL_n\qty(\faktor{\mathbb Z}{p\mathbb Z}) \).
	A list of \( n \) vectors in \( \faktor{\mathbb Z}{p\mathbb Z} \) are the columns of a matrix \( A \in G \) if and only if the vectors are linearly independent.
	Hence, by considering dimensionality of subspaces generated by each column,
	\begin{align*}
		\abs{G} & = (p^n - 1)(p^n - p)(p^n - p^2) \cdots (p^n - p^{n-1})      \\
		        & = p^{1+2+\dots+(n-1)} (p^n - 1)(p^{n-1} - 1) \cdots (p - 1) \\
		        & = p^{\binom{n}{2}} \prod_{i=1}^n (p^i - 1)
	\end{align*}
	Hence the Sylow \( p \)-subgroups have size \( p^{\binom{n}{2}} \).
	Let \( U \) be the set of upper triangular matrices with ones on the diagonal.
	This forms a Sylow \( p \)-subgroup of \( G \), since there are \( \binom{n}{2} \) entries in a given upper triangular matrix, and there are \( p \) choices for such an entry.
\end{example}

\subsection{M\"obius maps in modular arithmetic}
Recall that \( PGL_2(\mathbb C) \) acts on \( \mathbb C \cup \qty{\infty} \) by M\"obius transformations.
Likewise, \( PGL_2\qty(\faktor{\mathbb Z}{p\mathbb Z}) \) acts on \( \faktor{\mathbb Z}{p\mathbb Z} \cup \qty{\infty} \) by M\"obius transformations.
For a matrix
\[
	A = \begin{pmatrix}
		a & b \\
		c & d
	\end{pmatrix} \in GL_2\qty(\faktor{\mathbb Z}{p\mathbb Z});\quad A \colon z \mapsto \frac{az+b}{cz+d}
\]
Since the scalar matrices act trivially, we obtain an action on the projective general linear group instead of the general linear group.
We can represent \( \infty \) as an integer, say, \( p \), for the purposes of constructing a permutation representation.
\begin{lemma}
	The permutation representation \( PGL_2\qty(\faktor{\mathbb Z}{p\mathbb Z}) \to S_{p+1} \) is injective (and is an isomorphism if \( p = 2 \) or \( p = 3 \)).
\end{lemma}
\begin{proof}
	Suppose that \( \frac{az+b}{cz+d} = z \) for all \( z \in \faktor{\mathbb Z}{p\mathbb Z} \cup \qty{\infty} \).
	Since \( z = 0 \), we have \( b = 0 \).
	Since \( z = \infty \), we find \( c = 0 \).
	Thus the matrix is diagonal.
	Finally, since \( z = 1 \), \( \frac{a}{d} = 1 \) hence \( a = d \).
	Thus the matrix is scalar.
	So the permutation representation from \( PGL_2\qty(\faktor{\mathbb Z}{p \mathbb Z}) \) has trivial kernel, giving injectivity as required.

	If \( p = 2 \) or \( p = 3 \) we can compute the orders of relevant groups manually and show that the permutation representation is an isomorphism.
\end{proof}
\begin{lemma}
	Let \( p \) be an odd prime.
	Then
	\[
		\abs{PSL_2\qty(\faktor{\mathbb Z}{p\mathbb Z})} = \frac{(p-1)p(p+1)}{2}
	\]
\end{lemma}
\begin{proof}
	By the example above,
	\[
		\abs{GL_2\qty(\faktor{\mathbb Z}{p\mathbb Z})} = p(p^2 - 1)(p - 1)
	\]
	The homomorphism \( GL_2\qty(\faktor{\mathbb Z}{p\mathbb Z}) \to \qty(\faktor{\mathbb Z}{p\mathbb Z})^\times \) given by the determinant is surjective.
	Since \( SL_2\qty(\faktor{\mathbb Z}{p\mathbb Z}) \) is the kernel of this homomorphism, we have
	\[
		\abs{SL_2\qty(\faktor{\mathbb Z}{p\mathbb Z})} = p(p-1)(p+1)
	\]
	Now, if \(
	\begin{pmatrix}
		\lambda & 0 \\ 0 & \lambda
	\end{pmatrix}
	\) is an element of the special linear group, then \( \lambda^2 \equiv 1 \text{ mod } p \).
	Then, \( p \mid (\lambda - 1)(\lambda + 1) \) hence \( \lambda \equiv \pm 1 \text{ mod } p \).
	Thus,
	\[
		Z \cap SL_2\qty(\faktor{\mathbb Z}{p\mathbb Z}) = \qty{\pm 1}
	\]
	and the elements are distinct since \( p > 2 \).
	Hence the order of the projective special linear group is half the order of the special linear group as required.
\end{proof}
\begin{example}
	Let \( G = PSL_2\qty(\faktor{\mathbb Z}{5\mathbb Z}) \).
	Then by the previous lemma, \( \abs{G} = 60 \).
	Let \( G \acts \faktor{\mathbb Z}{5\mathbb Z} \cup \qty{\infty} \) by M\"obius transformations.
	The permutation representation \( \varphi \colon G \to \Sym(\qty{0,1,2,3,4,\infty}) \) is injective, since the permutation representation of \( PGL_2\qty(\faktor{\mathbb Z}{p\mathbb Z}) \) is known to be injective by a previous lemma.

	We claim that \( \Im \varphi \subseteq A_6 \).
	Let \( \psi = \sgn \circ \varphi \).
	If we can show \( \psi \) is trivial, \( \Im \varphi \subseteq A_6 \).
	Let \( h\in G \), and suppose \( h \) has order \( 2^n m \) for odd \( m \).
	If \( \psi(h^m) = 1 \), then since \( \psi \) is a group homomorphism we have \( \psi(h)^m = 1 \) giving \( \psi(h) \neq -1 \implies \psi(h) = 1 \).
	So to show \( \psi \) is trivial, it suffices to show \( \psi(g) = 1 \) for all \( g \in G \) with order a power of 2.
	By the second Sylow theorem, if \( g \) has order a power of 2, it is contained in a Sylow 2-subgroup.
	Then it suffices to show that \( \psi(H) = 1 \) for all Sylow 2-subgroups \( H \).
	But since \( \ker \psi \) is normal and all Sylow 2-subgroups are conjugate, it suffices to show \( \psi(H) = 1 \) for a single Sylow 2-subgroup \( H \).
	The Sylow 2-subgroup must have order 4.
	Hence consider
	\[
		H = \genset{ \begin{pmatrix}
				2 & 0 \\
				0 & 3
			\end{pmatrix} \qty{\pm I}, \begin{pmatrix}
				0  & 1 \\
				-1 & 0
			\end{pmatrix} \qty{\pm I} }
	\]
	Both of these elements square to the identity element inside the projective special linear group.
	This generates a group of order 4 which is necessarily a Sylow 2-subgroup.
	We can explicitly compute the action of \( H \) on \( \qty{0,1,2,3,4,\infty} \).
	\[
		\varphi\qty(\begin{pmatrix}
				2 & 0 \\
				0 & 3
			\end{pmatrix}) = (1\ 4)(2\ 3);\quad \varphi\qty(\begin{pmatrix}
				0  & 1 \\
				-1 & 0
			\end{pmatrix}) = (0\ \infty)(1\ 4)
	\]
	These are products of two transpositions, hence even permutations.
	Thus \( \psi(H) = 1 \), proving the claim that \( G \leq A_6 \).
	We can prove that for any \( G \leq A_6 \) of order 60, we have \( G \cong A_5 \); this is a question from the example sheets.
\end{example}

\subsection{Properties}
The following properties will not be proven in this course.
\begin{itemize}
	\item \( PSL_n\qty(\faktor{\mathbb Z}{p\mathbb Z}) \) is simple for all \( n \geq 2 \) and \( p \) prime, except where \( n = 2 \) and \( p = 2, 3 \).
	      Such groups are called finite groups of \textit{Lie type}.
	\item The smallest non-abelian simple groups are \( A_5 \cong PSL_2\qty(\faktor{\mathbb Z}{5\mathbb Z}) \), then \( PSL_2\qty(\faktor{\mathbb Z}{7\mathbb Z}) \cong GL_3\qty(\faktor{\mathbb Z}{2\mathbb Z}) \) which has order 168.
\end{itemize}
