\subsection{Estimators}
Suppose \( X_1, \dots, X_n \) are i.i.d.\ observations with a p.d.f.\ (or p.m.f.) \( f_X(x \mid \theta) \), where \( \theta \) is an unknown parameter in some parameter space \( \Theta \).
Let \( X = (X_1, \dots, X_n) \).
\begin{definition}
	An \textit{estimator} is a statistic, or a function of the data, written \( T(X) = \hat\theta \), which is used to approximate the true value of \( \theta \).
	This does not depend (explicitly) on \( \theta \).
	The distribution of \( T(X) \) is called its \textit{sampling distribution}.
\end{definition}
\begin{example}
	Let \( X_1, \dots, X_n \sim N(0,1) \) be i.i.d.
	Let \( \hat \mu = T(X) = \overline X_n \).
	The sampling distribution is \( T(X) \sim N\qty(\mu, \frac{1}{n}) \).
	Note that this sampling distribution in general depends on the true parameter \( \mu \).
\end{example}
\begin{definition}
	The \textit{bias} of \( \hat \theta \) is
	\[ \mathrm{bias}\qty(\hat \theta) = \esub{\theta}{\hat \theta} - \theta \]
	Note that \( \hat \theta \) is a function only of \( X_1, \dots, X_n \), and the expectation operator \( \mathbb E_\theta \) assumes that the true value of the parameter is \( \theta \).
\end{definition}
\begin{remark}
	In general, the bias is a function of the true parameter \( \theta \), even though it is not explicit in the notation.
\end{remark}
\begin{definition}
	An estimator with zero bias for all \( \theta \) is called an \textit{unbiased estimator}.
\end{definition}
\begin{example}
	The estimator \( \hat \mu \) in the above example is unbiased, since
	\[ \esub{\mu}{\hat \mu} = \exp_\mu(\overline X_n) = \mu \]
	for all \( \mu \in \mathbb R \).
\end{example}
\begin{definition}
	The \textit{mean squared error} of \( \theta \) is defined as
	\[ \mathrm{mse}\qty(\hat \theta) = \esub{\theta}{\qty(\hat \theta - \theta)^2} \]
\end{definition}
\begin{remark}
	Like the bias, the mean squared error is, in general, a function of the true parameter \( \theta \).
\end{remark}

\subsection{Bias-variance decomposition}
The mean squared error can be written as
\[ \mathrm{mse}\qty(\hat \theta) = \esub{\theta}{\qty(\hat \theta - \esub{\theta}{\hat\theta} + \esub{\theta}{\hat\theta} - \theta)^2} = \Varsub{\theta}\qty(\hat \theta) + \mathrm{bias}^2\qty(\hat\theta) \]
Note that both the variance and bias squared terms are positive.
This implies a tradeoff between bias and variance when minimising error.
\begin{example}
	Let \( X \sim \mathrm{Bin}(n, \theta) \) where \( n \) is known and \( \theta \) is an unknown probability.
	Let \( T_U = X / n \).
	This is the proportion of successes observed.
	This is an unbiased estimator, since \( \esub{\theta}{T_U} = \esub{\theta}{X}/n = \theta \).
	The mean squared error for the estimator is then
	\[ \Varsub{\theta}\qty(T_n) = \Varsub{\theta}\qty(\frac{X}{n}) = \frac{\Varsub{\theta}(X)}{n^2} = \frac{\theta(1-\theta)}{n} \]
	Now, consider an alternative estimator which has some bias:
	\[ T_B = \frac{X+1}{n+2} = w \underbrace{\frac{X}{n}}_{T_U} + (1-w)\frac{1}{2};\quad w = \frac{n}{n+2} \]
	This interpolates between the estimator \( T_U \) and the fixed estimator \( \frac{1}{2} \).
	Here,
	\[ \mathrm{bias}(T_B) = \esub{\theta}{T_B} - \theta = \frac{n}{n+2}\theta - frac{1}{n+2}\theta \]
	The bias is nonzero for all but one value of \( \theta \).
	Further,
	\[ \Varsub{\theta}(T_B) = \frac{\Varsub{\theta}(X+1)}{(n+2)^2} = \frac{n\theta(1-\theta)}{(n+2)^2} \]
	We can calculate
	\[ \mathrm{mse}\qty(T_B) = (1-w)^2 \qty(\frac{1}{2} - \theta)^2 + w^2\underbrace{\frac{\theta(1-\theta)}{n}}_{\mathrm{mse}(T_U)} \]
	There exists a range of \( \theta \) such that \( T_B \) has a lower mean squared error, and similarly there exists a range such that \( T_U \) has a lower error.
	This indicates that prior judgement of the true value of \( \theta \) can be used to determine which estimator is better.
\end{example}
It is not necessarily desirable that an estimator is unbiased.
\begin{example}
	Suppose \( X \sim \mathrm{Poisson}(\lambda) \) and we wish to estimate \( \theta = \prob{X = 0}^2 = e^{-2\lambda} \).
	For some estimator \( T(X) \) of \( \theta \) to be unbiased, we need that
	\[ \esub{\lambda}{T(X)} = \sum_{x=0}^\infty T(x) \frac{\lambda^x e^{-lambda}}{x!} = e^{-2\lambda} \]
	Hence,
	\[ \sum_{x=0}^\infty T(x) \frac{\lambda^x}{x!} = e^{-\lambda} \]
	But \( e^{-\lambda} \) has a known power series expansion, giving \( T(X) \equiv (-1)^X \) for all \( X \).
	This is not a good estimator, for example because it often predicts negative numbers for a positive quantity.
\end{example}

\subsection{Sufficiency}
\begin{definition}
	A statistic \( T(X) \) is \textit{sufficient} for \( \theta \) if the conditional distribution of \( X \) given \( T(X) \) does not depend on \( \theta \).
	Note that \( \theta \) and \( T(X) \) may be vector-valued, and need not have the same dimension.
\end{definition}
\begin{example}
	Let \( X_1, \dots, X_n \) be i.i.d.\ Bernoulli random variables with parameter \( \theta \) where \( \theta \in [0,1] \).
	The mass function is
	\[ f_X(x \mid \theta) = \prod_{i=1}^n \theta^{x_i}(1-\theta)^{1-x_i} = \theta^{\sum x_i} (1-\theta)^{n - \sum x_i} \]
	Note that this dependent only on \( x \) via the statistic \( T(X) = \sum_{n=1}^n x_i \).
	Here,
	\[ f_{X \mid T = t}(x \mid \theta) = \frac{\psub{\theta}{X = x, T(X) = t}}{\psub{\theta}{T(x) = t}} \]
	If \( \sum x_i = t \), we have
	\[ f_{X \mid T = t}(x \mid \theta) = \frac{\theta^{\sum x_i} (1-\theta)^{n-\sum x_i}}{\binom{n}{t} \theta^t (1-\theta)^{n-\sum x_i}} = \frac{1}{\binom{n}{t}} \]
	Hence \( T(X) \) is sufficient for \( \theta \).
\end{example}

\subsection{Factorisation criterion}
\begin{theorem}
	\( T \) is sufficient for \( \theta \) if and only if
	\[ f_X(x \mid \theta) = g(T(x), \theta) h(x) \]
	for suitable functions \( g,h \).
\end{theorem}
\begin{proof}
	This will be proven in the discrete case; the continuous case can be handled analogously.
	Suppose that the factorisation criterion holds.
	Then, if \( T(x) = t \),
	\begin{align*}
		f_{X \mid T = t}(x \mid T = t) &= \frac{\psub{\theta}(X = x, T(x) = t)}{\psub{\theta}{T(x) = t}} \\
		&= \frac{g(T(x),\theta)h(x)}{\sum_{x' \colon T(x') = t} g(T(x'),\theta)h(x')} \\
		&= \frac{h(x)}{\sum_{x' \colon T(x') = t} h(x')}
	\end{align*}
	which does not depend on \( \theta \).
	By definition, \( T(X) \) is sufficient.

	Conversely, suppose that \( T(X) \) is sufficient.
	\begin{align*}
		f_X(x \mid \theta) &= \psub{\theta}{X = x} \\
		&= \psub{\theta}{X = x, T(X) = T(x)} \\
		&= \underbrace{\psub{\theta}{X = x \mid T(X) = T(x)}}_{h(x)} \underbrace{\psub{\theta}{T(X) = T(x)}}_{g(T(X), \theta)}
	\end{align*}
\end{proof}
\begin{example}
	Consider the above example with \( n \) Bernoulli random variables with mass function
	\[ f_X(x \mid \theta) = \theta^{\sum x_i} (1-\theta)^{n - \sum x_i} \]
	Let \( T(X) = \sum x_i \), and then the above mass function is in the form of \( g(T(X), \theta) \) and we can set \( h(x) \equiv 1 \).
	Hence \( T(X) \) is sufficient.
\end{example}
\begin{example}
	Let \( X_1, \dots, X_n \) be i.i.d.\ from a uniform distribution on the interval \( [0,\theta] \) for some \( \theta > 0 \).
	The mass function is
	\[ f_X(x \mid \theta) = \prod_{i=1}^n \frac{1}{\theta} 1\qty{x_i \in [0,\theta]} = \qty(\frac{1}{\theta})^{n} 1\qty{\min_i x_i \geq 0} 1\qty{\max_i x_i \leq \theta} \]
	Let \( T(X) = \max_i X_i \).
	Then
	\[ g(T(X), \theta) = \qty(\frac{1}{\theta})^n 1\qty{\max_i x_i \leq \theta};\quad h(x) \equiv 1\qty{\min_i x_i \geq 0} \]
	We can then conclude that \( T(X) \) is sufficient for \( \theta \).
\end{example}

\subsection{Minimal sufficiency}
Sufficient statistics are not unique.
For instance, any bijection applied to a sufficient statistic is also sufficient.
Further, \( T(X) = X \) is always sufficient.
We instead seek statistics that maximally compress and summarise the relevant data in \( X \) and that discard extraneous data.
\begin{definition}
	A sufficient statistic \( T(X) \) for \( \theta \) is \textit{minimal} if it is a function of every other sufficient statistic for \( \theta \).
	More precisely, if \( T'(X) \) is sufficient, \( T'(x) = T'(y) \implies T(x) = T(y) \).
\end{definition}
\begin{remark}
	Any two minimal statistics \( S, T \) for the same \( \theta \) are bijections of each other.
	That is, \( T(x) = T(y) \) if and only if \( S(x) = S(y) \).
\end{remark}
\begin{theorem}
	Suppose that \( f_X(x \mid \theta)/f_X(y \mid \theta) \) is constant in \( \theta \) if and only if \( T(x) = T(y) \).
	Then \( T \) is minimal sufficient.
\end{theorem}
\begin{remark}
	This theorem essentially states the following.
	Let \( x \overset{1}{\sim} y \) if the above ratio of probability density or mass functions is constant in \( \theta \).
	This is an equivalence relation.
	Similarly, we can define \( \overset{2}{\sim} y \) if \( T(x) = T(y) \).
	This is also an equivalence relation.
	The hypothesis in the theorem is that the equivalence classes of \( \overset{1}{\sim} \) and \( \overset{2}{\sim} \) are equal.
	Further, we may always construct a minimal sufficient statistic for any parameter since we can use the construction \( \overset{1}{\sim} \) to create equivalence classes, and set \( T \) to be constant for all such equivalence classes.
\end{remark}
\begin{proof}
	Let \( t \in \Im T \).
	Then let \( z_t \) be a representative of the equivalence class \( \qty{x \colon T(x) = t} \).
	Then
	\[ f_X(x \mid \theta) = f_X(z_{T(x)} \mid \theta) \frac{f_X(x \mid \theta)}{f_X(z_{T(x)} \mid \theta)} \]
	By the hypothesis, the ratio on the right hand side does not depend on \( \theta \), so let this ratio be \( h(x) \).
	Further, the other term depends only on \( T(x) \), so it may be \( g(T(x), \theta) \).
	Hence \( T \) is sufficient by the factorisation criterion.

	To prove minimality, let \( S \) be any other sufficient statistic, and then by the factorisation criterion there exist \( g_S \) and \( h_S \) such that \( f_X(x \mid \theta) = g_S(S(x), \theta) h_S(x) \).
	Now, suppose \( S(x) = S(y) \) for some \( x, y \).
	Then,
	\[ \frac{f_X(x \mid \theta)}{f_X(y \mid \theta)} = \frac{g_S(S(x), \theta) h_S(x)}{g_S(S(y), \theta) h_S(y)} = \frac{h_S(x)}{h_S(y)} \]
	which is constant in \( \theta \).
	Hence, \( x \overset{1}{\sim} y \).
	By the hypothesis, we have \( x \overset{2}{\sim} y \), so \( T(x) = T(y) \), which is the requirement for minimality.
\end{proof}
\begin{example}
	Let \( X_1, \dots, X_n \) be normal with unknown \( \mu, \sigma^2 \).
	\[ \frac{f_X(x \mid \mu, \sigma^2)}{f_X(y \mid \mu, \sigma^2)} = \frac{(2 \pi \sigma^2)^{-n/2} \exp{-\frac{1}{2\sigma^2} \sum_i (x_i - \mu)^2}}{(2 \pi \sigma^2)^{-n/2} \exp{-\frac{1}{2\sigma^2 \sum_i (y_i - \mu)^2}}} = \exp{-\frac{1}{2\sigma^2} \qty(\sum_i x_i^2 - \sum_i y_i^2) + \frac{\mu}{\sigma^2} \qty(\sum_i x_i - \sum_i y_i)} \]
	Hence, for minimality, this is constant in the parameters \( \mu, \sigma^2 \) if and only if \( \sum_i x_i^2 = \sum_i y_i^2 \) and \( \sum_i x_i = \sum_i y_i \).
	Thus, a minimal sufficient statistic is \( \qty(\sum_i x_i^2, \sum_i x_i) \) is a minimal sufficient statistic.
	A more common way of expressing the minimal sufficient statistic is
	\[ S(x) = \qty(\overline X_n, S_{xx});\quad \overline X_n = \frac{1}{n} \sum_i x_i;\quad S_{xx} = \sum_i \qty(X_i - \overline X_n)^2 \]
	which is a bijection of the above.
\end{example}
\begin{example}
	\( \theta \) and a minimal statistic \( T \) need not have the same dimension.
	Consider \( X_1, \dots, X_n \sim N(\mu, \mu^2) \).
	Here, there is a single parameter \( \mu \) but the minimal sufficient statistic is still \( S(x) \) as defined above.
\end{example}

\subsection{Rao-Blackwell theorem}
Previously, the notation \( \mathbb E_\theta \) and \( \mathbb P_\theta \) have been used to denote expectations and probabilities under the model where the observations are i.i.d.\ with p.d.f.\ or p.m.f.\ \( f_X \).
From now, we omit this subscript, as it will be implied for much of the remainder of the course.
\begin{theorem}
	Let \( T \) be a sufficient statistic for \( \theta \), and define an estimator \( \widetilde \theta \) with \( \expect{{\widetilde \theta}^2} < \infty \) for all \( \theta \).
	Now we define another estimator
	\[ \hat \theta = \expect{\widetilde \theta \mid T(x)} \]
	Then, for all values of \( \theta \), we have
	\[ \expect{\qty(\hat \theta - \theta)^2} \leq \expect{\qty(\widetilde \theta - \theta)^2} \]
	In other words, the mean squared error of \( \hat \theta \) is not greater than the mean squared error of \( \widetilde \theta \).
	Further, the inequality is strict unless \( \widetilde \theta \) is a function of \( T \).
\end{theorem}
\begin{remark}
	Starting from any estimator \( \widetilde \theta \), if we condition on the sufficient statistic \( T \) we obtain a `better' statistic \( \hat \theta \).
	Note that \( T \) must be sufficient, otherwise \( \hat \theta \) may be a function of \( \theta \) and thus not an estimator:
	\[ \hat \theta(X) = \hat \theta(T) = \int \hat \theta(x) \underbrace{f_{X \mid T}(x \mid T)}_{\mathclap{\text{does not depend on } \theta \text{ as } T \text{ is sufficient}}} \dd{x} \]
\end{remark}
\begin{proof}
	By the tower property of the expectation, we can find
	\[ \expect{\hat \theta} = \expect{\expect{\widetilde \theta \mid T(x)}} = \expect{\widetilde \theta} \]
	Hence, subtracting \( \widetilde \theta \) from both sides, we find \( \mathrm{bias}\qty(\hat\theta) = \mathrm{bias}\qty(\widetilde\theta) \).
	By the conditional variance formula,
	\[ \Var\qty(\widetilde \theta) = \expect{\underbrace{\Var\qty(\widetilde \theta \mid T)}_{\geq 0}} + \underbrace{\Var\qty(\expect{\widetilde \theta \mid T})}_{\Var\qty(\hat\theta)} \geq \Var\qty(\hat \theta) \]
	By the bias-variance decomposition, we know that \( \mathrm{mse}\qty(\widetilde \theta) \geq \mathrm{mse}\qty(\hat \theta) \).
	The inequality is strict unless \( \Var\qty(\widetilde \theta \mid T) = 0 \) almost surely.
	This requires that \( \widetilde \theta \) is a function of \( T \).
\end{proof}
\begin{example}
	Let \( X_1, \dots, X_n \) be i.i.d.\ Poisson random variables with parameter \( \lambda \).
	Then let \( \theta = \prob{X_1 = 0} = e^{-\lambda} \).
	Here,
	\[ f_X(x \mid \lambda) = \frac{e^{-n \lambda} \lambda^{\sum x_i}}{\prod x_i!} \implies f_X(x \mid \theta) = \frac{\theta^n (-\log \theta)^{\sum x_i}}{\prod x_i!} \]
	Using the factorisation criterion, we find
	\[ g(T(x), \theta) = g\qty(\sum x_i, \theta) = \theta^n (-\log\theta)^{\sum x_i};\quad h(x) = \frac{1}{\prod x_i!} \]
	so \( T(x) = \sum x_i \) is sufficient.
	Note that \( \sum X_i \) has a Poisson distribution with parameter \( n \lambda \).
	Consider the estimator \( \widetilde \theta = 1\qty{X_1 = 0} \).
	This depends only on \( X_1 \), hence it is a weak estimator.
	However, it is unbiased, so when we apply the Rao-Blackwell theorem we will construct an unbiased \( \hat \theta \), which is precisely
	\[ \hat \theta = \expect{\widetilde \theta \mid \sum X_i = t} = \prob{X_1 = 0 \mid \sum X_i = t} = \frac{\prob{X_1 = 0, \sum X_i = t}}{\prob{\sum X_i = t}} = \frac{\prob{X_1 = 0}\prob{\sum_{i=2}^n X_i = t}}{\prob{\sum_{i=1}^n X_i = t}} = \qty(\frac{n-1}{n})^t \]
	This may also be written
	\[ \hat \theta = \qty(1 - \frac{1}{n})^{\sum x_i} \]
	which is an estimator with lower mean squared error than \( \widetilde 1 \) for all \( \theta \).
	Note that \( \hat \theta = \qty(1 = \frac{1}{n})^{n \overline X_n} \) converges in the limit to \( e^{-\overline X_n} \).
	By the strong law of large numbers, \( \overline X_n \to \expect{X_1} = \lambda \), so we arrive at \( \hat \theta \to e^{-\lambda} = \theta \) almost surely.
\end{example}
