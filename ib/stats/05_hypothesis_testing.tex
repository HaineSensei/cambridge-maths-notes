\subsection{Hypotheses}
\begin{definition}
	A \textit{hypothesis} is an assumption about the distribution of the data \( X \).
	Scientific questions are often phrased as a decision between two hypotheses.
	The \textit{null hypothesis} \( H_0 \) is usually a basic hypothesis, often representing the simplest possible distribution of the data.
	The \textit{alternative hypothesis} \( H_1 \) is the alternative, if \( H_0 \) were found to be false.
\end{definition}
\begin{example}
	Let \( X = (X_1, \dots, X_n) \) be i.i.d.\ Bernoulli random variables with parameter \( \theta \).
	We could take, for example, \( H_0 \colon \theta = \frac{1}{2} \) and \( H_1 \colon \theta = \frac{3}{4} \).
	Alternatively, we could take \( H_0 \colon \theta = \frac{1}{2} \) and \( H_1 \colon \theta \neq \frac{1}{2} \).
\end{example}
\begin{example}
	Suppose \( X_i \) takes values \( 0, 1, \dots \).
	We can take \( H_0 \colon X_i \overset{\mathrm{iid}}{\sim} \mathrm{Poi}(\lambda) \) for some \( \lambda \), and \( H_1 \colon X_i \overset{\mathrm{iid}}{\sim} f_1 \) for some other distribution \( f_1 \).
	This is known as a \textit{goodness of fit} test, which checks how well the model used for the data fits.
\end{example}
\begin{definition}
	A \textit{simple hypothesis} is a hypothesis which fully specifies the p.d.f.\ or p.m.f.\ of the data.
	A hypothesis that is not simple is called \textit{composite}.
\end{definition}
\begin{example}
	In the first example above, \( H_0 \colon \theta = \frac{1}{2} \) is simple, and \( H_1 \colon \theta \neq \frac{1}{2} \) is composite.
	In the second example, \( H_0 \colon X_i \overset{\mathrm{iid}}{\sim} \mathrm{Poi}(\lambda) \) is composite since \( \lambda \) was not fixed.
\end{example}

\subsection{Testing hypotheses}
\begin{definition}
	A \textit{test} of the null hypothesis \( H_0 \) is defined by a \textit{critical region} \( C \subseteq \mathcal X \).
	When \( X \in C \), we \textit{reject} the null hypothesis.
	This is a positive result.
	When \( X \not\in C \) we \textit{fail to reject} the null hypothesis, or find \textit{no sufficient evidence against} the null hypothesis.
	This is the negative result.

	A \textit{type I} error, or a \textit{false positive}, is the error made by rejecting the null hypothesis when it is true.
	A \textit{type II} error, or a \textit{false negative}, is the error made by failing to reject the null hypothesis when it is not true.
	When \( H_0, H_1 \) are simple, we define
	\[ \alpha = \psub{H_0}{H_0 \text{ is rejected}} = \psub{H_0}{X \in C};\quad \beta = \psub{H_1}{H_0 \text{ is not rejected}} = \psub{H_1}{X \not\in C} \]
	The \textit{size} of a test is \( \alpha \), which is the probability of a type I error.
	The \textit{power} of a test is \( 1 - \beta \), which is the probability of not finding a type II error.

	There is typically a tradeoff between \( \alpha \) and \( \beta \).
	Often, statisticians will choose an `acceptable' value for the probability of type I errors \( \alpha \), and then maximise the power with respect to this fixed \( \alpha \).
	Computing the size of a test is typically simpler since it does not depend on \( H_1 \).
\end{definition}

\subsection{Neyman-Pearson lemma}
Let \( H_0 \) and \( H_1 \) be simple, and let \( X \) have a p.d.f.\ or p.m.f.\ \( f_i \) under \( H_i \).
The \textit{likelihood ratio statistic} is defined by
\[ \Lambda_x(H_0; H_1) = \frac{f_1(x)}{f_0(x)} \]
The \textit{likelihood ratio test} is a test that rejects \( H_0 \) when \( \Lambda_x \) exceeds a set value \( k \), or more formally, \( C = \qty{ x \colon \Lambda_x(H_0; H_1) > k } \).
\begin{lemma}
	Suppose that \( f_0, f_1 \) are nonzero on the same set, and suppose that there exists \( k > 0 \) such that the likelihood ratio test with critical region \( C = \qty{x \colon \Lambda_x(H_0; H_1) > k} \) has size \( \alpha \).
	Then out of all tests of size upper bounded by \( \alpha \), this test has the largest power.
\end{lemma}
\begin{remark}
	A likelihood ratio test with size \( \alpha \) does not always exist for any given \( \alpha \).
	However, in general we can find a \textit{randomised test} with arbitrary size \( \alpha \).
	This is a test where, for some values of \( X \), we reject the null hypothesis; for some values, we fail to reject the null hypothesis; and for some values we reject the null hypothesis with a random chance of rejecting the null hypothesis.
\end{remark}
\begin{proof}
	Let \( \overline C \) be the complement of \( C \) in \( \mathcal X \).
	Then, the likelihood ratio test has
	\[ \alpha = \int_C f_0(x) \dd{x};\quad \beta = \int_{\overline C} f_1(x) \dd{x} \]
	Let \( C^\star \) be a critical region for a different test, with type I and II error probabilities \( \alpha^\star, \beta^\star \).
	Here,
	\[ \alpha^\star = \int_{C^\star} f_0(x) \dd{x};\quad \beta^\star = \int_{\overline {C^\star}} f_1(x) \dd{x} \]
	Suppose \( \alpha^\star \leq \alpha \).
	Then, we will show \( \beta \leq \beta^\star \).
	\[ \beta - \beta^\star = \int_{\overline C} f_1(x) \dd{x} - \int_{\overline{C^\star}} f_1(x) \dd{x} \]
	By cancelling the integrals on the intersection, and using the definition of \( C \),
	\begin{align*}
		\beta - \beta^\star &= \int_{\overline C \cap C^\star} f_1(x) \dd{x} - \int_{\overline{C^\star} \cap C} f_1(x) \dd{x} \\
		&= \int_{\overline C \cap C^\star} \underbrace{\frac{f_1(x)}{f_0(x)}}_{\leq k} f_0(x) \dd{x} - \int_{\overline{C^\star} \cap C} \underbrace{\frac{f_1(x)}{f_0(x)}}_{\geq k} f_0(x) \dd{x} \\
		&\leq k \qty[ \int_{\overline C \cap C^\star} f_0(x) \dd{x} - \int_{\overline C^\star \cap C} f_0(x) \dd{x} ] \\
		&= k \qty[ \int_{\overline C \cap C^\star} f_0(x) \dd{x} + \int_{C \cap C^\star} f_0(x) \dd{x} - \int_{C \cap C^\star} f_0(x) \dd{x} - \int_{\overline C^\star \cap C} f_0(x) \dd{x} ] \\
		&= k \qty[ \int_{\cap C^\star} f_0(x) \dd{x} - \int_{C} f_0(x) \dd{x} ] \\
		&= k \qty[ \alpha^\star - \alpha ] \\
		&\leq 0
	\end{align*}
\end{proof}
\begin{example}
	Let \( X_1, \dots, X_n \sim N(\mu, \sigma_0^2) \) be i.i.d., where \( \sigma_0^2 \) is known and \( \mu \) is an unknown.
	We wish to find the most powerful test of fixed size \( \alpha \) for the hypotheses \( H_0 \colon \mu = \mu_0 \) and \( H_1 \colon \mu = \mu_1 > \mu_0 \).
	The likelihood ratio is
	\begin{align*}
		\Lambda_x(H_0;H_1) &= \frac{(2\pi \sigma_0^2)^{-n/2} \exp{\frac{-1}{2\sigma_0^2} \sum (x_i - \mu_0)^2}}{(2\pi \sigma_0^2)^{-n/2} \exp{\frac{-1}{2\sigma_0^2} \sum (x_i - \mu_1)^2}} \\
		&= \exp{\underbrace{\frac{\mu_1 - \mu_0}{\sigma_0^2}}_{\geq 0} n \overline X + \frac{n(\mu_0 - \mu_1)^2}{2\sigma_0^2}}
	\end{align*}
	which depends only on \( \overline X \), and is monotonically increasing with respect to the sample mean \( \overline X \).
	Therefore, this is also monotonically increasing with respect to the statistic
	\[ Z = \sqrt{n}\frac{\overline X - \mu_0}{\sigma_0} \]
	Thus, \( \Lambda_x > k \) if and only if \( Z > k' \) for some \( k' \).
	Hence, the likelihood ratio test has critical region \( \qty{x\colon Z(x) > k'} \) for some \( k' \).
	It thus suffices to find a critical region of \( Z \) with size \( \alpha \) in order to construct the most powerful test of this size.
	Under \( H_0 \), \( Z \sim N(0,1) \).
	Hence, the critical region is given by \( k' = \Phi^{-1}(1-\alpha) \).
	This is known as a \textit{\( Z \)-test}, since we are using the \( Z \) statistic.
\end{example}

\subsection{\( p \)-values}
\begin{definition}
	Let \( C \) be a critical region of the form \( \qty{ x: T(x) > k } \) for some test statistic \( T \).
	Let \( x^\star \) denote the observed data.
	Then, the \textit{\( p \)-value} is
	\[ \psub{H_0}{T(X) > T(x^\star)} \]
\end{definition}
Typically, when reporting the results of a test, we describe the conclusion of the test as well as the \( p \)-value.
In the example above, suppose \( \mu_0 = 5 \), \( \mu_1 = 6 \), \( \alpha = 0.05 \), and \( x^\star = (5.1, 5.5, 4.9, 5.3) \).
Here, \( \overline{x^\star} = 5.2 \) and \( z^\star = 0.4 \).
The likelihood ratio test has critical region
\[ \qty{x: Z(x) > \Phi^{-1}(0.95) \approx 1.645} \]
The conclusion of the test here is to not reject \( H_0 \).
The \( p \)-value is \( 1 - \Phi(z^\star) \approx 0.35 \).
\begin{proposition}
	Under the null hypothesis \( H_0 \), the \( p \)-value is a uniform random variable in \( [0,1] \).
\end{proposition}
\begin{proof}
	Let \( F \) be the distribution of the test statistic \( T \), which we will assume for this proof is continuous.
	Then,
	\begin{align*}
		\psub{H_0}{p < u} &= \psub{H_0}{1 - F(T) < u} \\
		&= \psub{H_0}{F(T) > 1-u} \\
		&= \psub{H_0}{T > F^{-1}(1-u)} \\
		&= 1 - F(F^{-1}(1-u)) = u
	\end{align*}
\end{proof}

\subsection{Composite hypotheses}
Let \( X \sim f_X(\wildcard \mid \theta) \) where \( \theta \in \Theta \).
Let \( H_0 = \theta \in \Theta_0 \subset \Theta \) and \( H_1 = \theta \in \Theta_1 \subseteq \Theta \).
The probabilities of type I and type II error are now dependent on the precise value of \( \theta \), rather than simply on which hypothesis is taken.
\begin{definition}
	The \textit{power function} for a test \( C \) is
	\[ W(\theta) = \psub{\theta}{X \in C} \]
	The \textit{size} of a test \( C \) is
	\[ \alpha = \sup_{\theta \in \Theta_0} W(\theta) \]
	A test is \textit{uniformly most powerful} of size \( \alpha \) if, for any test \( C^\star \) with power function \( W^\star \) and size upper bounded by \( \alpha \), for all \( \theta \in \Theta_1 \) we have \( W(\theta) \geq W^\star(\theta) \).
	Such tests need not exist.
	In simple models, many likelihood ratio tests are uniformly most powerful.
\end{definition}
\begin{example}[one-sided test for normal location]
	Let \( X_1, \dots, X_n \sim N(\mu, \sigma_0^2) \) be i.i.d.\ where \( \sigma_0^2 \) is known and \( \mu \) is unknown.
	Let \( H_0 \colon \mu \leq \mu_0 \) and \( H_1 \colon \mu > \mu_0 \) for some fixed \( \mu_0 \).
	We claim that the simple hypothesis test given by \( H_0' \colon \mu = \mu_0 \) and \( H_1' \colon \mu = \mu_1 > \mu_0 \) is uniformly most powerful for \( H_0 \) and \( H_1 \).
	The power function is
	\begin{align*}
		W(\mu) &= \psub{\mu}{\frac{\sqrt{n}(\overline X-\mu_0)}{\sigma_0} = Z < z_\alpha = \Phi^{-1}(1-\alpha)} \\
		&= \psub{\mu}{\frac{\sqrt{n}(\overline X - \mu)}{\sigma_0} > z_\alpha + \frac{\sqrt{n}(\mu_0 - \mu)}{\sigma_0}} \\
		&= 1 - \Phi\qty(x_\alpha + \sqrt{n} \frac{\mu_0 - \mu}{\sigma_0})
	\end{align*}
	The test has size \( \alpha \) since \( \sup_{w \in \Theta_0} W(\mu) = \alpha \).
	It remains to show that this power function dominates all other power functions \( W^\star \) of size \( \alpha \) in the alternative space \( \Theta_1 \).
	First, observe that the critical region depends only on \( \mu_0 \), and not on \( \mu_1 \).
	In particular, for any \( \mu_1 > \mu_0 \), we have that the critical region \( C \) is the likelihood ratio test for the simple hypothesis test \( H_0' \colon \mu = \mu_0 \) and \( H_1' \colon \mu = \mu_1 \).
	We can also see \( C^\star \) as a test of \( H_0' \) versus \( H_1' \), and for these simple hypotheses, \( C^\star \) has size
	\[ W^\star(\mu_0) \leq \sup_{\mu < \mu_0} W^\star(\mu) \leq \alpha \]
	By the Neyman-Pearson lemma, \( C \) has power no smaller than \( C^\star \) for \( H_0' \) against \( H_1' \):
	\[ W(\mu_1) \geq W^\star(\mu_1) \]
	Since this is true for all \( \mu_1 > \mu_0 \), the result holds, and the test \( C \) satisfies the property for being uniformly most powerful.
\end{example}

\subsection{Generalised likelihood ratio test}
\begin{definition}
	Suppose we have \textit{nested hypotheses}, i.e.\ \( H_0 \colon \theta \in \Theta_0 \) and \( H_1 \colon \theta \in \Theta_1 \), where \( \Theta_0 \subset \Theta_1 \).
	The \textit{generalised likelihood ratio} is given by
	\[ \Lambda_x(H_0; H_1) = \frac{\sup_{\theta \in \Theta_1} f_X(x \mid \theta)}{\sup_{\theta \in \Theta_0} f_X(x \mid \theta)} \]
	Large values indicate a better fit under the alternative hypothesis.
	The \textit{generalised likelihood ratio test} rejects the null hypothesis when \( \Lambda_x \) is sufficiently large.
\end{definition}
\begin{example}[two-sided test for normal location]
	Let \( X_1, \dots, X_n \sim N(\mu, \sigma_0^2) \) be i.i.d.\ where \( \sigma_0^2 \) is known and \( \mu \) is unknown.
	Let \( H_0 \colon \mu = \mu_0 \) and \( H_1 \colon \mu \in \mathbb R \) for some fixed \( \mu_0 \).
	In this model, the generalised likelihood ratio is
	\begin{align*}
		\Lambda_x(H_0; H_1) &= \frac{(2 \pi \sigma_0^2)^{-n/2} \exp{\frac{-1}{2\sigma_0^2} \Sigma_{i=1}^n (x_i - \overline X)^2}}{(2 \pi \sigma_0^2)^{-n/2} \exp{\frac{-1}{2\sigma_0^2} \Sigma_{i=1}^n (x_i - \mu_0)^2}} \\
		2\log \Lambda_x &= \frac{n}{\sigma_0^2}(\overline X - \mu_0)^2
	\end{align*}
	Under \( H_0 \), \( \sqrt{n} \frac{\overline X - \mu_0}{\sigma_0} \sim N(0,1) \).
	Hence, \( 2 \log \Lambda_x \sim \chi_1^2 \).
	Therefore, the critical region of this generalised likelihood ratio test is
	\[ C = \qty{x \colon \frac{n}{\sigma_0^2}(\overline X - \mu_0)^2 > \chi_1^2(\alpha)} \]
	where \( \chi_1^2(\alpha) \) is the upper \( \alpha \) point of \( \chi_1^2 \).
	This is called a \textit{two-sided test} since there are two tails on the critical region, plotting with respect to \( \sqrt{n} \frac{\overline X - \mu_0}{\sigma_0} \).
\end{example}

\subsection{Wilks' theorem}
\begin{definition}
	The \textit{dimension} of a hypothesis \( H_0 \colon \theta \in \Theta_0 \) is the number of `free parameters' in this space.
\end{definition}
\begin{example}
	If \( \Theta_0 = \qty{\theta \in \mathbb R^k \colon \theta_1 = \dots = \theta_p = 0 } \), then the dimension of \( H_0 \) is \( k - p \).

	Let \( A \in \mathbb R^{p \times k} \) be a \( p \times k \) matrix with linearly independent rows.
	Let \( b \in \mathbb R^p \) for \( p < k \), then we define \( \Theta_0 = \qty{\theta \in \mathbb R^k \colon A\theta = b} \).
	Then the dimension of \( \theta \) is \( k - p \).

	Let \( \Theta_0 \) be a Riemannian manifold.
	We use differential geometry to deduce the dimensionality of such a manifold.
\end{example}
\begin{theorem}
	Suppose \( \Theta_0 \subset \Theta_1 \), and \( \dim \Theta_1 - \dim \Theta_0 = p \).
	Let \( X = (X_1, \dots, X_n) \) be i.i.d.\ random variables under \( f_x(\wildcard\mid\theta) \) where \( \theta \in \Theta_0^\circ \).
	Then, under some regularity conditions, as \( n \to \infty \) we have
	\[ 2\log\Lambda_x \sim \chi_p^2 \]
	More precisely, for all \( \ell \in \mathbb R_+ \),
	\[ \lim_{n \to \infty} \psub{\theta}{2\log \Lambda_x \leq \ell} = \prob{\Xi \leq \ell};\quad \Xi \sim \chi_p^2 \]
\end{theorem}
\begin{remark}
	If \( n \) is large, this theorem allows us to implement a generalised likelihood ratio test even if we cannot find the exact distribution of \( 2 \log \Lambda_x \).
	Frequentist guarantees obtained from such a test will be approximate.
\end{remark}
\begin{example}
	In the two-sided test for normal location, \( \dim \Theta_1 = 1 \) and \( \dim \Theta_0 = 0 \) hence the difference in dimensions is 1.
	Then, Wilks' theorem implies that \( 2 \log \Lambda_x \) is approximately distributed according to \( \chi_1^2 \), although the result is exact in this particular case.
\end{example}

\subsection{Goodness of fit}
Let \( X_1, \dots, X_n \) be i.i.d.\ samples taking values in \( \qty{1, \dots, k} \).
Let \( p_i = \prob{X_1 = i} \), and let \( N_i \) be the number of samples equal to \( i \), so \( \sum_i p_i = 1\) and \( \sum_i N_i = n \).
The parameters here are \( p = (p_1, \dots, p_k) \), which has \( k - 1 \) dimensions.
A \textit{goodness of fit test} has a null hypothesis of the form \( H_0 \colon p_i = \widetilde p_i \) for all \( i \), for a fixed \( \widetilde p = (\widetilde p_1, \dots, \widetilde p_k) \).
The alternative hypothesis \( H_1 \) does not constrain \( p \).

The model is \( (N_1, \dots, N_k) \sim \mathrm{Multi}(n; p_1, \dots, p_k) \).
The likelihood function is
\[ L(p) \propto p_1^{N_1} \cdots p_k^{N_k} \implies \ell(p) = \text{constant} + \sum_i N_i \log p_i \]
The generalised likelihood ratio is
\[ 2 \log \Lambda_x = 2\qty(\sup_{p \in \Theta_1} \ell(p) - \sup_{p \in \Theta_0} \ell(p)) = 2\qty(\ell(\hat p) - \ell(\widetilde p)) \]
where \( \hat p \) is the maximum likelihood estimator under \( H_1 \).
To find \( \hat p \), we typically use the method of Lagrange multipliers.
\[ \mathcal L(p, \lambda) = \sum_i N_i \log p_i - \lambda \qty(\sum p_i - 1) \]
We can compute that
\[ \hat p_i = \frac{N_i}{n} \]
This is simply the fraction of observed samples of type \( i \).

\subsection{Pearson statistic}
Let \( o_i = N_i \) be the observed number of samples of type \( i \), and \( e_i = n \widetilde p_i \) be the expected value under the null hypothesis of the number of samples of type \( i \).
Here, we can write
\[ 2 \log \Lambda = 2 \sum_i o_i \log \frac{o_i}{e_i} \]
Let \( \delta_i = o_i - e_i \).
Then
\[ 2 \log \Lambda = 2 \sum_i (e_i + \delta_i) \log\qty(1 + \underbrace{\frac{\delta_i}{e_i}}_{\text{small when } n \text{ large}}) \]
By taking the Taylor expansion, we arrive at
\[ 2 \sum_i \qty(\delta_i + \frac{\delta_i^2}{e_i} - \frac{\delta_i^2}{2e_i}) \]
Note that \( \sum_i \delta_i = \sum_i (o_i - e_i) = n - n = 0 \), so we can simplify and find
\[ \sum_i \frac{\delta_i^2}{e_i} = \sum_i \frac{(o_i - e_i)^2}{e_i} \]
This is \textit{Pearson's \( \chi^2 \) statistic}.
This is also referred to a \( \chi^2_{k-1} \) when performing a hypothesis test.
\begin{example}
	Mendel performed an experiment in which 556 different pea plants were created from a small set of ancestors.
	Each descendent was either yellow or green, and either wrinkled or smooth, giving four possibilities in total.
	The observed result was
	\[ N = \qty(\underbrace{315}_{SG}, \underbrace{108}_{SY}, \underbrace{102}_{WG}, \underbrace{31}_{WY}) \]
	Mendel's theory gives a null hypothesis \( H_0 \colon p = \widetilde p = \qty(\frac{9}{16}, \frac{3}{16}, \frac{3}{16}, \frac{1}{16}) \).
	Here,
	\[ 2 \log \Lambda = 0.618;\quad \sum_i \frac{(o_i - e_i)^2}{e_i} = 0.604 \]
	These are referred to a \( \chi^2_3 \) distribution.
	We observe that \( \chi^2_3(0.05) = 7.815 \), so we fail to reject the null hypothesis with a test of size \( 5\% \).
	We can compute that the \( p \)-value is \( \prob{\chi^2_3 > 0.6} \approx 0.96 \), so there is a very high probability of observing a more extreme value than observed.
\end{example}

\subsection{Goodness of fit for composite null}
Suppose \( H_0 \colon p_i = p_i(\theta) \) for some \( \theta \in \Theta_0 \), and \( H_1 \colon p \) has any distribution on \( \qty{1, \dots, k} \).
We can compute
\[ 2 \log \Lambda = 2 \qty( \sup_p \ell(p) - \sup_{\theta \in \Theta} \ell(p(\theta)) ) \]
We can sometimes compute these quantities explicitly, and hence find a test which refers this test statistic to a \( \chi^2_p \) distribution where \( p = \dim \Theta_1 - \dim \Theta_0 = (k-1) - \dim \Theta_0 \).
\begin{example}
	Consider a population of individuals who may have one of three genotypes, which occur with probabilities \( (p_1, p_2, p_3) = (\theta^2, 2\theta(1-\theta), (1-\theta)^2) \).
	In this case, we can find the maximum likelihood estimator under the null hypothesis to be
	\[ \hat \theta = \frac{2N_1 + N_2}{2n} \]
	Hence,
	\[ 2 \log \Lambda = 2(\ell(\hat p) - \ell(\hat \theta)) \]
	where \( \hat p_i = \frac{N_1}{n} \) as found previously.
	This can be computed explicitly and referred to a \( \chi^2_1 \) distribution.
	We can check that, in this model,
	\[ 2 \log \Lambda = \sum_i o_i \log \frac{o_i}{e_i} \]
	where \( o_i = N_i \) and \( e_i = n p_i(\hat \theta) \).
	We can approximate this using the Pearson statistic, \( \sum_i \frac{(o_i - e_i)^2}{e_i} \).
\end{example}

\subsection{Testing independence in contingency tables}
Suppose we have observations \( (X_1, Y_1), \dots, (X_n, Y_n) \) which are i.i.d., where the \( X_i \) take values in \( 1, \dots, r \) and the \( Y_i \) take values in \( 1, \dots, c \).
We wish to test whether the \( X_i \) and \( Y_i \) are independent.
We will summarise this data into a sufficient statistic known as a \textit{contingency table} \( N \), given by
\[ N_{ij} = \abs{ \qty{\ell \colon 1 \leq \ell \leq n, (X_\ell, Y_\ell) = (i,j)} } \]
So \( N_{ij} \) is the number of samples of type \( (i,j) \).
\begin{example}
	Suppose we observe \( n \) samples, and each sample has probability \( p_{ij} \) of being of type \( (i,j) \).
	Flattening \( (N_{ij}) \) into a vector, this has a multinomial distribution with parameters \( (p_{ij}) \) (also flattened into a vector).
	The null hypothesis is \( H_0 \colon p_{ij} = p_{i+} p_{+j} \) where \( p_{i+} = \sum_j p_{ij} \) and \( p_{+j} = \sum_i p_{ij} \).
	The alternative hypothesis places no restrictions on the \( p_{ij} \) apart from that it sums to 1 and has nonnegative entries.
	We find
	\[ 2 \log \Lambda = 2\sum_{i=1}^r \sum_{j=1}^c N_{ij} \log \frac{\hat p_{ij}}{\hat p_{i+} \hat p_{+j}} \]
	where \( \hat p_{ij} \) is the maximum likelihood estimator under \( H_1 \), and where \( \hat p_{i+} \) and \( \hat p_{+j} \) are the maximum likelihood estimators under \( H_0 \).
	These can be found using the method of Lagrange multipliers.
	In particular,
	\[ \hat p_{ij} = \frac{N_{ij}}{n};\quad \hat p_{i+} = \frac{N_{i+}}{n} = \frac{1}{n} \sum_{j=1}^c N_{ij};\quad \hat p_{+j} = \frac{N_{+j}}{n} = \frac{1}{n} \sum_{i=1}^r N_{ij} \]
	Writing \( o_{ij} = N_{ij} \) and \( e_{ij} = n \hat p_{i+} \hat p_{+j} \),
	\[ 2 \log \Lambda = \sum_{i,j} o_{ij} \log \frac{o_{ij}}{e_{ij}} \approx \sum_{i,j} \frac{(o_{ij} - e_{ij})^2}{e_{ij}} \]
	By Wilks' theorem, these test statistics have an approximate \( \chi^2_p \) distribution, where \( p = \dim \Theta_1 - \dim \Theta_0 = (rc-1) - (r-1 + c-1) = (r-1)(c-1) \).
\end{example}
The \( \chi^2 \) test for independence has a number of weaknesses.
\begin{enumerate}
	\item The \( \chi^2 \) approximation requires \( n \) to be large.
		A reasonable heuristic is to require \( N_{ij} \geq 5 \) for \textit{all} \( i,j \).
		If this is not possible, we can perform an \textit{exact test} (which is non-examinable).
	\item The \( \chi^2 \) test often has a low power.
		Heuristically, this is because the alternative space \( \Theta_1 \) is too large, and there are many possible models that lie in this space.
\end{enumerate}
Note that this test also applies when \( n \) is a random variable with a Poisson distribution.
This is often the case when we do not fix the number of samples.
The proof is not provided in this course.

\subsection{Testing homogeneity in contingency tables}
\begin{example}
	Suppose we perform a clinical trial on 150 patients, who are randomly assigned to one of three groups of equal size.
	The first two sets take a drug with different doses, and the third set takes a placebo.

	\begin{center}
		\begin{tabular}{c | c c c | c}
			& improved & no difference & worse & \\\hline
			placebo & 18 & 17 & 15 & 50 \\
			half dose & 20 & 10 & 20 & 50 \\
			full dose & 25 & 13 & 12 & 50
		\end{tabular}
	\end{center}

	In the previous section, we fixed the total number of samples.
	Here, we fix the total number of samples, and the total number of samples in each row.
	We suppose
	\[ N_{i1}, \dots, N_{ic} \sim \mathrm{Multinomial}(n_{i+}; p_{i1}, \dots, p_{ic}) \]
	which are independent for each row \( i \) of the table.
	The null hypothesis for homogeneity is that \( p_{1j} = p_{2j} = \dots = p_{rj} \) for all \( j \).
	The alternative hypothesis assumes that \( p_{i1}, \dots, p_{ic} \) is any arbitrary probability vector for each row \( i \).
	Under the alternative hypothesis,
	\[ L(p) = \prod_{i=1}^r \frac{n_{i+}!}{N_{i1}! \cdots N_{ic}!} p_{i1}^{N_{i1}} \cdots p_{ic}^{N_{ic}} \]
	Hence,
	\[ \ell(p) = \text{constant} + \sum_{i,j} N_{ij} \log p_{ij} \]
	This is the same likelihood as the independence test above.
	To define the maximum likelihood estimator we can again use the method of Lagrange multipliers with constraints \( \sum_j p_{ij} = 1 \) for each \( i \).
	We find
	\[ \hat p_{ij} = \frac{N_{ij}}{n_{i+}} \]
	Under the null hypothesis, we let \( p_j = p_{ij} \) for any \( i \).
	\[ \ell(p) = \text{constant} + \sum_{i,j} N_{ij} \log p_j = \sum_j N_{+j} \log p_j \]
	We have the constraint \( \sum_j p_j = 1 \).
	Using the method of Lagrange multipliers,
	\[ \hat p_j = \frac{N_{+j}}{n_{++}} \]
	Hence,
	\[ 2 \log \Lambda = 2 \sum_{i,j} N_{ij} \log \frac{\hat p_{ij}}{\hat p_j} = 2 \sum_{i,j} N_{ij} \log \frac{N_{ij}}{n_{i+} N_{+j} / n_{++}} \]
	This is precisely the same test statistic as the test for independence above.
	The only difference is that \( n_{i+} \) is fixed in this model.
	Further, if \( o_{ij} = N_{ij} \) and \( e_{ij} = n_{i+} \hat p_j = \frac{n_{i+} N_{+j}}{n_{++}} \), we have
	\[ 2 \log \Lambda = 2 \sum_{i,j} o_{ij} \log \frac{o_{ij}}{e_{ij}} \approx \sum_{i,j} \frac{(o_{ij} - e_{ij})^2}{e_{ij}} \]
	By Wilks' theorem, this is asymptotically a \( \chi^2_p \) distribution.
	Here,
	\[ p = \dim \Theta_1 - \dim \Theta_0 = r(c-1) - (c-1) = (r-1)(c-1) \]
	This is again exactly the same as in the \( \chi^2 \) test for independence.
	Operationally, the tests for homogeneity and independence are therefore completely identical; we reject the null hypothesis for one test if and only if we reject the null for the other.
	In the example above,
	\[ 2 \log \Lambda = 5.129;\quad \sum_{i,j} \frac{(o_{ij} - e_{ij})^2}{e_{ij}} = 5.173 \]
	Referring this to a \( \chi^2_4 \) distribution, the upper \( 0.05 \)-point is \( 9.488 \).
	Hence, we do not reject the null hypothesis at the 5\% significance level.
\end{example}

\subsection{Tests and confidence sets}
\begin{definition}
	The \textit{acceptance region} \( A \) of a test is the complement of the critical region.
\end{definition}
\begin{theorem}
	Let \( X \sim f_X(\wildcard\mid\theta) \) for some \( \theta \in \Theta \).
	Suppose that for each \( \theta_0 \in \Theta \), there exists a test of size \( \alpha \) with acceptance region \( A(\theta_0) \) for the  null hypothesis \( \theta = \theta_0 \).
	Then
	\[ I(X) = \qty{\theta \colon X \in A(\theta)} \]
	is a \( 100(1-\alpha)\% \) confidence set.

	Now suppose there exists a set \( I(X) \) which is a \( 100(1-\alpha)\% \) confidence set for \( \theta \).
	Then
	\[ A(\theta_0) = \qty{x \colon \theta_0 \in I(x)} \]
	is the acceptance region of a test of size \( \alpha \) for the hypothesis \( \theta = \theta_0 \).
\end{theorem}
\begin{proof}
	Observe that for both parts of the theorem,
	\[ \theta_0 \in I(X) \iff X \in A(\theta_0) \iff \text{fail to reject } H_0 \text{ with data } X \]
	For the first part, we assume that \( \psub{\theta}{\text{fail to reject } H_0 \text{ with data } X} = 1 - \alpha \), and we want to show \( \psub{\theta}{\theta_0 \in I(X)} = 1 - \alpha \).
	The second part is the converse.
\end{proof}
\begin{example}
	Let \( X_1, \dots, X_n \sim N(\mu, \sigma_0^2) \) be i.i.d.\ with \( \sigma_0^2 \) known and \( \mu \) unknown.
	We found that a \( 100(1-\alpha)\% \) confidence interval for \( \mu \) is
	\[ I(X) = \qty(\overline X \pm \frac{Z_{\alpha/2}\sigma_0}{\sqrt{n}}) \]
	Hence, by the second part of the theorem above, we can find a test for \( H_0 \colon \mu = \mu_0 \) with size \( \alpha \) by
	\[ A(\mu_0) = \qty{ x \colon \mu_0 \in I(x) } = \qty{ x \colon \mu_0 \in \qty[\overline x \pm \frac{Z_{\alpha/2} \sigma_0}{\sqrt{n}}] } \]
	This is equivalent to rejecting \( H_0 \) when
	\[ \abs{\sqrt{n} \frac{\mu_0 - \overline X}{\sigma_0}} > Z_{\alpha/2} \]
	This is a two-sided test for normal location.
\end{example}
