\subsection{Multivariate normal distribution}
Let \( X = (X_1, \dots, X_n) \) be a vector of random variables.
Then we define
\[
	\expect{X} = \begin{pmatrix}
		\expect{X_1} \\
		\vdots       \\
		\expect{X_n}
	\end{pmatrix};\quad \Var{X} = \qty( \expect{\qty(X_i - \expect{X_i})\qty(X_j - \expect{X_j})} )_{i,j}
\]
The familiar linearity results are
\[
	\expect{AX+b} = A\expect{X} + b;\quad A \Var{X} A^\transpose
\]
where \( A \in \mathbb R^{k \times n}, b \in \mathbb R^k \) are constant.
\begin{definition}
	We say that \( X \) has a \textit{multivariate normal distribution} if, for any fixed \( t \in \mathbb R^n \), we have \( t^\transpose X \sim N(\mu, \sigma^2) \) for some parameters \( \mu, \sigma^2 \).
\end{definition}
\begin{proposition}
	Let \( X \) be multivariate normal.
	Then \( AX+b \) is multivariate normal, where \( A \in \mathbb R^{k \times n}, b \in \mathbb R^k \) are constant.
\end{proposition}
\begin{proof}
	Let \( t \in \mathbb R^k \).
	Then,
	\[
		t^\transpose (Ax+b) = \underbrace{(A^\transpose t)^\transpose X}_{\sim N(\mu, \sigma^2)} + t^\transpose b
	\]
	which is the sum of a normal random variable and a constant.
	So this is \( N(\mu + t^\transpose b, \sigma^2) \).
\end{proof}
\begin{proposition}
	A multivariate normal distribution is fully specified by its mean and covariance matrix.
\end{proposition}
\begin{proof}
	Let \( X_1, X_2 \) be multivariate normal vectors with the same mean \( \mu \) and the same covariance matrix \( \Sigma \).
	We will show that these two random variables have the same moment generating function, and hence the same distribution.
	\[
		M_{X_1}(t) = \expect{ e^{1 \cdot t^\transpose X_1} }
	\]
	Note that \( t^\transpose X_1 \) is univariate normal.
	Hence, this is equal to
	\[
		M_{X_1}(t) = \exp(1 \cdot \expect{t^\transpose X_1} + \frac{1}{2} \Var{t^\transpose X_1} \cdot 1^2) = \exp(t^\transpose \mu + \frac{1}{2} t^\transpose \Sigma t)
	\]
	This depends only on \( \mu \) and \( \Sigma \), and we obtain the same moment generating function for \( X_2 \).
\end{proof}

\subsection{Orthogonal projections}
\begin{definition}
	A matrix \( P \in \mathbb R^{n \times n} \) is an \textit{orthogonal projection} onto its column space \( \mathrm{col}(P) \) if, for all \( v \in \mathrm{col}(P) \), we have \( Pv = v \), and for all \( w \in \mathrm{col}(P)^\perp \), we have \( Pw = 0 \).
\end{definition}
\begin{proposition}
	\( P \) is an orthogonal projection if and only if it is idempotent and symmetric.
\end{proposition}
\begin{proof}
	If \( P \) is idempotent and symmetric, let \( v \in \mathrm{col}(P) \), so \( v = Pa \) for some \( a \in \mathbb R^n \).
	Then, \( Pv = PPa = Pa = v \).
	Now, let \( w \in \mathrm{col}(P)^\perp \).
	By definition, \( P^\transpose w = 0 \).
	By symmetry, \( Pw = 0 \).

	Now, suppose \( P \) is an orthogonal projection.
	Any vector \( a \in \mathbb R^n \) can be uniquely written as \( a = v + w \) where \( v \in \mathrm{col}(P) \) and \( w \in \mathrm{col}(P)^\perp \).
	Then \( PPa = PPv + PPw = Pv = P(v+w) = Pa \).
	As this holds for all \( a \), we have that \( P \) is idempotent.
	Let \( u_1, u_2 \in \mathbb R^n \), and note \( (P u_1) \cdot ((I-P) u_2) = 0 \), as \( P u_1 \in \mathrm{col}(P) \) and \( (I-P) u_2 \in \mathrm{col}(P)^\perp \).
	We have \( u_1^\transpose P^\transpose (I-P) u_2 = 0 \).
	Since this holds for all \( u_1, u_2 \), \( P^\transpose (I-P) = 0 \) so \( P^\transpose = P^\transpose P \).
	Note that \( P^\transpose P \) is symmetric, so \( P^\transpose \) is symmetric, and hence \( P \) is symmetric.
\end{proof}
\begin{corollary}
	Let \( P \) be an orthogonal projection matrix.
	Then \( I-P \) is also an orthogonal projection matrix.
\end{corollary}
\begin{proof}
	Clearly, if \( P \) is symmetric, so is \( I-P \), so it suffices to prove idempotence.
	We have \( (I-P)(I-P) = I - 2P + P^2 = I - 2P + P = I - P \) as required.
\end{proof}
\begin{proposition}
	If \( P \) is an orthogonal projection, then \( P = UU^\transpose \) where the columns of \( U \) are an orthonormal basis for the column space of \( P \).
\end{proposition}
\begin{proof}
	First, we show that \( UU^\transpose \) is an orthogonal projection.
	This is clearly symmetric.
	It is idempotent: \( U U^\transpose U U^\transpose = U U^\transpose \) since \( U^\transpose U = I \), as the columns of \( U \) form an orthonormal basis for the column space of \( P \).
	Further, the column space of \( P \) is exactly the column space of \( U U^\transpose \).
\end{proof}
\begin{proposition}
	The rank of an orthogonal projection matrix is equal to its trace.
\end{proposition}
\begin{proof}
	The rank is the dimension of the column space, which is \( \rank P = \rank(U^\transpose U) = \tr(U^\transpose U) = \tr(U U^\transpose) = \tr P \).
\end{proof}
\begin{theorem}
	Let \( X \) be multivariate normal, where \( X \sim N(0,\sigma^2 I) \), and let \( P \) be an orthogonal projection.
	Then
	\begin{enumerate}
		\item \( PX \sim N(0,\sigma^2 P) \), and \( (I-P)X \sim N(0,\sigma^2(I-P)) \), and these two random variables are independent;
		\item \( \frac{\norm{PX}^2}{\sigma^2} \sim \chi^2_{\rank P} \).
	\end{enumerate}
\end{theorem}
\begin{proof}
	The vector \( (P, I-P)^\transpose X \) is multivariate normal, since it is a linear function of \( X \).
	This distribution is fully specified by its mean and variance.
	\[
		\expect{\begin{pmatrix}
				PX \\
				(I-P)X
			\end{pmatrix}} = \begin{pmatrix}
			P \\
			I-P
		\end{pmatrix} \expect{X} = 0
	\]
	Further,
	\[
		\Var{\begin{pmatrix}
				PX \\
				(I-P)X
			\end{pmatrix}} = \begin{pmatrix}
			P \\
			I - P
		\end{pmatrix} \sigma^2 I \begin{pmatrix}
			P \\
			I - P
		\end{pmatrix} = \sigma^2 \begin{pmatrix}
			P^2    & P(I-P)  \\
			P(I-P) & (I-P)^2
		\end{pmatrix}^\transpose = \sigma^2 \begin{pmatrix}
			P & 0   \\
			0 & I-P
		\end{pmatrix}
	\]
	Now we must show that the variables \( PX, (I-P)X \) are independent.
	Let \( Z \sim N(0,\sigma^2 P), Z' \sim N(0,\sigma^2(I-P)) \) be independent.
	Then we can see that \( (Z, Z')^\transpose \) is multivariate normal with
	\[
		\mu = 0;\quad \Sigma = \begin{pmatrix}
			P & 0     \\
			0 & I - P
		\end{pmatrix}
	\]
	Hence \( (PX, (1-P)X)^\transpose \) is equal in distribution to \( (Z, Z')^\transpose \).
	So \( PX \) is independent of \( (I-P)X \).

	We must show that \( \frac{\norm{PX}^2}{\sigma^2} \sim \chi^2_{\rank P} \).
	Note that
	\[
		\frac{\norm{PX}^2}{\sigma^2} = \frac{X^\transpose P^\transpose P X}{\sigma^2} = \frac{X^\transpose \qty(U U^\transpose)^\transpose U U^\transpose}{\sigma^2} = \frac{\norm{U^\transpose X}^2}{\sigma^2}
	\]
	Note, \( U^\transpose X \sim N(0,\sigma^2 U^\transpose U) = N(0,\sigma^2 I_{\rank P}) \).
	So
	\[
		\frac{(U^\transpose X)_i}{\sigma} \overset{\text{iid}}{\sim} N(0,1)
	\]
	for \( i = 1, \dots, \rank P \).
	Hence
	\[
		\frac{\norm{PX}^2}{\sigma^2} = \sum_{i=1}^{\rank P} \qty(\frac{(U^\transpose X)_i}{\sigma})^2 \sim \chi^2_{\rank P}
	\]
\end{proof}
\begin{theorem}
	Let \( X_1, \dots, X_n \overset{\text{iid}}{\sim} N(\mu,\sigma^2) \) for some unknown \( \mu \in \mathbb R \) and \( \sigma^2 > 0 \).
	The maximum likelihood estimators for \( \mu \) and \( \sigma \) are
	\[
		\hat \mu = \overline X = \frac{1}{n} \sum_i X_i;\quad \hat \sigma^2 = \frac{S_{xx}}{n} = \frac{\sum_i \qty(X_i - \overline X)^2}{n}
	\]
	Further,
	\begin{enumerate}
		\item \( \overline X \sim N\qty(\mu, \frac{\sigma^2}{n}) \);
		\item \( \frac{S_{xx}}{\sigma^2} \sim \chi^2_{n-1} \);
		\item \( \overline X, S_{xx} \) are independent.
	\end{enumerate}
\end{theorem}
\begin{proof}
	Let \( P \) be the square \( n \times n \) matrix with all entries \( \frac{1}{n} \).
	This is an orthogonal projection matrix, as it is symmetric and idempotent.
	Note that
	\[
		PX = \begin{pmatrix}
			\overline X \\
			\vdots      \\
			\overline X
		\end{pmatrix}
	\]
	We will write the observations \( X \) as
	\[
		X = \underbrace{\begin{pmatrix}
				\mu    \\
				\vdots \\
				\mu
			\end{pmatrix}}_{M} + \varepsilon;\quad \varepsilon \sim N(0,\sigma^2 I)
	\]
	Note that \( \overline X \) is a function of \( P \varepsilon \), since \( \overline X = (PX)_1 = (PM + P\varepsilon)_1 \).
	Further,
	\[
		S_{xx} = \sum_i \qty(X_i - \overline X)^2 = \norm{X - PX}^2 = \norm{(I-P)X}^2 = \norm{(I-P)\varepsilon}^2
	\]
	Hence \( S_{xx} \) is a function of \( (I-P)\varepsilon \).
	Since \( P\varepsilon \) and \( (I-P)\varepsilon \) are independent, \( \overline X \) and \( S_{xx} \) are independent.
	Since \( I-P \) is a projection with rank equal to its trace \( n-1 \), we apply the previous theorem to obtain
	\[
		S_{xx} = \norm{(I-P)\varepsilon}^2 \chi^2_{n-1}
	\]
\end{proof}

\subsection{Linear model}
Suppose we have data in pairs \( (x_1, Y_1), \dots, (x_n, Y_n) \), where \( Y_i \in \mathbb R, x_i \in \mathbb R^p \).
The \( Y_i \) are known as the \textit{response} variables, or the \textit{dependent} variables.
The \( x_{i1}, x_{ip} \) are the \textit{predictors}, or \textit{independent} variables.
We will model the expectation of the response \( Y_i \) as a linear function of the predictors \( (x_{i1}, \dots, x_{ip}) \).
\begin{example}
	Let \( Y_i \) be the number of insurance claims that driver \( i \) makes in a given year, and \( x_{i1}, \dots, x_{ip} \) is a set of variables about the specific driver.
	Predictors include age, the number of years they have held their license, and the number of points on their license, for instance.
\end{example}
We assume that
\[
	Y_i = \alpha + \beta_1 x_{i1} + \dots + \beta_p x_{ip} + \varepsilon_i
\]
where \( \alpha \in \mathbb R \) is an \textit{intercept}, \( \beta_i \) are the \textit{coefficients}, and \( \varepsilon \) is a \textit{noise vector}, which is a random variable.
The intercept and coefficients are the parameters of interest.
We will often eliminate the intercept by making one of the predictors \( x_{i1} = 1 \) for all \( i \), so \( \beta_1 \) plays the role of the intercept.

Note that we can use a linear model to model nonlinear relationships.
For example, suppose \( Y_i = a + bz_i + cz_i^2 + \varepsilon_i \).
We can rephrase this as a linear model with \( x_i = (1, z_i, z_i^2) \).

The coefficient \( \beta_j \) can be interpreted as the effect on \( Y_i \) of increasing \( x_{ij} \) by one, while keeping all other predictors fixed.
This cannot be interpreted as a causal relationship, unless this is a randomised control experiment.

\subsection{Matrix formulation}
Let
\[
	Y = \begin{pmatrix}
		Y_1    \\
		\vdots \\
		Y_n
	\end{pmatrix};\quad X = \begin{pmatrix}
		x_{11} & \cdots & x_{1p} \\
		\vdots & \ddots & \vdots \\
		x_{n1} & \cdots & x_{np}
	\end{pmatrix};\quad \beta = \begin{pmatrix}
		\beta_1 \\
		\vdots  \\
		\beta_p
	\end{pmatrix};\quad \varepsilon = \begin{pmatrix}
		\varepsilon_1 \\
		\vdots        \\
		\varepsilon_n
	\end{pmatrix}
\]
We call \( X \) the \textit{design matrix}.
The linear model is that
\[
	Y = X\beta + \varepsilon
\]
\( X\beta \) is considered fixed.
Since \( \varepsilon \) is random, this makes \( Y \) into a random variable.

\subsection{Assumptions}
We make a number of \textit{moment assumptions} on the noise vector \( \varepsilon \).
This allows us to deduce more results about the linear model.
\begin{enumerate}
	\item \( \expect{\varepsilon} = 0 \implies \expect{Y_i} = x_i^\transpose \beta \);
	\item \( \Var \varepsilon = \sigma^2 I \), which is equivalent to both \( \Var \varepsilon_i = \sigma^2 \) and \( \Cov{\varepsilon_i, \varepsilon_j} = 0 \) for all \( i \neq j \).
	      This property is known as \textit{homoscedasticity}.
\end{enumerate}
We will always assume that the design matrix \( X \) has full rank \( p \), or equivalently, that it has linearly independent columns.
Since \( X \in \mathbb R^{n \times p} \), this requires that \( n \geq p \), so we need at least as many samples as we have predictors.

\subsection{Least squares estimation}
\begin{definition}
	The \textit{least squares estimator} \( \hat \beta \) minimises the \textit{residual sum of squares}, which is
	\[
		S(\beta) = \norm{Y-X\beta}^2 = \sum_i \qty(Y_i - x_i^\transpose \beta)^2
	\]
	The term \( Y_i - x_i^\transpose \beta \) is called the \text{\( i \)th residual}.
\end{definition}
Since \( S(\beta) \) is a positive definite quadratic in \( \beta \), it is minimised at the stationary point.
\[
	\eval{\pdv{S(\beta)}{\beta_k}}_{\beta = \hat \beta} = 0 \iff \forall k,\;-2\sum_{i=1}^n x_{ik} \qty(Y_i - \sum_k x_{ij} \hat \beta_j) = 0 \iff X^\transpose X \hat \beta = X^\transpose Y
\]
As \( X \) has full column rank, \( X^\transpose X \) is invertible.
\[
	\hat \beta = (X^\transpose X)^{-1} X^\transpose Y
\]
This is notably a linear function of \( Y \), given fixed \( X \).
Note that
\[
	\expect{\hat \beta} = (X^\transpose X)^{-1} X^\transpose \expect{Y} = (X^\transpose X)^{-1} X^\transpose X\beta = \beta
\]
So \( \hat \beta \) is an unbiased estimator.
Further,
\begin{align*}
	\Var{\hat \beta} & = (X^\transpose X)^{-1} X^\transpose \Var Y \qty[(X^\transpose X)^{-1} X^\transpose]^\transpose     \\
	                 & = (X^\transpose X)^{-1} X^\transpose \sigma^2 I \qty[(X^\transpose X)^{-1} X^\transpose]^\transpose \\
	                 & = \sigma^2 (X^\transpose X)^{-1}
\end{align*}
\begin{theorem}[Gauss-Markov theorem]
	Let an estimator \( \beta^\star \) of \( \beta \) be unbiased and a linear function of \( Y \), so \( \beta^\star = CY \).
	Then, for any fixed \( t \in \mathbb R^p \), we have
	\[
		\Var{t^\transpose \hat \beta} \leq \Var{t^\transpose \beta^\star}
	\]
	where \( \hat \beta \) is the least squares estimator.
	We say that \( \hat \beta \) is the best linear unbiased estimator (BLUE).
\end{theorem}
\begin{remark}
	We can think of \( t \in \mathbb R^p \) as a vector of predictors for a new sample.
	Then \( t^\transpose \hat \beta \) is the prediction for \( \expect{Y_i} \) for this new sample, using the least squares estimator.
	\( t^\transpose \beta^\star \) is the prediction with \( \beta^\star \).
	In both cases, the prediction is unbiased.
\end{remark}
\begin{proof}
	Note that
	\[
		\Var{t^\transpose \beta^\star} - \Var{t^\transpose \hat \beta} = t^\transpose \qty[\Var{\beta^\star} - \Var{\hat\beta}] t
	\]
	To prove that this quantity is always non-negative, we must show that \( \Var{\beta^\star} - \Var{\hat\beta} \) is positive semidefinite.
	Let \( A = C - (X^\transpose X)^{-1} X^\transpose \).
	Note that \( \expect{AY} = \expect{\beta^\star} - \expect{\hat\beta} = 0 \).
	Also, \( \expect{AY} = A \expect{Y} = AX\beta \).
	This holds for all \( \beta \), so \( AX = 0 \).
	Now, since \( X^\transpose X \) is symmetric,
	\begin{align*}
		\Var{\beta^\star}                   & = \Var{CY}                                                                                                                          \\
		                                    & = \Var{(A+(X^\transpose X)^{-1} X^\transpose)Y}                                                                                     \\
		                                    & = \qty[A+(X^\transpose X)^{-1} X^\transpose] \Var Y \qty[A+(X^\transpose X)^{-1} X^\transpose]^\transpose                           \\
		                                    & = \qty[A+(X^\transpose X)^{-1} X^\transpose] \sigma^2 I \qty[A+(X^\transpose X)^{-1} X^\transpose]^\transpose                       \\
		                                    & = \sigma^2 \qty(A A^\transpose + (X^\transpose X)^{-1} + AX(X^\transpose X)^{-1} + (X^\transpose X)^{-1} X^\transpose A^\transpose) \\
		                                    & = \sigma^2 AA^\transpose + \Var{\hat \beta}                                                                                         \\
		\Var{\beta^\star} - \Var{\hat\beta} & = \sigma^2 AA^\transpose
	\end{align*}
	Note that the outer product \( A A^\transpose \) is always positive semidefinite.
\end{proof}

\subsection{Fitted values and residuals}
\begin{definition}
	The \textit{fitted values} are \( \hat Y = X \hat \beta = X(X^\transpose X)^{-1} X^\transpose Y \), where \( P = X (X^\transpose X)^{-1} X^\transpose \) is the \textit{hat matrix}.
	The \textit{residuals} are \( Y - \hat Y = (I-P)Y \).
\end{definition}
\begin{proposition}
	\( P \) is the orthogonal projection onto the column space of the design matrix.
\end{proposition}
\begin{proof}
	If \( v \) is in the column space of \( X \), then \( v = Xb \) for some \( b \).
	Hence
	\[
		Pv = X(X^\transpose X)^{-1} X^\transpose X b = Xb = v
	\]
	If \( w \) is in the orthogonal complement, then
	\[
		Pw = X(X^\transpose X)^{-1} \underbrace{X^\transpose w}_{0} = 0
	\]
\end{proof}
\begin{corollary}
	The fitted values are an orthogonal projection of the response variables to the column space of the design matrix.
	The residuals are orthogonal to the column space.
\end{corollary}

\subsection{Normal linear model}
The normal linear model is a linear model under the assumption that \( \varepsilon \sim N(0,\sigma^2 I) \), where \( \sigma^2 \) is unknown.
The parameters in the model are now \( (\beta, \sigma^2) \).
The likelihood function in the normal linear model is
\[
	L(\beta, \sigma^2) = f_Y(y\mid \beta,\sigma^2) = (2\pi \sigma^2)^{-\frac{n}{2}} \exp{-\frac{1}{2\sigma^2} \sum_i (Y_i - x_i^\transpose \beta)^2}
\]
The log-likelihood is
\[
	\ell(\beta,\sigma^2) = \text{constant} - \frac{n}{2}\log \sigma^2 - \frac{1}{2\sigma^2} \norm{Y-X\beta}^2
\]
To maximise this as a function of \( \beta \) for any fixed \( \sigma^2 \), we must minimise the residual sum of squares \( S(\beta) = \norm{Y-X\beta}^2 \).
So \( \hat \beta = (X^\transpose X)^{-1} X^\transpose Y \) is the maximum likelihood estimator of \( \beta \).
Further, \( \hat \sigma^2 = n^{-1} \norm{Y-X\hat\beta}^2 = n^{-1} \norm{\hat Y - Y}^2 = n^{-1} \norm{(I-P)Y}^2 \).
\begin{theorem}
	In the normal linear model,
	\begin{enumerate}
		\item \( \hat \beta \sim N(\beta,\sigma^2(X^\transpose X)^{-1}) \);
		\item \( n\frac{\hat\sigma^2}{\sigma^2} \sim \chi^2_{n-p} \);
		\item \( \hat \beta, \hat \sigma^2 \) are independent.
	\end{enumerate}
\end{theorem}
\begin{proof}
	We prove each part separately.
	\begin{enumerate}
		\item We already know that \( \expect{\hat \beta} = \beta \), and \( \Var{\hat \beta} = \sigma^2 (X^\transpose X)^{-1} \).
		      So it suffices to show that \( \hat \beta \) is a normal vector.
		      Since \( \hat \beta = (X^\transpose X)^{-1} X^\transpose Y \), it is a linear function of a normal vector, so is a normal vector.
		\item Observe that
		      \[
			      n\frac{\hat\sigma^2}{\sigma^2} = \frac{\norm{(I-P)Y}^2}{\sigma^2} = \frac{\norm{(I-P)(X\beta+\varepsilon)}^2}{\sigma^2}
		      \]
		      Since \( (I-P)X = 0 \) as \( P \) is the orthogonal projection onto the column space of \( X \),
		      \[
			      n\frac{\hat\sigma^2}{\sigma^2} = \frac{\norm{(I-P)\varepsilon}^2}{\sigma^2} \sim \chi^2_{\tr (I-P)}
		      \]
		      where \( \tr (I-P) = \tr I - \tr P = n - p \) since \( X \in \mathbb R^{n \times p} \) is assumed to have full rank.
		\item Note that \( \hat \sigma^2 \) is a function of \( (I-P)\varepsilon \), and
		      \begin{align*}
			      \hat\beta & = (X^\transpose X)^{-1} X^\transpose Y                    \\
			                & = (X^\transpose X)^{-1} X^\transpose (X\beta+\varepsilon) \\
			                & = \beta + (X^\transpose X)^{-1} X^\transpose \varepsilon  \\
			                & = \beta + (X^\transpose X)^{-1} X^\transpose P\varepsilon
		      \end{align*}
		      is a function of \( P\varepsilon \).
		      Since \( (I-P)\varepsilon \) and \( P \varepsilon \) are independent, so are \( \hat \beta, \hat \sigma^2 \).
	\end{enumerate}
\end{proof}
Note,
\[
	\expect{\frac{n\hat\sigma^2}{\sigma^2}} = \expect{\chi^2_{n-p}} = n-p \implies \expect{\hat\sigma^2} = \sigma^2 \cdot \frac{n-p}{n} < \sigma^2
\]
Hence this \( \hat \sigma^2 \) is a biased estimator, but asymptotically unbiased.

\subsection{Inference}
\begin{definition}
	Let \( U \sim N(0,1) \) and \( V \sim \chi^2_n \) be independent random variables.
	Then
	\[
		T = \frac{U}{\sqrt{\frac{V}{n}}}
	\]
	has a \textit{\( t_n \)-distribution}.
\end{definition}
As \( n \to \infty \), this approaches the standard normal distribution.
\begin{definition}
	Let \( V \sim \chi^2_n \) and \( W \sim \chi^2_m \) be independent random variables.
	Then
	\[
		F = \frac{V/n}{W/m}
	\]
	has an \textit{\( F_{n,m} \)-distribution}.
\end{definition}
\begin{example}
	We consider a \( 100(1-\alpha)\% \) confidence interval for one of the coefficients \( \beta \) in the normal linear model \( Y = X\beta + \varepsilon \).
	Without loss of generality, we will consider \( \beta_1 \).

	We begin by finding a \textit{pivot}, which is a distribution that does not depend on the parameters of the model.
	By standardising the above form of \( \hat \beta \),
	\[
		\frac{\beta_1 - \hat \beta_1}{\sqrt{\sigma^2(X^\transpose X)^{-1}_{11}}} \sim N(0,1)
	\]
	where \( M^{-1}_{11} \) is the top left entry in the matrix \( M^{-1} \).
	This random variable is independent from \( \frac{n\hat \sigma^2}{\sigma^2} \sim \chi^2_{n-p} \).
	Now, to construct a pivot, we find
	\[
		\frac{\frac{\beta_1 - \hat \beta_1}{\sqrt{\sigma^2(X^\transpose X)^{-1}_{11}}}}{\sqrt{\frac{\hat \sigma^2}{\sigma^2} \cdot \frac{n}{n-p}}} \sim \frac{U}{\sqrt{\frac{V}{n}}} \sim t_{n-p}
	\]
	The \( \sigma^2 \) terms cancel, so the statistic is a function only of \( \beta_1 \) and functions of the data.
	Then,
	\[
		\psub{\beta,\sigma^2}{ -t_{n-p}\qty(\frac{\alpha}{2}) \leq \frac{\hat \beta_1 - \beta_1}{\sqrt{(X^\transpose X)^{-1}_{11}}} \sqrt{\frac{n-p}{n\hat\sigma^2}} \leq t_{n-p}\qty(\frac{\alpha}{2}) } = 1-\alpha
	\]
	since the \( t \) distribution is symmetric about zero.
	Rearranging to find an interval for \( \beta_1 \),
	\[
		\psub{\beta,\sigma^2}{
			\hat \beta_1 - t_{n-p}\qty(\frac{\alpha}{2}) \frac{\sqrt{(X^\transpose X)^{-1}_{11} \hat\sigma^2}}{\sqrt{(n-p)/n}}
			\leq \beta_1 \leq
			\hat \beta_1 + t_{n-p}\qty(\frac{\alpha}{2}) \frac{\sqrt{(X^\transpose X)^{-1}_{11} \hat\sigma^2}}{\sqrt{(n-p)/n}}
		} = 1-\alpha
	\]
	Hence,
	\[
		I = \qty[\hat \beta_1 \pm t_{n-p}\qty(\frac{\alpha}{2}) \frac{\sqrt{(X^\transpose X)^{-1}_{11} \hat \sigma^2}}{\sqrt{(n-p)/n}} ]
	\]
	is a \( 100(1-\alpha)\% \) confidence interval for \( \beta_1 \).

	Consider a test for \( H_0 \colon \beta_1 = 0 \), \( H_1 \colon \beta_1 \neq 0 \).
	By connecting tests and confidence intervals, we can test \( H_0 \) with size \( \alpha \) by rejecting this null hypothesis when zero is not contained within the confidence interval \( I \).

	Consider a special case where \( Y_1, \dots, Y_n \overset{\text{iid}}{\sim} N(\mu,\sigma^2) \) where \( \mu, \sigma^2 \) are unknown, and we want to infer results about \( \mu \).
	Note that this is a special case of the normal linear model where
	\[
		X = \begin{pmatrix}
			1      \\
			\vdots \\
			1
		\end{pmatrix};\quad \beta = \begin{pmatrix}
			\mu
		\end{pmatrix}
	\]
	So we can infer a confidence interval for \( \mu \) using the above statistic.
\end{example}
\begin{example}
	Consider a \( 100(1-\alpha)\% \) confidence set for \( \beta \) as a whole.
	Note that
	\[
		\hat \beta - \beta \sim N(0,\sigma^2 (X^\transpose X)^{-1})
	\]
	Then,
	\[
		(X^\transpose X)^{1/2} (\hat \beta - \beta) \sim N(0,\sigma^2(X^\transpose X)^{1/2} (X^\transpose X)^{-1} (X^\transpose X)^{1/2}) \sim N(0,\sigma^2 I)
	\]
	where \( (X^\transpose X)^{1/2} \) is obtained using the eigendecomposition of the positive definite matrix \( X^\transpose X \).
	Hence,
	\[
		\frac{\norm{(X^\transpose X)^{1/2} (\hat \beta - \beta)}^2}{\sigma^2} \sim \chi^2_p
	\]
	We can also write this as
	\[
		\frac{\norm{(X^\transpose X)^{1/2} (\hat \beta - \beta)}^2}{\sigma^2} = \frac{\norm{X(\hat \beta - \beta)}^2}{\sigma^2}
	\]
	Since this is a function of \( \hat \beta \), this is independent of any function of \( \hat \sigma^2 \).
	In particular, it is independent of \( \frac{n\hat\sigma^2}{\sigma^2} \sim \chi^2_{n-p} \).
	Thus, we can form a pivot by
	\[
		\frac{\norm{X(\hat \beta - \beta)}^2 / (\sigma^2 p)}{\hat\sigma^2 n / (\sigma^2(n-p))} \sim \frac{\chi^2_p / p}{\chi^2_{n-p} / (n-p)} \sim F_{p,n-p}
	\]
	This does not depend on \( \sigma^2 \).
	For all \( \beta, \sigma^2 \),
	\[
		\psub{\beta,\sigma^2}{
			\frac{\norm{X(\hat \beta - \beta)}^2 / p}{\hat\sigma^2 n / (n-p)}
			\leq F_{p,n-p}(\alpha)} = 1-\alpha
	\]
	because the \( F \) distribution has support only on the positive real line.
	It is nontrivial to express this as a region for \( \beta \) since it is vector-valued.
	We can say, however, that
	\[
		\qty{\beta' \in \mathbb R^p \colon \frac{\norm{X(\hat \beta - \beta)}^2/p}{\hat \sigma^2 n/(n-p)} \leq F_{p,n-p}(\alpha)}
	\]
	is a \( 100(1-\alpha)\% \) confidence set for \( \beta \).

	This set is an ellipsoid centred at \( \hat \beta \).
	The shape of the ellipsoid depends on the design matrix \( X \); the principal axes are given by eigenvectors of \( X^\transpose X \).
\end{example}
The above two results are exact; no approximations were made.

\subsection{F-tests}
We wish to test whether a collection of predictors \( \beta_i \) are equal to zero.
Without loss of generality, we will take the first \( p_0 \leq p \) predictors.
We have \( H_0 \colon \beta_1 = \dots = \beta_{p_0} = 0 \), and \( H_1 = \beta \in \mathbb R^p \).
We denote \( X = (X_0, X_1) \) as a block matrix with \( X_0 \in \mathbb R^{n \times p_0} \) and \( X_1 \in \mathbb R^{n \times (p-p_0)} \), and we denote \( \beta = (\beta^0, \beta^1)^\transpose \) similarly.
The null model has \( \beta^0 = 0 \).
This is a linear model \( Y = X\beta + \varepsilon = X_1 \beta^1 + \varepsilon \).
We will write \( P = X(X^\transpose X)^{-1} X^\transpose \) and \( P_1 = X_1 (X_1^\transpose X_1)^{-1} X_1^\transpose \).
Note that as \( X \) and \( P \) have full rank, so must \( X_1, P_1 \).
\begin{lemma}
	\( (I-P)(P - P_1) = 0 \), and \( P - P_1 \) is an orthogonal projection with rank \( p_0 \).
\end{lemma}
\begin{proof}
	\( P - P_1 \) is symmetric since \( P \) and \( P_1 \) are symmetric.
	It is also idempotent, since
	\[
		(P-P_1)(P-P_1) = P^2 - P_1 P - P P_1 + P_1^2 = P - P_1 - P_1 + P_1 = P - P_1
	\]
	since \( P_1 \) projects onto the column space of \( X_1 \).
	Hence \( P - P_1 \) is indeed an orthogonal projection matrix.
	The rank is \( \rank(P - P_1) = \tr(P - P_1) = \tr P - \tr P_1 = p - (p - p_0) = p_0 \).
	Also,
	\[
		(I-P)(P-P_1) = P-P_1 - P + PP_1 = P-P_1 - P + P_1 = 0
	\]
\end{proof}
Recall that the maximum log-likelihood in the normal linear model is given by
\[
	\ell(\hat \beta, \hat \sigma^2) = \frac{-n}{2} \log \hat\sigma^2 - \frac{n}{2} \cdot \text{constant} = \frac{-n}{2} \log \frac{\norm{(I-P)Y}^2}{n} + \text{constant}
\]
The generalised likelihood ratio statistic is
\begin{align*}
	2 \log \Lambda & = 2 \sup_{\beta \in \mathbb R^p,\sigma^2>0} \ell(\beta, \sigma^2) - 2 \sup_{\beta_0 = 0, \beta_1 \in \mathbb R^{p-p_0},\sigma^2>0} \ell(\beta, \sigma^2) \\
	               & = n \qty[ -\log \frac{\norm{(I-P)Y}^2}{n} + \log \frac{\norm{(I-P_1)Y}^2}{n} ]
\end{align*}
Wilks' theorem applies here, showing that \( 2 \log \Lambda \sim \chi^2_{p_0} \) asymptotically as \( n \to \infty \) with \( p, p_0 \) fixed.
However, we can find an exact test, so using Wilks' theorem will not be necessary.
\( 2 \log \Lambda \) is monotone in
\begin{align*}
	\frac{\norm{(I-P_1)Y}^2}{\norm{(I-P)Y}^2} & = \frac{\norm{(I-P+P-P_1)Y}^2}{\norm{(I-P)Y}^2}                                             \\
	                                          & = \frac{\norm{(I-P)Y}^2 + \norm{(P-P_1)Y}^2 + 2Y^\transpose (I-P)(P-P_1)}{\norm{(I-P)Y}^2 } \\
	                                          & = \frac{\norm{(I-P)Y}^2 + \norm{(P-P_1)Y}^2}{\norm{(I-P)Y}^2}                               \\
	                                          & = 1 + \frac{\norm{(P-P_1)Y}^2}{\norm{(I-P)Y}^2}
\end{align*}
The generalised likelihood ratio test rejects when the \( F \)-statistic
\[
	F = \frac{\norm{(P-P_1)Y}^2}{\norm{(I-P)Y}^2} \cdot \frac{1/p_0}{1/(n-p)}
\]
is large.
\begin{theorem}
	Under \( H_0\colon \beta_1 = \dots = \beta_{p_0} = 0 \), in the normal linear model,
	\[
		F = \frac{\norm{(P-P_1)Y}^2}{\norm{(I-P)Y}^2} \cdot \frac{1/p_0}{1/(n-p)} \sim F_{p_0,n-p}
	\]
\end{theorem}
\begin{proof}
	Recall that
	\[
		\norm{(I-P)Y}^2 = \norm{(I-P)\varepsilon}^2 \sim \chi^2_{n-p} \cdot \sigma^2
	\]
	Therefore it suffices to show that \( \norm{(P-P_1)Y}^2 \) is an independent \( \chi^2_{p_0} \cdot \sigma^2 \) random variable.
	Under \( H_0 \), we have that
	\[
		(P-P_1)Y = (P-P_1)(X\beta+\varepsilon) = (P-P_1)(X_1 \beta^1 + \varepsilon) = (P-P_1)\varepsilon
	\]
	since \( P, P_1 \) preserve \( X_1 \).
	Hence, \( \norm{(P-P_1)Y}^2 = \norm{(P-P_1)\varepsilon}^2 \sim \chi^2_{\rank (P-P_1)} \cdot \sigma^2 = \chi^2_{p_0} \cdot \sigma^2 \).
	We must now show independence between \( (I-P)Y \) and \( (P-P_1)Y \).
	The vectors \( (I-P)\varepsilon, (P-P_1)\varepsilon \) are independent; indeed,
	\[
		E = \begin{pmatrix}
			(I-P)\varepsilon \\
			(P-P_1)\varepsilon
		\end{pmatrix}
	\]
	is a multivariate normal vector, and
	\[
		\expect{E} = 0;\quad \Var E = \begin{pmatrix}
			I-P          & (I-P)(P-P_1) \\
			(I-P)(P-P_1) & P-P_1
		\end{pmatrix} = \begin{pmatrix}
			I-P & 0     \\
			0   & P-P_1
		\end{pmatrix}
	\]
	and since \( (I-P)\varepsilon \) and \( (P-P_1)\varepsilon \) are elements of a multivariate normal vector and are uncorrelated, they are independent as required.
\end{proof}
The generalised likelihood ratio test of size \( \alpha \) rejects \( H_0 \) when \( F > F^{-1}_{p_0,n-p}(\alpha) \).
This is an exact test for all \( n, p, p_0 \).
Previously, we found a test for \( H_0\colon \beta_1 = 0 \) against \( H_1 \colon \beta_1 \neq 0 \).
This is a special case of the \( F \)-test derived above, where \( p_0 = 1 \).
The previous test of size \( \alpha \) rejects \( H_0 \) when
\[
	\abs{\hat\beta} > t_{n-p}\qty(\frac{\alpha}{2}) \sqrt{\frac{\hat\sigma^2 n (X^\transpose X)^{-1}_{11}}{n-p}}
\]
We will show that these two tests are equivalent; they reject \( H_0 \) in the same critical region.
The \( t \)-test rejects if and only if
\[
	\hat \beta_1^2 > t_{n-p}\qty(\frac{\alpha}{2})^2 \frac{\hat\sigma^2 n (X^\transpose X)^{-1}_{11}}{n-p}
\]
Note that \( t_{n-p}\qty(\frac{\alpha}{2})^2 = F_{1,n-p}(\alpha) \), since
\[
	U \sim N(0,1);\;W \sin \chi^2_n \implies T = \frac{U}{\sqrt{W/n}} \implies T^2 = \frac{U^2}{W/n} = \frac{V/1}{W/n} \sim F_{1,n}
\]
where \( V \sim \chi^2_1 \).
Hence,
\[
	\frac{\hat \beta_1/(X^\transpose X)^{-1}_{11}}{\hat\sigma^2 n/(n-p)} > F_{1,n-p}(\alpha)
\]
It suffices to show that
\[
	\frac{\hat \beta_1}{(X^\transpose X)^{-1}_{11}} = \frac{\norm{(P-P_1)Y}^2}{\underbrace{p_0}_{=1}};\quad \frac{\hat \sigma^2 n}{n-p} = \frac{\norm{(I-P)Y}^2}{n-p}
\]
We have already shown the latter part.
For \( \hat \beta_1 \), note that in this case, \( P - P_1 \) is a projection of rank 1 onto the one-dimensional subspace spanned by the vector \( v = (I-P)X^0 \) where \( X^0 \) is the first column in the matrix \( X \).
First, note the following identity.
\[
	X_0^\transpose (I - P_1) = v^\transpose = v^\transpose (P-P_1) = X_0^\transpose (I-P_1)(P-P_1) = X_0^\transpose (I-P_1)P
\]
Then,
\begin{align*}
	\norm{(P-P_1)Y}^2 & = \norm{\frac{v}{\norm{v}} \qty(\frac{v}{\norm{v}})^\transpose Y}^2                                 \\
	                  & = \frac{(v^\transpose Y)^2}{\norm{v}^2} = \frac{(X_0^\transpose (I-P_1) Y)^2}{\norm{(I-P_1) X_0}^2} \\
	                  & = \frac{(X_0^\transpose (I-P_1) P Y)^2}{\norm{(I-P_1) X_0}^2}                                       \\
	                  & = \frac{(X_0^\transpose (I-P_1) X \hat\beta)^2}{\norm{(I-P_1) X_0}^2}
\end{align*}
Note that \( (I-P_1) X = [(I-P_1)X_0, 0, \dots, 0] \).
Hence,
\begin{align*}
	\norm{(P-P_1)Y}^2 & = \frac{\norm{(I-P_1)X_0}^4 \hat \beta_1}{\norm{(I-P_1)X_0}^2} \\
	                  & = \norm{(I-P_1)X_0}^2 \hat \beta_1
\end{align*}
Finally, we show that
\[
	(X^\transpose X)^{-1}_{11} = \frac{1}{\norm{(I-P_1)X_0}^2}
\]
using the Woodbury identity for blockwise matrix inversion.
Hence,
\[
	\frac{\hat\beta_1^2}{(X^\transpose X)^{-1}_{11}} = \norm{(P-P_1) Y}^2
\]
as required.

\subsection{Analysis of variance}
Suppose we investigate responses of patients after receiving one of three treatments, including a control, which will be given index 1.
We will consider only one predictor, denoting which treatment a given patient received.
Consider the linear model
\[
	Y_{ij} = \alpha + \mu_j + \varepsilon_{ij}
\]
where \( j = 1, 2, 3 \) is the treatment index, and \( i = 1, \dots, N \) is the index of a patient in a given group.
Let \( (\varepsilon_{ij}) \sim N(0, \sigma^2) \) be independent.
Without loss of generality, we can set \( \mu_1 = 0 \), since we have an additional parameter \( \alpha \); this is known as a \textit{corner point} constraint.
Then, \( \mu_j \) should be interpreted as the effect of treatment \( j \) relative to treatment 1, which in this case is the control.

\begin{definition}
	The \textit{analysis of variance (ANOVA)} test on the linear model
	\[
		Y_{ij} = \alpha + \mu_j + \varepsilon_{ij}
	\]
	where \( \mu_1 = 0 \) is given by
	\[
		H_0 \colon \mu_2 = \mu_3 = \dots = 0;\quad H_1 \colon \mu_2, \mu_3, \dots \in \mathbb R
	\]
	In particular, \( H_0 \) gives \( \expect{Y_{ij}} = \alpha \).
\end{definition}

In our example, \( H_0 \colon \mu_2 = \mu_3 = 0 \) and \( H_1 \colon \mu_2, \mu_3 \in \mathbb R \).
This is a special case of the \( F \)-test, since we are testing whether the coefficients \( \mu_i \) are equal to zero.
\[
	X = \begin{pmatrix}
		1      & 0      & 0      \\
		1      & 0      & 0      \\
		\vdots & \vdots & \vdots \\
		1      & 1      & 0      \\
		1      & 1      & 0      \\
		\vdots & \vdots & \vdots \\
		1      & 0      & 1      \\
		1      & 0      & 1
	\end{pmatrix} = \begin{pmatrix}
		X_1 & X_0
	\end{pmatrix}
\]
The first column of \( X \), denoted \( X_1 \), represents \( \alpha \), and the other columns, denoted \( X_0 \), represent \( \mu_2, \mu_3 \).
\( X_0 \) is eliminated under the null hypothesis.
The predictor can be called \textit{categorical}; it is discrete, and entirely dependent on which treatment category a given patient is placed in.
Note that \( X \) has \( 3N \) rows, where each block of \( N \) consecutive rows is identical.
Recall that the \( F \)-test uses the test statistic
\[
	F = \frac{\norm{(P-P_1)Y}^2}{\norm{(I-P)Y}^2} \cdot \frac{1/p_0}{1/(n-p)} \sim F_{p_0,n-p}
\]
For this test, \( P \) projects onto the space of vectors in \( \mathbb R^{3N} \) which are constant over treatment groups.
In other words, let
\[
	\overline Y_j = \frac{1}{N} \sum_{i=1}^N Y_{ij}
\]
Then,
\[
	P Y = \qty( \underbrace{\overline Y_1, \dots, \overline Y_1}_{N \text{ entries}}, \underbrace{\overline Y_2, \dots, \overline Y_2}_{N \text{ entries}}, \underbrace{\overline Y_3, \dots, \overline Y_3}_{N \text{ entries}} )^\transpose
\]
\( P_1 \) projects onto the subspace of constant vectors in \( \mathbb R^{3N} \), so
\[
	\overline Y = \frac{1}{3N} \sum_{i=1}^N \sum_{j=1}^3 Y_{ij} \implies P_1 Y = \qty( \underbrace{\overline Y, \dots, \overline Y}_{3N \text{ entries}} )^\transpose
\]
Hence, we can write the \( F \) statistic as
\[
	F = \frac{\sum_{j=1}^3 N \qty(\overline Y_j - \overline Y)^2 / 2}{\sum_{i=1}^N \sum_{j=1}^3 \qty(Y_{ij} - \overline Y_j)^2 / (3N-3)}
\]
We can generatlise this to the case where there are \( J > 3 \) treatment groups:
\[
	F = \frac{\sum_{j=1}^J N \qty(\overline Y_j - \overline Y)^2 / (J-1)}{\sum_{i=1}^N \sum_{j=1}^J \qty(Y_{ij} - \overline Y_j)^2 / (JN-J)} = \frac{\text{variance between treatments}}{\text{variance within treatments}}
\]
\begin{remark}
	This test is sometimes called \textit{one-way} analysis of variance.
	\textit{Two-way} analysis of variance is a similar analysis in an experiment where groups are defined according to two variables.
	For instance, the response could be a student's performance in an exam, where the treatments are
	\begin{enumerate}
		\item completion of supervisions (zero representing not complete, one representing complete); and
		\item whether a monetary incentive was given (zero representing no incentive, one representing an incentive).
	\end{enumerate}
	Here, we would have the result \( Y_{ijk} \) as the number of marks of student \( i \) in group \( (j,k) \).
	The model would be
	\[
		Y_{ijk} = \alpha + \mu_j + \lambda_k + \varepsilon_{ijk}
	\]
	with a constraint without loss of generality that \( \mu_0 = \lambda_0 = 0 \).
	The two-way analysis of variance test is then
	\[
		H_0 \colon \mu_1 = \lambda_1 = 0;\quad H_1 \colon \mu_1, \lambda_1 \in \mathbb R
	\]
\end{remark}

\subsection{Simple linear regression}
In a linear regression model, we often centre predictors to simplify certain expressions.
\[
	Y_i = \alpha + \beta (x - \overline x) + \varepsilon_i
\]
where \( \overline x = \frac{1}{n} \sum_{i=1}^n x_i \), and the \( \varepsilon_i \) independently have the usual \( N(0, \sigma^2) \) distribution.
In this case, the maximum likelihood estimator \( (\hat \alpha, \hat \beta) \) takes a simple form.
Recall that \( (\hat \alpha, \hat \beta) \) minimises
\[
	S(\alpha, \beta) = \sum_{i=1}^n \qty(Y_i - \alpha - \beta(x_i - \overline x))^2
\]
Hence,
\[
	\pdv{S(\alpha, \beta)}{\alpha} = \sum_{i=1}^n -2\qty(Y_i - \alpha - \beta(x_i - \overline x)) = \sum_{i=1}^n -2\qty(Y_i - \alpha)
\]
This gives the simple expression
\[
	\alpha = \frac{\sum_{i=1}^n Y_i}{n} = \overline Y
\]
Now,
\[
	\eval{\pdv{S(\alpha, \beta)}{\beta}}_{\alpha = \hat \alpha} = \sum_{i=1}^n -2\qty(Y_i - \overline Y - \beta(x_i - \overline x))(x_i - \overline x)
\]
This vanishes when
\[
	\hat \beta = \frac{\sum_{i=1}^n \qty(Y_i - \overline Y)(x_i - \overline x)}{\sum_{i=1}^n (x_i - \overline x)^2} = \frac{S_{xy}}{S_{xx}}
\]
Note that \( \frac{S_{xy}}{n} \) is the sample covariance of \( X \) and \( Y \), and \( \frac{S_{xx}}{n} \) is the sample variance of \( X \).
