\subsection{Periodic Functions}
A function \( f(x) \) is \textit{periodic} if \( f(x+T) = f(x) \) for all \( x \), where \( T \) is the period.
For example, simple harmonic motion is periodic.
In space, we consider the wavelength \( \lambda = \frac{2\pi}{k} \), and the (angular) wave number \( k \) is defined conversely by \( k = \frac{2\pi}{\lambda} \).

\subsection{Properties of Trigonometric Functions}
Consider the set of functions
\[
	g_n(x) = \cos \frac{n\pi x}{L};\quad h_n(x) = \sin \frac{n\pi x}{L}
\]
where \( n \in \mathbb N \).
These functions are periodic with period \( T = 2L \).
Recall that
\begin{align*}
	\cos A \cos B & = \frac{1}{2}\qty(\cos(A-B) + \cos(A+B)); \\
	\sin A \sin B & = \frac{1}{2}\qty(\cos(A-B) - \cos(A+B)); \\
	\sin A \cos B & = \frac{1}{2}\qty(\sin(A-B) + \sin(A+B))
\end{align*}

\subsection{Periodic Function Space}
We define the inner product
\[
	\inner{f, g} = \int_0^{2L} f(x) g(x) \dd{x}
\]
The functions \( g_n \) and \( h_n \) are mutually orthogonal on the interval \( [0, 2L) \) with respect to the inner product above.
\begin{align*}
	\inner{h_n, h_m} & = \int_0^{2L} \sin \frac{n\pi x}{L} \sin \frac{m\pi x}{L} \dd{x}                                                           \\
	                 & = \frac{1}{2} \int_0^{2L} \qty(\cos\frac{(n-m)\pi x}{L} - \cos \frac{(n+m)\pi x}{L}) \dd{x}                                \\
	                 & = \frac{1}{2} \frac{L}{\pi} \qty[ \frac{1}{n-m}\sin\frac{(n-m)\pi x}{L} - \frac{1}{n+m} \sin \frac{(n+m)\pi x}{L} ]_0^{2L} \\
	                 & = 0 \text{ when } n \neq m
\end{align*}
If \( n = m \), we have
\[
	\inner{h_n, h_n} = \int_0^{2L} \sin^2 \frac{n\pi x}{L} \dd{x} = \frac{1}{2} \int_0^{2L} \qty(1-\cos \frac{2\pi n x}{L}) \dd{x} = L
\]
Thus,
\[
	\inner{h_n, h_m} = \begin{cases}
		L \delta_{nm} & n,m \neq 0 \\
		0             & nm = 0
	\end{cases}
\]
Similarly, we can show
\[
	\inner{g_n, g_m} = \begin{cases}
		L \delta_{nm} & n,m \neq 0                                 \\
		0             & \text{exactly one of } m,n \text{ is zero} \\
		2L            & n,m = 0
	\end{cases}
\]
and
\[
	\inner{h_n, g_m} = 0
\]
Now, we assert that \( \qty{g_n, h_n} \) form a complete orthogonal set; they span the space of all `well-behaved' periodic functions of period \( 2L \).
Further, the set \( \qty{g_n, h_n} \) is linearly independent.

\subsection{Fourier Series}
Since \( g_n, h_n \) span the space of `well-behaved' periodic functions of period \( 2L \), we can express any such function as a sum of such eigenfunctions.
\begin{definition}
	The Fourier series of \( f \) is
	\[
		f(x) = \frac{1}{2}a_0 + \sum_{n=1}^\infty a_n \cos \frac{n \pi x}{L} + \sum_{n=1}^\infty b_n \sin \frac{n \pi x}{L}
	\]
	where \( a_n, b_n \) are constants such that the right hand side is convergent for all \( x \) where \( f \) is continuous.
	At a discontinuity \( x \), the Fourier series approaches the midpoint of the supremum and infimum of the function in a close neighbourhood of \( x \).
	That is, we replace the left hand side with
	\[
		\frac{1}{2}f(x_+) + \frac{1}{2}f(x_-)
	\]
\end{definition}
\noindent Let \( m > 0 \), and consider taking the inner product \( \inner{h_m, f} \) and substituting the Fourier series of \( f \).
\begin{align*}
	\inner{h_m, f} & = \int_0^{2L} \sin\frac{m \pi x}{L} f(x) \dd{x} \\
	               & = \inner{h_m, b_m h_m}                          \\
	               & = L b_m
\end{align*}
Thus,
\[
	b_n = \frac{1}{L} \inner{h_n, f} = \frac{1}{L} \int_0^{2L} \sin\frac{n \pi x}{L} f(x) \dd{x}
\]
and analogously
\[
	a_n = \frac{1}{L} \inner{g_n, f} = \frac{1}{L} \int_0^{2L} \cos\frac{n \pi x}{L} f(x) \dd{x}
\]
Note that \( \frac{1}{2} a_0 \) is the average of the function.
Note further that we may integrate over any range as long as the total length is one period, \( 2L \).
Notably, we may integrate over the interval \( [-L, L] \).

\begin{example}
	Consider the \textit{sawtooth wave}; defined by \( f(x) = x \) for \( x \in [-L, L) \) and periodic elsewhere.
	Here,
	\[
		a_n = \frac{1}{L} \int_{-L}^L x \cos \frac{n\pi x}{L} \dd{x} = 0
	\]
	and
	\begin{align*}
		b_n & = \frac{2}{L} \int_0^L x \sin \frac{n\pi x}{L} \dd{x}                                                       \\
		    & = \frac{-2}{n\pi} \qty[x \cos \frac{n\pi x}{L}]_0^L + \frac{2}{n\pi} \int_0^L \cos \frac{n \pi x}{L} \dd{x} \\
		    & = \frac{-2L}{n\pi} \cos n \pi + \frac{2L}{(n\pi)^2} \sin n \pi                                              \\
		    & = \frac{2L}{n\pi} (-1)^{n+1}
	\end{align*}
\end{example}
