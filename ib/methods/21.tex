\subsection{Characteristics of a first order PDE}
Consider
\[
	\alpha(x,y) \pdv{\phi}{x} + \beta(x,y) \pdv{\phi}{y} = 0
\]
with Cauchy data on an initial curve \( B \), defined by \( (x(t), y(t)) \):
\[
	\phi(x(t), y(t)) = f(t)
\]
Note,
\[
	\alpha \phi_x + \beta \phi_y = u \cdot \grad{\phi} = \eval{\dv{\phi}{s}}_C
\]
This is exactly the directional derivative along the integral curve \( C \), defined by \( u = (\alpha, \beta) \).
Since \( \dv{\phi}{s} = \alpha \phi_x + \beta \phi_y = 0 \) from the original PDE, the function \( \phi(x,y) \) is constant along this curve \( C \).
In other words, the Cauchy data \( f(t) \) defined on \( B \) at \( s = 0 \) is propagated constantly along the integral curves.
This gives the solution
\[
	\phi(s,t) = \phi(x(s,t), y(s,t)) = f(t)
\]
To obtain \( \phi \) in the original coordinates, we need to transform from \( s,t \)-space into \( x,y \)-space.
Provided that the Jacobian \( J = x_t y_s - x_s y_t \) is non-zero, we can invert the transformation and find \( s,t \) as functions of \( x,y \).
This gives
\[
	\phi(x,y) = f(t(x,y))
\]
To solve such a PDE, we will typically use the following steps.
\begin{enumerate}[(i)]
	\item Find the characteristic equations \( \dv{x}{s} = \alpha, \dv{y}{s} = \beta \).
	\item Parametrise the initial conditions on \( B \) by \( (x(t), y(t)) \).
	\item Solve the characteristic equations to find \( x = x(s,t) \) and \( y = y(s,t) \) subject to the initial conditions at \( s = 0 \).
	\item Solve the equation for \( \phi \) given by \( \dv{\phi}{x} = \alpha \phi_x + \beta \phi_y = 0 \), so \( \phi \) is constant along the integral curves, giving \( \phi(s,t) = f(t) \).
	\item Invert the relations \( s = s(x,y) \) and \( t = t(x,y) \), then find \( \phi \) in terms of \( x,y \).
\end{enumerate}
\begin{example}
	Consider the equation
	\[
		\dv{\phi(x,y)}{x} = 0
	\]
	such that
	\[
		\phi(0,y) = h(y)
	\]
	The characteristic equations are given by
	\[
		\dv{x}{s} = \alpha = 1;\quad \dv{y}{s} = \beta = 0
	\]
	The initial curve \( B \) is given by
	\[
		(x(t), y(t)) = (0,t)
	\]
	Solving the characteristic equations,
	\[
		x = s + c(t);\quad y = d(t)
	\]
	At \( x = 0 \), we must have \( s = 0 \), so \( c = 0 \).
	Further, \( y = t \) hence \( d = t \).
	Thus,
	\[
		x = s;\quad y = t
	\]
	Thus,
	\[
		\dv{\phi}{x} = 0 \implies \phi(s,t) = h(t) \implies \phi(x,y) = h(y)
	\]
\end{example}
\begin{example}
	Consider
	\[
		e^x \phi_x + \phi_y = 0;\quad \phi(x,0) = \cosh x
	\]
	The characteristic equations are
	\[
		\dv{x}{s} = e^x;\quad \dv{y}{s} = 1
	\]
	The initial conditions are
	\[
		x(t) = t;\quad y(t) = 0
	\]
	We solve the characteristic equation subject to these initial conditions, giving
	\[
		-e^{-x} = s + c(t);\quad y = s + d(t)
	\]
	\( s = 0 \) implies \( -e^{-t} = c(t) \) and \( y = 0 = d(t) \).
	Hence
	\[
		e^{-x} = e^{-t} - s;\quad y = s
	\]
	Now,
	\[
		\dv{\phi}{s} = 0 \implies \phi(s,t) = \cosh t
	\]
	Since \( s = y, e^{-t} = y + e^{-x} \), we have \( t = -\log(y + e^{-x}) \).
	Thus,
	\[
		\phi(x,y) = \cosh[-\log(y + e^{-x})]
	\]
\end{example}

\subsection{Inhomogeneous first order PDEs}
Suppose we now wish to solve
\[
	\alpha(x,y) \phi_x + \beta(x,y) \phi_y = \gamma(x,y)
\]
with Cauchy data \( \phi(x(t), y(t)) = f(t) \) along a curve \( B \).
The characteristic curves are the same as the homogeneous case.
However, the directional derivative no longer vanishes:
\[
	\eval{\dv{\phi}{s}}_C = u \cdot \grad{\phi} = \gamma(x,y)
\]
where \( \phi = f(t) \) at \( s = 0 \) on \( B \).
So \( f(t) \) is no longer propagated constantly across characteristic polynomials, but is instead propagated according to the ODE in \( s \) above.
We must therefore solve this ODE along \( C \) before reverting to \( x,y \) coordinates.
\begin{example}
	Consider
	\[
		\phi_x + 2 \phi_y = ye^x;\quad \phi(x,x) = \sin x
	\]
	The characteristic equation is given by
	\[
		\dv{x}{s} = 1;\quad \dv{y}{s} = 2
	\]
	The initial conditions are
	\[
		x(t) = y(t) = t
	\]
	From the characteristic equations,
	\[
		x = s + c(t);\quad y = 2s + d(t)
	\]
	Thus,
	\[
		x = t = c(t);\quad y = t = d(t)
	\]
	So the solutions to the characteristics are
	\[
		x = s + t;\quad y = 2s + t
	\]
	Now we solve
	\[
		\dv{\phi}{s} = \gamma = y e^x = (2s+t)e^{s+t}
	\]
	Note that \( \dv{s} \qty(2se^s) = 2e^s + 2se^s \), so the solution is
	\[
		\phi(s,t) = (2s - 2 + t)e^{s+t} + c(s)
	\]
	for some constant term \( c(s) \).
	But \( \phi(0,t) = \sin t \), hence
	\[
		\sin t = (t-2)e^t + c(s) \implies \phi(s,t) = (2s-2+t)e^{s+t} + \sin t - (2-t)e^t
	\]
	Inverting into \( x,y \) space,
	\[
		\phi(x,y) = (y-2)e^x + (y-2x+2)e^{2x-y} + \sin(2x-y)
	\]
\end{example}

\subsection{Classication of second order PDEs}
In two dimensions, the general second order PDE is
\begin{align*}
	\mathcal L \phi & \equiv a(x,y) \pdv[2]{\phi}{x} + 2 b(x,y) \pdv{\phi}{x}{y} + c(x,y) \pdv[2]{\phi}{y} \\
	                & + d(x,y) \pdv{\phi}{x} + e(x,y) \pdv{\phi}{y} + f(x,y) \phi(x,y)
\end{align*}
The \textit{principal part} is given by
\[
	\sigma_P (x,y,k_x,k_y) \equiv k^\transpose A k = \begin{pmatrix}
		k_x & k_y
	\end{pmatrix} \begin{pmatrix}
		a(x,y) & b(x,y) \\
		b(x,y) & c(x,y)
	\end{pmatrix} \begin{pmatrix}
		k_x \\ k_y
	\end{pmatrix}
\]
The PDE is classified by the properties of the eigenvalues of \( A \).
\begin{enumerate}[(i)]
	\item If \( b^2 - ac < 0 \), the equation is \textit{elliptic}.
	      The eigenvalues have the same sign.
	      An example is the Laplace equation.
	\item If \( b^2 - ac > 0 \), the equation is \textit{hyperbolic}.
	      The eigenvalues have opposite signs.
	      An example is the wave equation.
	\item If \( b^2 - ac = 0 \), the equation is \textit{parabolic}, where at lease one eigenvalue is zero.
	      An example is the heat equation.
\end{enumerate}
Note that a differential equation may have different classifications at different points \( (x,y) \) in space.

\subsection{Characteristic curves of second order PDEs}
A curve defined by \( f(x,y) \) constant is a characteristic if
\[
	\begin{pmatrix}
		f_x & f_y
	\end{pmatrix} \begin{pmatrix}
		a & b \\
		b & c
	\end{pmatrix} \begin{pmatrix}
		f_x \\ f_y
	\end{pmatrix} = 0
\]
This is a generalisation of the first order case \( u \cdot \grad{f} = 0 \) where \( u = (\alpha, \beta) \).
The curve can be written as \( y = y(x) \) by the chain rule.
\[
	\pdv{f}{x} + \pdv{f}{y} \dv{y}{x} = 0 \implies \frac{f_x}{f_y} = -\dv{y}{x}
\]
Substituting into the quadratic form,
\[
	a \qty(\dv{y}{x})^2 - 2b \dv{y}{x} + c = 0
\]
for which we have a quadratic solution given by
\[
	\dv{y}{x} = \frac{b \pm \sqrt{b^2 - ac}}{a}
\]
\begin{enumerate}[(i)]
	\item Hyperbolic equations have two such solutions, since \( b^2 - ac > 0 \).
	\item Parabolic equations have one solution.
	\item Elliptic equations have no real characteristics.
\end{enumerate}
