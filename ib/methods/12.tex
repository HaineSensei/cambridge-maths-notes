\subsection{Cylindrican polar coordinates}
In cylindrical coordinates,
\[ \frac{1}{r} \pdv{r} \qty(r \pdv{\phi}{r}) + \frac{1}{4^2} \pdv[2]{\phi}{\theta} + \pdv[2]{\phi}{z} = 0 \]
With \( \phi = R(r) \Theta(\theta) Z(z) \), we find
\[ \Theta'' = -\mu \Theta;\quad Z'' = \lambda Z;\quad r(rR')' + (\lambda r^2 - \mu) R = 0 \]
The polar equation can be easily solved by
\[ \mu_m = m^2;\quad \Theta_m(\theta) = \cos m\theta, \sin m\theta \]
The radial equation is Bessel's equation, giving solutions
\[ R = J_m(kr), Y_m(kr) \]
Setting boundary conditions in the usual way, defining \( R=0 \) at \( r = a \) means that
\[ J_m(ka) = 0 \implies k = \frac{j_{mn}}{a} \]
The radial solution is
\[ R_{mn}(r) = J_m\qty(\frac{j_{mn}}{a}r) \]
We have eliminated the \( Y_n \) term since we require \( r = 0 \) to give a finite \( \phi \).
Finally, the \( z \) equation gives
\[ Z'' = k^2 Z \implies Z = e^{-kz}, e^{kz} \]
We typically eliminate the \( e^{kz} \) mode due to boundary conditions, such as \( Z \to 0 \) as \( z \to \infty \).
The general solution is therefore
\[ \phi(r,\theta,z) = \sum_{m=0}^\infty \sum_{n = 1}^\infty \qty(a_{mn} \cos m\theta + b_{mn} \sin m\theta) J_m\qty(\frac{j_{mn}}{a} r) e^{-frac{j_{mn}r}{a}} \]

\subsection{Spherical polar coordinates}
In spherical polar coordinates,
\[ \frac{1}{r^2} \pdv{r} \qty(r^2 \pdv{\Phi}{r}) + \frac{1}{r^2 \sin\theta}\pdv{\theta} \qty(\sin\theta \pdv{\Phi}{\theta}) + \frac{1}{r^2 \sin^2 \theta} \pdv[2]{\Phi}{\phi} = 0 \]
We will consider the \textit{axisymmetric case}; supposing that there is no \( \phi \) dependence.
We seek a separable solution of the form
\[ \Phi(r,\theta) = R(r) \Theta(\theta) \]
which gives
\[ (\sin\theta \Theta')' + \lambda \sin\theta \Theta = 0;\quad (r^2R')' - \lambda R = 0 \]
Consider the substitution \( x = \cos\theta, \dv{x}{\theta} = -\sin\theta \) in the polar equation.
This gives \( \dv{\Theta}{\theta} = -\sin\theta \dv{\Theta}{x} \) and hence
\[ -\sin\theta \dv{x}\qty[-\sin^2\theta \dv{\Theta}{x}] + \lambda \sin\theta \Theta = 0 \implies \dv{x}[(1-x^2)\dv{\Theta}{x}] + \lambda \Theta = 0 \]
This gives Legendre's equation, so it has solutions of eigenvalues \( \lambda_\ell = \ell (\ell + 1) \) and eigenfunctions
\[ \Theta_\ell(\theta) = P_\ell(x) = P_\ell(\cos\theta) \]
The radial equation then gives
\[ (r^2 R')' - \ell (\ell + 1) R = 0 \]
We will seek power law solutions: \( R = \alpha r^\beta \).
This gives
\[ \beta(\beta + 1) - \ell(\ell + 1) = 0 \implies \beta = \ell, \beta = -\ell - 1 \]
Thus the radial eigenmodes are
\[ R_\ell = r^{\ell}, r^{-\ell - 1} \]
Therefore the general axisymmetric solution for spherical polar coordinates is
\[ \Phi(r,\theta) = \sum_{\ell = 0}^\infty (a_\ell r^{\ell} + b_\ell r^{-\ell - 1}) P_\ell(\cos\theta) \]
The \( a_\ell, b_\ell \) are determined by the boundary conditions.
Orthogonality conditions for the \( P_\ell \) can be used to determine coefficients.
Consider a solution to Laplace's equation on the unit sphere with axisymmetric boundary conditions given by
\[ \Phi(1,\theta) = f(\theta) \]
Given that we wish to find the interior solution, \( b_n = 0 \) by regularity.
Then,
\[ f(\theta) = \sum_{\ell=0}^\infty a_\ell P_\ell(\cos\theta) \]
By defining \( f(\theta) = F(\cos\theta) \),
\[ F(x) = \sum_{\ell=0}^\infty a_\ell P_\ell(x) \]
We can then find the coefficients in the usual way, giving
\[ a_\ell = \frac{2\ell + 1}{2} \int_{-1}^1 F(x) P_{\ell}(x) \dd{x} \]

\subsection{Generating function for Legendre polynomials}
Consider a charge at \( r_0 = (x,y,z) = (0,0,1) \).
Then, the potential at a point \( P \) becomes
\begin{align*}
    \Phi(r) &= \frac{1}{\abs{r - r_0}} = \frac{1}{(x^2 + y^2 + (x-1)^2)^{1/2}} \\
    &= \frac{1}{(r^2 (\sin^2 \phi + \cos^2 \phi) \sin^2 \theta + r^2 \cos^2 \theta - 2r \cos\theta + 1)^{1/2}} \\
    &= \frac{1}{(r^2 \sin^2 \theta + r^2 \cos^2 \theta - 2r \cos\theta + 1)^{1/2}} \\
    &= \frac{1}{(r^2 - 2r \cos\theta + 1)^{1/2}} \\
    &= \frac{1}{(r^2 - 2r \overline x + 1)^{1/2}}
\end{align*}
where \( \overline x \equiv \cos \theta \).
This function \( \Phi \) is a solution to Laplace's equation where \( r \neq r_0 \).
Note that we can represent any axisymmetric solution as a sum of Legendre polynomials.
Now,
\[ \frac{1}{\sqrt{r^2 - 2rx + 1}} = \sum_{\ell = 0}^\infty a_\ell P_\ell(x) r^\ell \]
With the normalisation condition for the Legendre polynomials \( P_\ell(1) = 1 \), we find
\[ \frac{1}{1-r} = \sum_{\ell=0}^\infty a_\ell r^\ell \]
Using the geometric series expansion, we arrive at \( a_\ell = 1 \).
This gives
\[ \frac{1}{\sqrt{r^2 - 2rx + 1}} = \sum_{\ell = 0}^\infty P_\ell(x) r^\ell \]
which is the generating function for the Legendre polynomials.
