\subsection{???}
Consider scalar functions \( \phi, \psi \) which are twice differentiable on a domain \( D \).
By the divergence theorem, \textit{Green's first identity} is
\[
	\int_D \div(\phi \grad{\psi}) \dd[3]{r} = \int_D \qty(\phi \laplacian{\psi} + \grad{\phi} \cdot \grad{\psi}) \dd[3]{r} = \int_{\partial D} \phi \grad{\psi} \cdot \hat n \dd{S}
\]
Switching \( \psi \) and \( \phi \) and subtracting from the above, we arrive at \textit{Green's second identity}:
\[
	\int_{\partial D} \qty(\phi \pdv{\psi}{\hat n} - \psi \pdv{\phi}{\hat n}) \dd{S} = \int_D \qty(\phi \laplacian{\psi} - \psi \laplacian{\phi}) \dd[3]{r}
\]
Suppose we remove a ball \( \mathcal B_\varepsilon(r') \) from the domain.
Without loss of generality let \( r' = 0 \).
Let \( \phi \) be a solution to Poisson's equation, so \( \laplacian{\phi} = -\rho \) and let \( \psi \) be the free-space Green's function.
Thus, the right hand side of the second identity becomes
\[
	\int_{D \setminus \mathcal B_\varepsilon} \qty(\phi \laplacian{G_{\mathrm{FS}}} - G_{\mathrm{FS}} \laplacian{\phi}) \dd[3]{r} = \int_{D \setminus \mathcal B_\varepsilon} G_{\mathrm{FS}} \rho \dd[3]{r}
\]
The left hand side is
\[
	\int_{\partial D} \qty(\phi \pdv{G_{\mathrm{FS}}}{\hat n} - G_{\mathrm{FS}} \pdv{\phi}{\hat n}) \dd{S} + \int_{\partial \mathcal B_\varepsilon} \qty(\phi \pdv{G_{\mathrm{FS}}}{\hat n} - G_{\mathrm{FS}} \pdv{\phi}{\hat n}) \dd{S}
\]
For the second integral, we take the limit as \( \varepsilon \to 0 \).
Let \( \phi \) be regular, and let \( \overline\phi \) be the average value and \( \overline{\pdv{\phi}{\hat n}} \) be the average derivative.
This integral then becomes
\[
	\qty(\overline\phi \frac{-1}{4 \pi \varepsilon^2} - \frac{1}{4 \pi \varepsilon} \overline{\pdv{\phi}{\hat n}}) 4 \pi \varepsilon^2 \to -\phi(0)
\]
Combining the above, we find \textit{Green's third identity}, which is
\[
	\phi(r') = \int_D G_{\mathrm{FS}}(r;r')  \qty(-\rho(r)) \dd[3]{r} + \int_{\partial D} \qty(\phi(r) \pdv{G_{\mathrm{FS}}}{\hat n} \qty(r;r') - G_{\mathrm{FS}}(r;r') \pdv{\phi}{\hat n}\qty(r)) \dd{S}
\]
The second integral provides the ability to use inhomogeneous boundary conditions

\subsection{Dirichlet Green's function}
We will solve Poisson's equation \( \laplacian{\phi} = -\rho \) on \( D \) with inhomogeneous boundary conditions \( \phi(r) = h(r) \) on \( \partial D \).
The Dirichlet Green's function satisfies
\begin{enumerate}[(i)]
	\item \( \laplacian{G(r;r')} = 0 \) for all \( r \neq r \);
	\item \( G(r;r') = 0 \) on \( \partial D \);
	\item \( G(r;r') = G_{\mathrm{FS}}(r;r') + H(r;r') \) where \( H \) satisfies Laplace's equation, the homogeneous version of Poisson's equation, for all \( r \in D \).
\end{enumerate}
Green's second identity with \( \laplacian{\phi} = -\rho, \laplacian{H} = 0 \) gives
\[
	\int_{\partial D} \qty(\phi \pdv{H}{\hat n} - H \pdv{\phi}{\hat n}) \dd{S} = \int_D H \rho \dd[3]{r}
\]
Now, we set \( G_{\mathrm{FS}} = G - H \) into Green's third identity to find
\[
	\phi(r') = \int_D (G - H)(-\rho) \dd[3]{r} + \int_{\partial D} \qty(\phi \pdv{(G - H)}{\hat n} - (G-H) \pdv{\phi}{n}) \dd{S}
\]
All of the \( H \) terms can be cancelled by substituting the form of the second identity the derived above.
Now, given \( G = 0, \phi = h \) on \( \partial D \), we have
\[
	\phi(r') = \int_D G(r;r')(-\rho(r)) \dd[3]{r} + \int_{\partial D} h(r) \pdv{G(r;r')}{\hat n} \dd{S}
\]
This is the general solution.
The first integral is the Green's function solution, and the second integral yields the inhomogeneous boundary conditions.

\subsection{Method of images for Laplace's equation}
For symmetric domains \( D \), we can construct Green's functions with \( G = 0 \) on \( \partial D \) by cancelling the boundary potential out by using an opposite `mirror image' Green's function placed outside the domain.
Consider Laplace's equation \( \laplacian{\phi} = 0 \) on half of \( \mathbb R^3 \), in particular, the subset of \( \mathbb R^3 \) such that \( z > 0 \).
Let \( \phi(x,y,0) = h(x,y) \) and \( \phi \to 0 \) as \( r \to \infty \).
The free space Green's function satisfies \( G_{\mathrm{FS}} \to 0 \) as \( r \to \infty \), but does not satisfy the boundary condition that \( G_{\mathrm{FS}} = 0 \) at \( z = 0 \).
For \( G_{\mathrm{FS}} \) at \( r' = (x',y',z') \), we will subtract a copy of \( G_{\mathrm{FS}} \) located at \( r'' = (x',y',-z') \).
This gives
\begin{align*}
	G(r,r') & = \frac{-1}{4\pi \abs{r - r'}} - \frac{-1}{4\pi \abs{r = r''}}                                                 \\
	        & = \frac{-1}{4\pi \sqrt{(x-x')^2 + (y-y')^2 + (z-z')^2}} + \frac{1}{4\pi \sqrt{(x-x')^2 + (y-y')^2 + (z+z')^2}}
\end{align*}
Hence \( G((x,y,0), r') = 0 \), so this function satisfies the Dirichlet boundary conditions on all of the boundary \( \partial D \).
We have
\[
	\eval{\pdv{G}{\hat n}}_{z = 0} = \eval{\pdv{G}{z}}_{z = 0} = \frac{-1}{4\pi} \qty(\frac{z - z'}{\abs{r - r'}^3} - \frac{z + z'}{\abs{r-r'}^3}) = \frac{z'}{2\pi} \qty((x-x')^2 + (y-y')^2 + (z')^2)^{-3/2}
\]
The solution is then given by
\[
	\Phi(x',y',z') = \frac{z'}{2\pi} \int_{-\infty}^\infty \int_{-\infty}^\infty \qty[ (x - x')^2 + (y-y')^2 + (z')^2 ]^{-3/2} h(x,y) \dd{x}\dd{y}
\]

\subsection{Method of images for wave equation}
Consider the one-dimensional wave equation
\[
	\ddot \phi - c^2 \phi'' = f(x,t)
\]
with Dirichlet boundary conditions \( \phi(0,t) = 0 \).
We create matching Green's functions with an opposite sign centred at \( -\xi \).
\[
	G(x,t;\xi,\tau) = \frac{1}[2c] H(c(t-\tau) - \abs{x-\xi}) - \frac{1}{2c} H(c,(t-\tau) - \abs{x+\xi})
\]
We can replace the addition of the two terms with a subtraction to instead use Neumann boundary conditions.
Suppose we wish to solve the homogeneous problem with \( f = 0 \) for initial conditions of a Gaussian pulse.
Here, for \( x > 0 \) we have
\[
	\phi(x,t) = \exp[-(x-\xi + ct)^2] - \exp[-(-x - \xi + ct)^2]
\]
The solution travels to the left, cancelling with the image at \( t = \frac{\xi}{c} \), which emerges and travels right as the reflected wave.
