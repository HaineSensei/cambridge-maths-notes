\subsection{Fourier series expansion of delta function}
Consider a complex Fourier series expansion,
\[
	\delta(x) = \sum_{n=-\infty}^\infty c_n e^{in\pi x/L};\quad c_n = \frac{1}{2L}\int_{-L}^L \delta(x) e^{-i n \pi x / L} \dd{x} = \frac{1}{2L}
\]
Hence,
\[
	\delta(x) = \frac{1}{2L} \sum_{n=-\infty}^\infty e^{in\pi x/L}
\]
Let \( f(x) \) be a function, so \( f(x) = \sum_{n=-\infty}^\infty d_n e^{in \pi x / L} \).
Then, their inner product is given by
\[
	\int_{-L}^L f^\star(x) \delta(x) \dd{x} = \frac{1}{2L} \sum_{n = -\infty}^\infty d_n \int_{-L}^L e^{in \pi x/L} e^{in \pi x/L} \dd{x} = \sum_{n = -\infty}^\infty d_n = f(0)
\]
The Fourier expansion of the \( \delta \) function can be extended periodically to the whole real line.
This infinite set of \( \delta \) functions is known as the Dirac comb, given by
\[
	\sum_{m = -\infty}^\infty \delta(x-2mL) = \sum_{n = -\infty}^\infty e^{in \pi x/L}
\]

\subsection{Arbitrary eigenfunction expansion of delta function}
In general, suppose
\[
	\delta(x-\xi) = \sum_{n=1}^\infty a_n y_n(x)
\]
with coefficients
\[
	a_n = \frac{\int_a^b w(x) y_n(x) \delta(x-\xi) \dd{x}}{\int_a^b w(x) y_n(x)^2 \dd{x}} = \frac{w(\xi) y_n(\xi)}{\int_a^b w(x) y_n(x)^2 \dd{x}} = w_n(\xi) Y_n(\xi)
\]
Then,
\[
	\delta(x-\xi) = w(\xi) \sum_{n=1}^\infty Y_n(\xi) Y_n(x) = w(x) \sum_{n=1}^\infty Y_n(\xi) Y_n(x)
\]
since \( \frac{w(x)}{w(\xi)} \delta(x - \xi) = \delta(x - \xi) \).
Hence,
\[
	\delta(x-\xi) = w(x) \sum_{n=1}^\infty \frac{y_n(\xi) y_n(x)}{N_n}
\]
where \( N_n = \int_a^b w y_n^2 \dd{x} \) is a normalisation factor.
\begin{example}
	Consider a Fourier series for \( y(0) = y(1) = 0 \), with \( y_n(x) = \sin n \pi x \).
	From the sine series coefficient expression,
	\[
		\delta(x-\xi) = 2\sum_{n=1}^\infty \sin n \pi \xi \sin n \pi x
	\]
	where \( 0 < \xi < 1 \).
\end{example}

\subsection{Motivation for Green's functions}
Consider a massive static string with tension \( T \) and linear mass density \( \mu \), suspended between fixed ends \( y(0) = y(1) = 0 \).
By resolving forces, we have the time independent form
\[
	T \dv[2]{y}{x} - \mu g = 0
\]
We will solve the inhomogeneous ODE \( - \dv[2]{y}{x} = f(x) \) with \( f(x) = -\frac{\mu g}{T} \).
This has been placed in Sturm-Liouville form.
We can integrate directly and find
\[
	-y = -\frac{\mu g}{2T} x^2 + k_1 x + k_2
\]
Imposing boundary conditions,
\[
	y(x) = \qty(-\frac{\mu g}{T}) \cdot \frac{1}{2}x(1-x)
\]
Consider alternatively a solution obtained by solving the equation for a single point mass \( \delta m = \mu \delta x \) suspended at \( x = \xi \) on an very light string.
We can then superimpose the solutions for each point mass to find the overall solution.
For a single point mass, the solution is given by two straight lines from \( (0,0) \) and \( (1,0) \) to the point mass \( (\xi_i, y_i(\xi_i)) \).
The angles of these straight lines from the horizontal are given by \( \theta_1, \theta_2 \).
Resolving in the \( y \) direction,
\begin{align*}
	0                                           & = T (\sin \theta_1 + \sin \theta_2) - \delta m g                \\
	                                            & = T\qty(\frac{-y_i}{\xi_i} + \frac{-y_i}{1-\xi_i}) - \delta m g \\
	\therefore -T\qty(y_i(1-\xi_i) + y_i \xi_i) & = \delta m g \xi_i(1-\xi_i)                                     \\
	\therefore y_i(\xi_i)                       & = \frac{-\delta m g}{T} \xi_i (1-\xi_i)
\end{align*}
So the solution is
\[
	y_i(x) = \frac{-\delta m g}{T} \begin{cases}
		x(1-\xi_i)    & x < \xi_i \\
		\xi_i (1 - x) & x > \xi_i
	\end{cases}
\]
which is the generalised sawtooth.
This can alternatively be written
\[
	f_i(\xi) G(x,\xi)
\]
where \( f_i \) is a source term, and \( G(x,\xi) \) is the Green's function, the solution for a unit point source.
Since the differential equation is linear, we can sum the solutions, giving
\[
	y(x) = \sum_{i=1}^N f_i(\xi) G(x, \xi_i)
\]
Taking a continuum limit,
\[
	f_i(\xi) = \frac{-\delta m g}{T} = \frac{-\mu \delta x g}{T} \equiv f(x) \dd{x} \implies f(x) = \frac{-\mu g}{T}
\]
which gives
\[
	y(x) = \int_0^1 f(\xi) G(x,\xi) \dd{\xi}
\]
Substituting the Green's function,
\begin{align*}
	y(x) & = (\frac{-\mu g}{T}) \qty[ \int_0^x \xi(1-x) \dd{\xi} + \int_x^1 x(1-\xi) \dd{\xi}]              \\
	     & = (\frac{-\mu g}{T}) \qty{ \qty[\frac{\xi^2}{2}(1-x)]_0^x + \qty[x(\xi - \frac{\xi^2}{2})]_x^1 } \\
	     & = (\frac{-\mu g}{T}) \qty(\frac{x^2}{2}(1-x) - 0 + \frac{x}{2} - x\qty(x-\frac{x^2}{2}))         \\
	     & = (\frac{-\mu g}{T}) \cdot \frac{1}{2}x(1-x)
\end{align*}
So we have found the correct solution in two ways; once by direct integration, and once by superimposing point solutions.
In general, direct integration is not trivial, and Green's functions are useful in this case.

\subsection{Definition of Green's function}
We wish to solve the inhomogeneous ODE
\[
	\mathcal L y \equiv \alpha(x) y'' + \beta(x) y' + \gamma(x) y = f(x)
\]
on \( a \leq x \leq b \), where \( \alpha \neq 0 \) and \( \alpha, \beta, \gamma \) are continuous and bounded, taking homogeneous boundary conditions \( y(a) = y(b) = 0 \).
The Green's function for \( \mathcal L \) in this case is defined to be the solution for a unit point source at \( x = \xi \).
That is, \( G(x,\xi) \) is the function that satisfies the boundary conditions and
\[
	\mathcal L G(x,\xi) = \delta(x-\xi)
\]
so \( G(a,\xi) = G(b,\xi) = 0 \).
Then, by linearity, the general solution is given by
\[
	y(x) = \int_a^b f(\xi) G(x,\xi) \dd{\xi}
\]
where \( y(x) \) satisfies the homogeneous boundary conditions.
We can verify this by checking
\[
	\mathcal L y = \int_a^b \mathcal L G(x,\xi) f(\xi) \dd{\xi} = \int_a^b \delta(x-\xi) f(\xi) \dd{\xi} = f(x)
\]
So the solution is given by the inverse operator
\[
	y = \mathcal L^{-1} f;\quad \mathcal L^{-1} = \int_a^b \dd{\xi} G(x,\xi)
\]
The Green's function spits into two parts;
\[
	G(x,\xi) = \begin{cases}
		G_1(x,\xi) & a \leq x < \xi \\
		G_2(x,\xi) & \xi < x < b
	\end{cases}
\]
For all \( x \neq \xi \), we have \( \mathcal L G_1 = \mathcal L G_2 = 0 \), so the parts are homogeneous solutions.
\( G \) satisfies the homogeneous boundary conditions, so \( G_1(a, \xi) = 0 \) and \( G_2(b, \xi) = 0 \).
\( G \) must be continuous at \( x = \xi \), hence \( G_1(\xi, \xi) = G_2(\xi, \xi) \).
There is a jump condition; the derivative of \( G \) is discontinuous at \( x = \xi \).
This satisfies
\[
	[G']_{\xi_-}^{\xi_+} = \eval{\dv{G_2}{x}}_{x = \xi_+} - \eval{\dv{G_1}{x}}_{x = \xi_-} = \frac{1}{\alpha(\xi)}
\]
