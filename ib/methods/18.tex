\subsection{Fourier transforms of generalised functions}
We can apply Fourier transforms to generalised functions by considering limiting distributions.
Consider the inversion
\begin{align*}
	f(x)
	 & = \mathcal F^{-1}(\mathcal F(f))(x)                                                                                                          \\
	 & = \frac{1}{2\pi} \int_{-\infty}^\infty \qty[\int_{-\infty}^\infty f(u) e^{-iku} \dd{u}] e^{ikx} \dd{k}                                       \\
	 & = \frac{1}{2\pi} \int_{-\infty}^\infty f(u) \underbrace{\qty[\frac{1}{2\pi} \int_{-\infty}^\infty e^{-ik(x-u)} \dd{k}]}_{\delta(x-u)} \dd{u}
\end{align*}
In order to reconstruct \( f(x) \) on the right hand side for any function \( f \), we must have that the bracketed term is \( \delta(x-u) \).
So we identify
\[
	\delta(x-u) = \frac{1}{2\pi} \int_{-\infty}^\infty e^{ik(x-u)} \dd{k}
\]
If \( f(x) = \delta(x) \),
\[
	\widetilde f(k) = \int_{-\infty}^\infty \delta(x) e^{ikx} \dd{x} = 1
\]
This can be thought of as the Fourier transform of an infinitely thin Gaussian, which becomes an infinitely wide Gaussian (a constant).
If \( f(x) = 1 \), then
\[
	\widetilde f(k) = \int_{-\infty}^\infty e^{-ikx}\dd{x} = 2\pi \delta(k)
\]
This can also be found by the duality formula.
If \( f(x) = \delta(x - a) \), we have
\[
	\widetilde f(k) = e^{-ika}
\]
This is a translation of the original Fourier transform for the \( \delta \) function above.

\subsection{Trigonometric functions}
Let \( f(x) = \cos \omega x = \frac{1}{2} \qty(e^{ix} + e^{-ix}) \).
Then,
\[
	\widetilde f(k) = \pi\qty(\delta(k+\omega) + \delta(k-\omega))
\]
For \( f(x) = \sin \omega x \), we have
\[
	\widetilde f(k) = i\pi\qty(\delta(k+\omega) - \delta(k-\omega))
\]
Using duality,
\begin{align*}
	f(x) & = \frac{1}{2}\qty(\delta(x+a) + \delta(x-a)) \implies \widetilde f(k) = \cos ka  \\
	f(x) & = \frac{1}{2i}\qty(\delta(x+a) - \delta(x-a)) \implies \widetilde f(k) = \sin ka
\end{align*}

\subsection{Heaviside functions}
Let \( H(x) \) be the Heaviside function, such that \( H(0) = \frac{1}{2} \).
Then, \( H(x) + H(-x) = 1 \) for all \( x \).
We can take the Fourier transform of this and find
\[
	\widetilde H(k) + \widetilde H(-k) = 2\pi \delta(k)
\]
Recall that \( H'(x) = \delta(x) \).
Thus,
\[
	ik \widetilde H(x) = \widetilde \delta(k) = 1
\]
Since \( k \delta(k) = 0 \), the two equations for \( \widetilde H \) can be consistent if we take
\[
	\widetilde H(k) = \pi\delta(k) + \frac{1}{ik}
\]

\subsection{Dirichlet discontinuous formula}
Recall the Dirichlet discontinuous formula:
\[
	\int_0^\infty \frac{\sin ax}{x} \dd{x} = \frac{\pi}{2} \sgn a = \begin{cases}
		\frac{\pi}{2}  & a > 0 \\
		0              & a = 0 \\
		-\frac{\pi}{2} & a < 0
	\end{cases}
\]
We can rewrite this as
\[
	\frac{1}{2} \sgn x = \frac{1}{2\pi} \int_{-\infty}^\infty \frac{e^{ikx}}{ik} \dd{k}
\]
since the cosine term divided by \( ik \) is odd.
Hence,
\[
	f(x) = \frac{1}{2} \sgn x \iff \widetilde f(k) = \frac{1}{ik}
\]
This is the preferred form for a Heaviside-type function when used in Fourier transforms.

\subsection{Solving ODEs for boundary value problems}
Consider \( y'' - y = f(x) \) with homogeneous boundary conditions \( y \to 0 \) as \( x \to \pm \infty \).
Taking the Fourier transform of this expression, we find
\[
	(-k^2 - 1) \widetilde y = \widetilde f
\]
Thus, the solution is
\[
	\widetilde y(k) = \frac{-\widetilde f(k)}{1+k^2} \equiv \widetilde f(k) \widetilde g(k)
\]
where \( \widetilde g(k) = \frac{-1}{1 + k^2} \).
Note that \( \widetilde g(k) \) is the Fourier transform of \( g(x) = -\frac{1}{2} e^{-\abs{x}} \).
Applying the convolution theorem,
\begin{align*}
	y(x)
	 & = \int_{-\infty}^\infty f(u) g(x-u) \dd{u}                                                    \\
	 & = -\frac{1}{2} \int_{-\infty}^\infty f(u) e^{-\abs{x-u}}\dd{u}                                \\
	 & = -\frac{1}{2} \qty[ \int_{-\infty}^x f(u) e^{u-x}\dd{u} + \int_x^\infty f(u) e^{x-u}\dd{u} ]
\end{align*}
This is in the form of a boundary value problem Green's function.
We can construct the same results by constructing the Green's function directly.

\subsection{Signal processing}
Suppose we have an input signal \( \mathcal I(t) \), which is acted on by some linear operator \( \mathcal L_{\text{in}} \) to yield an output \( \mathcal O(t) \).
The Fourier transform of the input \( \widetilde{\mathcal I}(\omega) \) is called the \textit{resolution}.
\[
	\widetilde{\mathcal I}(\omega) = \int_{-\infty}^\infty \mathcal I(t) e^{-i\omega t} \dd{t}
\]
In the frequency domain, the action of \( \mathcal L_{\text{in}} \) on \( \mathcal I(t) \) means that \( \widetilde{\mathcal I}(\omega) \) is multiplied by a transfer function \( \widetilde{\mathcal R}(\omega) \).
Thus,
\[
	\mathcal O(t) = \frac{1}{2\pi} \int_{-\infty}^\infty \widetilde{\mathcal R}(\omega) \widetilde{\mathcal I}(\omega) e^{i\omega t} \dd{\omega}
\]
The inverse Fourier transform of the transfer function, \( \mathcal R \), is called the \textit{response function}, which is given by
\[
	\mathcal R(t) = \frac{1}{2\pi} \int_{-\infty}^\infty \widetilde{\mathcal R}(\omega) e^{i \omega t}\dd{\omega}
\]
By the convolution theorem,
\[
	\mathcal O(t) = \int_{-\infty}^\infty \mathcal I(u) \mathcal R(t-u) \dd{u}
\]
Suppose there is no input (\( \mathcal I(t) = 0 \)) for \( t < 0 \).
By causality, there should be zero output for the response function (\( \mathcal R(t) = 0 \)) for \( t < 0 \).
Therefore, we require \( 0 < u < t \) and hence
\[
	\mathcal O(t) = \int_0^t \mathcal I(u) \mathcal R(t-u) \dd{u}
\]
which resembles an initial value problem Green's function.

\subsection{General transfer functions for ODEs}
Suppose an input-output relationship is given by a linear ODE.\@
\[
	\mathcal L \mathcal O(t) \equiv \qty(\sum_{i=0}^n a_i \dv[i]{x}) \mathcal O(t) \equiv \mathcal I(t)
\]
Here, \( \mathcal L_{\text{in}} = 1 \).
We want to solve this ODE using a Fourier transform.
\[
	(a_0 + a_1 i\omega - a_2 \omega^2 - a_3 i\omega^3 + \dots + a_n (i \omega)^n) \widetilde{\mathcal O}(\omega) = \widetilde{\mathcal I}(\omega)
\]
We can solve this algebraically in Fourier transform space.
The transfer function is
\[
	\widetilde{\mathcal R}(\omega) = \frac{1}{a_0 + \dots + a_n (i \omega)^n}
\]
We factorise the denominator to find partial fractions.
Suppose there are \( J \) distinct roots \( (i \omega - c_j)^{k_j} \), where \( k_j \) is the algebraic multiplicity of the \( j \)th root, so \( \sum_{j=1}^J k_j = n \).
So we can write
\[
	\widetilde{\mathcal R}(\omega) = \frac{1}{(i \omega - c_1)^{k_1} \dots (i \omega - c_J)^{k_J}}
\]
Expressing this as partial fractions,
\[
	\widetilde{\mathcal R}(\omega) = \sum_{j=1}^J \sum_{m=1}^{k_i} \frac{\Gamma_{jm}}{(i\omega - c_j)^m}
\]
The \( \Gamma_{jm} \) terms are constant.
To solve this, we must find the inverse Fourier transform of \( (i\omega - a)^{-m} \).
Recall that
\[
	\mathcal F^{-1}\qty(\frac{1}{i\omega - a}) = \begin{cases}
		e^{at} & t > 0 \\
		0      & t < 0
	\end{cases}
\]
for \( \Re a < 0 \).
So we will require \( \Re c_j < 0 \) for all \( j \) to eliminate exponentially growing solutions.
Note that for \( n = 2 \),
\[
	i \dv{\omega} \qty(\frac{1}{(i \omega - a)^2})
\]
and recall that
\[
	\mathcal F (t f(t)) = i \mathcal F'(\omega)
\]
Hence,
\[
	\mathcal F^{-1}\qty(\frac{1}{(i \omega - a)^2}) = \begin{cases}
		t e^{at} & t > 0 \\
		0        & t < 0
	\end{cases}
\]
Inductively, we arrive at
\[
	\mathcal F^{-1}\qty(\frac{1}{(i \omega - a)^m}) = \begin{cases}
		\frac{t^{m-1}}{(m-1)!} e^{at} & t > 0 \\
		0                             & t < 0
	\end{cases}
\]
We can therefore invert any transfer function to obtain the response function.
Thus the response function takes the form
\[
	\mathcal R(t) = \sum_{j=1}^J \sum_{m=1}^{k_i} \Gamma_{jm} \frac{t^{m-1}}{(m-1)!} e^{c_j t};\quad t > 0
\]
and zero for \( t < 0 \).
We can now solve such differential equations in Green's function form, or directly invert \( \widetilde{\mathcal R}(\omega) \widetilde{\mathcal I}(\omega) \) for a polynomial \( \widetilde{\mathcal I}(\omega) \).

\subsection{Damped oscillator}
We can use the Fourier transform method to solve the differential equation
\[
	\mathcal L y \equiv y'' + 2py' + (p^2 + q^2)y = f(t)
\]
where \( p > 0 \).
Consider homogeneous boundary conditions \( y(0) = y'(0) = 0 \).
The Fourier transform is
\[
	(i\omega)^2 \widetilde y + 2 i p \omega \widetilde y + (p^2 + q^2) \widetilde y = \widetilde f
\]
Hence,
\[
	\widetilde y = \frac{\widetilde f}{-\omega^2 + 2ip\omega + p^2 + q^2} \equiv \widetilde R \widetilde f
\]
We can invert this using the convolution theorem by inverting \( \widetilde R \).
\[
	y(t) = \int_0^t \mathcal R(t-\tau) f(\tau) \dd{\tau}
\]
where the response function is
\[
	\mathcal R(t - \tau) = \frac{1}{2\pi} \int_{-\infty}^\infty \frac{e^{i\omega(t-\tau)}}{p^2 + q^2 + 2ip\omega - \omega^2} \dd{\omega}
\]
We can show that \( \mathcal L \mathcal R(t-\tau) = \delta(t-\tau) \); in other words, \( \mathcal R \) is the Green's function.
