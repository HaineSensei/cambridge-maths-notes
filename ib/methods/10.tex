\subsection{Derivation with Fourier's law}
In a volume \( V \), the overall heat energy \( Q \) is given by
\[
	Q = \int_V c_V \rho \theta \dd{V}
\]
where \( c_V \) is the specific heat of the material, \( \rho \) is the mass density, and \( \theta \) is the temperature.
The rate of change due to heat flow is
\[
	\dv{Q}{t} = \int_V c_V \rho \pdv{\theta}{t} \dd{V}
\]
Fourier's law for heat flow is
\[
	q = -k \grad{\theta}
\]
where \( q \) is the heat flux.
We will integrate this over the surface \( S = \partial V \), giving
\[
	-\dv{Q}{t} = \int_S q \cdot \hat n \dd{S}
\]
The negative sign is due to the normals facing outwards.
This is exactly
\[
	-\dv{Q}{t} = \int_S (-k \grad{\theta}) \cdot \hat n \dd{S} = \int_V -k \laplacian{\theta} \dd{V}
\]
Equating these two forms for \( \dv{Q}{t} \), we find
\[
	\int_V (c_V \rho \pdv{\theta}{t} - k \laplacian{\theta}) \dd{V} = 0
\]
Since \( V \) was arbitrary, the integrand must be zero.
So we have
\[
	\pdv{\theta}{t} - \frac{k}{c_V \rho} \laplacian{\theta} = 0
\]
Let \( D = \frac{k}{c_V \rho} \) be the diffusion constant.
Then we have the diffusion equation
\[
	\pdv{\theta}{t} - D \laplacian{\theta} = 0
\]

\subsection{Derivation with statistical dynamics}
We can derive this equation in another way, using statistical dynamics.
Gas particles diffuse by scattering every fixed time step \( \Delta t \) with probability density function \( p(\xi) \) of moving by a displacement \( \xi \).
On average, we have
\[
	\inner{\xi} = \int p(\xi) \xi \dd{\xi} = 0
\]
since there is no bias the direction in which any given particle is travelling.
Suppose that the probability density function after \( N\Delta t \) time is described by \( P_{N \Delta t}(x) \).
Then, for the next time step,
\[
	P_{(N+1)\Delta t}(x) = \int_{-\infty}^\infty p(\xi) P_{N \Delta t}(x - \xi) \dd{\xi}
\]
Using the Taylor expansion,
\begin{align*}
	P_{(N+1)\Delta t}(x) & \approx \int_{-\infty}^\infty p(\xi) \qty[P_{N \Delta t}(x) + P_{N \Delta t}'(x)(-\xi) + P_{N \Delta t}''(x)\frac{\xi^2}{2} + \cdots] \dd{\xi} \\
	                     & \approx P_{N \Delta t}(x) - P_{N \Delta t}'(x) \inner{\xi} + P_{N \Delta t}''(x) \frac{\inner{\xi^2}}{2} + \cdots                              \\
	                     & \approx P_{N \Delta t}(x) + P_{N \Delta t}''(x) \frac{\inner{\xi^2}}{2} + \cdots
\end{align*}
since \( \int p(\xi) \dd{\xi} = 1 \).
Identifying \( P_{N \Delta t}(x) = P(x, N\Delta t) \), we can write
\[
	P(x, (N+1)\Delta t) - P(x, N \Delta t) = \pdv[2]{x} P(x, N\Delta t) \frac{\inner{\xi^2}}{2}
\]
Assuming that the variance \( \frac{\inner{\xi^2}}{2} \) is proportional to \( D \Delta t \), then for small \( \Delta t \), we find
\[
	\pdv{P}{t} = D \pdv[2]{P}{x}
\]
which is exactly the diffusion equation.

\subsection{Similarity solutions}
The characteristic relation between the variance and time suggests that we seek solutions with a dimensionless parameter.
If we can a change of variables of the form \( \theta(\eta) = \theta(x,t) \), then it will likely be easier to solve.
Consider
\[
	\eta \equiv \frac{x}{2\sqrt{Dt}}
\]
Then,
\[
	\pdv{\theta}{t} = \pdv{\eta}{t} \pdv{\theta}{\eta} = \frac{-1}{2} \frac{x}{\sqrt{D} t^{3/2}} \theta' = \frac{-1}{2} \frac{\eta}{t} \theta'
\]
and
\[
	D \pdv[2]{\theta}{x} = D \pdv{x} \qty(\pdv{\eta}{x} \pdv{\theta}{\eta}) = D \pdv{x} \qty(\frac{1}{2\sqrt{Dt}} \theta') = \frac{D}{4Dt} \theta'' = \frac{1}{4t} \theta''
\]
Substituting into the diffusion equation,
\[
	\theta'' = -2 \eta \theta'
\]
Let \( \psi = \theta' \).
Then
\[
	\frac{\psi'}{\psi} = -2\eta \implies \ln \psi = -\eta^2 + \text{constant}
\]
Then, choosing a constant of \( c\frac{2}{\sqrt{\pi}} \),
\[
	\psi = c\frac{2}{\sqrt{\pi}} e^{-\eta^2} \implies \theta(\eta) = c\frac{2}{\sqrt{\pi}} \int_0^\eta e^{-u^2} \dd{u} = c \erf(\eta) = c \erf(\frac{x}{2\sqrt{Dt}})
\]
where
\[
	\erf(z) = \frac{2}{\sqrt{\pi}} \int_0^z e^{-u^2} \dd{u}
\]
This describes discontinuous initial conditions that spread over time.

\subsection{Heat conduction in a finite bar}
Suppose we have a bar of length \( 2L \) with \( -L \leq x \leq L \) and initial temperature
\[
	\theta(x,0) = H(x) = \begin{cases}
		1 & \text{if } 0 \leq x \leq L \\
		0 & \text{if } -L \leq x < 0
	\end{cases}
\]
with boundary conditions \( \theta(L, t) = 1 \), \( \theta(-L, t) = 0 \).
Currently the boundary conditions are not homoegeneous, so Sturm-Liouville theory cannot be used directly.
If we can identify a steady-state solution (time-independent) that reflects the late-time behaviour, then we can turn it into a homoegeneous set of boundary conditions.
We will try a solution of the form
\[
	\theta_s(x) = Ax + B
\]
since this certainly satisfies the diffusion equation.
To satisfy the boundary conditions,
\[
	A = \frac{1}{2L};\quad B = \frac{1}{2}
\]
Hence we have a solution
\[
	\theta_s = \frac{x + L}{2L}
\]
We will subtract this solution from our original equation for \( \theta \), giving
\[
	\hat \theta(x,t) = \theta(x,t) - \theta_s(x)
\]
with homogeneous boundary conditions
\[
	\hat \theta(-L, t) = \hat \theta(L, t) = 0
\]
and initial conditions
\[
	\theta(x,0) = H(x) - \frac{x+L}{2L}
\]
We will now separate variables in the usual way.
We will consider the ansatz
\[
	\hat \theta(x,t) = X(x) T(t) \implies X'' = - \lambda X; \dot T = -D \lambda T
\]
The boundary conditions imply \( \lambda > 0 \) and give the Fourier modes \( X(x) = A \cos \sqrt{\lambda} x + B \sin \sqrt{\lambda} x \).
For \( \cos \sqrt{\lambda} L = 0 \), we require \( \sqrt{\lambda_m} = \frac{m \pi}{2L} \) for \( m \) odd.
Also, \( \sin \sqrt{\lambda} L = 0 \) gives \( \sqrt{\lambda_n} = \frac{n \pi}{L} \) for \( n \) even.
Since \( \hat \theta \) is odd due to our initial conditions, we can take
\[
	X_n = B_n \sin \frac{n \pi x}{L}; \quad \lambda_n = \frac{n^2 \pi^2}{L^2}
\]
Substituting into \( \dot T = -D \lambda T \), we have
\[
	T_n(t) = c_n \exp(-\frac{Dn^2 \pi^2}{L^2} t )
\]
In general, the solution is
\[
	\hat \theta(x,t) = \sum_{n=1}^\infty b_n \sin \frac{n \pi x}{L} \exp(-\frac{Dn^2 \pi^2}{L^2} t )
\]
