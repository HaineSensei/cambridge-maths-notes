\subsection{Separation of variables}
We wish to solve the wave equation subject to certain boundary and initial conditions.
Consider a possible solution of separable form:
\[
	y(x,t) = X(x) T(t)
\]
Substituting into the wave equation,
\[
	\frac{1}{c^2} \ddot y = y'' \implies \frac{1}{c^2} X \ddot T = X'' T
\]
Then
\[
	\frac{1}{c^2}\frac{\ddot T}{T} = \frac{X''}{X}
\]
However, \( \frac{\ddot T}{T} \) depends only on \( t \) and \( \frac{X''}{X} \) depends only on \( x \).
Thus, both sides must be equal to some \textit{separation constant} \( -\lambda \).
\[
	\frac{1}{c^2}\frac{\ddot T}{T} = \frac{X''}{X} = -\lambda
\]
Hence,
\[
	X'' + \lambda X = 0;\quad \ddot T + \lambda c^2 T = 0
\]

\subsection{Boundary conditions and normal modes}
We will begin by first solving the spatial part of the solution.
One of \( \lambda > 0, \lambda < 0, \lambda = 0 \) must be true.
The boundary conditions restrict the possible \( \lambda \).
\begin{enumerate}[(i)]
	\item First, suppose \( \lambda < 0 \).
	      Take \( \chi^2 = -\lambda \).
	      Then,
	      \[
		      X(x) = Ae^{\chi x} + Be^{-\chi x} = C \cosh (\chi x) + D \sinh (\chi x)
	      \]
	      The boundary conditions are \( x(0) = x(L) = 0 \), so only the trivial solution is possible: \( C = D = 0 \).
	\item Now, suppose \( \lambda = 0 \).
	      Then
	      \[
		      X(x) = Ax + B
	      \]
	      Again, the boundary conditions impose \( A = B = 0 \) giving only the trivial solution.
	\item Finally, the last possibility is \( \lambda > 0 \).
	      \[
		      X(x) = A \cos \qty(\sqrt{\lambda} x) + B \sin \qty(\sqrt{\lambda} x)
	      \]
	      The boundary conditions give
	      \[
		      A = 0;\quad B \sin \qty(\sqrt{\lambda} L) = 0 \implies \sqrt{\lambda} L = n \pi
	      \]
\end{enumerate}
\noindent The following are the eigenfunctions and eigenvalues.
\[
	X_n(x) = B_n \sin \frac{n \pi x}{L};\quad \lambda_n = \qty(\frac{n \pi}{L})^2
\]
These are also called the `normal modes' of the system.
The spatial shape in \( x \) does not change in time, but the amplitude may vary.
The fundamental mode is the lowest frequency of vibration, given by
\[
	n = 1 \implies \lambda_1 = \frac{\pi^2}{L^2}
\]
The second mode is the first overtone, and is given by
\[
	n = 2 \implies \lambda_2 = \frac{4\pi^2}{L^2}
\]

\subsection{Initial conditions and temporal solutions}
Substituting \( \lambda_n \) into the time ODE,
\[
	\ddot T + \frac{n^2 \pi^2 c^2}{L^2}T = 0
\]
Hence,
\[
	T_n(t) = C_n \cos \frac{n \pi c t}{L} + D_n \sin \frac{n \pi c t}{L}
\]
Therefore, a specific solution of the wave equation satisfying the boundary conditions is (absorbing the \( B_n \) into the \( C_n, D_n \)):
\[
	y_n(x,t) = T_n(t) X_n(x) = \qty(C_n \cos \frac{n \pi c t}{L} + D_n \sin \frac{n \pi c t}{L}) \sin \frac{n \pi x}{L}
\]
To find a particular solution for a given set of initial conditions, we must consider a linear superposition of all possible \( y_n \).
\[
	y(x,t) = \sum_{n=1}^\infty \qty(C_n \cos \frac{n \pi c t}{L} + D_n \sin \frac{n \pi c t}{L}) \sin \frac{n \pi x}{L}
\]
By construction, this \( y(x,t) \) satisfies the boundary conditions, so now we can impose the initial conditions.
\[
	y(x,0) = p(x) = \sum_{n=1}^\infty C_n \sin \frac{n \pi x}{L}
\]
We can find the \( C_n \) using standard Fourier series techniques, since this is exactly a half-range sine series.
Further,
\[
	\pdv{y(x,0)}{t} = q(x) = \sum_{n=1}^\infty \frac{n \pi c}{L} D_n \sin \frac{n \pi x}{L}
\]
Again we can solve for the \( D_n \) in a similar way.
In particular,
\[
	C_n = \frac{2}{L} \int_0^L p(x) \sin \frac{n \pi x}{L} \dd{x}
\]
\[
	D_n = \frac{2}{n \pi c} \int_0^L q(x) \sin \frac{n \pi x}{L} \dd{x}
\]
\begin{example}
	Consider the initial condition of a see-saw wave parametrised by \( \xi \), and let \( L = 1 \).
	This can be visualised as plucking the string at position \( \xi \).
	\[
		y(x,0) = p(x) = \begin{cases}
			x(1-\xi) & 0 \leq x < \xi \\
			\xi(1-x) & \xi \leq x < 1
		\end{cases}
	\]
	We also define
	\[
		\pdv{y(x,0)}{t} = q(x) = 0
	\]
	The Fourier series for \( p \) is given by
	\[
		C_n = \frac{2 \sin n \pi \xi}{(n \pi)^2};\quad D_n = 0
	\]
	Hence the solution to the wave equation is
	\[
		y(x,t) = \sum_{n=1}^\infty \frac{2}{(n \pi)^2} \sin n \pi \xi \sin n \pi x \cos n \pi c t
	\]
\end{example}

\subsection{Separation of variables methodology}
A general strategy for solving higher-dimensional partial differential equations is as follows.
\begin{enumerate}[(i)]
	\item Obtain a linear PDE system, using boundary and initial conditions.
	\item Separate variables to yield decoupled ODEs.
	\item Impose homogeneous boundary conditions to find eigenvalues and eigenfunctions.
	\item Use these eigenvalues (constants of separation) to find the eigenfunctions in the other variables.
	\item Sum over the products of separable solutions to find the general series solution.
	\item Determine coefficients for this series using the initial conditions.
\end{enumerate}
\begin{example}
	We will solve the wave equation instead in characteristic coordinates.
	Recall the sine and cosine summation identities:
	\begin{align*}
		y(x,t) & = \frac{1}{2} \sum_{n=1}^\infty \Bigg[ \qty(C_n \sin \frac{n \pi}{L}(x-ct) + D_n \cos \frac{n \pi}{L}(x-ct)) \\&
		+ \qty(C_n \sin \frac{n \pi}{L}(x+ct) - D_n \cos \frac{n \pi}{L}(x+ct)) \Bigg]                                        \\
		       & = f(x-ct) + g(x+ct)
	\end{align*}
	The standing wave solution can be interpreted as a superposition of a right-moving wave and a left-moving wave.
	A special case is \( q(x) = 0 \), implying \( f = g = \frac{1}{2} p \).
	Then,
	\[
		y(x,t) = \frac{1}{2}\qty[p(x-ct) + p(x+ct)]
	\]
\end{example}

\subsection{Energy of oscillations}
A vibrating string has kinetic energy due to its motion.
\[
	\text{Kinetic energy} = \frac{1}{2} \mu \int_0^L \qty(\pdv{y}{t})^2 \dd{x}
\]
It has potential energy given by
\[
	\text{Potential energy} = T \Delta x = T \int_c^T \qty(\sqrt{1 + \qty(\pdv{y}{x})^2}-1)\dd{x} \approx \frac{1}{2} T \int_0^L \qty(\pdv{y}{x})^2 \dd{x}
\]
assuming that the disturbances on the string are small, that is, \( \abs{\pdv{y}{x}} \ll 1 \).
The total energy on the string, given \( c^2 = T/\mu \), is given by
\[
	E = \frac{1}{2}\mu \int_0^L \qty[\qty(\pdv{y}{t})^2 + c^2 \qty(\pdv{y}{x})^2] \dd{x}
\]
Substituting the solution, using the orthogonality conditions,
\begin{align*}
	E & = \frac{1}{2}\mu \sum_{n=1}^\infty \int_0^L \Bigg[-\qty(\frac{n \pi c}{L} C_n \sin \frac{n \pi c t}{L} + \frac{n \pi c}{L} D_n \cos \frac{n \pi c t}{L})^2 \sin^2 \frac{n \pi x}{L} \\
	  & + c^2 \qty(C_n \cos \frac{n \pi c t}{L} + D_n \sin \frac{n \pi c t}{L})^2 \frac{n^2 \pi^2}{L^2} \cos^2 \frac{n \pi x}{L} \Bigg] \dd{x}                                              \\
	  & = \frac{1}{4} \mu \sum_{n=1}^\infty \frac{n^2 \pi^2 c^2}{L} \qty(C_n^2 + D_n^2)
\end{align*}
which is an analogous result to Parseval's theorem.
This is true since \[
	\int \cos^2 \frac{n \pi x}{L}\dd{x} = \frac{1}{2}
\] and \( \cos^2 + \sin^2 = 1 \).
We can think of this energy as the sum over all the normal modes of the energy in that specific mode.
Note that this quantity is constant over time.

\subsection{Wave reflection and transmission}
The travelling wave has left-moving and right-moving modes.
A \textit{simple harmonic} travelling wave is
\[
	y = \Re\qty[ A e^{i \omega(t-x/c)} ] = A \cos \qty[\omega(t-x/c) + \phi]
\]
where the phase \( \phi \) is equal to \( \arg A \), and the wavelength \( \lambda \) is \( 2 \pi c / \omega \).
In further discussion, we assume only the real part is used.
Consider a density discontinuity on the string at \( x = 0 \) with the following properties.
\[
	\mu = \begin{cases}
		\mu_- & \text{for } x < 0 \\
		\mu_+ & \text{for } x > 0
	\end{cases} \implies c = \begin{cases}
		c_- = \sqrt{\frac{T}{\mu_-}} & \text{for } x < 0 \\
		c_+ = \sqrt{\frac{T}{\mu_+}} & \text{for } x > 0 \\
	\end{cases}
\]
assuming a constant tension \( T \).
As a wave from the negative direction approaches the discontinuity, some of the wave will be reflected, given by \( B e^{i \omega(t + x/c_-)} \), and some of the wave will be transmitted, given by \( D e^{i \omega(t - x/c_+)} \).
The boundary conditions at \( x = 0 \) are
\begin{enumerate}[(i)]
	\item \( y \) is continuous for all \( t \) (the string does not break), so
	      \begin{equation}
		      A + B = D \tag{\(\ast\)}
	      \end{equation}
	\item The forces balance, \( T \eval{\pdv{y}{x}}_{x = 0^-} = T \eval{\pdv{y}{x}}_{x = 0^+} \) which means \( \pdv{y}{x} \) must be continuous for all \( t \).
	      This gives
	      \begin{equation}
		      \frac{-i\omega A}{c_-} + \frac{i \omega B}{c_-} = \frac{-i \omega D}{c_+} \tag{\(\dagger\)}
	      \end{equation}
\end{enumerate}
We can eliminate \( B \) from \( (\ast) \) by subtracting \( \frac{c_-}{i \omega}(\dagger) \).
\[
	2A = D + D \frac{c_-}{c_+} = \frac{D}{c_+}(c_+ + c_-)
\]
Hence, given \( A \), we have the solution for the transmitted amplitude and reflected amplitude to be
\[
	D = \frac{2 c_+}{c_- + c_+} A;\quad B = \frac{c_+ - c_-}{c_- + c_+}
\]
In general \( A, B, D \) are complex, hence different phase shifts are possible.

There are a number of limiting cases, for example
\begin{enumerate}[(i)]
	\item If \( c_- = c_+ \) we have \( D = A \) and \( B = 0 \) so we have full transmission and no reflection.
	\item (Dirichlet boundary conditions) If \( \frac{\mu_+}{\mu_-} \to \infty \), this models a fixed end at \( x = 0 \).
	      We have \( \frac{c_+}{c_-} \to 0 \) giving \( D = 0 \) and \( B = -A \).
	      Notice that the reflection has occurred with opposite phase, \( \phi = \pi \).
	\item (Neumann boundary conditions) Consider \( \frac{\mu_+}{\mu_-} \to 0 \), this models a free end.
	      Then \( \frac{c_+}{c_-} \to \infty \) giving \( D = 2A \), \( B = A \).
	      This gives total reflection but with the same phase.
\end{enumerate}

\subsection{Wave equation in plane polar coordinates}
Consider the two-dimensional wave equation for \( u(r,\theta,t) \) given by
\[
	\frac{1}{c^2} \pdv[2]{u}{t} = \laplacian u
\]
with boundary conditions at \( r = 1 \) on a unit disc given by
\[
	u(1,\theta,t) = 0
\]
and initial conditions for \( t = 0 \) given by
\[
	u(r,\theta,0) = \phi(r,\theta);\quad \pdv{u}{t}(r,\theta,0) = \psi(r,\theta)
\]
Suppose that this equation is separable.
First, let us consider temporal separation.
Suppose that
\[
	u(r,\theta,t) = T(t) V(r,\theta)
\]
Then we have
\[
	\ddot T + \lambda c^2 T = 0;\quad \laplacian V + \lambda V = 0
\]
In plane polar coordinates, we can write the spatial equation as
\[
	\pdv[2]{V}{r} + \frac{1}{r} \pdv{V}{r} + \frac{1}{r^2}\pdv[2]{V}{\theta} + \lambda V = 0
\]
We will perform another separation, supposing
\[
	V(r,\theta) = R(r) \Theta(\theta)
\]
to give
\[
	\Theta'' + \mu \Theta = 0;\quad r^2 R'' + r R' + \qty(\lambda r^2 - \mu) R = 0
\]
where \( \lambda, \mu \) are the separation constants.
The polar solution is constrained by periodicity \( \Theta(0) = \Theta(2 \pi) \), since we are working on a disc.
We also consider only \( \mu > 0 \).
The eigenvalue is then given by \( \mu = m^2 \), where \( m \in \mathbb N \).
\[
	\Theta_m(\theta) = A_m \cos m \theta + B_m \sin m \theta
\]
Or, in complex exponential form,
\[
	\Theta_m(\theta) = C_m e^{im\theta};\quad m \in \mathbb Z
\]

\subsection{Bessel's equation}
We can solve the radial equation (in the previous subsection) by converting it first into Sturm-Liouville form, which can be accomplished by dividing by \( r \).
\[
	\dv{r} \qty(r R') - \frac{m^2}{r} = -\lambda r R
\]
where \( p(r) = r, q(r) = \frac{m^2}{r}, w(r) = r \), with self-adjoint boundary conditions with \( R(1) = 0 \).
We will require \( R \) is bounded at \( R(0) \), and since \( p(0) = 0 \) there is a regular singular point at \( r = 0 \).
This particular equation for \( R \) is known as Bessel's equation.
We will first substitute \( z \equiv \sqrt{\lambda} r \), then we find the usual form of Bessel's equation,
\[
	z^2 \dv[2]{R}{z} + z \dv{R}{z} + (z^2 - m^2)R = 0
\]
We can use the method of Frobenius by substituting the following power series:
\[
	R = z^p \sum_{n=0}^\infty a_n z^n
\]
to find
\[
	\sum_{n=0}^\infty \qty[ a_n (n+p)(n+p-1) z^{n+p} + (n+p) z^{n+p} + z^{n+p+2} + m^2 z^{n+p} ] = 0
\]
Equating powers of \( z \), we can find the indicial equation
\[
	p^2 - m^2 = 0 \implies p = m, -m
\]
The regular solution, given by \( p = m \), has recursion relation
\[
	(n+m)^2 a_n + a_{n-2} - m^2 a_n = 0
\]
which gives
\[
	a_n = \frac{-1}{n(n+2m)} a_{n-2}
\]
Hence, we can find
\[
	a_{2n} = a_0 \frac{(-1)^n}{2^{2n} n!
		(n+m)(n+m-1) \dots (m+1)}
\]
If, by convention, we let
\[
	a_0 = \frac{1}{2^m m!}
\]
we can then write the \textit{Bessel function of the first kind} by
\[
	J_m(z) = \qty(\frac{z}{2})^m \sum_{n=0}^\infty \frac{(-1)^n}{n!
		(n+m)!} \qty(\frac{z}{2})^{2n}
\]

\subsection{Asymptotic behaviour of Bessel functions}
If \( z \) is small, the leading-order behaviour of \( J_m(z) \) is
\begin{align*}
	J_0(z) & \approx 1                                \\
	J_m(z) & \approx \frac{1}{m!} \qty(\frac{z}{2})^m
\end{align*}
Now, let us consider large \( z \).
In this case, the function becomes oscillatory;
\[
	J_m(z) \approx \sqrt{\frac{2}{\pi z}} \cos(z - \frac{m \pi}{2} - \frac{\pi}{4})
\]

\subsection{Zeroes of Bessel functions}
We can see from the asymptotic behaviour that there are infinitely many zeroes of the Bessel functions of the first kind as \( z \to \infty \).
We define \( j_{mn} \) to be the \( n \)th zero of \( J_m \), for \( z > 0 \).
Approximately,
\[
	\cos(z - \frac{m \pi}{2} - \frac{\pi}{4}) = 0 \implies z - \frac{m \pi}{2} - \frac{\pi}{4} = n \pi - \frac{\pi}{2}
\]
Hence
\[
	z \approx n \pi + \frac{m \pi}{2} - \frac{\pi}{4} \equiv \widetilde j_{mn}
\]

\subsection{Solving the vibrating drum}
Recall that the radial solutions become
\[
	R_m(z) = R_m(\sqrt{\lambda} x) = A J_m(\sqrt{\lambda} x) + B Y_m(\sqrt{\lambda} x)
\]
Imposing the boundary condition of boundedness at \( r = 0 \), we must have \( B = 0 \).
Further imposing \( r = 1 \) and \( R = 0 \) gives \( J_m(\sqrt{\lambda}) = 0 \).
These zeroes occur at \( j_{mn} \approx n \pi + \frac{m \pi}{2} - \frac{\pi}{4} \).
Hence, the eigenvalues must be \( j^2_{mn} \).
Therefore, the spatial solution is
\[
	V_{mn}(r, \theta) = \Theta_m(\theta) R_{mn}(\sqrt{\lambda_{mn}} r) = (A_{mn} \cos m \theta + B_{mn} \sin m \theta) J_m (j_{mn} r)
\]
The temporal solution is
\[
	\ddot T = -\lambda c T \implies T_{mn}(t) = \cos(j_{mn} ct), \sin(j_{mn} ct)
\]
Combining everything together, the full solution is
\begin{align*}
	u(r,\theta,t) & = \sum_{n=1}^\infty J_0(j_{0n} r) \qty( A{0n}\cos j_{0n}ct + C_{0n}\sin j_{0n}ct )                                    \\
	              & + \sum_{m=1}^\infty \sum_{n=1}^\infty J_m (j_{mn}r) \qty( A_{mn} \cos m \theta + B_{mn} \sin m\theta ) \cos j_{mn} ct \\
	              & + \sum_{m=1}^\infty \sum_{n=1}^\infty J_m (j_{mn}r) \qty( C_{mn} \cos m \theta + D_{mn} \sin m\theta ) \sin j_{mn} ct
\end{align*}
Now, we impose the boundary conditions
\[
	u(r,\theta,0) = \phi(r,\theta) = \sum_{m=0}^\infty \sum_{n=1}^\infty J_m (j_{mn}r) \qty( A_{mn} \cos m \theta + B_{mn} \sin m\theta )
\]
and
\[
	\pdv{u}{t}\qty(r,\theta,0) = \psi(r,\theta) = \sum_{m=0}^\infty \sum_{n=1}^\infty j_{mn} c J_m (j_{mn}r) \qty( C_{mn} \cos m \theta + D_{mn} \sin m\theta )
\]
We need to find the coefficients by multiplying by \( J_m, \cos, \sin \) and using the orthogonality relations, which are
\[
	\int_0^1 J_m(j_{mn} r) J_m(j_{mk} r) r \dd{r} = \frac{1}{2}\qty[J_m'(j_{mn})]^2 \delta_{nk} = \frac{1}{2}\qty[J_{m+1}(j_{mn})]^2 \delta_{nk}
\]
by using a recursion relation of the Bessel functions.
We can then integrate to obtain the coefficients \( A_{mn} \).
\[
	\int_0^{2\pi} \dd{\theta} \cos p\theta \int_0^1 r \dd{r} J_p(j_{pq} r) \phi(r,\theta) = \frac{\pi}{2}\qty[J_{p+1}(j_{pq})]^2 A_{pq}
\]
where the \( \frac{\pi}{2} \) coefficient is \( 2\pi \) for \( p = 0 \).
We can find analogous results for the \( B_{mn}, C_{mn}, D_{mn} \).
\begin{example}
	Consider an initial radial profile \( u(r,\theta,0) = \phi(r) = 1 - r^2 \).
	Then, \( m = 0, B_{mn} = 0 \) for all \( m \) and \( A_{mn} = 0 \) for all \( m \neq 0 \).
	Then
	\[
		\pdv{u}{t}\qty(r,0,0) = 0
	\]
	hence \( C_{mn}, D_{mn} = 0 \).
	We just now need to find
	\[
		A_{0n} = \frac{2}{J_0(j_{0n})^2} \int_0^1 J_0(j_{0n}r)(1-r)^2 r\dd{r} = \frac{2}{J_0(j_{0n})^2} \frac{J_2(j_{0n})}{j_{0n}^2} \approx \frac{J_2(j_{0n})}{n} \text{ as } n \to \infty
	\]
	Then the approximate solution is
	\[
		u(r,\theta,t) = \sum_{n=1}^\infty A_{0n} J_0(j_{0n}r)\cos j_{0n} ct
	\]
	The fundamental frequency is \( \omega_d = j_{01} c \frac{2}{d} \approx 4.8\frac{c}{d} \) where \( d \) is the diameter of the drum.
	Comparing this to a string with length \( d \), this has a fundamental frequency of \( \omega_s = \frac{\pi c}{d} \approx 0.77 \omega_d \).
\end{example}

\subsection{Diffusion equation derivation with Fourier's law}
In a volume \( V \), the overall heat energy \( Q \) is given by
\[
	Q = \int_V c_V \rho \theta \dd{V}
\]
where \( c_V \) is the specific heat of the material, \( \rho \) is the mass density, and \( \theta \) is the temperature.
The rate of change due to heat flow is
\[
	\dv{Q}{t} = \int_V c_V \rho \pdv{\theta}{t} \dd{V}
\]
Fourier's law for heat flow is
\[
	q = -k \grad{\theta}
\]
where \( q \) is the heat flux.
We will integrate this over the surface \( S = \partial V \), giving
\[
	-\dv{Q}{t} = \int_S q \cdot \hat n \dd{S}
\]
The negative sign is due to the normals facing outwards.
This is exactly
\[
	-\dv{Q}{t} = \int_S (-k \grad{\theta}) \cdot \hat n \dd{S} = \int_V -k \laplacian{\theta} \dd{V}
\]
Equating these two forms for \( \dv{Q}{t} \), we find
\[
	\int_V (c_V \rho \pdv{\theta}{t} - k \laplacian{\theta}) \dd{V} = 0
\]
Since \( V \) was arbitrary, the integrand must be zero.
So we have
\[
	\pdv{\theta}{t} - \frac{k}{c_V \rho} \laplacian{\theta} = 0
\]
Let \( D = \frac{k}{c_V \rho} \) be the diffusion constant.
Then we have the diffusion equation
\[
	\pdv{\theta}{t} - D \laplacian{\theta} = 0
\]

\subsection{Diffusion equation derivation with statistical dynamics}
We can derive this equation in another way, using statistical dynamics.
Gas particles diffuse by scattering every fixed time step \( \Delta t \) with probability density function \( p(\xi) \) of moving by a displacement \( \xi \).
On average, we have
\[
	\inner{\xi} = \int p(\xi) \xi \dd{\xi} = 0
\]
since there is no bias the direction in which any given particle is travelling.
Suppose that the probability density function after \( N\Delta t \) time is described by \( P_{N \Delta t}(x) \).
Then, for the next time step,
\[
	P_{(N+1)\Delta t}(x) = \int_{-\infty}^\infty p(\xi) P_{N \Delta t}(x - \xi) \dd{\xi}
\]
Using the Taylor expansion,
\begin{align*}
	P_{(N+1)\Delta t}(x) & \approx \int_{-\infty}^\infty p(\xi) \qty[P_{N \Delta t}(x) + P_{N \Delta t}'(x)(-\xi) + P_{N \Delta t}''(x)\frac{\xi^2}{2} + \cdots] \dd{\xi} \\
	                     & \approx P_{N \Delta t}(x) - P_{N \Delta t}'(x) \inner{\xi} + P_{N \Delta t}''(x) \frac{\inner{\xi^2}}{2} + \cdots                              \\
	                     & \approx P_{N \Delta t}(x) + P_{N \Delta t}''(x) \frac{\inner{\xi^2}}{2} + \cdots
\end{align*}
since \( \int p(\xi) \dd{\xi} = 1 \).
Identifying \( P_{N \Delta t}(x) = P(x, N\Delta t) \), we can write
\[
	P(x, (N+1)\Delta t) - P(x, N \Delta t) = \pdv[2]{x} P(x, N\Delta t) \frac{\inner{\xi^2}}{2}
\]
Assuming that the variance \( \frac{\inner{\xi^2}}{2} \) is proportional to \( D \Delta t \), then for small \( \Delta t \), we find
\[
	\pdv{P}{t} = D \pdv[2]{P}{x}
\]
which is exactly the diffusion equation.

\subsection{Similarity solutions}
The characteristic relation between the variance and time suggests that we seek solutions with a dimensionless parameter.
If we can a change of variables of the form \( \theta(\eta) = \theta(x,t) \), then it will likely be easier to solve.
Consider
\[
	\eta \equiv \frac{x}{2\sqrt{Dt}}
\]
Then,
\[
	\pdv{\theta}{t} = \pdv{\eta}{t} \pdv{\theta}{\eta} = \frac{-1}{2} \frac{x}{\sqrt{D} t^{3/2}} \theta' = \frac{-1}{2} \frac{\eta}{t} \theta'
\]
and
\[
	D \pdv[2]{\theta}{x} = D \pdv{x} \qty(\pdv{\eta}{x} \pdv{\theta}{\eta}) = D \pdv{x} \qty(\frac{1}{2\sqrt{Dt}} \theta') = \frac{D}{4Dt} \theta'' = \frac{1}{4t} \theta''
\]
Substituting into the diffusion equation,
\[
	\theta'' = -2 \eta \theta'
\]
Let \( \psi = \theta' \).
Then
\[
	\frac{\psi'}{\psi} = -2\eta \implies \ln \psi = -\eta^2 + \text{constant}
\]
Then, choosing a constant of \( c\frac{2}{\sqrt{\pi}} \),
\[
	\psi = c\frac{2}{\sqrt{\pi}} e^{-\eta^2} \implies \theta(\eta) = c\frac{2}{\sqrt{\pi}} \int_0^\eta e^{-u^2} \dd{u} = c \erf(\eta) = c \erf(\frac{x}{2\sqrt{Dt}})
\]
where
\[
	\erf(z) = \frac{2}{\sqrt{\pi}} \int_0^z e^{-u^2} \dd{u}
\]
This describes discontinuous initial conditions that spread over time.

\subsection{Heat conduction in a finite bar}
Suppose we have a bar of length \( 2L \) with \( -L \leq x \leq L \) and initial temperature
\[
	\theta(x,0) = H(x) = \begin{cases}
		1 & \text{if } 0 \leq x \leq L \\
		0 & \text{if } -L \leq x < 0
	\end{cases}
\]
with boundary conditions \( \theta(L, t) = 1 \), \( \theta(-L, t) = 0 \).
Currently the boundary conditions are not homoegeneous, so Sturm-Liouville theory cannot be used directly.
If we can identify a steady-state solution (time-independent) that reflects the late-time behaviour, then we can turn it into a homoegeneous set of boundary conditions.
We will try a solution of the form
\[
	\theta_s(x) = Ax + B
\]
since this certainly satisfies the diffusion equation.
To satisfy the boundary conditions,
\[
	A = \frac{1}{2L};\quad B = \frac{1}{2}
\]
Hence we have a solution
\[
	\theta_s = \frac{x + L}{2L}
\]
We will subtract this solution from our original equation for \( \theta \), giving
\[
	\hat \theta(x,t) = \theta(x,t) - \theta_s(x)
\]
with homogeneous boundary conditions
\[
	\hat \theta(-L, t) = \hat \theta(L, t) = 0
\]
and initial conditions
\[
	\theta(x,0) = H(x) - \frac{x+L}{2L}
\]
We will now separate variables in the usual way.
We will consider the ansatz
\[
	\hat \theta(x,t) = X(x) T(t) \implies X'' = - \lambda X; \dot T = -D \lambda T
\]
The boundary conditions imply \( \lambda > 0 \) and give the Fourier modes \( X(x) = A \cos \sqrt{\lambda} x + B \sin \sqrt{\lambda} x \).
For \( \cos \sqrt{\lambda} L = 0 \), we require \( \sqrt{\lambda_m} = \frac{m \pi}{2L} \) for \( m \) odd.
Also, \( \sin \sqrt{\lambda} L = 0 \) gives \( \sqrt{\lambda_n} = \frac{n \pi}{L} \) for \( n \) even.
Since \( \hat \theta \) is odd due to our initial conditions, we can take
\[
	X_n = B_n \sin \frac{n \pi x}{L}; \quad \lambda_n = \frac{n^2 \pi^2}{L^2}
\]
Substituting into \( \dot T = -D \lambda T \), we have
\[
	T_n(t) = c_n \exp(-\frac{Dn^2 \pi^2}{L^2} t )
\]
In general, the solution is
\[
	\hat \theta(x,t) = \sum_{n=1}^\infty b_n \sin \frac{n \pi x}{L} \exp(-\frac{Dn^2 \pi^2}{L^2} t )
\]

\subsection{Particular solution to diffusion equation}
Recall that
\[
	\hat \theta(x,t) = \sum_{n=1}^\infty b_n \sin \frac{n \pi x}{L} \exp(-\frac{Dn^2 \pi^2}{L^2} t )
\]
At \( t = 0 \), we have a pure Fourier sine series.
We can then impose the initial conditions, to give
\[
	b_n = \frac{1}{L} \int_{-L}^L \hat \phi(x,0) \sin \frac{n \pi x}{L} \dd{x}
\]
where
\[
	\hat\phi(x,0) = H(x) - \frac{x+L}{2L}
\]
Hence, we can use the half-range sine series and find
\[
	b_n = \underbrace{ \frac{2}{L} \int_0^L \qty(H(x) = \frac{1}{2}) \sin \frac{n \pi x}{L} \dd{x} }_{\text{square wave}/2} - \underbrace{ \frac{2}{L}\frac{x}{2L} \sin \frac{n \pi x}{L} \dd{x} }_{\text{sawtooth}/2L}
\]
which gives
\[
	b_n = \frac{2}{(2m-1)\pi} - \frac{(-1)^{n+1}}{n\pi}
\]
where \( n = 2m - 1 \), and the first term vanishes for \( n \) even.
For \( n \) odd or even, we find the same result
\[
	b_n = \frac{1}{n\pi}
\]
Hence
\[
	\hat\theta(x,t) = \sum_{n=1}^\infty \frac{1}{n \pi} \sin \frac{n \pi x}{L} e^{-D \frac{n^2 \pi^2}{L^2} t}
\]
For the inhomogeneous boundary conditions,
\[
	\theta(x,t) = \frac{x+L}{2L} + \sum_{n=1}^\infty \frac{1}{n \pi} \sin \frac{n \pi x}{L} e^{-D \frac{n^2 \pi^2}{L^2} t}
\]
The similarity solution \( \frac{1}{2}\qty(1 + \erf(\frac{x}{2\sqrt{Dt}})) \) is a good fit for early \( t \), but it does not necessarily satisfy the boundary conditions, so for large \( t \) it is a bad approximation.

\subsection{Laplace's equation}
Laplace's equation is
\[
	\laplacian \phi = 0
\]
This equation describes (among others) steady-state heat flow, potential theory \( F = -\grad{\phi} \), and incompressible fluid flow \( v = \grad{\phi} \).
The equation is solved typically on a domain \( D \), where boundary conditions are specified often on the boundary surface.
The Dirichlet boundary conditions fix \( \phi \) on the boundary surface \( \partial D \).
The Neumann boundary conditions fix \( \hat n \cdot \grad{\phi} \) on \( \partial D \).

\subsection{Laplace's equation in three-dimensional Cartesian coordinates}
In \( \mathbb R^3 \) with Cartesian coordinates, Laplace's equation becomes
\[
	\pdv[2]{\phi}{x} + \pdv[2]{\phi}{y} + \pdv[2]{\phi}{z} = 0
\]
We seek separable solutions in the usual way:
\[
	\phi(x,y,z) = X(x)Y(y)Z(z)
\]
Substituting,
\[
	X''YZ + XY''Z + XYZ'' = 0
\]
Dividing by \( XYZ \) as usual,
\begin{align*}
	\frac{X''}{X} = \frac{-Y''}{Y} - \frac{Z''}{Z} & = -\lambda_\ell                         \\
	\frac{Y''}{Y} = \frac{-Z''}{Z} - \frac{X''}{X} & = -\lambda_m                            \\
	\frac{Z''}{Z} = \frac{-X''}{X} - \frac{Y''}{Y} & = -\lambda_n = \lambda_\ell + \lambda_m
\end{align*}
From the eigenmodes, our general solution will be of the form
\[
	\phi(x,y,z) = \sum_{\ell,m,n} a_{\ell mn} X_\ell(x) Y_m(y) Z_n(z)
\]
Consider steady (\(\pdv{\phi}{t} = 0 \)) heat flow in a semi-infinite rectangular bar, with boundary conditions \( \phi = 0 \) at \( x = 0 \), \( x = a \), \( y = 0 \) and \( y = b \); and \( \phi = 1 \) at \( z = 0 \) and \( \phi \to 0 \) as \( z \to \infty \).
We will solve for each eigenmode successively.
First, consider \( X'' = -\lambda_\ell X \) with \( X(0) = X(a) = 0 \).
This gives
\[
	\lambda_\ell = \frac{l^2 \pi^2}{a^2};\quad X_\ell = \sin \frac{\ell \pi x}{a}
\]
where \( \ell > 0, \ell \in \mathbb N \).
By symmetry,
\[
	\lambda_m = \frac{m^2 \pi^2}{b^2};\quad Y_m = \sin \frac{m \pi y}{b}
\]
For the \( z \) mode,
\[
	Z'' = -\lambda_n Z = (\lambda_\ell + \lambda_m) Z = \pi^2\qty(\frac{\ell^2}{a^2} + \frac{m^2}{b^2}) Z
\]
Since \( \phi \to 0 \) as \( z \to \infty \), the growing exponentials must vanish.
Therefore,
\[
	Z_{\ell m} = \exp[-\qty(\frac{\ell^2}{a^2} + \frac{m^2}{b^2})^{1/2} \pi z]
\]
Thus the general solution is
\[
	\phi(x,y,z) = \sum_{\ell, m} a_{\ell m} \sin \frac{\ell \pi x}{a} \sin \frac{m \pi y}{b} \exp[-\qty(\frac{\ell^2}{a^2} + \frac{m^2}{b^2})^{1/2} \pi z]
\]
Now, we will fix \( a_{\ell m} \) using \( \phi(x,y,0) = 1 \) using the Fourier sine series.
\[
	a_{\ell m} = \frac{2}{b} \int_0^b \frac{2}{a} \int_0^a \underbrace{1 \sin \frac{\ell \pi x}{a}}_{\text{square wave}} \underbrace{\sin \frac{m \pi y}{b}}_{\text{square wave}} \dd{x} \dd{y}
\]
So only the odd terms remain, giving
\[
	a_{\ell m} = \frac{4a}{a(2k-1)\pi} \cdot \frac{4b}{b(2p-1) \pi}
\]
where \( \ell = 2k-1 \) is odd and \( m = 2p-1 \) is odd.
Simplifying,
\[
	a_{\ell m} = \frac{16}{\pi^2 \ell m} \quad \text{ for } \ell, m \text{ odd}
\]
So the heat flow solution is
\[
	\phi(x,y,z) = \sum_{\ell, m \text{ odd}} \frac{16}{\pi^2 \ell m} \sin \frac{\ell \pi x}{a} \sin \frac{\ell \pi y}{b} \exp[-\qty(\frac{\ell^2}{a^2} + \frac{m^2}{b^2})^{1/2} \pi z]
\]
As \( z \) increases, every contribution but the lowest mode will be very small.
So low \( \ell, m \) dominate the solution.

\subsection{Laplace's equation in plane polar coordinates}
In plane polar coordinates, Laplace's equation becomes
\[
	\frac{1}{r} \pdv{r} \qty(r \pdv{\phi}{r}) + \frac{1}{r^2} \pdv[2]{\phi}{\theta}
\]
Consider a separable form of the answer, given by
\[
	\phi(r,\theta) = R(r) \Theta(\theta)
\]
We then have
\[
	\Theta'' + \mu \Theta = 0;\quad r(rR')' - \mu R = 0
\]
The polar equation can be solved easily by considering periodic boundary conditions.
This gives \( \mu = m^2 \) and the eigenmodes
\[
	\Theta_m(\theta) = \cos m \theta, \sin m \theta
\]
The radial equation is \textit{not} Bessel's equation, since there is no second separation constant.
We simply have
\[
	r(rR')' - m^2 R = 0
\]
We will try a power law solution, \( r = \alpha r^\beta \).
We find
\[
	\beta^2 - m^2 = 0 \implies \beta = \pm m
\]
So the eigenfunctions are
\[
	R_m(r) = r^m, r^{-m}
\]
which is one regular solution at the origin and one singular solution.
In the case \( m = 0 \), we have
\[
	(rR') = 0 \implies rR' = \text{constant} \implies R = \log r
\]
So
\[
	R_0(r) = \text{constant}, \log r
\]
The general solution is therefore
\[
	\phi(r,\theta) = \frac{a_0}{2} + c_0 \log r + \sum_{m=1}^\infty \qty(a_m \cos m\theta + b_m \sin m\theta) r^m + \sum_{m=1}^\infty \qty(c_m \cos m\theta + d_m \sin m\theta) r^{-m}
\]
\begin{example}
	Consider a soap film on a unit disc.
	We wish to solve Laplace's equation with a vertically distorted circular wire of radius \( r = 1 \) with boundary conditions \( \phi(1, \theta) = f(\theta) \).
	The \( z \) displacement of the wire produces the \( f(\theta) \) term.
	We wish to find \( \phi(r,\theta) \) for \( r < 1 \), assuming regularity at \( r = 0 \).
	Then, \( c_m = d_m = 0 \) and the solution is of the form
	\[
		\phi(r,\theta) = \frac{a_0}{2} + \sum_{m=1}^\infty \qty(a_m \cos m\theta + b_m \sin m\theta) r^m
	\]
	At \( r = 1 \),
	\[
		\phi(1,\theta) = f(\theta) = \frac{a_0}{2} + \sum_{m=1}^\infty \qty(a_m \cos m\theta + b_m \sin m\theta)
	\]
	which is exactly the Fourier series.
	Thus,
	\[
		a_m = \frac{1}{\pi} \int_0^{2\pi} f(\theta) \cos m \theta \dd{\theta};\quad b_m = \frac{1}{\pi} \int_0^{2\pi} f(\theta) \sin m \theta \dd{\theta}
	\]
	We can see from the equation that high harmonics are confined to have effects only near \( r = 1 \).
\end{example}

\subsection{Laplace's equation in cylindrical polar coordinates}
In cylindrical coordinates,
\[
	\frac{1}{r} \pdv{r} \qty(r \pdv{\phi}{r}) + \frac{1}{4^2} \pdv[2]{\phi}{\theta} + \pdv[2]{\phi}{z} = 0
\]
With \( \phi = R(r) \Theta(\theta) Z(z) \), we find
\[
	\Theta'' = -\mu \Theta;\quad Z'' = \lambda Z;\quad r(rR')' + (\lambda r^2 - \mu) R = 0
\]
The polar equation can be easily solved by
\[
	\mu_m = m^2;\quad \Theta_m(\theta) = \cos m\theta, \sin m\theta
\]
The radial equation is Bessel's equation, giving solutions
\[
	R = J_m(kr), Y_m(kr)
\]
Setting boundary conditions in the usual way, defining \( R=0 \) at \( r = a \) means that
\[
	J_m(ka) = 0 \implies k = \frac{j_{mn}}{a}
\]
The radial solution is
\[
	R_{mn}(r) = J_m\qty(\frac{j_{mn}}{a}r)
\]
We have eliminated the \( Y_n \) term since we require \( r = 0 \) to give a finite \( \phi \).
Finally, the \( z \) equation gives
\[
	Z'' = k^2 Z \implies Z = e^{-kz}, e^{kz}
\]
We typically eliminate the \( e^{kz} \) mode due to boundary conditions, such as \( Z \to 0 \) as \( z \to \infty \).
The general solution is therefore
\[
	\phi(r,\theta,z) = \sum_{m=0}^\infty \sum_{n = 1}^\infty \qty(a_{mn} \cos m\theta + b_{mn} \sin m\theta) J_m\qty(\frac{j_{mn}}{a} r) e^{-frac{j_{mn}r}{a}}
\]

\subsection{Laplace's equation in spherical polar coordinates}
In spherical polar coordinates,
\[
	\frac{1}{r^2} \pdv{r} \qty(r^2 \pdv{\Phi}{r}) + \frac{1}{r^2 \sin\theta}\pdv{\theta} \qty(\sin\theta \pdv{\Phi}{\theta}) + \frac{1}{r^2 \sin^2 \theta} \pdv[2]{\Phi}{\phi} = 0
\]
We will consider the \textit{axisymmetric case}; supposing that there is no \( \phi \) dependence.
We seek a separable solution of the form
\[
	\Phi(r,\theta) = R(r) \Theta(\theta)
\]
which gives
\[
	(\sin\theta \Theta')' + \lambda \sin\theta \Theta = 0;\quad (r^2R')' - \lambda R = 0
\]
Consider the substitution \( x = \cos\theta, \dv{x}{\theta} = -\sin\theta \) in the polar equation.
This gives \( \dv{\Theta}{\theta} = -\sin\theta \dv{\Theta}{x} \) and hence
\[
	-\sin\theta \dv{x}\qty[-\sin^2\theta \dv{\Theta}{x}] + \lambda \sin\theta \Theta = 0 \implies \dv{x}\qty[(1-x^2)\dv{\Theta}{x}] + \lambda \Theta = 0
\]
This gives Legendre's equation, so it has solutions of eigenvalues \( \lambda_\ell = \ell (\ell + 1) \) and eigenfunctions
\[
	\Theta_\ell(\theta) = P_\ell(x) = P_\ell(\cos\theta)
\]
The radial equation then gives
\[
	(r^2 R')' - \ell (\ell + 1) R = 0
\]
We will seek power law solutions: \( R = \alpha r^\beta \).
This gives
\[
	\beta(\beta + 1) - \ell(\ell + 1) = 0 \implies \beta = \ell, \beta = -\ell - 1
\]
Thus the radial eigenmodes are
\[
	R_\ell = r^{\ell}, r^{-\ell - 1}
\]
Therefore the general axisymmetric solution for spherical polar coordinates is
\[
	\Phi(r,\theta) = \sum_{\ell = 0}^\infty (a_\ell r^{\ell} + b_\ell r^{-\ell - 1}) P_\ell(\cos\theta)
\]
The \( a_\ell, b_\ell \) are determined by the boundary conditions.
Orthogonality conditions for the \( P_\ell \) can be used to determine coefficients.
Consider a solution to Laplace's equation on the unit sphere with axisymmetric boundary conditions given by
\[
	\Phi(1,\theta) = f(\theta)
\]
Given that we wish to find the interior solution, \( b_n = 0 \) by regularity.
Then,
\[
	f(\theta) = \sum_{\ell=0}^\infty a_\ell P_\ell(\cos\theta)
\]
By defining \( f(\theta) = F(\cos\theta) \),
\[
	F(x) = \sum_{\ell=0}^\infty a_\ell P_\ell(x)
\]
We can then find the coefficients in the usual way, giving
\[
	a_\ell = \frac{2\ell + 1}{2} \int_{-1}^1 F(x) P_{\ell}(x) \dd{x}
\]

\subsection{Generating function for Legendre polynomials}
Consider a charge at \( r_0 = (x,y,z) = (0,0,1) \).
Then, the potential at a point \( P \) becomes
\begin{align*}
	\Phi(r) & = \frac{1}{\abs{r - r_0}} = \frac{1}{(x^2 + y^2 + (x-1)^2)^{1/2}}                                         \\
	        & = \frac{1}{(r^2 (\sin^2 \phi + \cos^2 \phi) \sin^2 \theta + r^2 \cos^2 \theta - 2r \cos\theta + 1)^{1/2}} \\
	        & = \frac{1}{(r^2 \sin^2 \theta + r^2 \cos^2 \theta - 2r \cos\theta + 1)^{1/2}}                             \\
	        & = \frac{1}{(r^2 - 2r \cos\theta + 1)^{1/2}}                                                               \\
	        & = \frac{1}{(r^2 - 2r \overline x + 1)^{1/2}}
\end{align*}
where \( \overline x \equiv \cos \theta \).
This function \( \Phi \) is a solution to Laplace's equation where \( r \neq r_0 \).
Note that we can represent any axisymmetric solution as a sum of Legendre polynomials.
Now,
\[
	\frac{1}{\sqrt{r^2 - 2rx + 1}} = \sum_{\ell = 0}^\infty a_\ell P_\ell(x) r^\ell
\]
With the normalisation condition for the Legendre polynomials \( P_\ell(1) = 1 \), we find
\[
	\frac{1}{1-r} = \sum_{\ell=0}^\infty a_\ell r^\ell
\]
Using the geometric series expansion, we arrive at \( a_\ell = 1 \).
This gives
\[
	\frac{1}{\sqrt{r^2 - 2rx + 1}} = \sum_{\ell = 0}^\infty P_\ell(x) r^\ell
\]
which is the generating function for the Legendre polynomials.
