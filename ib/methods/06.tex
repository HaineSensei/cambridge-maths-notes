\subsection{Properties of legendre polynomials}
Since Legendre polynomials come from a self-adjoint operator, they must have certain conditions, such as orthogonality.
For \( n \neq m \),
\[
	\int_{-1}^1 P_n P_m \dd{x} = 0
\]
They are also normalisable,
\[
	\int_{-1}^1 P_n^2 \dd{x} = \frac{2}{2n+1}
\]
We can prove this with Rodrigues' formula:
\[
	P_n(x) = \frac{1}{2^n n!} \qty( \dv{x} )^n (x^2 - 1)^n
\]
Alternatively we could use a generating function:
\begin{align*}
	\sum_{n=0}^\infty P_n(x) t^n = \frac{1}{\sqrt{1 - 2xt + t^2}} & = 1 + \frac{1}{2}\qty(2xt - t^2) + \frac{3}{8}\qty(2xt - t^2)^2 + \dots \\
	                                                              & = 1 + xt + \frac{1}{2}\qty(3x^2 - 1)t^2 + \dots
\end{align*}
There are some useful recursion relations.
\[
	\ell(\ell + 1) P_{\ell + 1} = (2 \ell + 1) x P_\ell(x) - \ell P_{\ell - 1}(x)
\]
Also,
\[
	(2\ell + 1)P_\ell(x) = \dv{x} \qty[ P_{\ell + 1}(x) - P_{\ell - 1}(x) ]
\]

\subsection{Legendre polynomials as eigenfunctions}
Any (well-behaved) function on \( [-1,1] \) can be expressed as
\[
	f(x) = \sum_{\ell = 0}^\infty a_\ell P_\ell(x)
\]
where
\[
	a_\ell = \frac{2\ell + 1}{2} \int_{-1}^1 f(x) P_\ell(x) \dd{x}
\]
with no boundary conditions (e.g.\ periodicity conditions) on \( f \).

\subsection{Solving inhomogeneous differential equations}
\textit{This can be thought of as the general case of Fourier series discussed previously.}

\noindent Consider the problem
\[
	\mathcal L y = f(x) \equiv w(x) F(x)
\]
on \( x \in [a,b] \) assuming homogeneous boundary conditions.
Given eigenfunctions \( y_n(x) \) satisfying \( \mathcal L y_n = \lambda_n w y_n \), we wish to expand this solution as
\[
	y(x) = \sum_n c_n y_n(x)
\]
and
\[
	F(x) = \sum_n a_n y_n(x)
\]
where \( a_n \) are known and \( c_n \) are unknown:
\[
	a_n = \frac{\int_a^b w F y_n \dd{x}}{\int_a^b w y_n^2 \dd{x}}
\]
Substituting,
\[
	\mathcal L y = \mathcal L \sum_n c_n y_n = \sum_n c_n \lambda_n y_n = w \sum_n a_n y_n
\]
By orthogonality,
\[
	c_n \lambda_n = a_n \implies c_n = \frac{a_n}{\lambda_n}
\]
In particular,
\[
	y(x) = \sum_{n=1}^\infty \frac{a_n}{\lambda_n}y_n(x)
\]
We can further generalise; we can permit a driving force, which often induces a linear response term \( \widetilde\lambda w y \).
\[
	\mathcal L y - \widetilde \lambda w y = f(x)
\]
where \( \widetilde \lambda \) is fixed.
The solution becomes
\[
	y(x) = \sum_{n=1}^\infty \frac{a_n}{\lambda_n - \widetilde \lambda} y_n(x)
\]

\subsection{Integral solutions}
Recall that
\[
	y(x) = \sum_{n=1}^\infty \frac{a_n}{\lambda_n} y_n(x) = \sum_n \frac{y_n(x)}{\lambda_n \lambda_n N_n} \int_a^b w(\xi) F(\xi) y_n(\xi) \dd{\xi}
\]
where
\[
	N_n = \int w y_n^2 \dd{x}
\]
This then gives
\[
	y(x) = \int_a^b \underbrace{\sum_{n=1}^\infty \frac{y_n(x) y_n(\xi)}{\lambda_n N_n}}_{G(x,\xi)} \underbrace{w(\xi) F(\xi)}_{f(\xi)} \dd{\xi} = \int_a^b G(x;\xi) f(\xi) \dd{\xi}
\]
where
\[
	G(x,\xi) = \sum_{n=1}^\infty \frac{y_n(x) y_n(\xi)}{\lambda_n N_n}
\]
is the eigenfunction expansion of the Green's function.
Note that the Green's function does not depend on \( f \), but only on \( \mathcal L \) and the boundary conditions.
In this sense, it acts like an inverse operator
\[
	\mathcal L^- \equiv \int \dd{\xi} G(x,\xi)
\]
analogously to how \( Ax = b \implies x = A^{-1} b \) for matrix equations.

\subsection{Waves on an elastic string}
Consider a small displacement \( y(x,t) \) on a stretched string with fixed ends at \( x = 0 \) and \( x = L \), that is, with boundary conditions \( y(0,t) = y(L,t) = 0 \).
We can determine the string's motion for specified initial conditions \( y(x,0) = p(x) \) and \( \pdv{y}{t}(x,0) = q(x) \).
We derive the equation of motion governing the motion of the string by balancing forces on a string segment \( (x,x+\delta x) \) and take the limit as \( \delta x \to 0 \).
Let \( T_1 \) be the tension force acting to the left at angle \( \theta_1 \) from the horizontal.
Analogously, let \( T_2 \) be the rightwards tension force at angle \( \theta_2 \).
We assume at any point on the string that \( \abs{\pdv{y}{x}} \ll 1 \), so the angles of the forces are small.
In the \( x \) dimension,
\[
	T_1 \cos \theta_1 = T_2 \cos \theta_2 \implies T_1 \approx T_2 = T
\]
So the tension \( T \) is constant up to an error of order \( O\qty(\abs{\pdv{y}{x}
}^2) \).
In the \( y \) dimension, since \( \theta \) are small,
\[
	F_T = T_2 \sin \theta_2 - T_1 \sin \theta_1 \approx T (\eval{\pdv{y}{x}}_{x + \delta x} - \eval{\pdv{y}{x}}_x) \approx T \pdv[2]{y}{x} \delta x
\]
By \( F = ma \),
\[
	F_T + F_g = (\mu \delta x) \pdv[2]{y}{t} = T \pdv[2]{y}{x} \delta x - g \mu \delta x
\]
where \( F_g \) is the gravitational force and \( \mu \) is the linear mass density.
We define the wave speed as
\[
	c = \sqrt{\frac{T}{\mu}}
\]
and find
\[
	\pdv[2]{y}{t} = \frac{T}{\mu} \pdv[2]{y}{x} - g = c^2 \pdv[2]{y}{x}
\]
We often assume gravity is negligible to produce the pure wave equation
\[
	\frac{1}{c^2} \pdv[2]{y}{t} = \pdv[2]{y}{x}
\]
