\subsection{Separation of Variables}
We wish to solve the wave equation subject to certain boundary and initial conditions.
Consider a possible solution of separable form:
\[
	y(x,t) = X(x) T(t)
\]
Substituting into the wave equation,
\[
	\frac{1}{c^2} \ddot y = y'' \implies \frac{1}{c^2} X \ddot T = X'' T
\]
Then
\[
	\frac{1}{c^2}\frac{\ddot T}{T} = \frac{X''}{X}
\]
However, \( \frac{\ddot T}{T} \) depends only on \( t \) and \( \frac{X''}{X} \) depends only on \( x \).
Thus, both sides must be equal to some \textit{separation constant} \( -\lambda \).
\[
	\frac{1}{c^2}\frac{\ddot T}{T} = \frac{X''}{X} = -\lambda
\]
Hence,
\[
	X'' + \lambda X = 0;\quad \ddot T + \lambda c^2 T = 0
\]

\subsection{Boundary Conditions and Normal Modes}
We will begin by first solving the spatial part of the solution.
One of \( \lambda > 0, \lambda < 0, \lambda = 0 \) must be true.
The boundary conditions restrict the possible \( \lambda \).
\begin{enumerate}[(i)]
	\item First, suppose \( \lambda < 0 \).
	      Take \( \chi^2 = -\lambda \).
	      Then,
	      \[
		      X(x) = Ae^{\chi x} + Be^{-\chi x} = C \cosh (\chi x) + D \sinh (\chi x)
	      \]
	      The boundary conditions are \( x(0) = x(L) = 0 \), so only the trivial solution is possible: \( C = D = 0 \).
	\item Now, suppose \( \lambda = 0 \).
	      Then
	      \[
		      X(x) = Ax + B
	      \]
	      Again, the boundary conditions impose \( A = B = 0 \) giving only the trivial solution.
	\item Finally, the last possibility is \( \lambda > 0 \).
	      \[
		      X(x) = A \cos \qty(\sqrt{\lambda} x) + B \sin \qty(\sqrt{\lambda} x)
	      \]
	      The boundary conditions give
	      \[
		      A = 0;\quad B \sin \qty(\sqrt{\lambda} L) = 0 \implies \sqrt{\lambda} L = n \pi
	      \]
\end{enumerate}
\noindent The following are the eigenfunctions and eigenvalues.
\[
	X_n(x) = B_n \sin \frac{n \pi x}{L};\quad \lambda_n = \qty(\frac{n \pi}{L})^2
\]
These are also called the `normal modes' of the system.
The spatial shape in \( x \) does not change in time, but the amplitude may vary.
The fundamental mode is the lowest frequency of vibration, given by
\[
	n = 1 \implies \lambda_1 = \frac{\pi^2}{L^2}
\]
The second mode is the first overtone, and is given by
\[
	n = 2 \implies \lambda_2 = \frac{4\pi^2}{L^2}
\]

\subsection{Initial Conditions and Temporal Solutions}
Substituting \( \lambda_n \) into the time ODE,
\[
	\ddot T + \frac{n^2 \pi^2 c^2}{L^2}T = 0
\]
Hence,
\[
	T_n(t) = C_n \cos \frac{n \pi c t}{L} + D_n \sin \frac{n \pi c t}{L}
\]
Therefore, a specific solution of the wave equation satisfying the boundary conditions is (absorbing the \( B_n \) into the \( C_n, D_n \)):
\[
	y_n(x,t) = T_n(t) X_n(x) = \qty(C_n \cos \frac{n \pi c t}{L} + D_n \sin \frac{n \pi c t}{L}) \sin \frac{n \pi x}{2}
\]
To find a particular solution for a given set of initial conditions, we must consider a linear superposition of all possible \( y_n \).
\[
	y(x,t) = \sum_{n=1}^\infty \qty(C_n \cos \frac{n \pi c t}{L} + D_n \sin \frac{n \pi c t}{L}) \sin \frac{n \pi x}{2}
\]
By construction, this \( y(x,t) \) satisfies the boundary conditions, so now we can impose the initial conditions.
\[
	y(x,0) = p(x) = \sum_{n=1}^\infty C_n \sin \frac{n \pi x}{L}
\]
We can find the \( C_n \) using standard Fourier series techniques, since this is exactly a half-range sine seroes.
Further,
\[
	\pdv{y(x,0)}{t} = q(x) = \sum_{n=1}^\infty \frac{n \pi c}{L} D_n \sin \frac{n \pi x}{L}
\]
Again we can solve for the \( D_n \) in a similar way.
In particular,
\[
	C_n = \frac{2}{L} \int_0^L p(x) \sin \frac{n \pi x}{L} \dd{x}
\]
\[
	D_n = \frac{2}{n \pi c} \int_0^L q(x) \sin \frac{n \pi x}{L} \dd{x}
\]
\begin{example}
	Consider the initial condition of a see-saw wave parametrised by \( \xi \), and let \( L = 1 \).
	This can be visualised as plucking the string at position \( \xi \).
	\[
		y(x,0) = p(x) = \begin{cases}
			x(1-\xi) & 0 \leq x < \xi \\
			\xi(1-x) & \xi \leq x < 1
		\end{cases}
	\]
	We also define
	\[
		\pdv{y(x,0)}{t} = q(x) = 0
	\]
	The Fourier series for \( p \) is given by
	\[
		C_n = \frac{2 \sin n \pi \xi}{(n \pi)^2};\quad D_n = 0
	\]
	Hence the solution to the wave equation is
	\[
		y(x,t) = \sum_{n=1}^\infty \frac{2}{(n \pi)^2} \sin n \pi \xi \sin n \pi x \cos n \pi c t
	\]
\end{example}

\subsection{Separation of Variables Methodology}
A general strategy for solving higher-dimensional partial differential equations is as follows.
\begin{enumerate}[(i)]
	\item Obtain a linear PDE system, using boundary and initial conditions.
	\item Separate variables to yield decoupled ODEs.
	\item Impose homogeneous boundary conditions to find eigenvalues and eigenfunctions.
	\item Use these eigenvalues (constants of separation) to find the eigenfunctions in the other variables.
	\item Sum over the products of separable solutions to find the general series solution.
	\item Determine coefficients for this series using the initial conditions.
\end{enumerate}
\begin{example}
	We will solve the wave equation instead in characteristic coordinates.
	Recall the sine and cosine summation identities:
	\[
		y(x,t) = \frac{1}{2} \sum_{n=1}^\infty \qty[ \qty(C_n \sin \frac{n \pi}{L}(x-ct) + D_n \cos \frac{n \pi}{L}(x-ct)) + \qty(C_n \sin \frac{n \pi}{L}(x+ct) - D_n \cos \frac{n \pi}{L}(x+ct)) ] = f(x-ct) + g(x+ct)
	\]
	The standing wave solution can be interpreted as a superposition of a right-moving wave and a left-moving wave.
	A special case is \( q(x) = 0 \), implying \( f = g = \frac{1}{2} p \).
	Then,
	\[
		y(x,t) = \frac{1}{2}\qty[p(x-ct) + p(x+ct)]
	\]
\end{example}
