\subsection{Diffusion equation and Fourier transform}
Recall the heat equation for a conducting wire given by
\[
	\pdv{\Theta}{t}\qty(x,t) - D\pdv[2]{\Theta}{x}\qty(x,t) = 0
\]
with initial conditions \( \Theta(x,0) = h(x) \) and boundary conditions \( \Theta \to 0 \) as \( x \to \pm \infty \).
Taking the Fourier transform with respect to \( x \),
\[
	\pdv{t} \widetilde \Theta(k,t) = -D k^2 \widetilde \Theta(k,t)
\]
Integrating, we find
\[
	\widetilde \Theta(k,t) = C e^{-D k^2 t}
\]
The initial conditions give \( \widetilde \Theta(k,0) = \widetilde h(k) \) and therefore
\[
	\widetilde \Theta(k,t) = \widetilde h(k) e^{-Dk^2 t}
\]
We take the inverse Fourier transform to find
\[
	\Theta(x,t) = \frac{1}{2\pi} \int_{-\infty}^\infty \widetilde h(k) \underbrace{e^{-Dk^2 t}}_{\mathclap{\text{FT of Gaussian}}} e^{ikx} \dd{k}
\]
Hence, by the convolution theorem,
\begin{align*}
	\Theta(x,t) & = \frac{1}{\sqrt{4 \pi D t}} \int_{-\infty}^\infty h(u) \exp(-\frac{(x-u)^2}{4Dt}) \dd{u} \\
	            & \equiv \int_{-\infty}^\infty h(u) S_d(x-u,t) \dd{u}
\end{align*}
where the \textit{fundamental solution} is
\[
	S_d(x,t) = \frac{1}{\sqrt{4 \pi D t}} \exp(-\frac{x^2}{4Dt})
\]
which is the Fourier transform of \( \exp(-D k^2 t) \).
Note, with localised initial conditions \( \Theta(x,0) = \Theta_0 \delta(x) \), the solution is exactly the fundamental solution:
\[
	\Theta(x,t) = \Theta_0 S_d(x,t) = \frac{\Theta_0}{\sqrt{4 \pi D t}} \exp(-\eta^2);\quad \eta = \frac{x}{2\sqrt{Dt}}
\]
where \( \eta \) is the similarity parameter.

\subsection{Gaussian pulse for heat equation}
Suppose that the initial conditions for the head equation are given by
\[
	f(x) = \sqrt{\frac{a}{\pi}} \Theta_0 e^{-ax^2}
\]
Then, our previous solution gives
\begin{align*}
	\Theta(x,t) & = \frac{\Theta_0 \sqrt{a}}{\sqrt{4 \pi^2 D t}} \int_{-\infty}^\infty \exp[-au^2 - \frac{(x-u)^2}{4 D t}] \dd{u}                                                      \\
	            & = \frac{\Theta_0 \sqrt{a}}{\sqrt{4 \pi^2 D t}} \int_{-\infty}^\infty \exp[-\frac{(1 + 4 a D t)u^2 - 2xu + x^2}{4 D t}] \dd{u}                                        \\
	            & = \frac{\Theta_0 \sqrt{a}}{\sqrt{4 \pi^2 D t}} \int_{-\infty}^\infty \exp[-\frac{1 + 4 a D t}{4 D t} \qty(u - \frac{x}{1+4aDt}) ] \exp[ \frac{-ax^2}{1+4aDt}] \dd{u}
\end{align*}
Recall that
\[
	\int_{-\infty}^\infty \exp[\frac{-(u - \mu)^2}{\sigma^2}] \dd{u} = \sigma \sqrt{\pi}
\]
The integral above is a Gaussian, so its solution can be read off directly as
\[
	\Theta(x,t) = \frac{\Theta_0 \sqrt{a}}{\sqrt{\pi (1 + 4 \pi^2 D t)}} \exp[\frac{-ax^2}{1+4aDt}]
\]
So the width of the Gaussian pulse will get wider over time, according to \( \sigma^2 \sim t \), as it evolves according to the heat equation.
The area is constant, so heat energy is conserved in the system.

\subsection{Forced diffusion equation}
Consider the equation
\[
	\pdv{t}\Theta(x,t) - D \pdv[2]{\Theta}{x}(x,t) = f(x,t)
\]
subject to homogeneous initial conditions \( \Theta(x,0) = 0 \).
We construct a two-dimensional Green's function \( G(x,t; \xi, \tau) \) such that
\[
	\pdv{t}G(x,t) - D \pdv[2]{G}{x}(x,t) = \delta(x - \xi)\delta(t - \tau)
\]
subject to the same homogeneous boundary conditions \( G(x,0;\xi,\tau) = 0 \).
Consider the Fourier transform with respect to \( x \).
\[
	\pdv{\widetilde G}{t} + D k^2 \widetilde G = e^{-ik\xi} \delta(t - \tau)
\]
We can solve this using an integrating factor \( e^{Dk^2 t} \) and integrating with respect to time.
Since \( G = 0 \) at \( t = 0 \),
\begin{align*}
	\pdv{t} \qty[ e^{D k^2 t} \widetilde G ]                    & = e^{-ik\xi + D k^2 t} \delta(t - \tau)                      \\
	\int_0^t \pdv{t'} \qty[ e^{D k^2 t'} \widetilde G ] \dd{t'} & = \int_0^t e^{-ik\xi + D k^2 t'} \delta(t' - \tau) \dd{t'}   \\
	e^{D k^2 t} \widetilde G                                    & = e^{-ik\xi} \int_0^t e^{D k^2 t'} \delta(t' - \tau) \dd{t'} \\
	e^{D k^2 t} \widetilde G                                    & = e^{-ik\xi} e^{D k^2 \tau} H(t - \tau)
\end{align*}
where \( H \) is the Heaviside step function.
Thus,
\[
	\widetilde G(k,t;\xi,\tau) = e^{-ik\xi} e^{-D k^2 (t - \tau)} H(t - \tau)
\]
The inverse Fourier transform gives the Green's function.
\[
	G(x,t;\xi,\tau) = \frac{H(t - \tau)}{2\pi} \int_{-\infty}^\infty e^{-ik\xi} e^{-D k^2 (t - \tau)} e^{ikx} \dd{k}
\]
This is a Gaussian; by changing variables into \( x' = x - \xi \) and \( t' = t - \tau \) we find
\[
	G(x,t;\xi,\tau) = \frac{H(t')}{2\pi} \int_{-\infty}^\infty e^{ikx'} e^{-D k^2 t'} \dd{k} = \frac{H(t')}{\sqrt{4 \pi D t'}} \exp[-\frac{(x')^2}{4Dt'}]
\]
Converting back,
\[
	G(x,t;\xi,\tau) = \frac{H(t - \tau)}{\sqrt{4 \pi D (t - \tau)}} \exp[-\frac{(x - \xi)^2}{4D(t - \tau)}] = H(t-\tau) S_d(x-\xi, t-\tau)
\]
where \( S_d \) is the fundamental solution as above.
Thus, the general solution is
\[
	\Theta(x,t) = \int_0^\infty \dd{\tau} \int_{-\infty}^\infty \dd{\xi} G(x,t;\xi,\tau) f(\xi, \tau)
\]
Let \( \xi = u \), then
\[
	\Theta(x,t) = \int_0^t \dd{\tau} \int_{-\infty}^\infty \dd{u} f(u, \tau) S_d(x-u, t-\tau)
\]

\subsection{Duhamel's principle}
In the above equation, omitting the integral over time, this is exactly the solution as found earlier with initial conditions at \( t = \tau \), which was
\[
	\Theta(x,t) = \int_{-\infty}^\infty \dd{u} f(u) S_d(x-u, t-\tau)
\]
The forced PDE with homogeneous boundary conditions can be related to solutions of the homogeneous PDE with inhomogeneous boundary conditions.
The forcing term \( f(x,t) \) at \( t = \tau \) acts as an initial condition for subsequent evolution.
Thus, the solution is a superposition of the effects of the initial conditions integrated over \( 0 < \tau < t \).
This relation between the homogeneous and inhomogeneous problems is known as \textit{Duhamel's principle}.

\subsection{Forced wave equation}
Consider the forced wave equation, given by
\[
	\pdv[2]{\phi}{t} - c^2 \pdv[2]{\phi}{x} = f(x,t)
\]
with \( \phi(x,0) = \phi_t(x,0) = 0 \).
We construct the Green's function using
\[
	\pdv[2]{G}{t} - c^2 \pdv[2]{G}{x} = \delta(x-\xi)\delta(t-\tau)
\]
with \( G(x,0) = \phi_t(x,0) = 0 \).
We take the Fourier transform with respect to \( x \), and find
\[
	\pdv[2]{\widetilde G}{t} + c^2 k^2 \widetilde G = e^{-ik\xi} \delta(t - \tau)
\]
We can solve this by inspection by comparing with the corresponding initial value problem Green's function, and find
\[
	\widetilde G = \begin{cases}
		0                                       & t < \tau \\
		e^{-ik\xi} \frac{\sin kc(t - \tau)}{kc} & t > \tau
	\end{cases}
\]
Using the Heaviside function.
\[
	\widetilde G = e^{-ik\xi} \frac{\sin kc(t - \tau)}{kc} H(t - \tau)
\]
We invert the Fourier transform.
\[
	G(x,t;\xi,\tau) = \frac{H(t-\tau)}{2\pi c} \int_{-\infty}^\infty e^{ik(x - \xi)} \frac{\sin kc(t - \tau)}{k} \dd{k}
\]
Let \( A = x - \xi \), and \( B = ct - \tau \).
By oddness of sine, only the cosine term of the complex exponential remains.
Noting the similarity to the Dirichlet discontinuous function,
\begin{align*}
	G(x,t;\xi,\tau) & = \frac{H(t-\tau)}{\pi c} \int_0^\infty \frac{\cos(kA) \sin(kB)}{k} \dd{k}          \\
	                & = \frac{H(t-\tau)}{2\pi c} \int_0^\infty \frac{\sin k(A+B) - \sin k(A-B)}{k} \dd{k} \\
	                & = \frac{H(t-\tau)}{4c} \qty[ \sgn(A+B) - \sgn(A-B) ]
\end{align*}
Since the \( H(t - \tau) \) term is nonzero only for \( t > \tau \), we must have \( B = c(t-\tau) > 0 \).
The only way that the bracketed term can be nonzero is when \( \abs{A} < B \); so \( \abs{x - \xi} < c(t-\tau) \).
This is the domain of dependence as found before, demonstrating the causality of the relation.
Hence,
\[
	G(x,t;\xi,\tau) = \frac{1}{2c} H(c(t-\tau) - \abs{x - \xi})
\]
Thus, the solution is
\begin{align*}
	\phi(x,t) & = \int_0^\infty \dd{\tau} \int_{-\infty}^\infty \dd{\xi} f(\xi, t) G(x,t;\xi,\tau)          \\
	          & = \frac{1}{2c} \int_0^t \dd{\tau} \int_{x - c(t-\tau)}^{x + c(t - \tau)} \dd{\xi} f(\xi, \tau)
\end{align*}

\subsection{Poisson's equation}
Consider
\[
	\laplacian{\phi} = - \rho(r)
\]
defined on a three-dimensional domain \( D \), with Dirichlet boundary conditions \( \phi = 0 \) on a boundary \( \partial D \).
The Dirac \( \delta \) function, when defined in \( \mathbb R^3 \), has the following properties.
\begin{enumerate}[(i)]
	\item \( \delta(r - r') = 0 \) for all \( r \neq r' \);
	\item \( \int_D \delta(r - r') \dd[3]{r} = 1 \) if \( r' \in D \), and zero otherwise;
	\item \( \int_D f(r) \delta(r - r') \dd[3]{r} = f(r') \).
\end{enumerate}
First, we consider \( D = \mathbb R^3 \) with the homogeneous boundary conditions that \( G \to 0 \) as \( \norm{r} \to \infty \).
This is known as the \textit{free-space} Green's function, denoted \( G_{\mathrm{FS}} \).
The potential here is spherically symmetric, so the Green's function is a function only of the distance between the point and the source.
WIthout loss of generality, let \( r' = 0 \), so \( G \) is a function only of the radius, now denoted \( r \).
Integrating the left hand side of Poisson's equation over a ball \( B \) with radius \( r \) around zero, we find
\[
	\int_B \laplacian{G_{\mathrm{FS}}} \dd[3]{r} = \int_{\partial B} \grad{G_{\mathrm{FS}}} \cdot \hat n \dd{S} = \int_{\partial B} \pdv{G}{r} r^2 \dd{\Omega}
\]
where \( \dd{\Omega} \) is the angle element.
This gives
\[
	\int_B \laplacian{G_{\mathrm{FS}}} \dd[3]{r} = 4 \pi r^2 \pdv{G_{\mathrm{FS}}}{r}
\]
The right hand side of Poisson's equation gives unity, since zero is contained in the ball.
Therefore,
\[
	\pdv{G_{\mathrm{FS}}}{r} = \frac{1}{4 \pi r^2} \implies G_{\mathrm{FS}} = \frac{-1}{4 \pi r} + c
\]
Since \( G \to 0 \) as \( r \to \infty \), we must have \( c = 0 \).
The fundamental solution is therefore the free-space Green's function given by
\[
	G(r; r') = \frac{-1}{4 \pi \norm{r - r'}}
\]
Thus, Poisson's equation is solved by
\[
	\Phi(r) = \frac{1}{4 \pi} \int_{\mathbb R^3} \frac{\rho(r')}{\norm{r - r'}} \dd[3]{r'}
\]

\subsection{Green's identities}
Consider scalar functions \( \phi, \psi \) which are twice differentiable on a domain \( D \).
By the divergence theorem, \textit{Green's first identity} is
\[
	\int_D \div(\phi \grad{\psi}) \dd[3]{r} = \int_D \qty(\phi \laplacian{\psi} + \grad{\phi} \cdot \grad{\psi}) \dd[3]{r} = \int_{\partial D} \phi \grad{\psi} \cdot \hat n \dd{S}
\]
Switching \( \psi \) and \( \phi \) and subtracting from the above, we arrive at \textit{Green's second identity}:
\[
	\int_{\partial D} \qty(\phi \pdv{\psi}{\hat n} - \psi \pdv{\phi}{\hat n}) \dd{S} = \int_D \qty(\phi \laplacian{\psi} - \psi \laplacian{\phi}) \dd[3]{r}
\]
Suppose we remove a ball \( \mathcal B_\varepsilon(r') \) from the domain.
Without loss of generality let \( r' = 0 \).
Let \( \phi \) be a solution to Poisson's equation, so \( \laplacian{\phi} = -\rho \) and let \( \psi \) be the free-space Green's function.
Thus, the right hand side of the second identity becomes
\[
	\int_{D \setminus \mathcal B_\varepsilon} \qty(\phi \laplacian{G_{\mathrm{FS}}} - G_{\mathrm{FS}} \laplacian{\phi}) \dd[3]{r} = \int_{D \setminus \mathcal B_\varepsilon} G_{\mathrm{FS}} \rho \dd[3]{r}
\]
The left hand side is
\[
	\int_{\partial D} \qty(\phi \pdv{G_{\mathrm{FS}}}{\hat n} - G_{\mathrm{FS}} \pdv{\phi}{\hat n}) \dd{S} + \int_{\partial \mathcal B_\varepsilon} \qty(\phi \pdv{G_{\mathrm{FS}}}{\hat n} - G_{\mathrm{FS}} \pdv{\phi}{\hat n}) \dd{S}
\]
For the second integral, we take the limit as \( \varepsilon \to 0 \).
Let \( \phi \) be regular, and let \( \overline\phi \) be the average value and \( \overline{\pdv{\phi}{\hat n}} \) be the average derivative.
This integral then becomes
\[
	\qty(\overline\phi \frac{-1}{4 \pi \varepsilon^2} - \frac{1}{4 \pi \varepsilon} \overline{\pdv{\phi}{\hat n}}) 4 \pi \varepsilon^2 \to -\phi(0)
\]
Combining the above, we find \textit{Green's third identity}, which is
\[
	\phi(r') = \int_D G_{\mathrm{FS}}(r;r')  \qty(-\rho(r)) \dd[3]{r} + \int_{\partial D} \qty(\phi(r) \pdv{G_{\mathrm{FS}}}{\hat n} \qty(r;r') - G_{\mathrm{FS}}(r;r') \pdv{\phi}{\hat n}\qty(r)) \dd{S}
\]
The second integral provides the ability to use inhomogeneous boundary conditions

\subsection{Dirichlet Green's function}
We will solve Poisson's equation \( \laplacian{\phi} = -\rho \) on \( D \) with inhomogeneous boundary conditions \( \phi(r) = h(r) \) on \( \partial D \).
The Dirichlet Green's function satisfies
\begin{enumerate}[(i)]
	\item \( \laplacian{G(r;r')} = 0 \) for all \( r \neq r \);
	\item \( G(r;r') = 0 \) on \( \partial D \);
	\item \( G(r;r') = G_{\mathrm{FS}}(r;r') + H(r;r') \) where \( H \) satisfies Laplace's equation, the homogeneous version of Poisson's equation, for all \( r \in D \).
\end{enumerate}
Green's second identity with \( \laplacian{\phi} = -\rho, \laplacian{H} = 0 \) gives
\[
	\int_{\partial D} \qty(\phi \pdv{H}{\hat n} - H \pdv{\phi}{\hat n}) \dd{S} = \int_D H \rho \dd[3]{r}
\]
Now, we set \( G_{\mathrm{FS}} = G - H \) into Green's third identity to find
\[
	\phi(r') = \int_D (G - H)(-\rho) \dd[3]{r} + \int_{\partial D} \qty(\phi \pdv{(G - H)}{\hat n} - (G-H) \pdv{\phi}{n}) \dd{S}
\]
All of the \( H \) terms can be cancelled by substituting the form of the second identity the derived above.
Now, given \( G = 0, \phi = h \) on \( \partial D \), we have
\[
	\phi(r') = \int_D G(r;r')(-\rho(r)) \dd[3]{r} + \int_{\partial D} h(r) \pdv{G(r;r')}{\hat n} \dd{S}
\]
This is the general solution.
The first integral is the Green's function solution, and the second integral yields the inhomogeneous boundary conditions.

\subsection{Method of images for Laplace's equation}
For symmetric domains \( D \), we can construct Green's functions with \( G = 0 \) on \( \partial D \) by cancelling the boundary potential out by using an opposite `mirror image' Green's function placed outside the domain.
Consider Laplace's equation \( \laplacian{\phi} = 0 \) on half of \( \mathbb R^3 \), in particular, the subset of \( \mathbb R^3 \) such that \( z > 0 \).
Let \( \phi(x,y,0) = h(x,y) \) and \( \phi \to 0 \) as \( r \to \infty \).
The free space Green's function satisfies \( G_{\mathrm{FS}} \to 0 \) as \( r \to \infty \), but does not satisfy the boundary condition that \( G_{\mathrm{FS}} = 0 \) at \( z = 0 \).
For \( G_{\mathrm{FS}} \) at \( r' = (x',y',z') \), we will subtract a copy of \( G_{\mathrm{FS}} \) located at \( r'' = (x',y',-z') \).
This gives
\begin{align*}
	G(r,r') & = \frac{-1}{4\pi \abs{r - r'}} - \frac{-1}{4\pi \abs{r - r''}}                                                 \\
	        & = \frac{-1}{4\pi \sqrt{(x-x')^2 + (y-y')^2 + (z-z')^2}} + \frac{1}{4\pi \sqrt{(x-x')^2 + (y-y')^2 + (z+z')^2}}
\end{align*}
Hence \( G((x,y,0), r') = 0 \), so this function satisfies the Dirichlet boundary conditions on all of the boundary \( \partial D \).
We have
\[
	\eval{\pdv{G}{\hat n}}_{z = 0} = \eval{\pdv{G}{z}}_{z = 0} = \frac{-1}{4\pi} \qty(\frac{z - z'}{\abs{r - r'}^3} - \frac{z + z'}{\abs{r-r'}^3}) = \frac{z'}{2\pi} \qty((x-x')^2 + (y-y')^2 + (z')^2)^{-3/2}
\]
The solution is then given by
\[
	\Phi(x',y',z') = \frac{z'}{2\pi} \int_{-\infty}^\infty \int_{-\infty}^\infty \qty[ (x - x')^2 + (y-y')^2 + (z')^2 ]^{-3/2} h(x,y) \dd{x}\dd{y}
\]

\subsection{Method of images for wave equation}
Consider the one-dimensional wave equation
\[
	\ddot \phi - c^2 \phi'' = f(x,t)
\]
with Dirichlet boundary conditions \( \phi(0,t) = 0 \).
We create matching Green's functions with an opposite sign centred at \( -\xi \).
\[
	G(x,t;\xi,\tau) = \frac{1}{2c} H(c(t-\tau) - \abs{x-\xi}) - \frac{1}{2c} H(c,(t-\tau) - \abs{x+\xi})
\]
We can replace the addition of the two terms with a subtraction to instead use Neumann boundary conditions.
Suppose we wish to solve the homogeneous problem with \( f = 0 \) for initial conditions of a Gaussian pulse.
Here, for \( x > 0 \) we have
\[
	\phi(x,t) = \exp[-(x-\xi + ct)^2] - \exp[-(-x - \xi + ct)^2]
\]
The solution travels to the left, cancelling with the image at \( t = \frac{\xi}{c} \), which emerges and travels right as the reflected wave.
