\subsection{Half-range series}
Consider \( f(x) \) defined only on \( 0 \leq x < L \).
We can extend the range of \( f \) to be the full range \( -1 \leq x < L \) in two simple ways:
\begin{enumerate}[(i)]
	\item require \( f \) to be odd, so \( f(-x) = -f(x) \).
	      Hence, \( a_n = 0 \) and
	      \[
		      b_n = \frac{2}{L} \int_0^L f(x) \sin \frac{n \pi x}{L} \dd{x}
	      \]
	      So
	      \[
		      So f(x) = \sum_{n=1}^\infty b_n \sin \frac{n\pi x}{L} \dd{x}
	      \]
	      which is called a Fourier sine series.
	\item require \( f \) to be even, so \( f(-x) = f(x) \).
	      In this case, \( b_n = 0 \) and
	      \[
		      a_n = \frac{2}{L} \int_0^L f(x) \cos \frac{n \pi x}{L} \dd{x}
	      \]
	      and
	      \[
		      So f(x) = \frac{1}{2}a_0 + \sum_{n=1}^\infty a_n \cos \frac{n\pi x}{L} \dd{x}
	      \]
	      which is a Fourier cosine series.
\end{enumerate}

\subsection{Complex representation of Fourier series}
Recall that
\begin{align*}
	\cos \frac{n \pi x}{L} & = \frac{1}{2}\qty(e^{i n \pi x / L} + e^{-i n \pi x / L});  \\
	\sin \frac{n \pi x}{L} & = \frac{1}{2i}\qty(e^{i n \pi x / L} - e^{-i n \pi x / L});
\end{align*}
Therefore, a Fourier series can be written as
\begin{align*}
	f(x) & = \frac{1}{2} a_0 + \frac{1}{2} \sum_{n=1}^\infty \qty[ (a_n - i b_n) e^{i n \pi x / L} + (a_n + i b_n) e^{-i n \pi x / L} ]
	     & = \sum_{m=-\infty}^\infty c_m e^{i m \pi x / L}
\end{align*}
where for \( m > 0 \) we have \( m=n, c_m = \frac{1}{2}(a_n - ib_n) \), and for \( m < 0 \) we have \( n = -m, c_m = \frac{1}{2}(a_{-m} + ib_{-m}) \), and where \( m = 0 \) we have \( c_0 = \frac{1}{2} a_0 \).
In particular,
\[
	c_m = \frac{1}{2L} \int_{-L}^L f(x) e^{-i m \pi x / L} \dd{x}
\]
where the negative sign comes from the complex conjugate.
This is because, for complex-valued \( f, g \), we have
\[
	\inner{f,g} = \int_{-L}^L f^\star g \dd{x}
\]
The orthogonality conditions are
\[
	\int_{-L}^L e^{-i m \pi x / L} e^{i n \pi x / L} \dd{x} = 2L \delta_{mn}
\]
Parseval's theorem now states
\[
	\int_{-L}^L f^\star(x) f(x) \dd{x} = \int_{-L}^L \abs{f(x)}^2 \dd{x} = 2L \sum_{m=-\infty}^\infty \abs{c_m}^2
\]

\subsection{Self-adjoint matrices}
\textit{Much of this section is a recap of IA Vectors and Matrices.}
Suppose that \( u, v \in \mathbb C^N \) with inner product
\[
	\inner{u,v} = u^\dagger v
\]
The \( N \times N \) matrix \( A \) is \textit{self-adjoint}, or \textit{Hermitian}, if
\[
	\forall u,v \in \mathbb C^N, \inner{Au, v} = \inner{u, Av} \iff A^\dagger = A
\]
The eigenvalues \( \lambda_n \) and eigenvectors \( v_n \) satisfy
\[
	A v_n = \lambda_n v_n
\]
They have the following properties:
\begin{enumerate}[(i)]
	\item \( \lambda_n^\star = \lambda_n \);
	\item \( \lambda_n \neq \lambda_m \implies \inner{ v_n, v_m } = 0 \);
	\item we can create an orthonormal basis from the eigenvectors.
\end{enumerate}
Given \( b \in \mathbb C^n \), we can solve for \( x \) in the general matrix equation \( Ax = b \) by expressing \( b \) in terms of the eigenvector basis:
\[
	b = \sum_{n=1}^N b_n v_n
\]
We seek a solution of the form
\[
	x = \sum_{n=1}^N c_n v_n
\]
At this point, the \( b_n \) are known and the \( c_n \) are our target.
Substituting into the matrix equation, orthogonality of basis vectors gives
\begin{align*}
	A \sum_{n=1}^N c_n v_n         & = \sum_{n=1}^N b_n v_n  \\
	\sum_{n=1}^N c_n \lambda_n v_n & = \sum_{n=1}^N b_n v_n  \\
	c_n \lambda_n                  & = b_n                   \\
	c_n                            & = \frac{b_n}{\lambda_n} \\
\end{align*}
Therefore,
\[
	x = \sum_{n=1}^N \frac{b_n}{\lambda_n} v_n
\]
provided \( \lambda_n \neq 0 \), or equivalently, the matrix is invertible.

\subsection{Solving inhomogeneous ODEs with Fourier series}
We wish to find \( y(x) \) given a source term \( f(x) \) for the general differential equation
\[
	\mathcal L y \equiv -\dv[2]{y}{x} = f(x)
\]
with boundary conditions \( y(0) = y(L) = 0 \).
The related eigenvalue problem is
\[
	\mathcal L y_n = \lambda_n y_n,\quad y_n(0) = y_n(L) = 0
\]
which has solutions
\[
	y_n(x) = \sin \frac{n \pi x}{L}, \lambda_n = \qty(\frac{n\pi}{L})^2
\]
We can show that this is a self-adjoint linear operator with orthogonal eigenfunctions.
We seek solutions of the form of a half-range sine series.
Consider
\[
	y(x) = \sum_{n=1}^\infty c_n \sin\frac{n \pi x}{L}
\]
The right hand side is
\[
	f(x) = \sum_{n=1}^\infty b_n \sin \frac{n \pi x}{L}
\]
We can find \( b_n \) by
\[
	b_n = \frac{2}{L} \int_0^L f(x) \sin \frac{n \pi x}{L} \dd{x}
\]
Substituting, we have
\[
	\mathcal L y = -\dv[2]{x} \qty(\sum_n c_n \sin \frac{n \pi x}{L}) = \sum_n c_n \qty(\frac{n\pi}{L})^2 \sin \frac{n \pi x}{L} = \sum_n b_n \sin \frac{n \pi x}{L}
\]
By orthogonality,
\[
	c_n \qty(\frac{n \pi}{L})^2 = b_n \implies c_n = \qty(\frac{L}{n \pi})^2 b_n
\]
Therefore the solution is
\[
	y(x) = \sum_n \qty(\frac{L}{n \pi})^2 b_n \sin \frac{n \pi x}{L} = \sum_n \frac{b_n}{\lambda_n} y_n
\]
which is equivalent to the solution we found for self-adjoint matrices for which the eigenvalues and eigenvectors are known.
\begin{example}
	Consider an odd square wave with \( L = 1 \), so \( f(x) = 1 \) from \( 0 \leq x < 1 \).
	\[
		f(x) = 4 \sum_m \frac{\sin(2m-1)\pi x}{(2m-1)\pi}
	\]
	Then the solution to \( \mathcal L y = f \) should be (with odd \( n = 2m-1 \))
	\[
		y(x) = \sum_n \frac{b_n}{\lambda_n} y_n = 4 \sum_n \frac{\sin (2m-1) \pi x}{((2m - 1) \pi)^3}
	\]
	This is exactly the Fourier series for
	\[
		y(x) = \frac{1}{2}x(1-x)
	\]
	so this \( y \) is the solution to the differential equation.
	We can in fact integrate \( \mathcal L y = 1 \) directly with the boundary conditions to verify the solution.
	We can also differentiate the Fourier series for \( y \) twice to find the square wave.
\end{example}
