\subsection{???}
Independence results are found across mathematical disciplines.
\begin{enumerate}
    \item The \emph{parallel postulate} is independent from the other four postulates of Euclidean geometry.
    It states that for any given point not on a line, there is a unique line passing through that point that does not intersect the given line.
    In the 19th century, it was shown that the other four postulates are satisfied by hyperbolic geometry, but this postulate is not satisfied.
    This shows that the other four axioms are insufficient to prove the parallel postulate.
    \item Let \( \varphi \) be the statement in the language of fields describing the existence of a square root of 2.
    We know that \( \mathbb Q \) is a field satisfying \( \neg\varphi \), and \( \mathbb Q[\sqrt{2}] \) satisfies \( \varphi \).
    The fields \( \mathbb Q \) and \( \mathbb Q[\sqrt{2}] \) are models of the theory of fields, one of which satisfies \( \varphi \), and one of which satisfies \( \neg\varphi \).
    This shows that the theory of fields does not prove \( \varphi \) or \( \neg\varphi \).
    A similar result holds for the statement \( \varphi \) that says that there are no roots of \( x^4 = -1 \).
    \item G\"odel's incompleteness theorem implies that there must always be an independence result in a sufficiently powerful consistent set theory.
\end{enumerate}
We will show that there are other independence results in set theory that are not self-referential like the G\"odel incompleteness theorems.
\begin{theorem}[Cantor]
    \( \abs{\mathbb N} < \abs{\mathcal P(\mathbb N)} \).
\end{theorem}
The continuum hypothesis is that there are no sets of cardinality strictly between \( \abs{\mathbb N} \) and \( \abs{\mathcal P(N)} = \abs{\mathbb R} \).
\begin{definition}
    The \emph{continuum hypothesis} \( \mathsf{CH} \) states that if \( X \subseteq \mathbb P(\mathbb N) \) is infinite, then either \( \abs{X} = \abs{\mathbb N} \) or \( \abs{X} = \abs{\mathcal P(\mathbb N)} \), or equivalently,
    \[ 2^{\aleph_0} = \aleph_1 \]
\end{definition}
Progress was made on the continuum hypothesis in the 19th and 20th centuries.
\begin{enumerate}
    \item In 1883, Cantor showed that any closed subset of \( \mathbb R \) satisfies \( \mathsf{CH} \).
    \item In 1916, Alexandrov and Hausdorff showed that any Borel set of \( \mathbb R \) satisfies \( \mathsf{CH} \).
    \item In 1930, Suslin strengthened this result to analytic subsets of \( \mathbb R \).
    \item In 1938, G\"odel showed that if \( \mathsf{ZF} \) is consistent, then so is \( \mathsf{ZFC} + \mathsf{CH} \).
    \item However, in 1963, Cohen showed that if \( \mathsf{ZF} \) is consistent, then so is \( \mathsf{ZFC} + \neg\mathsf{CH} \).
\end{enumerate}
In this course, we will prove results (iv) and (v), thus establishing the independence of the continuum hypothesis from \( \mathsf{ZFC} \).

\subsection{Systems of set theory}
The language of set theory \( \mathcal L = \mathcal L_\in \) is a first-order predicate logic with equality and membership as primitive relations.
We assume the existence of infinitely many variables \( v_1, v_2, \dots \) denoting sets.
We will only use the logical connectives \( \vee \) and \( \neg \) as well as the existential quantifier \( \exists \).
Conjunction, implication, and universal quantification can be defined in terms of disjunction, negation, and existential quantification.

We say that an occurrence of a variable \( x \) is \emph{bound} in a formula \( \varphi \) if is in a quantifier \( \exists x \) or lies in the scope of such a quantifier.
An occurrence is called \emph{free} if it is not bound.
We write \( FV(\varphi) \) for the set of free variables of \( \varphi \).
We will write \( \varphi(u_1, \dots, u_n) \) to emphasise the dependence of \( \varphi \) on its free variables \( u_1, \dots, u_n \).
By doing so, we will allow ourselves to freely change the names of the free variables, and assume that substituted variables are free.
The syntax \( \varphi(u_0, \dots, u_n) \) does not imply that \( u_i \) occurs freely, or even at all.

Some of the most common axioms of set theory are as follows.
\begin{enumerate}
    \item \emph{Axiom of extensionality.}
    \[ \forall x.\, \forall y.\, (\forall z.\, (z \in x \leftrightarrow x \in y) \to x = y) \]
    \item \emph{Axiom of empty set.}
    \[ \exists x.\, \forall y \in x.\, y \neq y \]
    \item \emph{Axiom of pairing.}
    \[ \forall x.\, \forall y.\, \exists z.\, (x \in z \wedge y \in z) \]
    \item \emph{Axiom of union.}
    \[ \forall a.\, \exists x.\, \forall y.\, (y \in x \leftrightarrow \exists z \in a.\, y \in z) \]
    \item \emph{Axiom of foundation.}
    \[ \forall x.\, (\exists y.\, y \in x \to \exists y \in x.\, \neg\exists z \in x.\, z \in y) \]
    \item \emph{Axiom scheme of separation.} For any formula \( \varphi \),
    \[ \forall a.\, \exists x.\, \forall y.\, (y \in x \leftrightarrow (y \in a \wedge \varphi(y))) \]
    \item \emph{Axiom of infinity.}
    \[ \exists a.\, (\exists x.\, (x \in a) \wedge \forall x \in a.\, \exists y \in a.\, x \in y) \]
    \item \emph{Axiom of power set.}
    \[ \forall a.\, \exists x.\, \forall y.\, (y \in x \leftrightarrow \forall z.\, (z \in y \to z \in a)) \]
    \item \emph{Axiom scheme of replacement.} For any formula \( \varphi \),
    \[ \forall a.\, (\forall x \in a.\, \exists! y.\, \varphi(x, y) \to \exists b.\, \forall x \in a.\, \exists y \in b.\, \varphi(x, y)) \]
    \item[(ix\('\))] \emph{Axiom scheme of collection.} For any formula \( \varphi \),
    \[ \forall a.\, (\forall x \in a.\, \exists y.\, \varphi(x, y) \to \exists b.\, \forall x \in a.\, \exists y \in b.\, \varphi(x, y)) \]
    \item \emph{Axiom of choice.}
    \[ \forall X.\, \qty(\varnothing \notin X \to \exists f : \qty(X \to \bigcup X).\, \forall a \in X.\, f(a) \in a) \]
    \item[(x\('\))] \emph{Well-ordering principle.}
    \[ \forall a.\, \exists R.\, R \text{ is a well-ordering of } a \]
\end{enumerate}
Some common set theories are as follows.
\begin{itemize}
    \item \emph{Zermelo set theory} \( \mathsf{Z} \) consists of axioms (i) to (viii).
    Axioms (ix) and (ix\('\)) are equivalent relative to \( \mathsf{Z} \).
    \item \emph{Zermelo--Fraenkel set theory} \( \mathsf{ZF} \) consists of axioms (i) to (ix).
    Axioms (x) and (x\('\)) are equivalent relative to \( \mathsf{ZF} \).
    \item \emph{Zermelo--Fraenkel set theory with choice} \( \mathsf{ZFC} \) consists of axioms (i) to (x).
    \item \emph{Zermelo--Fraenkel set theory without power set} \( \mathsf{ZF}^- \) consists of axioms (i) to (vii), with the axiom of collection (ix\('\)) instead of replacement (ix); it has been shown that (ix) is weaker than (ix\('\)) in the presence of axioms (i) to (vii).
    \item \emph{Zermelo--Fraenkel set theory with choice and without power set} \( \mathsf{ZFC}^- \) consists of axioms (i) to (vii), with the axiom of collection (ix\('\)) and the well-ordering principle (x\('\)).
\end{itemize}
In this course, our main metatheory will be \( \mathsf{ZF} \), and we will be explicit about the use of choice.

We say that a class \( X \) is \emph{definable} over \( M \) if there exists a formula \( \varphi \) and sets \( a_1, \dots, a_n \in M \) such that for all \( z \in M \), we have \( z \in X \) if and only if \( \varphi(z, a_1, \dots, a_n) \).
A class is \emph{proper} over \( M \) if it is not a set in \( M \).
In this course, we will assume that all mentioned classes are definable.
For example, the universe class \( V = \qty{x \mid x = x} \), the Russell class \( R = \qty{x \mid x \notin x} \), and the class of ordinals are all definable.
Any set is a definable class.
Classes are heavily dependent on the underlying model: if \( M = 2 \) then \( \mathrm{Ord} = 2 = M \), and if \( M = 3 \cup \qty{1} \) then \( \mathrm{Ord} = 3 \neq M \).

\subsection{Adding defined functions}
Often in set theory, we use symbols such as \( 0, 1, \subseteq, \cap, \wedge, \forall \); they do not exist in our language.
\begin{definition}
    Suppose that \( \mathcal L \subseteq \mathcal L' \) and \( T \) is a set of sentences in \( \mathcal L \).
    We say that \( P \) is a \emph{defined \( n \)-ary predicate} symbol over \( T \) if there is a formula \( \varphi \) in \( \mathcal L \) such that
    \[ T \vdash \forall x_1, \dots, x_n.\, (P(x_1, \dots, x_n) \leftrightarrow \varphi(x_1, \dots, x_n)) \]
    Similarly, we say that \( f \) is a \emph{defined \( n \)-ary function} symbol over \( T \) if there is a formula \( \varphi \) in \( \mathcal L \) such that
    \[ f(x_1, \dots, x_n) = y \text{ if and only if } T \vdash \varphi(x_1, \dots, x_n, y) \]
    and
    \[ T \vdash \forall x_1, \dots, x_n.\, \exists! y.\, \varphi(x_1, \dots, x_n, y) \]
    We say that a set of sentences \( T' \) of \( \mathcal L' \) is an \emph{extension by definitions} of \( T \) over \( \mathcal L \) when \( T' = T \cup S \) and \( S = \qty{\varphi_s \mid s \in \mathcal L' \setminus \mathcal L'} \) and each \( \varphi_s \) is a definition of \( s \) in the language \( \mathcal L \) over \( T \).
\end{definition}
Commonly used symbols such as \( 0, 1, \subseteq, \cap, \mathcal P, \bigcup \) are defined over \( \mathsf{ZF} \).
\begin{theorem}
    Suppose that \( \mathcal L \subseteq \mathcal L' \), and that \( T \) is a set of \( \mathcal L \)-sentences and \( T' \) is an extension by definitions of \( T \) to \( \mathcal L' \).
    Then
    \begin{enumerate}
        \item (conservativity) If \( \varphi \) is a sentence of \( \mathcal L \), then \( T \vdash \varphi \leftrightarrow T' \vdash \varphi \).
        \item (abbreviations) If \( \varphi \) is a formula of \( \mathcal L' \), then there exists a formula \( \hat\varphi \) of \( \mathcal L \) whose free variables are exactly those of \( \varphi \), such that \( T' \vdash \forall x.\, (\varphi \leftrightarrow \hat\varphi) \).
    \end{enumerate}
\end{theorem}
\begin{example}
    The intersection \( a \cap b \) can be defined as the unique set \( c \) such that
    \[ \forall x.\, (x \in c \iff x \in a \wedge x \in b) \]
    This definition makes sense only if there is a unique \( c \) satisfying this formula \( \varphi(c) \).
    If
    \[ M = \qty{a, c, d, \qty{a}, \qty{a, b}, \qty{a, b, c}, \qty{a, b, d}} \]
    then it is easy to check that both \( \qty{a} \) and \( \qty{a, d} \) satisfy \( \varphi \), so intersection cannot be defined.
\end{example}
