In this section, we will prove
\[ \Con(\mathsf{ZF}) \to \Con(\mathsf{ZFC} + \mathsf{GCH}) \]

\subsection{Definable sets}
Recall that the \( \mathrm{V}_\alpha \) hierarchy has the property that \( \mathrm{V}_{\alpha + 1} = \mathcal P(\mathrm{V}_\alpha) \).
We will construct a universe \( \mathrm{L} \) in which we restrict to the `nice' subsets.
\begin{definition}
    A set \( x \) is said to be \emph{definable} over \( (M, \in) \) if there exist \( a_1, \dots, a_n \in M \) and a formula \( \varphi \) such that
    \[ x = \qty{z \in M \mid (M, \in) \vDash \varphi(z, a_1, \dots, a_n)} \]
    We write
    \[ \operatorname{Def}(M) = \qty{x \subseteq M \mid x \text{ is definable over } M} \]
\end{definition}
\begin{remark}
    \begin{enumerate}
        \item \( M \in \operatorname{Def}(M) \).
        \item \( M \subseteq \operatorname{Def}(M) \subseteq \mathcal \mathcal P(M) \).
    \end{enumerate}
\end{remark}
This definition involves a quantification over infinitely many formulas, so is not yet fully formalised.
One method to do this is to code formulas as elements of \( \mathrm{V}_\omega \), called \emph{G\"odel codes}.
We can then use Tarski's satisfaction relation to define a formula \( \mathsf{Sat} \), and can then prove
\[ \mathsf{Sat}(M, E, \ulcorner \varphi \urcorner, x_1, \dots, x_n) \leftrightarrow (M, \in) \vDash \varphi(x_1, \dots, x_n) \]
where \( \ulcorner \varphi \urcorner \in V_\omega \) is the G\"odel code for \( \varphi \).
We will later use a different method to formalise it, but for now we will assume that this is well-defined.

\subsection{Defining the constructible universe}
We define the \( \mathrm{L}_\alpha \) hierarchy by transfinite recursion as follows.
\[ \mathrm{L}_0 = \varnothing;\quad \mathrm{L}_{\alpha + 1} = \operatorname{Def}(\mathrm{L}_\alpha);\quad \mathrm{L}_\lambda = \bigcup_{\alpha < \lambda} \mathrm{L}_\alpha;\quad \mathrm{L} = \bigcup_{\alpha \in \mathrm{Ord}} \mathrm{L}_\alpha \]
\begin{lemma}
    For any ordinals \( \alpha, \beta \),
    \begin{enumerate}
        \item if \( \beta \leq \alpha \) then \( \mathrm{L}_\beta \subseteq \mathrm{L}_\alpha \);
        \item if \( \beta < \alpha \) then \( \mathrm{L}_\beta \in \mathrm{L}_\alpha \);
        \item \( \mathrm{L}_\alpha \) is transitive;
        \item the ordinals of \( \mathrm{L}_\alpha \) are precisely \( \alpha \);
        \item \( \mathrm{L} \) is transitive and \( \mathrm{Ord} \subseteq L \).
    \end{enumerate}
\end{lemma}
\begin{definition}
    Let \( T \) be a set of axioms in \( \mathcal L_\in \), and let \( W \) be a class.
    Then \( W \) is called an \emph{inner model} of \( T \) if
    \begin{enumerate}
        \item \( W \) is a transitive class;
        \item \( \mathrm{Ord} \subseteq W \);
        \item \( T^W \) is true; that is, for every formula \( \varphi \) in \( T \), we have \( \varphi^W \).
    \end{enumerate}
\end{definition}
\begin{theorem}
    \( \mathrm{L} \) is an inner model of \( \mathsf{ZF} \).
\end{theorem}
This is a theorem scheme; for every axiom of \( \mathsf{ZF} \), we can prove its relativisation to \( \mathrm{L} \).
\begin{proof}
    By the previous lemma, it suffices to check that \( \mathsf{ZF}^{\mathrm{L}} \) holds.
    \begin{itemize}
        \item Since \( \mathrm{L} \) is transitive, \( \mathrm{L} \) satisfies extensionality and foundation.
        \item For the axiom of empty set, we use the fact that \( \varnothing^{\mathrm{L}} = \varnothing = \mathrm{L}_0 = \mathrm{L} \).
        \item For pairing, given \( a, b \in \mathrm{L} \), we must show \( \qty{a, b} \in \mathrm{L} \).
        Fix \( \alpha \) such that \( a, b \in \mathrm{L}_\alpha \).
        Then
        \[ \qty{a, b} = \qty{x \in \mathrm{L}_\alpha \mid (\mathrm{L}_\alpha, \in) \vDash x = a \vee x = b} \in \operatorname{Def}(\mathrm{L}_\alpha) \]
        \item For union, let \( a \in \mathrm{L}_\alpha \).
        By transitivity, \( \bigcup a \subseteq \mathrm{L}_\alpha \).
        Then
        \[ \bigcup a = \qty{x \in \mathrm{L}_\alpha \mid (\mathrm{L}_\alpha, \in) \vDash \exists z.\, (z \in a \wedge x \in z)} \in \operatorname{Def}(\mathrm{L}_\alpha) \]
        \item For infinity, note that
        \[ \omega = \qty{n \in \mathrm{L}_\omega \mid (\mathrm{L}_\omega, \in) \vDash n \in \mathrm{Ord}} \in \operatorname{Def}(\mathrm{L}_\alpha) \]
        \item Consider separation.
        Let \( \varphi \) be a formula, and let \( a, \vb u \in \mathrm{L}_\alpha \).
        We claim that
        \[ b = \qty{x \in a \mid \varphi^{\mathrm{L}}(x, \vb u)} \in \mathrm{L} \]
        This implicitly uses the fact that \( \mathrm{L} \) is definable.
        Using the reflection theorem, there is \( \beta > \alpha \) such that
        \[ \mathsf{ZF} \vdash \forall x \in \mathrm{L}_\beta.\, (\varphi^{\mathrm{L}}(x, \vb u) \leftrightarrow \varphi^{\mathrm{L}_\beta}(x, \vb u)) \]
        Moreover, \( \varphi^{\mathrm{L}_\beta}(x, \vb u) \) holds if and only if \( (\mathrm{L}_\beta, \in) \vDash \varphi(x, \vb u) \).
        We thus obtain
        \[ \qty{x \in a \mid \varphi^{\mathrm{L}}(x, \vb u)} = \qty{x \in a \mid \varphi^{\mathrm{L}_\beta}(x, \vb u)} = \qty{x \in \mathrm{L}_\beta \mid (\mathrm{L}_\beta, \in) \vDash \varphi(x, \vb u) \wedge x \in a} \in \operatorname{Def}(\mathrm{L}_\beta) \]
        \item We now consider replacement.
        It suffices to show that if \( a \in \mathrm{L} \) and \( f : a \to \mathrm{L} \) is a definable function, then there exists \( \gamma \in \mathrm{Ord} \) such that \( f '' a \subseteq \mathrm{L}_\gamma \), since then we can use separation.
        First, observe that for every \( x \in a \), there exists \( \beta \in \mathrm{Ord} \) such that \( f(x) \in \mathrm{L}_\beta \).
        Using replacement in \( \mathrm{V} \), there exists an ordinal \( \gamma \) such that for all \( x \in a \), there exists \( \beta < \gamma \) such that \( f(x) \in \mathrm{L}_\beta \).
        As \( \mathrm{L}_\beta \subseteq \mathrm{L}_\gamma \), we thus obtain for all \( x \in a \) that \( f(x) \in \mathrm{L}_\gamma \).
        \item Finally, consider the axiom of power set.
        It suffices to prove that if \( x \in \mathrm{L} \) then \( \mathcal P(x) \cap \mathrm{L} \in \mathrm{L} \).
        Take \( x \in \mathrm{L} \).
        Using replacement in \( \mathrm{V} \), we can fix an ordinal \( \gamma \) such that \( \mathcal P(x) \cap \mathrm{L} \subseteq \mathrm{L}_\gamma \).
        Then
        \[ \mathcal P(x) \cap \mathrm{L} = \qty{z \in \mathrm{L}_\gamma \mid (\mathrm{L}, \in) \vDash z \subseteq x} \in \operatorname{Def}(\mathrm{L}_\gamma) \]
    \end{itemize}
\end{proof}

\subsection{G\"odel functions}
We will now formally define \( \mathrm{L} \).
For clarity, we will define the ordered triple \( \langle a, b, c \rangle \) to be \( \langle a, \langle b, c \rangle \rangle \).
\begin{definition}
    The \emph{G\"odel functions} are the following collection of functions on two variables.
    \begin{enumerate}
        \item \( \mathcal F_1(x, y) = \qty{x, y} \);
        \item \( \mathcal F_2(x, y) = \bigcup x \);
        \item \( \mathcal F_3(x, y) = x \setminus y \);
        \item \( \mathcal F_4(x, y) = x \times y \);
        \item \( \mathcal F_5(x, y) = \dom x = \qty{\pi_1(z) \mid z \in x \wedge z \text{ is an ordered pair}} \);
        \item \( \mathcal F_6(x, y) = \operatorname{ran} x \qty{\pi_2(z) \mid z \in x \wedge z \text{ is an ordered pair}} \);
        \item \( \mathcal F_7(x, y) = \qty{\langle u, v, w \rangle \mid \langle u, v \rangle \in x, w \in y} \);
        \item \( \mathcal F_8(x, y) = \qty{\langle u, w, v \rangle \mid \langle u, v \rangle \in x, w \in y} \);
        \item \( \mathcal F_9(x, y) = \qty{\langle v, u \rangle \in y \in x \mid u = v} \);
        \item \( \mathcal F_{10}(x, y) = \qty{\langle v, u \rangle \in y \times x \mid u \in v} \).
    \end{enumerate}
\end{definition}
\begin{proposition}
    The following can all be written as a finite combination of G\"odel functions (i)--(vii).
    \[ \qty{x};\quad x \cup y;\quad x \cap y;\quad \langle x, y \rangle;\quad \langle x, y, z \rangle \]
\end{proposition}
\begin{proposition}
    For every \( i \in \qty{1, \dots, 10} \), the statement \( z = \mathcal F_i(x, y) \) can be written using a \( \Delta_0 \) formula.
    Hence, these formulas are absolute.
\end{proposition}
\begin{lemma}[G\"odel normal form]
    For every \( \Delta_0 \) formula \( \varphi(x_1, \dots, x_n) \) with free variables contained in \( \qty{x_1, \dots, x_n} \), there is a term \( \mathcal F_\varphi \) built from the symbols \( \mathcal F_1, \dots, \mathcal F_{10} \) such that
    \[ \mathsf{ZF} \vdash \forall a_1, \dots, a_n.\, \mathcal F_\varphi(a_1, \dots, a_n) = \qty{\langle x_n, \dots, x_1 \rangle \in a_n \times \dots \times a_1 \mid \varphi(x_1, \dots, x_n)} \]
\end{lemma}
\begin{remark}
    \begin{enumerate}
        \item The reversed order of the free variables is done purely for technical reasons.
        \item \( \mathcal F_2 \) will correspond to disjunction for \( \Delta_0 \) formulas, intersection will correspond to intersection, \( \mathcal F_3 \) will give negation, and \( \mathcal F_9 \) and \( \mathcal F_{10} \) will give atomic formulas.
        \item \( \mathcal F_7 \) and \( \mathcal F_8 \) will deal with ordered \( n \)-tuples.
        The triple \( \langle x_1, x_2, x_3 \rangle \), this is formed using \( x_1 \) and \( \langle x_2, x_3 \rangle \).
        However, it cannot be formed using \( \langle x_1, x_2 \rangle \) and \( x_3 \).
    \end{enumerate}
\end{remark}
\begin{proof}
    We show this by induction on the class \( \Delta_0 \).
    We call a formula \( \varphi \) a \emph{termed formula} if the conclusion of the lemma holds for \( \varphi \); we aim to show that every \( \Delta_0 \)-formula is a termed tormula.
    We will only use the logical symbols \( \wedge, \vee, \neg, \exists \), and the only occurrence of existential quantification will be in formulas of the form
    \[ \varphi(x_1, \dots, x_n) \equiv \exists x_{n+1} \in x_j.\, \psi(x_1, \dots, x_{n+1}) \]
    where \( j \leq m \leq n \).
    For example, we allow \( \exists x_3 \in x_1.\, (x_1 \in x_2 \wedge x_3 = x_1) \), but we disallow \( \exists x_1 \in x_2.\, \psi \) and \( \exists x_3 \in x_1.\, (x_3 \in x_2 \wedge \exists x_4 \in x_1.\, \psi) \).
    Every \( \Delta_0 \)-formula is equivalent to one of this form.
    We allow for dummy variables, so \( \varphi(x_1, x_2) \equiv x_1 \in x_2 \) and \( \varphi(x_1, x_2, x_3) \equiv x_1 \in x_2 \) are distinct.
    This proof will take place in four parts: first some logical points, then we consider propositional formulas, then atomic formulas, and finally bounded existentials.

    \emph{Part (i): logical points.}
    We make the following remarks.
    \begin{itemize}
        \item If \( \mathsf{ZF} \vdash \varphi(\vb x) \leftrightarrow \psi(\vb x) \) and \( \varphi(\vb x) \) is a termed formula, then \( \psi \) is also a termed formula.
        This is immediate from the definition, since we can let \( \mathcal F_\psi = \mathcal F_\varphi \).
        \item For all \( m, n \), if \( \varphi(x_1, \dots, x_n) \equiv \psi(x_1, \dots, x_m) \) and \( \psi \) is a termed formula, then so is \( \varphi \).
        If \( n \geq m \), we can show this by induction on \( n \).
        The base case \( n = m \) is trivial.
        For the inductive step, suppose
        \[ \varphi(x_1, \dots, x_{n+1}) \equiv \psi(x_1, \dots, x_m) \]
        Then, we can write
        \[ \varphi(x_1, \dots, x_{n+1}) \equiv \theta(x_1, \dots, x_n) \]
        where \( \theta \) is a termed formula.
        Then
        \[ \mathcal F_\varphi(a_1, \dots, a_n, a_{n+1}) = a_{n+1} \times \mathcal F_\theta(a_1, \dots, a_n) = \mathcal F_4(a_{n+1}, \mathcal F_\theta(a_1, \dots, a_n)) \]
        giving the result by the inductive hypothesis.
        This is the reason for reversing the order: because the ordered triple \( \langle x, y, z \rangle \) is \( \langle x, \langle y, z \rangle \rangle \), the map
        \[ \qty{\langle x_1, x_2 \rangle \in a_1 \times a_2 \mid \theta(x_1, x_2)} \mapsto \qty{\langle x_1, x_2, x_3 \rangle \in a_1 \times a_2 \times a_3 \mid \theta(x_1, x_2)} \]
        is much more complicated to implement in G\"odel functions.
        We prove the case \( n \leq m \) by induction; if
        \[ \varphi(x_1, \dots, x_{n-1}) \equiv \psi(x_1, \dots, x_m) \]
        then
        \[ \varphi(x_1, \dots, x_{n-1}) \equiv \theta(x_1, \dots, x_n) \]
        and
        \[ \qty{0} = \qty{\mathcal F_3(a_1, a_1)} = \mathcal F_1(\mathcal F_3(a_1, a_1), \mathcal F_3(a_1, a_1)) \]
        Then
        \begin{align*}
            \mathcal F_\varphi(a_1, \dots, a_{n-1}) &= \qty{\langle x_{n-1}, \dots, x_1 \rangle \in a_{n-1} \times \dots \times a_1 \mid \varphi(x_1, \dots, x_{n-1})} \\
            &= \operatorname{ran}(\qty{\langle 0, x_{n-1}, \dots, x_1 \rangle \in \qty{0} \times a_{n-1} \times \dots \times a_1 \mid \theta(x_1, \dots, x_{n-1}, 0)}) \\
            &= \mathcal F_6(\mathcal F_\theta(a_1, \dots, a_{n-1}), \mathcal F_1(\mathcal F_3(a_1, a_1), \mathcal F_3(a_1, a_1)), a_1)
        \end{align*}
        \item If \( \psi(x_1, \dots, x_n) \) is a termed formula and
        \[ \varphi(x_1, \dots, x_{n+1}) = \psi(x_1, \dots, x_{n-1}, x_{n+1}/x_n) \]
        then \( \varphi \) is a termed formula.
        First, if \( n = 1 \), we have a termed formula \( \psi(x_1) \) and consider \( \psi(x_2/x_1) \).
        Then
        \begin{align*}
            \mathcal F_\varphi(a_1, a_2) &= \qty{\langle x_2, x_1 \rangle \in a_2 \times a_1 \mid \psi(x_2)} \\
            &= \qty{\langle x_2, x_1 \rangle \mid x_1 \in a_1 \wedge x_2 \in \mathcal F_\psi(a_2)} \\
            &= \mathcal F_\psi(a_2) \times a_1 \\
            &= \mathcal F_4(\mathcal F_\psi(a_2), a_1)
        \end{align*}
        If \( n > 1 \), we have
        \begin{align*}
            \mathcal F_\varphi(a_1, \dots, a_{n+1}) &= \qty{\langle x_{n+1}, \dots, x_1 \rangle \mid x_n \in a_n \wedge \langle x_{n+1}, x_{n-1}, \dots, x_1 \rangle \in \mathcal F_\psi(a_1, \dots, a_{n-1}, a_{n+1})} \\
            &= \mathcal F_8(\mathcal F_\psi(a_1, \dots, a_{n-1}, a_{n+1}), a_n)
        \end{align*}
        \item If \( \psi(x_1, x_2) \) is a termed formula, and
        \[ \varphi(x_1, \dots, x_n) \equiv \psi(x_{n-1}/x_1, x_n/x_2) \]
        then \( \varphi \) is a termed formula.
        This is trivial if \( n = 2 \), so we assume \( n > 2 \).
        Then
        \begin{align*}
            \mathcal F_\varphi(a_1, \dots, a_n) &= \qty{\langle x_n, \dots, x_1 \rangle \in a_n \times \dots \times a_1 \mid \langle x_n, x_{n-1} \rangle \in \mathcal F_\psi(a_{n-1}, a_n)} \\
            &= \mathcal F_7(\mathcal F_\psi(a_{n-1}, a_n), a_{n-2} \times \dots \times a_1)
        \end{align*}
    \end{itemize}

    \emph{Part (ii): propositional connectives.}
    \begin{itemize}
        \item If \( \varphi \) is a termed formula, then so is \( \neg\varphi \).
        \[ \mathcal F_{\neg\varphi}(a_1, \dots, a_n) = (a_n \times \dots \times a_1) \setminus \mathcal F_\varphi(a_1, \dots, a_n) \]
        \item If \( \varphi, \psi \) are termed formulas, then so is \( \varphi \vee \psi \).
        \[ \mathcal F_{\varphi \vee \psi}(a_1, \dots, a_n) = \mathcal F_\varphi(a_1, \dots, a_n) \cup \mathcal F_\psi(a_1, \dots, a_n) \]
        It is easy to see that unions can be formed using G\"odel functions.
        \item Conjunctions are similar to disjunctions.
        \[ \mathcal F_{\varphi \wedge \psi}(a_1, \dots, a_n) = \mathcal F_\varphi(a_1, \dots, a_n) \cap \mathcal F_\psi(a_1, \dots, a_n) \]
    \end{itemize}
\end{proof}
\begin{definition}
    A class \( C \) is \emph{closed under G\"odel functions} if whenever \( x, y \in C \), we have \( \mathcal F_i(x, y) \in C \) for \( i \in \qty{1, \dots, 10} \).
    Given a set \( b \), we let \( \mathrm{cl}(b) \) be the smallest set \( C \) containing \( b \) as a subset that is closed under G\"odel functions.
\end{definition}
For example, \( \mathrm{cl}(\varnothing) = \varnothing \), \( \qty{a, b} \in \mathrm{cl}(\qty{a, b}) \), and \( \mathrm{cl}(b) = \mathrm{cl}(\mathrm{cl}(b)) \).
\begin{definition}
    Let \( b \) be a set.
    Define \( \mathcal D^n(b) \) inductively by
    \[ \mathcal D^0(b) = b;\quad \mathcal D^{n+1}(b) = \qty{\mathcal F_i(x, y) \mid x, y \in \mathcal D^n(b), i \in \qty{1, \dots, 10}} \]
\end{definition}
One can easily check that \( \mathrm{cl}(b) = \bigcup_{n \in \omega} \mathcal D^n(b) \).
\begin{lemma}
    If \( M \) is a transitive class that is closed under G\"odel functions, then \( M \) satisfies \( \Delta_0 \)-separation.
\end{lemma}
\begin{proof}
    Let \( \varphi(x_1, \dots, x_n) \) be a \( \Delta_0 \)-formula, and let \( a, b_1, \dots, b_{i-1}, b_{i+1}, \dots, b_n \in M \).
    Let
    \[ Y = \qty{x_i \in a \mid \varphi(b_1, \dots, b_{i-1}, x_i, b_{i+1}, \dots, b_n)} \]
    We must show \( Y \in M \).
    Let \( \mathcal F_\varphi \) be the formula built from G\"odel's normal form theorem.
    Then for any \( c_1, \dots, c_n \in M \), we have
    \[ \mathcal F_\varphi(c_1, \dots, c_n) = \qty{\langle x_n, \dots, x_1 \rangle \in c_n \times \dots \times c_1 \mid \varphi(x_1, \dots, x_n)} \in M \]
    Hence, as \( \qty{b_j} = \mathcal F_1(b_j, b_j) \in M \), we obtain
    \[ \mathcal F_\varphi(\qty{b_1}, \dots, \qty{b_{i-1}}, a, \qty{b_{i+1}}, \dots, \qty{b_n}) \in M \]
    Then, we can show that \( Y \in M \) by taking the range \( \mathcal F_6 \) a total of \( n - i \) times and then taking the domain \( \mathcal F_5 \).
\end{proof}
\begin{theorem}
    For every transitive set \( M \), the collection of definable subsets is
    \[ \operatorname{Def}(M) = \mathrm{cl}(M \cup \qty{M}) \cap \mathcal P(M) \]
\end{theorem}
\begin{proof}
    We first prove the forward direction.
    Let \( \varphi \) be a formula.
    Then \( \varphi^M \) is \( \Delta_0 \), so there is a term \( \mathcal G \) built from the G\"odel functions \( \mathcal F_1, \dots, \mathcal F_{10} \) such that for \( a_1, \dots, a_n \in M \), we have
    \[ \qty{x \in M \mid (M, \in) \vDash \varphi(x, a_1, \dots, a_n)} = \qty{x \in M \mid \varphi^M(x, a_1, \dots, a_n)} = \mathcal G(M, a_1, \dots, a_n) \in \mathrm{cl}(M \cup \qty{M}) \]
    We now show the converse.
    We first claim that if \( \mathcal G \) is built from the G\"odel functions, then for any \( x, a_1, \dots, a_n \), the formulas
    \[ x = \mathcal G(a_1, \dots, a_n);\quad x \in \mathcal G(a_1, \dots, a_n) \]
    are \( \Delta_0 \).
    This can be proven inductively using the iterative construction of \( \mathrm{cl}(M \cup \qty{M}) \).
    For example, if \( X, Y \in \mathcal D^k(a_1, \dots, a_n) \), then \( x = \mathcal F_1(X, Y) \) is equivalent to the statement
    \[ (\forall z \in x.\, z = X \vee z = Y) \wedge (\exists w \in x.\, w = X) \wedge (\exists w \in x.\, w = Y) \]
    so the result holds for \( \mathcal F_1 \); very similar proofs show the result for both equality and membership for all other G\"odel functions.

    Let \( Z \in \mathrm{cl}(M \cup \qty{M}) \cap \mathcal P(M) \).
    Since \( Z \in \mathrm{cl}(M \cup \qty{M}) \), we can fix a term \( \mathcal G \) built from the \( \mathcal F_1, \dots, \mathcal F_{10} \) such that \( Z = \mathcal G(M, a_1, \dots, a_n) \).
    Let \( \varphi \) be a \( \Delta_0 \) formula such that \( x \in \mathcal G(M, a_1, \dots, a_n) \) if and only if \( \varphi(x, M, a_1, \dots, a_n) \).
    Then \( \mathcal G(M, a_1, \dots, a_n) = \qty{x \in M \mid \varphi(x, M, a_1, \dots, a_n)} \) as \( Z \subseteq M \).
    It therefore remains to prove that there is a formula \( \psi \) such that
    \[ \psi^M(x, a_1, \dots, a_n) \leftrightarrow \varphi(x, M, a_1, \dots, a_n) \]
    For example, we can define \( \psi \) from \( \varphi \) by the following replacements.
    \begin{enumerate}
        \item \( \exists v_i \in M \mapsto \exists v_i \);
        \item \( v_i \in M \mapsto v_i = v_i \);
        \item \( M = M \mapsto v_0 = v_0 \);
        \item \( M \in M, M \in v_i, M = v_i \mapsto v_0 \neq v_0 \).
    \end{enumerate}
    Finally, we obtain
    \[ Z = \mathcal G(M, a_1, \dots, a_n) = \qty{x \in M \mid \psi^M(x, a_1, \dots, a_n)} \in \operatorname{Def}(M) \]
\end{proof}
