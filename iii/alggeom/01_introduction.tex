\subsection{Course description}
The course consists of four parts.
\begin{enumerate}
    \item The theory of sheaves on topological spaces.
    \item The definitions of schemes and morphisms between them.
    \item Properties of schemes, such as the algebraic geometry analogues of compactness and other similar properties.
    \item Rapid introduction to the cohomology of sheaves.
\end{enumerate}

\subsection{Motivation from moduli theory}
In moduli theory, we study families of varieties instead of one at a time.
In the extreme, we study all varieties of a given `type' simultaneously.
For now, let
\[ \mathbb P^n = \mathbb P^n_{\mathbb C} = \faktor{\mathbb C^{n+1} \setminus \qty{\vb 0}}{\sim} \]
where \( \vb x \sim \lambda \vb x \) for nonzero \( \lambda, \vb x \).
A variety is the vanishing locus \( \mathbb V(S) \) of a set \( S \) of homogeneous polynomials in \( n + 1 \) variables.
These are subsets of \( \mathbb P^n \).
We present some examples of moduli.
\begin{example}
    The set of all lines in \( \mathbb P^2 \).
    A line in \( \mathbb P^2 \) is given by
    \[ \qty{aX_0 + bX_1 + cX_2 = 0} \]
    where not all of \( a, b, c \) are zero.
    The set of all lines in \( \mathbb P^2 \) are given by triples \( (a, b, c) \).
    Note that \( (\lambda a, \lambda b, \lambda c) \) gives the same line as \( (a, b, c) \), so really lines in \( \mathbb P^2 \) correspond exactly to points in \( \mathbb P^2 \).
    We call the set of all lines in \( \mathbb P^2 \) the dual space \( \mathbb P^2_{\text{dual}} \).
    This property is known as projective duality.

    The same logic applies to the set of degree \( d \) hypersurfaces in \( \mathbb P^n \); this space corresponds directly to
    \[ \mathbb P^{\binom{n+d}{d} - 1} \]
\end{example}
There is an unfortunate consequence of this method of study.
Some polynomials are of the form \( f = f_1^2 f_2 \) for some non-constant \( f_1 \), but then \( \mathbb V(f) = \mathbb V(f_1 f_2) \).
For example, \( (X_0 + X_1 + X_2)^2 \subseteq \mathbb P^2 \) is a line not a conic.
In particular, the limit of a sequence of conics may not be a conic.
The solution is to take the set
\[ U_d \subseteq \mathbb P^{\binom{n+d}{d} - 1} \]
in which \( [f] \in U_d \) has no repeated factors.
But then, \( U_d \) is `not compact', as some points have been removed.

We will now describe the impact of scheme theory on this situation.
Fix some \( \mathbb P^n \), and we will produce a `space'
\[ \operatorname{Var}(\mathbb P^n) \subsetneq \operatorname{Hilb}(\mathbb P^n) \]
The set \( \operatorname{Var}(\mathbb P^n) \) bijects onto the set of varieties of \( \mathbb P^n \).
The set \( \operatorname{Hilb}(\mathbb P^n) \) bijects onto the set of subschemes of \( \mathbb P^n \), and is compact in the Euclidean topology.
In particular, limits of varieties need not be varieties, but limits of schemes are always schemes.
One consequence is that in scheme theory,
\[ \mathbb V(X_0 + X_1 + X_2),\quad \mathbb V((X_0 + X_1 + X_2)^2) \]
are not isomorphic as schemes in \( \mathbb P^2 \).

\subsection{Motivation from the Weil conjectures}
Fix some homogeneous polynomial \( f \in \mathbb Z[X_0, \dots, X_{n+1}] \).
First, consider
\[ X = \mathbb V(f) \subseteq \mathbb P^{n+1}_{\mathbb C} \]
and assume that \( X \) is smooth.
As \( X \) is a compact topological space, we can find its Betti numbers \( b_0(X), \dots, b_{2n}(X) \), where
\[ b_i(X) = \operatorname{rank} H_i(X; \mathbb Z) \]
In particular, we can find its Euler characteristic.
\[ \chi(X) = \sum (-1)^i b_i(X) \]
Second, fix a prime \( p \) and let \( N_m \) be the number of solutions of \( f \) over \( \mathbb F_{p^m} \).
Define the Weil zeta function
\[ \zeta(X;t) = \exp(\sum_m \frac{N_m}{m} \cdot t^m) \]
One of the Weil conjectures states the following.
\begin{theorem}[Grothendieck]
    \begin{enumerate}
        \item \( \zeta(X; t) \) is a rational function in \( t \), so
        \[ \zeta(X; t) = \frac{P_X(t)}{Q_X(t)} \]
        \item Further, \( \zeta(X; t) \) can be written as a ratio of the form
        \[ \frac{P_0(t) P_2(t) \dots P_{2n}(t)}{P_1(t) P_3(t) \dots P_{2n-1}(t)} \]
        where
        \[ \deg P_i(t) = b_i(X) \]
    \end{enumerate}
\end{theorem}
The proof relies fundamentally on scheme theory: we need a space \( \mathcal X \) that interpolates between the algebraic closure \( \overline{\mathbb F_p} \) and \( \mathbb C \).
