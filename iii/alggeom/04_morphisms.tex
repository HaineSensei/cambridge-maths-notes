\subsection{Morphisms of ringed spaces}
Let \( (X, \mathcal O_X) \) be a scheme.
The stalks \( \mathcal O_{X,\mathfrak p} \) are local rings: they have a unique maximal ideal, which is the set of all non-unit elements.
Given \( f \in \mathcal O_X(U) \), we can meaningfully ask whether \( f \) vanishes at \( \mathfrak p \); that is, if the image of \( f \) in \( \mathcal O_{X, \mathfrak p} \) is contained in the maximal ideal.
\begin{definition}
    A morphism of ringed spaces \( f : (X, \mathcal O_X) \to (Y, \mathcal O_Y) \) consists of a continuous function \( f : X \to Y \) and a morphism \( f^\sharp : \mathcal O_Y \to f_\star \mathcal O_X \) between sheaves of rings on \( Y \).
\end{definition}
\( f^\sharp \) represents function composition with \( f^{-1} \), although the ring \( \mathcal O_X \) may not be a ring of functions.
It is possible to find a morphism \( (f, f^\sharp) \) between schemes \( (X, \mathcal O_X) \) and \( (Y, \mathcal O_Y) \) such that there exists \( q \in U \subseteq Y \) and \( h \in \mathcal O_Y(U) \) such that \( h \) vanishes at \( q \) but \( f^\sharp(h) \in \mathcal O_X(f^{-1}(U)) \) does not vanish at some \( p \in X \) with \( f(p) = q \).
This motivates the definition of a morphism of schemes.

Let \( f : X \to Y \) be a morphism of ringed spaces.
Given any point \( p \in X \), there is an induced map \( f^\sharp : \mathcal O_{Y,f(p)} \to \mathcal O_{X,p} \).
Explicitly, given \( s \in \mathcal O_{Y,f(p)} \), we can represent it by \( (s_U, U) \) where \( U \) is open, \( f(p) \in U \), and \( s_U \in \mathcal O_Y(U) \).
Now, \( f^\sharp(s_U) \in \mathcal O_X(f^{-1}(U)) \), so the pair \( (f^\sharp(s_U), f^{-1}(U)) \) defines an element of \( \mathcal O_{X,p} \).
\begin{definition}
    A ringed space \( (X, \mathcal O_X) \) is called a \emph{locally ringed space} if for all \( p \in X \), the stalk \( \mathcal O_{X,p} \) is is a local ring.
    A morphism of locally ringed spaces \( (f, f^\sharp) : (X, \mathcal O_X) \to (Y, \mathcal O_Y) \) is a morphism of ringed spaces such that if \( \mathfrak m_p \) denotes the maximal ideal in \( \mathcal O_{X,p} \), then \( f^\sharp(\mathfrak m_{f(p)}) \subseteq \mathfrak m_p \).
\end{definition}
This encapsulates the idea that functions vanishing on the codomain must also vanish on the domain after the inverse image, as the maximal ideal represents functions vanishing at the point.

\subsection{Morphisms of schemes}
Note that all schemes are locally ringed spaces.
\begin{definition}
    A \emph{morphism of schemes} \( X \to Y \) is a morphism of locally ringed spaces \( X \to Y \).
\end{definition}
\begin{theorem}
    There is a natural bijection
    \[ \qty{\text{morphisms of schemes } \Spec B \to \Spec A} \leftrightarrow \qty{\text{homomorphisms of rings } A \to B} \]
\end{theorem}
\begin{proof}
    First, recall that a section \( s \) of a sheaf \( \mathcal F \) on \( U \) is a coherent collection of elements of the stalks \( s(p) \in \mathcal F_p \) for all \( p \in U \).
    We will construct a map of schemes \( \Spec B \to \Spec A \) for every ring homomorphism \( A \to B \), and then show that every morphism of schemes arises in this way.

    Let \( \varphi : A \to B \) be a ring homomorphism.
    Let \( \varphi^{-1} : \Spec B \to \Spec A \) be the map of topological spaces; this is a continuous function.
    We now build
    \[ \varphi^\sharp : \mathcal O_{\Spec A} \to \varphi_\star^{-1} \mathcal O_{\Spec B} \]
    At the level of stalks, the map \( A_{\varphi^{-1}(\mathfrak p)} \to B_{\mathfrak p} \) is induced by \( \varphi \) by mapping \( \frac{a}{s} \) to \( \frac{\varphi(a)}{\varphi(s)} \).
    This is well-defined, as for \( s \notin \varphi^{-1}(\mathfrak p) \), then \( \varphi(s) \notin \mathfrak p \).
    Observe that this is automatically a local homomorphism.
    
    We must now show that this choice of maps on stalks extends to a map between sheaves.
    Given \( U \subseteq \Spec A \), we need to define
    \[ \varphi^\sharp : \mathcal O_{\Spec A}(U) \to \mathcal O_{\Spec B}((\varphi^{-1})^{-1}(U)) \]
    An element \( s \in \mathcal O_{\Spec A}(U) \) is a collection of assignments \( (\mathfrak p \mapsto s(\mathfrak p))_{\mathfrak p \in U} \) for \( \mathfrak p \in U \) and \( s(\mathfrak p) \in A_{\mathfrak p} \).
    We then define \( \varphi^\sharp \) by
    \[ (\mathfrak p \mapsto s(\mathfrak p))_{\mathfrak p \in U} \mapsto (\mathfrak q \mapsto \varphi_{\mathfrak q}(s(\varphi^{-1}(\mathfrak q))))_{\mathfrak q \in (\varphi^{-1})^{-1}(U)} \]
    One can check that the gluing conditions are satisfied.

    Conversely, suppose \( (f, f^\sharp) : \Spec B \to \Spec A \) is a morphism of schemes.
    Using the fact that we have a map of global sections \( \mathcal O_{\Spec A}(\Spec A) \to \mathcal O_{\Spec B}(\Spec B) \), we obtain a ring homomorphism \( g : A \to B \).
    We must check that \( g^{-1} : \Spec B \to \Spec A \) gives the correct map \( f \) on topological spaces, and that the construction above yields the correct map \( f^\sharp \) on sheaves.
    The maps on stalks are compatible with restriction, so the following diagram commutes for all \( \mathfrak p \in \Spec B \).
    % https://q.uiver.app/#q=WzAsNCxbMCwwLCJcXEdhbW1hKFxcU3BlYyBBLCBcXG1hdGhjYWwgT197XFxTcGVjIEF9KSJdLFsxLDAsIlxcR2FtbWEoXFxTcGVjIEIsIFxcbWF0aGNhbCBPX3tcXFNwZWMgQn0pIl0sWzEsMSwiXFxtYXRoY2FsIE9fe1xcU3BlYyBCLCBcXG1hdGhmcmFrIHB9Il0sWzAsMSwiXFxtYXRoY2FsIE9fe1xcU3BlYyBBLCBmKFxcbWF0aGZyYWsgcCl9Il0sWzAsMV0sWzEsMl0sWzAsM10sWzMsMl1d
\[\begin{tikzcd}
	{\Gamma(\Spec A, \mathcal O_{\Spec A})} & {\Gamma(\Spec B, \mathcal O_{\Spec B})} \\
	{\mathcal O_{\Spec A, f(\mathfrak p)}} & {\mathcal O_{\Spec B, \mathfrak p}}
	\arrow[from=1-1, to=1-2]
	\arrow[from=1-2, to=2-2]
	\arrow[from=1-1, to=2-1]
	\arrow[from=2-1, to=2-2]
\end{tikzcd}\]
    Equivalently, the following diagram commutes for all \( \mathfrak p \in \Spec B \).
    % https://q.uiver.app/#q=WzAsNCxbMCwwLCJBIl0sWzEsMCwiQiJdLFsxLDEsIkJfe1xcbWF0aGZyYWsgcH0iXSxbMCwxLCJBX3tmKFxcbWF0aGZyYWsgcCl9Il0sWzAsMV0sWzEsMl0sWzAsM10sWzMsMl1d
\[\begin{tikzcd}
	A & B \\
	{A_{f(\mathfrak p)}} & {B_{\mathfrak p}}
	\arrow[from=1-1, to=1-2]
	\arrow[from=1-2, to=2-2]
	\arrow[from=1-1, to=2-1]
	\arrow[from=2-1, to=2-2]
\end{tikzcd}\]
    Since the moprhism is local, \( (f^\sharp)^{-1}(\mathfrak p B_{\mathfrak p}) = f(\mathfrak p) A_{f(\mathfrak p)} \).
    As the above diagram commutes, \( g^{-1} = f \) as maps of topological spaces, and the maps of structure sheaves agree at the level of stalks by construction so they must agree everywhere.
\end{proof}

\subsection{Immersions}
\begin{definition}
    Let \( X, Y \) be schemes.
    A morphism of schemes \( f : X \to Y \) is an \emph{open immersion} if \( f \) induces an isomorphism of \( X \) onto an open subscheme \( \qty(U, \eval{\mathcal O_Y}_U) \) of \( Y \).
    A moprhism \( f : X \to Y \) is a \emph{closed immersion} if \( f \) is a homeomorphism onto a closed subset of \( Y \), and \( g^\sharp : \mathcal O_Y \to g_\star \mathcal O_X \) is surjective.
\end{definition}
\begin{example}
    Let \( k[t] \to \faktor{k[t]}{(t^2)} \).
    The induced map \( \Spec \faktor{k[t]}{(t^2)} \to \Spec k[t] \) is a closed immersion.
\end{example}
