\subsection{Definitions}
\begin{example}
    Let \( \mathbb CP^n \) be the variety \( \mathbb C^{n+1} \setminus \qty{0} \) modulo scaling by \( \mathbb C \).
    We have a structure sheaf \( \mathcal O_{\mathbb CP^n} \), where if \( U \subseteq \mathbb CP^n \) is Zariski open, we define
    \[ \mathcal O_{\mathbb CP^n}(U) = \qty{\frac{P(\vb x)}{Q(\vb x)} \midd P, Q \text{ homogeneous of the same degree, and the ratio is regular at all } p \in U} \]
    For any integer \( d \), we can consider a sheaf \( \mathcal O_{\mathbb CP^n}(d) \) given by
    \[ \mathcal O_{\mathbb CP^n}(d)(U) = \qty{\frac{P(\vb x)}{Q(\vb x)} \midd P, Q \text{ homogeneous, } \deg P - \deg Q = d\text{, and regular at all } p \in U} \]
    This is a sheaf of groups, but not a sheaf of rings as it is not closed under multiplication for \( d \neq 0 \).
    Note that \( \mathcal O_{\mathbb CP^n}(d)(U) \) is a module over \( \mathcal O_{\mathbb CP^n}(U) \), and the multiplication commutes with restriction.
\end{example}
\begin{example}
    Let \( A \) be a ring, and let \( M \) be an \( A \)-module.
    We define the sheaf \( \mathcal F_M = M^{\mathrm{sh}} \) on \( \Spec A \) as follows.
    If \( U \subseteq \Spec A \) is a distinguished open \( U = U_f \), then we set
    \[ \mathcal F_M(U) = M_f \]
    which is the module \( M \) localised at \( f \).
    This defines a sheaf on a base, and hence extends to a unique sheaf on \( \Spec A \).
\end{example}
\begin{definition}
    Let \( (X, \mathcal O_X) \) be a ringed space.
    A \emph{sheaf of \( \mathcal O_X \)-modules} on \( X \) is a sheaf \( \mathcal F \) of abelian groups together with a multiplication \( \mathcal F(U) \times \mathcal O_X(U) \to \mathcal F(U) \) that makes \( \mathcal F(U) \) into an \( \mathcal O_X(U) \)-module, that is compatible with restriction.
    % https://q.uiver.app/#q=WzAsNCxbMCwwLCJcXG1hdGhjYWwgRihWKSBcXHRpbWVzIFxcbWF0aGNhbCBPX1goVikiXSxbMSwwLCJcXG1hdGhjYWwgRihWKSJdLFsxLDEsIlxcbWF0aGNhbCBGKFUpIl0sWzAsMSwiXFxtYXRoY2FsIEYoVSkgXFx0aW1lcyBcXG1hdGhjYWwgT19YKFUpIl0sWzAsMV0sWzEsMl0sWzAsM10sWzMsMl1d
\[\begin{tikzcd}
	{\mathcal F(V) \times \mathcal O_X(V)} & {\mathcal F(V)} \\
	{\mathcal F(U) \times \mathcal O_X(U)} & {\mathcal F(U)}
	\arrow[from=1-1, to=1-2]
	\arrow[from=1-2, to=2-2]
	\arrow[from=1-1, to=2-1]
	\arrow[from=2-1, to=2-2]
\end{tikzcd}\]
\end{definition}
Similarly, we can define a sheaf of \( \mathcal O_X \)-algebras.
A morphism between sheaves of modules \( \varphi : \mathcal F \to \mathcal G \) on \( X \) is a homomorphism of sheaves of abelian groups that is compatible with multiplication.

Given morphisms of sheaves of modules on \( X \), we can locally take kernels, cokernels, images, direct sums, tensor products, hom functors, and all of these extend to sheaves of modules.
In the case of cokernels, images, and tensor products, we require a sheafification step.
For example, the presheaf tensor product \( \mathcal F \otimes_{\mathcal O_X} \mathcal G \) associated to an open set \( U \subseteq X \) is given by \( \mathcal F(U) \otimes_{\mathcal O_X(U)} \mathcal G(U) \); the sheaf tensor product is given by sheafification.

Given a morphism of ringed spaces or schemes \( f : X \to Y \), the pushforward of an \( \mathcal O_X \)-module \( \mathcal F \) is the sheaf of abelian groups \( f_\star \mathcal F \).
As a morphism of ringed spaces, we also have a map \( f^\sharp : \mathcal O_Y \to f_\star \mathcal O_X \), giving \( f_\star \mathcal F \) an \( \mathcal O_Y \)-module structure.
Given an open set \( U \subseteq Y \), \( a \in \mathcal O_Y(U) \), and \( m \in f_\star \mathcal F(U) = \mathcal F(f^{-1}(U)) \), we define \( a \cdot m = f^\sharp(a) \cdot m \), where \( f^\sharp(a) \in \mathcal O_X(f^{-1}(U)) \).

Conversely, if \( \mathcal G \) is a sheaf of \( \mathcal O_Y \)-modules, we define
\[ f^\star \mathcal G = f^{-1} \mathcal G \otimes_{f^{-1} \mathcal O_Y} \mathcal O_X \]
where the \( f^{-1} \mathcal O_Y \)-module structure on \( \mathcal O_X \) is defined via the adjoint to \( f^\sharp \).

\subsection{Quasi-coherence}
\begin{definition}
    A \emph{quasi-coherent sheaf} \( \mathcal F \) on a scheme \( X \) is a sheaf of \( \mathcal O_X \)-modules such that there exists a cover of \( X \) by affines \( \qty{U_i} \) such that \( \eval{\mathcal F}_{U_i} \) is the sheaf associated to a module over the ring \( \mathcal O_X(U_i) \).
    If these modules can be taken to be finitely generated, we say \( \mathcal F \) is \emph{coherent}.
\end{definition}
\begin{example}
    \begin{enumerate}
        \item On any scheme \( X \), \( \mathcal O_X \) is quasi-coherent (and, in fact, coherent).
        \item \( \bigoplus_I \mathcal O_X \) is quasi-coherent, but not coherent if \( I \) is infinite.
        \item If \( i : X \to Y \) is a closed immersion, then \( i_\star \mathcal O_X \) is a quasi-coherent \( \mathcal O_Y \)-module.
        Let \( U \subseteq Y \) be an affine open set, so \( U = \Spec A \).
        Then \( X \cap U \to U \) gives an ideal \( I \subseteq A \) which is the kernel of the surjection \( \mathcal O_Y(U) \to \mathcal O_X(X \cap U) \).
        On \( U \), \( \eval{i_\star \mathcal O_X}_U \) is the sheaf associated to the \( A \)-module \( \faktor{A}{I} \).
    \end{enumerate}
\end{example}
\begin{proposition}
    An \( \mathcal O_X \)-module \( \mathcal F \) is quasi-coherent if and only if for any affine open \( U \subseteq X \) with \( U = \Spec A \), \( \eval{\mathcal F}_U \) is the sheaf associated to a module over \( A \).
\end{proposition}
We first prove the following key technical lemma.
\begin{lemma}
    Let \( X = \Spec A \), \( f \in A \), and \( \mathcal F \) a quasi-coherent \( \mathcal O_X \)-module.
    Let \( s \in \Gamma(X, \mathcal F) \).
    Then
    \begin{enumerate}
        \item If \( s \) restricts to 0 on \( U_f \), then \( f^n s = 0 \) for some \( n \geq 1 \).
        \item If \( t \in \mathcal F(U_f) \), then \( f^n t \) is the restriction of a global section of \( \mathcal F \) over \( X \) for some \( n \geq 1 \).
    \end{enumerate}
\end{lemma}
\begin{proof}
    There exists some cover of \( X \) by schemes of the form \( \Spec B = V \), such that \( \eval{\mathcal F}_V = M^{\mathrm{sh}} \) for \( M \) a \( B \)-module.
    We can cover each such \( V \) by distinguished affines of the form \( U_g \) for some \( g \in A \).
    Then \( \eval{\mathcal F}_{U_g} = (M \otimes_B A_g)^{\mathrm{sh}} \), as \( \eval{F}_V \) is quasi-coherent.
    But recall that \( \Spec A \) is quasi-compact: every open cover has a finite subcover.
    So finitely many \( U_{g_i} \) will suffice to cover \( A \) by open sets such that \( \mathcal F \) restricts to \( M_i^{\mathrm{sh}} \) on \( U_{g_i} \).
    Then the lemma follows from formal properties of localisation.
\end{proof}
We now prove the main proposition.
\begin{proof}
    Given \( U \subseteq X \), observe that \( \eval{\mathcal F}_U \) is also quasi-coherent.
    We can thus reduce the statement to the case where \( X = \Spec A \).
    Now we take \( M = \Gamma(X, \mathcal F) \), and let \( M^{\mathrm{sh}} \) be the associated sheaf.
    We claim that \( M^{\mathrm{sh}} \cong \mathcal F \).
    Let \( \alpha : M^{\mathrm{sh}} \to \mathcal F \) be the map given by restriction (for example via stalks).
    Then \( \alpha \) is an isomorphism at the level of stalks by the above lemma, so is an isomorphism globally.
\end{proof}
In particular, the quasi-coherent sheaves of modules over \( \Spec A \) are precisely the modules over \( A \).
The coherent sheaves of modules over \( \Spec A \) are precisely the finitely-generated modules over \( A \).
\begin{proposition}
    \begin{enumerate}
        \item Images, kernels, and cokernels of maps of (quasi-)coherent sheaves remain (quasi-)coherent.
        \item If \( f : X \to S \) is a morphism of schemes and \( \mathcal F \) is a (quasi-)coherent sheaf of modules on \( S \), then \( f^\star \mathcal F \) is also (quasi-)coherent.
        \item If \( f : X \to S \) is a morphism of schemes and \( \mathcal G \) is a quasi-coherent sheaf on \( X \), then \( f_\star \mathcal G \) is also quasi-coherent.
    \end{enumerate}
\end{proposition}
The proofs are omitted and non-examinable.
Note that (iii) need not hold for coherent sheaves: let \( f : \mathbb A^1_k \to \Spec k \) be the obvious map, and consider \( f_\star \mathcal O_{\mathbb A^1_k} \).
This is a quasi-coherent sheaf on \( \Spec k \), so is a \( k \)-vector space, which is \( k[t] \).
As a module, this is not finitely generated.
Observe that if \( f : \mathbb P^1_k \to \Spec k \), then \( f_\star \mathcal O_{\mathbb P^1_k} \) is the sheaf associated to \( k \).
In general, if \( \mathcal G \) is a coherent sheaf on \( X \) and \( f : X \to S \) is proper, then \( f_\star \mathcal G \) is coherent.

Let \( A \) be a graded ring, with the usual assumptions on its generators.
To build \( \Proj A \), we consider the cover by \( \Spec \qty(A\qty[\frac{1}{f}]_0) \) for \( f \in A_1 \).
We can produce a similar construction for modules.

Let \( M \) be a \emph{graded \( A \)-module}, that is,
\[ M = \bigoplus_{d \in \mathbb Z} M_d \]
where each \( M_d \) is an abelian group, \( M \) is an \( A \)-module, and \( A_i M_j \subseteq M_{i+j} \).
Consider the sheaf determined by the association
\[ \Proj A \supseteq U_f \mapsto \qty(M\qty[\frac{1}{f}])_0 \]
To each \( U_f = \mathbb V(f)^c \), we associate the degree zero elements of the localisation of \( M \) at \( f \).
This gives a quasi-coherent sheaf on \( \Proj A \) by identical arguments as in the \( \Proj \) construction.
\begin{definition}
    Let \( X \) be a scheme and \( \mathcal F \) be a quasi-coherent \( \mathcal O_X \)-module.
    We say that \( \mathcal F \) is
    \begin{enumerate}
        \item \emph{free}, if \( \mathcal F \simeq \mathcal O_X^{\oplus I} \) for some set \( I \);
        \item an \emph{(algebraic) vector bundle} or \emph{locally free} if there exists an open cover \( \qty{U_i} \) such that \( \eval{\mathcal F}_{U_i} \) is free;
        \item a \emph{line bundle} or an \emph{invertible sheaf} if it is a vector bundle that is locally isomorphic to \( \mathcal O_X \).
    \end{enumerate}
\end{definition}
Note that such sheaves are coherent if and only if the index sets \( I \) can be taken to be finite.

\subsection{Coherent sheaves on projective schemes}
\begin{definition}
    Let \( A \) be a graded ring, and let \( M \) be a graded \( A \)-module.
    For \( d \in \mathbb Z \), we define \( M(d) \), called \( M \) \emph{twisted by} \( d \), to be the module such that
    \[ (M(d))_k = M_{k+d} \]
\end{definition}
\begin{definition}
    Let \( X = \Proj A \) where \( A \) is a graded ring and let \( d \in \mathbb Z \).
    The sheaf \( \mathcal O_X(d) \) is defined to be the sheaf associated to the graded module \( A(d) \).
    In particular, \( \mathcal O_X(1) \) is called the \emph{twisting sheaf}.
\end{definition}
\begin{remark}
    \( \mathcal O_X(d) = \mathcal O_X(1)^{\otimes d} \).
    Note that the tensor product of graded modules is additive in the grading.
\end{remark}
\begin{example}
    Consider \( \Proj k[x_0, \dots, x_n] = \mathbb P^n_k \).
    The global sections of \( \mathcal O_{\mathbb P^n_k}(d) \) are homogeneous degree \( d \) polynomials in the \( x_i \).
    In particular, if \( d < 0 \), then \( \Gamma(\mathbb P^n_k, \mathcal O_{\mathbb P^n_k}(d)) = 0 \).
\end{example}
\begin{definition}
    An \( \mathcal O_X \)-module \( \mathcal F \) is called \emph{globally generated} or \emph{generated by global sections} if it is a quotient of \( \mathcal O_X^{\oplus r} \) for some \( r \); that is, is there is a surjective map of coherent sheaves \( \mathcal O_X^{\oplus r} \to \mathcal F \).
    Equivalently, there exist elements \( s_1, \dots, s_r \in \Gamma(X, \mathcal F) \) such that \( \qty{s_i} \) generate the stalks \( \mathcal F_p \) over \( \mathcal O_{X,p} \) for all \( p \in X \).
\end{definition}
\begin{theorem}
    Let \( i : X \rightarrowtail \mathbb P^n_R \) be a closed immersion.
    Let \( \mathcal O_X(1) \) be the restriction of \( \mathcal O_{\mathbb P^n_R}(1) \), so \( \mathcal O_X(1) = i^\star \mathcal O_{\mathbb P^n_R}(1) \).
    Let \( \mathcal F \) be a coherent sheaf on \( X \).
    Then there exists an integer \( d_0 \) such that for all \( d \geq d_0 \), the sheaf
    \[ \mathcal F(d) = \mathcal F \otimes_{\mathcal O_X} \mathcal O_X(d) \]
    is globally generated.
\end{theorem}
As a consequence, every \( \mathcal F \) as above is a quotient of a vector bundle.
\begin{proof}
    By formal properties, it is equivalent to show the statement for \( i_\star \mathcal F \); that is, \( i_\star \mathcal F(d) \) is globally generated on \( \mathbb P^n_R \).
\end{proof}
