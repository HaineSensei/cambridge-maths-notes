We will now use the notation \( \eval{f}_U \) for \( \res^V_U f \).
% Idea: Spec A has a sheaf O_{Spec A} such that value at U_f is A_f; globalise this to get the notion of a scheme.

\subsection{Localisation}
\begin{definition}
    Let \( A \) be a ring and \( S \subseteq A \) be a multiplicatively closed set.
    The \emph{localisation} of \( A \) at \( S \) is
    \[ S^{-1}A = \faktor{\qty{(a, s) \mid a \in A, s \in S}}{\sim} \]
    where
    \[ (a, s) \sim (a', s') \iff \exists s'' \in S,\, s''(as' - a's) = 0 \in A \]
\end{definition}
Examples of multiplicatively closed sets include the set of powers of a fixed element, or the complement of a prime ideal.
The pair \( (a, s) \) represents \( \frac{a}{s} \).
The extra \( s'' \) term represents a unit in this new ring, which may be needed in rings that are not integral domains.
\begin{remark}
    The natural map \( A \to S^{-1} A \) need not be injective, for example, if \( S \) contains a zero divisor.
\end{remark}
% We aim to define a sheaf \( \mathcal O_{\Spec A} \) on the topological space \( \Spec A \), such that the stalk at a prime \( \mathfrak p \) is \( (A \setminus \mathfrak p)^{-1} A \), and if \( U_f \) is a distinguished open, then \( \mathcal O_{\Spec A}(U_f) = A_f \).

\subsection{Sheaves on a base}
\begin{definition}
    Let \( X \) be a topological space and \( \mathcal B \) be a basis for the topology.
    A \emph{sheaf on the base \( \mathcal B \)} consists of assignments \( B_i \mapsto F(B_i) \) of abelian groups, with restriction maps \( \res_{B_j}^{B_i} : F(B_i) \to F(B_j) \) whenever \( B_j \subseteq B_i \) such that,
    \begin{enumerate}
        \item \( \res^{B_i}_{B_i} = \id_{B_i} \);
        \item \( \res^{B_j}_{B_k} \circ \res^{B_i}_{B_j} = \res^{B_i}_{B_k} \)
    \end{enumerate}
    with the additional axioms that
    \begin{enumerate}
        \item if \( B = \bigcup B_i \) with \( B, B_i \in \mathcal B \) and \( f, g \in F(B) \) such that \( \eval{f}_{B_i} = \eval{g}_{B_i} \) for all \( i \), then \( f = g \);
        \item if \( B = \bigcup B_i \) as above, with \( f_i \in F(B_i) \) such that for all \( i, j \) and \( B' \subseteq B_i \cap B_j \) with \( B' \in \mathcal B \), \( \eval{f_i}_{B'} = \eval{f_j}_{B'} \), then there exists \( f \in F(B) \) with \( \eval{f}_{B_i} = f_i \).
    \end{enumerate}
\end{definition}
This is very similar to the definition of a sheaf, but only specified on the basis.
\begin{proposition}
    Let \( F \) be a sheaf on a base \( \mathcal B \) of \( X \).
    This determines a sheaf \( \mathcal F \) on \( X \) such that \( \mathcal F(B) = F(B) \) for all \( B \in \mathcal B \), agreeing with restriction maps.
    Moreover, \( \mathcal F \) is unique up to unique isomorphism.
\end{proposition}
\begin{proof}
    We first define the stalks using \( F \):
    \[ \mathcal F_p = \faktor{\qty{(s_B, B) \mid p \in B \in \mathcal B, s_B \in F(B)}}{\sim} \]
    We then use a sheafification idea to define \( \mathcal F(U) \).
    The elements are the dependent functions \( f \in \prod_{p \in U} \mathcal F_p \) such that for each \( p \in U \), there exists a basic open set \( B \) containing \( p \) and a section \( s \in F(B) \) such that \( s_q = f_q \) in \( \mathcal F_q \) for all \( q \in B \).
    This is then clearly a sheaf.
    The natural maps \( F(B) \to \mathcal F(B) \) are isomorphisms by the sheaf axioms.
\end{proof}

\subsection{The structure sheaf}
Recall that the distinguished opens \( U_f, U_g \) coincide if and only if \( f, g \) are powers of some \( h \in A \).
Also, if \( U_f = U_g \) then \( A_f \cong A_g \).
Therefore, the assignment \( U_f \mapsto A_f \) is well-defined.
\begin{proposition}
    The assignment \( U_f \mapsto A_f \) defines a sheaf of rings on the base \( \qty{U_f} \) of the topological space \( \Spec A \).
\end{proposition}
\begin{remark}
    If \( \qty{U_{f_i}}_{i \in I} \) covers \( \Spec A \), there exists a finite subcover.
    Indeed, since the \( U_{f_i} \) cover \( \Spec A \), there is no prime ideal \( \mathfrak p \subseteq A \) containing all \( (f_i)_{i \in I} \).
    Equivalently, \( \sum_{i \in I} (f_i) = (1) \).
    In particular, \( 1 = \sum_{i \in J} a_i f_i \) for \( J \subseteq I \) finite.
    So \( \sum_{i \in J} (f_i) = (1) \), and thus \( \qty{U_{f_i}}_{i \in J} \) covers \( \Spec A \).
    We say that \( \Spec A \) is \emph{quasi-compact}; traditionally the word `compact' is reserved for Hausdorff spaces in the context of algebraic geometry.
\end{remark}
\begin{proof}
    We will check the axioms for the basic open set \( B = \Spec A \); the general case follows by applying this result to a localisation.
    Suppose \( \Spec A = \bigcup_{i = 1}^n U_{f_i} \); this union is finite by the previous remark.
    Let \( s \in A \) be such that \( \eval{s}_{U_i} = 0 \) for all \( i \).
    By the definition of localisation, as the set \( \qty{U_{f_i}} \) is finite there exists \( m \) such that \( f_i^m s = 0 \) for all \( i \).
    But note that \( (1) = (f_i^m)_{i = 1}^n \) for any \( m > 0 \) because the \( \qty{U_{f_i}}_{i=1}^n \) cover \( \Spec A \).
    Thus \( \qty{U_{f_i^m}}_{i=1}^n \) cover \( \Spec A \).
    \[ 1 = \sum r_i f_i^m \implies s = \sum r_i f_i^m s = 0 \]
    Now suppose \( \Spec A = \bigcup_{i \in I} U_{f_i} \), and \( s_i \in A_{f_i} \) are elements that agree in \( A_{f_i f_j} \).
    We need to build an element in \( A \) with these restrictions.

    First, suppose \( I \) is finite.
    On \( U_{f_i} \), we have chosen \( \frac{a_i}{f_i^{\ell_i}} \in A_{f_i} \); we write \( g_i = f_i^{\ell_i} \), noting that \( U_{f_i} = U_{g_i} \).
    On the overlaps, by hypothesis we have
    \[ (g_i g_j)^{m_{ij}} (a_i g_j - a_j g_i) = 0 \]
    Rewriting this using the fact that \( U_f = U_{f^k} \) for all \( k > 0 \), and assuming \( m = m_{ij} \) by taking the largest, we obtain
    \[ b_i = a_i g_i^m;\quad h_i = g_i^{m+1} \]
    so on each \( U_{h_i} \) we have chosen an element \( \frac{b_i}{h_i} \).
    Now, as the \( U_{h_i} = U_{f_i} \) cover \( \Spec A \), we have \( 1 = \sum r_i h_i \) for some \( r_i \in A \).
    We can thus construct \( r = \sum r_i b_i \) with the \( r_i \) as above.
    This construction then has the correct restrictions to \( \frac{b_i}{h_i} \) in \( U_{h_i} \).

    When \( I \) is infinite, choose \( (f_i)_{i=1}^n \) such that the \( U_{f_i} \) for \( i \in \qty{1, \dots, n} \) form a cover, and use the finite case to build \( r \in A \).
    This has the correct restrictions to the \( U_{f_i} \) for \( i \in \qty{1, \dots, n} \).
    Given \( (f_1, \dots, f_n, f_\alpha) = A \), the same construction gives a new \( r' \in A \), but then by the first sheaf axiom, \( r = r' \).
\end{proof}
\begin{definition}
    The \emph{structure sheaf} on \( \Spec A \) is the sheaf \( \mathcal O_{\Spec A} \) associated to the sheaf on the base of distinguished opens mapping \( U_f \) to \( A_f \).
\end{definition}
\begin{remark}
    The stalk \( \mathcal O_{\Spec A, \mathfrak p} \) is equal to \( A_{\mathfrak p} \).
\end{remark}

\subsection{Definitions and examples}
\begin{definition}
    A \emph{ringed space} \( (X, \mathcal O_X) \) is a topological space \( X \) with a sheaf of rings \( \mathcal O_X \).
    An isomorphism of ringed spaces \( (X, \mathcal O_X) \to (Y, \mathcal O_Y) \) is a homeomorphism \( \pi : X \to Y \) and an isomorphism \( \mathcal O_Y \to \pi_\star \mathcal O_X \) of sheaves on \( Y \).
\end{definition}
Note that for \( U \subseteq X \) open, \( U \) is naturally a ringed space with \( \mathcal O_U(V) = \mathcal O_X(V) \).
\begin{definition}
    An \emph{affine scheme} is a ringed space \( (X, \mathcal O_X) \) that is isomorphic to \( (\Spec A, \mathcal O_{\Spec A}) \).
\end{definition}
\begin{definition}
    A \emph{scheme} is a ringed space \( (X, \mathcal O_X) \) where every point \( p \in X \) has a neighbourhood \( U_p \) such that the ringed space \( (U_p, \mathcal O_{U_p}) \) is isomorphic to some affine scheme.
\end{definition}
\begin{example}
    \begin{enumerate}
        \item The spectrum \( \Spec A \) of each ring \( A \) is a scheme.
        \item Let \( X \) be a scheme and \( U \subseteq X \) an open set.
        Take \( \mathcal O_U = \eval{\mathcal O_X}_U \) to be the structure sheaf of \( U \).
        Thus \( U \) is a scheme: as the \( U_f \) form a basis, there is always a sufficiently small neighbourhood inside \( U \) isomorphic to an affine scheme.
        \item Let \( X = \Spec \mathbb C[x, y] \) and \( U = \qty{(x, y)}^c \).
        Then the scheme \( U \) is not an affine scheme.
        % exercise: why?
    \end{enumerate}
\end{example}
