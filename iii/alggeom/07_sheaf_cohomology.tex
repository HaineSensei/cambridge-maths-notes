\subsection{Introduction and properties}
We have previously seen that if \( X = \mathbb A^2 \setminus \qty{(0, 0)} \), then \( \mathcal O_X(X) \cong \mathcal O_{\mathbb A^2}(\mathbb A^2) \cong k[x, y] \).
Given a topological space \( X \) and a sheaf \( \mathcal F \) of abelian groups, there is a series of \emph{cohomology} groups \( H^i(X, \mathcal F) \) for \( i \in \mathbb N \).
The definition will be omitted.
These groups have the following features.
\begin{enumerate}
    \item The group \( H^0(X, \mathcal F) \) is precisely \( \Gamma(X, \mathcal F) \).
    \item If \( f : Y \to X \) is continuous, there is an induced map \( f^\star : H^i(X, \mathcal F) \to H^i(Y, f^{-1} \mathcal F) \).
    \item Given a short exact sequence of sheaves
    % https://q.uiver.app/#q=WzAsNSxbMCwwLCIwIl0sWzEsMCwiXFxtYXRoY2FsIEYiXSxbMiwwLCJcXG1hdGhjYWwgRiciXSxbMywwLCJcXG1hdGhjYWwgRicnIl0sWzQsMCwiMCJdLFswLDFdLFsxLDJdLFsyLDNdLFszLDRdXQ==
\[\begin{tikzcd}
	0 & {\mathcal F} & {\mathcal F'} & {\mathcal F''} & 0
	\arrow[from=1-1, to=1-2]
	\arrow[from=1-2, to=1-3]
	\arrow[from=1-3, to=1-4]
	\arrow[from=1-4, to=1-5]
\end{tikzcd}\]
    we obtain a long exact sequence
    % https://q.uiver.app/#q=WzAsOSxbMCwwLCIwIl0sWzEsMCwiSF4wKFgsIFxcbWF0aGNhbCBGKSJdLFsyLDAsIkheMChYLCBcXG1hdGhjYWwgRicpIl0sWzMsMCwiSF4wKFgsIFxcbWF0aGNhbCBGJycpIl0sWzEsMSwiSF4xKFgsIFxcbWF0aGNhbCBGKSJdLFsyLDEsIkheMShYLCBcXG1hdGhjYWwgRicpIl0sWzMsMSwiSF4xKFgsIFxcbWF0aGNhbCBGJycpIl0sWzEsMiwiSF4yKFgsIFxcbWF0aGNhbCBGKSJdLFsyLDIsIlxcY2RvdHMiXSxbMCwxXSxbMSwyXSxbMiwzXSxbMyw0XSxbNCw1XSxbNSw2XSxbNiw3XSxbNyw4XV0=
\[\begin{tikzcd}
	0 & {H^0(X, \mathcal F)} & {H^0(X, \mathcal F')} & {H^0(X, \mathcal F'')} \\
	& {H^1(X, \mathcal F)} & {H^1(X, \mathcal F')} & {H^1(X, \mathcal F'')} \\
	& {H^2(X, \mathcal F)} & \cdots
	\arrow[from=1-1, to=1-2]
	\arrow[from=1-2, to=1-3]
	\arrow[from=1-3, to=1-4]
	\arrow[from=1-4, to=2-2]
	\arrow[from=2-2, to=2-3]
	\arrow[from=2-3, to=2-4]
	\arrow[from=2-4, to=3-2]
	\arrow[from=3-2, to=3-3]
\end{tikzcd}\]
	\item If \( X \) is an affine scheme and \( \mathcal F \) is a quasi-coherent sheaf, then \( H^i(X, \mathcal F) = 0 \) for all \( i > 0 \).
	\item Cohomology commutes with taking direct sums of sheaves.
	\item If \( X \) is a Noetherian separated scheme, then \( H^i(X, \mathcal F) \) can be computed from the sections of \( \mathcal F \) on an open affine cover \( \qty{U_i} \) and from the data of the restrictions to \( \mathcal F(U_i \cap U_j), \mathcal F(U_i \cap U_j \cap U_k) \) and so on.
	This can be done by considering \emph{\v{C}ech cohomology}.
\end{enumerate}

\subsection{\v{C}ech cohomology}
Let \( X \) be a topological space, and let \( \mathcal F \) be a sheaf on \( X \).
Let \( \mathcal U = \qty{U_i}_{i \in I} \) be a fixed open cover of \( X \), indexed by a well-ordered set \( I \).
In this course, we will take \( I = \qty{1, \dots, N} \), and write \( U_{i_0 \dots i_p} = U_{i_0} \cap \dots \cap U_{i_p} \).
\v{C}ech cohomology attaches data to the triple \( (X, \mathcal F, \mathcal U) \).
The group of \emph{\v{C}ech \( p \)-cochains} is
\[ C^p(\mathcal U, \mathcal F) = \prod_{i_0 < \dots < i_p} \mathcal F(U_{i_0 \dots i_p}) \]
There is a \emph{differential}
\[ d : C^p(\mathcal U, \mathcal F) \to C^{p+1}(\mathcal U, \mathcal F) \]
where the \( i_0, \dots, i_{p+1} \) component of \( d\alpha \) is given by
\[ (d\alpha)_{i_0 \dots i_{p+1}} = \sum_{k=0}^{p+1} (-1)^k \eval{\alpha_{i_0\dots \hat i_k \dots i_{p+1}}}_{U_{i_0 \dots i_{p+1}}} \]
where \( \hat i_k \) denotes that the element \( i_k \) of the sequence is omitted.
One can easily show that \( d^2 : C^p \to C^{p+2} \) is the zero map.
Thus, \( \qty{C^p(\mathcal U, \mathcal F)}_p \) has the structure of a \emph{cochain complex}.
\begin{definition}
	The \emph{\( i \)th \v{C}ech cohomology} of \( (X, \mathcal F, \mathcal U) \) is the \( i \)th cohomology group of the cochain complex:
	\[ \check{H}^i(X, \mathcal F) = \frac{\ker(C^i(\mathcal U, \mathcal F) \xrightarrow d C^{i+1}(\mathcal U, \mathcal F))}{\im(C^{i-1}(\mathcal U, \mathcal F) \xrightarrow d C^i(\mathcal U, \mathcal F))} \]
\end{definition}
\begin{example}
	Let \( X = S^1 \) be the usual circle.
	Let \( \mathcal F \) be the constant sheaf \( \underline{\mathbb Z} \); on any connected open set this sheaf has value \( \mathbb Z \), and for a general open set with \( n \) connected components, this sheaf has value \( \mathbb Z^n \).
	Let \( \mathcal U = \qty{U, V} \) where \( U, V \) are obtained by deleting disjoint closed intervals from the circle, giving an open cover with \( U, V \cong \mathbb R \).
	We have
	\[ C^0(\mathcal U, \underline{\mathbb Z}) = \mathbb Z^2 \]
	as there is one copy of \( \mathbb Z \) for \( U \) and one for \( V \).
	Also,
	\[ C^1(\mathcal U, \underline{\mathbb Z}) = \mathbb Z^2 \]
	given by \( \underline{\mathbb Z}(U \cap V) \).
	The differential is \( (a, b) \mapsto (b-a, b-a) \), so
	\[ \check{H}^0(\mathcal U, \underline{\mathbb Z}) \cong \mathbb Z = \ker d \]
	and
	\[ \check{H}^1(\mathcal U, \underline{\mathbb Z}) \cong \mathbb Z = \coker d \]
\end{example}
\begin{remark}
	\begin{enumerate}
		\item These \v{C}ech cohomology groups are equal to the corresponding singular cohomology groups of \( S^1 \).
		\item Note that \( \check{H} \) is typically only well-behaved when \( \mathcal U \) is also well-behaved.
		That is, \( \check{H}^i(\mathcal U, \mathcal F) \) depends on \( \mathcal U \) and not just \( X \).
		In the example above, we could have chosen \( \mathcal U = \qty{S^1} \), and in this case, \( \check{H}^1(\mathcal U, \underline{\mathbb Z}) = 0 \).
		Also note that \( \underline{\mathbb Z} \) is not a quasi-coherent sheaf.
		\item Let \( X = \mathbb P^1_k, U = X \setminus \qty{0}, V = X \setminus \qty{\infty}, \mathcal U = \qty{U, V} \).
		Then
		\[ \check{H}^0(\mathcal U, \mathcal O_X) = k;\quad \check{H}^1(\mathcal U, \mathcal O_X) = 0 \]
		\item Let \( X \) be Noetherian and separated, and let \( \qty{U_i}_{i \in I} \) be an affine cover of \( X \), so all \( U_{i_0 \dots i_p} \) are affine.
		Let \( \mathcal F \) be a quasi-coherent sheaf on \( X \).
		Then
		\[ \check{H}^p(\mathcal U, \mathcal F) \cong H^p(X, \mathcal F) \]
		and the isomorphism is natural.
		Thus, in this particular case, the cohomology is easy to calculate by going via \v{C}ech cohomology.
	\end{enumerate}
\end{remark}
\begin{theorem}
	Let \( X = \mathbb P^n_k \) and \( \mathcal F = \bigoplus_{d \in \mathbb Z} \mathcal O_{\mathbb P^n_k}(d) \).
	Then there are isomorphisms of graded \( k \)-vector spaces
	\begin{enumerate}
		\item \( H^0(X, \mathcal F) \cong k[x_0, \dots, x_n] \);
		\item \( H^n(X, \mathcal F) \cong \frac{1}{x_0 \dots x_n} k[x_0^{-1}, \dots, x_n^{-1}] \);
		\item \( H^p(X, \mathcal F) = 0 \) for \( p \neq 0, n \).
	\end{enumerate}
	In particular, \( H^0(\mathbb P^n_k, \mathcal O(d)) \) has dimension \( \binom{n+d}{d} \), and \( H^n(\mathbb P^n_k, \mathcal O(d)) \) has dimension \( \binom{-d-1}{n} \).
\end{theorem}
\begin{proof}
	We prove this result using \v{C}ech cohomology.
	Part (i) follows from earlier discussions, as \( H^0(X, \mathcal F) = \bigoplus_{d \in \mathbb Z} \Gamma(\mathbb P^n_k, \mathcal O(d)) \).

	\emph{Part (ii).}
	Consider the standard cover \( \mathcal U \) of \( \mathbb P^n_k \) by affines \( U_i = \mathbb V(x_i)^c \).
	Observe that
	\[ \mathcal F(U_{i_0 \dots i_p}) = k[x_0, \dots, x_n]_{x_{i_0} \dots x_{i_p}} \]
	This \( k \)-module is spanned by monomials \( x_0^{k_0} \dots x_n^{k_n} \) where \( k_{i_0}, \dots, k_{i_p} \in \mathbb Z \) and the other coefficients are nonnegative.
	In the associated \v{C}ech complex, we have
	\[ \check{C}^{n-1} = \bigoplus_{i=0}^n k[x_0, \dots, x_n]_{x_0 \dots \hat x_i \dots x_n};\quad \check{C}^n = k[x_0, \dots, x_n]_{x_0 \dots x_n} \]
	Since \( \mathcal U \) contains only \( n + 1 \) elements, \( \check{C}^{n+1} \) vanishes.
	Thus,
	\begin{align*}
		H^n(\mathbb P^n_k, \mathcal F) &= \check{H}^n(\mathcal U, \mathcal F) \\
		&= \frac{\check{C}^n}{\im (\check{C}^{n-1} \to \check{C}^n)} \\
		&= \frac{\vecspan_k \qty{x_0^{k_0} \dots x_n^{k_n} \mid k_i \in \mathbb Z}}{\vecspan_k \qty{x_0^{k_0} \dots x_n^{k_n} \mid \text{at least one } k_i \geq 0}}
	\end{align*}
	as required.

	\emph{Part (iii).}
	We will use the long exact sequence associated to a short exact sequence of sheaves and use induction on the dimension \( n \).
	First, observe that \( \mathbb P^{n-1}_k \) is isomorphic to the closed subscheme \( \mathbb V(x_0) \subseteq \mathbb P^n_k \).
	Let \( i : \mathbb P^{n-1}_k \to \mathbb P^n_k \) be the inclusion.
	Recall that \( \mathcal O_{\mathbb P^n_k}(-1) = L(-H) \) where \( H = \mathbb V(x_0) \).
	By a result on the example sheets, we obtain the \emph{ideal sheaf sequence}
	% https://q.uiver.app/#q=WzAsNSxbMCwwLCIwIl0sWzEsMCwiXFxtYXRoY2FsIE9fe1xcbWF0aGJiIFBebl9rfSgtMSkiXSxbMiwwLCJcXG1hdGhjYWwgT197XFxtYXRoYmIgUF5uX2t9Il0sWzMsMCwiaV9cXHN0YXIgXFxtYXRoY2FsIE9fe1xcbWF0aGJiIFBee24tMX1fa30iXSxbNCwwLCIwIl0sWzAsMV0sWzEsMl0sWzIsM10sWzMsNF1d
\[\begin{tikzcd}
	0 & {\mathcal O_{\mathbb P^n_k}(-1)} & {\mathcal O_{\mathbb P^n_k}} & {i_\star \mathcal O_{\mathbb P^{n-1}_k}} & 0
	\arrow[from=1-1, to=1-2]
	\arrow[from=1-2, to=1-3]
	\arrow[from=1-3, to=1-4]
	\arrow[from=1-4, to=1-5]
\end{tikzcd}\]
	where the map \( \mathcal O_{\mathbb P^n_k}(-1) \to \mathcal O_{\mathbb P^n_k} \) is given by multiplication by \( x_0 \).
	This is analogous to the fact that for an ideal \( I \) of a ring \( A \), we have a short exact sequence
	% https://q.uiver.app/#q=WzAsNSxbMCwwLCIwIl0sWzEsMCwiSSJdLFsyLDAsIkEiXSxbMywwLCJcXGZha3RvcntBfXtJfSJdLFs0LDAsIjAiXSxbMCwxXSxbMSwyXSxbMiwzXSxbMyw0XV0=
\[\begin{tikzcd}
	0 & I & A & {\faktor{A}{I}} & 0
	\arrow[from=1-1, to=1-2]
	\arrow[from=1-2, to=1-3]
	\arrow[from=1-3, to=1-4]
	\arrow[from=1-4, to=1-5]
\end{tikzcd}\]
	We obtain an associated long exact sequence for the homology.
	Assuming the result for dimension up to \( n - 1 \), we can break this into three smaller exact sequences.
	% https://q.uiver.app/#q=WzAsNyxbMCwwLCIwIl0sWzEsMCwiSF4wKFxcbWF0aGJiIFBebl9rLCBcXG1hdGhjYWwgRikiXSxbMiwwLCJIXjAoXFxtYXRoYmIgUF5uX2ssIFxcbWF0aGNhbCBGKSJdLFszLDAsIkheMChcXG1hdGhiYiBQXntuLTF9X2ssIFxcbWF0aGNhbCBGX3tcXG1hdGhiYiBQXntuLTF9X2t9KSJdLFs0LDAsIkheMShcXG1hdGhiYiBQXm5faywgXFxtYXRoY2FsIEYpIl0sWzUsMCwiSF4xKFxcbWF0aGJiIFBebl9rLCBcXG1hdGhjYWwgRikiXSxbNiwwLCIwIl0sWzAsMV0sWzEsMiwiXFxjZG90XFwsIHhfMCJdLFsyLDNdLFszLDRdLFs0LDUsIlxcY2RvdFxcLHhfMCJdLFs1LDZdXQ==
	\begin{equation}
		\begin{tikzcd}[column sep=small]
	0 & {H^0(\mathbb P^n_k, \mathcal F)} & {H^0(\mathbb P^n_k, \mathcal F)} & {H^0(\mathbb P^{n-1}_k, \mathcal F_{\mathbb P^{n-1}_k})} & {H^1(\mathbb P^n_k, \mathcal F)} & {H^1(\mathbb P^n_k, \mathcal F)} & 0
	\arrow[from=1-1, to=1-2]
	\arrow["{\cdot\, x_0}", from=1-2, to=1-3]
	\arrow[from=1-3, to=1-4]
	\arrow[from=1-4, to=1-5]
	\arrow["{\cdot\,x_0}", from=1-5, to=1-6]
	\arrow[from=1-6, to=1-7]
\end{tikzcd}
\tag{a}
\end{equation}
	where \( \mathcal F_{\mathbb P^{n-1}_k} = \bigoplus_{d \in \mathbb Z} \mathcal O_{\mathbb P^{n-1}_k}(d) \);
	% https://q.uiver.app/#q=WzAsNCxbMCwwLCIwIl0sWzEsMCwiSF5wKFxcbWF0aGJiIFBebl9rLCBcXG1hdGhjYWwgRikiXSxbMiwwLCJIXnAoXFxtYXRoYmIgUF5uX2ssIFxcbWF0aGNhbCBGKSJdLFszLDAsIjAiXSxbMCwxXSxbMSwyLCJcXGNkb3RcXCwgeF8wIl0sWzIsM11d
\begin{equation}\begin{tikzcd}
	0 & {H^p(\mathbb P^n_k, \mathcal F)} & {H^p(\mathbb P^n_k, \mathcal F)} & 0
	\arrow[from=1-1, to=1-2]
	\arrow["{\cdot\, x_0}", from=1-2, to=1-3]
	\arrow[from=1-3, to=1-4]
\end{tikzcd}\tag{b}\end{equation}
	for \( 1 < p < n - 1 \); and
	% https://q.uiver.app/#q=WzAsNyxbMCwwLCIwIl0sWzEsMCwiSF57bi0xfShcXG1hdGhiYiBQXm5faywgXFxtYXRoY2FsIEYpIl0sWzIsMCwiSF57bi0xfShcXG1hdGhiYiBQXm5faywgXFxtYXRoY2FsIEYpIl0sWzMsMCwiSF57bi0xfShcXG1hdGhiYiBQXntuLTF9X2ssIFxcbWF0aGNhbCBGX3tcXG1hdGhiYiBQXntuLTF9X2t9KSJdLFs0LDAsIkhebihcXG1hdGhiYiBQXm5faywgXFxtYXRoY2FsIEYpIl0sWzUsMCwiSF5uKFxcbWF0aGJiIFBebl9rLCBcXG1hdGhjYWwgRikiXSxbNiwwLCIwIl0sWzAsMV0sWzEsMiwiXFxjZG90XFwsIHhfMCJdLFsyLDNdLFszLDRdLFs0LDUsIlxcY2RvdFxcLHhfMCJdLFs1LDZdXQ==
\begin{equation}\begin{tikzcd}[column sep=small]
	0 & {H^{n-1}(\mathbb P^n_k, \mathcal F)} & {H^{n-1}(\mathbb P^n_k, \mathcal F)} & {H^{n-1}(\mathbb P^{n-1}_k, \mathcal F_{\mathbb P^{n-1}_k})} & {H^n(\mathbb P^n_k, \mathcal F)} & {H^n(\mathbb P^n_k, \mathcal F)} & 0
	\arrow[from=1-1, to=1-2]
	\arrow["{\cdot\, x_0}", from=1-2, to=1-3]
	\arrow[from=1-3, to=1-4]
	\arrow[from=1-4, to=1-5]
	\arrow["{\cdot\,x_0}", from=1-5, to=1-6]
	\arrow[from=1-6, to=1-7]
\end{tikzcd}\tag{c}\end{equation}
	By using (a) and (c), we observe that (b) is also exact for \( p = 1 \) and \( p = n - 1 \) by explicit computation in the \v{C}ech complex.
	Now, multiplication by \( x_0 \) makes \( H^p(\mathbb P^n_k, \mathcal F) \) into a \( k[x_0] \)-module.
	We will calculate the localisation \( H^p(\mathbb P^n_k, \mathcal F)_{x_0} \).
	As localisation is exact, \( H^p(\mathbb P^n_k, \mathcal F)_{x_0} = H^p\qty(U_0, \eval{\mathcal F}_{U_0}) \).
	But the right-hand side vanishes for \( p > 0 \) as \( U_0 \) is affine.
	Hence, for any \( \alpha \in H^p(\mathbb P^n_k, \mathcal F) \), there exists \( k \) such that \( x_0^k \alpha = 0 \).
	But multiplication by \( x_0 \) is an isomorphism on cohomology by (b), so in fact \( H^p(\mathbb P^n_k, \mathcal F) = 0 \) for all \( 1 \leq p \leq n - 1 \).
\end{proof}
Given the exact sequence
\[\begin{tikzcd}
	0 & {\mathcal O_{\mathbb P^n_k}(-1)} & {\mathcal O_{\mathbb P^n_k}} & {i_\star \mathcal O_{\mathbb P^{n-1}_k}} & 0
	\arrow[from=1-1, to=1-2]
	\arrow[from=1-2, to=1-3]
	\arrow[from=1-3, to=1-4]
	\arrow[from=1-4, to=1-5]
\end{tikzcd}\]
taking the tensor product with \( \mathcal O_{\mathbb P^n_k}(d) \), one can show that we obtain an exact sequence
\[\begin{tikzcd}
	0 & {\mathcal O_{\mathbb P^n_k}(d-1)} & {\mathcal O_{\mathbb P^n_k}(d)} & {i_\star \mathcal O_{\mathbb P^{n-1}_k}(d)} & 0
	\arrow[from=1-1, to=1-2]
	\arrow[from=1-2, to=1-3]
	\arrow[from=1-3, to=1-4]
	\arrow[from=1-4, to=1-5]
\end{tikzcd}\]
Note that \( \mathcal O_{\mathbb P^n_k}(d) \) is locally free.

Let \( X \) be proper over \( \Spec k \) and let \( \mathcal F \) be a coherent sheaf on \( X \).
\begin{remark}
	\begin{enumerate}
		\item We have observed that \( H^0(X, \mathcal F) \) is a finite-dimensional \( k \)-vector space.
		The same holds for all \( H^p(X, \mathcal F) \).
		\item If \( X \) has dimension \( n \), then \( H^p(X, \mathcal F) \) vanishes for \( p > n \).
		Thus, given \( (X, \mathcal F) \), there are finitely many numbers \( h^p(X, \mathcal F) = \dim_k H^p(X, \mathcal F) \).
	\end{enumerate}
\end{remark}
\begin{definition}
	The \emph{Euler characteristic} of \( \mathcal F \) is
	\[ \chi(\mathcal F) = \sum_{p = 0}^\infty (-1)^p h^p(X, \mathcal F) \]
\end{definition}
Suppose that
% https://q.uiver.app/#q=WzAsNSxbMCwwLCIwIl0sWzEsMCwiXFxtYXRoY2FsIEYiXSxbMiwwLCJcXG1hdGhjYWwgRiciXSxbMywwLCJcXG1hdGhjYWwgRicnIl0sWzQsMCwiMCJdLFswLDFdLFsxLDJdLFsyLDNdLFszLDRdXQ==
\[\begin{tikzcd}
	0 & {\mathcal F} & {\mathcal F'} & {\mathcal F''} & 0
	\arrow[from=1-1, to=1-2]
	\arrow[from=1-2, to=1-3]
	\arrow[from=1-3, to=1-4]
	\arrow[from=1-4, to=1-5]
\end{tikzcd}\]
is an exact sequence of such sheaves.
Then the associated long exact sequence gives
\[ \chi(F') = \chi(F) + \chi(F'') \]
