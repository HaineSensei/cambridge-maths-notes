\subsection{???}
Let \( G \) be a group.
\begin{definition}
    The \emph{integral group ring} \( \mathbb Z G \) is the set of formal sums \( \sum n_g g \), where \( n_g \in \mathbb Z \), \( g \in G \), and only finitely many of the \( n_g \) are nonzero.
    An addition operation makes this set a free abelian group:
    \[ \qty(\sum m_g g) + \qty(\sum n_g g) = \sum (m_g + n_g) g \]
    Multiplication is defined by
    \[ \qty(\sum_{h \in G} m_h h)(\sum_{k \in G} n_k k) = \sum \qty(\sum_{hk = g} m_h n_k) g \]
    The multiplicative identity is \( 1 e \) where \( e \) is the identity of \( G \).
    This produces an associative ring, which underlies the integral representation theory of \( G \).
\end{definition}
\begin{definition}
    A \emph{(left) \( \mathbb Z G \)-module} \( M \) is an abelian group under addition together with a map \( \mathbb Z G \times M \to M \) denoted \( (r, m) \mapsto rm \), satisfying
    \begin{enumerate}
        \item \( r(m_1 + m_2) = rm_1 + rm_2 \);
        \item \( (r_1 + r_2)m = r_1 m + r_2 m \);
        \item \( r_1(r_2 m) = (r_1 r_2) m \);
        \item \( 1 m = m \).
    \end{enumerate}
\end{definition}
A module is \emph{trivial} if \( gm = m \) for all \( g \in G \) and \( m \in M \).
We call \( \mathbb Z \) \emph{the} trivial module, given by the trivial action \( gn = n \) for all \( n \in \mathbb Z \) and \( g \in G \).

The \emph{free} \( \mathbb Z G \)-module on a set \( X \) is the module of formal sums \( \sum r_x x \) where \( r_x \in \mathbb Z G \) and \( x \in X \), and only finitely many of the \( r_x \) are nonzero.
This has the obvious \( G \)-action.
This module will be denoted \( \mathbb Z G\qty{X} \).

We can define submodules, quotient modules, and so on as one would expect.

\begin{definition}
    A \emph{(left) \( \mathbb Z G \)-map} or \emph{morphism} \( \alpha : M_1 \to M_2 \) is a map of abelian groups with \( \alpha(r m) = r \alpha(m) \) for all \( r \in \mathbb Z G \) and \( m \in M_1 \).
\end{definition}
\begin{example}
    The \emph{augmentation map} \( \varepsilon : \mathbb Z G \to \mathbb Z \) is the \( \mathbb Z G \)-map between left \( \mathbb Z G \)-modules given by
    \[ \sum n_g g \mapsto \sum n_g \]
    This is also a right \( \mathbb Z G \)-map, and also a map of rings.
\end{example}
We will write \( \Hom_G(M, N) \) to be the set of \( \mathbb Z G \)-maps \( M \to N \), which is made into an abelian group under pointwise addition.
\begin{example}
    Regarding \( \mathbb Z G \) as a left \( \mathbb Z G \)-module, then
    \[ \Hom_G(\mathbb Z G, M) \cong M \]
    for any left \( \mathbb Z G \)-module \( M \).
    This isomorphism is given by \( \varphi \mapsto \varphi(1) \); the \( \mathbb Z G \)-map is determined by the image of \( 1 \).
    \[ \varphi(r) = \varphi(r \cdot 1) = r \varphi(1) \]
\end{example}
Note that \( \Hom_G(\mathbb Z G, M) \) can be viewed as a left \( \mathbb Z G \)-module, given by
\[ (s \varphi)(r) = \varphi(rs);\quad s \in \mathbb Z G \]
Note that the isomorphism
\[ \Hom_G(\mathbb Z G, \mathbb Z G) \cong \mathbb Z G;\quad \varphi \mapsto \varphi(1) \]
satisfies \( \varphi(r) = r \varphi(1) \) and so \( \varphi \) corresponds to multiplication on the right by \( \varphi(1) \).
\begin{remark}
    \( G \) may not be abelian, and so we must carefully distinguish left and right actions.
\end{remark}
\begin{definition}
    If \( f : M_1 \to M_2 \) is a \( \mathbb Z G \)-map, its \emph{dual maps} \( f^\star \) are \( \mathbb Z G \)-maps \( \Hom_G(M_2, N) \to \Hom_G(M_1, N) \) for each \( \mathbb Z G \)-module \( N \), given by composition on the right with \( f \).
    If \( f : N_1 \to N_2 \), its \emph{induced maps} \( f_\star \) are \( \Hom_G(M, N_1) \to \Hom_G(M, N_2) \) given by composition on the left with \( f \).
    These are maps of abelian groups.
\end{definition}
We will now present a prototypical example.
\begin{example}
    Let \( G = \langle t \rangle \) be an infinite cyclic group.
    Consider the graph whose vertices are \( v_i \) for \( i \in \mathbb Z \), where \( v_i \) is joined to \( v_{i+1} \) and \( v_{i-1} \).
    Let \( V \) be its set of vertices, and \( E \) be its set of edges.
    \( G \) acts by translations on this graph, where \( t \) maps \( v_i \) to \( v_{i+1} \).
    The formal sums \( \mathbb Z V \) and \( \mathbb Z E \) can be regarded as \( \mathbb Z G \)-modules.
    They are free: \( \mathbb Z V = \mathbb Z G \qty{v_0} \), and \( \mathbb Z E = \mathbb Z G \qty{e} \) where \( e \) is the edge connecting \( v_0 \) and \( v_1 \).
    The boundary map is a \( \mathbb Z G \)-map \( d : \mathbb Z E \to \mathbb Z V \) given by \( e \mapsto v_1 - v_0 \).
    There is also a \( \mathbb Z G \)-map \( \mathbb Z V \to \mathbb Z \) given by \( v_0 \mapsto 1 \); this corresponds to the augmentation map.
\end{example}
\begin{definition}
    A \emph{chain complex} of \( \mathbb Z G \)-modules is a sequence
    % https://q.uiver.app/#q=WzAsNSxbMCwwLCJNX3MiXSxbMSwwLCJNX3tzLTF9Il0sWzIsMCwiTV97cy0yfSJdLFszLDAsIlxcY2RvdHMiXSxbNCwwLCJNX3QiXSxbMCwxLCJkX3MiXSxbMSwyLCJkX3tzLTF9Il0sWzIsM10sWzMsNCwiZF97dCsxfSJdXQ==
    \[\begin{tikzcd}
        {M_s} & {M_{s-1}} & {M_{s-2}} & \cdots & {M_t}
        \arrow["{d_s}", from=1-1, to=1-2]
        \arrow["{d_{s-1}}", from=1-2, to=1-3]
        \arrow[from=1-3, to=1-4]
        \arrow["{d_{t+1}}", from=1-4, to=1-5]
    \end{tikzcd}\]
    such that for every \( t < n < s \), we have \( d_n d_{n+1} = 0 \), and so \( \im d_{n+1} \subseteq \ker d_n \).
    We will refer to the entire sequence as \( M_\bullet = (M_n, d_n)_{t \leq n \leq s} \).
\end{definition}
We say that \( M_\bullet \) is \emph{exact} at \( M_n \) if \( \im d_{n+1} = \ker d_n \), and we say it is \emph{exact} if it is exact at all \( M_n \) for \( t < n < s \).
The \emph{homology} of this chain complex is
\[ H_s(M_\bullet) = \ker d_s;\quad H_n(M_\bullet) = \faktor{\ker d_n}{\im d_{n+1}};\quad H_t(M_\bullet) = \coker d_{t-1} = \faktor{M_t}{\im d_{t-1}} \]
A \emph{short exact sequence} is an exact chain complex of the form
% https://q.uiver.app/#q=WzAsNSxbMCwwLCIwIl0sWzEsMCwiTV8xIl0sWzIsMCwiTV8yIl0sWzMsMCwiTV8zIl0sWzQsMCwiMCJdLFswLDFdLFsxLDIsIlxcYWxwaGEiXSxbMiwzLCJcXGJldGEiXSxbMyw0XV0=
\[\begin{tikzcd}
0 & {M_1} & {M_2} & {M_3} & 0
\arrow[from=1-1, to=1-2]
\arrow["\alpha", from=1-2, to=1-3]
\arrow["\beta", from=1-3, to=1-4]
\arrow[from=1-4, to=1-5]
\end{tikzcd}\]
That is, \( \alpha \) is injective, \( \beta \) is surjective, and \( \im \alpha = \ker \beta \).
\begin{example}
    In our example above, we have the short exact sequence
    % https://q.uiver.app/#q=WzAsNSxbMCwwLCIwIl0sWzEsMCwiXFxtYXRoYmIgWiBFIl0sWzIsMCwiXFxtYXRoYmIgWiBWIl0sWzMsMCwiXFxtYXRoYmIgWiJdLFs0LDAsIjAiXSxbMCwxXSxbMSwyXSxbMiwzXSxbMyw0XV0=
\[\begin{tikzcd}
	0 & {\mathbb Z E} & {\mathbb Z V} & {\mathbb Z} & 0
	\arrow[from=1-1, to=1-2]
	\arrow[from=1-2, to=1-3]
	\arrow[from=1-3, to=1-4]
	\arrow[from=1-4, to=1-5]
\end{tikzcd}\]
    This corresponds to a short exact sequence
    % https://q.uiver.app/#q=WzAsNSxbMCwwLCIwIl0sWzEsMCwiXFxtYXRoYmIgWiBHIl0sWzIsMCwiXFxtYXRoYmIgWiBHIl0sWzMsMCwiXFxtYXRoYmIgWiJdLFs0LDAsIjAiXSxbMCwxXSxbMSwyXSxbMiwzXSxbMyw0XV0=
\[\begin{tikzcd}
	0 & {\mathbb Z G} & {\mathbb Z G} & {\mathbb Z} & 0
	\arrow[from=1-1, to=1-2]
	\arrow[from=1-2, to=1-3]
	\arrow[from=1-3, to=1-4]
	\arrow[from=1-4, to=1-5]
\end{tikzcd}\]
    where \( G = \langle t \rangle \) is an infinite cyclic group, and the map \( \mathbb Z G \to \mathbb Z G \) is given by multiplication on the right by \( t - 1 \).
\end{example}
\begin{definition}
    A \( \mathbb Z G \)-module \( P \) is \emph{projective} if, for every surjective \( \mathbb Z G \)-map \( \alpha : M_1 \to M_2 \) and every \( \mathbb Z G \)-map \( \beta : P \to M_2 \), there is a map \( \overline\beta : P \to M_1 \) such that \( \alpha \circ \overline \beta = \beta \).
    % https://q.uiver.app/#q=WzAsNCxbMSwwLCJQIl0sWzEsMSwiTV8yIl0sWzAsMSwiTV8xIl0sWzIsMSwiMCJdLFswLDEsIlxcYmV0YSJdLFswLDIsIlxcb3ZlcmxpbmVcXGJldGEiLDIseyJzdHlsZSI6eyJib2R5Ijp7Im5hbWUiOiJkYXNoZWQifX19XSxbMiwxLCJcXGFscGhhIiwyXSxbMSwzXV0=
\[\begin{tikzcd}
	& P \\
	{M_1} & {M_2} & 0
	\arrow["\beta", from=1-2, to=2-2]
	\arrow["\overline\beta"', dashed, from=1-2, to=2-1]
	\arrow["\alpha"', from=2-1, to=2-2]
	\arrow[from=2-2, to=2-3]
\end{tikzcd}\]
\end{definition}
Given any short exact sequence
% https://q.uiver.app/#q=WzAsNSxbMCwwLCIwIl0sWzEsMCwiTiJdLFsyLDAsIk1fMSJdLFszLDAsIk1fMiJdLFs0LDAsIjAiXSxbMCwxXSxbMSwyLCJmIl0sWzIsMywiXFxhbHBoYSJdLFszLDRdXQ==
\[\begin{tikzcd}
	0 & N & {M_1} & {M_2} & 0
	\arrow[from=1-1, to=1-2]
	\arrow["f", from=1-2, to=1-3]
	\arrow["\alpha", from=1-3, to=1-4]
	\arrow[from=1-4, to=1-5]
\end{tikzcd}\]
we can consider
% https://q.uiver.app/#q=WzAsNSxbMCwwLCIwIl0sWzEsMCwiXFxIb21fRyhQLCBOKSJdLFsyLDAsIlxcSG9tX0coUCwgTV8xKSJdLFszLDAsIlxcSG9tX0coUCwgTV8yKSJdLFs0LDAsIjAiXSxbMCwxXSxbMSwyLCJmX1xcc3RhciJdLFsyLDMsIlxcYWxwaGFfXFxzdGFyIl0sWzMsNF1d
\[\begin{tikzcd}
	0 & {\Hom_G(P, N)} & {\Hom_G(P, M_1)} & {\Hom_G(P, M_2)} & 0
	\arrow[from=1-1, to=1-2]
	\arrow["{f_\star}", from=1-2, to=1-3]
	\arrow["{\alpha_\star}", from=1-3, to=1-4]
	\arrow[from=1-4, to=1-5]
\end{tikzcd}\]
We could have defined projectivity by saying that this new sequence is exact.
Note that this sequence is always a chain complex regardless if \( P \) is projective, and we always have exactness except possibly at \( \Hom_G(P, M_2) \).
