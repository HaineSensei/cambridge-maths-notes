\subsection{???}
Let \( G \) be a group.
\begin{definition}
    The \emph{integral group ring} \( \mathbb Z G \) is the set of formal sums \( \sum n_g g \), where \( n_g \in \mathbb Z \), \( g \in G \), and only finitely many of the \( n_g \) are nonzero.
    An addition operation makes this set a free abelian group:
    \[ \qty(\sum m_g g) + \qty(\sum n_g g) = \sum (m_g + n_g) g \]
    Multiplication is defined by
    \[ \qty(\sum_{h \in G} m_h h)(\sum_{k \in G} n_k k) = \sum \qty(\sum_{hk = g} m_h n_k) g \]
    The multiplicative identity is \( 1 e \) where \( e \) is the identity of \( G \).
    This produces an associative ring, which underlies the integral representation theory of \( G \).
\end{definition}
\begin{definition}
    A \emph{(left) \( \mathbb Z G \)-module} \( M \) is an abelian group under addition together with a map \( \mathbb Z G \times M \to M \) denoted \( (r, m) \mapsto rm \), satisfying
    \begin{enumerate}
        \item \( r(m_1 + m_2) = rm_1 + rm_2 \);
        \item \( (r_1 + r_2)m = r_1 m + r_2 m \);
        \item \( r_1(r_2 m) = (r_1 r_2) m \);
        \item \( 1 m = m \).
    \end{enumerate}
\end{definition}
A module is \emph{trivial} if \( gm = m \) for all \( g \in G \) and \( m \in M \).
We call \( \mathbb Z \) \emph{the} trivial module, given by the trivial action \( gn = n \) for all \( n \in \mathbb Z \) and \( g \in G \).

The \emph{free} \( \mathbb Z G \)-module on a set \( X \) is the module of formal sums \( \sum r_x x \) where \( r_x \in \mathbb Z G \) and \( x \in X \), and only finitely many of the \( r_x \) are nonzero.
This has the obvious \( G \)-action.
This module will be denoted \( \mathbb Z G\qty{X} \).

We can define submodules, quotient modules, and so on as one would expect.

\begin{definition}
    A \emph{(left) \( \mathbb Z G \)-map} or \emph{morphism} \( \alpha : M_1 \to M_2 \) is a map of abelian groups with \( \alpha(r m) = r \alpha(m) \) for all \( r \in \mathbb Z G \) and \( m \in M_1 \).
\end{definition}
\begin{example}
    The \emph{augmentation map} \( \varepsilon : \mathbb Z G \to \mathbb Z \) is the \( \mathbb Z G \)-map between left \( \mathbb Z G \)-modules given by
    \[ \sum n_g g \mapsto \sum n_g \]
    This is also a right \( \mathbb Z G \)-map, and also a map of rings.
\end{example}
We will write \( \Hom_G(M, N) \) to be the set of \( \mathbb Z G \)-maps \( M \to N \), which is made into an abelian group under pointwise addition.
\begin{example}
    Regarding \( \mathbb Z G \) as a left \( \mathbb Z G \)-module, then
    \[ \Hom_G(\mathbb Z G, M) \cong M \]
    for any left \( \mathbb Z G \)-module \( M \).
    This isomorphism is given by \( \varphi \mapsto \varphi(1) \); the \( \mathbb Z G \)-map is determined by the image of \( 1 \).
    \[ \varphi(r) = \varphi(r \cdot 1) = r \varphi(1) \]
\end{example}
Note that \( \Hom_G(\mathbb Z G, M) \) can be viewed as a left \( \mathbb Z G \)-module, given by
\[ (s \varphi)(r) = \varphi(rs);\quad s \in \mathbb Z G \]
Note that the isomorphism
\[ \Hom_G(\mathbb Z G, \mathbb Z G) \cong \mathbb Z G;\quad \varphi \mapsto \varphi(1) \]
satisfies \( \varphi(r) = r \varphi(1) \) and so \( \varphi \) corresponds to multiplication on the right by \( \varphi(1) \).
\begin{remark}
    \( G \) may not be abelian, and so we must carefully distinguish left and right actions.
\end{remark}
\begin{definition}
    If \( f : M_1 \to M_2 \) is a \( \mathbb Z G \)-map, its \emph{dual maps} \( f^\star \) are \( \mathbb Z G \)-maps \( \Hom_G(M_2, N) \to \Hom_G(M_1, N) \) for each \( \mathbb Z G \)-module \( N \), given by composition on the right with \( f \).
    If \( f : N_1 \to N_2 \), its \emph{induced maps} \( f_\star \) are \( \Hom_G(M, N_1) \to \Hom_G(M, N_2) \) given by composition on the left with \( f \).
    These are maps of abelian groups.
\end{definition}
We will now present a prototypical example.
\begin{example}
    Let \( G = \langle t \rangle \) be an infinite cyclic group.
    Consider the graph whose vertices are \( v_i \) for \( i \in \mathbb Z \), where \( v_i \) is joined to \( v_{i+1} \) and \( v_{i-1} \).
    Let \( V \) be its set of vertices, and \( E \) be its set of edges.
    \( G \) acts by translations on this graph, where \( t \) maps \( v_i \) to \( v_{i+1} \).
    The formal sums \( \mathbb Z V \) and \( \mathbb Z E \) can be regarded as \( \mathbb Z G \)-modules.
    They are free: \( \mathbb Z V = \mathbb Z G \qty{v_0} \), and \( \mathbb Z E = \mathbb Z G \qty{e} \) where \( e \) is the edge connecting \( v_0 \) and \( v_1 \).
    The boundary map is a \( \mathbb Z G \)-map \( d : \mathbb Z E \to \mathbb Z V \) given by \( e \mapsto v_1 - v_0 \).
    There is also a \( \mathbb Z G \)-map \( \mathbb Z V \to \mathbb Z \) given by \( v_0 \mapsto 1 \); this corresponds to the augmentation map.
\end{example}
\begin{definition}
    A \emph{chain complex} of \( \mathbb Z G \)-modules is a sequence
    % https://q.uiver.app/#q=WzAsNSxbMCwwLCJNX3MiXSxbMSwwLCJNX3tzLTF9Il0sWzIsMCwiTV97cy0yfSJdLFszLDAsIlxcY2RvdHMiXSxbNCwwLCJNX3QiXSxbMCwxLCJkX3MiXSxbMSwyLCJkX3tzLTF9Il0sWzIsM10sWzMsNCwiZF97dCsxfSJdXQ==
    \[\begin{tikzcd}
        {M_s} & {M_{s-1}} & {M_{s-2}} & \cdots & {M_t}
        \arrow["{d_s}", from=1-1, to=1-2]
        \arrow["{d_{s-1}}", from=1-2, to=1-3]
        \arrow[from=1-3, to=1-4]
        \arrow["{d_{t+1}}", from=1-4, to=1-5]
    \end{tikzcd}\]
    such that for every \( t < n < s \), we have \( d_n d_{n+1} = 0 \), and so \( \im d_{n+1} \subseteq \ker d_n \).
    We will refer to the entire sequence as \( M_\bullet = (M_n, d_n)_{t \leq n \leq s} \).
\end{definition}
We say that \( M_\bullet \) is \emph{exact} at \( M_n \) if \( \im d_{n+1} = \ker d_n \), and we say it is \emph{exact} if it is exact at all \( M_n \) for \( t < n < s \).
The \emph{homology} of this chain complex is
\[ H_s(M_\bullet) = \ker d_s;\quad H_n(M_\bullet) = \faktor{\ker d_n}{\im d_{n+1}};\quad H_t(M_\bullet) = \coker d_{t-1} = \faktor{M_t}{\im d_{t-1}} \]
A \emph{short exact sequence} is an exact chain complex of the form
% https://q.uiver.app/#q=WzAsNSxbMCwwLCIwIl0sWzEsMCwiTV8xIl0sWzIsMCwiTV8yIl0sWzMsMCwiTV8zIl0sWzQsMCwiMCJdLFswLDFdLFsxLDIsIlxcYWxwaGEiXSxbMiwzLCJcXGJldGEiXSxbMyw0XV0=
\[\begin{tikzcd}
0 & {M_1} & {M_2} & {M_3} & 0
\arrow[from=1-1, to=1-2]
\arrow["\alpha", from=1-2, to=1-3]
\arrow["\beta", from=1-3, to=1-4]
\arrow[from=1-4, to=1-5]
\end{tikzcd}\]
That is, \( \alpha \) is injective, \( \beta \) is surjective, and \( \im \alpha = \ker \beta \).
\begin{example}
    In our example above, we have the short exact sequence
    % https://q.uiver.app/#q=WzAsNSxbMCwwLCIwIl0sWzEsMCwiXFxtYXRoYmIgWiBFIl0sWzIsMCwiXFxtYXRoYmIgWiBWIl0sWzMsMCwiXFxtYXRoYmIgWiJdLFs0LDAsIjAiXSxbMCwxXSxbMSwyXSxbMiwzXSxbMyw0XV0=
\[\begin{tikzcd}
	0 & {\mathbb Z E} & {\mathbb Z V} & {\mathbb Z} & 0
	\arrow[from=1-1, to=1-2]
	\arrow[from=1-2, to=1-3]
	\arrow[from=1-3, to=1-4]
	\arrow[from=1-4, to=1-5]
\end{tikzcd}\]
    This corresponds to a short exact sequence
    % https://q.uiver.app/#q=WzAsNSxbMCwwLCIwIl0sWzEsMCwiXFxtYXRoYmIgWiBHIl0sWzIsMCwiXFxtYXRoYmIgWiBHIl0sWzMsMCwiXFxtYXRoYmIgWiJdLFs0LDAsIjAiXSxbMCwxXSxbMSwyXSxbMiwzXSxbMyw0XV0=
\[\begin{tikzcd}
	0 & {\mathbb Z G} & {\mathbb Z G} & {\mathbb Z} & 0
	\arrow[from=1-1, to=1-2]
	\arrow[from=1-2, to=1-3]
	\arrow[from=1-3, to=1-4]
	\arrow[from=1-4, to=1-5]
\end{tikzcd}\]
    where \( G = \langle t \rangle \) is an infinite cyclic group, and the map \( \mathbb Z G \to \mathbb Z G \) is given by multiplication on the right by \( t - 1 \).
\end{example}
\begin{definition}
    A \( \mathbb Z G \)-module \( P \) is \emph{projective} if, for every surjective \( \mathbb Z G \)-map \( \alpha : M_1 \to M_2 \) and every \( \mathbb Z G \)-map \( \beta : P \to M_2 \), there is a map \( \overline\beta : P \to M_1 \) such that \( \alpha \circ \overline \beta = \beta \).
    % https://q.uiver.app/#q=WzAsNCxbMSwwLCJQIl0sWzEsMSwiTV8yIl0sWzAsMSwiTV8xIl0sWzIsMSwiMCJdLFswLDEsIlxcYmV0YSJdLFswLDIsIlxcb3ZlcmxpbmVcXGJldGEiLDIseyJzdHlsZSI6eyJib2R5Ijp7Im5hbWUiOiJkYXNoZWQifX19XSxbMiwxLCJcXGFscGhhIiwyXSxbMSwzXV0=
\[\begin{tikzcd}
	& P \\
	{M_1} & {M_2} & 0
	\arrow["\beta", from=1-2, to=2-2]
	\arrow["\overline\beta"', dashed, from=1-2, to=2-1]
	\arrow["\alpha"', from=2-1, to=2-2]
	\arrow[from=2-2, to=2-3]
\end{tikzcd}\]
\end{definition}
Given any short exact sequence
% https://q.uiver.app/#q=WzAsNSxbMCwwLCIwIl0sWzEsMCwiTiJdLFsyLDAsIk1fMSJdLFszLDAsIk1fMiJdLFs0LDAsIjAiXSxbMCwxXSxbMSwyLCJmIl0sWzIsMywiXFxhbHBoYSJdLFszLDRdXQ==
\[\begin{tikzcd}
	0 & N & {M_1} & {M_2} & 0
	\arrow[from=1-1, to=1-2]
	\arrow["f", from=1-2, to=1-3]
	\arrow["\alpha", from=1-3, to=1-4]
	\arrow[from=1-4, to=1-5]
\end{tikzcd}\]
we can consider
% https://q.uiver.app/#q=WzAsNSxbMCwwLCIwIl0sWzEsMCwiXFxIb21fRyhQLCBOKSJdLFsyLDAsIlxcSG9tX0coUCwgTV8xKSJdLFszLDAsIlxcSG9tX0coUCwgTV8yKSJdLFs0LDAsIjAiXSxbMCwxXSxbMSwyLCJmX1xcc3RhciJdLFsyLDMsIlxcYWxwaGFfXFxzdGFyIl0sWzMsNF1d
\[\begin{tikzcd}
	0 & {\Hom_G(P, N)} & {\Hom_G(P, M_1)} & {\Hom_G(P, M_2)} & 0
	\arrow[from=1-1, to=1-2]
	\arrow["{f_\star}", from=1-2, to=1-3]
	\arrow["{\alpha_\star}", from=1-3, to=1-4]
	\arrow[from=1-4, to=1-5]
\end{tikzcd}\]
We could have defined projectivity by saying that this new sequence is exact.
Note that this sequence is always a chain complex regardless if \( P \) is projective, and we always have exactness except possibly at \( \Hom_G(P, M_2) \).
\begin{lemma}
    Free modules are projective.
\end{lemma}
\begin{proof}
    Let \( \alpha : M_1 \to M_2 \) be a surjective \( \mathbb Z G \)-map, and let \( \beta : \mathbb Z G\qty{X} \to M_2 \).
    Then for each generator \( x \in X \), there exists some \( m_x \in M_1 \) such that \( \alpha(m_x) = \beta(x) \).
    We then define \( \overline\beta : \mathbb Z G\qty{X} \to M_1 \) by mapping
    \[ \sum r_x x \mapsto \sum r_x m_x \]
    which satisfies the required equation \( \alpha \overline\beta = \beta \).
\end{proof}
\begin{definition}
    A \emph{projective (free) resolution} of the trivial module \( \mathbb Z \) is an exact sequence
    % https://q.uiver.app/#q=WzAsNSxbMCwwLCJcXGNkb3RzIl0sWzEsMCwiUF8xIl0sWzIsMCwiUF8wIl0sWzMsMCwiXFxtYXRoYmIgWiJdLFs0LDAsIjAiXSxbMCwxLCJkXzIiXSxbMSwyLCJkXzEiXSxbMiwzLCJkXzAiXSxbMyw0XV0=
\[\begin{tikzcd}
	\cdots & {P_1} & {P_0} & {\mathbb Z} & 0
	\arrow["{d_2}", from=1-1, to=1-2]
	\arrow["{d_1}", from=1-2, to=1-3]
	\arrow["{d_0}", from=1-3, to=1-4]
	\arrow[from=1-4, to=1-5]
\end{tikzcd}\]
    where the \( P_i \) are projective (respectively free).
    This is a chain complex.
\end{definition}
\begin{example}
    Let \( G = \langle t \rangle \) be an infinite cyclic group.
    Then we have a finite free resolution of \( \mathbb Z \) given by the exact sequence
    % https://q.uiver.app/#q=WzAsNSxbMCwwLCIwIl0sWzEsMCwiXFxtYXRoYmIgWiBHIl0sWzIsMCwiXFxtYXRoYmIgWiBHIl0sWzMsMCwiXFxtYXRoYmIgWiJdLFs0LDAsIjAiXSxbMCwxXSxbMSwyLCJcXGNkb3QgXFwsKHQgLSAxKSJdLFsyLDMsIlxcdmFyZXBzaWxvbiJdLFszLDRdXQ==
\[\begin{tikzcd}
	0 & {\mathbb Z G} & {\mathbb Z G} & {\mathbb Z} & 0
	\arrow[from=1-1, to=1-2]
	\arrow["{\cdot \,(t - 1)}", from=1-2, to=1-3]
	\arrow["\varepsilon", from=1-3, to=1-4]
	\arrow[from=1-4, to=1-5]
\end{tikzcd}\]
    where \( \varepsilon \) is the augmentation map.
\end{example}
\begin{example}
    Let \( G = \langle t \rangle \) be a cyclic group of order \( n \).
    Then we have a resolution
    % https://q.uiver.app/#q=WzAsOCxbMywwLCJcXG1hdGhiYiBaIEciXSxbNCwwLCJcXG1hdGhiYiBaIEciXSxbNSwwLCJcXG1hdGhiYiBaIEciXSxbNiwwLCJcXG1hdGhiYiBaIl0sWzcsMCwiMCJdLFsyLDAsIlxcbWF0aGJiIFogRyJdLFsxLDAsIlxcbWF0aGJiIFogRyJdLFswLDAsIlxcY2RvdHMiXSxbMCwxLCJcXGJldGEiXSxbMSwyLCJcXGFscGhhIl0sWzIsMywiXFx2YXJlcHNpbG9uIl0sWzMsNF0sWzUsMCwiXFxhbHBoYSJdLFs2LDUsIlxcYmV0YSJdLFs3LDZdXQ==
\[\begin{tikzcd}
	\cdots & {\mathbb Z G} & {\mathbb Z G} & {\mathbb Z G} & {\mathbb Z G} & {\mathbb Z G} & {\mathbb Z} & 0
	\arrow["\beta", from=1-4, to=1-5]
	\arrow["\alpha", from=1-5, to=1-6]
	\arrow["\varepsilon", from=1-6, to=1-7]
	\arrow[from=1-7, to=1-8]
	\arrow["\alpha", from=1-3, to=1-4]
	\arrow["\beta", from=1-2, to=1-3]
	\arrow[from=1-1, to=1-2]
\end{tikzcd}\]
    where
    \[ \alpha(x) = x(t-1);\quad \beta(x) = x(1 + t + \dots + t^{n-1}) \]
\end{example}
From algebraic topology, if we have a connected simplicial complex \( X \) with fundamental group \( \pi_1(X) = G \), such that the universal cover \( \widetilde X \) is contractible, we obtain a free resolution of \( \mathbb Z \) given by the universal cover.
In this way, the simplicial complex \( X \) contains a lot of information about its fundamental group; this is what we aim to replicate algebraically.

For calculation purposes, we are interested in `small' resolutions, for instance where the free modules have small rank.
However, for theory development, we often want general constructions, and resolutions given by generic theory tend to be large.
\begin{definition}
    \( G \) is of \emph{type \( FP_n \)} if \( \mathbb Z \) has a projective resolution
    \[\begin{tikzcd}
        \cdots & {P_1} & {P_0} & {\mathbb Z} & 0
        \arrow["{d_2}", from=1-1, to=1-2]
        \arrow["{d_1}", from=1-2, to=1-3]
        \arrow["{d_0}", from=1-3, to=1-4]
        \arrow[from=1-4, to=1-5]
    \end{tikzcd}\]
    which may be infinite, but where \( P_n, P_{n-1}, \dots, P_0 \) are finitely generated as \( \mathbb Z G \)-modules.

    We say \( G \) is of \emph{type \( FP_\infty \)} if \( \mathbb Z \) has a projective resolution where all of the \( P_i \) are finitely generated as \( \mathbb Z G \)-modules.
    Finally, \( G \) is of \emph{type \( FP \)} if \( \mathbb Z \) has a projective resolution where all of the \( P_i \) are finitely generated as \( \mathbb Z G \)-modules, and the resolution is of finite length, so \( P_s = 0 \) for sufficiently large \( s \).
\end{definition}
\begin{example}
    \begin{enumerate}
        \item Let \( G = \langle t \rangle \) be the infinite cyclic group.
        Then \( G \) is of type \( FP \).
        \item Let \( G = \langle t \rangle \) be a finite cyclic group.
        Then \( G \) is of type \( FP_\infty \); we will show later that it is not of type \( FP \).
    \end{enumerate}
\end{example}
These can be regarded as finiteness conditions on the group \( G \).
The topological version of \( FP_n \) would be that a simplicial complex \( X \) with fundamental group \( G \) has a finite \( n \)-skeleton.

\subsection{???}
Consider a partial projective resolution
% https://q.uiver.app/#q=WzAsNyxbMCwwLCJQX3MiXSxbMSwwLCJQX3tzLTF9Il0sWzIsMCwiXFxjZG90cyJdLFszLDAsIlBfMSJdLFs0LDAsIlBfMCJdLFs1LDAsIlxcbWF0aGJiIFoiXSxbNiwwLCIwIl0sWzAsMV0sWzEsMl0sWzIsM10sWzMsNF0sWzQsNV0sWzUsNl1d
\[\begin{tikzcd}
	{P_s} & {P_{s-1}} & \cdots & {P_1} & {P_0} & {\mathbb Z} & 0
	\arrow[from=1-1, to=1-2]
	\arrow[from=1-2, to=1-3]
	\arrow[from=1-3, to=1-4]
	\arrow[from=1-4, to=1-5]
	\arrow[from=1-5, to=1-6]
	\arrow[from=1-6, to=1-7]
\end{tikzcd}\]
Then we can set \( P_{s+1} \) to be the free module \( \mathbb Z G\qty{X_{s+1}} \) where \( X_{s+1} \) is the kernel of \( d_s \).
We can then set \( d_{s+1} \) to be
\[ \underbrace{\sum r_x x}_{\in P_{s+1}} \mapsto \underbrace{\sum r_x x}_{\in P_s} \]
where the left-hand side is a formal sum, and the right-hand sum takes place in \( P_s \).
We thus obtain a longer partial projective resolution
% https://q.uiver.app/#q=WzAsOCxbMSwwLCJQX3MiXSxbMiwwLCJQX3tzLTF9Il0sWzMsMCwiXFxjZG90cyJdLFs0LDAsIlBfMSJdLFs1LDAsIlBfMCJdLFs2LDAsIlxcbWF0aGJiIFoiXSxbNywwLCIwIl0sWzAsMCwiUF97cysxfSJdLFswLDFdLFsxLDJdLFsyLDNdLFszLDRdLFs0LDVdLFs1LDZdLFs3LDAsImRfe3MrMX0iXV0=
\[\begin{tikzcd}
	{P_{s+1}} & {P_s} & {P_{s-1}} & \cdots & {P_1} & {P_0} & {\mathbb Z} & 0
	\arrow[from=1-2, to=1-3]
	\arrow[from=1-3, to=1-4]
	\arrow[from=1-4, to=1-5]
	\arrow[from=1-5, to=1-6]
	\arrow[from=1-6, to=1-7]
	\arrow[from=1-7, to=1-8]
	\arrow["{d_{s+1}}", from=1-1, to=1-2]
\end{tikzcd}\]
since exactness holds at \( P_s \) by construction.
We could alternatively take \( X_{s+1} \) to be a \( \mathbb Z G \)-generating set of \( \ker d_s \); this would have the effect of reducing the size of \( P_{s+1} \), which is most useful in direct calculation if \( \ker d_s \) is finitely generated.
Continuing in this way, we obtain a resolution of \( \mathbb Z \).
\begin{definition}
    The \emph{standard} or \emph{bar} resolution of \( \mathbb Z \) is constructed as follows.
    Let \( G^{(n)} \) be the set of formal symbols
    \[ G^{(n)} = \qty{[g_1 | \dots | g_n] \mid g_1, \dots, g_n \in G} \]
    where \( G^{(0)} \) is the set containing only the empty symbol \( [] \).
    Let \( F_n = \mathbb Z G \qty{G^{(n)}} \) be the corresponding free modules.
    We define the boundary maps \( d_n : F_n \to F_{n-1} \) by
    \begin{align*}
        d_n([g_1 | \dots | g_n]) &= g_1[g_2 | \dots | g_n] \\
        &- [g_1 g_2 | g_3 | \dots | g_n] \\
        &+ [g_1 | g_2 g_3 | \dots | g_n] - \dots \\
        &+ (-1)^{n-1} [g_1 | \dots | g_{n-1} g_n] \\
        &+ (-1)^n [g_1 | \dots | g_{n-1}]
    \end{align*}
    One can verify explicitly that there are chain maps as required, giving a free resolution
    % https://q.uiver.app/#q=WzAsNCxbMCwwLCJcXGNkb3RzIl0sWzEsMCwiRl8xIl0sWzIsMCwiRl8wIl0sWzMsMCwiXFxtYXRoYmIgWiJdLFswLDFdLFsxLDJdLFsyLDNdXQ==
\[\begin{tikzcd}
	\cdots & {F_1} & {F_0} & {\mathbb Z}
	\arrow[from=1-1, to=1-2]
	\arrow[from=1-2, to=1-3]
	\arrow[from=1-3, to=1-4]
\end{tikzcd}\]
\end{definition}
\begin{remark}
    The bar resolution corresponds to the standard resolution in algebraic topology.
    Consider the free abelian group \( \mathbb Z G^{n+1} \) generated by the \( (n + 1) \)-tuples with elements in \( G \).
    Then \( G \) acts on \( G^{n+1} \) diagonally:
    \[ g(g_0, \dots, g_n) = (gg_0, \dots, gg_n) \]
    Thus \( \mathbb Z G^{n+1} \) is a free \( \mathbb Z G \)-module on the basis of \( (n + 1) \)-tuples with first element \( 1 \).
    The symbol \( [g_1 | \dots | g_n] \) corresponds to the \( (n + 1) \)-tuple
    \[ (1, g_1, g_1 g_2, \dots, g_1 \dots g_n) \]
    Removing the first entry gives
    \[ g_1 (1, g_2, g_2 g_3, \dots, g_2 \dots g_n) \]
    and removing the second entry gives
    \[ (1, g_1 g_2, \dots, g_1 \dots g_n) \]
\end{remark}
\begin{lemma}
    The bar resolution is exact.
\end{lemma}
\begin{proof}
    We will just consider the \( d_n \) as maps of abelian groups.
    \( F_n \) has basis \( G \times G^{(n)} \) as a free abelian group.
    \[ G \times G^{(n)} = \qty{g_0 [g_1 | \dots | g_n] \mid g_0, \dots, g_n \in G} \]
    We define \( \mathbb Z \)-maps \( s_n : F_n \to F_{n+1} \) such that
    \[ \id_{F_n} = d_{n+1} s_n + s_{n-1} d_n \]
    by
    \[ s_n(g_0[g_1 | \dots | g_n]) = [g_0 | g_1 | \dots | g_n] \]
    This is not a \( \mathbb Z G \)-map.
    One can check that the required equation holds.
    If \( x \in \ker d_n \), then
    \[ x = \id x = d_{n+1} s_n(x) + s_{n-1} d_n(x) = d_{n+1} s_n(x) \in \im d_{n+1} \]
\end{proof}
\begin{corollary}
    Any finite group is of type \( FP_\infty \).
\end{corollary}
\begin{proof}
    The bar resolution gives a suitable resolution.
\end{proof}
