\subsection{???}
Recall that \( H^0(G, M) \), the group \( M^G \) of invariants of \( M \) under \( G \).
A derivation is a 1-cocycle, or equivalently a map \( \varphi : G \to M \) such that \( \varphi(g_1 g_2) = g_1 \varphi(g_2) + \varphi(g_1) \), and an inner derivation is a map of the form \( \varphi(g) = gm - m \).
We present two interpretations of (inner) derivations.

\emph{First interpretation.}
Consider possible \( \mathbb Z G \)-actions on the abelian group \( M \oplus \mathbb Z \) of the form \( g(m, n) = (gm + n \varphi(g), n) \).
Then
\[ g_1(g_2(m, n)) = g_1(g_2 m + n \varphi(g_2), n) = (g_1 g_2 m + n g_1 \varphi(g_2) + n \varphi(g_1), n) \]
and
\[ (g_1 g_2)(m, n) = (g_1 g_2 m + n \varphi(g_1 g_2), n) \]
For these to coincide, we must require \( \varphi(g_1 g_2) = g_1 \varphi(g_2) + \varphi(g_1) \), which is to say that \( \varphi \) is a derivation.
In particular, if \( M \) is a free \( \mathbb Z \)-module of finite rank, then we obtain a map
\[ g \mapsto \begin{pmatrix}
    \theta_1(g) & \varphi(g) \\
    0 & 1
\end{pmatrix} \]
where \( \theta_1(g) \) is a matrix corresponding to the action of \( g \) on \( M \).
This is a group homomorphism only if \( \varphi \) is a derivation.
One can check that \( \varphi \) is an inner derivation if \( (-m, 1) \) generates a \( \mathbb Z G \)-submodule of \( M \) which is the trivial module.
