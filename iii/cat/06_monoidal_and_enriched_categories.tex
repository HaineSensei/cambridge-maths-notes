\subsection{Monoidal categories}
There are many examples of categories \( \mathcal C \) equipped with a functor \( \otimes : \mathcal C \times \mathcal C \to \mathcal C \) and an object \( I \in \ob \mathcal C \) that turn \( \mathcal C \) into a monoid up to isomorphism.
Such a structure on a category is called a \emph{monoidal structure}, which will be defined precisely at the end of this subsection.
\begin{example}
    \begin{enumerate}
        \item Let \( \mathcal C \) be a category with finite products.
        Let \( \otimes \) be the categorical product \( \times \), and let \( I = 1 \) be the terminal object.
        This is known as the \emph{cartesian monoidal structure}.
        Dually, if \( \mathcal C \) is a category with finite coproducts, it has a \emph{cocartesian monoidal structure}, given by \( \otimes = + \) and \( I = 0 \).
        \item In \( \mathbf{Met} \), the different metrics on \( X \times Y \) yield different monoidal structures on \( \mathbf{Met} \).
        Each of these have the one-point space, which is the terminal object, as the unit of the monoid.
        \item In \( \mathbf{AbGp} \), the tensor product gives a monoidal structure, where \( \mathbb Z \) is the unit.
        Similarly, if \( R \) is a commutative ring, the tensor product \( \otimes_R \) gives a monoidal structure on \( \mathbf{Mod}_R \) with unit \( R \).
        \item For any category \( \mathcal C \), its category of endofunctors \( [\mathcal C, \mathcal C] \) has a monoidal structure given by composition.
        The unit is the identity endofunctor \( 1_{\mathcal C} \).
        \item For posets with top and bottom elements \( 1 \) and \( 0 \), we can define the \emph{ordinal sum} \( A \ast B \) to be the poset obtained from their disjoint union, by identifying the top element of \( A \) with the bottom element of \( B \).
        This is a monoidal structure, where the unit is the one-element poset.
    \end{enumerate}
\end{example}
\begin{definition}
    A \emph{monoidal category} is a category \( \mathcal C \) equipped with a functor \( \otimes : \mathcal C \times \mathcal C \to \mathcal C \) and a distinguished object \( I \), together with three natural isomorphisms
    \[ \alpha_{A, B, C} : (A \otimes B) \otimes C \to A \otimes (B \otimes C);\quad \lambda_A : I \otimes A \to A;\quad \rho_A : A \otimes I \to A \]
    such that the diagrams
    % https://q.uiver.app/#q=WzAsNSxbMCwwLCIoKEEgXFxvdGltZXMgQikgXFxvdGltZXMgQykgXFxvdGltZXMgRCJdLFswLDEsIihBIFxcb3RpbWVzIEIpIFxcb3RpbWVzIChDIFxcb3RpbWVzIEQpIl0sWzEsMiwiQSBcXG90aW1lcyAoQiBcXG90aW1lcyAoQyBcXG90aW1lcyBEKSkiXSxbMiwwLCIoQSBcXG90aW1lcyAoQiBcXG90aW1lcyBDKSkgXFxvdGltZXMgRCJdLFsyLDEsIkEgXFxvdGltZXMgKChCIFxcb3RpbWVzIEMpIFxcb3RpbWVzIEQpIl0sWzAsMSwiXFxhbHBoYV97QSBcXG90aW1lcyBCLCBDLCBEfSIsMl0sWzEsMiwiXFxhbHBoYV97QSxCLEMgXFxvdGltZXMgRH0iLDJdLFswLDMsIlxcYWxwaGFfe0FCQ30gXFxvdGltZXMgMV9EIl0sWzMsNCwiXFxhbHBoYV97QSxCIFxcb3RpbWVzIEMsRH0iXSxbNCwyLCIxX0EgXFxvdGltZXMgXFxhbHBoYV97QkNEfSJdXQ==
\[\begin{tikzcd}
	{((A \otimes B) \otimes C) \otimes D} && {(A \otimes (B \otimes C)) \otimes D} \\
	{(A \otimes B) \otimes (C \otimes D)} && {A \otimes ((B \otimes C) \otimes D)} \\
	& {A \otimes (B \otimes (C \otimes D))}
	\arrow["{\alpha_{A \otimes B, C, D}}"', from=1-1, to=2-1]
	\arrow["{\alpha_{A,B,C \otimes D}}"', from=2-1, to=3-2]
	\arrow["{\alpha_{ABC} \otimes 1_D}", from=1-1, to=1-3]
	\arrow["{\alpha_{A,B \otimes C,D}}", from=1-3, to=2-3]
	\arrow["{1_A \otimes \alpha_{BCD}}", from=2-3, to=3-2]
\end{tikzcd}\]
% https://q.uiver.app/#q=WzAsMyxbMCwwLCIoQSBcXG90aW1lcyBJKSBcXG90aW1lcyBCIl0sWzIsMCwiQSBcXG90aW1lcyAoSSBcXG90aW1lcyBCKSJdLFsxLDEsIkEgXFxvdGltZXMgQiJdLFswLDEsIlxcYWxwaGFfe0EsIEksIEJ9Il0sWzEsMiwiMV9BIFxcb3RpbWVzIFxcbGFtYmRhX0IiXSxbMCwyLCJcXHJob19BIFxcb3RpbWVzIDFfQiIsMl1d
\[\begin{tikzcd}
	{(A \otimes I) \otimes B} && {A \otimes (I \otimes B)} \\
	& {A \otimes B}
	\arrow["{\alpha_{A, I, B}}", from=1-1, to=1-3]
	\arrow["{1_A \otimes \lambda_B}", from=1-3, to=2-2]
	\arrow["{\rho_A \otimes 1_B}"', from=1-1, to=2-2]
\end{tikzcd}\]
    commute, and \( \lambda_I = \rho_I : I \otimes I \to I \).
    A monoidal category is \emph{strict} if \( \alpha, \lambda, \rho \) are identities.
\end{definition}
\( \alpha \) is called the \emph{associator}, and \( \lambda \) and \( \rho \) are the \emph{left} and \emph{right unitors}.

These diagrams suffice to prove the commutativity of the following two diagrams.
% https://q.uiver.app/#q=WzAsMyxbMCwwLCIoSSBcXG90aW1lcyBBKSBcXG90aW1lcyBCIl0sWzEsMCwiSSBcXG90aW1lcyAoQSBcXG90aW1lcyBCKSJdLFswLDEsIkEgXFxvdGltZXMgQiJdLFswLDEsIlxcYWxwaGFfe0ksQSxCfSJdLFswLDIsIlxcbGFtYmRhX0EgXFxvdGltZXMgMV9CIiwyXSxbMSwyLCJcXGxhbWJkYV97QSBcXG90aW1lcyBCfSJdXQ==
\[\begin{tikzcd}
	{(I \otimes A) \otimes B} & {I \otimes (A \otimes B)} \\
	{A \otimes B}
	\arrow["{\alpha_{I,A,B}}", from=1-1, to=1-2]
	\arrow["{\lambda_A \otimes 1_B}"', from=1-1, to=2-1]
	\arrow["{\lambda_{A \otimes B}}", from=1-2, to=2-1]
\end{tikzcd}\quad\quad\begin{tikzcd}
	{(A \otimes B) \otimes I} & {A \otimes (B \otimes I)} \\
	& {A \otimes B}
	\arrow["{\alpha_{A,B,I}}", from=1-1, to=1-2]
	\arrow["{1_A \otimes \rho_B}", from=1-2, to=2-2]
	\arrow["{\rho_{A \otimes B}}"', from=1-1, to=2-2]
\end{tikzcd}\]
Note that in the category of abelian groups with the usual tensor product, the obvious choice for \( \alpha_{A,B,C} \) is the map sending \( (a \otimes b) \otimes c \) to \( a \otimes (b \otimes c) \).
However, there is also a natural isomorphism sending \( (a \otimes b) \otimes c \) to \( -a \otimes (b \otimes c) \).
But this choice does not satisfy the pentagon equation, as a pentagon has an odd number of sides.

\subsection{The coherence theorem}
Given a monoidal category \( (\mathcal C, \otimes, I) \), we define a \emph{word} recursively.
\begin{enumerate}
    \item We have a stack of \emph{variables} \( A, B, C, \dots \), which are all words.
    \item The unit \( I \) is a word.
    \item If \( u, v \) are words, then \( u \otimes v \) is a word.
\end{enumerate}
A word with \( n \) variables defines a functor \( \mathcal C^n \to \mathcal C \).
\begin{theorem}[Mac Lane's coherence theorem]
    For any two words \( w, w' \) with the same sequence of variables in the same order, there is a unique natural isomorphism \( w \to w' \) obtained by composing instances of \( \alpha, \lambda, \rho \) and their inverses.
\end{theorem}
\begin{proof}
    We define the \emph{height} of a word \( w \) to be \( a(w) + i(w) \), where
    \begin{enumerate}
        \item \( a(w) \) is the \emph{associator height}, which is the number of closing parentheses occurring immediately before \( \otimes \) in \( w \);
        \item \( i(w) \) is the number of occurrences of \( I \) in \( w \).
    \end{enumerate}
    Applying any instance of \( \alpha, \lambda, \rho \) to a word reduces its height.
    For example, \( \alpha \dots : w \to w' \), then \( a(w') < a(w) \) and \( i(w') = i(w) \), and correspondingly if \( \lambda\dots w \to w' \), then \( i(w') = i(w) - 1 \) and \( a(w') \leq a(w) \).
    In particular, any string of instances of \( \alpha, \lambda, \rho \) starting from \( w \) has length at most \( a(w) + i(w) \).

    We say that a word \( w \) is \emph{reduced} if either \( a(w) = i(w) = 0 \) or \( w = I \).
    If \( a(w) > 0 \), then \( w \) is the domain of an instance of \( \alpha \), and if \( i(w) > 0 \) and \( w \neq I \), then \( w \) is the domain of an instance of either \( \lambda \) or \( \rho \).
    Thus, for any word \( w \), there is a string \( w \to \dots \to w_0 \) where \( w_0 \) is the unique reduced word containing the same variables of \( w \) in the same order.
    We must show that any two such strings have the same composite
    Given
    % https://q.uiver.app/#q=WzAsMyxbMSwwLCJ3Il0sWzAsMSwidyciXSxbMiwxLCJ3JyciXSxbMCwxLCJcXHZhcnBoaSIsMl0sWzAsMiwiXFxwc2kiXV0=
\[\begin{tikzcd}
	& w \\
	{w'} && {w''}
	\arrow["\varphi"', from=1-2, to=2-1]
	\arrow["\psi", from=1-2, to=2-3]
\end{tikzcd}\]
    where \( \varphi, \psi \) are instances of \( \alpha, \lambda \), or \( \rho \), we need to find a word \( w''' \) completing the commutative square
    % https://q.uiver.app/#q=WzAsNCxbMSwwLCJ3Il0sWzAsMSwidyciXSxbMiwxLCJ3JyciXSxbMSwyLCJ3JycnIl0sWzAsMSwiXFx2YXJwaGkiLDJdLFswLDIsIlxccHNpIl0sWzEsMywiXFx0aGV0YSIsMl0sWzIsMywiXFxjaGkiXV0=
\[\begin{tikzcd}
	& w \\
	{w'} && {w''} \\
	& {w'''}
	\arrow["\varphi"', from=1-2, to=2-1]
	\arrow["\psi", from=1-2, to=2-3]
	\arrow["\theta"', from=2-1, to=3-2]
	\arrow["\chi", from=2-3, to=3-2]
\end{tikzcd}\]
    where \( \theta, \chi \) are composites of instances of \( \alpha, \lambda \), and \( \rho \).

    If \( \varphi, \psi \) act on disjoint subwords of \( w \), so \( w = u \otimes v \) where \( \varphi = \varphi' \otimes 1_v \) and \( \psi = 1_u \otimes \psi' \), then we can fill in the square as follows.
    % https://q.uiver.app/#q=WzAsNCxbMSwwLCJ1IFxcb3RpbWVzIHYiXSxbMCwxLCJ1JyBcXG90aW1lcyB2Il0sWzIsMSwidSBcXG90aW1lcyB2JyJdLFsxLDIsInUnIFxcb3RpbWVzIHYnIl0sWzAsMSwiXFx2YXJwaGknIFxcb3RpbWVzIDFfdiIsMl0sWzAsMiwiMV91IFxcb3RpbWVzIFxccHNpJyJdLFsxLDMsIjFfe3UnfSBcXG90aW1lcyBcXHBzaSciLDJdLFsyLDMsIlxcdmFycGhpJyBcXG90aW1lcyAxX3t2J30iXV0=
\[\begin{tikzcd}
	& {u \otimes v} \\
	{u' \otimes v} && {u \otimes v'} \\
	& {u' \otimes v'}
	\arrow["{\varphi' \otimes 1_v}"', from=1-2, to=2-1]
	\arrow["{1_u \otimes \psi'}", from=1-2, to=2-3]
	\arrow["{1_{u'} \otimes \psi'}"', from=2-1, to=3-2]
	\arrow["{\varphi' \otimes 1_{v'}}", from=2-3, to=3-2]
\end{tikzcd}\]
    Now suppose one acts within the argument of the other, for example, if \( \varphi \) is \( \alpha_{t, u, v} \) and \( \psi = (1_t \otimes \psi') \otimes 1_v \).
    Then by naturality of \( \alpha \), we can complete the diagram with \( 1_t \otimes (\psi' \otimes 1_v) \) and \( \alpha_{t, u', v} \).

    Now suppose that \( \varphi \) and \( \psi \) interfere.
    If \( \varphi \) and \( \psi \) are both instances of \( \alpha \), then the pentagon equation completes the commutative square.

    Suppose one is an instance of \( \alpha \) and the other is an instance of \( \lambda \) or \( \rho \).
    Then \( I \) must occur as one of the three arguments to \( \alpha \).
    If it is the middle argument, the two diagrams in the definition of a monoidal category complete the square.
    If if is the left or right argument, the other two diagrams defined immediately after will complete the square.

    Finally, if one is an instance of \( \lambda \) and the other is an instance of \( \rho \), then they must be \( \lambda_I \) and \( \rho_I \), and so must agree.
    This completes the proof that there is a unique natural isomorphism to a reduced word.

    Now suppose we have a string
    % https://q.uiver.app/#q=WzAsNixbMCwwLCJ3XzEiXSxbMSwwLCJ3XzIiXSxbMiwwLCJ3XzMiXSxbMywwLCJ3XzQiXSxbNSwwLCJ3X24iXSxbNCwwLCJcXGNkb3RzIl0sWzAsMV0sWzIsM10sWzIsMV1d
\[\begin{tikzcd}
	{w_1} & {w_2} & {w_3} & {w_4} & \cdots & {w_n}
	\arrow[from=1-1, to=1-2]
	\arrow[from=1-3, to=1-4]
	\arrow[from=1-3, to=1-2]
\end{tikzcd}\]
    Then there are unique `forwards' morphisms
    % https://q.uiver.app/#q=WzAsNyxbMCwwLCJ3XzEiXSxbMSwwLCJ3XzIiXSxbMiwwLCJ3XzMiXSxbMywwLCJ3XzQiXSxbNSwwLCJ3X24iXSxbNCwwLCJcXGNkb3RzIl0sWzIsMiwid18wIl0sWzAsMV0sWzIsM10sWzAsNl0sWzEsNl0sWzIsNl0sWzMsNl0sWzQsNl0sWzIsMV1d
\[\begin{tikzcd}
	{w_1} & {w_2} & {w_3} & {w_4} & \cdots & {w_n} \\
	\\
	&& {w_0}
	\arrow[from=1-1, to=1-2]
	\arrow[from=1-3, to=1-4]
	\arrow[from=1-1, to=3-3]
	\arrow[from=1-2, to=3-3]
	\arrow[from=1-3, to=3-3]
	\arrow[from=1-4, to=3-3]
	\arrow[from=1-6, to=3-3]
	\arrow[from=1-3, to=1-2]
\end{tikzcd}\]
    to \( w_0 \), which is the reduced word with the same sequence of variables.
    Each of the triangles must commute by the uniqueness result proven above.
    Hence the composite of the arrows along the top edge is equal to the composite \( w_1 \rightarrow w_0 \leftarrow w_n \).
\end{proof}

\begin{definition}
    A \emph{symmetry} on a monoidal category \( (\mathcal C, \otimes, I) \) is a natural isomorphism \( \gamma_{A, B} : A \otimes B \to B \otimes A \) such that the following diagrams commute.
    % https://q.uiver.app/#q=WzAsNixbMCwwLCIoQSBcXG90aW1lcyBCKSBcXG90aW1lcyBDIl0sWzEsMCwiQSBcXG90aW1lcyAoQiBcXG90aW1lcyBDKSJdLFsyLDAsIkEgXFxvdGltZXMgKEMgXFxvdGltZXMgQikiXSxbMiwxLCIoQSBcXG90aW1lcyBDKSBcXG90aW1lcyBCIl0sWzEsMSwiKEMgXFxvdGltZXMgQSkgXFxvdGltZXMgQiJdLFswLDEsIkMgXFxvdGltZXMgKEEgXFxvdGltZXMgQikiXSxbMCwxLCJcXGFscGhhX3tBLEIsQ30iXSxbMSwyLCIxX0EgXFxvdGltZXMgXFxnYW1tYV97QixDfSJdLFsyLDMsIlxcYWxwaGFfe0EsQyxCfV57LTF9Il0sWzMsNCwiXFxnYW1tYV97QSxDfSBcXG90aW1lcyAxX0IiXSxbMCw1LCJcXGdhbW1hX3tBIFxcb3RpbWVzIEIsIEN9IiwyXSxbNCw1LCJcXGFscGhhX3tDLEEsQn0iXV0=
\[\begin{tikzcd}
	{(A \otimes B) \otimes C} & {A \otimes (B \otimes C)} & {A \otimes (C \otimes B)} \\
	{C \otimes (A \otimes B)} & {(C \otimes A) \otimes B} & {(A \otimes C) \otimes B}
	\arrow["{\alpha_{A,B,C}}", from=1-1, to=1-2]
	\arrow["{1_A \otimes \gamma_{B,C}}", from=1-2, to=1-3]
	\arrow["{\alpha_{A,C,B}^{-1}}", from=1-3, to=2-3]
	\arrow["{\gamma_{A,C} \otimes 1_B}", from=2-3, to=2-2]
	\arrow["{\gamma_{A \otimes B, C}}"', from=1-1, to=2-1]
	\arrow["{\alpha_{C,A,B}}", from=2-2, to=2-1]
\end{tikzcd}\]
    % https://q.uiver.app/#q=WzAsMyxbMCwwLCJBIFxcb3RpbWVzIEkiXSxbMiwwLCJJIFxcb3RpbWVzIEEiXSxbMSwxLCJBIl0sWzAsMSwiXFxnYW1tYV97QSxJfSJdLFsxLDIsIlxcbGFtYmRhX0EiXSxbMCwyLCJcXHJob19BIiwyXV0=
\[\begin{tikzcd}
	{A \otimes I} && {I \otimes A} \\
	& A
	\arrow["{\gamma_{A,I}}", from=1-1, to=1-3]
	\arrow["{\lambda_A}", from=1-3, to=2-2]
	\arrow["{\rho_A}"', from=1-1, to=2-2]
\end{tikzcd}\quad\quad\begin{tikzcd}
	{A \otimes B} & {B \otimes A} \\
	& {A \otimes B}
	\arrow["{\gamma_{A,B}}", from=1-1, to=1-2]
	\arrow["{\gamma_{B,A}}", from=1-2, to=2-2]
	\arrow["{1_{A \otimes B}}"', from=1-1, to=2-2]
\end{tikzcd}\]
\end{definition}
For the weaker notion of a \emph{braiding}, we can omit the last of the three diagrams, but add an additional hexagonal equation, since it can no longer be derived from the first.

There is a coherence theorem for symmetric monoidal categories, which is also due to Mac Lane.
The theorem shows that for any two words \( w, w' \) involving the same set of variables without repetition, there is a unique natural isomorphism between \( w \) and \( w' \) obtained from compositions of instances of \( \alpha, \lambda, \gamma \) and their inverses.
Note that \( \rho \) is not necessary, as it can be produced from instances of \( \lambda \) and \( \gamma \).
The examples of monoidal categories above are all symmetric, except for (iv) and (v).

\subsection{Monoidal functors}
\begin{definition}
    Let \( (\mathcal C, \otimes, I), (\mathcal D, \oplus, J) \) be monoidal categories.
    A \emph{(lax) monoidal functor} \( F : (\mathcal C, \otimes, I) \to (\mathcal D, \oplus, J) \) is a functor \( F: \mathcal C \to \mathcal D \) equipped with a natural transformation \( \varphi_{A, B} : FA \oplus FB \to F(A \otimes B) \) and a morphism \( \iota : J \to FI \), such that the following diagrams commute.
    % https://q.uiver.app/#q=WzAsNixbMCwwLCIoRkEgXFxvcGx1cyBGQikgXFxvcGx1cyBGQyJdLFswLDEsIkZBIFxcb3BsdXMgKEZCIFxcb3BsdXMgRkMpIl0sWzEsMCwiRihBIFxcb3RpbWVzIEIpIFxcb3BsdXMgRkMiXSxbMiwwLCJGKChBIFxcb3RpbWVzIEIpIFxcb3RpbWVzIEMpIl0sWzIsMSwiRihBIFxcb3RpbWVzIChCIFxcb3RpbWVzIEMpKSJdLFsxLDEsIkZBIFxcb3BsdXMgRihCIFxcb3RpbWVzIEMpIl0sWzAsMSwiXFxhbHBoYV97RkEsRkIsIEZDfSIsMl0sWzAsMiwiXFx2YXJwaGlfe0EsQn0gXFxvcGx1cyAxX3tGQ30iXSxbMiwzLCJcXHZhcnBoaV97QSBcXG90aW1lcyBCLCBDfSJdLFszLDQsIkZcXGFscGhhX3tBLEIsQ30iXSxbMSw1LCIxX3tGQX0gXFxvcGx1cyBcXHZhcnBoaV97QixDfSIsMl0sWzUsNCwiXFx2YXJwaGlfe0EsQiBcXG90aW1lcyBDfSIsMl1d
\[\begin{tikzcd}
	{(FA \oplus FB) \oplus FC} & {F(A \otimes B) \oplus FC} & {F((A \otimes B) \otimes C)} \\
	{FA \oplus (FB \oplus FC)} & {FA \oplus F(B \otimes C)} & {F(A \otimes (B \otimes C))}
	\arrow["{\alpha_{FA,FB, FC}}"', from=1-1, to=2-1]
	\arrow["{\varphi_{A,B} \oplus 1_{FC}}", from=1-1, to=1-2]
	\arrow["{\varphi_{A \otimes B, C}}", from=1-2, to=1-3]
	\arrow["{F\alpha_{A,B,C}}", from=1-3, to=2-3]
	\arrow["{1_{FA} \oplus \varphi_{B,C}}"', from=2-1, to=2-2]
	\arrow["{\varphi_{A,B \otimes C}}"', from=2-2, to=2-3]
\end{tikzcd}\]
% https://q.uiver.app/#q=WzAsNCxbMCwwLCJKIFxcb3BsdXMgRkEiXSxbMSwwLCJGSSBcXG9wbHVzIEZBIl0sWzEsMSwiRihJIFxcb3RpbWVzIEEpIl0sWzAsMSwiRkEiXSxbMCwxLCJcXGlvdGEgXFxvcGx1cyAxX3tGQX0iXSxbMSwyLCJcXHZhcnBoaV97SSxBfSJdLFswLDMsIlxcbGFtYmRhX3tGQX0iLDJdLFsyLDMsIkZcXGxhbWJkYV9BIl1d
\[\begin{tikzcd}
	{J \oplus FA} & {FI \oplus FA} \\
	FA & {F(I \otimes A)}
	\arrow["{\iota \oplus 1_{FA}}", from=1-1, to=1-2]
	\arrow["{\varphi_{I,A}}", from=1-2, to=2-2]
	\arrow["{\lambda_{FA}}"', from=1-1, to=2-1]
	\arrow["{F\lambda_A}", from=2-2, to=2-1]
\end{tikzcd}\quad\quad\begin{tikzcd}
	{FA \oplus J} & {FA \oplus FI} \\
	FA & {F(A \otimes I)}
	\arrow["{1_{FA} \oplus \iota}", from=1-1, to=1-2]
	\arrow["{\varphi_{A, I}}", from=1-2, to=2-2]
	\arrow["{\rho_{FA}}"', from=1-1, to=2-1]
	\arrow["{F\rho_A}", from=2-2, to=2-1]
\end{tikzcd}\]
    We say \( F \) is \emph{strong monoidal} (respectively \emph{strict monoidal}) if \( \varphi \) and \( \iota \) are isomorphisms (respectively identities).
    An \emph{oplax} monoidal functor is the same definition, but where the directions of the maps \( \varphi \) and \( \iota \) are reversed.
\end{definition}
Note that the same letters are used for the associators and unitors in both monoidal categories.
\begin{example}
    \begin{enumerate}
        \item The forgetful functor \( U : (\mathbf{AbGp}, \otimes, \mathbb Z) \to (\mathbf{Set}, \times, 1) \) is lax monoidal.
        We define \( \iota : 1 \to \mathbb Z \) to map the element of \( 1 \) to the generator \( 1 \in \mathbb Z \), and define \( \varphi : UA \times UB \to U(A \otimes B) \) by \( (a, b) \mapsto a \otimes b \).
        One can easily verify that the required diagrams commute.
        \item The free functor \( F : (\mathbf{Set}, \times, 1) \to (\mathbf{AbGp}, \otimes, \mathbb Z) \) is strong monoidal, because \( F1 \cong \mathbb Z \) and \( F(A \times B) \cong FA \otimes FB \).
        \item Let \( R \) be a commutative ring.
        Then the forgetful functor \( \mathbf{Mod}_R \to \mathbf{AbGp} \) is lax monoidal, where \( \iota : \mathbb Z \to R \) is the natural map, and \( \varphi : A \otimes_{\mathbb Z} B \to A \otimes_R B \) is the quotient map.
        Its left adjoint, the free functor \( \mathbf{AbGp} \to \mathbf{Mod}_R \), is strong monoidal.
        \item If \( \mathcal C \) and \( \mathcal D \) have the cartesian monoidal structure, then any functor \( F : \mathcal C \to \mathcal D \) is oplax monoidal.
        \( \iota : F1 \to 1 \) is the unique morphism to the terminal object of \( \mathcal D \), and \( \varphi_{A,B} : F(A \times B) \to FA \times FB \) is given by \( (F\pi_1, F\pi_2) \).
        \( F \) is strong monoidal if and only if it preserves finite products.
        \item If \( X \) and \( Y \) are metric spaces, then \( 1_{X \times Y} \) is non-expansive as a map \( (X \times Y, d_1) \to (X \times Y, d_\infty) \), making the identity functor \( 1_{\mathbf{Met}} \) into a monoidal functor \( (\mathbf{Met}, \times_\infty, 1) \to (\mathbf{Met}, \times_1, 1) \).
        Note that the \( d_\infty \) metric on \( X \times Y \) defines the categorical product.
    \end{enumerate}
\end{example}
\begin{lemma}
    Let \( \mathcal C \) and \( \mathcal D \) be monoidal categories.
    Let \( F \dashv G \), where \( F : \mathcal C \to \mathcal D \) and \( G : \mathcal D \to \mathcal C \).
    Then there is a bijection between lax monoidal structures on \( G \) and oplax monoidal structures on \( F \).
\end{lemma}
\begin{proof}[Proof sketch]
    Suppose we have \( (\varphi, \iota) \) on \( G \).
    Then the transpose of \( \iota : J \to GI \) is a morphism \( FJ \to I \), and we have a natural transformation
    % https://q.uiver.app/#q=WzAsNCxbMCwwLCJGKEEgXFxvdGltZXMgQikiXSxbMSwwLCJGKEdGQSBcXG90aW1lcyBHRkIpIl0sWzIsMCwiRkcoRkEgXFxvcGx1cyBGQikiXSxbMywwLCJGQSBcXG9wbHVzIEZCIl0sWzAsMSwiRihcXGV0YV9BIFxcdGltZXMgXFxldGFfQikiXSxbMSwyLCJGIFxcdmFycGhpX3tGQSwgRkJ9Il0sWzIsMywiXFxlcHNpbG9uX3tGQSBcXG9wbHVzIEZCfSJdXQ==
\[\begin{tikzcd}[column sep=large]
	{F(A \otimes B)} & {F(GFA \otimes GFB)} & {FG(FA \oplus FB)} & {FA \oplus FB}
	\arrow["{F(\eta_A \times \eta_B)}", from=1-1, to=1-2]
	\arrow["{F \varphi_{FA, FB}}", from=1-2, to=1-3]
	\arrow["{\epsilon_{FA \oplus FB}}", from=1-3, to=1-4]
\end{tikzcd}\]
    One can check that each of the required diagrams commute, defining an oplax monoidal structure on \( F \).
    By duality, an oplax monoidal structure on \( F \) yields a lax monoidal structure on \( G \), and it can be shown that these constructions are inverse to each other.
\end{proof}

\subsection{Closed monoidal categories}
\begin{definition}
    We say that a monoidal category \( (\mathcal C, \otimes, I) \) is (\emph{left/right/bi})-\emph{closed} if \( A \otimes(-), (-) \otimes A \), or both have right adjoints for all \( A \).
    If \( \otimes \) is symmetric, we say in any of these cases that \( \mathcal C \) is \emph{closed}.
\end{definition}
Right adjoints for \( (-) \otimes A \) are denoted \( [A, -] \) if they exist.
\begin{example}
    \begin{enumerate}
        \item A cartesian closed category is a monoidal category with \( \otimes = \times \), that is closed as a monoidal category.
        In particular, \( \mathbf{Set} \) and \( \mathbf{Cat} \) are cartesian closed.
        \item The metric \( d_1 \) on the set \( [X, Y] \) of non-expansive maps \( X \to Y \) yields a closed structure on \( (\mathbf{Met}, \times_1, 1) \).
        \item \( \mathbf{AbGp} \) and \( \mathbf{Mod}_R \) for any commutative ring \( R \) are monoidal closed, where \( [A, B] \) is the set of homomorphisms \( A \to B \), turned into an abelian group or \( R \)-module by pointwise addition and scalar multiplication.
        The homomorphisms \( C \to [A,B] \) correspond under \( \lambda \)-conversion to bilinear maps \( C \times A \to B \), and thus to homomorphisms \( C \otimes_R A \to B \).
        \item The cartesian monoidal structure on the category of pointed sets \( \mathbf{Set}_\star \) is not closed, but the monoidal structure given by the \emph{smash product} \( (-) \wedge (-) \) is closed, where
        \[ (A, a_0) \wedge (B, b_0) = \faktor{A \times B}{\sim} \]
        and \( \sim \) identifies all elements where either coordinate is the basepoint.
        Basepoint-preserving maps \( A \wedge B \to C \) correspond to basepoint-preserving maps from \( A \) to the set \( [B,C] \) of basepoint-preserving maps \( B \to C \).
        \item Consider the set \( \operatorname{Rel}(A \times A) = P(A \times A) \) of relations on \( A \).
        This is a poset under inclusion, and is a monoid under relational composition.
        Composition is order-preserving in each variable, making \( \operatorname{Rel}(A \times A) \) into a strict monoidal category.
        It is not symmetric, but biclosed.
        For the right adjoint to \( (-) \circ R \), we define \( R \Rightarrow T \) to be
        \[ (R \Rightarrow T) = \qty{(b, c) \in A \times A \mid \forall a \in A,\, (a, b) \in R \Rightarrow (a, c) \in T} \]
        Then \( S \subseteq (R \Rightarrow T) \) if and only if \( S \circ R \subseteq T \).
    \end{enumerate}
\end{example}
