\subsection{???}
There are many examples of categories \( \mathcal C \) equipped with a functor \( \otimes : \mathcal C \times \mathcal C \to \mathcal C \) and an object \( I \in \ob \mathcal C \) that turn \( \mathcal C \) into a monoid up to isomorphism.
Such a structure on a category is called a \emph{monoidal structure}, which will be defined precisely at the end of this subsection.
\begin{example}
    \begin{enumerate}
        \item Let \( \mathcal C \) be a category with finite products.
        Let \( \otimes \) be the categorical product \( \times \), and let \( I = 1 \) be the terminal object.
        This is known as the \emph{cartesian monoidal structure}.
        Dually, if \( \mathcal C \) is a category with finite coproducts, it has a \emph{cocartesian monoidal structure}, given by \( \otimes = + \) and \( I = 0 \).
        \item In \( \mathbf{Met} \), the different metrics on \( X \times Y \) yield different monoidal structures on \( \mathbf{Met} \).
        Each of these have the one-point space, which is the terminal object, as the unit of the monoid.
        \item In \( \mathbf{AbGp} \), the tensor product gives a monoidal structure, where \( \mathbb Z \) is the unit.
        Similarly, if \( R \) is a commutative ring, the tensor product \( \otimes_R \) gives a monoidal structure on \( \mathbf{Mod}_R \) with unit \( R \).
        \item For any category \( \mathcal C \), its category of endofunctors \( [\mathcal C, \mathcal C] \) has a monoidal structure given by composition.
        The unit is the identity endofunctor \( 1_{\mathcal C} \).
        \item For posets with top and bottom elements \( 1 \) and \( 0 \), we can define the \emph{ordinal sum} \( A \ast B \) to be the poset obtained from their disjoint union, by identifying the top element of \( A \) with the bottom element of \( B \).
        This is a monoidal structure, where the unit is the one-element poset.
    \end{enumerate}
\end{example}
