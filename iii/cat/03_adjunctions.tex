\subsection{Definition and examples}
\begin{definition}
    Let \( \mathcal C, \mathcal D \) be categories.
    An \emph{adjunction} between \( \mathcal C \) and \( \mathcal D \) is a pair of functors \( F : \mathcal C \to \mathcal D \) and \( G : \mathcal D \to \mathcal C \), together with a bijection between morphisms \( FA \to B \) in \( \mathcal D \) and \( A \to HB \) in \( \mathcal C \), which is natural in both variables \( A, B \).
    We say that \( F \) is the \emph{left adjoint} to \( G \), and that \( G \) is the \emph{right adjoint} to \( F \), and write \( F \dashv G \).
\end{definition}
If \( \mathcal C, \mathcal D \) are locally small, then the naturality condition is that
\[ \mathcal D(F-, -);\quad \mathcal C(-, G-) \]
are naturally isomorphic functors \( \mathcal C^\cop \times \mathcal D \to \mathbf{Set} \).
\begin{example}
    \begin{enumerate}
        \item The free group functor \( F : \mathbf{Set} \to \mathbf{Gp} \) is left adjoint to the forgetful functor \( U : \mathbf{Gp} \to \mathbf{Set} \).
        \[ \mathbf{Gp}(FA, G) \leftrightarrow \mathbf{Set}(A, UG) \]
        \item The forgetful functor \( U : \mathbf{Top} \to \mathbf{Set} \) has a left adjoint \( D : \mathbf{Set} \to \mathbf{Top} \) which equips each set with its discrete topology.
        \[ \mathbf{Top}(DX, Y) \leftrightarrow \mathbf{Set}(X, UY) \]
        It also has a right adjoint \( I : \mathbf{Set} \to \mathbf{Top} \) which equips each set with its indiscrete topology.
        \[ \mathbf{Set}(UX, Y) \leftrightarrow \mathbf{Top}(X, IY) \]
        \item Consider the functor \( \ob : \mathbf{Cat} \to \mathbf{Set} \) which maps each category to each set of objects.
        It has a left adjoint \( D \) which turns each set \( X \) into a discrete category in which the objects are elements of \( X \), and the only morphisms are identities.
        It also has a right adjoint \( I \) which turns each set \( X \) into an indiscrete category in which the objects are elements of \( X \), and there is exactly one morphism between any two elements of \( X \).
        In addition, \( D : \mathbf{Set} \to \mathbf{Cat} \) has a left adjoint \( \pi_0 : \mathbf{Cat} \to \mathbf{Set} \), where \( \pi_0 \mathcal C \) is the set of connected components of \( \ob \mathcal C \) under the graph induced by its morphisms.
        \[ \mathbf{Set}(\pi_0 \mathcal C, X) \leftrightarrow \mathbf{Cat}(\mathcal C, DX);\quad \mathbf{Cat}(D X, \mathcal C) \leftrightarrow \mathbf{Set}(X, \ob \mathcal C);\quad \mathbf{Set}(\ob \mathcal C, X) \leftrightarrow \mathbf{Cat}(\mathcal C, IX) \]
        Thus we have a chain
        \[ \pi_0 \dashv D \dashv \ob \dashv I \]
        \item For any set \( A \), we have a functor \( (-) \times A : \mathbf{Set} \to \mathbf{Set} \).
        This functor has a right adjoint, which is the functor \( \mathbf{Set}(A, -) : \mathbf{Set} \to \mathbf{Set} \).
        \[ \mathbf{Set}(B \times A, C) \leftrightarrow \mathbf{Set}(B, \mathbf{Set}(A, C)) \]
        Applying this bijection is sometimes called \emph{currying} or \emph{\( \lambda \)-conversion}.
        We say that a category \( \mathcal C \) with binary products is \emph{cartesian closed} if \( (-) \times A : \mathcal C \to \mathcal C \) has a right adjoint, written \( [A, -] \) or \( (-)^A \), for each \( A \).
        For example, \( \mathbf{Cat} \) is cartesian closed, where \( \mathcal D^{\mathcal C} = [\mathcal C, \mathcal D] \) is the functor category that this notation already refers to.
        \item An equivalence \( F : \mathcal C \to \mathcal D \), \( G : \mathcal D \to \mathcal C \) forms adjunctions both ways: \( F \dashv G, G \dashv F \).
        \item Let \( \mathbf{Idem} \) be the category of pairs \( (A, e) \) where \( A \) is a set and \( e \) is an idempotent endomorphism \( A \to A \).
        The morphisms in \( \mathbf{Idem} \) are the maps of sets which commute with the idempotents.
        We have a functor \( F : \mathbf{Set} \to \mathbf{Idem} \) sending \( A \) to \( (A, 1_A) \).
        Consider \( G : \mathbf{Idem} \to \mathbf{Set} \) sending \( (A, e) \) to the set of fixed points of \( e \).
        Then \( F \dashv G \) since any morphism \( FA \to (B, e) \) takes values in \( G(B, e) \).
        But also \( G \dashv F \), since a morphism \( (A, e) \to FB \) is entirely determined by its action on the fixed points in \( A \) under \( e \), because \( f(a) = f(ea) \).
        This is not an equivalence of categories, because \( G \) is not faithful.
        So not all pairs of functors that are adjoint in both directions form an equivalence.
        \item Let \( \mathcal C \) be a category.
        There is a unique functor \( G : \mathcal C \to \mathbf 1 \), where \( \mathbf 1 \) is the discrete category on a single object.
        A left adjoint for \( G \), if it exists, sends the object in \( \mathbf 1 \) to an \emph{initial object} \( I \) of \( \mathcal C \), which is an object with a unique morphism to every object in \( \mathcal C \).
        Dually, a right adjoint sends the object in \( \mathbf 1 \) to a \emph{terminal object} \( T \), which is an object with a unique morphism from every object in \( \mathcal C \).
        In \( \mathbf{Set} \), the empty set is initial, and any singleton is terminal.
        In \( \mathbf{Gp} \), the trivial group is initial and terminal.
        \item Let \( f : A \to B \) be a function of sets, and let \( A' \subseteq A, B' \subseteq B \).
        Then \( Pf(A') \subseteq B' \) if and only if \( A' \subseteq P^\star f(B') \).
        Thus \( Pf \dashv P^\star f \) as functors between \( PA \) and \( PB \) as posets.
        \item Let \( A, B \) be sets with a relation \( R \subseteq A \times B \).
        We define mappings \( (-)^r : PA \to PB \) by
        \[ S^r = \qty{b \in B \mid \forall a \in S,\, (a, b) \in R} \]
        and \( (-)^\ell : PB \to PA \) by
        \[ T^\ell = \qty{a \in A \mid \forall b \in T,\, (a, b) \in R} \]
        These are contravariant functors, and
        \[ S \subseteq T^\ell \iff S \times T \subseteq R \iff T \subseteq S^r \]
        We say that \( (-)^\ell \) and \( (-)^r \) are \emph{adjoint on the right}.
        This pair is called a \emph{Galois connection}.
        \item The contravariant power-set functor \( P^\star \) is self-adjoint on the right, since functions \( A \to P^\star B \) and \( B \to P^\star A \) naturally correspond bijectively to subsets of \( A \times B \).
        \item The dual vector space functor \( (-)^{\star\star} : \mathbf{Vect}_k \to \mathbf{Vect}_k \) is self-adjoint on the right, as linear maps \( V \to W^\star \) and linear maps \( W \to V^\star \) both naturally correspond to bilinear forms on \( V \times W \).
    \end{enumerate}
\end{example}

\subsection{???}
\begin{theorem}
    Let \( G : \mathcal D \to \mathcal C \) be a functor and \( A \in \ob \mathcal C \).
    Write \( (A \downarrow G) \) for the category whose objects are pairs \( (B, f) \) where \( B \in \ob \mathcal D \) and \( f : A \to GB \) in \( \mathcal C \), and whose morphisms \( (B, f) \to (B', f') \) are morphisms \( g : B \to B' \) which commute with \( f, f' \):
    % https://q.uiver.app/#q=WzAsMyxbMCwwLCJBIl0sWzEsMCwiR0IiXSxbMSwxLCJHQiciXSxbMCwxLCJmIl0sWzEsMiwiR2ciXSxbMCwyLCJmJyIsMl1d
\[\begin{tikzcd}
	A & GB \\
	& {GB'}
	\arrow["f", from=1-1, to=1-2]
	\arrow["Gg", from=1-2, to=2-2]
	\arrow["{f'}"', from=1-1, to=2-2]
\end{tikzcd}\]
    Then specifying a left adjoint for \( G \) is equivalent to specifying an initial object of \( (A \downarrow G) \) for each \( A \).
\end{theorem}
The category \( (A \downarrow G) \) is sometimes called a \emph{comma category}.
\begin{proof}
    Suppose \( F \dashv G \).
    Then let \( \eta_A : A \to GFA \) correspond to the identity \( 1_{FA} \) under the adjunction.
    We show that \( (FA, \eta_A) \) is initial in \( (A \downarrow G) \).
    Indeed, given \( f : A \to GB \), then
    % https://q.uiver.app/#q=WzAsMyxbMCwwLCJBIl0sWzEsMCwiR0ZBIl0sWzEsMSwiR0IiXSxbMCwxLCJcXGV0YV9BIl0sWzEsMiwiR2ciXSxbMCwyLCJmIiwyXV0=
\[\begin{tikzcd}
	A & GFA \\
	& GB
	\arrow["{\eta_A}", from=1-1, to=1-2]
	\arrow["Gg", from=1-2, to=2-2]
	\arrow["f"', from=1-1, to=2-2]
\end{tikzcd}\]
    commutes if and only if \( g \) is the morphism corresponding to \( f \) under the adjunction.
    In particular, for any \( f \), there is a unique such \( g \).

    Conversely, suppose \( (FA, \eta_A) \) is initial in \( (A \downarrow G) \) for each \( A \).
    Then we define the action of \( F \) on objects by mapping \( A \) to \( FA \).
    We make \( F \) into a functor by mapping \( f : A \to A' \) to the unique morphism that makes the following square commute.
    % https://q.uiver.app/#q=WzAsNCxbMCwwLCJBIl0sWzEsMCwiR0ZBIl0sWzEsMSwiR0ZBJyJdLFswLDEsIkEnIl0sWzAsMSwiXFxldGFfQSJdLFsxLDIsIkdGZiJdLFswLDMsImYiLDJdLFszLDIsIlxcZXRhX3tBJ30iLDJdXQ==
\[\begin{tikzcd}
	A & GFA \\
	{A'} & {GFA'}
	\arrow["{\eta_A}", from=1-1, to=1-2]
	\arrow["GFf", from=1-2, to=2-2]
	\arrow["f"', from=1-1, to=2-1]
	\arrow["{\eta_{A'}}"', from=2-1, to=2-2]
\end{tikzcd}\]
    Functoriality of \( F \) follows from the uniqueness of \( Ff \).
    The bijection between morphisms \( f : A \to GB \) and \( g : FA \to B \) sends \( f \) to the unique \( g \) giving \( (Gg)\eta_A = f \).
    Naturality of the bijection in \( A \) was built in to the definition of \( F \) as a functor, and naturality in \( B \) is easy.
\end{proof}
\begin{corollary}
    Let \( F, F' : \mathcal C \to \mathcal D \) be left adjoints to \( G : \mathcal D \to \mathcal C \).
    Then \( F \simeq F' \) in \( [\mathcal C, \mathcal D] \).
\end{corollary}
\begin{proof}
    \( (FA, \eta_A) \) and \( (F'A, \eta'_A) \) are both initial objects in \( (A \downarrow G) \), and so there is a unique isomorphism \( \alpha_A : (FA, \eta_A) \to (F'A, \eta'_A) \) in this category.
    The map \( A \mapsto \alpha_A \) is natural, because given \( f : A \to A' \), \( \alpha_{A'}(Ff) \) and \( (F'f) \alpha_A \) are both morphisms \( (FA, \eta_A) \rightrightarrows (F'A', \eta'_{A'} f) \) from an initial object in \( (A \downarrow G) \), so must be equal.
\end{proof}
\begin{lemma}
    Suppose
    % https://q.uiver.app/#q=WzAsMyxbMCwwLCJcXG1hdGhjYWwgQyJdLFsxLDAsIlxcbWF0aGNhbCBEIl0sWzIsMCwiXFxtYXRoY2FsIEUiXSxbMCwxLCJGIiwwLHsib2Zmc2V0IjotMn1dLFsxLDIsIkgiLDAseyJvZmZzZXQiOi0yfV0sWzIsMSwiSyIsMCx7Im9mZnNldCI6LTJ9XSxbMSwwLCJHIiwwLHsib2Zmc2V0IjotMn1dXQ==
\[\begin{tikzcd}
	{\mathcal C} & {\mathcal D} & {\mathcal E}
	\arrow["F", shift left=2, from=1-1, to=1-2]
	\arrow["H", shift left=2, from=1-2, to=1-3]
	\arrow["K", shift left=2, from=1-3, to=1-2]
	\arrow["G", shift left=2, from=1-2, to=1-1]
\end{tikzcd}\]
    where \( F \dashv G \) and \( H \dashv K \).
    Then \( HF \dashv GK \).
\end{lemma}
\begin{proof}
    We have bijections between morphisms \( HFA \to C \), morphisms \( FA \to KC \), and morphisms \( A \to GKC \), which are natural in \( A \) and \( C \), so their composite is also natural.
\end{proof}
\begin{corollary}
    Suppose the square of functors
    % https://q.uiver.app/#q=WzAsNCxbMCwwLCJcXG1hdGhjYWwgQyJdLFsxLDAsIlxcbWF0aGNhbCBEIl0sWzEsMSwiXFxtYXRoY2FsIEYiXSxbMCwxLCJcXG1hdGhjYWwgRSJdLFswLDEsIkYiXSxbMSwyLCJIIl0sWzAsMywiRyIsMl0sWzMsMiwiSyIsMl1d
\[\begin{tikzcd}
	{\mathcal C} & {\mathcal D} \\
	{\mathcal E} & {\mathcal F}
	\arrow["F", from=1-1, to=1-2]
	\arrow["H", from=1-2, to=2-2]
	\arrow["G"', from=1-1, to=2-1]
	\arrow["K"', from=2-1, to=2-2]
\end{tikzcd}\]
    commutes, and all of the functors \( F, G, H, K \) have left adjoints \( F', G', H', K' \).
    Then the square of left adjoints
    % https://q.uiver.app/#q=WzAsNCxbMCwwLCJcXG1hdGhjYWwgQyJdLFsxLDAsIlxcbWF0aGNhbCBEIl0sWzEsMSwiXFxtYXRoY2FsIEYiXSxbMCwxLCJcXG1hdGhjYWwgRSJdLFsxLDAsIkYnIiwyXSxbMiwxLCJIJyIsMl0sWzMsMCwiRyciXSxbMiwzLCJLJyJdXQ==
\[\begin{tikzcd}
	{\mathcal C} & {\mathcal D} \\
	{\mathcal E} & {\mathcal F}
	\arrow["{F'}"', from=1-2, to=1-1]
	\arrow["{H'}"', from=2-2, to=1-2]
	\arrow["{G'}", from=2-1, to=1-1]
	\arrow["{K'}", from=2-2, to=2-1]
\end{tikzcd}\]
    commutes up to natural isomorphism.
\end{corollary}
This result holds for any shape of diagram, not just a square.
The hypothesis can be weakened to only require that the first diagram commutes up to natural isomorphism.
\begin{proof}
    The two composites \( F'H' \) and \( G'K' \) are left adjoints to \( HF = KG \), so must be naturally isomorphic.
\end{proof}
