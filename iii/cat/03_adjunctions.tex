\subsection{Definition and examples}
\begin{definition}
    Let \( \mathcal C, \mathcal D \) be categories.
    An \emph{adjunction} between \( \mathcal C \) and \( \mathcal D \) is a pair of functors \( F : \mathcal C \to \mathcal D \) and \( G : \mathcal D \to \mathcal C \), together with a bijection between morphisms \( FA \to B \) in \( \mathcal D \) and \( A \to HB \) in \( \mathcal C \), which is natural in both variables \( A, B \).
    We say that \( F \) is the \emph{left adjoint} to \( G \), and that \( G \) is the \emph{right adjoint} to \( F \), and write \( F \dashv G \).
\end{definition}
If \( \mathcal C, \mathcal D \) are locally small, then the naturality condition is that
\[ \mathcal D(F-, -);\quad \mathcal C(-, G-) \]
are naturally isomorphic functors \( \mathcal C^\cop \times \mathcal D \to \mathbf{Set} \).
\begin{example}
    \begin{enumerate}
        \item The free group functor \( F : \mathbf{Set} \to \mathbf{Gp} \) is left adjoint to the forgetful functor \( U : \mathbf{Gp} \to \mathbf{Set} \).
        \[ \mathbf{Gp}(FA, G) \leftrightarrow \mathbf{Set}(A, UG) \]
        \item The forgetful functor \( U : \mathbf{Top} \to \mathbf{Set} \) has a left adjoint \( D : \mathbf{Set} \to \mathbf{Top} \) which equips each set with its discrete topology.
        \[ \mathbf{Top}(DX, Y) \leftrightarrow \mathbf{Set}(X, UY) \]
        It also has a right adjoint \( I : \mathbf{Set} \to \mathbf{Top} \) which equips each set with its indiscrete topology.
        \[ \mathbf{Set}(UX, Y) \leftrightarrow \mathbf{Top}(X, IY) \]
        \item Consider the functor \( \ob : \mathbf{Cat} \to \mathbf{Set} \) which maps each category to each set of objects.
        It has a left adjoint \( D \) which turns each set \( X \) into a discrete category in which the objects are elements of \( X \), and the only morphisms are identities.
        % fix
        % \[ \mathbf{Set}(\pi_0 \mathcal C, X) \leftrightarrow \mathbf{Cat}(\mathcal C, DX) \]
        It also has a right adjoint \( I \) which turns each set \( X \) into an indiscrete category in which the objects are elements of \( X \), and there is exactly one morphism between any two elements of \( X \).
        % fix
        % \[ \mathbf{Set}(\pi_0 \mathcal C, X) \leftrightarrow \mathbf{Cat}(\mathcal C, DX) \]
        In addition, \( D : \mathbf{Set} \to \mathbf{Cat} \) has a left adjoint \( \pi_0 : \mathbf{Cat} \to \mathbf{Set} \), where \( \pi_0 \mathcal C \) is the set of connected components of \( \ob \mathcal C \) under the graph induced by its morphisms.
        \[ \mathbf{Set}(\pi_0 \mathcal C, X) \leftrightarrow \mathbf{Cat}(\mathcal C, DX) \]
        Thus we have a chain
        \[ \pi_0 \dashv D \dashv \ob \dashv I \]
        \item For any set \( A \), we have a functor \( (-) \times A : \mathbf{Set} \to \mathbf{Set} \).
        This functor has a right adjoint, which is the functor \( \mathbf{Set}(A, -) : \mathbf{Set} \to \mathbf{Set} \).
        \[ \mathbf{Set}(B \times A, C) \leftrightarrow \mathbf{Set}(B, \mathbf{Set}(A, C)) \]
        Applying this bijection is sometimes called \emph{currying} or \emph{\( \lambda \)-conversion}.
        We say that a category \( \mathcal C \) with binary products is \emph{cartesian closed} if \( (-) \times A \mathcal C \to \mathcal C \) has a right adjoint, written \( [A, -] \) or \( (-)^A \), for each \( A \).
        For example, \( \mathbf{Cat} \) is cartesian closed, where \( \mathcal D^{\mathcal C} = [\mathcal C, \mathcal D] \) is the functor category that this notation already refers to.
    \end{enumerate}
\end{example}
