\subsection{Definition and examples}
\begin{definition}
    Let \( \mathcal C, \mathcal D \) be categories.
    An \emph{adjunction} between \( \mathcal C \) and \( \mathcal D \) is a pair of functors \( F : \mathcal C \to \mathcal D \) and \( G : \mathcal D \to \mathcal C \), together with a bijection between morphisms \( FA \to B \) in \( \mathcal D \) and \( A \to HB \) in \( \mathcal C \), which is natural in both variables \( A, B \).
    We say that \( F \) is the \emph{left adjoint} to \( G \), and that \( G \) is the \emph{right adjoint} to \( F \), and write \( F \dashv G \).
\end{definition}
If \( \mathcal C, \mathcal D \) are locally small, then the naturality condition is that
\[ \mathcal D(F-, -);\quad \mathcal C(-, G-) \]
are naturally isomorphic functors \( \mathcal C^\cop \times \mathcal D \to \mathbf{Set} \).
\begin{example}
    \begin{enumerate}
        \item The free group functor \( F : \mathbf{Set} \to \mathbf{Gp} \) is left adjoint to the forgetful functor \( U : \mathbf{Gp} \to \mathbf{Set} \).
        \[ \mathbf{Gp}(FA, G) \leftrightarrow \mathbf{Set}(A, UG) \]
        \item The forgetful functor \( U : \mathbf{Top} \to \mathbf{Set} \) has a left adjoint \( D : \mathbf{Set} \to \mathbf{Top} \) which equips each set with its discrete topology.
        \[ \mathbf{Top}(DX, Y) \leftrightarrow \mathbf{Set}(X, UY) \]
        It also has a right adjoint \( I : \mathbf{Set} \to \mathbf{Top} \) which equips each set with its indiscrete topology.
        \[ \mathbf{Set}(UX, Y) \leftrightarrow \mathbf{Top}(X, IY) \]
        \item Consider the functor \( \ob : \mathbf{Cat} \to \mathbf{Set} \) which maps each category to each set of objects.
        It has a left adjoint \( D \) which turns each set \( X \) into a discrete category in which the objects are elements of \( X \), and the only morphisms are identities.
        It also has a right adjoint \( I \) which turns each set \( X \) into an indiscrete category in which the objects are elements of \( X \), and there is exactly one morphism between any two elements of \( X \).
        In addition, \( D : \mathbf{Set} \to \mathbf{Cat} \) has a left adjoint \( \pi_0 : \mathbf{Cat} \to \mathbf{Set} \), where \( \pi_0 \mathcal C \) is the set of connected components of \( \ob \mathcal C \) under the graph induced by its morphisms.
        \[ \mathbf{Set}(\pi_0 \mathcal C, X) \leftrightarrow \mathbf{Cat}(\mathcal C, DX);\quad \mathbf{Cat}(D X, \mathcal C) \leftrightarrow \mathbf{Set}(X, \ob \mathcal C);\quad \mathbf{Set}(\ob \mathcal C, X) \leftrightarrow \mathbf{Cat}(\mathcal C, IX) \]
        Thus we have a chain
        \[ \pi_0 \dashv D \dashv \ob \dashv I \]
        \item For any set \( A \), we have a functor \( (-) \times A : \mathbf{Set} \to \mathbf{Set} \).
        This functor has a right adjoint, which is the functor \( \mathbf{Set}(A, -) : \mathbf{Set} \to \mathbf{Set} \).
        \[ \mathbf{Set}(B \times A, C) \leftrightarrow \mathbf{Set}(B, \mathbf{Set}(A, C)) \]
        Applying this bijection is sometimes called \emph{currying} or \emph{\( \lambda \)-conversion}.
        We say that a category \( \mathcal C \) with binary products is \emph{cartesian closed} if \( (-) \times A : \mathcal C \to \mathcal C \) has a right adjoint, written \( [A, -] \) or \( (-)^A \), for each \( A \).
        For example, \( \mathbf{Cat} \) is cartesian closed, where \( \mathcal D^{\mathcal C} = [\mathcal C, \mathcal D] \) is the functor category that this notation already refers to.
        \item An equivalence \( F : \mathcal C \to \mathcal D \), \( G : \mathcal D \to \mathcal C \) forms adjunctions both ways: \( F \dashv G, G \dashv F \).
        \item Let \( \mathbf{Idem} \) be the category of pairs \( (A, e) \) where \( A \) is a set and \( e \) is an idempotent endomorphism \( A \to A \).
        The morphisms in \( \mathbf{Idem} \) are the maps of sets which commute with the idempotents.
        We have a functor \( F : \mathbf{Set} \to \mathbf{Idem} \) sending \( A \) to \( (A, 1_A) \).
        Consider \( G : \mathbf{Idem} \to \mathbf{Set} \) sending \( (A, e) \) to the set of fixed points of \( e \).
        Then \( F \dashv G \) since any morphism \( FA \to (B, e) \) takes values in \( G(B, e) \).
        But also \( G \dashv F \), since a morphism \( (A, e) \to FB \) is entirely determined by its action on the fixed points in \( A \) under \( e \), because \( f(a) = f(ea) \).
        This is not an equivalence of categories, because \( G \) is not faithful.
        So not all pairs of functors that are adjoint in both directions form an equivalence.
        \item Let \( \mathcal C \) be a category.
        There is a unique functor \( G : \mathcal C \to \mathbf 1 \), where \( \mathbf 1 \) is the discrete category on a single object.
        A left adjoint for \( G \), if it exists, sends the object in \( \mathbf 1 \) to an \emph{initial object} \( I \) of \( \mathcal C \), which is an object with a unique morphism to every object in \( \mathcal C \).
        Dually, a right adjoint sends the object in \( \mathbf 1 \) to a \emph{terminal object} \( T \), which is an object with a unique morphism from every object in \( \mathcal C \).
        In \( \mathbf{Set} \), the empty set is initial, and any singleton is terminal.
        In \( \mathbf{Gp} \), the trivial group is initial and terminal.
        \item Let \( f : A \to B \) be a function of sets, and let \( A' \subseteq A, B' \subseteq B \).
        Then \( Pf(A') \subseteq B' \) if and only if \( A' \subseteq P^\star f(B') \).
        Thus \( Pf \dashv P^\star f \) as functors between \( PA \) and \( PB \) as posets.
        \item Let \( A, B \) be sets with a relation \( R \subseteq A \times B \).
        We define mappings \( (-)^r : PA \to PB \) by
        \[ S^r = \qty{b \in B \mid \forall a \in S,\, (a, b) \in R} \]
        and \( (-)^\ell : PB \to PA \) by
        \[ T^\ell = \qty{a \in A \mid \forall b \in T,\, (a, b) \in R} \]
        These are contravariant functors, and
        \[ S \subseteq T^\ell \iff S \times T \subseteq R \iff T \subseteq S^r \]
        We say that \( (-)^\ell \) and \( (-)^r \) are \emph{adjoint on the right}.
        This pair is called a \emph{Galois connection}.
        \item The contravariant power-set functor \( P^\star \) is self-adjoint on the right, since functions \( A \to P^\star B \) and \( B \to P^\star A \) naturally correspond bijectively to subsets of \( A \times B \).
        \item The dual vector space functor \( (-)^\star : \mathbf{Vect}_k \to \mathbf{Vect}_k \) is self-adjoint on the right, as linear maps \( V \to W^\star \) and linear maps \( W \to V^\star \) both naturally correspond to bilinear forms on \( V \times W \).
    \end{enumerate}
\end{example}

\subsection{Comma categories}
\begin{definition}
    Let \( G : \mathcal D \to \mathcal C \) be a functor and \( A \in \ob \mathcal C \).
    Then, the \emph{comma category} \( (A \downarrow G) \) is the category whose objects are pairs \( (B, f) \) where \( B \in \ob \mathcal D \) and \( f : A \to GB \) in \( \mathcal C \), and whose morphisms \( (B, f) \to (B', f') \) are morphisms \( g : B \to B' \) which commute with \( f, f' \):
\[\begin{tikzcd}
	A & GB \\
	& {GB'}
	\arrow["f", from=1-1, to=1-2]
	\arrow["Gg", from=1-2, to=2-2]
	\arrow["{f'}"', from=1-1, to=2-2]
\end{tikzcd}\]
\end{definition}
\begin{theorem}
    Let \( G : \mathcal D \to \mathcal C \) be a functor.
    Then specifying a left adjoint for \( G \) is equivalent to specifying an initial object of the comma categories \( (A \downarrow G) \) for each \( A \).
\end{theorem}
\begin{proof}
    First, note that an object \( (B, f) \) is initial in \( (A \downarrow G) \) if and only if for every \( (B', f') \), there is a unique morphism \( g : B \to B' \) such that the following triangle commutes.
    \[\begin{tikzcd}
        A & GB \\
        & {GB'}
        \arrow["f", from=1-1, to=1-2]
        \arrow["Gg", from=1-2, to=2-2]
        \arrow["{f'}"', from=1-1, to=2-2]
    \end{tikzcd}\]
    Suppose \( F \dashv G \).
    Then let \( \eta_A : A \to GFA \) correspond to the identity \( 1_{FA} \) under the adjunction.
    We show that \( (FA, \eta_A) \) is initial in \( (A \downarrow G) \).
    Indeed, given \( f : A \to GB \), then
    % https://q.uiver.app/#q=WzAsMyxbMCwwLCJBIl0sWzEsMCwiR0ZBIl0sWzEsMSwiR0IiXSxbMCwxLCJcXGV0YV9BIl0sWzEsMiwiR2ciXSxbMCwyLCJmIiwyXV0=
\[\begin{tikzcd}
	A & GFA \\
	& GB
	\arrow["{\eta_A}", from=1-1, to=1-2]
	\arrow["Gg", from=1-2, to=2-2]
	\arrow["f"', from=1-1, to=2-2]
\end{tikzcd}\]
    commutes if and only if \( g \) is the morphism corresponding to \( f \) under the adjunction.
    In particular, for any \( f \), there is a unique such \( g \).

    Conversely, suppose \( (FA, \eta_A) \) is initial in \( (A \downarrow G) \) for each \( A \).
    Then we define the action of \( F \) on objects by mapping \( A \) to \( FA \).
    We make \( F \) into a functor by mapping \( f : A \to A' \) to the unique morphism that makes the following square commute; this exists as \( (FA, \eta_A) \) is initial.
    % https://q.uiver.app/#q=WzAsNCxbMCwwLCJBIl0sWzEsMCwiR0ZBIl0sWzEsMSwiR0ZBJyJdLFswLDEsIkEnIl0sWzAsMSwiXFxldGFfQSJdLFsxLDIsIkdGZiJdLFswLDMsImYiLDJdLFszLDIsIlxcZXRhX3tBJ30iLDJdXQ==
\[\begin{tikzcd}
	A & GFA \\
	{A'} & {GFA'}
	\arrow["{\eta_A}", from=1-1, to=1-2]
	\arrow["GFf", from=1-2, to=2-2]
	\arrow["f"', from=1-1, to=2-1]
	\arrow["{\eta_{A'}}"', from=2-1, to=2-2]
\end{tikzcd}\]
    Functoriality of \( F \) follows from the uniqueness of \( Ff \).
    The bijection between morphisms \( f : A \to GB \) and \( g : FA \to B \) sends \( f \) to the unique \( g \) giving \( (Gg)\eta_A = f \).
    Naturality of the bijection in \( A \) was built in to the definition of \( F \) as a functor, and naturality in \( B \) is easy.
\end{proof}
\begin{corollary}
    Let \( F, F' : \mathcal C \to \mathcal D \) be left adjoints to \( G : \mathcal D \to \mathcal C \).
    Then \( F \simeq F' \) in \( [\mathcal C, \mathcal D] \).
\end{corollary}
\begin{proof}
    \( (FA, \eta_A) \) and \( (F'A, \eta'_A) \) are both initial objects in \( (A \downarrow G) \), and so there is a unique isomorphism \( \alpha_A : (FA, \eta_A) \to (F'A, \eta'_A) \) in this category.
    The map \( A \mapsto \alpha_A \) is natural, because given \( f : A \to A' \), \( \alpha_{A'}(Ff) \) and \( (F'f) \alpha_A \) are both morphisms \( (FA, \eta_A) \rightrightarrows (F'A', \eta'_{A'} f) \) from an initial object in \( (A \downarrow G) \), so must be equal.
\end{proof}
\begin{lemma}
    Suppose
    % https://q.uiver.app/#q=WzAsMyxbMCwwLCJcXG1hdGhjYWwgQyJdLFsxLDAsIlxcbWF0aGNhbCBEIl0sWzIsMCwiXFxtYXRoY2FsIEUiXSxbMCwxLCJGIiwwLHsib2Zmc2V0IjotMn1dLFsxLDIsIkgiLDAseyJvZmZzZXQiOi0yfV0sWzIsMSwiSyIsMCx7Im9mZnNldCI6LTJ9XSxbMSwwLCJHIiwwLHsib2Zmc2V0IjotMn1dXQ==
\[\begin{tikzcd}
	{\mathcal C} & {\mathcal D} & {\mathcal E}
	\arrow["F", shift left=2, from=1-1, to=1-2]
	\arrow["H", shift left=2, from=1-2, to=1-3]
	\arrow["K", shift left=2, from=1-3, to=1-2]
	\arrow["G", shift left=2, from=1-2, to=1-1]
\end{tikzcd}\]
    where \( F \dashv G \) and \( H \dashv K \).
    Then \( HF \dashv GK \).
\end{lemma}
\begin{proof}
    We have bijections
    \[ \mathcal E(HFA, C) \leftrightarrow \mathcal D(FA, KC) \leftrightarrow \mathcal C(A, GKC) \]
    which are natural in \( A \) and \( C \), so their composite is also natural.
\end{proof}
\begin{corollary}
    Suppose the square of functors
    % https://q.uiver.app/#q=WzAsNCxbMCwwLCJcXG1hdGhjYWwgQyJdLFsxLDAsIlxcbWF0aGNhbCBEIl0sWzEsMSwiXFxtYXRoY2FsIEYiXSxbMCwxLCJcXG1hdGhjYWwgRSJdLFswLDEsIkYiXSxbMSwyLCJIIl0sWzAsMywiRyIsMl0sWzMsMiwiSyIsMl1d
\[\begin{tikzcd}
	{\mathcal C} & {\mathcal D} \\
	{\mathcal E} & {\mathcal F}
	\arrow["F", from=1-1, to=1-2]
	\arrow["H", from=1-2, to=2-2]
	\arrow["G"', from=1-1, to=2-1]
	\arrow["K"', from=2-1, to=2-2]
\end{tikzcd}\]
    commutes, and all of the functors \( F, G, H, K \) have left adjoints \( F', G', H', K' \).
    Then the square of left adjoints
    % https://q.uiver.app/#q=WzAsNCxbMCwwLCJcXG1hdGhjYWwgQyJdLFsxLDAsIlxcbWF0aGNhbCBEIl0sWzEsMSwiXFxtYXRoY2FsIEYiXSxbMCwxLCJcXG1hdGhjYWwgRSJdLFsxLDAsIkYnIiwyXSxbMiwxLCJIJyIsMl0sWzMsMCwiRyciXSxbMiwzLCJLJyJdXQ==
\[\begin{tikzcd}
	{\mathcal C} & {\mathcal D} \\
	{\mathcal E} & {\mathcal F}
	\arrow["{F'}"', from=1-2, to=1-1]
	\arrow["{H'}"', from=2-2, to=1-2]
	\arrow["{G'}", from=2-1, to=1-1]
	\arrow["{K'}", from=2-2, to=2-1]
\end{tikzcd}\]
    commutes up to natural isomorphism.
\end{corollary}
This result holds for any shape of diagram, not just a square.
The hypothesis can be weakened to only require that the first diagram commutes up to natural isomorphism.
\begin{proof}
    The two composites \( F'H' \) and \( G'K' \) are left adjoints to \( HF = KG \), so must be naturally isomorphic.
\end{proof}

\subsection{Units and counits}
Given an adjunction \( F \dashv G \), the proof of the previous theorem demonstrated a naturality square between the morphisms \( \eta_A : A \to GFA \) corresponding to \( 1_{FA} \) under the adjunction.
We call \( \eta : 1_{\mathcal C} \to GF \) the \emph{unit} of the adjunction.
Dually, the map \( \epsilon : FG \to 1_{\mathcal D} \) is called the \emph{counit} of the adjunction; each \( \epsilon_B : FGB \to B \) corresponds to \( 1_{GB} \).
\begin{theorem}
    Let \( F : \mathcal C \to \mathcal D \), \( G : \mathcal D \to \mathcal C \).
    Specifying an adjunction \( F \dashv G \) is equivalent to specifying natural transformations \( \eta : 1_{\mathcal C} \to GF \), \( \epsilon : FG \to 1_{\mathcal D} \), satisfying the \emph{triangular identities}
    % https://q.uiver.app/#q=WzAsNixbMCwwLCJGIl0sWzEsMCwiRkdGIl0sWzEsMSwiRiJdLFszLDAsIkciXSxbNCwwLCJHRkciXSxbNCwxLCJHIl0sWzAsMSwiRlxcZXRhIl0sWzEsMiwiXFxlcHNpbG9uX0YiXSxbMCwyLCIxX0YiLDJdLFszLDQsIlxcZXRhX0ciXSxbNCw1LCJHXFxlcHNpbG9uIl0sWzMsNSwiMV9HIiwyXV0=
\[\begin{tikzcd}
	F & FGF && G & GFG \\
	& F &&& G
	\arrow["F\eta", from=1-1, to=1-2]
	\arrow["{\epsilon_F}", from=1-2, to=2-2]
	\arrow["{1_F}"', from=1-1, to=2-2]
	\arrow["{\eta_G}", from=1-4, to=1-5]
	\arrow["G\epsilon", from=1-5, to=2-5]
	\arrow["{1_G}"', from=1-4, to=2-5]
\end{tikzcd}\]
\end{theorem}
\begin{proof}
    Suppose we have an adjunction \( F \dashv G \).
    We have seen how to define \( \eta \) and \( \epsilon \); it thus suffices to check the triangular identities.
    Since they are dual to each other, it suffices to check the first.
    The morphism \( \epsilon_{FA} \) corresponds under the adjunction to \( 1_{GFA} \), so by naturality, the composite \( \epsilon_{FA} (F\eta_A) \) corresponds to \( 1_{GFA} \eta_A = \eta_A \).
    But \( 1_{FA} \) corresponds to \( \eta_A \), giving the commutative triangle \( \epsilon_{FA} (F\eta_A) = 1_{FA} \).

    Conversely, suppose \( \eta \) and \( \epsilon \) are natural transformations satisfying the triangular identities.
    We map \( f : A \to GB \) to the composite \( \Phi(f) \) given by
    % https://q.uiver.app/#q=WzAsMyxbMCwwLCJGQSJdLFsxLDAsIkZHQiJdLFsyLDAsIkIiXSxbMCwxLCJGZiJdLFsxLDIsIlxcZXBzaWxvbl9CIl1d
\[\begin{tikzcd}
	FA & FGB & B
	\arrow["Ff", from=1-1, to=1-2]
	\arrow["{\epsilon_B}", from=1-2, to=1-3]
\end{tikzcd}\]
    and \( g : FA \to B \) to the composite \( \Psi(g) \) given by
    % https://q.uiver.app/#q=WzAsMyxbMCwwLCJBIl0sWzEsMCwiR0ZBIl0sWzIsMCwiR0IiXSxbMCwxLCJcXGV0YV9BIl0sWzEsMiwiR2ciXV0=
    \[\begin{tikzcd}
        A & GFA & GB
        \arrow["{\eta_A}", from=1-1, to=1-2]
        \arrow["Gg", from=1-2, to=1-3]
    \end{tikzcd}\]
    These assignments are natural in \( A \) and \( B \) as \( \eta \) and \( \epsilon \) are natural transformations.
    Thus it suffices to show \( \Psi \Phi \) and \( \Phi \Psi \) are the relevant identity maps; again they are dual so it suffices to show \( \Psi \Phi(f) = f \).
    \( \Psi \Phi(f) \) is the composite% https://q.uiver.app/#q=WzAsNCxbMCwwLCJBIl0sWzEsMCwiR0ZBIl0sWzIsMCwiR0ZHQiJdLFszLDAsIkdCIl0sWzAsMSwiXFxldGFfQSJdLFsxLDIsIkdGZiJdLFsyLDMsIkdcXGVwc2lsb25fQiJdXQ==
    \[\begin{tikzcd}
        A & GFA & GFGB & GB
        \arrow["{\eta_A}", from=1-1, to=1-2]
        \arrow["GFf", from=1-2, to=1-3]
        \arrow["{G\epsilon_B}", from=1-3, to=1-4]
    \end{tikzcd}\]
    which by naturality of \( \eta \) is equal to
    % https://q.uiver.app/#q=WzAsNCxbMCwwLCJBIl0sWzEsMCwiR0IiXSxbMiwwLCJHRkdCIl0sWzMsMCwiR0IiXSxbMCwxLCJmIl0sWzEsMiwiXFxldGFfe0dCfSJdLFsyLDMsIkdcXGVwc2lsb25fQiJdXQ==
\[\begin{tikzcd}
	A & GB & GFGB & GB
	\arrow["f", from=1-1, to=1-2]
	\arrow["{\eta_{GB}}", from=1-2, to=1-3]
	\arrow["{G\epsilon_B}", from=1-3, to=1-4]
\end{tikzcd}\]
    which is equal to \( f \) by the triangular identity.
\end{proof}
Recall that an equivalence of categories consisted of isomorphisms \( \alpha : 1_{\mathcal C} \to GF \) and \( \beta : FG \to 1_{\mathcal D} \).
These isomorphisms may not satisfy the triangular identities, but we can always choose \( \alpha \) and \( \beta \) in such a way that these identities hold.
\begin{proposition}
    Let \( (F, G, \alpha, \beta) \) be an equivalence of categories.
    Then there exist natural isomorphisms \( \alpha' : 1_{\mathcal C} \to GF \) and \( \beta' : FG \to 1_{\mathcal D} \) which satisfy the triangular identities.
    In particular, \( F \dashv G \dashv F \).
\end{proposition}
\begin{proof}
    We will set \( \alpha' = \alpha \), and construct \( \beta' \) to be the composite
    % https://q.uiver.app/#q=WzAsNCxbMCwwLCJGRyJdLFsxLDAsIkZHRkciXSxbMiwwLCJGRyJdLFszLDAsIjFfe1xcbWF0aGNhbCBEfSJdLFswLDEsIihGR1xcYmV0YSleey0xfSJdLFsxLDIsIihGXFxhbHBoYV9HKV57LTF9Il0sWzIsMywiXFxiZXRhIl1d
\[\begin{tikzcd}
	FG & FGFG & FG & {1_{\mathcal D}}
	\arrow["{(FG\beta)^{-1}}", from=1-1, to=1-2]
	\arrow["{(F\alpha_G)^{-1}}", from=1-2, to=1-3]
	\arrow["\beta", from=1-3, to=1-4]
\end{tikzcd}\]
    Note that \( FG\beta = \beta_{FG} \), since
\[\begin{tikzcd}
	FGFG & FG \\
	FG & {1_{\mathcal D}}
	\arrow["FG\beta", from=1-1, to=1-2]
	\arrow["\beta", from=1-2, to=2-2]
	\arrow["{\beta_{FG}}"', from=1-1, to=2-1]
	\arrow["\beta"', from=2-1, to=2-2]
\end{tikzcd}\]
    commutes by naturality of \( \beta \).
    Note also that \( \beta \) is monic.
    Dually, note that \( GF\alpha = \alpha_{GF} \).
    For the triangular identities, consider the diagrams
\[\begin{tikzcd}
	F & FGF & FGFGF \\
	& F & FGF \\
	&& F
	\arrow["{F\alpha}", from=1-1, to=1-2]
	\arrow["{(\beta_{FGF})^{-1}}", from=1-2, to=1-3]
	\arrow["{(F\alpha_{GF})^{-1} = (FGF \alpha)^{-1}}", from=1-3, to=2-3]
	\arrow["{\beta_F}", from=2-3, to=3-3]
	\arrow["{(F\alpha)^{-1}}", from=1-2, to=2-2]
	\arrow["{\beta_F}"{description}, from=2-2, to=2-3]
	\arrow["{1_F}"', from=1-1, to=2-2]
	\arrow["{1_F}"', from=2-2, to=3-3]
\end{tikzcd}\]
    and
\[\begin{tikzcd}
	G & GFG & GFGFG \\
	& G & GFG \\
	&& G
	\arrow["{(GFG\beta)^{-1}}", from=1-2, to=1-3]
	\arrow["{(GF\alpha_G)^{-1} = (\alpha_{GFG})^{-1}}", from=1-3, to=2-3]
	\arrow["G\beta", from=2-3, to=3-3]
	\arrow["{\alpha_G^{-1}}", from=1-2, to=2-2]
	\arrow["{(G\beta)^{-1}}"{description}, from=2-2, to=2-3]
	\arrow["{1_G}"', from=1-1, to=2-2]
	\arrow["{1_G}"', from=2-2, to=3-3]
	\arrow["{\alpha_G}", from=1-1, to=1-2]
\end{tikzcd}\]
    where the squares commute by naturality of \( \beta \) and \( \alpha \) respectively.
    Thus \( \alpha', \beta' \) are the unit and counit of an adjunction \( F \dashv G \) as required.
    Similarly, \( (\beta')^{-1}, (\alpha')^{-1} \) are the unit and counit of an adjunction \( G \dashv F \).
\end{proof}
\begin{lemma}
    Let \( F \dashv G \) be an adjunction with counit \( \epsilon : FG \to 1_{\mathcal D} \).
    Then
    \begin{enumerate}
        \item \( \epsilon \) is pointwise epimorphic if and only if \( G \) is faithful;
        \item \( \epsilon \) is a (pointwise) isomorphism if and only if \( G \) is full and faithful.
    \end{enumerate}
\end{lemma}
\begin{proof}
    \emph{Part (i).}
    Given \( g : B \to B' \) in \( \mathcal D \), the composite \( g \epsilon_B \) corresponds under the adjunction to \( Gg : GB \to GB' \).
    Thus for morphisms \( g \) with specified domain and codomain, the map \( g \mapsto g \epsilon_B \) is injective if and only if the action of \( G \) is injective.
    This is true for all \( B \) and \( B' \) if and only if \( \epsilon \) is pointwise epimorphic, if and only if \( G \) is faithful.

    \emph{Part (ii).}
    Similarly, \( G \) is full and faithful if and only if the map \( g \mapsto g \epsilon_B \) is a bijection on morphisms with specified domain and codomain.
    This clearly holds if \( \epsilon_B \) is an isomorphism for all \( B \).
    Conversely, if the condition holds, there is a unique map \( g : B \to FGB \) such that \( \epsilon_B g = 1_B \).
    Then \( \epsilon_B g \epsilon_B = \epsilon_B \), so \( g \epsilon_B \) and \( 1_{FGB} \) have the same composite with \( \epsilon_B \), so they are equal.
\end{proof}

\subsection{Reflections}
\begin{definition}
    An adjunction \( F \dashv G \) is called a \emph{reflection} if the counit is an isomorphism.
    Dually, it is called a \emph{coreflection} if the unit is an isomorphism.
    A full subcategory is called \emph{reflective} if the inclusion functor has a left adjoint; in this case the adjunction is a reflection.
\end{definition}
\begin{remark}
    If \( F \dashv G \) is a reflection, then \( G : \mathcal D \to \mathcal C \) induces an equivalence of categories between \( \mathcal D \) and the full subcategory of \( \mathcal C \) on the objects in the image of \( G \).
    This subcategory is reflective.
\end{remark}
If \( \mathcal D \subseteq \mathcal C \) is a reflective subcategory, there is intuitively a best possible way to get \emph{into} \( \mathcal D \) from some object in \( \mathcal C \).
The left adjoint sends an object in \( \mathcal C \) to its `best approximation' in \( \mathcal D \).
If \( \mathcal D \) is coreflective, there is a best possible way to get \emph{out of} \( \mathcal D \) to some object in \( \mathcal C \).
\begin{example}
    \begin{enumerate}
        \item \( \mathbf{AbGp} \) is reflective in \( \mathbf{Gp} \); the left adjoint to the inclusion map sends a group \( G \) to its abelianisation \( G^{\mathrm{ab}} = \faktor{G}{H} \), the quotient of \( G \) by its commutator subgroup \( H = \qty{aba^{-1}b^{-1} \mid a, b \in G} \trianglelefteq G \).
        Note that any homomorphism \( G \to A \) where \( A \) is abelian factors uniquely through the quotient map \( G \to G^{\mathrm{ab}} \), giving the adjunction as required.
        \item Recall that an abelian group is called \emph{torsion} if all of its elements have finite order, and \emph{torsion-free} if all of its nonzero elements have infinite order.
        For an abelian group \( A \), its set of torsion elements forms a subgroup \( A_t \), which is a torsion group.
        Any homomorphism from a torsion group to \( A \) must factor through \( A_t \).
        Thus \( A_t \) is the coreflection of \( A \) in the category of torsion abelian groups, and \( \faktor{A}{A_t} \) is the reflection of \( A \) in the category of torsion-free abelian groups.
        \item The full subcategory \( \mathbf{KHaus} \) of compact Hausdorff spaces is reflective in the category \( \mathbf{Top} \) of topological spaces.
        The left adjoint to the inclusion map is the \emph{Stone--\v{C}ech compactification} functor \( \beta \).
        We will construct this functor using the special adjoint functor theorem, which is explored in the next section.
        \item Recall that a subset \( C \) of a topological space \( X \) is called \emph{sequentially closed} if for every sequence \( x_n \in C \) converging to a limit \( x \in X \), we have \( x \in C \).
        We say that \( X \) is a \emph{sequential space} if all sequentially closed subsets are closed.
        The full subcategory \( \mathbf{Seq} \) of sequential spaces is coreflective in \( \mathbf{Top} \).
        Given a space \( X \), let \( X_s \) denote the same set, but where the topology is such that all sequentially closed sets are also taken to be closed.
        The identity map \( X_s \to X \) is continuous, and forms the counit of the adjunction.
        \item The category \( \mathbf{Preord} \) of preorders is reflective in \( \mathbf{Cat} \).
        The left adjoint maps a category \( \mathcal C \) to the quotient category \( \faktor{\mathcal C}{\sim} \) where \( \sim \) identifies all parallel pairs of morphisms.
        \item Let \( X \) be a topological space.
        Then the poset \( \Omega X \) of open sets in \( X \) is coreflective in the poset \( PX \), since if \( U \) is open and \( A \) is an arbitrary subset of \( X \), then \( U \subseteq A \) if and only if \( U \subseteq A^\circ \).
        Thus the interior operator \( (-)^\circ \) is right adjoint to the inclusion \( \Omega X \to PX \).
        Dually, the poset of closed sets is reflective in \( PX \); the closure operator \( \overline{(-)} \) is left adjoint to the inclusion.
    \end{enumerate}
\end{example}
