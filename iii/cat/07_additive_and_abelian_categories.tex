\subsection{Additive categories}
In this section, we will study categories enriched over \( (\mathbf{AbGp}, \otimes, \mathbb Z) \); these are called \emph{additive} categories.
We will also consider other weaker enrichments: a category enriched over \( (\mathbf{Set}_\star, \wedge, 2) \) is called \emph{pointed}, and a category enriched over \( (\mathbf{CMon}, \otimes, \mathbb N) \), where \( \mathbf{CMon} \) is the category of commutative monoids, is called \emph{semi-additive}.

In a pointed category \( \mathcal C \), each \( \mathcal C(A, B) \) has a distinguished element 0, and all composites with zero morphisms are zero morphisms.
In a semi-additive category \( \mathcal C \), each \( \mathcal C(A, B) \) has a binary addition operation which is associative, commutative, and has an identity \( 0 \).
Composition in a semi-additive category is bilinear, so \( (f + g)(h + k) = fh + gh + fk + gk \) whenever the composites are defined.
In an additive category, each morphism \( f \in \mathcal C(A, B) \) has an additive inverse \( -f \in \mathcal C(A, B) \).
\begin{lemma}
    \begin{enumerate}
        \item For an object \( A \) in a pointed category \( \mathcal C \), the following are equivalent.
        \begin{enumerate}
            \item \( A \) is a terminal object of \( \mathcal C \).
            \item \( A \) is an initial object of \( \mathcal C \).
            \item \( 1_A = 0 : A \to A \).
        \end{enumerate}
        \item For objects \( A, B, C \) in a semi-additive category \( \mathcal C \), the following are equivalent.
        \begin{enumerate}
            \item there exist morphisms \( \pi_1 : C \to A \) and \( \pi_2 : C \to B \) making \( C \) into a product of \( A \) and \( B \);
            \item there exist morphisms \( \nu_1 : A \to C \) and \( \nu_2 : B \to C \) making \( C \) into a coproduct of \( A \) and \( B \);
            \item there exist morphisms \( \pi_1 : C \to A, \pi_2 : C \to B, \nu_1 : A \to C, \nu_2 : B \to C \) satisfying
            \[ \pi_1 \nu_1 = 1_A;\quad \pi_2 \nu_2 = 1_B;\quad \pi_1 \nu_2 = 0;\quad \pi_2 \nu_1 = 0;\quad \nu_1 \pi_1 + \nu_2 \pi_1 = 1_C \]
        \end{enumerate}
    \end{enumerate}
\end{lemma}
\begin{proof}
    In each part, as (a) and (b) are dual and (c) is self-dual, it suffices to prove the equivalence of (a) and (b).

    \emph{Part (i).}
    If \( A \) is terminal, then it has exactly one morphism \( A \to A \), so this must be the zero morphism.
    Conversely, if \( 1_A = 0 \), then \( A \) is terminal, as for any \( f : B \to A \), we have \( f = 1_A f = 0 f = 0 \), so the only morphism \( B \to A \) is the zero morphism.

    \emph{Part (ii).}
    If (a) holds, take \( \nu_1, \nu_2 \) to be defined by the first four equations in (c); it suffices to verify the last equation, \( \nu_1 \pi_1 + \nu_2 \pi_2 = 1_C \).
    Composing with \( \pi_1 \),
    \[ \pi_1 \nu_1 \pi_1 = 1_A \pi_1 + 0 \pi_2 = \pi_1 \]
    and similarly, composing with \( \pi_2 \) gives \( \pi_2 \).
    So by uniqueness of factorisations through limit cones, \( \nu_1 \pi_1 + \nu_2 \pi_2 \) must be the identity.
    Conversely, if (c) holds, given a pair \( f : D \to A \) and \( g : D \to B \), the morphism
    \[ h = \nu_1 f + \nu_2 g \]
    satisfies
    \[ \pi_1 h = 1_A f + 0 g = f;\quad \pi_2 h = 0 f + 1_A g = g \]
    giving a factorisation, and if \( h' \) also satisfies these equations, then
    \[ h' = (\nu_1 \pi_1 + \nu_2 \pi_2) h' = \nu_1 f + \nu_2 g = h \]
    so the factorisation is unique.
\end{proof}
In any category, an object which is both initial and terminal is called a \emph{zero object}, denoted \( 0 \).
An object that is a product and a coproduct of \( A \) and \( B \) is called a \emph{biproduct}, denoted \( A \oplus B \).
\begin{lemma}
    Let \( \mathcal C \) be a locally small category.
    \begin{enumerate}
        \item If \( \mathcal C \) has a zero object, then it has a unique pointed structure.
        \item Suppose \( \mathcal C \) has a zero object and has binary products and coproducts.
        Suppose further that for each pair \( A, B \in \ob \mathcal C \), the canonical morphism \( c : A + B \to A \times B \) defined by
        \[ \pi_i c \nu_j = \begin{cases}
            1 & \text{if } i = j \\
            0 & \text{if } i \neq j
        \end{cases} \]
        is an isomorphism.
        Then \( \mathcal C \) has a unique semi-additive structure.
    \end{enumerate}
\end{lemma}
We adopt the convention that morphisms into a product are denoted with column vectors, and morphisms out of a product are denoted with row vectors.
\begin{proof}
    \emph{Part (i).}
    The unique morphism \( 0 \to 0 \) is both the identity and a zero morphism.
    So for any two \( A, B : \ob \mathcal C \), the unique composite \( A \to 0 \to B \) must be the zero element of \( \mathcal C(A, B) \).
    We can define a pointed structure on \( \mathcal C \) in this way.

    \emph{Part (ii).}
    This technique is known as the \emph{Eckmann--Hilton argument}.
    Given \( f, g : A \rightrightarrows B \), we define the \emph{left sum} \( f +_\ell g \) to be the composite
    % https://q.uiver.app/#q=WzAsNCxbMCwwLCJBIl0sWzEsMCwiQiBcXHRpbWVzIEIiXSxbMiwwLCJCK0IiXSxbMywwLCJCIl0sWzAsMSwiXFxiZWdpbntwbWF0cml4fWYgXFxcXCBnXFxlbmR7cG1hdHJpeH0iXSxbMSwyLCJjXnstMX0iXSxbMiwzLCJcXGJlZ2lue3BtYXRyaXh9MSYxXFxlbmR7cG1hdHJpeH0iXV0=
\[\begin{tikzcd}[ampersand replacement=\&]
	A \& {B \times B} \& {B+B} \& B
	\arrow["{\begin{pmatrix}f \\ g\end{pmatrix}}", from=1-1, to=1-2]
	\arrow["{c^{-1}}", from=1-2, to=1-3]
	\arrow["{\begin{pmatrix}1&1\end{pmatrix}}", from=1-3, to=1-4]
\end{tikzcd}\]
    and the \emph{right sum} \( f +_r g \) to be
    % https://q.uiver.app/#q=WzAsNCxbMCwwLCJBIl0sWzEsMCwiQSBcXHRpbWVzIEEiXSxbMiwwLCJCK0IiXSxbMywwLCJCIl0sWzAsMSwiXFxiZWdpbntwbWF0cml4fTEgXFxcXCAxXFxlbmR7cG1hdHJpeH0iXSxbMSwyLCJjXnstMX0iXSxbMiwzLCJcXGJlZ2lue3BtYXRyaXh9ZiZnXFxlbmR7cG1hdHJpeH0iXV0=
\[\begin{tikzcd}[ampersand replacement=\&]
	A \& {A \times A} \& {B+B} \& B
	\arrow["{\begin{pmatrix}1 \\ 1\end{pmatrix}}", from=1-1, to=1-2]
	\arrow["{c^{-1}}", from=1-2, to=1-3]
	\arrow["{\begin{pmatrix}f&g\end{pmatrix}}", from=1-3, to=1-4]
\end{tikzcd}\]
    Note that \( (f +_\ell g)h = fh +_\ell gh \), since
    \[ \begin{pmatrix}
        f \\ g
    \end{pmatrix} h = \begin{pmatrix}
        fh \\ gh
    \end{pmatrix} \]
    and similarly,
    \[ k(f +_r g) = kf +_r kg \]
    So if we show that the two sums coincide, we obtain the required distributive laws.
    First, note that \( 0 : A \to B \) is a two-sided identity for both \( +_\ell \) and \( +_r \).
    For example, \( f +_\ell 0 = f \), since
    % https://q.uiver.app/#q=WzAsNSxbMCwwLCJBIl0sWzEsMSwiQiBcXHRpbWVzIEIiXSxbMiwwLCJCIl0sWzMsMSwiQiArIEIiXSxbNCwwLCJCIl0sWzAsMSwiXFxiZWdpbntwbWF0cml4fWYgXFxcXCAwXFxlbmR7cG1hdHJpeH0iLDJdLFswLDIsImYiXSxbMiwxLCJcXGJlZ2lue3BtYXRyaXh9MVxcXFwwXFxlbmR7cG1hdHJpeH0iXSxbMiwzLCJcXG51XzEiXSxbMSwzLCJjXnstMX0iLDJdLFsyLDQsIjFfQiJdLFszLDQsIlxcYmVnaW57cG1hdHJpeH0xJjFcXGVuZHtwbWF0cml4fSIsMl1d
\[\begin{tikzcd}[ampersand replacement=\&]
	A \&\& B \&\& B \\
	\& {B \times B} \&\& {B + B}
	\arrow["{\begin{pmatrix}f \\ 0\end{pmatrix}}"', from=1-1, to=2-2]
	\arrow["f", from=1-1, to=1-3]
	\arrow["{\begin{pmatrix}1\\0\end{pmatrix}}", from=1-3, to=2-2]
	\arrow["{\nu_1}", from=1-3, to=2-4]
	\arrow["{c^{-1}}"', from=2-2, to=2-4]
	\arrow["{1_B}", from=1-3, to=1-5]
	\arrow["{\begin{pmatrix}1&1\end{pmatrix}}"', from=2-4, to=1-5]
\end{tikzcd}\]
    commutes.
    Suppose we have morphisms \( f, g, h, k : A \to B \), and consider the composite
    % https://q.uiver.app/#q=WzAsNixbMCwwLCJBIl0sWzEsMCwiQSBcXHRpbWVzIEEiXSxbMiwwLCJBICsgQSJdLFszLDAsIkIgXFx0aW1lcyBCIl0sWzQsMCwiQiArIEIiXSxbNSwwLCJCIl0sWzAsMSwiXFxiZWdpbntwbWF0cml4fTFcXFxcMVxcZW5ke3BtYXRyaXh9Il0sWzEsMiwiY157LTF9Il0sWzIsMywiXFxiZWdpbntwbWF0cml4fWYmZ1xcXFxoJmtcXGVuZHtwbWF0cml4fSJdLFszLDQsImNeey0xfSJdLFs0LDUsIlxcYmVnaW57cG1hdHJpeH0xJjFcXGVuZHtwbWF0cml4fSJdXQ==
\[\begin{tikzcd}[ampersand replacement=\&]
	A \& {A \times A} \& {A + A} \& {B \times B} \& {B + B} \& B
	\arrow["{\begin{pmatrix}1\\1\end{pmatrix}}", from=1-1, to=1-2]
	\arrow["{c^{-1}}", from=1-2, to=1-3]
	\arrow["{\begin{pmatrix}f&g\\h&k\end{pmatrix}}", from=1-3, to=1-4]
	\arrow["{c^{-1}}", from=1-4, to=1-5]
	\arrow["{\begin{pmatrix}1&1\end{pmatrix}}", from=1-5, to=1-6]
\end{tikzcd}\]
    The composite of the first three factors is
    \[ \begin{pmatrix}
        f +_r g \\
        h +_r k
    \end{pmatrix} \]
    so the whole composite is \( (f +_r g) +_\ell (h +_r k) \).
    Evaluating from other end, we obtain
    \[ (f +_r g) +_\ell (h +_r k) = (f +_\ell h) +_r (g +_\ell k) \]
    This is known as the \emph{interchange law}.
    Substituting \( g = k = 0 \), we obtain \( f +_\ell k = f +_r k \).
    Substituting \( f = k = 0 \) (and dropping the subscripts) we obtain the commutative law \( g + h = h + g \).
    Substituting \( h = 0 \), we obtain the associativity law \( (f + g) + k = f + (g + k) \).

    For uniqueness, suppose we have some semi-additive structure \( + \) on \( \mathcal C \).
    Then \( \nu_1 \pi_1 + \nu_2 \pi_2 \) must be the inverse of \( c = \begin{pmatrix}
        1 & 0 \\ 0 & 1
    \end{pmatrix} : A + B \to A \times B \), since
    \[ \nu_1 \pi_1 c = \nu_1 \begin{pmatrix}
        1 & 0
    \end{pmatrix} = \begin{pmatrix}
        \nu_1 & 0
    \end{pmatrix};\quad \nu_2 \pi_2 c =\begin{pmatrix}
        0 & \nu_2
    \end{pmatrix} \]
    so
    \[ (\nu_1 \pi_1 + \nu_2 \pi_2) c = \begin{pmatrix}
        \nu_1 + 0 & 0 + \nu_2
    \end{pmatrix} = \begin{pmatrix}
        \nu_1 & \nu_2
    \end{pmatrix} = 1_{A + B} \]
    Hence the definitions of \( +_\ell \) and \( +_r \) both reduce to \( + \).
\end{proof}
Note that if \( \mathcal C \) and \( \mathcal D \) are semi-additive categories with finite biproducts, then a functor \( F : \mathcal C \to \mathcal D \) is semi-additive (that is, enriched over \( \mathbf{CMon} \)) if and only if it preserves either finite products or finite coproducts.
In particular, if \( F \) has either a left or right adjoint, then it is semi-additive, and the adjunction is enriched over \( \mathbf{CMon} \); the bijection \( \mathcal C(A, GB) \to \mathcal D(FA, B) \) is an isomorphism of commutative monoids, since the operations \( F(-) \) and \( (-) \epsilon_B \) both respect addition.

\subsection{Kernels and cokernels}
\begin{definition}
    Let \( f : A \to B \) be a morphism in a pointed category \( \mathcal C \).
    The \emph{kernel} of \( f \) is the equaliser of the pair \( (f, 0) \); dually the \emph{cokernel} is the coequaliser of \( (f, 0) \).
    A monomorphism that occurs as the kernel of a morphism is called \emph{normal}.
\end{definition}
In an additive category, the normal monomorphisms are precisely the regular monomorphisms, since the equaliser of \( (f, g) \) is the kernel of \( f - g \).
In \( \mathbf{Gp} \), all inclusions of subgroups are regular, but not all inclusions are normal, since a normal monomorphism corresponds to a normal subgroup.
In \( \mathbf{Set}_\star \), all surjections are regular epimorphisms, but \( (A, a_0) \to (B, b_0) \) is a normal epimorphism if \( f \) is bijective on elements not mapped to \( b_0 \).
We say that a morphism \( f : A \to B \) is a \emph{pseudomonomorphism} if its kernel is a zero morphism; that is, \( fg = 0 \) implies \( g = 0 \).
\begin{lemma}
    In a pointed category with kernels and cokernels, \( f : A \to B \) is normal monic if and only if \( f \cong \ker \coker f \).
\end{lemma}
\begin{proof}
    If \( f \cong \ker \coker f \), it is clearly normal.
    Now suppose \( f = \ker g \).
    Then \( g \) factors through the cokernel of \( f \), so \( g (\ker \coker f) = 0 \).
    Thus \( \ker coker f \leq f \) in \( \operatorname{Sub}(B) \).
    But \( (\coker f) f = 0 \), so \( f \leq \ker \coker f \), so they are isomorphic as subobjects of \( B \).
\end{proof}
\begin{corollary}
    In a pointed category with kernels and cokernels, the operations \( \ker \) and \( \coker \) induce an order-reversing bijection between isomorphism classes of normal subobjects and isomorphism classes of normal quotients of any object.
\end{corollary}
\begin{remark}
    For any morphism \( f : A \to B \) in such a category, \( \ker \coker f \) is the smallest normal subobject of \( B \) through which \( f \) factors.
\end{remark}

\subsection{Abelian categories}
\begin{definition}
    An \emph{abelian category} is an additive category with all finite limits and colimits.
    Equivalently, an abelian category is a category with a zero object, finite biproducts, kernels, and cokernels, such that all monomorphisms and epimorphisms are normal.
\end{definition}
