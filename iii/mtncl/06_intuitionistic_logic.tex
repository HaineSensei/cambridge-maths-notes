\subsection{The Brouwer--Heyting--Kolmogorov interpretation}
We will construct a system of logic in which every proof contains evidence of its truth.
Our system will have the following properties, known as the Brouwer--Heyting--Kolmogorov interpretation.
\begin{enumerate}
    \item \( \bot \) has no proof.
    \item To prove \( \varphi \wedge \psi \), one must provide a proof of \( \varphi \) together with a proof of \( \psi \).
    \item To prove \( \varphi \Rightarrow \psi \), one must provide a mechanism for translating a proof of \( \varphi \) into a proof of \( \psi \).
    In particular, to prove \( \neg\varphi \), we must provide a way to turn a proof of \( \varphi \) into a contradiction.
    \item To prove \( \varphi \vee \psi \), we must specify either \( \varphi \) or \( \psi \), and then provide a proof for it.
    Note that in a classical setting, a proof of \( \varphi \vee \psi \) need not specify which of the two disjuncts is true.
    \item The law of the excluded middle \( \mathsf{LEM} \), which states \( \varphi \vee \neg \varphi \), is not valid.
    If this held for some proposition, we could decide whether the proposition was true or its negation is true, because any proof of \( \varphi \vee \neg \varphi \) contains this information.
    \item To prove \( \exists x.\, \varphi(x) \), one must provide a term \( t \) together with a proof of \( \varphi(t) \).
    \item To prove \( \forall x.\, \varphi(x) \), one must provide a mechanism that converts any term \( t \) into a proof of \( \varphi(t) \).
\end{enumerate}
This will be called \emph{intuitionistic (propositional) logic} \( \mathsf{IPC} \).
\begin{theorem}[Diaconescu]
    In intuitionistic \( \mathsf{ZF} \) set theory, the law of the excluded middle \( \mathsf{LEM} \) can be deduced from the axiom of choice \( \mathsf{AC} \).
\end{theorem}
\begin{proof}
    Let \( \varphi \) be a proposition; we want a proof of \( \varphi \vee \neg \varphi \).
    Using the axiom of separation, we have proofs that the following sets exist.
    \[ A = \qty{x \in \qty{0, 1} \mid \varphi \vee (x = 0)};\quad B = \qty{x \in \qty{0, 1} \mid \varphi \vee (x = 1)} \]
    These sets are \emph{inhabited}: there exists an element in each of them; in particular, \( 0 \in A \) and \( 1 \in A \) are intuitionistically valid.
    Note that being inhabited is strictly stronger than being nonempty in intuitionistic logic.
    This is because any proof that a set is inhabited contains information about an element in the set.
    The set \( \qty{A, B} \) is a family of inhabited sets, so by the axiom of choice, we have a choice function \( f : \qty{A, B} \to A \cup B \), and we have a proof that \( f(A) \in A \) and \( f(B) \in B \).
    Thus, we have a proof of
    \[ (\varphi \vee (f(A) = 0)) \wedge (\varphi \vee (f(B) = 1)) \]
    We also have a proof that \( f(A), f(B) \in \qty{0,1} \).
    In particular, we either have a proof that \( f(A) = 0 \) or we have a proof that \( f(A) = 1 \), and the same holds for \( B \).
    We have the following cases.
    \begin{enumerate}
        \item Suppose we have a proof that \( f(A) = 1 \).
        Then we have a proof of \( \varphi \vee (1 = 0) \), so we must have a proof of \( \varphi \).
        \item Suppose we have a proof that \( f(B) = 0 \).
        Then similarly we have a proof of \( \varphi \vee (0 = 1) \), so we must have a proof of \( \varphi \).
        \item Suppose we have proofs that \( f(A) = 0 \) and \( f(B) = 1 \).
        We will prove \( \neg \varphi \).
        Suppose that we have a proof of \( \varphi \).
        Then from a proof of \( \varphi \vee (x = 0) \) or \( \varphi \vee (x = 1) \) we can derive a proof of the other, so by the axiom of extensionality, \( A = B \).
        Then \( 0 = f(A) = f(B) = 1 \) as \( f \) is a function, giving a contradiction.
        Thus, we have constructed a proof of \( \neg\varphi \).
    \end{enumerate}
    We can always specify a proof of \( \varphi \) or a proof of \( \neg \varphi \), so we have \( \varphi \vee \neg \varphi \).
\end{proof}
\begin{remark}
    \begin{enumerate}
        \item Intuitionistic mathematics is more general than classical mathematics, because it operates on fewer assumptions.
        \item Notions that are classically conflated may be different in intuitionistic logic.
        For example, there is no classical distinction between inhabited and nonempty sets, but they are not the same in intuitionistic logic.
        Other examples include finiteness, or disequality and apartness.
        \item Intuitionistic proofs have computational content attached to them, but classical proofs may not.
        \item Intuitionistic logic is the internal logic of an arbitrary topos.
        % look at the Zariski topos
    \end{enumerate}
\end{remark}

\subsection{Natural deduction}
We will use the notation \( \Gamma \vdash \varphi \), or \( \Gamma \vdash_{\mathsf{IPC}} \varphi \), to denote that the set of \emph{open assumptions} \( \Gamma \) let us conclude \( \varphi \).
\( \Gamma \) is also called the \emph{context}.
We will inductively define this provability relation.
Some rules, called \emph{introduction rules}, let us construct proofs.
\begin{mathpar}
    \inferrule[\( \wedge \)-I]{\Gamma \vdash A \and \Gamma \vdash B}{\Gamma \vdash A \wedge B}
    \and
    \inferrule[\( \vee \)-I]{\Gamma \vdash A}{\Gamma \vdash A \vee B}
    \and
    \inferrule[\( \vee \)-I]{\Gamma \vdash B}{\Gamma \vdash A \vee B}
\end{mathpar}
Dually, some rules, called \emph{elimination rules}, let us extract information from proofs.
\begin{mathpar}
    \inferrule[\( \wedge \)-E]{\Gamma \vdash A \wedge B}{\Gamma \vdash A}
    \and
    \inferrule[\( \wedge \)-E]{\Gamma \vdash A \wedge B}{\Gamma \vdash B}
    \and
    \inferrule[\( \vee \)-E]{\Gamma, A \vdash C \and \Gamma, B \vdash C \and \Gamma \vdash A \vee B}{\Gamma \vdash C}
\end{mathpar}
We now define the principle of explosion, which is an elimination rule for \( \bot \).
We do not construct an introduction rule for \( \bot \).
\[ \inferrule[\( \bot \)-E]{\Gamma \vdash \bot}{\Gamma \vdash A} \]
We now define the introduction and elimination rules for implication.
The elimination rule is known as \emph{modus ponens}.
\begin{mathpar}
    \inferrule[\( \Rightarrow \)-I]{\Gamma, A \vdash B}{\Gamma \vdash A \Rightarrow B}
    \and
    \inferrule[\( \Rightarrow \)-E]{\Gamma \vdash A \Rightarrow B \and \Gamma \vdash A}{\Gamma \vdash B}
\end{mathpar}
We finally define a rule called the \emph{axiom schema}, that allows us to prove our assumptions.
\[ \inferrule[Ax]{\ }{\Gamma, A \vdash A} \]
If an inference rule moves an assumption out of the context, we say that the assumption is \emph{discharged} or \emph{closed}.
We are allowed to drop assumptions that we do not use; this is called the \emph{weakening} rule.
We obtain classical propositional logic \( \mathsf{CPC} \) by additionally adding one of the following two rules.
\begin{mathpar}
    \inferrule[\( \mathsf{LEM} \)]{\ }{\Gamma \vdash A \vee \neg A}
    \and
    \inferrule[\( \neg\neg \)-E]{\Gamma, \neg A \vdash \bot}{\Gamma \vdash A}
\end{mathpar}
We will additionally use the informal notation
\[ \inferrule*[right={\( (A, B) \)}]{{\begin{matrix} [A]\\\vdots\\X \end{matrix}} \and {\begin{matrix} [B]\\\vdots\\Y \end{matrix}}}{C} \]
to mean that if we can prove \( X \) assuming \( A \) and we can prove \( Y \) assuming \( B \), then we can infer \( C \) by discharging the open assumptions \( A \) and \( B \).
For example, we can write an instance of \( \Rightarrow \)-I as
\[ \inferrule*[right={\( (A) \)}]{\ {\begin{matrix} \Gamma, [A]\\\vdots\\B \end{matrix}}\ }{\Gamma \vdash A \Rightarrow B} \]
To extend this to intuitionistic predicate logic \( \mathsf{IQC} \), we need to add rules for quantifiers.
\begin{mathpar}
    \inferrule[\( \exists \)-I]{\Gamma \vdash \varphi[x := t]}{\Gamma \vdash \exists x.\, \varphi(x)}
    \and
    \inferrule[\( \forall \)-I]{\Gamma \vdash \varphi \and x \text{ not free in } \Gamma}{\Gamma \vdash \forall x.\, \varphi}
    \\\\
    \inferrule[\( \exists \)-E]{\Gamma \vdash \exists x.\, \varphi \and \Gamma, \varphi \vdash \psi \and x \text{ not free in } \Gamma}{\Gamma \vdash \psi}
    \and
    \inferrule[\( \forall \)-E]{\Gamma \vdash \forall x.\, \varphi}{\Gamma \vdash \varphi[x := t]}
\end{mathpar}
\begin{example}
    We will show that \( \vdash_{\mathsf{IPC}} A \wedge B \Rightarrow B \wedge A \).
    \[ \inferrule*[right=\( \Rightarrow \)-I]{\inferrule*[right=\( \wedge \)-I]{
        \inferrule*[right=\( \wedge \)-E]{[A \wedge B]}{B}
        \and
        \inferrule*[right=\( \wedge \)-E]{[A \wedge B]}{A}
    }{B \wedge A}}{A \wedge B \Rightarrow B \wedge A} \]
\end{example}
\begin{example}
    We will show that the logical axioms
    \[ \varphi \Rightarrow (\psi \Rightarrow \varphi);\quad (\varphi \Rightarrow (\psi \Rightarrow \chi)) \Rightarrow ((\varphi \Rightarrow \psi) \Rightarrow (\varphi \Rightarrow \chi)) \]
    are intuitionistically valid.
    \[ \inferrule*[right={(\( \Rightarrow \)-I, \( \varphi \))}]{
        \inferrule*[right={(\( \Rightarrow \)-I, \( \psi \))}]{
            \inferrule*[right=Ax]{[\varphi]}{\varphi} \and [\psi]
        }{\psi \Rightarrow \varphi}
    }{\varphi \Rightarrow (\psi \Rightarrow \varphi)} \]
    For the second axiom,
    \[ \inferrule*[right=\( (\varphi \Rightarrow (\psi \Rightarrow \chi)) \)]{
        \inferrule*[right=\( (\varphi \Rightarrow \psi) \)]{
            \inferrule*[right=\( (\varphi) \)]{
                \inferrule*[right=\( \Rightarrow \)-E]{
                    \inferrule*[right=\( \Rightarrow \)-E]{[\varphi \Rightarrow (\psi \Rightarrow \chi)] \and [\varphi]}{\psi \Rightarrow \chi}
                    \and
                    \inferrule*[right=\( \Rightarrow \)-E]{[\varphi \Rightarrow \psi] \and [\varphi]}{\psi}
                }{\chi}
            }{\varphi \Rightarrow \chi}
        }{(\varphi \Rightarrow \psi) \Rightarrow (\varphi \Rightarrow \chi)}
    }{(\varphi \Rightarrow (\psi \Rightarrow \chi)) \Rightarrow ((\varphi \Rightarrow \psi) \Rightarrow (\varphi \Rightarrow \chi))} \]
\end{example}
\begin{lemma}
    If \( \Gamma \vdash_{\mathsf{IPC}} \varphi \), then \( \Gamma, \psi \vdash_{\mathsf{IPC}} \varphi \).
    Moreover, if \( p \) is a primitive proposition and \( \psi \) is any proposition, then
    \[ \Gamma[p := \psi] \vdash_{\mathsf{IPC}} \varphi[p := \psi] \]
\end{lemma}
\begin{proof}
    This follows easily by induction over the length of the proof.
\end{proof}

\subsection{The simply typed lambda calculus}
For now, we will assume we are given a set \( \Pi \) of \emph{simple types}, generated by the grammar
\[ \Pi ::= \mathcal U \mid \Pi \to \Pi \]
where \( \mathcal U \) is a countable set of \emph{primitive types} or \emph{type variables}.
\begin{definition}
    Let \( V \) be an infinite set of variables.
    The set \( \Lambda_\Pi \) of \emph{simply typed \( \lambda \)-terms} is defined by the grammar
    \[ \Lambda_\Pi ::= V \mid \underbrace{\lambda V : \Pi .\, \Lambda_\Pi}_{\text{\( \lambda \)-abstraction}} \mid \underbrace{\Lambda_\Pi \, \Lambda_\Pi}_{\text{\( \lambda \)-application}} \]
\end{definition}
