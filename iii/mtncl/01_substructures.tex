\subsection{Notation}
The interpretation of a function symbol \( f \) in a model \( \mathcal M \) is denoted by \( f^{\mathcal M} \), and similarly the interpretation of a relation symbol \( R \) in \( \mathcal M \) is denoted by \( R^{\mathcal M} \).
If \( \mathcal M \) is an \( \mathcal L \)-structure, and \( A \subseteq \mathcal M \) is a subset, we will write \( \mathcal L_A \) for the language obtained by adding a new constant symbol \( a \) to the signature of \( \mathcal L \) for each element \( a \) of \( A \).
Then \( \mathcal M \) is naturally an \( \mathcal L_A \)-structure by interpreting the constants in the obvious way.
We will allow for the empty set to be an \( \mathcal L \)-structure.

\subsection{Homomorphisms and substructures}
\begin{definition}
    Let \( \mathcal M \) and \( \mathcal N \) be \( \mathcal L \)-structures.
    An \emph{\( \mathcal L \)-homomorphism} is a map \( \eta : \mathcal M \to \mathcal N \) that preserves the interpretations of the symbols in the language: given \( \vb a = (a_1, \dots, a_n) \in \mathcal M^n \),
    \begin{enumerate}
        \item for all function symbols \( f \) of arity \( n \), we have that
        \[ \eta(f^{\mathcal M}(\vb a)) = f^{\mathcal N}(\eta(\vb a)) \]
        \item for all relation symbols \( r \) of arity \( n \), we have that
        \[ \vb a \in R^{\mathcal M} \iff \eta(\vb a) \in R^{\mathcal N} \]
    \end{enumerate}
    An injective \( \mathcal L \)-homomorphism is called an \emph{\( \mathcal L \)-embedding}.
    An invertible \( \mathcal L \)-homomorphism is called an \emph{\( \mathcal L \)-isomorphism}.
\end{definition}
\begin{definition}
    If \( \mathcal M \subseteq \mathcal N \) and the inclusion map is an \( \mathcal L \)-homomorphism, we say that \( \mathcal M \) is a \emph{substructure} of \( \mathcal N \), and that \( \mathcal N \) is an \emph{extension} of \( \mathcal M \).
\end{definition}
\begin{example}
    \begin{enumerate}
        \item Let \( \mathcal L \) be the language of groups.
        Then \( (\mathbb N, +, 0) \) is a substructure of \( (\mathbb Z, +, 0) \), but it is not a subgroup.
        \item If \( \mathcal M \) is an \( \mathcal L \)-structure, \( X \) is the domain of a substructure of \( \mathcal M \) if and only if it is closed under the interpretations of all function symbols.
        The forward implication is clear.
        If \( f \) is a function symbol of arity \( n \) and \( X \) is closed under \( f^{\mathcal M} \), \( \eval{f^{\mathcal M}}_{X^n} \) is a function \( X^n \to X \) interpreting \( f \) on the domain \( X \), as required.
        In particular, any substructure should also contain all of the constants in the language.
        \item The substructure \emph{generated by} a subset \( X \subseteq \mathcal M \) is given by the smallest set that contains \( X \) and is closed under the interpretations of all function symbols in \( \mathcal M \).
        This is denoted \( \langle X \rangle_{\mathcal M} \), and one can check that for infinite \( \mathcal L \) (but not necessarily infinite signature),
        \[ \abs{\langle X \rangle_{\mathcal M}} \leq \abs{X} + \abs{\mathcal L} \]
        We prove this by iteratively closing up \( X \) by applying interpretations of function symbols to elements of \( X \), and then taking the union of the resulting sets.
        At each stage, for each function symbol \( f \) of arity \( n \), we add at most \( \abs{X}^n \leq \abs{X} \cdot \aleph_0 \) new elements.
        So in a single stage, we add at most \( \abs{X} \cdot \aleph_0 \cdot \abs{\mathcal L} = \abs{X} \cdot \abs{\mathcal L} \) new elements to \( X \).
        Repeating this \( \omega \) times, the final set has size at most
        \begin{align*}
            \abs{X} + \abs{X} \cdot \abs{\mathcal L} + \abs{X} \cdot \abs{\mathcal L}^2 + \cdots &= \abs{X} (1 + \abs{\mathcal L} + \abs{\mathcal L}^2 + \cdots) \\
            &\leq \abs{X} (\abs{\mathcal L} + \abs{\mathcal L} + \abs{\mathcal L} + \cdots) \\
            &= \abs{X} \cdot \abs{\mathcal L} \cdot \aleph_0 \\
            &= \abs{X} \cdot \abs{\mathcal L}
        \end{align*}
        We say that \( \mathcal M \) is \emph{finitely generated} if there exists a finite subset \( X \subseteq \mathcal M \) such that \( \mathcal M = \langle X \rangle_{\mathcal M} \).
        \item Consider
        \[ (\mathbb R, \cdot, -1) \vDash \neg \exists x.\, (x^2 = -1) \]
        But it has an extension \( (\mathbb C, \cdot, -1) \) that does not model this sentence.
    \end{enumerate}
\end{example}
\begin{proposition}
    Let \( \varphi(\vb x) \) be a quantifier-free \( \mathcal L \)-formula with \( n \) free variables.
    Let \( \mathcal M \) be an \( \mathcal L \)-structure, and let \( \vb a \) be an \( n \)-tuple in \( \mathcal M \).
    Then for every extension \( \mathcal N \) of \( \mathcal M \),
    \[ \mathcal M \vDash \varphi(\vb a) \iff \mathcal N \vDash \varphi(\vb a) \]
\end{proposition}
\begin{proof}
    We proceed by induction on the structure of formulae.
    First, we show that if \( t(\vb x) \) is a term with \( k \) free variables, then
    \[ t^{\mathcal M}(\vb b) = t^{\mathcal N}(\vb b) \]
    for all \( \vb b \in \mathcal M^k \).
    It is clearly the case if \( t = x_i \) is a variable, as both structures interpret \( t(\vb b) \) as \( b_i \).
    Suppose \( t \) is a term of the form \( t = f(q_1, \dots, q_\ell) \) for \( f \) a function symbol of arity \( \ell \) and the \( q_i \) are terms.
    By the inductive hypothesis we have
    \[ q_i^{\mathcal M}(\vb b) = q_i^{\mathcal N}(\vb b) \]
    Therefore,
    \begin{align*}
        t^{\mathcal M}(\vb b) &= f^{\mathcal M}(q_1^{\mathcal M}(\vb b), \dots, q_\ell^{\mathcal M}(\vb b)) \\
        &= f^{\mathcal N}(q_1^{\mathcal M}(\vb b), \dots, q_\ell^{\mathcal M}(\vb b)) \\
        &= f^{\mathcal N}(q_1^{\mathcal N}(\vb b), \dots, q_\ell^{\mathcal N}(\vb b)) \\
        &= t^{\mathcal N}(\vb b)
    \end{align*}
    Thus terms are interpreted the same way in both models.
    For terms \( t_1, t_2 \) with the same free variables \( \vb x \), then for any choice of \( \vb a \),
    \begin{align*}
        \mathcal M \vDash (t_1(\vb x) = t_2(\vb x)) &\iff t_1^{\mathcal M}(\vb a) = t_2^{\mathcal M}(\vb a) \\
        &\iff t_1^{\mathcal N}(\vb a) = t_2^{\mathcal M}(\vb b) \\
        &\iff \mathcal N \vDash (t_1(\vb x) = t_2(\vb x))
    \end{align*}
    Let \( R \) be a relation symbol of arity \( n \), and let \( t_1, \dots, t_n \) be terms with the same free variables \( \vb x \).
    \begin{align*}
        \mathcal M \vDash R(t_1(\vb x), \dots, t_n(\vb x)) &\iff (t_1^{\mathcal M}(\vb a), \dots, t_n^{\mathcal M}(\vb a)) \in R^{\mathcal M} \\
        &\iff (t_1^{\mathcal M}(\vb a), \dots, t_n^{\mathcal M}(\vb a)) \in R^{\mathcal N} \\
        &\iff (t_1^{\mathcal N}(\vb a), \dots, t_n^{\mathcal N}(\vb a)) \in R^{\mathcal N} \\
        &\iff \mathcal N \vDash R(t_1(\vb x), \dots, t_n(\vb x))
    \end{align*}
    So the result holds for all atomic formulae.
    For connectives, note that
    \begin{align*}
        \mathcal M \vDash \neg \varphi &\iff \mathcal M \nvDash \varphi \\
        &\iff \mathcal N \nvDash \varphi \\
        &\iff \mathcal N \vDash \neg \varphi
    \end{align*}
    and
    \begin{align*}
        \mathcal M \vDash \varphi \wedge \psi &\iff (\mathcal M \vDash \varphi) \wedge (\mathcal M \vDash \psi) \\
        &\iff (\mathcal N \vDash \varphi) \wedge (\mathcal N \vDash \psi) \\
        &\iff \mathcal N \vDash \varphi \wedge \psi
    \end{align*}
    As quantifier-free formulae can be built out of atomic formulae, negation, and conjunction, we have completed the proof.
\end{proof}
