\subsection{Notation}
The interpretation of a function symbol \( f \) in a model \( \mathcal M \) is denoted by \( f^{\mathcal M} \), and similarly the interpretation of a relation symbol \( R \) in \( \mathcal M \) is denoted by \( R^{\mathcal M} \).
If \( \mathcal M \) is an \( \mathcal L \)-structure, and \( A \subseteq \mathcal M \) is a subset, we will write \( \mathcal L_A \) for the language obtained by adding a new constant symbol \( a \) to the signature of \( \mathcal L \) for each element \( a \) of \( A \).
Then \( \mathcal M \) is naturally an \( \mathcal L_A \)-structure by interpreting the constants in the obvious way.
We will allow for the empty set to be an \( \mathcal L \)-structure.

\subsection{Homomorphisms and substructures}
\begin{definition}
    Let \( \mathcal M \) and \( \mathcal N \) be \( \mathcal L \)-structures.
    An \emph{\( \mathcal L \)-homomorphism} is a map \( \eta : \mathcal M \to \mathcal N \) that preserves the interpretations of the symbols in the language: given \( \vb a = (a_1, \dots, a_n) \in \mathcal M^n \),
    \begin{enumerate}
        \item for all function symbols \( f \) of arity \( n \), we have that
        \[ \eta(f^{\mathcal M}(\vb a)) = f^{\mathcal N}(\eta(\vb a)) \]
        \item for all relation symbols \( r \) of arity \( n \), we have that
        \[ \vb a \in R^{\mathcal M} \iff \eta(\vb a) \in R^{\mathcal N} \]
    \end{enumerate}
    An injective \( \mathcal L \)-homomorphism is called an \emph{\( \mathcal L \)-embedding}.
    An invertible \( \mathcal L \)-homomorphism is called an \emph{\( \mathcal L \)-isomorphism}.
\end{definition}
\begin{definition}
    If \( \mathcal M \subseteq \mathcal N \) and the inclusion map is an \( \mathcal L \)-homomorphism, we say that \( \mathcal M \) is a \emph{substructure} of \( \mathcal N \), and that \( \mathcal N \) is an \emph{extension} of \( \mathcal M \).
    We will typically use the notation \( \mathcal M \subseteq \mathcal N \) to indicate that \( \mathcal M \) is a substructure of \( \mathcal N \) when both are \( \mathcal L \)-structures, not just that it is a subset.
\end{definition}
\begin{example}
    \begin{enumerate}
        \item Let \( \mathcal L \) be the language of groups.
        Then \( (\mathbb N, +, 0) \) is a substructure of \( (\mathbb Z, +, 0) \), but it is not a subgroup.
        \item If \( \mathcal M \) is an \( \mathcal L \)-structure, \( X \) is the domain of a substructure of \( \mathcal M \) if and only if it is closed under the interpretations of all function symbols.
        The forward implication is clear.
        If \( f \) is a function symbol of arity \( n \) and \( X \) is closed under \( f^{\mathcal M} \), \( \eval{f^{\mathcal M}}_{X^n} \) is a function \( X^n \to X \) interpreting \( f \) on the domain \( X \), as required.
        In particular, any substructure should also contain all of the constants in the language.
        \item The substructure \emph{generated by} a subset \( X \subseteq \mathcal M \) is given by the smallest set that contains \( X \) and is closed under the interpretations of all function symbols in \( \mathcal M \).
        This is denoted \( \langle X \rangle_{\mathcal M} \), and one can check that for infinite \( \mathcal L \) (but not necessarily infinite signature),
        \[ \abs{\langle X \rangle_{\mathcal M}} \leq \abs{X} + \abs{\mathcal L} \]
        We prove this by iteratively closing up \( X \) by applying interpretations of function symbols to elements of \( X \), and then taking the union of the resulting sets.
        At each stage, for each function symbol \( f \) of arity \( n \), we add at most \( \abs{X}^n \leq \abs{X} \cdot \aleph_0 \) new elements.
        So in a single stage, we add at most \( \abs{X} \cdot \aleph_0 \cdot \abs{\mathcal L} = \abs{X} \cdot \abs{\mathcal L} \) new elements to \( X \).
        Repeating this \( \omega \) times, the final set has size at most
        \begin{align*}
            \abs{X} + \abs{X} \cdot \abs{\mathcal L} + \abs{X} \cdot \abs{\mathcal L}^2 + \cdots &= \abs{X} (1 + \abs{\mathcal L} + \abs{\mathcal L}^2 + \cdots) \\
            &\leq \abs{X} (\abs{\mathcal L} + \abs{\mathcal L} + \abs{\mathcal L} + \cdots) \\
            &= \abs{X} \cdot \abs{\mathcal L} \cdot \aleph_0 \\
            &= \abs{X} \cdot \abs{\mathcal L}
        \end{align*}
        We say that \( \mathcal M \) is \emph{finitely generated} if there exists a finite subset \( X \subseteq \mathcal M \) such that \( \mathcal M = \langle X \rangle_{\mathcal M} \).
        \item Consider
        \[ (\mathbb R, \cdot, -1) \vDash \neg \exists x.\, (x^2 = -1) \]
        But it has an extension \( (\mathbb C, \cdot, -1) \) that does not model this sentence.
    \end{enumerate}
\end{example}
\begin{proposition}
    Let \( \varphi(\vb x) \) be a quantifier-free \( \mathcal L \)-formula with \( n \) free variables.
    Let \( \mathcal M \) be an \( \mathcal L \)-structure, and let \( \vb a \) be an \( n \)-tuple in \( \mathcal M \).
    Then for every extension \( \mathcal N \) of \( \mathcal M \),
    \[ \mathcal M \vDash \varphi(\vb a) \iff \mathcal N \vDash \varphi(\vb a) \]
\end{proposition}
\begin{proof}
    We proceed by induction on the structure of formulae.
    First, we show that if \( t(\vb x) \) is a term with \( k \) free variables, then
    \[ t^{\mathcal M}(\vb b) = t^{\mathcal N}(\vb b) \]
    for all \( \vb b \in \mathcal M^k \).
    It is clearly the case if \( t = x_i \) is a variable, as both structures interpret \( t(\vb b) \) as \( b_i \).
    Suppose \( t \) is a term of the form \( t = f(q_1, \dots, q_\ell) \) for \( f \) a function symbol of arity \( \ell \) and the \( q_i \) are terms.
    By the inductive hypothesis we have
    \[ q_i^{\mathcal M}(\vb b) = q_i^{\mathcal N}(\vb b) \]
    Therefore,
    \begin{align*}
        t^{\mathcal M}(\vb b) &= f^{\mathcal M}(q_1^{\mathcal M}(\vb b), \dots, q_\ell^{\mathcal M}(\vb b)) \\
        &= f^{\mathcal N}(q_1^{\mathcal M}(\vb b), \dots, q_\ell^{\mathcal M}(\vb b)) \\
        &= f^{\mathcal N}(q_1^{\mathcal N}(\vb b), \dots, q_\ell^{\mathcal N}(\vb b)) \\
        &= t^{\mathcal N}(\vb b)
    \end{align*}
    Thus terms are interpreted the same way in both models.
    For terms \( t_1, t_2 \) with the same free variables \( \vb x \), then for any choice of \( \vb a \),
    \begin{align*}
        \mathcal M \vDash (t_1(\vb x) = t_2(\vb x)) &\iff t_1^{\mathcal M}(\vb a) = t_2^{\mathcal M}(\vb a) \\
        &\iff t_1^{\mathcal N}(\vb a) = t_2^{\mathcal M}(\vb b) \\
        &\iff \mathcal N \vDash (t_1(\vb x) = t_2(\vb x))
    \end{align*}
    Let \( R \) be a relation symbol of arity \( n \), and let \( t_1, \dots, t_n \) be terms with the same free variables \( \vb x \).
    \begin{align*}
        \mathcal M \vDash R(t_1(\vb x), \dots, t_n(\vb x)) &\iff (t_1^{\mathcal M}(\vb a), \dots, t_n^{\mathcal M}(\vb a)) \in R^{\mathcal M} \\
        &\iff (t_1^{\mathcal M}(\vb a), \dots, t_n^{\mathcal M}(\vb a)) \in R^{\mathcal N} \\
        &\iff (t_1^{\mathcal N}(\vb a), \dots, t_n^{\mathcal N}(\vb a)) \in R^{\mathcal N} \\
        &\iff \mathcal N \vDash R(t_1(\vb x), \dots, t_n(\vb x))
    \end{align*}
    So the result holds for all atomic formulae.
    For connectives, note that
    \begin{align*}
        \mathcal M \vDash \neg \varphi &\iff \mathcal M \nvDash \varphi \\
        &\iff \mathcal N \nvDash \varphi \\
        &\iff \mathcal N \vDash \neg \varphi
    \end{align*}
    and
    \begin{align*}
        \mathcal M \vDash \varphi \wedge \psi &\iff (\mathcal M \vDash \varphi) \wedge (\mathcal M \vDash \psi) \\
        &\iff (\mathcal N \vDash \varphi) \wedge (\mathcal N \vDash \psi) \\
        &\iff \mathcal N \vDash \varphi \wedge \psi
    \end{align*}
    As quantifier-free formulae can be built out of atomic formulae, negation, and conjunction, we have completed the proof.
\end{proof}

\subsection{Elementary equivalence}
\begin{definition}
    Structures \( \mathcal M, \mathcal N \) are called \emph{elementarily equivalent} if for every \( \mathcal L \)-sentence,
    \[ \mathcal M \vDash \varphi \iff \mathcal N \vDash \varphi \]
    A map \( f : \mathcal M \to \mathcal N \) is an \emph{elementary embedding} if it is injective, and for all \( \mathcal L \)-formulae \( \varphi(x_1, \dots, x_n) \) and elements \( m_1, \dots, m_n \in \mathcal M \), we have
    \[ \mathcal M \vDash \varphi(m_1, \dots, m_n) \iff \mathcal N \vDash \varphi(f(m_1), \dots, f(m_n)) \]
\end{definition}
If there is an elementary embedding between two structures, they are elementarily equivalent.
If \( \mathcal M \) and \( \mathcal N \) are elementarily equivalent, we write \( \mathcal M \equiv \mathcal N \).
\begin{remark}
    If \( \mathcal M \) and \( \mathcal N \) are \( \mathcal L \)-structures, and \( \vb m \in \mathcal M, \vb n \in \mathcal N \) are ordered tuples of the same length \( k \), then by
    \[ (\mathcal M, \vb m) \equiv (\mathcal N, \vb n) \]
    we view \( (\mathcal M, \vb m) \) and \( (\mathcal N, \vb n) \) as structures over \( \mathcal L \) with \( k \) additional constants, interpreting these new constants as the elements of \( \vb m \) and \( \vb n \) respectively.
\end{remark}
\begin{proposition}
    If \( \mathcal M \cong \mathcal N \), then \( \mathcal M \equiv \mathcal N \).
\end{proposition}
This can be easily shown by induction.
The converse is generally not true, for example if the structures are infinite.
\begin{definition}
    A substructure \( \mathcal M \subseteq \mathcal N \) is an \emph{elementary substructure} if the inclusion map is an elementary embedding.
    In this case, we also say that \( \mathcal N \) is an \emph{elementary extension} of \( \mathcal M \).
    We write \( \mathcal M \preceq \mathcal N \).
\end{definition}

\subsection{Categorical and complete theories}
Recall that a theory \( \mathcal T \) is \emph{complete} if either \( \mathcal T \vdash \varphi \) or \( \mathcal T \vdash \neg\varphi \) for all sentences \( \varphi \).
Then any two models of a complete theory are elementarily equivalent, but they may have different cardinalities.
\begin{definition}
    A theory \( \mathcal T \) is \emph{model-complete} if every embedding between models of \( \mathcal T \) is elementary.
\end{definition}
\begin{definition}
    Let \( \kappa \) be an infinite cardinal.
    A theory \( \mathcal T \) is \emph{\( \kappa \)-categorical} if all models of \( \mathcal T \) of cardinality \( \kappa \) are isomorphic.
\end{definition}
It turns out that if theory on a countable language is categorical for some uncountable cardinal, then it is categorical for all infinite cardinals.
\begin{proposition}[Vaught's test]
    Let \( \mathcal T \) be a consistent \( \mathcal L \)-theory that has no finite models.
    If \( \mathcal T \) is \( \kappa \)-categorical for some infinite \( \kappa \geq \abs{\mathcal L} \), then \( \mathcal T \) is complete.
\end{proposition}
\begin{proof}
    Suppose there is some \( \varphi \) such that \( \mathcal T \nvdash \varphi \) and \( \mathcal T \nvdash \neg \varphi \).
    Then \( \mathcal T \cup \qty{\varphi} \) and \( \mathcal T \cup \qty{\neg\varphi} \) are consistent theories, so have models.
    As \( \mathcal T \) has no finite models, these two models are infinite.
    In fact, by the L\"owenheim--Skolem theorem, the models can be forced to have size \( \kappa \).
    But these models are in particular models of \( \mathcal T \), so they must be isomorphic.
    Since they are isomorphic, they are elementarily equivalent.
    But the models disagree on the truth value of \( \varphi \), giving a contradiction.
\end{proof}
\begin{example}
    \begin{enumerate}
        \item Any two countable dense linear orders are isomorphic, so the theory of dense linear orders without endpoints is \( \aleph_0 \)-categorical.
        Thus, by Vaught's test, the theory \( \mathsf{DLO} \) of dense linear orders without endpoints is complete.
        \item Let \( F \) be a field.
        The theory of infinite (not infinite-dimensional) \( F \)-vector spaces is \( \kappa \)-categorical for \( \kappa > \abs{F} \). % (exercise)
        Hence, the theory is complete.
    \end{enumerate}
\end{example}

\subsection{Tarski--Vaught test}
\begin{proposition}
    Let \( \mathcal N \) be an \( \mathcal L \)-structure, and let \( M \subseteq \mathcal N \).
    Then \( M \) is the domain of an elementary substructure if and only if for any formula \( \varphi(x, \vb t) \) and tuple \( \vb m \in M \), if there exists a witness \( n \in \mathcal N \) such that \( \mathcal N \vDash \varphi(n, \vb m) \), then there is a witness \( \hat n \in M \) such that \( \mathcal N \vDash \varphi(\hat n, \vb m) \).
\end{proposition}
\begin{proof}
    If \( M \) is the domain of an elementary substructure \( \mathcal M \), then \( \mathcal N \vDash \exists x.\, \varphi(x, \vb m) \) implies that \( \mathcal M \vDash \exists x.\, \varphi(x, \vb m) \).
    Thus \( \mathcal M \vDash \varphi(\hat m, \vb m) \) for some \( \hat m \in \mathcal M \).
    But then \( \mathcal N \vDash \varphi(\hat m, \vb m) \), as required.

    For the other implication, if \( M \subseteq \mathcal N \) has the stated property, we first show that \( M \) is closed under the interpretation of function symbols.
    Consider the formulae \( \varphi_f(x, \vb t) = (x = f(\vb t)) \) for each function symbol \( f \) in \( \mathcal L \).
    Then for any \( \vb m \in M \), there exists \( n \in \mathcal N \) such that \( \mathcal N \vDash n = f(\vb m) \), but then by hypothesis, there exists \( \hat m \in M \) such that \( \mathcal N \vDash \hat m = f(\vb m) \).
    Thus \( f(\vb m) = \hat m \in M \).
    Interpreting relation symbols on \( M \) in the obvious way, we turn \( M \) into an \( \mathcal L \)-structure \( \mathcal M \), which is clearly a substructure of \( \mathcal N \).

    It now remains to show that the substructure \( \mathcal M \) of \( \mathcal N \) is elementary.
    This follows from induction over the number of quantifiers in formulae, noting that the truth values of quantifier-free formulae are always preserved under any extension.
\end{proof}

\subsection{Universal theories and the method of diagrams}
\begin{definition}
    A formula \( \varphi \) is \emph{universal} if it is of the form \( \forall \vb x.\, \psi(\vb x, \vb y) \) where \( \psi \) is quantifier-free.
    A theory is \emph{universal} if all its axioms are universal sentences.
\end{definition}
\begin{definition}
    Let \( \mathcal N \) be an \( \mathcal L \)-structure.
    We define the \emph{diagram} of \( \mathcal N \) to be the set
    \[ \operatorname{Diag} \mathcal N = \qty{\varphi(n_1, \dots n_k) \mid \varphi \text{ is a quantifier-free } \mathcal L_{\mathcal N} \text{-formula}, \mathcal N \vDash \varphi(n_1, \dots, n_k)} \]
    The \emph{elementary diagram} of \( \mathcal N \) is
    \[ \operatorname{Diag}_{\text{el}} \mathcal N = \qty{\varphi(n_1, \dots n_k) \mid \varphi \text{ is an } \mathcal L_{\mathcal N} \text{-formula}, \mathcal N \vDash \varphi(n_1, \dots, n_k)} \]
\end{definition}
The diagram of a group is a slight generalisation of its multiplication table.
Note that a model of a diagram is the same as an extension, and a model of an elementary diagram is the same as an elementary extension.
\begin{lemma}
    Let \( \mathcal T \) be a consistent theory, and let \( \mathcal T_\forall \) be the theory of universal sentences proven by \( \mathcal T \).
    If \( \mathcal N \) is a model of \( \mathcal T_\forall \), then \( \mathcal T \cup \operatorname{Diag} {\mathcal N} \) is consistent.
\end{lemma}
\begin{proof}
    Suppose \( \mathcal T \cup \operatorname{Diag} {\mathcal N} \) is inconsistent.
    As \( \mathcal T \) is consistent, by compactness there must be a finite number of sentences in the diagram \( \operatorname{Diag} {\mathcal N} \) that are inconsistent with \( \mathcal T \).
    Taking the conjunction, we can reduce to the case where there is a single sentence \( \varphi(\vb n) \) that is inconsistent with \( \mathcal T \).
    Then as \( \mathcal T \cup \qty{\varphi(\vb n)} \) is inconsistent, \( \mathcal T \vdash \neg\varphi(\vb n) \).
    Since \( \mathcal T \) has nothing to say about the new constants \( \vb n \), we must in fact have \( \mathcal T \vdash \forall \vb x.\, \neg \varphi(\vb x) \).
    This is a universal consequence of \( \mathcal T \), so by assumption \( \mathcal N \) models it, giving a contradiction.
\end{proof}
\begin{corollary}[Tarski, \L{}o\'s]
    An \( \mathcal L \)-theory \( \mathcal T \) has a universal axiomatisation if and only if it is preserved under substructures.
    That is, if \( \mathcal M \subseteq \mathcal N \) are substructures and \( \mathcal M \vDash \mathcal T \) then \( \mathcal N \vDash \mathcal T \).
    Dually, a theory has an existential axiomatisation if and only if it is preserved under extensions.
\end{corollary}
\begin{proof}
    One direction is clear.
    Suppose \( \mathcal T \) is preserved under taking substructures.
    If \( \mathcal N \vDash \mathcal T \), then \( \mathcal N \vDash \mathcal T_\forall \); we show that the converse also holds.
    By the previous proposition, \( \mathcal T \cup \operatorname{Diag} \mathcal N \) is consistent.
    Let \( \mathcal N^\star \) be a model of this theory.
    So \( \mathcal N^\star \) is an extension of \( \mathcal N \), and also models \( \mathcal T \).
    But as \( \mathcal T \) is preserved under substructures, \( \mathcal N \) must model \( \mathcal T \).
\end{proof}
We can show much more with the same method.
\begin{theorem}[elementary amalgamation theorem]
    Let \( \mathcal M, \mathcal N \) be \( \mathcal L \)-structures, and \( \vb m \in \mathcal M, \vb n \in \mathcal N \) be tuples of the same size such that \( (\mathcal M, \vb m) \equiv (\mathcal N, \vb n) \).
    Then there is an elementary extension \( \mathcal K \) of \( M \) and an elementary embedding \( g : \mathcal N \hookrightarrow \mathcal K \) mapping each \( n_i \) to \( m_i \).
\end{theorem}
\begin{proof}
    Replacing \( \mathcal N \) with an isomorphic copy if required, we can assume \( \vb m = \vb n \), and that \( \mathcal M \) and \( \mathcal N \) have no other common elements.
    We show that the theory
    \[ \mathcal T = \operatorname{Diag}_{\text{el}} \mathcal M \cup \operatorname{Diag}_{\text{el}} \mathcal N \]
    is consistent, using compactness.
    Suppose that \( \Phi \) is a finite subset of sentences in \( \mathcal T \), which of course includes only finitely many sentences in \( \operatorname{Diag}_{\text{el}} \mathcal N \).
    Let the conjunction of those sentences be written as \( \varphi(\vb m, \vb k) \), where \( \varphi(\vb x, \vb y) \) is an \( \mathcal L_{\mathcal N} \)-formula, and \( \vb k \) are pairwise distinct elements of \( \mathcal N \setminus \vb m \).
    If \( \Phi \) is inconsistent, then \( \operatorname{Diag}_{\text{el}} \mathcal M \vdash \neg \varphi(\vb m, \vb k) \).
    Since the elements of \( \vb k \) are distinct and not in \( \mathcal M \), we in fact have \( \operatorname{Diag}_{\text{el}} \mathcal M \vdash \forall \vb y.\, \neg \varphi(\vb m, \vb y) \).
    % TODO: why not in \mathcal M? why not just not in \vb m?
    In particular, \( (\mathcal M, \vb m) \vDash \forall y.\, \neg \varphi(\vb m, \vb y) \).
    By hypothesis, \( (\mathcal N, \vb n) \vDash \forall y.\, \neg \varphi(\vb m, \vb y) \).
    This is a contradiction, as \( \varphi(\vb m, \vb k) \in \operatorname{Diag}_{\text{el}} \mathcal N \).
    Hence \( \mathcal T \) is consistent.
    Take \( \mathcal K \) to be the \( \mathcal L \)-reduct of a model of \( \mathcal T \).
\end{proof}
We can also use this technique to constrain the size of a model.
\begin{theorem}[L\"owenheim--Skolem theorem]
    Let \( \mathcal M \) be an infinite \( \mathcal L \)-structure.
    Let \( \kappa \geq \abs{\mathcal L} \) be an infinite cardinal.
    Then,
    \begin{enumerate}
        \item if \( \kappa < \abs{\mathcal M} \), there is an elementary substructure of \( \mathcal M \) of size \( \kappa \);
        \item if \( \kappa > \abs{\mathcal M} \), there is an elementary extension of \( \mathcal M \) of size \( \kappa \).
    \end{enumerate}
\end{theorem}
We postpone the proof of part (i).
\begin{proof}
    Expand the language \( \mathcal L \) by adding constant symbols for each \( m \in \mathcal M \) and \( c \in \kappa \).
    Let
    \[ \mathcal T = \operatorname{Diag}_{\text{el}} \mathcal M \cup \bigcup_{c \neq c' \in \kappa} \qty{\neg(c = c')} \]
    \( \mathcal T \) has a model by compactness, and this model must be an elementary extension of \( \mathcal M \) with size at least \( \kappa \).
    We then apply the downward L\"owenheim--Skolem theorem if necessary to obtain a model of size exactly \( \kappa \).
\end{proof}
For example, if \( \mathcal L \) is countable, every infinite \( \mathcal L \)-structure has a countable elementary substructure.
