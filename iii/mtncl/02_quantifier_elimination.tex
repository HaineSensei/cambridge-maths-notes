\subsection{Skolem functions}
\begin{definition}
    Let \( \mathcal T \) be an \( \mathcal L \)-theory, and let \( \varphi(\vb x, y) \) be an \( \mathcal L \)-formula where \( \vb x \) is nonempty.
    A \emph{Skolem function} for \( \varphi \) is an \( \mathcal L \)-term \( t \) such that
    \[ \mathcal T \vdash \forall \vb x.\, (\exists y.\, \varphi(\vb x, y) \Rightarrow \varphi(\vb x, t(\vb x))) \]
    A \emph{skolemisation} of an \( \mathcal L \)-theory \( \mathcal T \) is a language \( \mathcal L^+ \supseteq \mathcal L \) and an \( \mathcal L^+ \)-theory \( \mathcal T^+ \supseteq \mathcal T \) such that
    \begin{enumerate}
        \item every \( \mathcal L \)-structure that models \( \mathcal T \) can be expanded to an \( \mathcal L^+ \)-structure that models \( \mathcal T^+ \);
        \item \( \mathcal T^+ \) has Skolem functions for any \( \mathcal L^+ \)-formula \( \varphi(\vb x, y) \) where \( \vb x \) is nonempty.
    \end{enumerate}
    A theory is called a \emph{Skolem theory} if it is a skolemisation of itself.
\end{definition}
By `expanded', we mean that \( \mathcal T \) is given interpretations to the elements of \( \mathcal L^+ \setminus \mathcal L \), but no new objects are added.
\begin{proposition}
    Let \( \mathcal T \) be an \( \mathcal L \)-theory, and let \( \mathcal F \) be a collection of \( \mathcal L \)-formulae including all atomic formulae and closed under Boolean operations.
    Suppose that for every formula \( \psi(\vb x, y) \in F \), there exists \( \varphi(\vb x) \in F \) with
    \[ T \vdash \forall \vb x.\, (\exists y,\. \psi(\vb x, y) \Leftrightarrow \varphi(\vb x)) \]
    Then, every \( \mathcal L \)-formula is equivalent to one in \( F \) with the same free variables modulo \( \mathcal T \) (that is, \( \mathcal T \) proves they are equivalent).
\end{proposition}
This is proven on the first example sheet.
% TODO: just do the proof here
\begin{proposition}
    Let \( \mathcal T \) be a Skolem theory.
    Then,
    \begin{enumerate}
        \item every \( \mathcal L \)-formula \( \varphi(\vb x) \) where \( \vb x \) is nonempty is equivalent modulo \( \mathcal T \) to some quantifier-free \( \varphi^\star(\vb x) \);
        \item if \( \mathcal N \vDash \mathcal T \) and \( X \subseteq \mathcal N \), then either \( \langle X \rangle_{\mathcal N} = \varnothing \) or \( \langle X \rangle_{\mathcal N} \preceq \mathcal N \).
    \end{enumerate}
\end{proposition}
\begin{remark}
    When \( \mathcal N \) is a model of a Skolem theory, \( \langle X \rangle_{\mathcal N} \) is sometimes called the \emph{Skolem hull} of \( X \).
\end{remark}
\begin{proof}
    \emph{Part (i).}
    Clearly, \( \varphi(\vb x, t(\vb x)) \Rightarrow \exists y.\, \varphi(\vb x, y) \) in any model.
    So having Skolem functions mean that
    \[ T \vdash \forall \vb x.\, (\exists y.\, \varphi(\vb x, y) \Leftrightarrow \varphi(\vb x, t(\vb x))) \]
    completing the proof by the previous proposition.

    \emph{Part (ii).}
    We proceed by the Tarski--Vaught test.
    Let \( \mathcal M = \langle X \rangle_{\mathcal N}, \vb m \in \mathcal M \), and let \( \varphi(\vb x, y) \) be such that
    \[ \mathcal N \vDash \exists y.\, \varphi(\vb m, y) \]
    Then as \( \mathcal N \) has Skolem functions, there exists an \( \mathcal L \)-term \( t \) such that
    \[ \mathcal N \vDash \varphi(\vb m, t(\vb m)) \]
    But \( \mathcal M \) is closed under the interpretation of function symbols as it is a substructure, so \( t(\vb m) \in \mathcal M \).
    Thus
    \[ \mathcal M \vDash \exists y.\, \varphi(\vb m, y) \]
    as required.
\end{proof}

\subsection{Skolemisation theorem}
\begin{theorem}
    Every first-order language \( \mathcal L \) can be expanded to some \( \mathcal L^+ \supseteq \mathcal L \) that includes an \( \mathcal L^+ \)-theory \( \Sigma \) such that
    \begin{enumerate}
        \item \( \Sigma \) is a Skolem \( \mathcal L^+ \)-theory;
        \item any \( \mathcal L \)-structure can be expanded to an \( \mathcal L^+ \)-structure that models \( \Sigma \); and
        \item \( \abs{\mathcal L^+} = \abs{\mathcal L} \).
    \end{enumerate}
\end{theorem}
\begin{proof}
    We will design \( \mathcal L^+ \) to include Skolem functions for each suitable formula.
    If \( \chi(\vb x, y) \) is an \( \mathcal L \)-formula with \( \vb x \) nonempty, we add a function symbol \( F_\chi \) of arity \( \abs{\vb x} \).
    Performing this for all \( \mathcal L \)-formulae of this form, we obtain a language \( \mathcal L' \supseteq \mathcal L \).
    Next, define \( \Sigma(\mathcal L) \) to be the set of \( \mathcal L \)-sentences that enforce the correct behaviour of the \( F_\chi \):
    \[ \forall \vb x.\, (\exists y.\, \chi(\vb x, y) \Rightarrow \chi(\vb x, F_\chi(\vb x))) \]
    Note that \( \Sigma(\mathcal L) \) is an \( \mathcal L' \)-theory, not an \( \mathcal L \)-theory; there may be existentials in \( \mathcal L' \) without explicit witnesses.
    We can overcome this issue by iterating this construction \( \omega \) times and taking the union.
    Formally, we recursively define
    \[ \mathcal L_0 = \mathcal L;\quad \mathcal L_{n+1} = \mathcal L_n';\quad \Sigma_0 = \varnothing; \quad \Sigma_{n+1} = \Sigma_n \cup \Sigma(\mathcal L_n) \]
    Then we can set
    \[ \mathcal L^+ = \bigcup_{n < \omega} \mathcal L_n;\quad \Sigma = \bigcup_{n < \omega} \Sigma_n \]
    First, note that \( \Sigma \) is a Skolem theory.
    This is because each \( \mathcal L^+ \)-formula is in \( \mathcal L_n \) for some \( n < \omega \), so \( \Sigma_{n+1} \subseteq \Sigma \) asserts that it has a Skolem function.
    It is also clear to see that \( \abs{\mathcal L^+} = \abs{\mathcal L} \) using basic cardinal arithmetic.

    To prove property (ii), it suffices to show that each \( \mathcal L \)-theory can be expanded into an \( \mathcal L' \)-theory that models \( \Sigma(\mathcal L) \); we can then proceed by induction.
    Note that this argument will use the axiom of choice.
    Let \( \mathcal M \) be an \( \mathcal L \)-structure.
    We can assume \( \mathcal M \neq \varnothing \); if \( \mathcal M = \varnothing \) then all sentences in \( \Sigma \) would be vacuously true and there would be nothing to prove.
    We now expand \( \mathcal M \) into an \( \mathcal L' \)-structure \( \mathcal M \) in the following way.
    Consider \( \chi(\vb x, y) \) where \( \vb x \) is nonempty and \( \vb m \in \mathcal M \).
    If
    \[ \mathcal M \vDash \exists b.\, \chi(\vb m, b) \]
    then we can choose such a \( b \) and interpret \( F_\chi(\vb m) \) as this value.
    If
    \[ \mathcal M \nvDash \exists b.\, \chi(\vb m, b) \]
    then we interpret \( F_\chi(\vb m) \) as an arbitrary model element, say, \( \vb m_0 \).
    By construction, \( \mathcal M' \) models \( \Sigma(\mathcal L) \).
\end{proof}
\begin{corollary}
    Any \( \mathcal L \)-theory \( \mathcal T \) admits a skolemisation \( \mathcal T^+ \) in a language \( \mathcal L^+ \) of the same size as \( \mathcal L \).
\end{corollary}
\begin{proof}
    Take \( \mathcal T^+ = \mathcal T \cup \Sigma \).
    Any model of \( \mathcal T^+ \) models \( \Sigma \), so \( \mathcal T^+ \) has Skolem functions.
    Moreover, any \( \mathcal L \)-structure that models \( \mathcal T \) can be extended to one that models \( \Sigma \), which will therefore model \( \mathcal T^+ \).
\end{proof}
\begin{corollary}[downward L\"owenheim--Skolem theorem]
    Let \( \mathcal M \) be an \( \mathcal L \)-structure, and let \( X \subseteq \mathcal M \).
    Let \( \kappa \) be a cardinal such that
    \[ \abs{\mathcal L} + \abs{X} \leq \kappa \leq \abs{\mathcal M} \]
    Then \( \mathcal M \) has an elementary substructure of size \( \kappa \) that contains \( X \).
\end{corollary}
