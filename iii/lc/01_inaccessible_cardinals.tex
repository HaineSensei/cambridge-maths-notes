\subsection{Large cardinal properties}
Modern set theory largely concerns itself with the consequences of the incompleteness phenomenon.
Given any `reasonable' set theory \( T \), G\"odel's first incompleteness theorem shows that there is a sentence \( \varphi \) such that \( T \nvdash \varphi \) and \( T \nvdash \neg\varphi \).
To be `reasonable', the set of axioms must be computably enumerable, among other similar restrictions.
In particular, G\"odel's second incompleteness theorem shows that \( T \nvdash \Con(T) \), where \( \Con(T) \) is the statement that \( T \) is consistent.
Hence,
\[ \qty{\psi \mid T \vdash \psi} \subsetneq \qty{\psi \mid T + \varphi \vdash \psi} \]
We might say
\[ T <_{\text{consequence}} T + \varphi \]
so \( T \) has strictly fewer consequences than \( T + \varphi \).
Modern set theory is about understanding the relation \( \leq_{\text{consequence}} \) and other similar relations.
It turns out that large cardinal axioms are the most natural hierarchy that we can use to measure the strength of set theories.

In this course we will not provide a definition for the notion of `large cardinal', but we will provide an informal description.
A \emph{large cardinal property} is a formula \( \Phi \) such that \( \Phi(\kappa) \) implies that \( \kappa \) is a very large cardinal, so large that its existence cannot be proven in \( \mathsf{ZFC} \).
A \emph{large cardinal axiom} is an axiom of the form \( \exists \kappa.\, \Phi(\kappa) \), which we will abbreviate \( \Phi \mathsf{C} \).
We begin with some non-examples.

\begin{enumerate}
    \item \( \kappa \) is called an \emph{\( \aleph \) fixed point} if \( \kappa = \aleph_\kappa \).
    Note that, for example, \( \omega \), \( \omega_1 \), and \( \aleph_\omega \) are not \( \aleph \) fixed points.
    However, we have the following result.
    We say that \( F : \mathrm{Ord} \to \mathrm{Ord} \) is \emph{normal} if \( \alpha < \beta \) implies \( F(\alpha) < F(\beta) \), and if \( \lambda \) is a limit, \( F(\lambda) = \bigcup_{\alpha < \lambda} F(\alpha) \).
    One can show that every normal ordinal operation has arbitrarily large fixed points, and in particular that these fixed points may be enumerated by the ordinals.
    In particular, since the operation \( \alpha \mapsto \aleph_\alpha \) is normal, it admits fixed points.
    \item Let \( \Phi(\kappa) \) be the property
    \[ \kappa = \aleph_\kappa \wedge \Con(\mathsf{ZFC}) \]
    Clearly \( \Phi \mathsf{C} \) implies \( \Con(\mathsf{ZFC}) \), so \( \mathsf{ZFC} \nvdash \Phi \mathsf{C} \).
    We would like our large cardinal axioms to be unprovable by \( \mathsf{ZFC} \) because of the size of the cardinal in question, not because of any other arbitrary reasons that we may attach to these axioms.
\end{enumerate}

One source of large cardinal axioms is as follows.
Consider the ordinal \( \omega \); it is much larger than any ordinal smaller than it.
We can consider properties that encapsulate the notion that \( \omega \) is much larger than any smaller ordinal, and use these as large cardinal properties.

\begin{enumerate}
    \item If \( n < \omega \), then \( n^+ < \omega \), where \( n^+ \) is the cardinal successor of \( n \).
    We define
    \[ \Lambda(\kappa) \iff \forall \alpha.\, (\alpha < \kappa \to \alpha^+ < \kappa) \]
    where \( \alpha^+ \) is the least cardinal strictly larger than \( \alpha \).
    Then, \( \Lambda(\kappa) \) holds precisely when \( \kappa \) is a limit cardinal.
    These need not be very large, for example, \( \aleph_\omega \) is a limit cardinal, and the existence of this cardinal is proven by \( \mathsf{ZFC} \).
    \item If \( n < \omega \), then \( 2^n < \omega \), where \( 2^n \) is the size of the power set of \( n \).
    \[ \Sigma(\kappa) \iff \forall \alpha.\, (\alpha < \kappa \to 2^\alpha < \kappa) \]
    where \( 2^\alpha \) is the cardinality of \( \mathcal P(\alpha) \).
    Such cardinals are called \emph{strong limit cardinals}.
    We will show that these exist in all models of \( \mathsf{ZFC} \).
    Similarly to the aleph hierarchy, we can define the \emph{beth} hierarchy, denoted \( \beth_\alpha \).
    This is given by
    \[ \beth_0 = \aleph_0;\quad \beth_{\alpha + 1} = 2^{\beth_\alpha};\quad \beth_{\lambda} = \bigcup_{\alpha < \lambda} \beth_\alpha \]
    Cantor's theorem shows that \( \aleph_\alpha \leq \beth_\alpha \), and the continuum hypothesis is the assertion that \( \aleph_1 = \beth_1 \).
    Note that \( \kappa \) is a strong limit cardinal if and only if \( \kappa = \beth_\lambda \) for some limit ordinal \( \lambda \).
    In particular, \( \mathsf{ZFC} \vdash \Sigma \mathsf{C} \).
    \item If \( s : n \to \omega \) for \( n < \omega \), then \( \sup(s) = \bigcup \operatorname{ran}(s) < \omega \).
    This gives rise to the following definition.
    \begin{definition}
        Let \( \lambda \) be a limit ordinal.
        We say that \( C \subseteq \lambda \) is \emph{cofinal} or \emph{unbounded} if \( \bigcup C = \lambda \).
        We define the \emph{cofinality} of \( \lambda \), denoted \( \cf(\lambda) \), to be the cardinality of the smallest cofinal subset.
        If \( \lambda \) is a cardinal, then \( \cf(\lambda) \leq \lambda \).
        If this inequality is strict, the cardinal is called \emph{singular}; if this is an equality, it is called \emph{regular}.
    \end{definition}
    Note that if \( \kappa \) is regular, then if \( \lambda < \kappa \), and for each \( \alpha < \lambda \) we have a set \( X_\alpha \subseteq \kappa \) of size \( \abs{X_\alpha} < \kappa \), then \( \bigcup X_\alpha \neq \kappa \).
    It is easy to show that this property is equivalent to regularity.

    Then \( \omega \) is a regular cardinal.
    Note that \( \aleph_1 \) is also regular, since countable unions of countable sets are countable.
    This argument generalises to all succcessor cardinals, so all successor cardinals \( \aleph_{\alpha + 1} \) are regular.
    The cardinal \( \aleph_\omega \) is not regular, as it is the union of \( \qty{\aleph_n \mid n \in \mathbb N} \), which is a subset of \( \aleph_\omega \) of cardinality \( \aleph_0 \), giving \( \cf(\aleph_\omega) = \aleph_0 \).
    The cofinality of \( \aleph_{\aleph_\omega} \) is also \( \aleph_0 \).
    Limit cardinals are often singular.
\end{enumerate}

\subsection{Weakly inaccessible and inaccessible cardinals}
Motivated by these examples of properties of \( \omega \), we make the following definition.

\begin{definition}
    A cardinal \( \kappa \) is called \emph{weakly inacessible} if it is an uncountable regular limit, and \emph{(strongly) inaccessible} if it is an uncountable regular strong limit.
    We write \( \mathsf{WI}(\kappa) \) to denote that \( \kappa \) is weakly inaccessible, and \( \mathsf{I}(\kappa) \) if \( \kappa \) is inaccessible.
\end{definition}

To argue that these are large cardinal properties, we will show that they are very large, and that the existence of such cardinals cannot be proven in \( \mathsf{ZFC} \).
Note that we cannot actually prove this statement; if \( \mathsf{ZFC} \) were inconsistent, it would prove every statement.
Whenever we write statements such as \( \mathsf{ZFC} \nvdash \mathsf{IC} \), it should be interpreted to mean `if \( \mathsf{ZFC} \) is consistent, it does not prove \( \mathsf{IC} \)'.

Many things in the relationship of \( \mathsf{WI} \) and \( \mathsf{I} \) are unclear: \( 2^{\aleph_0} \) is clearly not inaccessible as it is not a strong limit, but it is not clear that this is not a limit.
The \emph{generalised continuum hypothesis} \( \mathsf{GCH} \) is that for all cardinals \( \alpha \), we have \( 2^{\aleph_\alpha} = \aleph_{\alpha + 1} \), and so \( \aleph_\alpha = \beth_\alpha \).
Thus, the notions of limit and strong limit coincide, and so the notions of inaccessible cardinals and weakly inaccessible cardinals coincide.

\begin{proposition}
    Weakly inaccessible cardinals are aleph fixed points.
\end{proposition}
\begin{proof}
    Suppose \( \kappa \) is weakly inaccessible but \( \kappa < \aleph_\kappa \).
    Fix \( \alpha \) such that \( \kappa = \aleph_\alpha \), then \( \alpha < \kappa \).
    As \( \kappa \) is a limit cardinal, \( \alpha \) must be a limit ordinal.
    But then \( \aleph_\alpha = \bigcup_{\beta < \alpha} \aleph_\beta \), so in particular, the set \( \qty{\aleph_\beta \mid \beta < \alpha} \) is cofinal in \( \kappa \), but this set is of size \( \abs{\alpha} < \kappa \).
    Hence \( \kappa \) is singular, contradicting regularity.
\end{proof}

\subsection{Second order replacement}
We will now show that \( \mathsf{ZFC} \) does not prove \( \mathsf{IC} \), and we omit the result for weakly inaccessible cardinals.
We could do this via model-theoretic means: we assume \( M \vDash \mathsf{ZFC} \), and construct a model \( N \vDash \mathsf{ZFC} + \neg \mathsf{IC} \).
However, there is another approach we will take here.
By G\"odel's second incompleteness theorem, under the assumption that \( \mathsf{ZFC} \) is consistent, we have \( \mathsf{ZFC} \nvdash \Con(\mathsf{ZFC}) \), so it suffices to show \( \mathsf{IC} \to \Con(\mathsf{ZFC}) \).
G\"odel's completeness theorem states that \( \Con(T) \) holds if and only if there exists a model \( M \) with \( M \vDash T \).
Thus, it suffices to show that under the assumption that there is an inaccessible cardinal, we can construct a model of \( \mathsf{ZFC} \).
Note that the metatheory in which the completeness theorem is proven actually matters; both theories and models are actually sets in the outer theory.

Recall that the \emph{cumulative hierarchy} inside a model of set theory is given by
\[ \mathrm{V}_0 = \varnothing;\quad \mathrm{V}_{\alpha + 1} = \mathcal P(\mathrm{V}_\alpha);\quad \mathrm{V}_\lambda = \bigcup_{\alpha < \lambda} \mathrm{V}_\alpha \]
\begin{enumerate}
    \item The axiom of foundation is equivalent to the statement that every set is an element of \( \mathrm{V}_\alpha \) for some \( \alpha \).
    \item \( (\mathrm{V}_\omega, \in) \) is a model of all of the axioms of set theory except for the axiom of infinity.
    This collection of axioms is called \emph{finite set theory} \( \mathsf{FST} \).
    \item \( (\mathrm{V}_{\omega + \omega}, \in) \) is a model of all of the axioms of set theory except for the axiom of replacement.
    This theory is called \emph{Zermelo set theory with choice} \( \mathsf{ZC} \).
    In fact, for any limit ordinal \( \lambda > \omega \), \( \mathsf{ZFC} \) proves that \( (\mathrm{V}_\lambda, \in) \vDash \mathsf{ZC} \).
    That is, \( \mathsf{ZFC} \) proves the existence of a model of \( \mathsf{ZC} \), or equivalently, \( \mathsf{ZFC} \vdash \Con(\mathsf{ZC}) \).
    Hence, \( \mathsf{ZC} \) cannot prove replacement, since G\"odel's second incompleteness theorem applies to \( \mathsf{ZC} \).
    In this way, replacement behaves like a large cardinal axiom for \( \mathsf{ZC} \).
    The same holds for infinity and \( \mathsf{FST} \).
\end{enumerate}
We briefly discuss why replacement fails in \( \mathrm{V}_{\omega + \omega} \).
Consider the set of ordinals \( \omega + n \) for \( n < \omega \); this set does not belong to \( \mathrm{V}_{\omega + \omega} \) as its rank is \( \omega + \omega \).
However, the class function \( F \) given by \( n \mapsto \omega + n \) is definable by a simple formula, and applying this to the set \( \omega \in \mathrm{V}_{\omega + \omega} \) gives a counterexample to replacement.
Our counterexample is thus a cofinal subset of \( \mathrm{V}_{\omega + \omega} \) whose union does not lie in \( \mathrm{V}_{\omega + \omega} \).
In some sense, the fact that \( \omega + \omega \) is singular is the reason why \( \mathrm{V}_{\omega + \omega} \) does not satisfy replacement.

Now, consider \( \alpha = \aleph_1 \), which is regular.
Consider \( \mathcal P(\omega) \in \mathrm{V}_{\omega + 2} \subseteq \mathrm{V}_{\omega_1} \).
There is a definable surjection from \( \mathcal P(\omega) \) to \( \omega_1 \), motivated by the proof of Hartogs' lemma.
Indeed, subsets of \( \omega \) can encode well-orders, and every countable well-order is encoded by a subset of \( \omega \), so the map
\[ g : A \mapsto \begin{cases}
    \alpha & \text{if } A \text{ codes a well-order of order type } \alpha \\
    0 & \text{otherwise}
\end{cases} \]
is a surjection \( \mathcal P(\omega) \to \omega_1 \).
This class function has cofinal range in \( \omega_1 \), and so \( \mathrm{V}_{\omega_1} \) does not satisfy replacement.

We will prove that \( \mathsf{I}(\kappa) \) implies that \( \mathrm{V}_\kappa \) models replacement.
A set \( M \) is said to satisfy \emph{second-order replacement} \( \mathsf{SOR} \) if for every \( F : M \to M \) and every \( x \in M \), the set \( \qty{F(y) \mid y \in x} \) is in \( M \).
Any model of \( \mathrm{V}_\alpha \) that satisfies second-order replacement is a model of \( \mathsf{ZFC} \), as the counterexamples to replacement are special cases of violations of second-order replacement.

\begin{theorem}[Zermelo]
    If \( \kappa \) is inaccessible, then \( \mathrm{V}_\kappa \) satisfies second-order replacement.
\end{theorem}
We first prove the following lemmas.
\begin{lemma}
    If \( \kappa \) is inaccessible and \( \lambda < \kappa \), then \( \abs{\mathrm{V}_\kappa} < \kappa \).
\end{lemma}
\begin{proof}
    This follows by induction.
    Note \( \abs{\mathrm{V}_0} = 0 < \kappa \).
    If \( \abs{\mathrm{V}_\alpha} < \kappa \), then as \( \kappa \) is a strong limit, \( \abs{\mathrm{V}_{\alpha + 1}} = \abs{\mathcal P(\mathrm{V}_\alpha)} = 2^{\abs{\mathrm{V}_\alpha}} < \kappa \).
    If \( \lambda \) is a limit and \( \abs{\mathrm{V}_\alpha} < \kappa \) for all \( \alpha < \lambda \), then if \( \abs{\mathrm{V}_\lambda} = \kappa \), we have written \( \kappa \) as a union of less than \( \kappa \) sets of size less than \( \kappa \), contradicting regularity.
\end{proof}
\begin{lemma}
    If \( \kappa \) is inaccessible and \( x \in \mathrm{V}_\kappa \), then \( \abs{x} < \kappa \).
\end{lemma}
\begin{proof}
    Suppose \( x \in \mathrm{V}_\kappa = \bigcup_{\alpha < \kappa} \mathrm{V}_\alpha \).
    Then there exists \( \alpha < \kappa \) such that \( x \in \mathrm{V}_\alpha \).
    Then \( x \subseteq \mathrm{V}_\alpha \) as the \( \mathrm{V}_\alpha \) are transitive, but then \( \abs{x} \leq \abs{\mathrm{V}_\alpha} < \kappa \).
\end{proof}
We can now prove Zermelo's theorem.
\begin{proof}
    Let \( F : \mathrm{V}_\kappa \to \mathrm{V}_\kappa \), and \( x \in \mathrm{V}_\kappa \); we must show that \( R = \qty{F(y) \mid y \in x} \in \mathrm{V}_\kappa \).
    By the second lemma above, \( \abs{x} < \kappa \), hence \( \abs{R} < \kappa \).
    For each \( y \in x \), define \( \alpha_y \) to be the rank of \( F(y) \).
    This is an ordinal less than \( \kappa \).
    Consider \( A = \qty{\alpha_y \mid y \in x} \); its cardinality is bounded by that of \( x \), so \( \abs{A} < \kappa \).
    But as \( \kappa \) is regular, \( \abs{A} \) is not cofinal, so there is \( \gamma < \kappa \) such that \( A \subseteq \mathrm{V}_\gamma \).
    By definition, \( R \subseteq \mathrm{V}_\gamma \), so \( R \in \mathrm{V}_{\gamma + 1} \subseteq \mathrm{V}_\kappa \), as required.
\end{proof}
The definition of inacessibility is precisely what is needed for this proof to work.
The following converse holds.
\begin{theorem}[Shepherdson]
    If \( \mathrm{V}_\kappa \) satisfies second-order replacement, then \( \kappa \) is inaccessible.
\end{theorem}
\begin{proof}
    Suppose \( \kappa \) is not inaccessible, so either \( \kappa \) is singular or there is \( \lambda < \kappa \) such that \( 2^\lambda \geq \kappa \).
    If \( \kappa \) is singular, then \( \kappa = \bigcup_{\alpha < \lambda} \kappa_\alpha \) for \( \lambda < \kappa \) and \( \kappa_\alpha < \kappa \).
    Consider \( C = \qty{\kappa_\alpha \mid \alpha < \lambda} \); this set is cofinal in \( \kappa \), but the cardinality of \( C \) is \( \lambda \).
    Therefore, \( C \notin \mathrm{V}_\kappa \).
    We simply take the function \( F : \alpha \mapsto \kappa_\alpha \), then the image of \( \lambda \) under \( F \) is \( C \notin \mathrm{V}_\kappa \), so \( F \) witnesses that \( \mathrm{V}_\kappa \) violates second-order replacement.

    Suppose there is \( \lambda < \kappa \) such that \( 2^\lambda \geq \kappa \).
    Let \( F : \mathcal P(\lambda) \to \kappa \) be a surjection.
    Since \( \lambda < \kappa \), we must have \( \mathcal P(\lambda) \in \mathrm{V}_{\lambda + 2} \subseteq \mathrm{V}_\kappa \).
    Then the image of \( \mathcal P(\lambda) \) under \( F \) is \( \kappa \notin \mathrm{V}_\kappa \) as required.
\end{proof}

\subsection{Countable transitive models of set theory}
It is not generally the case that if \( \mathrm{V}_\kappa \vDash \mathsf{ZFC} \) then \( \kappa \) is inaccessible.
Moreover, the existence of an inaccessible cardinal is strictly stronger than the consistency of \( \mathsf{ZFC} \).
We will show this second statement first.

Suppose \( \kappa \) is inaccessible, so \( \mathrm{V}_\kappa \vDash \mathsf{ZFC} \).
A standard model-theoretic argument shows there is a countable elementary substructure \( (N, \in) \preceq (\mathrm{V}_\kappa, \in) \).
In particular, \( (N, \in) \vDash \mathsf{ZFC} \).
The proof of the downwards L\"owenheim--Skolem theorem is a Skolem hull construction, given by
\[ N_0 = \varnothing;\quad N_{k+1} = N_k \cup W(N_k);\quad N = \bigcup_{k \in \mathbb N} N_k \]
where \( W(N_k) \) is a set of witnesses for all formulas of the form \( \exists x.\, \varphi \) with parameters in \( N_k \).
The fact that this is an elementary substructure follows from the Tarski--Vaught test.
We will now explore this model in more detail.

If \( n \in \omega \), there is a formula \( \varphi_n \) such that \( \mathrm{V}_\kappa \vDash \varphi_n(x) \) if and only if \( x = n \).
Clearly, the formula \( \exists x.\, \varphi_n(x) \) has precisely one witness, so \( \omega \subseteq N_1 \).
Similarly, there are formulas \( \varphi_\omega, \varphi_{\omega + \omega}, \varphi_{\omega \cdot 3} \) and so on.
There is also a formula \( \varphi_{\omega_1} \) such that \( x = \omega_1 \) if and only if \( \mathrm{V}_\kappa \vDash \varphi_{\omega_1}(x) \).
As before, because there is a unique witness to this formula in \( \mathrm{V}_\kappa \), we must have \( \omega_1 \in N_1 \).
But since the model \( N \) is countable, there must be a gap in the ordinals at some point below \( \omega_1 \).
By the same argument, the model contains \( \omega_2, \omega_3 \) and so on.
Therefore, \( N \) is a nontransitive model.

As \( (N, \in) \) is well-founded and extensional, by Mostowski's collapsing theorem there is a unique transitive \( M \) such that \( (M, \in) \cong (N, \in) \).
This fills all of the gaps in our model.
As this is an isomorphism, we obtain \( (M, \in) \preceq (N, \in) \preceq (\mathrm{V}_\kappa, \in) \), so \( (M, \in) \) is a countable transitive model of \( \mathsf{ZFC} \).
In particular, its height \( \alpha = \mathrm{Ord} \cap M \) is a countable ordinal.
There is an elementary embedding of \( M \) into \( \mathrm{V}_\kappa \) given by the inverse of the Mostowski collapse.
In particular, some \( \beta < \alpha \) has the property that \( M \vDash \varphi_{\omega_1}(\beta) \).

Therefore, the property `\( x \) is a cardinal' cannot be an \emph{absolute} property between \( M \) and \( \mathrm{V}_\kappa \).
A property is said to be absolute between \( M \) and some larger structure \( N \) if it holds in \( M \) precisely if it holds in \( N \), where parameters are allowed to take values in the smaller structure \( M \).
If the truth of the property in the smaller structure implies the truth in the larger structure, we say the property is \emph{upwards absolute}; conversely, if truth in the larger structure implies truth in the smaller one, we say the property is \emph{downwards absolute}.
The theory of absoluteness concerns the following classes of formulas, among others.
\begin{enumerate}
    \item \( \Delta_0 \) formulas, in which only bounded quantifiers are permitted, for example in \( \mathsf{ZFC} \), `\( x \) is an ordinal', `\( f \) is a function', `\( x \) is a subset of \( y \)', `\( x \) is \( \omega \)'.
    \item \( \Sigma_1 \) formulas, which are \( \Delta_0 \) formulas surrounded by a single existential quantifier.
    \item \( \Pi_1 \) formulas, which are \( \Delta_0 \) formulas surrounded by a single universal quantifier, for example `\( x \) is a cardinal' or `\( x \) is the power set of \( y \)'.
\end{enumerate}
One can show that \( \Delta_0 \) formulas are absolute between transitive models.
Further, \( \Sigma_1 \) formulas are upwards absolute and \( \Pi_1 \) formulas are downwards absolute.
The example above shows that `\( x \) is a cardinal' cannot be \( \Delta_0 \) as it is not upwards absolute.
Similarly, `\( x \) is the power set of \( y \)' cannot be \( \Delta_0 \), because the object \( p \) that \( M \) believes is the power set of \( \omega \) must be countable, and so cannot be the real power set in \( \mathrm{V}_\kappa \).
As being a subset is absolute, this object \( p \) must consist of subsets of \( \omega \), but must only contain very few of them.

As being \( \omega \) is \( \Delta_0 \), in fact all arithmetical statements (and therefore, by encoding, all syntactic statements) are \( \Delta_0 \).
\begin{theorem}
    \( \mathsf{IC} \to \Con(\mathsf{ZFC}) \) but \( \Con(\mathsf{ZFC}) \nrightarrow \mathsf{IC} \).
\end{theorem}
\begin{proof}
    The forward direction has already been proven.
    Since \( \mathsf{IC} \) proves the consistency of \( \mathsf{ZFC} \), there is a countable transitive model \( M \subseteq \mathrm{V}_\kappa \subseteq \mathrm{V} \) of \( \mathsf{ZFC} \).
    By absoluteness, \( M \vDash \Con(\mathsf{ZFC}) \), so \( M \vDash \mathsf{ZFC}^\star \) where we define \( \mathsf{ZFC}^\star = \mathsf{ZFC} + \Con(\mathsf{ZFC}) \).
    We have thus proven that \( \mathsf{IC} \) implies the consistency of \( \mathsf{ZFC}^\star \).
    So, by the second incompleteness theorem, \( \mathsf{ZFC}^\star \nvdash \mathsf{IC} \).
\end{proof}

\subsection{Worldly cardinals}
We now show that if \( \mathrm{V}_\kappa \vDash \mathsf{ZFC} \), it is not necessarily the case that \( \kappa \) is inaccessible.

Observe that \( M \neq \mathrm{V}_\alpha \) for any \( \alpha \).
Clearly \( M \neq \mathrm{V}_\omega \).
But \( \abs{\mathrm{V}_{\omega + 1}} = \abs{\mathcal P(\omega)} = 2^{\aleph_0} \), and \( \abs{\mathrm{V}_\alpha} > 2^{\aleph_0} \) for all \( \alpha \geq \omega + 1 \).
But \( M \) is countable, so it cannot be any of these.

Recall the definition of \( N \) by
\[ N_0 = \varnothing;\quad N_{k+1} = W(N_k);\quad N = \bigcup_{k \in \mathbb N} N_k \]
We wish to create a similar structure that is of the form \( \mathrm{V}_\alpha \) for some \( \alpha \).
We define
\[ \alpha_0 = 0;\quad \alpha_{k+1} = \sup\qty{\rank(x) \mid x \in W(\mathrm{V}_{\alpha_k})};\quad \alpha = \sup\qty{\alpha_n \mid n \in \mathbb N} \]
Note that \( N \subseteq \mathrm{V}_{\alpha_1} \).
\begin{theorem}
    \( \mathrm{V}_\alpha \preceq \mathrm{V}_\kappa \) and \( \alpha < \kappa \).
\end{theorem}
\begin{proof}
    The first statement follows from the Tarski--Vaught test.
    To show \( \alpha < \kappa \), we first show by induction that \( \alpha_k < \kappa \).
    This is clearly true for \( k = 0 \).
    Now, if \( \alpha_k < \kappa \), we have \( \abs{\mathrm{V}_{\alpha_k}} < \kappa \) by a previous lemma.
    Thus,
    \[ \abs{W(\mathrm{V}_{\alpha_k})} \leq \aleph_0 \cdot \abs{\mathrm{V}_{\alpha_k}^{<\omega}} = \abs{\mathrm{V}_{\alpha_k}} < \kappa \]
    where \( X^{<\omega} \) is the set of finite sequences of elements of \( X \).
    Hence \( \qty{\rank(x) \mid x \in W(\mathrm{V}_{\alpha_k})} \) is a set of less than \( \kappa \) ordinals less than \( \kappa \), so it must be bounded by regularity.
    Finally, as \( \alpha \) is a countable union of the \( \alpha_k \), regularity again shows \( \alpha < \kappa \).
\end{proof}
\begin{remark}
    The ordinal \( \alpha \) produced in this way has countable cofinality, so cannot be inaccessible.
    In particular, \( \mathrm{V}_\alpha \vDash \mathsf{ZFC} \) but \( \alpha \) is not inaccessible.
\end{remark}
\begin{definition}
    We call an ordinal \( \alpha \) \emph{worldly} if \( \mathrm{V}_\alpha \vDash \mathsf{ZFC} \), and write \( \mathsf{Wor}(\alpha) \).
\end{definition}
We have shown \( \mathsf{I}(\kappa) \to \mathsf{Wor}(\kappa) \), but not the other way round given that a wordly cardinal exists.
In particular,
\[ \mathsf{IC} \to \mathsf{WorC} \to \Con(\mathsf{ZFC}) \]
\begin{theorem}
    If \( \kappa \) is a wordly ordinal, \( \kappa \) is a cardinal.
\end{theorem}
\begin{proof}
    First, observe that \( \kappa \) is a limit ordinal; otherwise, its predecessor would be the largest ordinal in the model, but \( \mathsf{ZFC} \) proves that there is no largest ordinal.
    Suppose \( \kappa \) is not a cardinal, so there is \( \lambda < \kappa \) such that there is a bijection \( \lambda \to \kappa \).
    In particular, \( \lambda < \kappa < \lambda^+ \).
    By the proof of Hartogs' lemma, there is a relation \( R \subseteq \lambda \times \lambda \) such that \( (\lambda, R) \cong (\kappa, \in) \).
    Assuming Kuratowski's definition of ordered pairs, an element of \( \lambda \times \lambda \) is an element of \( \mathrm{V}_{\lambda} \), so \( \lambda \times \lambda \in \mathrm{V}_{\lambda + 1} \) and \( R \in \mathrm{V}_{\lambda + 1} \).
    The pair \( (\lambda, R) \) is an element of \( \mathrm{V}_{\lambda + 3} \subseteq \mathrm{V}_{\kappa} \).
    Thus \( \mathrm{V}_\kappa \) contains a well-order \( (\lambda, R) \) of order type \( \kappa \).
    But \( \mathsf{ZFC} \) proves that every well-ordering is isomorphic to a unique ordinal, so we must have \( \kappa \in \mathrm{V}_\kappa \), which is a contradiction.
\end{proof}

\subsection{The consistency strength hierarchy}
Let \( B \) be a base theory; we will often use \( \mathsf{ZFC} \).
If \( T, S \) are extensions of \( B \), we say that \( T \) has lower \emph{consistency strength} than \( S \), written \( T \leq_{\Con} S \), if \( B \vdash \Con(S) \to \Con(T) \).
We say that \( T \) and \( S \) is \emph{equiconsistent}, written \( T \equiv_{\Con} S \), if \( T \leq_{\Con} S \) and \( S \leq_{\Con} T \), and write \( T <_{\Con} S \) if \( T \leq_{\Con} S \) but \( S \nleq_{\Con} T \).
\begin{remark}
    \begin{enumerate}
        \item If \( I \) is inconsistent, then \( T \leq_{\Con} I \) for all \( T \).
        All inconsistent theories are equiconsistent.
        In particular, \( T \) is consistent if and only if \( T <_{\Con} I \).
        We typically write \( \bot \) for an inconsistent theory.
        \item \( <_{\Con} \) is more than just `proving more theorems'.
        If \( \varphi \) is such that \( \mathsf{ZFC} \nvdash \varphi \) and \( \mathsf{ZFC} \nvdash \neg\varphi \), it is not necessarily the case that \( \mathsf{ZFC} <_{\Con} \mathsf{ZFC} + \varphi \) or \( \mathsf{ZFC} <_{\Con} \mathsf{ZFC} + \neg\varphi \).
        For example, \( \mathsf{ZFC} + \mathsf{CH} \), \( \mathsf{ZFC} + \neg\mathsf{CH} \), and \( \mathsf{ZFC} \) are all equiconsistent.
        \item The second incompleteness theorem shows, for suitably nice theories \( T \), that if \( T \neq \bot \) then \( T <_{\Con} T + \Con(T) \).
        Note that it is possible that \( T \) is consistent but \( T + \Con(T) \) is inconsistent, so the incompleteness theorem does not necessarily give an infinite chain of strict consistency strength inequalities.
        For example, consider
        \[ \mathsf{ZFC}^\dagger = \mathsf{ZFC} + \neg\Con(\mathsf{ZFC}) \]
        Since \( \mathsf{ZFC}^\dagger \supseteq \mathsf{ZFC} \), we must have \( \Con(\mathsf{ZFC}^\dagger) \to \Con(\mathsf{ZFC}) \), but \( \mathsf{ZFC}^\dagger \to \neg\Con(\mathsf{ZFC}) \), so \( \mathsf{ZFC}^\dagger + \Con(\mathsf{ZFC}^\dagger) \) is inconsistent.
    \end{enumerate}
\end{remark}
In conclusion,
\[ \mathsf{ZFC} <_{\Con} \mathsf{ZFC} + \Con(\mathsf{ZFC}) <_{\Con} \mathsf{ZFC} + \mathsf{WorC} <_{\Con} \mathsf{ZFC} + \mathsf{IC} \]
where the second inequality uses the same argument as \( \mathsf{IC} \to \Con(\mathsf{ZFC} + \Con(\mathsf{ZFC})) \).

We will see that \( \mathsf{ZFC} \equiv_{\Con} \mathsf{ZFC} + \neg\mathsf{IC} \).
Many large cardinal axioms have this property that their negations are weak.

If \( \kappa \) is the least inaccessible cardinal, then \( \mathrm{V}_\kappa \) is a model of \( \mathsf{ZFC} \), but we can show that it cannot satisfy \( \mathsf{IC} \).
Note that the statement `\( \lambda \) is inaccessible' is a \( \Pi_1 \) statement, so is downwards absolute.
Given a model with two inaccessible cardinals \( \kappa_0 < \kappa_1 \), we have \( \mathrm{V}_{\kappa_1} \vDash \mathsf{ZFC} + \mathsf{I}(\kappa_0) \) so in particular, \( \mathrm{V}_{\kappa_1} \vDash \mathsf{ZFC} + \mathsf{IC} \).
\begin{lemma}
    If \( \alpha \) is a limit ordinal, then the formula `\( \lambda \) is inaccessible' is absolute for \( \mathrm{V}_\alpha \) and \( \mathrm{V} \).
\end{lemma}
In particular, \( \mathrm{V}_\kappa \) above does not satisfy \( \mathsf{IC} \).
\begin{proof}
    By downwards absoluteness, it suffices to show that if \( \mathrm{V}_\alpha \vDash \mathsf{I}(\lambda) \) then \( \mathsf{I}(\lambda) \).
    Suppose not, so \( \lambda \) is singular or not a strong limit.

    Let \( \lambda \) be singular, so there is a cofinal set \( C \subseteq \lambda \) with \( \abs{C} = \gamma < \lambda \), so there is a bijection \( f : \gamma \to C \).
    Note that being singular is \( \Sigma_1 \), witnessed by \( C, \gamma, f \).
    We have \( C \in \mathrm{V}_{\lambda + 1} \), \( \gamma \in \mathrm{V}_\lambda \), and \( f \in \mathrm{V}_{\lambda + 2} \).
    All of these are subsets of \( \mathrm{V}_\alpha \), so these witnesses exist in \( \mathrm{V}_\alpha \).
    Hence \( \mathrm{V}_\alpha \) believes that \( C \) is a cofinal set of cardinality less than \( \lambda \), so it believes \( \lambda \) is singular, contradicting inaccessibility.

    Now let \( \lambda \) not be a strong limit.
    Let \( \gamma < \lambda \), and let \( f : \mathcal P(\gamma) \to \lambda \) be a surjection.
    Then \( \mathcal P(\gamma) \in \mathrm{V}_{\gamma + 2} \subseteq \mathrm{V}_\lambda \subseteq \mathrm{V}_\alpha \), and so this function is an element of \( \mathrm{V}_{\lambda + 2} \subseteq V_\alpha \).
    The statement that it is a surjection is absolute, so \( \mathrm{V}_\alpha \) believes \( f \) is a surjection from \( \mathcal P(\gamma) \) to \( \lambda \), contradicting its belief that \( \lambda \) is a strong limit.
\end{proof}
Therefore, we have the following.
\begin{theorem}
    Suppose \( \mathsf{ZFC} + \mathsf{IC} \), and let \( \kappa \) be the least inaccessible.
    Then \( \mathrm{V}_\kappa \vDash \mathsf{ZFC} + \neg\mathsf{IC} \).
\end{theorem}
\begin{proof}
    Suppose \( \mathrm{V}_\kappa \vDash \mathsf{ZFC} + \mathsf{IC} \).
    Then there is \( \lambda < \kappa \) such that \( \mathrm{V}_\kappa \vDash \mathsf{I}(\lambda) \), but by the previous lemma this contradicts minimality of \( \kappa \).
\end{proof}
Therefore, we have the following.
\[ \mathsf{ZFC} + \mathsf{IC} \vdash \text{there is a transitive model of } \mathsf{ZFC} + \neg\mathsf{IC} \]
For any theory \( T \), we write
\[ T^\star = T + \Con(T) \]
We make the following remarks.
\begin{enumerate}
    \item Observe that if \( S \) proves that there is a transitive model of \( T \), then \( S \vdash \Con(T^\star) \) because consistency statements are downwards absolute between transitive models.
    % Does T need to be computably enumerable?
    \item Note also that if \( S \) proves every axiom of \( T \), then \( \Con(S) \to \Con(T) \).
    \item If \( T \) is not equiconsistent with \( \bot \), then \( \Con(T) \nrightarrow \Con(T^\star) \).
\end{enumerate}
We can therefore show
\[ \Con(\mathsf{ZFC} + \neg\mathsf{IC}) \nrightarrow \Con(\mathsf{ZFC} + \mathsf{IC}) \]
assuming that \( \mathsf{ZFC} + \neg\mathsf{IC} \) is consistent.
We have that \( \mathsf{ZFC} + \mathsf{IC} \) yields a transitive model of \( \mathsf{ZFC} + \neg\mathsf{IC} \).
Thus, by (i), \( \mathsf{ZFC} + \mathsf{IC} \) implies \( \Con((\mathsf{ZFC} + \neg\mathsf{IC})^\star) \).
% Why not just Con(Con(that thing))?
Hence \( \Con(\mathsf{ZFC} + \neg\mathsf{IC}) \to \Con((\mathsf{ZFC} + \neg\mathsf{IC})^\star) \), so if the given implication were to hold, it would contradict G\"odel's second incompleteness theorem.
Thus, if \( \mathsf{ZFC} + \neg\mathsf{IC} \) is consistent,
\[ \mathsf{ZFC} + \neg\mathsf{IC} <_{\Con} \mathsf{ZFC} + \mathsf{IC} \]
Observe that none of the proofs given in this section work for weakly inaccessible cardinals, so it is not clear that weakly inaccessible cardinals qualify as large cardinals.
However, that under the generalised continuum hypothesis, we have \( \aleph_\alpha = \beth_\alpha \) and so the notions of weakly inaccessible cardinal and inaccessible cardinal coincide.
In Part III Forcing and the Continuum Hypothesis, we see that if \( M \vDash \mathsf{ZFC} \), there is \( L \subseteq M \) such that \( L \) is transitive in \( M \), \( L \) contains all the ordinals of \( M \), and \( L \vDash \mathsf{ZFC} + \mathsf{GCH} \).
Thus, given a model \( M \vDash \mathsf{ZFC} + \mathsf{WIC} \), we obtain \( L \vDash \mathsf{ZFC} + \mathsf{IC} \), and thus the two axioms \( \mathsf{WIC} \) and \( \mathsf{IC} \) are equiconsistent.

Note that \( 2^{\aleph_0} \) is not a strong limit, but it is consistent that \( 2^{\aleph_0} \) is weakly inaccessible (under suitable assumptions), so the notions of weakly inaccessible cardinals and inaccessible cardinals do not coincide.
