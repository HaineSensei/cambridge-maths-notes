\subsection{The measure problem}
Let \( \mathbb I \) denote the unit interval \( [0,1] \subseteq \mathbb R \).
A function \( \mu : \mathcal P(\mathbb I) \to \mathbb I \) is called a \emph{measure} if
\begin{enumerate}
    \item \( \mu(\mathbb I) = 1 \) and \( \mu(\varnothing) = 0 \);
    \item (translation invariance) if \( X \subseteq \mathbb I \), \( r \in \mathbb R \), and \( X + r = \qty{x + r \mid x \in X} \subseteq \mathbb I \), then \( \mu(X) = \mu(X + r) \); and
    \item (countable additivity) if \( (A_n)_{n \in \mathbb N} \) is a family of pairwise disjoint subsets of \( \mathbb I \), then \( \mu\qty(\bigcup_{n \in \mathbb N} A_n) = \sum_{n \in \mathbb N}\mu(A_n) \).
\end{enumerate}
The \emph{Lebesgue measure problem} was the question of whether such a measure function exists.
Vitali proved that a measure cannot be defined on all of \( \mathcal P(\mathbb I) \).
This proof requires the axiom of choice nontrivially.
In 1970, Solovay proved that if \( \mathsf{ZFC} + \mathsf{IC} \) is consistent, then, there is a model of \( \mathsf{ZF} \) in which all sets are Lebesgue measurable.
In 1984, Shelah showed that the inaccessible cardinal was necessary to construct this model.

Now, replace translation invariance with the requirement that for all \( x \in \mathbb I \), we have \( \mu(\qty{x}) = 0 \), and call such measures \emph{Banach measures}.
\emph{Banach's measure problem} was the question of whether a Banach measure exists.
Note that this property is true for any Lebesgue measure.
If \( \mu(\qty{x}) > 0 \) for some \( x \), then by translation invariance, every singleton has the same measure \( \mu(\qty{x}) > 0 \).
There is some natural number \( n \) such that \( n \mu(\qty{x}) > 1 \), but this contradicts countable additivity using a set of \( n \) reals.
Observe that for any \( \varepsilon > 0 \), there can be only finitely many pairwise disjoint sets with measure at least \( \varepsilon \).

Banach and Kuratowski proved in 1929 that the continuum hypothesis implies that there are no Banach measures on \( \mathbb I \).
We can define Banach measures on any set \( S \) by also replacing property (i) with the requirement that \( \mu(S) = 1 \) and \( \mu(\varnothing) = 0 \).
Note that if \( \abs{S} = \abs{S'} \), then there is a Banach measure on \( S \) if and only if there is one on \( S' \).
Thus, having a Banach measure is a property of cardinals.

For larger cardinals, it may not be natural to just consider countable additivity.
\begin{definition}
    A Banach measure \( \mu \) is called \emph{\( \lambda \)-additive} if for all \( \gamma < \lambda \) and pairwise disjoint families \( \qty{A_\alpha \mid \alpha < \gamma} \), then
    \[ \mu\qty(\bigcup A_\alpha) = \sup\qty{\sum_{\alpha \in F} \mu(A_\alpha) \mid F \subseteq \gamma \text{ finite}} \]
\end{definition}
\begin{theorem}
    If \( \kappa \) is the smallest cardinal that has a Banach measure, then that measure is \( \kappa \)-additive.
\end{theorem}

\subsection{Real-valued measurable cardinals}
\begin{definition}
    A cardinal \( \kappa \) is \emph{real-valued measurable}, written \( \mathsf{RVM}(\kappa) \), if there is a \( \kappa \)-additive Banach measure on \( \kappa \).
\end{definition}
\begin{proposition}
    Every real-valued measurable cardinal is regular.
\end{proposition}
\begin{proof}
    Suppose that \( \kappa \) is a real-valued measurable cardinal, and that \( C \subseteq \kappa \) is cofinal with \( \abs{C} = \lambda < \kappa \).
    We can write
    \[ C = \qty{\gamma_\alpha \mid \alpha < \gamma} \]
    where \( \gamma_\alpha \) is increasing in \( \alpha \).
    Consider
    \[ C_\alpha = \qty{\xi \mid \gamma_\alpha \leq \xi < \gamma_{\alpha + 1}} \]
    Then \( \bigcup_{\alpha < \gamma} C_\alpha = \kappa \) as \( C \) is cofinal, and the \( C_\alpha \) are disjoint.
    Note that \( \abs{C_\alpha} \leq \abs{\gamma_{\alpha + 1}} < \kappa \).
    Writing \( C_\alpha = \bigcup_{x \in C_\alpha} \qty{x} \), we observe by \( \kappa \)-additivity that \( \mu(C_\alpha) = 0 \).
    But again by \( \kappa \)-additivity, \( \mu(\kappa) = 0 \), contradicting property (i).
\end{proof}
\begin{proposition}[the pigeonhole principle]
    Let \( \kappa \) be regular, \( \lambda < \kappa \), and \( f : \kappa \to \lambda \).
    Then there is some \( \alpha \in \lambda \) such that \( \abs{f^{-1}(\alpha)} = \kappa \).
\end{proposition}
\begin{proof}
    We have
    \[ \kappa = \bigcup_{\alpha \in \lambda} f^{-1}(\alpha) \]
    giving the result immediately by regularity of \( \kappa \).
\end{proof}
\begin{proposition}
    All successor cardinals are regular.
\end{proposition}
\begin{proposition}
    If \( \mu \) is a Banach measure on \( S \), and \( C \) is a family of pairwise disjoint sets of positive \( \mu \)-measure, then \( C \) is countable.
\end{proposition}
\begin{proof}
    Consider the collection
    \[ C_n = \qty{A \in C \mid \mu(A) > \frac{1}{n}} \]
    Observe that each \( C_n \) is finite, so \( C = \bigcup_{n \in \mathbb N} C_n \) must be countable.
\end{proof}
\begin{lemma}[Ulam]
    For any cardinal \( \lambda \), there is an \emph{Ulam matrix} \( A_\alpha^\xi \) indexed by \( \alpha < \lambda^+ \) and \( \xi < \lambda \) such that
    \begin{enumerate}
        \item for a given \( \xi \), the set \( \qty{A_\alpha^\xi \mid \alpha < \lambda^+} \) is a pairwise disjoint family; and
        \item for a given \( \alpha \), then
        \[ \abs{\lambda^+ \setminus \bigcup_{\xi < \lambda} A_\alpha^\xi} \leq \lambda \]
    \end{enumerate}
\end{lemma}
\begin{proof}
    For each \( \gamma < \lambda^+ \), fix a surjection \( f_\gamma : \lambda \to \gamma + 1 \).
    Define
    \[ A_\alpha^\xi = \qty{\gamma \mid f_\gamma(\xi) = \alpha} \]
    It is clear that property (i) holds.
    For property (ii), suppose
    \[ \gamma \in \lambda^+ \setminus \bigcup_{\xi < \lambda} A_\alpha^\xi \]
    Then \( \gamma < \lambda^+ \) and for all \( \xi \), we have \( f_\gamma(\xi) \neq \alpha \).
    Hence
    \[ \lambda^+ \setminus \bigcup_{\xi < \lambda} A_\alpha^\xi \subseteq \alpha \]
    so the size of this set is at most \( \lambda \)
\end{proof}
\begin{theorem}
    If \( \kappa \) is real-valued measurable, then \( \kappa \) is weakly inaccessible.
\end{theorem}
\begin{remark}
    If there is a Banach measure on \( [0,1] \), then in particular \( 2^{\aleph_0} \) is weakly inaccessible.
\end{remark}
\begin{proof}
    We have already shown regularity.
    Suppose \( \kappa \) is not a limit cardinal, so \( \kappa = \lambda^+ \).
    Let \( (A_\alpha^\xi)_{\alpha < \lambda^+; \xi < \lambda} \) be an Ulam matrix for \( \lambda \).
    By (ii),
    \[ \abs{Z_\alpha} \leq \lambda;\quad Z_\alpha = \lambda^+ \setminus \bigcup_{\xi < \lambda} A_\alpha^\xi \]
    so by \( \kappa \)-additivity, \( \mu(Z) = 0 \).
    Hence
    \[ \mu\qty(\bigcup_{\xi < \lambda} A_\alpha^\xi) = 1 \]
    This is a small union of sets of measure 1, so again by \( \kappa \)-additivity there is some \( \xi_\alpha \) such that \( \mu(A_\alpha^{\xi_\alpha}) > 0 \).
    Let \( f : \lambda^+ \to \lambda \) be the map \( \alpha \mapsto \xi_\alpha \).
    By the pigeonhole principle, there is some \( \xi \) and a set \( A \subseteq \lambda^+ \) with \( \abs{A} = \lambda^+ \) such that for all \( \alpha \in A \), we have \( \xi_\alpha = \xi \).
    By property (i), the collection \( \qty{A_\alpha^\xi \mid \alpha \in A} \) is a collection of uncountable size \( \lambda^+ \) of pairwise disjoint sets, all of which have positive measure, but we have already shown that such a collection must be countable.
\end{proof}

\subsection{Measurable cardinals}
\begin{definition}
    A Banach measure \( \mu \) is called \emph{two-valued} if \( \mu \) takes values in \( \qty{0,1} \).
\end{definition}
This removes any mention of the real numbers from the definition of a Banach measure.
\begin{remark}
    Two-valued measures correspond directly to ultrafilters.
    Recall that \( F \) is a \emph{filter} on \( S \) if
    \begin{enumerate}
        \item \( \varnothing \notin F, S \in F \);
        \item if \( A \subseteq B \) then \( A \in F \to B \in F \);
        \item if \( A, B \in F \) then \( A \cap B \in F \).
    \end{enumerate}
    We say that \( F \) is an \emph{ultrafilter} if \( A \in F \) or \( S \setminus A \in F \) for all \( A \subseteq S \).
    \( F \) is \emph{nonprincipal} if for all \( x \in S \), the singleton \( \qty{x} \) is not in \( F \).
    An ultrafilter is \( \lambda \)-complete if for all \( \gamma < \lambda \) and all families \( \qty{A_\alpha \mid \alpha < \gamma} \subseteq F \), we have \( \bigcap_{\alpha < \gamma} A_\alpha \in F \).
    In this way, the collection of sets of a two-valued Banach measure \( \mu \) that are assigned measure \( 1 \) form a nonprincipal ultrafilter.
    This filter is \( \lambda \)-complete if and only if \( \mu \) is \( \lambda \)-additive.
\end{remark}
\begin{definition}
    A cardinal \( \kappa \) is \emph{measurable} if there is a \( \kappa \)-complete nonprincipal ultrafilter on \( \kappa \).
\end{definition}
