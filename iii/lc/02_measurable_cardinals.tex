\subsection{The measure problem}
Let \( \mathbb I \) denote the unit interval \( [0,1] \subseteq \mathbb R \).
A function \( \mu : \mathcal P(\mathbb I) \to \mathbb I \) is called a \emph{measure} if
\begin{enumerate}
    \item \( \mu(\mathbb I) = 1 \) and \( \mu(\varnothing) = 0 \);
    \item (translation invariance) if \( X \subseteq \mathbb I \), \( r \in \mathbb R \), and \( X + r = \qty{x + r \mid x \in X} \subseteq \mathbb I \), then \( \mu(X) = \mu(X + r) \); and
    \item (countable additivity) if \( (A_n)_{n \in \mathbb N} \) is a family of pairwise disjoint subsets of \( \mathbb I \), then \( \mu\qty(\bigcup_{n \in \mathbb N} A_n) = \sum_{n \in \mathbb N}\mu(A_n) \).
\end{enumerate}
The \emph{measure problem} was the question as to whether such a measure function exists.
Vitali proved that a measure cannot be defined on all of \( \mathcal P(\mathbb I) \).
This proof requires the axiom of choice nontrivially.
In 1970, Solovay proved that if \( \mathsf{ZFC} + \mathsf{IC} \) is consistent, then, there is a model of \( \mathsf{ZF} \) in which all sets are Lebesgue measurable.
In 1984, Shelah showed that the inaccessible cardinal was necessary to construct this model.
