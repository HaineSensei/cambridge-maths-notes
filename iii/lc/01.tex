Modern set theory largely concerns itself with the consequences of the incompleteness phenomenon.
Given any `reasonable' set theory \( T \), then G\"odel's first incompleteness theorem shows that there is \( \varphi \) such that \( T \nvdash \varphi \) and \( T \nvdash \neg\varphi \).
To be `reasonable', the set of axioms must be computably enumerable, among other similar restrictions.
In particular, G\"odel's second incompleteness theorem shows that \( T \nvdash \Con(T) \), where \( \Con(T) \) is the statement that \( T \) is consistent.
Hence,
\[ \qty{\psi \mid T \vdash \psi} \subsetneq \qty{\psi \mid T + \varphi \vdash \psi} \]
We might say
\[ T <_{\text{consequence}} T + \varphi \]
so \( T \) has strictly fewer consequences than \( T + \varphi \).
Modern set theory is about understanding the relation \( \leq_{\text{consequence}} \) and other similar relations.
It turns out that large cardinal axioms are the most natural hierarchy that we can use to measure the strength of set theories.

In this course we will not provide a definition for the notion of `large cardinal', but we will provide an informal description.
A \emph{large cardinal property} is a formula \( \Phi \) such that \( \Phi(\kappa) \) implies that \( \kappa \) is a very large cardinal, so large that its existence cannot be proven in \( \mathsf{ZFC} \).
A \emph{large cardinal axiom} is an axiom of the form \( \exists \kappa.\, \Phi(\kappa) \), which we will abbreviate \( \Phi \mathsf{C} \).
We begin with some non-examples.

\begin{enumerate}
    \item \( \kappa \) is called an \emph{\( \aleph \) fixed point} if \( \kappa = \aleph_\kappa \).
    Note that, for example, \( \omega \), \( \omega_1 \), and \( \aleph_\omega \) are not \( \aleph \) fixed points.
    However, we have the following result.
    We say that \( F : \mathrm{Ord} \to \mathrm{Ord} \) is \emph{normal} if \( \alpha \leq \beta \) implies \( F(\alpha) \leq F(\beta) \), and if \( \lambda \) is a limit, \( F(\lambda) = \bigcup_{\alpha < \lambda} F(\alpha) \).
    One can show that every normal ordinal operation has arbitrarily large fixed points, and in particular that these fixed points may be enumerated by the ordinals.
    In particular, since the operation \( \alpha \mapsto \aleph_\alpha \) is normal, it admits fixed points.
    \item Let \( \Phi(\kappa) \) be the property
    \[ \kappa = \aleph_\kappa \wedge \Con(\mathsf{ZFC}) \]
    Clearly \( \Phi \mathsf{C} \) implies \( \Con(\mathsf{ZFC}) \), so \( \mathsf{ZFC} \nvdash \Phi \mathsf{C} \).
    We would like our large cardinal axioms to be unprovable by \( \mathsf{ZFC} \) because of their size, not because of any other arbitrary reasons that we may attach to these axioms.
\end{enumerate}

One source of large cardinal axioms is as follows.
Consider the ordinal \( \omega \); it is much larger than any ordinal smaller than it.
We can consider properties that encapsulate the notion that \( \omega \) is much larger than any smaller ordinal, and use these as large cardinal properties.

\begin{enumerate}
    \item If \( n < \omega \), then \( n^+ < \omega \), where \( n^+ \) is the cardinal successor of \( n \).
    We define
    \[ \Lambda(\kappa) \iff \forall \alpha (\alpha < \kappa \to \alpha^+ < \kappa) \]
    where \( \alpha^+ \) is the least cardinal strictly larger than \( \alpha \).
    Then, \( \Lambda(\kappa) \) holds precisely when \( \kappa \) is a limit cardinal.
    These need not be very large, for example, \( \aleph_\omega \) is a limit cardinal, and the existence of this cardinal is proven by \( \mathsf{ZFC} \).
    \item If \( n < \omega \), then \( 2^n < \omega \), where \( 2^n \) is the size of the power set of \( n \).
    \[ \Sigma(\kappa) \iff \forall \alpha (\alpha < \kappa \to 2^\alpha < \kappa) \]
    where \( 2^\alpha \) is the cardinality of \( \mathcal P(\alpha) \).
    Such cardinals are called \emph{strong limit cardinals}.
    We will show that these exist in all models of \( \mathsf{ZFC} \).
    Similarly to the aleph hierarchy, we can define the \emph{beth} hierarchy, denoted \( \beth_\alpha \).
    This is given by
    \[ \beth_0 = \aleph_0;\quad \beth_{\alpha + 1} = 2^{\beth_\alpha};\quad \beth_{\lambda} = \bigcup_{\alpha < \lambda} \beth_\alpha \]
    Cantor's theorem shows that \( \aleph_\alpha \leq \beth_\alpha \), and the continuum hypothesis is the assertion that \( \aleph_1 = \beth_1 \).
    Note that \( \kappa \) is a strong limit cardinal if and only if \( \kappa = \beth_\lambda \) for some limit ordinal \( \lambda \).
    In particular, \( \mathsf{ZFC} \vdash \Sigma \mathsf{C} \).
    \item If \( s : n \to \omega \) for \( n < \omega \), then \( \sup(s) = \bigcup \operatorname{ran}(s) < \omega \).
    This gives rise to the following definition.
    \begin{definition}
        Let \( \lambda \) be a limit ordinal.
        We say that \( C \subseteq \lambda \) is \emph{cofinal} or \emph{unbounded} if \( \bigcup C = \lambda \).
        We define the \emph{cofinality} of \( \lambda \), denoted \( \cf(\lambda) \), to be the cardinality of the smallest cofinal subset.
        If \( \lambda \) is a cardinal, then \( \cf(\lambda) \leq \lambda \).
        If this inequality is strict, the cardinal is called \emph{singular}; if this is an equality, it is called \emph{regular}.
    \end{definition}
    Then \( \omega \) is a regular cardinal.
    Note that \( \aleph_1 \) is also regular, since countable unions of countable sets are countable.
    This argument generalises to all succcessor cardinals, so all successor cardinals \( \aleph_{\alpha + 1} \) are regular.
    The cardinal \( \aleph_\omega \) is not regular, as it is the union of \( \qty{\aleph_n \mid n \in \mathbb N} \), which is a subset of \( \aleph_\omega \) of cardinality \( \aleph_0 \), giving \( \cf(\aleph_\omega) = \aleph_0 \).
    Note also that the cofinality of \( \aleph_{\aleph_\omega} \) is also \( \aleph_0 \).
    Limit cardinals are often singular.
\end{enumerate}

Motivated by these examples of properties of \( \omega \), we make the following definition.

\begin{definition}
    A cardinal \( \kappa \) is called \emph{weakly inacessible} if it is an uncountable regular limit, and \emph{(strongly) inaccessible} if it is an uncountable regular strong limit.
    We write \( \mathsf{WI}(\kappa) \) to denote that \( \kappa \) is weakly inaccessible, and \( \mathsf{I}(\kappa) \) if \( \kappa \) is inaccessible.
\end{definition}

To argue that these are large cardinal properties, we will show that they are very large, and that the existence of such cardinals cannot be proven in \( \mathsf{ZFC} \).
Note that we cannot actually prove this statement; if \( \mathsf{ZFC} \) were inconsistent, it would prove every statement.
Whenever we write statements such as \( \mathsf{ZFC} \nvdash \mathsf{IC} \), it should be interpreted to mean `if \( \mathsf{ZFC} \) is consistent, it does not prove \( \mathsf{IC} \)'.

\begin{proposition}
    Weakly inaccessible cardinals are aleph fixed points.
\end{proposition}
\begin{proof}
    Suppose \( \kappa \) is weakly inacessible but \( \kappa < \aleph_\kappa \).
    Fix \( \alpha \) such that \( \kappa = \aleph_\alpha \), then \( \alpha < \kappa \).
    As \( \kappa \) is a limit cardinal, \( \alpha \) must be a limit ordinal.
    But then \( \aleph_\alpha = \bigcup_{\beta < \alpha} \aleph_\beta \), so in particular, the set \( \qty{\aleph_\beta \mid \beta < \alpha} \) is cofinal in \( \kappa \), but this set is of size \( \abs{\alpha} < \kappa \).
    Hence \( \kappa \) is singular, contradicting regularity.
\end{proof}
