\subsection{???}
\begin{definition}
    Let \( A \) be an \( R \)-algebra, and let \( x \in A \).
    Then \( x \) is \emph{integral} over \( R \) if there exists a monic polynomial \( f \in R[T] \) such that \( f(x) = 0 \). 
\end{definition}
\begin{example}
    \begin{enumerate}
        \item If \( R = k \) is a field, then \( x \) is integral over \( k \) if and only if \( x \) is algebraic over \( k \).
        \item We will prove later that
        \begin{enumerate}
            \item the \( \mathbb Z \)-integral elements of \( \mathbb Q \) are \( \mathbb Z \);
            \item the \( \mathbb Z \)-integral elements of \( \mathbb Q\qty[\sqrt{2}] \) are \( \mathbb Z\qty[\sqrt{2}] \);
            \item the \( \mathbb Z \)-integral elements of \( \mathbb Q\qty[\sqrt{5}] \) are \( \mathbb Z\qty[\frac{1+\sqrt{5}}{2}] \supsetneq \mathbb Z\qty[\sqrt{5}] \).
        \end{enumerate}
    \end{enumerate}
\end{example}
\begin{definition}
    Let \( M \) be an \( R \)-module.
    We say that \( M \) is \emph{faithful} if the structure homomorphism \( \rho : R \to \End M \) is injective.
    Equivalently, for every nonzero ring element \( r \), there exists \( m \in M \) such that \( rm \neq 0 \).
\end{definition}
\begin{example}
    Let \( R \subseteq A \) be rings, and let \( A \) be an \( R \)-module in the natural way.
    Then \( A \) is a faithful \( R \)-module, as if \( r \neq 0 \), then \( r 1_A = r \neq 0 \).
\end{example}
\begin{proposition}
    Let \( R \subseteq A \) be rings and \( x \in A \), and consider \( A \) as an \( R[x] \)-module.
    Then \( x \) is integral over \( R \) if and only if there exists \( M \subseteq A \) such that
    \begin{enumerate}
        \item \( M \) is a faithful \( R[x] \)-module; and
        \item \( M \) is finitely generated as an \( R \)-module.
    \end{enumerate}
\end{proposition}
Condition (i) is that \( M \) is an \( R \)-submodule of \( A \), \( xM \subseteq M \), and \( M \) is faithful over \( R[x] \).
\begin{proof}
    First, assume conditions (i) and (ii) hold.
    We have an \( R \)-linear map \( f : M \to M \) given by multiplication by \( x \), as \( xM \subseteq M \).
    As \( M \) is a finitely generated \( R \)-module, we can apply the Cayley--Hamilton theorem to find
    \[ f^n + r_1 f^{n-1} + \dots + r_n f^0 =  0;\quad r_i \in R \]
    in \( \End_R M \).
    Then, evaluating at \( m \in M \),
    \[ (x^n + r_1 x^{n-1} + \dots + r_n x^0) m = 0 \]
    As this holds for all \( m \), and \( M \) is a faithful \( R[x] \)-module, we must have
    \[ x^n + r_1 x^{n-1} + \dots + r_n x^0 = 0 \]
    Thus \( x \) is integral over \( R \).

    Now suppose \( x \) is integral over \( R \).
    Then
    \[ x^n + r_1 x^{n-1} + \dots + r_n x^0 = 0 \]
    for some \( r_1, \dots, r_n \in R \).
    We define
    \[ M = \vecspan_R \qty{x_0, \dots, x^{n-1}} \]
    This is finitely generated, and satisfies \( xM \subseteq M \).
    \( M \) is faithful over \( R[x] \) as it contains \( x^0 = 1 \).
\end{proof}
