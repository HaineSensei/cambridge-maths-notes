\subsection{???}
\begin{definition}
    Let \( \mathfrak p \) be a prime ideal of \( R \).
    The \emph{height} of \( \mathfrak p \), denoted \( \operatorname{ht}(p) \), is
    \[ \operatorname{ht}(\mathfrak p) = \sup \qty{ d \mid \mathfrak p_0 \subsetneq \mathfrak p_1 \subsetneq \dots \subsetneq \mathfrak p_d = \mathfrak p; \mathfrak p_i \in \Spec R } \]
    The \emph{(Krull) dimension} of \( R \) is
    \[ \dim R = \sup \qty{\operatorname{ht}(\mathfrak p) \mid \mathfrak p \in \Spec R} = \sup \qty{\operatorname{ht}(\mathfrak m) \mid \mathfrak m \in \mSpec R} \]
\end{definition}
\begin{remark}
    The height of a prime ideal \( \mathfrak p \) is the Krull dimension of the localisation \( R_{\mathfrak p} \).
    In particular,
    \[ \dim R = \sup \qty{\dim R_{\mathfrak p} \mid \mathfrak p \in \Spec R} = \sup \qty{\dim R_{\mathfrak m} \mid \mathfrak m \in \mSpec R} \]
    So the problem of computing dimension can be reduced to computing dimension of local rings.
\end{remark}
\begin{definition}
    Let \( I \) be a proper ideal of \( R \).
    Then the \emph{height} of \( I \) is
    \[ \operatorname{ht}(I) = \inf \qty{\operatorname{ht}(\mathfrak p) \mid I \subseteq \mathfrak p} \]
\end{definition}
\begin{proposition}
    Let \( A \subseteq B \) be an integral extension of rings.
    Then,
    \begin{enumerate}
        \item \( \dim A = \dim B \); and
        \item if \( A, B \) are integral domains and \( k \)-algebras for some field \( k \), they have the same transcendence degree over \( k \).
    \end{enumerate}
\end{proposition}
We prove part (i); the second part is not particularly relevant for this course.
\begin{proof}
    First, we show that \( \dim A \leq \dim B \).
    Consider a chain of prime ideals \( \mathfrak p_0 \subsetneq \dots \subsetneq \mathfrak p_d \) in \( \Spec A \).
    By the lying over theirem and the going up theorem, we obtain a chain of prime ideals \( \mathfrak q_0 \subseteq \dots \subseteq \mathfrak q_d \) in \( \Spec B \).
    As \( \mathfrak p_i = \mathfrak q_i \cap A \) and \( \mathfrak p_i \neq \mathfrak p_{i+1} \), we must have \( \mathfrak q_i \neq \mathfrak q_{i+1} \).
    So this produces a chain of length \( d \) in \( B \), as required.

    Now consider a chain \( \mathfrak q_0 \subsetneq \dots \subsetneq \mathfrak q_d \) in \( \Spec B \).
    Contracting each ideal, we produce a chain \( \mathfrak p_0 \subseteq \dots \subseteq \mathfrak p_d \) in \( \Spec A \).
    Suppose that \( \mathfrak q_i \) and \( \mathfrak q_{i+1} \) contract to the same prime ideal \( \mathfrak p_i \) in \( \Spec A \).
    Note that \( \mathfrak q_i \subseteq \mathfrak q_{i+1} \), so by incomparability, they must be equal, but this is a contradiction.
\end{proof}
\begin{remark}
    If \( A \) is a finitely generated \( k \)-algebra for some field \( k \), then by Noether normalisation, we obtain a \( k \)-algebra embedding \( k[T_1, \dots, T_d] \to A \), and the extension is integral.
    Thus \( \dim A = \dim k[T_1, \dots, T_d] \).
    One can show that \( \dim k[T_1, \dots, T_d] = d \), and hence that the integer \( d \) obtained by Noether normalisation is uniquely determined by \( A \) and \( k \).
\end{remark}
