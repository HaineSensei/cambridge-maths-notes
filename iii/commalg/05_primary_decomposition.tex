\subsection{???}
\begin{definition}
    Let \( I \) be an ideal of \( R \).
    \( I \) is
    \begin{enumerate}
        \item \emph{prime} if \( \faktor{R}{I} \neq 0 \) and \( 0 \) is the only zero divisor of \( \faktor{R}{I} \);
        \item \emph{radical} if the only nilpotent element of \( \faktor{R}{I} \) is zero;
        \item \emph{primary} if \( \faktor{R}{I} \neq 0 \) and every zero divisor in \( \faktor{R}{I} \) is nilpotent.
    \end{enumerate}
\end{definition}
Prime ideals are radical and primary.
\( R \) is radical but not prime or primary.
\begin{example}
    \begin{enumerate}
        \item Let \( R = \mathbb Z \).
        The ideal \( (6) \) is radical but not primary, as \( \faktor{R}{(6)} \) contains zero divisors \( 2, 3 \) which are not nilpotent.
        The ideal \( (9) \) is primary but not radical.
    \end{enumerate}
\end{example}
