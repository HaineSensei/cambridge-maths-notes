\subsection{Introduction}
Let \( M \) and \( N \) be \( R \)-modules.
Informally, the tensor product of \( M \) and \( N \) over \( R \) is the set \( M \otimes_R N \) of all sums
\[ \sum_{i=1}^\ell m_i \otimes n_i;\quad m_i \in M, n_i \in N \]
subject to the relations
\begin{align*}
    (m_1 + m_2) \otimes n &= m_1 \otimes n + m_2 \otimes n \\
    m \otimes (n_1 + n_2) &= m \otimes n_1 + m \otimes n_2 \\
    (rm) \otimes n &= r(m \otimes n) \\
    m \otimes (rn) &= r(m \otimes n)
\end{align*}
This is a module that abstracts the notion of bilinearity between two modules.
\begin{example}
    Consider \( \faktor{\mathbb Z}{2\mathbb Z} \otimes_{\mathbb Z} \faktor{\mathbb Z}{3\mathbb Z} \).
    In this \( \mathbb Z \)-module,
    \[ x \otimes y = (3x) \otimes y = x \otimes (3y) = x \otimes 0 = x \otimes (0 \cdot 0) = 0 (x \otimes 0) = 0 \]
    Hence \( \faktor{\mathbb Z}{2\mathbb Z} \otimes_{\mathbb Z} \faktor{\mathbb Z}{3\mathbb Z} = 0 \).
\end{example}
\begin{example}
    Now consider \( {\mathbb R}^n \otimes_{\mathbb R} {\mathbb R}^\ell \).
    We will show later that this is isomorphic to \( \mathbb R^{n+\ell} \).
\end{example}

\subsection{Definitions}
\begin{definition}
    A map of \( R \)-modules \( f : M \times N \to L \) is \emph{\( R \)-bilinear} if for each \( m_0 \in M \) and \( n_0 \in N \), the maps \( n \mapsto f(m_0, n) \) and \( m \mapsto f(m, n_0) \) are \( R \)-linear (or equivalently, a homomorphism of \( R \)-modules).
\end{definition}
\begin{definition}
    Let \( M, N \) be \( R \)-modules.
    Let \( \mathcal F = R^{\oplus(M \times N)} \) be the free \( R \)-module with coordinates indexed by \( M \times N \).
    Define \( K \subseteq \mathcal F \) to be the submodule generated by the following set of relations:
    \begin{align*}
        &(m_1 + m_2, n) - (m_1, n) - (m_2, n) \\
        &(m, n_1 + n_2) - (m, n_1) - (m, n_2) \\
        &r (m, n) - (rm, n) \\
        &r (m, n) - (m, rn)
    \end{align*}
    The \emph{tensor product} \( M \otimes_R N \) is \( \faktor{\mathcal F}{K} \).
    We further define the \( R \)-bilinear map
    \[ i_{M \otimes N} : M \times N \to M \otimes N;\quad i_{M \otimes N}(m, n) = m \otimes n \]
\end{definition}
\begin{proposition}[universal property of the tensor product]
    The pair \( (M \otimes_R N, i_{M \otimes_R N}) \) satisfies the following universal property.
    For every \( R \)-module \( L \) and every \( R \)-bilinear map \( f : M \times N \to L \), there exists a unique homomorphism \( h : M \otimes_R N \to L \) such that the following diagram commutes.
    \[\begin{tikzcd}
        {M \times N} & {M \otimes_R N} \\
        & L
        \arrow["{i_{M \otimes_R N}}", from=1-1, to=1-2]
        \arrow["h", dashed, from=1-2, to=2-2]
        \arrow["f"', from=1-1, to=2-2]
    \end{tikzcd}\]
    Equivalently, \( h \circ i_{M \otimes_R N} = f \).
\end{proposition}
\begin{proof}
    The conclusion \( h \circ i_{M \otimes N} = f \) holds if and only if for all \( m, n \), we have
    \[ h(m \otimes n) = f(m, n) \]
    Note that the elements \( \qty{m \otimes n} \) generate \( M \otimes N \) as an \( R \)-module, so there is at most one \( h \).
    We now show that the definition of \( h \) on the pure tensors \( m \otimes n \) extends to an \( R \)-linear map \( M \otimes N \to L \).
    The map \( R^{\oplus(M \otimes N)} \to L \) given by \( (m, n) \mapsto f(m, n) \) exists by the universal property of the direct sum.
    However, this map vanishes on the generators of \( K \), so it factors through the quotient \( \faktor{\mathcal F}{K} \) as required.
\end{proof}
The universal property given above characterises the tensor product up to isomorphism.
\begin{proposition}
    Let \( M, N \) be \( R \)-modules, and \( (T, j) \) be an \( R \)-module and an \( R \)-bilinear map \( M \times N \to T \).
    Suppose that \( (T, j) \) satisfies the same universal property as \( M \otimes N \).
    Then there is a unique isomorphism of \( R \)-modules \( \varphi : M \otimes N \similarrightarrow T \) such that \( \varphi \circ i_{M \otimes N} = j \).
\end{proposition}
\begin{proof}
    By using the universal property of \( M \otimes N \) and \( T \), we obtain \( \varphi \) and \( \psi \) as follows.
    \[\begin{tikzcd}
        {M \otimes N} && T \\
        & {M \times N}
        \arrow["{i_{M \otimes N}}", from=2-2, to=1-1]
        \arrow["j"', from=2-2, to=1-3]
        \arrow["\varphi", shift left, dashed, from=1-1, to=1-3]
        \arrow["\psi", shift left, dashed, from=1-3, to=1-1]
    \end{tikzcd}\]
    The universal property states that \( \varphi \circ i_{M \otimes N} = j \) and \( \psi \circ j = i_{M \otimes N} \).
    Hence, \( \psi \circ \varphi \circ i_{M \otimes N} = i_{M \otimes N} \).
    This means that the following diagram commutes.
    \[\begin{tikzcd}
        {M \times N} && {M \otimes N} \\
        && {M \otimes N}
        \arrow["{i_{M \otimes N}}", from=1-1, to=1-3]
        \arrow["\id"', shift right=2, from=1-3, to=2-3]
        \arrow["{i_{M \otimes N}}"', from=1-1, to=2-3]
        \arrow["{\psi \circ \varphi}", shift left=2, from=1-3, to=2-3]
    \end{tikzcd}\]
    By the uniqueness condition of the universal property, \( \id = \psi \circ \varphi \).
    Similarly, \( \id = \varphi \circ \psi \).
    Hence, \( \varphi \) is an isomorphism \( M \otimes N \to T \) with \( \varphi \circ i_{M \otimes N} = j \).
    Uniqueness of \( \varphi \) is guaranteed by the universal property: it is the only solution to \( \varphi \circ i_{M \otimes N} = j \).
\end{proof}
