\begin{proposition}[Zariski's lemma]
    Let \( k \subseteq L \) be fields where \( L \) is finitely generated as a \( k \)-algebra.
    Then \( \dim_k L \) is finite.
\end{proposition}
\begin{proof}
    By Noether normalisation, we have
    \[ k \subseteq k[x_1, \dots, x_n] \subseteq L \]
    where \( x_1, \dots, x_n \) are algebraically independent over \( k \), and \( L \) is finite over \( k[x_1, \dots, x_n] \).
    As this is an integral extension of integral domains and \( L \) is a field, \( k[x_1, \dots, x_n] \) must be a field.
    But as \( k[x_1, \dots, x_n] \) is a polynomial algebra over \( k \), the \( x_i \) cannot be invertible.
    Hence \( n = 0 \), so \( k \subseteq L \) is finite as required.
\end{proof}
\begin{definition}
    Let \( k \subseteq \Omega \) be an extension of fields, where \( \Omega \) is algebraically closed.
    \begin{enumerate}
        \item Let \( S \subseteq k[T_1, \dots, T_n] \).
        We define
        \[ \mathbb V(S) = \qty{\vb x \in \Omega^n \mid \forall f \in S,\, f(\vb x) = 0} \]
        Sets of this form are called \emph{\( k \)-algebraic} subsets of \( \Omega^n \).
        \item Let \( X \subseteq \Omega^n \).
        We define
        \[ I(X) = \qty{f \in k[T_1, \dots, T_n] \mid \forall \vb x \in X,\, f(\vb x) = 0} \]
    \end{enumerate}
\end{definition}
Note that \( \mathbb V(S) = \mathbb V(I) \), where \( I \) is the ideal generated by \( S \).
Recall that for every finite field extension \( k \subseteq L \), there is a \( k \)-algebra embedding \( L \to \Omega \), because \( \Omega \) is algebraically closed.
\begin{theorem}
    Let \( \mathfrak a \subseteq k[T_1, \dots, T_n] \) be an ideal.
    Then
    \begin{enumerate}
        \item (weak Nullstellensatz) \( \mathbb V(\mathfrak a) = \varnothing \) if and only if \( 1 \in \mathfrak a \);
        \item (strong Nullstellensatz) \( I(\mathbb V(\mathfrak a)) = \sqrt{\mathfrak a} \).
    \end{enumerate}
\end{theorem}
\begin{proof}
    \emph{Weak Nullstellensatz.}
    Clearly if \( 1 \in \mathfrak a \) then \( \mathbb V(\mathfrak a) = \varnothing \), as \( 1 \neq 0 \).
    Now suppose \( 1 \notin \mathfrak a \).
    There is a maximal ideal \( \mathfrak m \in \mSpec k[T_1, \dots, T_n] \) such that \( \mathfrak a \subseteq \mathfrak m \).
    Then \( L = \faktor{k[T_1, \dots, T_n]}{\mathfrak m} \) is a field, which is finitely generated over \( k \) as an algebra.
    By Zariski's lemma, this extension is finitely generated as a module.
    Hence, there is an injective \( k \)-algebra homomorphism \( L \to \Omega \).
    Composing with the quotient map, we obtain a \( k \)-algebra homomorphism \( \varphi : k[T_1, \dots, T_n] \to \Omega \) with kernel \( \mathfrak m \).
    Now, let
    \[ \vb x = (\varphi(T_1), \dots, \varphi(T_n)) \in \Omega^n \]
    We claim that this is a common solution to all polynomials in \( \mathfrak a \).
    Note that for \( f \in k[T_1, \dots, T_n] \), we have \( \varphi(f) = f(\vb x) \).
    Therefore, for all \( f \in \mathfrak a \), we have \( f \in \ker \varphi \) so \( f(\vb x) = \varphi(f) = 0 \).

    \emph{Strong Nullstellensatz.}
    Let \( f \in \sqrt{\mathfrak a} \).
    Then \( f^\ell \in \mathfrak a \) for some \( \ell \geq 1 \), and therefore, \( f^\ell(\vb x) = 0 \) for all \( \vb x \in \mathbb V(\mathfrak a) \).
    As \( \Omega \) is an integral domain, \( f(\vb x) = 0 \) for all \( \vb x \in \mathbb V(\mathfrak a) \).
    Hence \( f \in I(\mathbb V(\mathfrak a)) \).

    Conversely, suppose \( f \in I(\mathbb V(\mathfrak a)) \), so for all \( \vb x \in \mathbb V(\mathfrak a) \), we have \( f(\vb x) = 0 \).
    We want to show that \( f \in \sqrt{\mathfrak a} \).
    To do this, we show that \( \overline f \) is nilpotent in \( \faktor{k[T_1, \dots, T_n]}{\mathfrak a} \).
    It suffices to show that
    \[ \qty(\faktor{k[T_1, \dots, T_n]}{\mathfrak a})_{\overline f} = 0 \]
    Note that
    \[ \qty(\faktor{k[T_1, \dots, T_n]}{\mathfrak a})_{\overline f} \simeq \faktor{k[T_1, \dots, T_n, T_{n+1}]}{\mathfrak b};\quad \mathfrak b = \mathfrak a^e + (T_{n+1} f - 1) \]
    We will show that \( 1 \in \mathfrak b \), or equivalently by the weak Nullstellensatz, \( \mathbb V(\mathfrak b) = \varnothing \).

    Suppose \( \vb x = (x_1, \dots, x_{n+1}) \in \mathbb V(\mathfrak b) \subseteq \Omega^{n+1} \).
    Define \( \vb x_0 = (x_1, \dots, x_n) \), so \( \vb x_0 \in \mathbb V(\mathfrak a) \).
    In particular, \( f(\vb x_0) = 0 \), as \( f \in I(\mathbb V(\mathfrak a)) \).
    Thus \( f(\vb x) = 0 \).
    Now, \( (T_{n+1} f - 1)(\vb x) = -1 \neq 0 \), but \( (T_{n+1} f - 1) \in \mathfrak b \), so \( \vb x \) is not a common solution to all polynomials in \( \mathfrak b \), which is a contradiction.
\end{proof}
One can easily derive the weak Nullstellensatz from the strong Nullstellensatz.

Note that
\begin{enumerate}
    \item \( \sqrt{\sqrt{\mathfrak a}} = \sqrt{\mathfrak a} \).
    \item If \( X \subseteq Y \subseteq \Omega^n \), then \( I(X) \supseteq I(Y) \).
    \item If \( S \subseteq T \subseteq k[T_1, \dots, T_n] \), then \( \mathbb V(S) \supseteq \mathbb V(T) \).
    \item If \( S \subseteq k[T_1, \dots, T_n] \), then \( S \subseteq I(\mathbb V(S)) \).
    \item If \( X \subseteq \Omega^n \), then \( X \subseteq \mathbb V(I(X)) \).
    \item If \( X \subseteq \Omega^n \) is an algebraic set, then \( X = \mathbb V(I(X)) \), as \( X = \mathbb V(\mathfrak a) \) gives
    \[ \mathbb V(\mathfrak a) \subseteq \mathbb V(I(\mathbb V(\mathfrak a))) \subseteq \mathbb V(\mathfrak a) \]
    \item If \( X \subseteq \Omega^n \), then \( I(X) \) is a radical ideal.
\end{enumerate}
\begin{proposition}
    Let \( k = \Omega \) be an algebraically closed field, and let \( n \geq 0 \).
    Then we have an inclusion-reversing bijection
    \[ \qty{\text{\( k \)-algebraic subsets of } \Omega^n} \leftrightarrow \qty{\text{radical ideals of } k[T_1, \dots, T_n]} \]
    given by \( X \mapsto I(X) \) and \( \mathbb V(\mathfrak a) \mapsfrom \mathfrak a \).
\end{proposition}
\begin{proof}
    We have already shown that \( I(X) \) is radical, and \( X = \mathbb V(I(X)) \) if \( X \) is an algebraic set.
    For the converse, let \( \mathfrak a \subseteq k[T_1, \dots, T_n] \) be a radical ideal.
    Then \( I(\mathbb V(\mathfrak a)) = \sqrt{\mathfrak a} = \mathfrak a \) by the strong Nullstellensatz.
\end{proof}
\begin{remark}
    Every prime ideal \( \mathfrak p \) is radical, as \( x^n \in \mathfrak p \) implies \( x \in \mathfrak p \).
    In particular, every maximal ideal is radical.
\end{remark}
\begin{corollary}
    Let \( k = \Omega \) be an algebraically closed field.
    Then we have a bijection
    \[ \Omega^n \leftrightarrow \mSpec k[T_1, \dots, T_n] \]
    given by \( \vb x = (x_1, \dots, x_n) \mapsto (T_1 - x_1, \dots, T_n - x_n) = \mathfrak m_{\vb x} \).
\end{corollary}
\begin{proof}
    First, note that \( \mathfrak m_{\vb x} \) is a maximal ideal for every \( \vb x \), since it is the kernel of the map \( k[T_1, \dots, T_n] \twoheadrightarrow \Omega \) given by \( T_i \to x_i \).
    Also, \( \mathfrak m_{\vb x} = I(\qty{\vb x}) \).
    Indeed, the inclusion \( \mathfrak m_{\vb x} \subseteq I(\qty{\vb x}) \) is clear, and \( I(\qty{\vb x}) \) is a proper ideal of \( k[T_1, \dots, T_n] \), so they must be equal by maximality.
    Note that \( \mathbb V(\mathfrak m_{\vb x}) = \qty{\vb x} \).
    Hence the claim follows from the inclusion-reversing bijection, as maximal ideals correspond to minimal nonempty \( k \)algebraic sets.
\end{proof}
\begin{definition}
    We say that \( X \subseteq \Omega^n \) is \emph{irreducible} if \( X \) cannot be expressed as the union of two strictly smaller algebraic subsets.
\end{definition}
\begin{proposition}
    \( X \subseteq \Omega^n \) is irreducible if and only if \( I(X) \) is prime.
\end{proposition}
% TODO: Fill in!
