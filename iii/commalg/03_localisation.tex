\subsection{Definitions}
\begin{definition}
    A \emph{multiplicative set} or \emph{multiplicatively closed set} \( S \subseteq R \) is a subset such that \( 1 \in S \) and if \( a, b \in S \), then \( ab \in S \).
    If \( U \subseteq R \) is any set, its \emph{multiplicative closure} \( S \) of \( U \) is the set
    \[ \qty{\prod_{i = 1}^n u_i \midd n \geq 0, u_i \in U} \]
    which is the smallest multiplicatively closed set containing \( U \).
\end{definition}
\begin{example}
    \begin{enumerate}
        \item If \( R \) is an integral domain, then \( S = R \setminus \qty{0} \) is multiplicative.
        \item More generally, if \( \mathfrak p \) is a prime ideal in \( R \), then \( S = R \setminus \mathfrak p \) is multiplicative.
        \item If \( x \in R \), then the set \( \qty{x^n \mid n \geq 0} \) is multiplicative.
    \end{enumerate}
\end{example}
\begin{remark}
    \( \mathbb Q \) is obtained from \( \mathbb Z \) by adding inverses for the elements of the multiplicative subset \( \mathbb Z \setminus \qty{0} \).
    We have a ring homomorphism \( \mathbb Z \hookrightarrow \mathbb Q \).
    We generalise this construction to arbitrary rings and multiplicative sets.
    In general, injectivity of the ring homomorphism in question may fail.
\end{remark}
\begin{definition}
    Let \( S \subseteq R \) be a multiplicative set, and let \( M \) be an \( R \)-module.
    Then the \emph{localisation} of \( M \) by \( S \) is the set \( S^{-1} M = \faktor{M \times S}{\sim} \) where \( (m_1, s_1) \sim (m_2, s_2) \) if and only if there exists \( u \in S \) such that \( u(s_2 m_1 - s_1 m_2) = 0 \).
    We write \( \frac{m}{s} \) for the equivalence class corresponding to \( (m, s) \).
    We make \( S^{-1} M \) into an \( R \)-module by defining
    \[ \frac{m_1}{s_1} + \frac{m_2}{s_2} = \frac{m_1 s_2 + m_2 s_1}{s_1 s_2};\quad r \cdot \frac{m}{s} = \frac{rm}{s} \]
    We can make \( S^{-1} R \) into a ring by defining
    \[ \frac{r_1}{s_1} \cdot \frac{r_2}{s_2} = \frac{r_1 r_2}{s_1 s_2} \]
    Then \( S^{-1}M \) is an \( S^{-1}R \)-module by
    \[ \frac{r}{s} \cdot \frac{m}{t} = \frac{r m}{s t} \]
    We have the localisation map \( R \to S^{-1}R \) given by \( r \mapsto \frac{r}{1} \), which is a ring homomorphism.
    We also have the localisation map \( M \to S^{-1}M \) given by \( m \mapsto \frac{m}{1} \), which is a homomorphism of \( R \)-modules.
\end{definition}
We must show that \( \sim \) is an equivalence relation.
The only nontrivial thing to prove is transitivity.
Let
\[ u(s_2 m_1 - s_1 m_2) = 0 = v(s_3 m_2 - s_2 m_3);\quad u, v \in S \]
Then
\[ 0 = uv(s_2 s_3 m_1 - s_1 s_3 m_2) + uv(s_1 s_3 m_2 - s_1 s_2 m_3) = uvs_2(s_3 m_1 - s_1 m_3);\quad uvs_2 \in S \]
as required.
All other operations mentioned are well-defined; the proofs are not enlightening so are omitted.

\subsection{Universal property}
\begin{proposition}
    Let \( U \subseteq R \), and let \( S \subseteq R \) be its multiplicative closure.
    Let \( f : R \to B \) be a ring homomorphism such that \( f(u) \) is a unit for all \( u \in U \).
    Then there is a unique ring homomorphism \( h : S^{-1}R \to B \) such that the following diagram commutes.
    \[\begin{tikzcd}
        R & {S^{-1}R} \\
        & B
        \arrow["{\iota_{S^{-1}R}}", from=1-1, to=1-2]
        \arrow["h", dashed, from=1-2, to=2-2]
        \arrow["f"', from=1-1, to=2-2]
    \end{tikzcd}\]
    where \( \iota_{S^{-1}R}(r) = \frac{r}{1} \), so in particular, \( f(r) = h\qty(\frac{r}{1}) \).
\end{proposition}
Thus
\[ \Hom_{\text{Ring}}(S^{-1}R, B) \simeq \qty{\varphi \in \Hom_{\text{Ring}}(R, B) \mid \varphi(U) \subseteq B^\times} \]
mapping
\[ f \mapsto \qty(r \mapsto \frac{r}{1});\quad \qty(\frac{r}{s} \mapsto \frac{\varphi(r)}{\varphi(s)}) \mapsfrom \varphi \]
\begin{proof}
    Let \( f : R \to B \) be a ring homomorphism such that \( f(u) \) is a unit for all \( u \in U \).
    Then \( f(s) \) is a unit for all \( s \in S \).
    We want to construct a ring homomorphism \( h : S^{-1}R \to B \) such that \( f(r) = h\qty(\frac{r}{1}) \) for all \( r \in R \).
    Such an \( h \) must satisfy the following condition.
    \[ 1 = h(1) = h\qty(\frac{1}{s} \cdot \frac{s}{1}) = h\qty(\frac{1}{s}) f(s) \]
    Thus \( h\qty(\frac{1}{s}) = f(s)^{-1} \).
    Hence, we must have
    \[ h\qty(\frac{r}{s}) = h\qty(\frac{1}{s}) h\qty(\frac{r}{1}) = f(s)^{-1} f(r) \]
    It thus suffices to show that this \( h \) is well-defined; it is then a ring homomorphism satisfying the correct property.
    If \( \frac{r_1}{s_1} = \frac{r_2}{s_2} \), then there is \( t \in S \) such that \( t s_2 r_1 = t s_1 r_2 \).
    Applying \( f \),
    \[ f(t) f(s_2) f(r_1) = f(t) f(s_1) f(r_2) \]
    As \( f(t), f(s_1), f(s_2) \) are invertible,
    \[ \frac{f(r_1)}{f(s_1)} = \frac{f(r_2)}{f(s_2)} \]
    so \( h \) is well-defined.
\end{proof}
\begin{proposition}
    Suppose \( (A, j) \) has the same universal property of \( (S^{-1}R, \iota_{S^{-1}R}) \) where \( \iota_{S^{-1}R}(r) = \frac{r}{1} \), then there is a unique ring isomorphism \( S^{-1}R \to A \) mapping \( \frac{r}{s} \) to \( j(s)^{-1} j(r) \).
\end{proposition}
\begin{remark}
    \begin{enumerate}
        \item Let \( \frac{r}{s} \in S^{-1}R \).
        Then \( \frac{r}{s} = \frac{0}{1} \) if and only if there exists \( u \in S \) such that \( ur = 0 \).
        \item In particular, \( S^{-1}R = 0 \) when \( \frac{1}{1} = \frac{0}{1} \), which occurs precisely when \( 0 \in S \).
        \item \( \ker \iota_{S^{-1}R} = \qty{r \in R \mid \exists u \in S,\, ur = 0} \).
        \item \( \iota_{S^{-1}R} \) is injective if and only if \( S \) contains no zero divisors.
        \item \( \iota_{S^{-1}R} \) is always an epimorphism, but usually not surjective.
        For example, the map \( \iota : \mathbb Z \hookrightarrow \mathbb Q \) is epic.
        Indeed, for \( f, g : \mathbb Q \to A \) are such that \( f \circ \iota = g \circ \iota \), then
        \[ f\qty(\frac{p}{q}) = \frac{f(\iota(p))}{f(\iota(q))} = \frac{g(\iota(p))}{g(\iota(q))} = g\qty(\frac{p}{q}) \]
    \end{enumerate}
\end{remark}
\begin{example}
    \begin{enumerate}
        \item Let \( f \in R \) and define \( S = \qty{f^n \mid n \geq 0} \).
        Define \( R_f = S^{-1}R \).
        Taking for instance \( R = \mathbb Z \) and \( f = 2 \),
        \[ R_f = \qty{\frac{a}{2^n} \midd a \in \mathbb Z,\, n \geq 0} = \mathbb Z\qty[\frac{1}{2}] \]
        producing the ring of dyadic rational numbers.
        Since we write \( \faktor{\mathbb Z}{n\mathbb Z} \) for the finite quotient ring and \( \mathbb Z_2 \) for the 2-adic integers, we must use the notation \( \mathbb Z\qty[\frac{1}{2}] \) for this particular construction instead.
        Thus \( R_f \) is the zero ring if and only if \( f \) is nilpotent.
        \item Let \( \mathfrak p \in \Spec R \), where \( \Spec R \) is the set of prime ideals in \( R \).
        Then \( S = R \setminus \mathfrak p \) is a multiplicative set.
        Consider \( (R \setminus \mathfrak p)^{-1} R = R_{\mathfrak p} \).
        For example,
        \[ \mathbb Z_{(3)} = \qty{\frac{a}{b} \mid a, b \in \mathbb Z,\, 3 \nmid b} \]
    \end{enumerate}
\end{example}

\subsection{Functoriality}
\begin{proposition}
    Let \( M \) be an \( R \)-module and \( S \subseteq R \) be a multiplicative set.
    Then there is an isomorphism of \( S^{-1}R \)-modules
    \[ S^{-1}R \otimes_R M \to S^{-1}M \]
    given by \( \frac{r}{s} \otimes m \mapsto \frac{rm}{s} \).
\end{proposition}
Thus the localisation of any module can be reduced to a tensor product with the localisation of a ring.
\begin{proof}
    Define the map \( S^{-1}R \times M \to S^{-1}M \) mapping \( \qty(\frac{r}{s}, m) \mapsto \frac{rm}{s} \); this is bilinear and thus gives rise to an \( R \)-linear map \( \varphi : S^{-1}R \otimes M \to S^{-1}M \) with the desired action on pure tensors.
    One can check that this is in fact \( S^{-1} R \)-linear.
    Clearly \( \varphi \) is surjective by \( \frac{1}{s} \otimes m \mapsto \frac{m}{s} \).
    For injectivity, we first show that every tensor
    \[ \sum_i \frac{r_i}{s_i} \otimes m_i \in S^{-1}R \otimes_R M \]
    is pure.
    We define
    \[ s = \prod_i s_i;\quad t_j = \prod_{j \neq i} s_j \]
    hence
    \[ \sum_i \frac{r_i}{s_i} \otimes m_i = \sum_i \frac{1}{s_i} \otimes r_i m_i = \sum_i \frac{t_i}{s} \otimes r_i m_i = \sum_i \frac{1}{s} \otimes t_i r_i m_i = \frac{1}{s} \otimes \sum_i t_i r_i m_i \]
    as required.
    Now, it suffices to prove injectivity on pure tensors.
    If \( \varphi\qty(\frac{1}{s} \otimes m) = \frac{0}{1} \), then there exists \( u \in S \) such that
    \[ u(1m - 0s) = 0 \implies um = 0 \]
    Thus
    \[ \frac{1}{s} \otimes m = \frac{u}{us} \otimes m = \frac{1}{us} \otimes um = \frac{1}{us} \otimes 0 = 0 \]
    as required.
\end{proof}
The map \( S^{-1}R \otimes (-) \) acts on modules and on morphisms.
The map \( S^{-1}(-) \) acts on modules, and can be extended to act on morphisms in the following way.
If \( f : N \to N' \) is \( R \)-linear, we produce the commutative diagram
% https://q.uiver.app/#q=WzAsNCxbMCwwLCJTXnstMX1SIFxcb3RpbWVzX1IgTiJdLFsxLDAsIlNeey0xfVIgXFxvdGltZXNfUiBOJyJdLFsxLDEsIlNeey0xfU4nIl0sWzAsMSwiU157LTF9TiJdLFswLDEsIlxcaWRfe1Neey0xfSBSfSBcXG90aW1lcyBmIl0sWzEsMiwiXFxzaW0iXSxbMCwzLCJcXHNpbSIsMl0sWzMsMiwiU157LTF9KGYpIiwyLHsic3R5bGUiOnsiYm9keSI6eyJuYW1lIjoiZGFzaGVkIn19fV1d
\[\begin{tikzcd}
	{S^{-1}R \otimes_R N} & {S^{-1}R \otimes_R N'} \\
	{S^{-1}N} & {S^{-1}N'}
	\arrow["{\id_{S^{-1} R} \otimes f}", from=1-1, to=1-2]
	\arrow["\sim", from=1-2, to=2-2]
	\arrow["\sim"', from=1-1, to=2-1]
	\arrow["{S^{-1}(f)}"', dashed, from=2-1, to=2-2]
\end{tikzcd}\]
with action
% https://q.uiver.app/#q=WzAsNCxbMCwxLCJcXGZyYWN7bn17c30iXSxbMCwwLCJcXGZyYWN7MX17c30gXFxvdGltZXMgbiJdLFsxLDAsIlxcZnJhY3sxfXtzfSBcXG90aW1lcyBmKG4pIl0sWzEsMSwiXFxmcmFje2Yobil9e3N9Il0sWzAsMSwiIiwwLHsic3R5bGUiOnsidGFpbCI6eyJuYW1lIjoibWFwcyB0byJ9fX1dLFsxLDIsIiIsMCx7InN0eWxlIjp7InRhaWwiOnsibmFtZSI6Im1hcHMgdG8ifX19XSxbMiwzLCIiLDAseyJzdHlsZSI6eyJ0YWlsIjp7Im5hbWUiOiJtYXBzIHRvIn19fV0sWzAsMywiIiwyLHsic3R5bGUiOnsidGFpbCI6eyJuYW1lIjoibWFwcyB0byJ9LCJib2R5Ijp7Im5hbWUiOiJkYXNoZWQifX19XV0=
\[\begin{tikzcd}
	{\frac{1}{s} \otimes n} & {\frac{1}{s} \otimes f(n)} \\
	{\frac{n}{s}} & {\frac{f(n)}{s}}
	\arrow[maps to, from=2-1, to=1-1]
	\arrow[maps to, from=1-1, to=1-2]
	\arrow[maps to, from=1-2, to=2-2]
	\arrow[dashed, maps to, from=2-1, to=2-2]
\end{tikzcd}\]
Then the functor \( S^{-1}R \otimes_R (-) \) is naturally isomorphic to the functor \( S^{-1}(-) \).
\begin{remark}
    If \( A \) is an \( R \)-algebra, then we have an \( S^{-1}R \)-linear isomorphism \( S^{-1}R \otimes_R A \similarrightarrow S^{-1}A \); this is also an isomorphism of \( S^{-1}R \)-algebras.
\end{remark}
\begin{lemma}
    Let \( M \) be an \( S^{-1}R \)-module.
    Treating \( M \) as an \( R \)-module, we can define \( S^{-1}M \).
    Then,
    \[ S^{-1}M \simeq M \]
    as \( S^{-1}R \)-modules, mapping \( \frac{m}{s} \mapsto \frac{1}{s} m \).
\end{lemma}
Equivalently, \( M \simeq S^{-1}R \otimes_R M \) as \( S^{-1}R \)-modules, mapping \( m \mapsto \frac{1}{1} \otimes m \).
\begin{proof}
    The localisation map \( M \to S^{-1}M \) maps \( m \mapsto \frac{m}{1} \).
    This is \( S^{-1}R \)-linear, and surjective as \( \frac{1}{s} \cdot m \mapsto \frac{m}{s} \).
    To show injectivity, note that \( \frac{m}{1} = \frac{0}{1} \) implies there exists \( u \in S \) with \( um = 0 \).
    Multiplying by \( \frac{1}{u} \) as \( M \) is an \( S^{-1}R \)-module we obtain \( m = 0 \) as required.
\end{proof}

\subsection{Universal property}
Recall that if \( U \) has multiplicative closure \( S \),
\[ \Hom_{\text{Ring}}(S^{-1}R, B) \simeq \qty{\varphi \in \Hom_{\text{Ring}}(R, B) \mid \varphi(U) \subseteq B^\times} \]
If \( M \) is a fixed \( R \)-module and \( L \) is an \( S^{-1}R \)-module, we have
\[ \Hom_R(M, L) \simeq \Hom_{S^{-1}R}(S^{-1}M, L) \]
\begin{proposition}
    Let \( M \) be an \( R \)-module and \( L \) be an \( S^{-1}R \)-module.
    Let \( f : M \to L \) be \( R \)-linear.
    Then there exists a unique \( S^{-1}R \)-linear map \( h : S^{-1}M \to L \) such that \( f = h \circ i_{S^{-1}M} \).
    % https://q.uiver.app/#q=WzAsMyxbMCwwLCJNIl0sWzEsMCwiU157LTF9TSJdLFsxLDEsIkwiXSxbMCwxLCJpX3tTXnstMX1NfSJdLFsxLDIsImgiLDAseyJzdHlsZSI6eyJib2R5Ijp7Im5hbWUiOiJkYXNoZWQifX19XSxbMCwyLCJmIiwyXV0=
\[\begin{tikzcd}
	M & {S^{-1}M} \\
	& L
	\arrow["{i_{S^{-1}M}}", from=1-1, to=1-2]
	\arrow["h", dashed, from=1-2, to=2-2]
	\arrow["f"', from=1-1, to=2-2]
\end{tikzcd}\]
\end{proposition}
As usual with universal properties, this characterises \( S^{-1}M \) uniquely up to unique isomorphism.
\begin{proof}
    We use the natural isomorphism between \( S^{-1}(-) \) and \( S^{-1}R \otimes_R (-) \).
    After applying this, we have a map
    \[ \iota : M \to S^{-1}R \otimes_R M;\quad m \mapsto \frac{1}{1} \otimes m \]
    Let \( f : M \to L \) be \( R \)-linear, and define
    \[ h = \id_{S^{-1}R} \otimes f : S^{-1}R \otimes_R M \to S^{-1}R \otimes_R L \]
    Note that \( S^{-1}R \otimes_R L \simeq L \), so we can consider \( h \) as mapping to \( L \), with action
    \[ h\qty(\frac{r}{s} \otimes m) = \frac{r}{s}f(m) \]
    Uniqueness of \( h \) follows from the fact that \( \qty{1 \otimes m}_{m \in M} \) generate \( S^{-1}R \otimes_R M \) as an \( S^{-1}R \)-module.
\end{proof}

\subsection{Exactness}
\begin{proposition}
    The functor \( S^{-1}(-) \) is exact.
    More explicitly, if
    % https://q.uiver.app/#q=WzAsMyxbMCwwLCJBIl0sWzEsMCwiQiJdLFsyLDAsIkMiXSxbMCwxLCJmIl0sWzEsMiwiZyJdXQ==
\[\begin{tikzcd}
	A & B & C
	\arrow["f", from=1-1, to=1-2]
	\arrow["g", from=1-2, to=1-3]
\end{tikzcd}\]
    is an exact sequence of \( R \)-modules, then
    % https://q.uiver.app/#q=WzAsMyxbMCwwLCJTXnstMX1BIl0sWzEsMCwiU157LTF9QiJdLFsyLDAsIlNeey0xfUMiXSxbMCwxLCJTXnstMX1mIl0sWzEsMiwiU157LTF9ZyJdXQ==
\[\begin{tikzcd}
	{S^{-1}A} & {S^{-1}B} & {S^{-1}C}
	\arrow["{S^{-1}f}", from=1-1, to=1-2]
	\arrow["{S^{-1}g}", from=1-2, to=1-3]
\end{tikzcd}\]
    is an exact sequence of \( S^{-1}R \)-modules.
\end{proposition}
\begin{proof}
    First,
    \[ (S^{-1}g) \circ (S^{-1}f) = S^{-1}(g \circ f) = S^{-1}0 = 0 \]
    so \( \Im S^{-1}f \subseteq \ker S^{-1}g \).
    Now suppose \( \frac{b}{s} \in \ker S^{-1}g \), so \( \frac{g(b)}{s} = \frac{0}{1} \).
    Hence there exists \( u \in S \) such that \( ug(b) = 0 \).
    As \( g \) is \( R \)-linear and \( u \in R \), we have \( g(ub) = 0 \).
    By exactness, \( ub \in \ker g = \Im f \).
    Thus there exists \( a \in A \) such that \( f(a) = ub \).
    Hence,
    \[ \frac{b}{s} = \frac{ub}{us} = \frac{f(a)}{us} = S^{-1}f\qty(\frac{a}{us}) \]
\end{proof}
In particular, \( S^{-1}R \) is a flat \( R \)-module, so for example \( \mathbb Q \) is a flat \( \mathbb Z \)-module.
\begin{remark}
    Suppose \( N \subseteq M \) are \( R \)-modules, and \( \iota : N \to M \) is the inclusion map.
    Then applying the localisation, the map \( S^{-1}\iota : S^{-1}N \to S^{-1}M \) given by \( \frac{n}{s} \mapsto \frac{n}{s} \) is still injective.
    Note that the similar result for tensor products fails.
\end{remark}
\begin{proposition}
    Let \( M \) be an \( R \)-module and \( N, P \) be submodules of \( M \).
    Then,
    \begin{enumerate}
        \item \( S^{-1}(N + P) = S^{-1}N + S^{-1}P \);
        \item \( S^{-1}(N \cap P) = S^{-1}N \cap S^{-1}P \);
        \item \( \faktor{S^{-1}M}{S^{-1}N} \similarrightarrow S^{-1}\qty(\faktor{M}{N}) \) given by \( \frac{m}{s} + S^{-1}N \mapsto \frac{m + N}{s} \).
    \end{enumerate}
\end{proposition}
Parts (i) and (ii) rely on a slight abuse of notation, thinking of \( S^{-1}N \) as a submodule of \( S^{-1}M \).
Due to the above remark, this should not cause confusion.
\begin{proof}
    \emph{Part (i).}
    Note that
    \[ \frac{n+p}{s} = \frac{n}{s} + \frac{p}{s} \in S^{-1}N + S^{-1}P \]
    and
    \[ \frac{n}{s_1} + \frac{p}{s_2} = \frac{s_2 n + s_1 p}{s_1 s_2} \in S^{-1}(N + P) \]
    
    \emph{Part (ii).}
    The forward inclusion is clear.
    Conversely, suppose \( x \in S^{-1}N \cap S^{-1}P \), so \( x = \frac{n}{s_1} = \frac{p}{s_2} \).
    Hence, there exists \( u \in S \) such that \( u s_2 n = u s_1 p = w \).
    Note \( u s_2 n \in N \) and \( u s_1 p \in P \), so \( w \in N \cap P \).
    Now,
    \[ x = \frac{n}{s_1} = \frac{us_2 n}{u s_1 s_2} = \frac{w}{u s_1 s_2} \in S^{-1}(N \cap P) \]
    
    \emph{Part (iii).}
    Consider the short exact sequence
    % https://q.uiver.app/#q=WzAsNSxbMCwwLCIwIl0sWzEsMCwiTiJdLFsyLDAsIk0iXSxbMywwLCJcXGZha3RvcntNfXtOfSJdLFs0LDAsIjAiXSxbMCwxXSxbMSwyLCJcXGlvdGEiXSxbMiwzLCJcXHBpIl0sWzMsNF1d
\[\begin{tikzcd}
	0 & N & M & {\faktor{M}{N}} & 0
	\arrow[from=1-1, to=1-2]
	\arrow["\iota", from=1-2, to=1-3]
	\arrow["\pi", from=1-3, to=1-4]
	\arrow[from=1-4, to=1-5]
\end{tikzcd}\]
    Applying the exact functor \( S^{-1}(-) \), we obtain the short exact sequence
    % https://q.uiver.app/#q=WzAsNSxbMCwwLCIwIl0sWzEsMCwiU157LTF9TiJdLFsyLDAsIlNeey0xfU0iXSxbMywwLCJTXnstMX1cXHF0eShcXGZha3RvcntNfXtOfSkiXSxbNCwwLCIwIl0sWzAsMV0sWzEsMiwiU157LTF9XFxpb3RhIl0sWzIsMywiU157LTF9XFxwaSJdLFszLDRdXQ==
\[\begin{tikzcd}
	0 & {S^{-1}N} & {S^{-1}M} & {S^{-1}\qty(\faktor{M}{N})} & 0
	\arrow[from=1-1, to=1-2]
	\arrow["{S^{-1}\iota}", from=1-2, to=1-3]
	\arrow["{S^{-1}\pi}", from=1-3, to=1-4]
	\arrow[from=1-4, to=1-5]
\end{tikzcd}\]
    Thus
    \[ (S^{-1}\iota)(S^{-1}N) = S^{-1}N \subseteq S^{-1}M \]
    and
    \[ (S^{-1}\pi)\qty(\frac{m}{s}) = \frac{m+N}{s} \]
    giving the isomorphism as required.
\end{proof}
\begin{proposition}
    Let \( M, N \) be \( R \)-modules.
    Then
    \[ S^{-1}M \otimes_{S^{-1}R} S^{-1}N \similarrightarrow S^{-1}(M \otimes_R N) \]
\end{proposition}
\begin{proof}
    We have already proven that
    \[ (S^{-1}R \otimes_R M) \otimes_{S^{-1}R} (S^{-1}R \otimes_R N) \simeq S^{-1}R \otimes_R (M \otimes_R N) \]
    giving the result as required.
\end{proof}
\begin{example}
    Let \( \mathfrak p \) be a prime ideal in \( R \).
    Then by setting \( S = R \setminus \mathfrak p \),
    \[ M_{\mathfrak p} \otimes_{R_{\mathfrak p}} N_{\mathfrak p} \simeq (M \otimes_R N)_{\mathfrak p} \]
\end{example}

\subsection{Extension and contraction of ideals}
If \( f : A \to B \) is a ring homomorphism and \( \mathfrak b \) is an ideal in \( B \), the preimage \( f^{-1}(\mathfrak b) = \mathfrak b^c \) is an ideal in \( A \), called its \emph{contraction}.
If \( \mathfrak a \) is an ideal in \( A \), we can generate an ideal \( (f(\mathfrak a)) = \mathfrak a^e \) in \( B \), called its \emph{extension}.
We show on the first example sheet that for any ring homomorphism \( f : A \to B \), there is a bijection
\[ \qty{\text{contracted ideals of } A} \leftrightarrow \qty{\text{extended ideals of } B} \]
noting that the contracted ideals are those ideals with \( \mathfrak a = \mathfrak a^{ec} \), and the extended ideals are those ideals with \( \mathfrak b = \mathfrak b^{ce} \), where the bijection maps \( \mathfrak a \mapsto \mathfrak a^e \) and \( \mathfrak b^c \mapsfrom \mathfrak b \).

We now study the special case where \( f : R \to S^{-1}R \) is the localisation map of a ring, given by \( r \mapsto \frac{r}{1} \).
In this case, the extension of an ideal is written \( S^{-1}\mathfrak a = \mathfrak a^e \).
We claim that
\[ \mathfrak a^e = \qty{\frac{a}{s} \midd a \in \mathfrak a, s \in S} \]
Indeed, \( \mathcal a^e \) is generated by \( \qty{\frac{a}{1} \midd a \in \mathfrak a} \), so \( \mathfrak a^e \) must contain \( \qty{\frac{a}{s} \midd a \in \mathfrak a, s \in S} \), but this is already an ideal.
We also claim that
\[ \mathcal a^{ec} = \bigcup_{s \in S} (\mathfrak a : s);\quad (\mathfrak a : s) = \qty{r \in R \mid rs \in \mathfrak a} \]
Indeed, for \( r \in \bigcup_{s \in S} (\mathfrak a : s) \), we have \( rs = a \) in \( R \) for some \( s \in S \) and \( a \in \mathfrak a \), so \( \frac{rs}{1} = \frac{a}{1} \), giving \( \frac{r}{1} = \frac{a}{s} \), so \( r \in \mathfrak a^{ec} \) as required.
In the other direction, if \( r \in \mathfrak a^{ec} \), then \( \frac{r}{1} = \frac{a}{s} \) for some \( s \in S \) and \( a \in \mathfrak a \), so there exists \( u \in S \) such that \( rus = ua \in \mathfrak a \), so \( r \in (\mathfrak a : us) \) as required.

Now, let \( \mathfrak b \) be an ideal of \( S^{-1}R \).
Then
\[ \mathfrak b^c = \qty{r \in R \midd \frac{r}{1} \in \mathfrak b} \]
We claim that \( \mathfrak b^{ce} = \mathfrak b \), so all ideals in \( S^{-1}R \) are extended.
Note that the inclusion \( \mathfrak b^{ce} \subseteq \mathfrak b \) holds for any pair of rings.
For the reverse inclusion, consider \( \frac{r}{s} \in \mathfrak b \), so \( \frac{r}{1} \in \mathfrak b \).
Hence \( r \in \mathfrak b^c \), so \( \frac{r}{1} \in \mathfrak b^{ce} \), thus \( \frac{r}{s} \in \mathfrak b^{ce} \) as \( \mathfrak b^{ce} \) is an ideal in \( S^{-1}R \).
\begin{proposition}
    Consider the localisation map \( R \to S^{-1}R \) given by \( r \mapsto \frac{r}{1} \).
    \begin{enumerate}
        \item Every ideal of \( S^{-1}R \) is extended.
        \item An ideal \( \mathfrak a \) of \( R \) is contracted if and only if the image of \( S \) in \( \faktor{R}{\mathfrak a} \) contains no zero divisors.
        \item \( \mathfrak a^e = S^{-1}R \) if and only if \( \mathfrak a \cap S \neq \varnothing \).
        \item There is a bijection
        \[ \qty{\mathfrak p \in \Spec R \midd \mathfrak p \cap S = \varnothing} \leftrightarrow \Spec S^{-1}R \]
        given by \( \mathfrak p \mapsto \mathfrak p^e \), \( \mathfrak q^c \mapsfrom \mathfrak q \).
    \end{enumerate}
\end{proposition}
\begin{proof}
    \emph{Part (i).}
    Follows from the fact that \( \mathfrak b^{ce} = \mathfrak b \) for all ideals \( \mathfrak b \) in \( S^{-1}R \).

    \emph{Part (ii).}
    \( \mathfrak a \) is contracted if and only if \( \mathfrak a^{ec} \subseteq \mathfrak a \), because the reverse inclusion always holds.
    This happens if and only if
    \[ \bigcup_{s \in S} (\mathfrak a : s) \subseteq \mathfrak a \]
    which occurs if and only if
    \[ \forall r \in R,\, \qty(Sr \cap \mathfrak a \neq \varnothing \implies r \in \mathfrak a) \]
    \[ \forall r \in R,\, \qty(0 + \mathfrak a \in S(r + \mathfrak a) \implies r + \mathfrak a = 0 + \mathfrak a) \]
    which in turn occurs if and only if the image of \( S \) in \( \faktor{R}{\mathfrak a} \) contains no zero divisors.

    \emph{Part (iii).}
    Suppose \( \mathfrak a \cap S \neq \varnothing \), so let \( x \in \mathfrak a \cap S \).
    Then \( \frac{x}{x} \in \mathfrak a^e \), so \( \mathfrak a^e = (1) = S^{-1}R \).
    Conversely, if \( \mathfrak a^e = S^{-1}R \), then \( \frac{1}{1} \in \mathfrak a^e \), so \( \frac{1}{1} = \frac{a}{s} \) for some \( a \in \mathfrak a \), \( s \in S \).
    Therefore there exists \( u \in S \) such that \( us = ua \in S \cap \mathfrak a \).

    \emph{Part (iv).}
    Consider the contraction map \( \Spec S^{-1}R \to \qty{\mathfrak p \in \Spec R \mid \mathfrak p \cap S = \varnothing} \) given by \( \mathfrak q \mapsto \mathfrak q^c \).
    We show this is well-defined.
    In general, a contraction of a prime ideal is always prime.
    Further, \( \mathfrak p \in \Spec R \) is contracted if and only if the image of \( S \) in \( \faktor{R}{\mathfrak p} \) contains no zero divisors, but \( \faktor{R}{\mathfrak p} \) is an integral domain, so its only zero divisor is zero itself.
    So this condition is equivalent to the condition \( \mathfrak p \cap S = \varnothing \).
    In particular, \( \qty{\mathfrak p \in \Spec R \mid \mathfrak p \cap S = \varnothing} \) is precisely the set of contracted prime ideals of \( R \).
    The map is injective, since if \( \mathfrak q \in \Spec S^{-1}R \), then \( \mathfrak q^{ce} = \mathfrak q \).

    In the other direction, for \( \mathfrak p \in \Spec R \) such that \( \mathfrak p \cap S = \varnothing \), it must be contracted, so \( \mathfrak p^{ec} = \mathfrak p \).
    It therefore remains to show that \( \mathfrak p^e \) is a prime ideal.
    We want to show that \( \faktor{S^{-1}R}{\mathfrak p^e} \) is an integral domain.
    We have that \( \mathfrak p^e \neq S^{-1}R \) by (iii), so \( \faktor{S^{-1}R}{\mathfrak p^e} \) is not the zero ring, so it suffices to show that this quotient has no zero divisors.
    To show this, we embed \( \faktor{S^{-1}R}{\mathfrak p^e} \) in the field \( FF\qty(\faktor{R}{\mathfrak p}) \).

    Consider the composite map
    \[ R \to \faktor{R}{\mathfrak p} \to FF\qty(\faktor{R}{\mathfrak p}) \]
    which is a surjection followed by an injection.
    This has the property that all elements of \( S \) are mapped to units, because \( S \cap \mathfrak p = \varnothing \).
    By the universal property of the localisation, we have a map
    \[ \varphi : S^{-1}R \to FF\qty(\faktor{R}{\mathfrak p});\quad \frac{r}{s} \mapsto \frac{r + \mathfrak p}{s + \mathfrak p} \]
    It suffices to show that \( \ker \varphi = \mathfrak p^e \), then the result holds by the isomorphism theorem.
    Let \( \frac{r}{s} \in \ker \varphi \), so \( \frac{r + \mathfrak p}{s + \mathfrak p} = \frac{0}{1} \) in \( FF\qty(\faktor{R}{\mathfrak p}) \).
    Observe that \( \Im \varphi \subseteq \overline S^{-1} \qty(\faktor{R}{\mathfrak p}) \), where \( \overline S \) is the image of \( S \) in \( \faktor{R}{\mathfrak p} \).
    Restricting the range, we can consider \( \varphi \) as a map from \( S^{-1}R \) to \( \overline S^{-1}\qty(\faktor{R}{\mathfrak p}) \).
    So \( \varphi\qty(\frac{r}{s}) = \frac{0}{1} \) implies that there exists \( u + \mathfrak p \in \overline S \) such that \( (u + \mathfrak p)(r + \mathfrak p) = 0 \), so \( ur + \mathfrak p = 0 \).
    In particular, \( u \in S \) and \( ur \in \mathfrak p \).
    Hence \( \frac{r}{s} = \frac{ur}{us} \) where \( ur \in \mathfrak p \) and \( us \in S \), so \( \frac{r}{s} \in \mathfrak p^e \).

    For the other direction, take \( x \in \mathfrak p^e \), so \( x = \frac{p}{s} \) for \( p \in \mathfrak p, s \in S \).
    Then \( \varphi(x) = \frac{p + \mathfrak p}{s + \mathfrak p} = 0 \), so \( x \in \ker \varphi \).
\end{proof}
It is not true in general that the extensions of prime ideals are prime.
\begin{definition}
    If \( I \) is an ideal in \( R \), the \emph{radical} of \( I \) is the ideal
    \[ \sqrt{I} = \qty{r \in R \mid \exists n \geq 1,\, r^n \in I} \]
\end{definition}
\begin{proposition}
    Let \( I \) be an ideal in a ring \( R \).
    Then
    \[ \sqrt{I} = \bigcap_{I \subseteq \mathfrak p \in \Spec R} \mathfrak p \]
\end{proposition}
\begin{proof}
    Let \( x \in \sqrt{I} \).
    Then \( x^n \in I \) for some \( n \geq 1 \).
    For every \( \mathfrak p \in \Spec R \), if \( I \subseteq \mathfrak p \), then \( x^n \in \mathfrak p \), so \( x \in \mathfrak p \).
    Conversely, suppose \( x^n \notin I \) for all \( n \geq 1 \).
    As \( I \neq R \), we have \( \faktor{R}{I} \neq 0 \).
    Let \( \overline x \) be the image of \( x \) in \( \faktor{R}{I} \), and consider
    \[ \qty(\faktor{R}{I})_{\overline x} = \qty{\overline x^n \mid n \geq 1}^{-1} \qty(\faktor{R}{I}) \]
    This is not the zero ring, because \( x^n \notin I \) for all \( n \geq 1 \).
    Therefore, \( \qty(\faktor{R}{I})_{\overline x} \) has a prime ideal, as it contains a maximal ideal.
    By the bijection described in part (iv) of the previous result, this prime ideal corresponds to a prime ideal of \( \faktor{R}{I} \) that avoids \( \overline x \).
    This in turn corresponds to a prime ideal \( \mathfrak p \in \Spec R \) that contains \( I \) and avoids \( x \).
    Hence \( x \notin \bigcap_{I \subseteq \mathfrak p \in \Spec R} \mathfrak p \). 
\end{proof}
