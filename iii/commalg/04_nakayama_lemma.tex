\begin{proposition}[Cayley--Hamilton theorem]
    Let \( M \) be a finitely generated \( R \)-module, and let \( f : M \to M \) be an \( R \)-linear endomorphism.
    Let \( \mathfrak a \) be an ideal in \( R \) such that \( f(M) \subseteq \mathfrak a M \).
    Then, we have an equality in \( \End_R M \)
    \[ f^n + a_1 f^{n-1} + \dots + a_n f^0 = 0;\quad f^r = \underbrace{f \circ \dots \circ f}_{r \text{ times}} \]
    where \( a_i \in \mathfrak a \).
\end{proposition}
\begin{proof}
    Let \( M = \vecspan_R\qty{m_1, \dots, m_n} \), so \( \mathfrak a M = \vecspan_{\mathfrak a} \qty{m_1, \dots, m_n} \).
    Then
    \[ \begin{pmatrix}
        f(m_1) \\
        \vdots \\
        f(m_n)
    \end{pmatrix} = P \begin{pmatrix}
        m_1 \\
        \vdots \\
        m_n
    \end{pmatrix};\quad P \in M_{n \times n}(\mathfrak a) \]
    Let \( \rho : R \to \End M \) be the structure ring homomorphism of \( M \) as an \( R \)-module.
    Then we can define \( R[T] \to \End M \) by \( T \mapsto f \), making \( M \) into an \( R[T] \)-module.
    Hence,
    \[ T \begin{pmatrix}
        m_1 \\
        \vdots \\
        m_n
    \end{pmatrix} = P \begin{pmatrix}
        m_1 \\
        \vdots \\
        m_n
    \end{pmatrix} \]
    Thus
    \[ Q \begin{pmatrix}
        m_1 \\
        \vdots \\
        m_n
    \end{pmatrix} = 0;\quad Q = T I_n - P \]
    Multiplying by the adjugate matrix \( \adj Q \) on the left on both sides,
    \[ (\det Q) \begin{pmatrix}
        m_1 \\
        \vdots \\
        m_n
    \end{pmatrix} = 0 \]
    In particular, \( (\det Q) m = 0 \) for all \( m \in M \), as the \( m_i \) generate \( M \).
    Hence, \( m \mapsto (\det Q) m = \eval{(\det Q)}_{T = f} \) is \( 0 \) in \( \End_R M \).
    Finally, note that \( \det Q \) is a monic polynomial, and all other coefficients lie in \( \mathfrak a \).
\end{proof}
\begin{corollary}
    Let \( M \) be a finitely generated \( R \)-module, and let \( \mathfrak a \) be an ideal in \( R \).
    If \( \mathfrak a M = M \), then there exists \( a \in \mathfrak a \) such that \( am = m \) for all \( m \in M \).
\end{corollary}
\begin{proof}
    Apply the Cayley--Hamilton theorem with \( f = \id_M \).
    We obtain a polynomial
    \[ (1 + a_1 + \dots + a_n) \id_M = 0 \]
    Take \( a = -(a_1 + \dots + a_n) \).
\end{proof}
\begin{definition}
    The \emph{Jacobson radical} of a ring \( R \), denoted \( J(R) \), is the intersection of all maximal ideals of \( R \).
\end{definition}
\begin{example}
    \begin{enumerate}
        \item If \( (R, \mathfrak m) \) is a local ring, then \( J(R) = \mathfrak m \).
        \item \( J(\mathbb Z) = \qty{0} \).
    \end{enumerate}
\end{example}
\begin{proposition}
    Let \( x \in R \).
    Then \( x \in J(R) \) if and only if \( 1 - xy \) is a unit for every \( y \in R \).
\end{proposition}
\begin{proof}
    First, let \( x \in J(R) \), and suppose \( y \in R \) is such that \( 1 - xy \) is not a unit.
    Then \( (1 - xy) \) is a proper ideal, so it is contained in a maximal ideal \( \mathfrak m \).
    But as \( x \in J(R) \), we must have \( x \in \mathfrak m \), giving \( 1 = 1 - xy + xy \in \mathfrak m \), contradicting that \( \mathfrak m \) is a maximal ideal.

    Now suppose \( x \notin J(R) \), so there is a maximal ideal \( \mathfrak m \) such that \( x \notin \mathfrak m \).
    Then \( \mathfrak m + (x) = R \) as \( \mathfrak m \) is maximal.
    In particular, there exists \( t \in \mathfrak m \) and \( y \in R \) such that \( t + xy = 1 \), or equivalently, \( 1 - xy = t \in \mathfrak m \).
    Note that \( t \) cannot be a unit, because it is contained in a proper ideal.
\end{proof}
\begin{proposition}[Nakayama's lemma]
    Let \( M \) be a finitely generated \( R \)-module, and let \( \mathfrak a \subseteq J(R) \) be an ideal of \( R \) such that \( \mathfrak a M = M \).
    Then \( M = 0 \).
\end{proposition}
This lemma is more useful when \( J(R) \) is large, so is particularly useful when applied to local rings.
\begin{proof}
    By the above corollary, there exists \( a \in \mathfrak a \) such that \( am = m \) for all \( m \in M \), or equivalently, \( (1 - a)m = 0 \).
    By assumption, \( a \in J(R) \), so \( 1 - a \) is a unit in \( R \).
    Hence \( m = 0 \).
\end{proof}
\begin{corollary}
    Let \( M \) be a finitely generated \( R \)-module, and let \( N \subseteq M \) be a submodule.
    Let \( \mathfrak a \subseteq J(R) \) be an ideal in \( R \) such that \( N + \mathfrak a M = M \).
    Then \( N = M \).
\end{corollary}
This can be applied to find generating sets for \( M \).
\begin{proof}
    Note that
    \[ \mathfrak a \qty(\faktor{M}{N}) = \faktor{\mathfrak a M + N}{N} = \faktor{M}{N} \]
    so \( \faktor{M}{N} = 0 \) by Nakayama's lemma.
\end{proof}
