\chapter[Probability and Measure \\ \textnormal{\emph{Lectured in Michaelmas \oldstylenums{2022} by \textsc{Prof.\ R.\ Nickl}}}]{Probability and Measure}
\emph{\Large Lectured in Michaelmas \oldstylenums{2022} by \textsc{Prof.\ R.\ Nickl}}

In this course, we study measure theory and integration, and its applications to probability theory.
We begin by defining the notion of a measure, which extends the notion of the length of an interval to a much larger class of `measurable' sets.
In the context of a probability space, a probability measure is a way to associate probabilities to events that could occur.
Measures have the countable additivity property, which allows us to compute the measure of certain limits of measurable sets.
Using this property, we can analyse limiting behaviour by considering the measure of a set on which a certain event occurs.

Measure theory allows us to define the Lebesgue integral.
This integral agrees with the Riemann integral on most well-behaved functions, but it has many more convenient properties concerning limits.
For example, the dominated convergence theorem gives a sufficient condition for when the limit of the integrals of functions is the integral of the limit.
Another example is that the set of Lebegsue integrable functions forms a complete normed vector space, but this is not true of the Riemann integral.

Using the Lebesgue integral, we can define the Fourier transform of an integrable function.
This linear operator is `almost' injective: if the Fourier transform of a function is also integrable, we can recover the original function almost everywhere.
Properties of the Fourier transform are used to deduce the central limit theorem.

\subfile{../../ii/pm/main.tex}
