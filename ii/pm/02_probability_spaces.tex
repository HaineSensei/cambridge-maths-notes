\subsection{Definitions}
\begin{definition}
	If a measure space \( (E, \mathcal E, \mu) \) has \( \mu(E) = 1 \), we call it a \emph{probability space}, and instead write \( (\Omega, \mathcal F, \mathbb P) \).
	We call \( \Omega \) the outcome space or sample space, \( \mathcal F \) the set of events, and \( \mathbb P \) the probability measure.
\end{definition}
The axioms of probability theory (Kolmogorov, 1933), are
\begin{enumerate}
	\item \( \prob{\Omega} = 1 \);
	\item \( 0 \leq \prob{E} \leq 1 \) for all \( E \in \mathcal F \);
	\item if \( A_n \) are a disjoint sequence of events in \( \mathcal F \), then \( \prob{\bigcup_n A_n} = \sum_n \prob{A_n} \).
\end{enumerate}
This is exactly what is required by our definition: \( \mathbb P \) is a measure on a \( \sigma \)-algebra.
\begin{definition}
	Events \( A_i, i \in I \) are \emph{independent} if for all finite \( J \subseteq I \), we have
	\[ \prob{\bigcap_{j \in J} A_j} = \prod_{j \in J} \prob{A_j} \]
	\( \sigma \)-algebras \( \mathcal A_i, I \in I \) are independent if for any \( A_j \in \mathcal A_j \) where \( J \subseteq I \) is finite, the \( A_j \) are independent.
\end{definition}
Kolmogorov showed that these definitions are sufficient to derive the law of large numbers.
\begin{proposition}
	Let \( \mathcal A_1, \mathcal A_2 \) be \( \pi \)-systems of sets in \( \mathcal F \).
	Suppose \( \prob{A_1 \cap A_2} = \prob{A_1} \prob{A_2} \) for all \( A_1 \in \mathcal A_1, A_2 \in \mathcal A_2 \).
	Then the \( \sigma \)-algebras \( \sigma(\mathcal A_1), \sigma(\mathcal A_2) \) are independent.
\end{proposition}
This follows by uniqueness.

\subsection{Borel--Cantelli lemmas}
\begin{definition}
	Let \( A_n \in \mathcal F \) be a sequence of events.
	Then the \emph{limit superior} of \( A_n \) is
	\[ \limsup_n A_n = \bigcap_n \bigcup_{m \geq n} A_m = \qty{A_n \text{ infinitely often}} \]
	The \emph{limit inferior} of \( A_n \) is
	\[ \liminf_n A_n = \bigcup_n \bigcap_{m \geq n} A_m = \qty{A_n \text{ eventually}} \]
\end{definition}
\begin{lemma}[First Borel--Cantelli lemma]
	Let \( A_n \in \mathcal F \) be a sequence of events such that \( \sum_n \prob{A_n} < \infty \).
	Then \( \prob{A_n \text{ infinitely often}} = 0 \).
\end{lemma}
\begin{proof}
	For all \( n \), we have
	\[ \prob{\limsup_n A_n} = \prob{\bigcap_n \bigcup_{m \geq n} A_m} \leq \prob{\bigcup_{m \geq n} A_m} \leq \sum_{m \geq n} \prob{A_m} \to 0 \]
\end{proof}
This proof did not require that \( \mathbb P \) be a probability measure, just that it is a measure.
Therefore, we can use this for arbitrary measures.
\begin{lemma}[Second Borel--Cantelli lemma]
	Let \( A_n \in \mathcal F \) be a sequence of independent events, and \( \sum_n \prob{A_n} = \infty \).
	Then \( \prob{A_n \text{ infinitely often}} = 1 \).
\end{lemma}
\begin{proof}
	By independence, for all \( N \geq n \in \mathbb N \) and using \( 1 - a \leq e^{-a} \), we find
	\[ \prob{\bigcap_{m=n}^N A_m^c} = \prod_{m=n}^N \qty(1 - \prob{A_m}) \leq \prod_{m=n}^N e^{-\prob{A_m}} = e^{-\sum_{m=n}^N \prob{A_m}} \]
	As \( N \to \infty \), this approaches zero.
	Since \( \bigcap_{m=n}^N A_m^c \) decreases to \( \bigcap_{m=n}^\infty A_m^c \), by countable additivity we must have \( \prob{\bigcap_{m=n}^\infty A_m^c} = 0 \).
	But then
	\[ \prob{A_n \text{ infinitely often}} = \prob{\bigcap_n \bigcup_{m \geq n} A_m} = 1 - \prob{\bigcup_n \bigcap_{m \geq n} A_m^c} \geq 1 - \sum_n \prob{\bigcap_{m \geq n} A_m^c} = 1 \]
	Hence this probability is equal to one.
\end{proof}
