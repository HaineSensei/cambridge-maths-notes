\subsection{Integration in product spaces}
Let \( (E_1, \mathcal E_1, \mu_1), (E_2, \mathcal E_2, \mu_2) \) be finite measure spaces.
On \( E = E_1 \times E_2 \), we can consider the \( \pi \)-system of `rectangles' \( \mathcal A = \qty{A_1 \times A_2 \mid A_1 \in \mathcal E_1, A_2 \in \mathcal E_2} \).
Then we define the \( \sigma \)-algebra \( \mathcal E_1 \otimes \mathcal E_2 = \sigma(\mathcal A) \) on the product space.
If the \( E_i \) are topological spaces with a countable base, then \( \mathcal B(E_1 \times E_2) = \mathcal B(E_1) \otimes \mathcal B(E_2) \).
\begin{lemma}
	Let \( E = E_1 \times E_2, \mathcal E = \mathcal E_1 \otimes \mathcal E_2 \).
	Let \( f \colon (E, \mathcal E) \to \mathbb R \) be measurable.
	Then for all \( x_1 \in E_1 \), the map \( (x_2 \mapsto f(x_1, x_2)) \colon (E_2, \mathcal E_2) \to \mathbb R \) is \( \mathcal E_2 \)-measurable.
\end{lemma}
\begin{proof}
	Let
	\[ \mathcal V = \qty{f \colon (E,\mathcal E) \to \mathbb R \mid f \text{ bounded, measurable, conclusion of the lemma holds}} \]
	This is a \( \mathbb R \)-vector space, and it contains \( \mathbbm 1_E, \mathbbm 1_A \) for all \( A \in \mathcal A \), since \( \mathbbm 1_A = \mathbbm 1_{A_1(x_1)} \mathbbm 1_{A_2(x_2)} \).
	Now, let \( 0 \leq f_n \) increase to \( f \), \( f_n \in \mathcal V \).
	Then \( (x_2 \mapsto f(x_1, x_2)) = \lim_n (x_2 \mapsto f_n(x_1, x_2)) \), so it is \( \mathcal E_2 \)-measurable as a limit of a sequence of measurable functions.
	Then by the monotone class theorem, \( \mathcal V \) contains all bounded measurable functions.
	This extends to all measurable functions by truncating the absolute value of \( f \) to \( n \in \mathbb N \), then the sequence of such bounded truncations converges pointwise to \( f \).
\end{proof}
\begin{lemma}
	Let \( E = E_1 \times E_2, \mathcal E = \mathcal E_1 \otimes \mathcal E_2 \).
	Let \( f \colon (E, \mathcal E) \to \mathbb R \) be measurable such that
	\begin{enumerate}
		\item \( f \) is bounded; or
		\item \( f \) is nonnegative.
	\end{enumerate}
	Then the map \( x_1 \mapsto \int_{E_2} f(x_1,x_2) \dd{\mu_2(x_2)} \) is \( \mu_1 \)-measurable and is bounded or nonnegative respectively.
\end{lemma}
\begin{remark}
	In case (ii), the map on \( x_1 \) may evaluate to infinity, but the set of values
	\[ \qty{x_1 \in E_1 \midd \int_{E_2} f(x_1,x_2) \dd{\mu_2(x_2)} = \infty} \]
	lies in \( \mathcal E_1 \).
\end{remark}
\begin{proof}
	Let
	\[ \mathcal V = \qty{f \colon (E,\mathcal E) \to \mathbb R \mid f \text{ bounded, measurable, conclusion of the lemma holds}} \]
	This is a vector space by linearity of the integral.
	\( \mathbbm 1_E \in \mathcal V \), since \( \mathbbm 1_{E_1} \mu_2(E_2) \) is nonnegative and bounded.
	\( \mathbbm 1_A \in \mathcal V \) for all \( A \in \mathcal A \), because \( \mathbbm 1_{A_1}(x_1) \mu_2(A_2) \) is \( \mathcal E_1 \)-measurable, nonnegative, and bounded since it is at most \( \mu_2(E_2) < \infty \).
	Now let \( f_n \) be a sequence of nonnegative functions that increase to \( f \), where \( f_n \in \mathcal V \).
	Then by the monotone convergence theorem,
	\[ \int_{E_2} \lim_{n \to \infty} f_n(x_1, x_2) \dd{\mu_2(x_2)} = \lim_{n \to \infty} \int_{E_2} f_n(x_1, x_2) \dd{\mu_2(x_2)} \]
	is an increasing limit of \( \mathcal E_1 \)-measurable functions, so is \( \mathcal E_1 \)-measurable.
	It is bounded by \( \mu_2(E_2) \norm{f}_\infty \), or nonnegative as required.
	So \( f \in \mathcal V \).
	By the monotone class theorem, the result for bounded functions holds.
	In case (ii), we can take a bounded approximation in \( \mathcal V \) of an arbitrary measurable function \( f \) to conclude the proof.
\end{proof}
\begin{theorem}[product measure]
	Let \( (E_1, \mathcal E_1, \mu_1), (E_2, \mathcal E_2, \mu_2) \) be finite measure spaces.
	There exists a unique measure \( \mu = \mu_1 \otimes \mu_2 \) on \( (E_1 \times E_2, \mathcal E_1 \otimes \mathcal E_2) \) such that \( \mu(A_1 \times A_2) = \mu_1(A_1) \mu_2(A_2) \) for all \( A_1 \in \mathcal E_1 \), \( A_2 \in \mathcal E_2 \).
\end{theorem}
\begin{proof}
	\( \mathcal A \) generates \( \mathcal E \otimes \mathcal E_2 \), so by the uniqueness theorem, there can only be one such measure.
	We define
	\[ \mu(A) = \int_{E_1} \qty( \int_{E_2} \mathbbm 1_A(x_1,x_2) \dd{\mu_2(x_2)} ) \dd{\mu_1(x_1)} \]
	We have
	\[ \mu(A_1 \times A_2) = \int_{E_1} \qty( \int_{E_2} \mathbbm 1_{A_1}(x_1) \mathbbm 1_{A_2}(x_2) \dd{\mu_2(x_2)} ) \dd{\mu_1(x_1)} = \int_{E_1} \mathbbm 1_{A_1}(x_1) \mu_2(A_2) \dd{\mu_1(x_1)} = \mu_1(A_1) \mu_2(A_2) \]
	Clearly \( \mu(\varnothing) = 0 \), so it suffices to show countable additivity.
	Let \( A_n \) be disjoint sets in \( \mathcal E_1 \otimes \mathcal E_2 \).
	Then
	\[ \mathbbm 1_{\qty(\bigcup_n A_n)} = \sum_n \mathbbm 1_{A_n} = \lim_{n \to \infty} \sum_{i=1}^n \mathbbm 1_{A_n} \]
	Then by the monotone convergence theorem and the previous lemmas,
	\begin{align*}
		\mu\qty(\bigcup_n A_n) &= \int_{E_1} \qty( \int_{E_2} \lim_{n \to \infty} \sum_{i=1}^n \mathbbm 1_{A_i} \dd{\mu_2(x_2)} ) \dd{\mu_1(x_1)} \\
		&= \int_{E_1} \qty( \lim_{n \to \infty} \int_{E_2} \sum_{i=1}^n \mathbbm 1_{A_i} \dd{\mu_2(x_2)} ) \dd{\mu_1(x_1)} \\
		&= \lim_{n \to \infty} \int_{E_1} \qty( \int_{E_2} \sum_{i=1}^n \mathbbm 1_{A_i} \dd{\mu_2(x_2)} ) \dd{\mu_1(x_1)} \\
		&= \lim_{n \to \infty} \sum_{i=1}^n \int_{E_1} \qty( \int_{E_2} \mathbbm 1_{A_i} \dd{\mu_2(x_2)} ) \dd{\mu_1(x_1)} \\
		&= \lim_{n \to \infty} \sum_{i=1}^n \mu(A_i) \\
		&= \sum_{n=1}^\infty \mu(A_n)
	\end{align*}
\end{proof}

\subsection{Fubini's theorem}
\begin{theorem}
	Let \( (E, \mathcal E, \mu) = (E_1 \times E_2, \mathcal E_1 \otimes \mathcal E_2, \mu_1 \otimes \mu_2) \) be a finite measure space.
	Let \( f \colon E \to \mathbb R \) be a nonnegative measurable function.
	Then
	\begin{align*}
		\mu(f) &= \int_E f \dd{\mu} \\
		&= \int_{E_1} \qty( \int_{E_2} f(x_1,x_2) \dd{\mu_2(x_2)} ) \dd{\mu_1(x_1)} \\
		&= \int_{E_2} \qty( \int_{E_1} f(x_1,x_2) \dd{\mu_1(x_1)} ) \dd{\mu_2(x_2)}
	\end{align*}
	Now, let \( f \colon E \to \mathbb R \) be a \( \mu \)-integrable function (on the product measure).
	Let \( A_1 = \qty{x_1 \in E_1 \midd \int_{E_2} \abs{f(x_1,x_2)} \dd{\mu_2(x_2)} < \infty} \).
	Define \( f_1 \) by \( f_1(x_1) = \int_{E_2} f(x_1,x_2) \dd{\mu_2(x_2)} \) on \( A_1 \) and zero elsewhere.
	Then \( \mu_1(A_1^c) = 0 \) and \( \mu(f) = \mu_1(f_1) = \mu_1(f_1 \mathbbm 1_{A_1}) \), and defining \( A_2 \) symmetrically, \( \mu(f) = \mu_2(f_2) = \mu_2(f_2 \mathbbm 1_{A_2}) \).
\end{theorem}
\begin{remark}
	If \( f \) is bounded, \( A_1 = E_1 \).
	Note, for \( f(x_1,x_2) = \frac{x_1^2-x_2^2}{(x_1^2+x_2^2)^2} \) on \( (0,1)^2 \), we have \( \mu_1(f_1) \neq \mu_2(f_2) \), but \( f \) is not Lebesgue integrable on \( (0,1)^2 \).
\end{remark}
\begin{proof}
	By the construction of the product measure \( \mu(A) \) for rectangles \( A = A_1 \times A_2 \) in the \( \pi \)-system \( \mathcal A \) generating \( \mathcal E \), the identities in the first part of the theorem clearly hold for \( f = \mathbbm 1_A \).
	By uniqueness, this extends to \( \mathbbm 1_A \) for all \( A \in \mathcal E \).
	Then, by linearity of the integral, this extends to simple functions.
	By the monotone convergence theorem, the first part of the theorem follows.

	Now let \( f \) be \( \mu \)-integrable.
	Let \( h(x_1) = \int_{E_2} \abs{f(x_1,x_2)} \dd{\mu_2(x_2)} \).
	Then by the first part, \( \mu_1(\abs{h}) \leq \mu(\abs{f}) < \infty \).
	So \( f_1 \) is \( \mu_1 \)-integrable.
	We have \( \mu_1(A_1^c) = 0 \), otherwise, we could compute a lower bound \( \mu_1(\abs{h}) \geq \mu_1(\abs{h} \mathbbm 1_{A_1^c}) = \infty \), but it must be finite.
	Note that \( f_1^\pm = \int_{E_2} f^\pm(x_1,x_2) \dd{\mu_2(x_2)} \), and \( \mu(f_1) = \mu_1(f_1^+) - \mu_1(f_1^-) \).
	Hence, by the first part, \( \mu(f) = \mu(f^+) - \mu(f^-) = \mu_1(f_1^+) - \mu_1(f_1^-) = \mu_1(f_1) \) as required.
\end{proof}
\begin{remark}
	The proofs above extend to \( \sigma \)-finite measures \( \mu \).

	Let \( (E_i, \mathcal E_i, \mu_i) \) be measure spaces with \( \sigma \)-finite measures.
	Note that \( (\mathcal E_1 \otimes \mathcal E_2) \otimes \mathcal E_3 = \mathcal E_1 \otimes (\mathcal E_2 \otimes \mathcal E_3) \), by a \( \pi \)-system argument using Dynkin's lemma.
	So we can iterate the construction of the product measure to obtain a measure \( \mu_1 \otimes \dots \mu_n \), which is a unique measure on \( \qty(\prod_{i=1}^n E_i \bigotimes_{i=1}^n \mathcal E_i) \) with the property that the measure of a hypercube \( \mu(A_1 \times A_n) \) is the product of the measures of its sides \( \mu_i(A_i) \).

	In particular, we have constructed the Lebesgue measure \( \mu^n = \bigotimes_{i=1}^n \mu \) on \( \mathbb R^n \).
	Applying Fubini's theorem, for functions \( f \) that are either nonnegative and measurable or \( \mu^n \)-integrable, we have
	\[ \int_{\mathbb R^n} f \dd{\mu^n} = \idotsint_{\mathbb R \dots \mathbb R} f(x_1, \dots, x_n) \dd{\mu(x_1)} \dots \dd{\mu(x_n)} \]
\end{remark}

\subsection{Product probability spaces and independence}
\begin{proposition}
	Let \( (\Omega, \mathcal F, \mathbb P) \), and \( (E, \mathcal E) = \qty(\prod_{i=1}^n E_i, \bigotimes_{i=1}^n \mathcal E_i) \).
	Let \( X \colon (\Omega, \mathcal F) \to (E, \mathcal E) \) be a measurable function, and define \( X(\omega) = (X_1(\omega), X_2(\omega), \dots, X_n(\omega)) \).
	Then the following are equivalent.
	\begin{enumerate}
		\item \( X_1, \dots, X_n \) are independent random variables;
		\item \( \mu_X = \bigotimes_{i=1}^n \mu_{X_i} \);
		\item for all bounded and measurable \( f_i \colon E_i \to \mathbb R \), \( \expect{\prod_{i=1}^n f_i(X_i)} = \prod_{i=1}^n \expect{f_i(X_i)} \).
	\end{enumerate}
\end{proposition}
\begin{proof}
	\emph{(i) implies (ii).}
	Consider the \( \pi \)-system \( \mathcal A \) of rectangles \( A = \prod_{i=1}^n A_i \) for \( A_i \in \mathcal E_i \).
	Since \( \mu_X \) is an image measure,
	Then
	\[ \mu_X(A_1 \times \dots \times A_n) = \prob{X_1 \in A_1, \dots, X_n \in A_n} = \prob{X_1} \dots \prob{A_n} = \prod_{i=1}^n \mu_{X_i}(A_i) \]
	So by uniqueness, the result follows.

	\emph{(ii) implies (iii).}
	By Fubini's theorem,
	\begin{align*}
		\expect{\prod_{i=1}^n f(X_i)} &= \mu_X\qty(\prod_{i=1}^n f(X_i)) \\
		&= \int_E f(x) \dd{\mu(x)} \\
		&= \idotsint_{E_i} \qty(\prod_{i=1}^n f_i(x_i)) \dd{\mu_{X_1}(x_1)} \dots \dd{\mu_{X_2}(x_2)} \\
		&= \prod_{i=1}^n \int_{E_i} f_i(x_i) \dd{\mu_{X_i}(x_i)} \\
		&= \prod_{i=1}^n \expect{f(X_i)}
	\end{align*}

	\emph{(iii) implies (i).}
	Let \( f_i = \mathbbm 1_{A_i} \) for any \( A_i \in \mathcal E_i \).
	These are bounded and measurable functions.
	Then
	\[ \prob{X_1 \in A_1, \dots, X_n \in A_n} = \expect{\prod_{i=1}^n \mathbbm 1_{A_i}(X_i)} = \prod_{i=1}^n \expect{\mathbbm 1_{A_i}(X_i)} = \prod_{i=1}^n \prob{X_i \in A_i} \]
	So the \( \sigma \)-algebras generated by the \( X_i \) are independent as required.
\end{proof}
