\subsection{Laws of large numbers}
\begin{proposition}
	Let \( (X_n)_{n \in \mathbb N} \) be independent and identically distributed random variables such that \( \expect{X_n} = 0 \) and \( \Var{X_n} = \sigma^2 < \infty \).
	Then \( \frac{1}{n} \sum_{i=1}^n X_i \to 0 \) in probability as \( n \to \infty \).
\end{proposition}
\begin{proof}
	By Chebyshev's inequality,
	\[ \prob{\abs{\frac{1}{n}\sum_{i=1}^n X_i} > \varepsilon} \leq \frac{1}{n^2 \varepsilon^2} \Var{\sum_{i=1}^n X_i} \leq \frac{\sigma^2}{n\varepsilon^2} \to 0 \]
	So \( \frac{1}{n} \sum_{i=1}^n X_i \to \expect{X_1} \) in probability.
\end{proof}
This is known as the weak law of large numbers.
However, this result has several weaknesses, and we can provide stronger results.
\begin{proposition}
	Let \( (X_n)_{n \in \mathbb N} \) be independent random variables such that \( \expect{X_n} = \mu \) and \( \expect{X_n^4} \leq M \) for all \( n \).
	Then \( \frac{1}{n} \sum_{i=1}^n X_i \to \mu \) almost surely as \( n \to \infty \).
\end{proposition}
\begin{proof}
	Let \( Y_n = X_n - \mu \).
	Then \( \expect{Y_n} = 0 \), and \( \expect{Y_n^4} \leq 2^4 \qty(\expect{X_n^4} + \mu^4) < \infty \).
	So we can assume \( \mu = 0 \).
	For distinct indices \( i, j, k, \ell \), by independence and the Cauchy--Schwarz inequality, we have
	\[ 0 = \expect{X_i X_j X_k X_\ell} = \expect{X_i^2 X_j X_j} = \expect{X_i^3 X_j};\quad \expect{X_i^2 X_j^2} \leq \sqrt{\expect{X_i^4}}\sqrt{\expect{X_j^4}} \leq M \]
	So we can compute
	\[ \expect{\qty(\sum_{i=1}^n X_i)^4} = \expect{\sum_{i=1}^n X_i^4} + 6\expect{\sum_{i < j} X_i^2 X_j^2} \leq nM + 3n(n-1)M \leq 3n^2 M \]
	Let \( S_n = \sum_{i=1}^n X_i \).
	Then,
	\[ \expect{\sum_{n=1}^\infty \qty(\frac{S_n}{n})^4} \leq \sum_{n=1}^\infty \frac{1}{n^4} 3n^2 M < \infty \]
	Hence \( \sum_{n=1}^\infty \qty(\frac{S_n}{n})^4 < \infty \) almost surely.
	But then \( \qty(\frac{S_n}{n})^4 \to 0 \) almost surely, so \( \frac{S_n}{n} \to 0 \) almost surely.
\end{proof}

\subsection{Invariants}
Let \( (E, \mathcal E, \mu) \) be a \( \sigma \)-finite measure space.
\begin{definition}
	A measurable transformation \( \Theta \colon E \to E \) is \emph{measure-preserving} if \( \mu(\Theta^{-1}(A)) = \mu(A) \) for all \( A \in \mathcal E \).
\end{definition}
In this case, for any integrable function \( f \in L^1(\mu) \), we have \( \int_E f \dd{\mu} = \int_E f \circ \Theta \dd{\mu} \).
\begin{definition}
	A measurable map \( f \colon E \to \mathbb R \) is called \emph{\( \Theta \)-invariant} if \( f \circ \Theta = f \).
	A set \( A \in \mathcal E \) is \( \Theta \)-invariant if \( \Theta^{-1}(A) = A \), or equivalently, \( \mathbbm 1_A \) is \( \Theta \)-invariant.
\end{definition}
The collection \( \mathcal E_\Theta \) of \( \Theta \)-invariant sets forms a \( \sigma \)-algebra over \( E \).
A function \( f \colon E \to \mathbb R \) is invariant if and only if \( f \) is \( \mathcal E_\Theta \)-measurable; this is a question on an example sheet.
\begin{definition}
	\( \Theta \) is called \emph{ergodic} if the \( \Theta \)-invariant sets \( A \) satisfy either \( \mu(A) = 0 \) or \( \mu(E \setminus A) = 0 \).
\end{definition}
If \( f \) is \( \Theta \)-invariant and \( \Theta \) is ergodic, then one can show that \( f \) is constant almost everywhere on \( E \).
\begin{example}
	Consider \( (E, \mathcal E) = ((0,1], \mathcal B) \) with the Lebesgue measure \( \mu \).
	The maps \( \Theta_a(x) = x + a \) modulo 1 and \( \Theta(x) = 2x \) modulo 1 are both measure-preserving, and ergodic unless \( a \in \mathbb Q \).
	This is a question on an example sheet.
\end{example}
\begin{lemma}[maximal ergodic lemma]
    Let \( (E, \mathcal E, \mu) \) be a \( \sigma \)-finite measure space.
	Let \( \Theta \colon E \to E \) be measure-preserving.
	For \( f \in L^1(\mu) \), we define \( S_0(f) = 0 \) and \( S_n(f) = \sum_{k=0}^{n-1} f \circ \Theta^k \).
    Let \( S^\star = S^\star(f) = \sup_{n \geq 0} S_n(f) \).
    Then \( \int_{\qty{S^\star > 0}} f \dd{\mu} \geq 0 \).
\end{lemma}
\begin{proof}
    Define \( S_n^\star = \max_{k \leq n} S_k \).
    Then clearly \( S_n^\star \uparrow S^\star \), and \( S_k \leq S_n^\star \) for all \( k \leq n \).
    Note that \( S_{k+1} = S_k \circ \Theta + f \leq S_n^\star \circ \Theta + f \).

    Define \( A_n = \qty{S_n^\star > 0} \), so \( A_n \uparrow \qty{S^\star > 0} \).
    On \( A_n \), we have
    \[ S_n^\star = \max_{1 \leq k \leq n} S_k \leq \max_{0 \leq k \leq n} S_{k+1} \leq S_n^\star \circ \Theta + f \]
    since \( S_0 = 0 \).
    We can integrate this inequality to find
    \[ \int_{A_n} S_n^\star \dd{\mu} \leq \int_{A_n} S_n^\star \circ \Theta \dd{\mu} + \int_{A_n} f \dd{\mu} \]
    On the complement \( A_n^c \), we must have \( S_n^\star = 0 \leq S_n^\star \circ \Theta \).
    Hence,
    \[ \int_E S_n^\star \dd{\mu} \leq \int_E S_n^\star \circ \Theta \dd{\mu} + \int_{A_n} f \dd{\mu} \]
    Since \( \Theta \) is measure-preserving,
    \[ \int_E S_n^\star \dd{\mu} \leq \int_E S_n^\star \dd{\mu} + \int_{A_n} f \dd{\mu} \]
    so we obtain
    \[ \int_{A_n} f \dd{\mu} \geq 0 \]
    Since \( f \mathbbm 1_{A_n} \to f \mathbbm 1_{\qty{S^\star > 0}} \) pointwise, and \( \abs{f \mathbbm 1_{A_n}} \leq \abs{f} \in L^1(\mu) \), we can apply the dominated convergence theorem to show that
    \[ \int_{\qty{S^\star > 0}} f \dd{\mu} = \lim_{n \to \infty} \int_{A_n} f \dd{\mu} \geq 0 \]
    as required.
\end{proof}

\subsection{Ergodic theorems}
\begin{theorem}[Birkhoff]
	Let \( (E, \mathcal E, \mu) \) be a \( \sigma \)-finite measure space.
	Let \( \Theta \colon E \to E \) be measure-preserving.
	For \( f \in L^1(\mu) \), we define \( S_0(f) = 0 \) and \( S_n(f) = \sum_{k=0}^{n-1} f \circ \Theta^k \).
	Then there exists a \( \Theta \)-invariant integrable function \( \overline f \in L^1(\mu) \) with \( \mu\qty(\abs{\overline f}) \leq \mu(\abs{f}) \) such that \( \frac{S_n(f)}{n} \to \overline f \) almost everywhere.
\end{theorem}
The proof of Birkhoff's ergodic theorem is non-examinable.
\begin{proof}[Proof (non-examinable)]
    Note that
    \[ \limsup_n \frac{S_n(f)}{n} =  \limsup_n \frac{S_n(f) \circ \Theta}{n} \]
    and the same holds for \( \liminf_n \).
    Hence \( \limsup_n \frac{S_n(f)}{n} \) and \( \liminf_n \frac{S_n(f)}{n} \) are invariant functions.
    So they are \( \mathcal E_\Theta \)-measurable.
    Hence
    \[ D = D_{a,b} = \qty{\liminf_n \frac{S_n(f)}{n} < a < b < \limsup_n \frac{S_n(f)}{n}} \]
    are measurable and invariant sets.
    Without loss of generality, let \( b > 0 \).
    Let \( B \in \mathcal E \), where \( B \subseteq D \) such that \( \mu(B) < \infty \).
    Let \( g = f - b\mathbbm 1_B \in L^1(\mu) \).
    Then,
    \[ S_n(g) = S_n(f) - bS_n(\mathbbm 1_B) \geq S_n(f) - bn \]
    which is positive on \( D \) for some \( n \) by the definition of \( \limsup_n \).
    We will apply the maximal ergodic lemma with \( E = D \) and \( \mu = \eval{\mu}_D \); \( \Theta \) is still measure-preserving on this new measure since
    \[ \eval{\mu}_D(A) = \mu(A \cap D) = \mu(\Theta^{-1}(A \cap D)) = \mu(\Theta^{-1}(A) \cap \Theta^{-1}(D)) = \mu(\Theta^{-1}(A) \cap D) = \eval{\mu}_D(\Theta^{-1}(A)) \]
    Note that \( \qty{S^\star > 0} \subseteq D \) as we restrict our measure space to \( D \), but by the previous inequality, \( S^\star > 0 \) on \( D \).
    So \( D = \qty{S^\star > 0} \).
    Then the maximal ergodic lemma gives
    \[ 0 \leq \int_{S^\star > 0} g \dd{\mu} = \int_D g \dd{\mu} = \int_D f \dd{\mu} - b \mu(B) \]
    Hence, \( b \mu(B) \leq \int_D f \dd{\mu} \).
    By \( \sigma \)-finiteness, this inequality extends to \( D \); one can choose an approximating sequence \( B_n \uparrow D \) where \( \mu(B_n) < \infty \), then take limits to show \( b\mu(D) = b \lim_n \mu(B_n) \leq \int_D f \dd{\mu} \).
    Repeating the above argument for \( -f \) and \( -a \), we obtain \( -a\mu(D) \leq \int_D -f \dd{\mu} \).
    Combining these two inequalities gives
    \[ b\mu(D) \leq \int_D f \dd{\mu} \leq a\mu(D) \]
    But \( a < b \), so \( \mu(D) = 0 \) or \( \infty \), but \( f \) is integrable, so \( \mu(D) = 0 \).
    Now, define
    \[ \Delta = \qty{\liminf_n \frac{S_n(f)}{n} < \limsup_n \frac{S_n(f)}{n}}  = \bigcup_{a < b \in \mathbb Q} D_{a,b} \]
    By countable additivity,
    \[ \mu(\Delta) = \mu\qty(\bigcup_{a < b \in \mathbb Q} D_{a,b}) = \sum_{a < b \in \mathbb Q} \mu(D_{a,b}) = 0 \]
    On \( \Delta^c \), \( \frac{S_n}{n} \) converges in \( [-\infty, \infty] \).
    We define the invariant function \( \overline f \) by
    \[ \overline f = \begin{cases}
        \lim_n \frac{S_n}{n} & x \in \Delta^c \\
        0 & x \in \Delta
    \end{cases} \]
    so \( \frac{S_n}{f} \to \overline f \) almost everywhere as \( n \to \infty \).
    Since \( \mu(\abs{f \circ \Theta^{n-1}}) = \mu(\abs{f}) \), we have \( \mu(\abs{S_n}) \leq n \mu(\abs{f}) \) and thus
    \[ \mu\qty(\abs{\overline f}) = \mu\qty(\liminf_n \abs{\frac{S_n}{n}}) \leq \liminf_n \mu\qty(\abs{\frac{S_n}{n}}) \leq \mu(\abs{f}) \]
    which is one of the results required by the theorem.
    In particular, \( \mu\qty(\abs{\overline f}) < \infty \) so \( \abs{\overline f} < \infty \) almost everywhere.
\end{proof}
\begin{theorem}[von Neumann]
	Let \( (E, \mathcal E, \mu) \) be a finite measure space (not \( \sigma \)-finite).
	Let \( \Theta \colon E \to E \) be measure-preserving.
	Let \( f \in L^p(E) \) with \( 1 \leq p < \infty \).
	Then \( \frac{S_n(f)}{n} \to \overline f \) in \( L^p \).
\end{theorem}
\begin{proof}
	Since \( \Theta \) is measure-preserving, we have
	\[ \norm{f \circ \Theta^i}_p^p = \int_E \abs{f}^p \circ \Theta^i \dd{\mu} = \int_E \abs{f}^p \dd{\mu} = \int_E \abs{f}^p \dd{\mu} = \norm{f}_p^p \]
	Thus, by Minkowski's inequality, for all \( f \in L^p \) we have
	\[ \norm{\frac{S_n(f)}{n}}_p \leq \frac{1}{n} \sum_{i=0}^{n-1} \norm{f \circ \Theta^i}_p = \norm{f}_p \]
	So \( \frac{S_n(f)}{n} \) is a contraction in \( L^p \).
	For each \( K > 0 \), we define \( f_K = \max(\min(f, K), -K) \).
	Then
	\[ \norm{f - f_K}_p^p = \int_E \abs{f - f_K}^p \mathbbm 1_{\abs{f} > K} \dd{\mu} \]
	Since \( \mathbbm 1_{\abs{f} > K} \) converges to zero pointwise, and \( \abs{f - f_K} \leq 2\abs{f}^p \in L^1 \), we find \( \norm{f - f_K}_p < \frac{\varepsilon}{3} \) by dominated convergence, for sufficiently large \( K = K_\varepsilon \).
	As \( \abs{f_K} \leq K \), we have \( \abs{\frac{S_n(f_K)}{n}} \leq K \).
	Since \( \mu \) is finite, \( f_K \in L^1(\mu) \), so by Birkhoff's ergodic theorem, \( \frac{S_n(f_K)}{n} \to \overline f_K \) almost everywhere for some invariant function \( \overline f_K \).
	Note that \( \overline f_k \) is bounded by \( K \) as \( \frac{S_n(f_K)}{n} \) is bounded by \( K \).
	By the bounded convergence theorem, we deduce that \( \norm{\frac{S_n(f_K)}{n} - \overline f_K} \to 0 \) as \( n \to \infty \).
	Further, this holds in \( L^p \) since
	\[ \norm{\frac{S_n(f_K)}{n} - \overline f_K}_p \leq (2K)^{\frac{p-1}{p}} \norm{\frac{S_n(f_K)}{n} - \overline f_K}_1 < \frac{\varepsilon}{3} \]
	where the last inequality holds for sufficiently large \( n \).
	Since \( \mu \) is a finite measure, \( L^p(\mu) \subseteq L^1(\mu) \), hence by Birkhoff's ergodic theorem, \( \frac{S_n(f)}{n} \to \overline f \) almost everywhere as \( f \to \infty \).
	Then, by the contraction property applied to \( f - f_K \),
	\begin{align*}
		\norm{\overline f - \overline f_K}_p^p &= \int_E \abs{\overline f - \overline f_K}^p \dd{\mu} \\
		&= \int_E \liminf_n \abs{\frac{S_n(f) - S_n(f_K)}{n}}^p \dd{\mu} \\
		&\leq \liminf_n \int_E \abs{\frac{S_n(f) - S_n(f_K)}{n}}^p \dd{\mu} \\
		&= \liminf_n \int_E \abs{\frac{S_n(f - f_K)}{n}}^p \dd{\mu} \\
		&\leq \liminf_n \norm{f - f_K}_p^p \\
		&= \norm{f - f_K}_p^p < \qty(\frac{\varepsilon}{3})^p
	\end{align*}
	So in particular, \( \overline f \in L^p \).
	Then by the triangle inequality,
	\begin{align*}
			\norm{\frac{S_n(f)}{n} - \overline f}_p &\leq \norm{\frac{S_n(f) - S_n(f_K)}{n}}_p + \norm{\frac{S_n(f_K)}{n} - \overline f_K}_p + \norm{\overline f - \overline f_K}_p \\
			&< \norm{\frac{S_n(f) - S_n(f_K)}{n}}_p + \frac{2\varepsilon}{3} \\
			&\leq \norm{f - f_K}_p + \frac{2\varepsilon}{3} = \varepsilon
	\end{align*}
	for sufficiently large \( n \).
\end{proof}

\subsection{Infinite product spaces}
Let \( E = \mathbb R^{\mathbb N} = \qty{x = (x_n)_{n \in \mathbb N}} \) be the space of real sequences.
Consider
\[ \mathcal C = \qty{A = \prod_{n=1}^\infty A_n \midd A_n \in \mathcal B, \exists N \in \mathbb N, \forall n > N, A_n = \mathbb R} \]
This forms a \( \pi \)-system, which generates the \emph{cylindrical \( \sigma \)-algebra} \( \sigma(\mathcal C) \).
One shows that \( \sigma(\mathcal C) = \sigma(\qty{f_n \mid n \in \mathbb N}) \) where \( f_n(x) = x_n \) are the coordinate projection functions on \( E \).
We can also show \( \sigma(\mathcal C) = \mathcal B(\mathbb R^{\mathbb N}) \) for the product topology.
Let \( (X_n)_{n \in \mathbb N} \) be a sequence of independent and identically distributed random variables defined on \( (\Omega, \mathcal F, \mathbb P) \) with marginal distributions \( \mu_{X_n} = m \) for all \( n \); this exists by an earlier theorem.
We define a map \( X \colon \Omega \to E \) by \( X(\omega)_n = X_n(\omega) \).
This is \( \mathcal F \)--\( \sigma(\mathcal C) \) measurable, since for all \( A \in \mathcal C \), we have
\[ X^{-1}(A) = \qty{\omega \mid X_1(\omega) \in A_1, \dots, X_N(\omega) \in A_n} = \bigcap_{n=1}^N X_n^{-1}(A_n) \in \mathcal F \]
We denote \( \mu = \mathbb P \circ X^{-1} \), which is the unique product probability measure in \( \mathbb R^{\mathbb N} \) satisfying
\begin{align*}
		\mu\qty(\prod_{n=1}^\infty A_n) &= \lim_{N \to \infty} \mu\qty(\prod_{n=1}^N A_n) \\
		&= \lim_{N \to \infty} \prob{X_1 \in A_1, \dots, X_N \in A_N} \\
		&= \lim_{N \to \infty} \prob{X_1 \in A_1} \cdots \prob{X_N \in A_N} \\
		&= \prod_{n=1}^\infty \prob{X_n \in A_n} \\
		&= \prod_{n=1}^\infty m(A_n)
\end{align*}
Note that we need to use finiteness of \( N \) to exploit independence of the \( X_i \).
We call \( (E, \mathcal E, \mu) = (\mathbb R^{\mathbb N}, \sigma(\mathcal C), m^{\mathbb N}) \) the \emph{canonical model} for an infinite sequence of random variables of law \( m \).
\begin{theorem}
	The shift map \( \Theta \colon E \to E \) defined by \( \Theta(x)_n = x_{n+1} \) is measure preserving and ergodic.
\end{theorem}
\begin{proof}
	For \( A \in \mathcal C \),
	\begin{align*}
			\mu(A) &= \prob{X_1 \in A_1, \dots, X_N \in A_N} \\
			&= \prob{X_1 \in A_1} \cdots \prob{X_N \in A_N} \\
			&= \prod_{n=1}^N m(A_n) \\
			&= \prob{X_2 \in A_1} \cdots \prob{X_{N+1} \in A_N} \\
			&= \mu(\Theta^{-1}(A))
	\end{align*}
	so \( \Theta \) is measure-preserving.
	Recall that the tail \( \sigma \)-algebra is defined by \( \mathcal T = \bigcap_n \mathcal T_n \) where \( \mathcal T_n = \sigma(\qty{X_k \mid k \geq n + 1}) \).
	Note that for all \( A \in \mathcal C \), we have \( \Theta^{-n}(A) = \qty{x \in \mathbb R^{\mathbb N} \mid (x_{n+1}, x_{n+2}, \dots) \in A} \in \mathcal T_n \).
	Now, if \( A \) is invariant, \( A = \Theta^{-n}(A) \in \mathcal T_n \) for all \( n \), so \( A \in \mathcal T \).
	By Kolmogorov's zero-one law, \( \mu(A) = 0 \) or \( \mu(A) = 1 \) as required for ergodicity.
\end{proof}
We can apply Birkhoff's ergodic theorem to \( \Theta \).
If \( f \in L^1(\mu) \), then \( \frac{S_n(f)}{n} \to \overline f \in L^1(\mu) \) almost surely.
Since \( \overline f \) is invariant and \( \Theta \) is ergodic, \( \overline f \) is almost surely constant.
% TODO: rearrange
By von Neumann's \( L^p \)-ergodic theorem, convergence holds in fact in \( L^1 \).

\subsection{Strong law of large numbers}
\begin{theorem}
	Let \( \int_{\mathbb R} \abs{x} \dd{m(x)} < \infty \), and let \( \int_{\mathbb R} x \dd{m(x)} = \nu \).
	Then
	\[ \mu\qty(\qty{x \in \mathbb R^{\mathbb N} \midd \frac{x_1 + x_2 + \dots + x_n}{n} \to \nu}) = 1 \]
\end{theorem}
\begin{proof}
	Let \( f(x) = x_1 \).
	Then \( f \in L^1(\mu) \), since \( \int_E \abs{f} \dd{\mu} = \int_{\mathbb R} \abs{x} \dd{m(x)} < \infty \).
	So by Birkhoff's ergodic theorem,
	\[ \mu\qty(\qty{ \frac{x_1 + \dots + x_n}{n} \to \nu }) = \mu\qty(\qty{ \frac{S_n(f)}{n} \to \overline f }) = 1 \]
	where we also use von Neumann's ergodic theorem to deduce that \( \overline f = \mu(\overline f) = \lim_n \mu\qty(\frac{S_n(f)}{n}) = \frac{n}{n} \nu = \nu \).
\end{proof}
\begin{theorem}[strong law of large numbers]
	Let \( (X_n)_{n \in \mathbb N} \) be independent and identically distributed random variables such that \( \expect{\abs{X_1}} < \infty \).
	Then \( \frac{1}{n} \sum_{i=1}^n X_i \to \expect{X} \) almost surely.
\end{theorem}
\begin{proof}
	Inject \( X \) from \( \Omega \) to \( E = \mathbb R^{\mathbb N} \) as before, and notice that
	\[ \prob{\frac{1}{n} \sum_{i=1}^n X_i \to \expect{X}} = \mu\qty(\qty{ x \midd \frac{x_1 + \dots + x_n}{n} \to \nu }) = 1 \]
\end{proof}
\begin{remark}
	The hypothesis \( \expect{\abs{X}} < \infty \) cannot be weakened; we see on an example sheet that \( \frac{1}{n} \sum_{i=1}^n X_i \) can exhibit various behaviours.
	Note that this notion of convergence is stronger than the weak convergence seen in the central limit theorem.
	The law of the iterated logarithm is that
	\[ \limsup_n \frac{X_1 + \dots + X_n}{\sqrt{2 n \log \log n}} = 1 \]
	almost surely, and \( -1 \) for the limit inferior.
	In particular, the central limit theorem does not hold almost surely.
\end{remark}
% TODO: rearrange following thm to above
\begin{corollary}
	In the strong law of large numbers, we have \( \expect{\abs{\frac{1}{n} \sum_{i=1}^n X_i - \expect{X}}} \to 0 \) as \( n \to \infty \).
\end{corollary}
