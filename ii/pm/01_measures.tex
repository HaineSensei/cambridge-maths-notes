\subsection{Definitions}
\begin{definition}
	Let \( E \) be a (nonempty) set. A collection \( \mathcal E \) of subsets of \( E \) is called a \emph{\( \sigma \)-algebra} if the following properties hold:
	\begin{itemize}
		\item \( \varnothing \in \mathcal E \);
		\item \( A \in \mathcal E \implies A^c = E \setminus A \in \mathcal E \);
		\item if \( (A_n)_{n \in \mathbb N} \) is a countable collection of sets in \( \mathcal E \), \( \bigcup_{n \in \mathbb N} A_n \in \mathcal E \).
	\end{itemize}
\end{definition}
\begin{example}
	Let \( \mathcal E = \qty{\varnothing, E} \).
	This is a \( \sigma \)-algebra.
	Also, \( \mathcal P(E) = \qty{A \subseteq E} \) is a \( \sigma \)-algebra.
\end{example}
\begin{remark}
	Since \( \bigcap_n A_n = \qty(\bigcup_n A_n^c)^c \), any \( \sigma \)-algebra \( \mathcal E \) is closed under countable intersections as well as under countable unions.
	Note that \( B \setminus A = B \cap A^c \in \mathcal E \), so \( \sigma \)-algebras are closed under set difference.
\end{remark}
\begin{definition}
	A set \( E \) with a \( \sigma \)-algebra \( \mathcal E \) is called a \emph{measurable space}.
	The elements of \( \mathcal E \) are called \emph{measurable sets}.
\end{definition}
\begin{definition}
	A \emph{measure} \( \mu \) is a set function \( \mu : \mathcal E \to [0,\infty] \), such that \( \mu(\varnothing) = 0 \), and for a sequence \( (A_n)_{n \in \mathbb N} \) such that the \( A_n \) are disjoint, we have \( \mu\qty(\bigcup_{n \in \mathbb N} A_n) = \sum_{n \in \mathbb N} \mu(A_n) \).
	This is the \emph{countable additivity} property of the measure.
\end{definition}
\begin{remark}
	If \( E \) is countable, then for any \( A \in \mathcal P(E) \) and measure \( \mu \), we have \( \mu(A) = \mu\qty(\bigcup_{x\in A} \qty{x}) = \sum_{x \in A} \mu(\qty{x}) \).
	Hence, measures are uniquely defined by the measure of each singleton.
	This corresponds to the notion of a probability mass function.
\end{remark}
\begin{definition}
	For a collection \( \mathcal A \) of subsets of \( E \), we define the \( \sigma \)-algebra \emph{\( \sigma(A) \) generated by \( \mathcal A \)} by
	\[ \sigma(\mathcal A) = \qty{A \subseteq E \colon A \in \mathcal E \text{ for all \( \sigma \)-algebras } \mathcal E \supseteq \mathcal A} \]
	So it is the smallest \( \sigma \)-algebra containing \( \mathcal A \).
	Equivalently,
	\[ \sigma(\mathcal A) = \bigcap_{\mathcal E \supseteq \mathcal A, \mathcal E \text{ a \( \sigma \)-algebra}} \mathcal E \]
\end{definition}

\subsection{Rings and algebras}
To construct good generators, we define the following.
\begin{definition}
	\( \mathcal A \subseteq \mathcal P(E) \) is called a \emph{ring} over \( E \) if \( \varnothing \in \mathcal A \) and \( A, B \in \mathcal A \) implies \( B \setminus A \in \mathcal A \) and \( A \cup B \in \mathcal A \).
\end{definition}
Rings are easier to manage than \( \sigma \)-algebras because there are only finitary operators.
\begin{definition}
	\( \mathcal A \) is called an \emph{algebra} over \( E \) if \( \varnothing \in \mathcal A \) and \( A, B \in \mathcal A \) implies \( A^c \in \mathcal A \) and \( A \cup B \in \mathcal A \).
\end{definition}
\begin{remark}
	Rings are closed under symmetric difference \( A \triangle B = (B \setminus A) \cup (A \setminus B) \), and are closed under intersections \( A \cap B = A \cup B \setminus A \triangle B \).
	Algebras are rings, because \( B \setminus A = B \cap A^c = (B^c \cup A)^c \).
	Not all rings are algebras, because rings do not need to include the entire space.
\end{remark}
\begin{proposition}[Disjointification of countable unions]
	Consider \( \bigcup_n A_n \) for \( A_n \in \mathcal E \), where \( \mathcal E \) is a \( \sigma \)-algebra (or a ring, if the union is finite).
	Then there exist \( B_n \in \mathcal E \) that are disjoint such that \( \bigcup_n A_n = \bigcup_n B_n \).
\end{proposition}
\begin{proof}
	Define \( \widetilde A_n = \bigcup_{j \leq n} A_j \), then \( B_{n+1} = \widetilde A_n \setminus \widetilde A_{n-1} \).
\end{proof}
\begin{definition}
	A \emph{set function} on a collection \( \mathcal A \) of subsets of \( E \), where \( \varnothing \in \mathcal A \), is a map \( \mu \colon \mathcal A \to [0,\infty] \) such that \( \mu(\varnothing) = 0 \).
	We say \( \mu \) is \emph{increasing} if \( \mu(A) \leq \mu(B) \) for all \( A \subseteq B \) in \( \mathcal A \).
	We say \( \mu \) is \emph{additive} if \( \mu(A \cup B) = \mu(A) + \mu(B) \) for disjoint \( A, B \in \mathcal A \) and \( A \cup B \in \mathcal A \).
	We say \( \mu \) is \emph{countably additive} if \( \mu\qty(\bigcup_n A_n) = \sum_n \mu(A_n) \) for disjoint sequences \( A_n \) where \( \bigcup_n A_n \) and each \( A_n \) lie in \( \mathcal A \).
	We say \( \mu \) is \emph{countably subadditive} if \( \mu\qty(\bigcup_n A_n) \leq \sum_n \mu(A_n) \) for arbitrary sequences \( A_n \) under the above conditions.
\end{definition}
\begin{remark}
	A measure satisfies all four of the above conditions.
\end{remark}
\begin{theorem}[Carath\'eodory's theorem]
	Let \( \mu \) be a countably additive set function on a ring \( \mathcal A \) of subsets of \( E \).
	Then there exists a measure \( \mu^\star \) on \( \sigma(\mathcal A) \) such that \( \eval{\mu^\star}_{\mathcal A} = \mu \).
\end{theorem}
We will later prove that this extended measure is unique.
\begin{proof}
	For \( B \subseteq E \), we define the \emph{outer measure} \( \mu^\star \) as
	\[ \mu^\star(B) = \inf \qty{\sum_{n \in \mathbb N} \mu(A_n), A_n \in \mathcal A, B \subseteq \bigcup_{n \in \mathbb N} A_n} \]
	If there is no sequence \( A_n \) such that \( B \subseteq \bigcup_{n \in \mathbb N} A_n \), we declare the outer measure \( \mu^\star(B) \) to be \( \infty \).
	We define the class
	\[ \mathcal M = \qty{A \subseteq E \mid \forall B \subseteq E.\,\mu^\star(A) = \mu^\star(A \cap B) + \mu^\star(A \cap B^c)} \]
	This is the class of \emph{\( \mu^\star \)-measurable sets}.
\end{proof}
