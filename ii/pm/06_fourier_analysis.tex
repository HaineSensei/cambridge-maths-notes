\subsection{Fourier transforms}
In this section, we will write \( L^p(\mathbb R^d) \) for the set of measurable functions \( f \colon \mathbb R^d \to \mathbb C \) such that \( \norm{f}_p = \qty(\int_{\mathbb R^d} \abs{f(x)}^p \dd{x})^{\frac 1p} < \infty \).
We can extend the integral as a complex linear map \( L^1(\mathbb R) \to \mathbb C \) by defining
\[ \int_{\mathbb R} (u + iv)(x) \dd{x} = \int_{\mathbb R} u(x) \dd{x} + i \int_{\mathbb R} v(x) \dd{x} \]
Note that for some \( u + iv = \alpha \in \mathbb C \) with \( \abs{\alpha} = 1 \),
\[ \abs{\int_{\mathbb R^d} f(x) \dd{x}} = \int_{\mathbb R^d} \alpha f(x) \dd{x} = \int_{\mathbb R^d} u(x) \dd{x} + i \int_{\mathbb R^d} v(x) \dd{x} \]
But since the left hand side is real-valued, the \( i \int_{\mathbb R^d} v(x) \dd{x} \) term vanishes.
So
\[ \abs{\int_{\mathbb R^d} f(x) \dd{x}} = \int_{\mathbb R^d} u(x) \dd{x} \leq \int_{\mathbb R^d} \abs{f(x)} \dd{x} \]
\begin{definition}
	Let \( f \in L^1(\mathbb R^d) \).
	We define the \emph{Fourier transform} \( \hat f \) by
	\[ \hat f(u) = \int_{\mathbb R^d} f(x) e^{i\inner{u,x}} \dd{x} \]
	where \( \inner{u,x} = \sum_{i=1}^d u_i x_i \).
\end{definition}
\begin{remark}
	Note that \( \abs{\hat f(u)} \leq \norm{f}_1 \).
	Also, if \( u_n \to u \), then \( e^{i\inner{u_n,x}} \to e^{i\inner{u,x}} \).
	By the dominated convergence theorem with dominating function \( \abs{f} \), we have \( \hat f(u_n) \to \hat f(u) \), so \( \hat f \) is a continuous bounded function.
\end{remark}
\begin{definition}
	Let \( f \in L^1(\mathbb R^d) \) such that \( \hat f \in L^1(\mathbb R^d) \).
	Then we say that the \emph{Fourier inversion formula} holds for \( f \) if
	\[ f(x) = \frac{1}{(2\pi)^d} \int_{\mathbb R^d} \hat f(u) e^{-i\inner{u,x}} \dd{u} \]
	almost everywhere in \( \mathbb R^d \).
\end{definition}
\begin{definition}
	Let \( f \in L^1(\mathbb R^d) \cap L^2(\mathbb R^d) \).
	Then the \emph{Plancherel identity} holds for \( f \) if
	\[ \norm{\hat f}_2 = (2\pi)^{\frac d2} \norm{f}_2 \]
\end{definition}
We will show that the Fourier inversion formula holds whenever \( \hat f \in L^1(\mathbb R^d) \), and the Plancherel identity holds for all \( f \in L^1(\mathbb R^d) \cap L^2(\mathbb R^d) \).
\begin{remark}
	Given the Plancherel identity, the Fourier transform is a linear isometry of \( L^2(\mathbb R^d) \), by approximating any function in \( L^2(\mathbb R^d) \) by integrable functions.
\end{remark}
\begin{definition}
	Let \( \mu \) be a finite Borel measure on \( \mathbb R^d \).
	We define the Fourier transform of the measure by
	\[ \hat\mu(u) = \int_{\mathbb R^d} e^{i\inner{u,x}} \dd{\mu(x)} \]
\end{definition}
Note that \( \abs{\hat \mu(u)} \leq \mu(\mathbb R^d) \), and \( \hat \mu \) is continuous by the dominated convergence theorem.
If \( \mu \) has a density \( f \) with respect to the Lebesgue measure, \( \hat\mu = \hat f \).
\begin{definition}
	Let \( X \) be an \( \mathbb R^d \)-valued random variable.
	The \emph{characteristic function} \( \varphi_X \) is given by
	\[ \varphi_X(u) = \expect{e^{i\inner{u,X}}} = \hat \mu_X(u) \]
	where \( \mu_X \) is the law of \( X \).
\end{definition}

\subsection{Convolutions}
\begin{definition}
	Let \( f \in L^1(\mathbb R^d) \) and \( \nu \) be a probability measure on \( \mathbb R^d \).
	We define their \emph{convolution} \( f \ast \nu \) by
	\[ (f \ast \nu)(x) = \begin{cases}
		\int_{\mathbb R^d} f(x-y) \dd{\nu(y)} & \text{if } (y \mapsto f(x-y)) \in L^1(\nu) \\
		0 & \text{else}
	\end{cases} \]
\end{definition}
\begin{remark}
	If \( 1 \leq p < \infty \), by Jensen's inequality,
	\[ \int_{\mathbb R^d} \qty( \int_{\mathbb R^d} \abs{f(x-y)} \dd{\nu(y)} )^p \dd{x} \leq \int_{\mathbb R^d} \int_{\mathbb R^d} \abs{f(x-y)}^p \dd{\nu(y)} \dd{x} = \int_{\mathbb R^d} \int_{\mathbb R^d} \abs{f(x-y)}^p \dd{x} \dd{\nu(y)} = \int_{\mathbb R^d} \int_{\mathbb R^d} \abs{f(x)} \dd{\nu(y)} \dd{x} = \int_{\mathbb R^d} \abs{f(x)} \dd{x} = \norm{f}_p^p \]
	So \( f \in L^p(\mathbb R^d) \), we have \( (y \mapsto f(x-y)) \in L^p(\nu) \) almost everywhere, and again by Jensen's inequality,
	\[ \norm{f \ast \nu}_p^p = \int_{\mathbb R^d} \abs{ \int_{\mathbb R^d} f(x-y)\dd{\nu(y)} }^p \dd{x} \leq \int_{\mathbb R^d} \qty( \int_{\mathbb R^d} \abs{f(x-y)} \dd{\nu(y)} )^p \dd{x} \leq \norm{f}_p^p \]
	Hence \( f \mapsto f \ast \nu \) is a contraction on \( L^p(\mathbb R^d) \).
\end{remark}
In the case where \( \nu \) has a density \( g \) with respect to the Lebesgue measure, we write \( f \ast g = f \ast \nu \).
\begin{definition}
	For probability measures \( \mu, \nu \) on \( \mathbb R^d \), their convolution \( \mu \ast \nu \) is a probability measure on \( \mathbb R^d \) given by the law of \( X + Y \) where \( X, Y \) are independent random variables with laws \( \mu \) and \( \nu \), so
	\begin{align*}
		(\mu \ast \nu)(A) &= \prob{X+Y \in A} \\
		&= \int_{\mathbb R^d \times \mathbb R^d} \symbb 1_A(x+y) \dd{(\mu \otimes \nu)(x, y)} \\
		&= \int_{\mathbb R^d} \int_{\mathbb R^d} \symbb 1_A(x+y) \dd{\nu(y)} \dd{\mu(x)}
	\end{align*}
\end{definition}
If \( \mu \) has density \( f \) with respect to the Lebesgue measure, \( \mu \ast \nu \) has density \( f \ast \nu \) with respect to the Lebesgue measure.
Indeed,
\begin{align*}
	(\mu \ast \nu)(A) &= \int_{\mathbb R^d} \int_{\mathbb R^d} \symbb 1_A(x+y) f(x) \dd{x} \dd{\nu(y)} \\
	&= \int_{\mathbb R^d} \int_{\mathbb R^d} \symbb 1_A(v) f(v-y) \dd{v} \dd{\nu(y)} \\
	&= \int_{\mathbb R^d} \symbb 1_A(v) \int_{\mathbb R^d}f(v-y) \dd{\nu(y)} \dd{v} \\
	&= \int_{\mathbb R^d} \symbb 1_A(v) (f \ast \nu)(v) \dd{v}
\end{align*}
\begin{proposition}
	\( \widehat{f \ast \nu}(u) = \hat f(u) \hat \nu(u) \).
\end{proposition}
\begin{proposition}
	\( \widehat{\mu \ast \nu}(u) = \expect{e^{i\inner{u,X+Y}}} = \expect{e^{i\inner{u,X}}e^{i\inner{u,Y}}} = \hat \mu(u) \hat \nu(u) \).
\end{proposition}

\subsection{Fourier transforms of Gaussians}
\begin{definition}
	The \emph{normal distribution} \( N(0,t) \) is given by the probability density function
	\[ g_t(x) = \frac{1}{\sqrt{2\pi t}} e^{-\frac{x^2}{2t}} \]
\end{definition}
If \( \varphi_X \) is the characteristic function of a standard normal random variable, by integration by parts,
\begin{align*}
	\dv{u} \varphi_X(u) &= \dv{u} \int_{\mathbb R} e^{iux} g_1(x) \dd{x} \\
	&= \int_{\mathbb R} g_1(x) \dv{u} e^{iux} \dd{x} \\
	&= \frac{i}{\sqrt{2\pi}} \int_{\mathbb R} \underbrace{e^{iux}}_{v} \underbrace{x e^{-\frac{x^2}{2}}}_{w'} \dd{x} \\
	&= \frac{i^2}{\sqrt{2\pi}} \int_{\mathbb R} u e^{iux} e^{-\frac{x^2}{2}} \dd{x} \\
	&= -u \varphi_X(u)
\end{align*}
Hence,
\[ \dv{u}\qty(e^{\frac{u^2}{2}} \varphi_X(u)) = ue^{\frac{u^2}{2}} \varphi_X(u) - e^{\frac{u^2}{2}} u \varphi_X(u) = 0 \]
In particular, \( \varphi_X(u) = \varphi_X(0) e^{-\frac{u^2}{2}} = e^{-\frac{u^2}{2}} \).
In other words, \( \hat g_1(u) = \sqrt{2\pi} g_1(u) \).

In \( \mathbb R^d \), consider a Gaussian random vector \( Z = (Z_1, \dots, Z_d) \) with independent and identically distributed entries \( Z_i \sim N(0,1) \).
Then, the joint probability density function of \( \sqrt{t}Z \) is
\[ g_t(x) = \prod_{j=1}^d \frac{1}{\sqrt{2\pi t}} e^{-\frac{x_j^2}{2t}} = (2\pi t)^{-\frac{d}{2}} e^{-\frac{\norm{x}^2}{2t}} \]
The Fourier transform of \( g_t \) is
\[ \hat g_t(u) = \expect{e^{i\inner{u,\sqrt{t}Z}}} = \expect{\prod_{j=1}^d e^{iu_j \sqrt{t} z_j}} = \prod_{j=1}^d \expect{e^{iu_j \sqrt{t} z_j}} = \prod_{j=1}^d e^{-u_j^2 \frac{t}{2}} = e^{-\frac{\norm{u}^2 t}{2}} \]
which implies that in general, \( \hat g_t(u) = (2\pi)^{\frac{d}{2}} t^{\frac{d}{2}} g_{\frac{1}{t}}(u) \).
Taking the Fourier transform with respect to \( u \), \( \hhat g_t = (2\pi)^d g_t \), and since \( g_t(-x) = g_t(x) \) and the Lebesgue measure is translation invariant, we have
\[ g_t(x) = \frac{1}{(2\pi)^d} \hhat g_t(x) = \frac{1}{(2\pi)^d} \int_{\mathbb R^d} e^{-i\inner{u,x}} \hat g_t(u) \dd{u} \]
so the Fourier inversion theorem holds for such Gaussian random vectors.
\begin{definition}
	We say that a function on \( \mathbb R^d \) is a \emph{Gaussian convolution} if it is of the form
	\[ f \ast g_t(x) = \int_{\mathbb R^d} f(x-y) g_t(y) \dd{y} \]
	where \( x \in \mathbb R^d, t > 0, f \in L^1(\mathbb R^d) \).
\end{definition}
We can show that \( f \ast g_t \) is continuous on \( \mathbb R^d \), and \( \norm{f \ast g_t}_1 \leq \norm{f}_1 \).
Note that \( \widehat{f \ast g_t}(u) = \hat f(u) e^{-\frac{\norm{u}^2 t}{2}} \), so \( \norm{\widehat{f \ast g_t}}_\infty \leq \norm{f}_1 \), giving \( \norm{\widehat{f \ast g_t}}_1 \leq \norm{f}_1 (2\pi)^{\frac{d}{2}} t^{-\frac{d}{2}} < \infty \).
\begin{lemma}
	The Fourier inversion theorem holds for all Gaussian convolutions.
\end{lemma}
\begin{proof}
	We can use the Fourier inversion theorem for \( g_t(y) \) to see that
	\begin{align*}
		(2\pi)^d f \ast g_t(x) &= (2\pi)^d \int_{\mathbb R^d} f(x-y) g_t(y) \dd{y} \\
		&= \int_{\mathbb R^d} f(x-y) \int_{\mathbb R^d} e^{-i\inner{u,y}} \hat g_t(u) \dd{u} \dd{y} \\
		&= \int_{\mathbb R^d} e^{-i\inner{u,x}} \int_{\mathbb R^d} f(x-y) e^{i\inner{u,x-y}} \dd{y} \hat g_t(u) \dd{u} \\
		&= \int_{\mathbb R^d} e^{-i\inner{u,x}} \int_{\mathbb R^d} f(z) e^{i\inner{u,z}} \dd{z} \hat g_t(u) \dd{u} \\
		&= \int_{\mathbb R^d} e^{-i\inner{u,x}} \hat f(u) \hat g_t(u) \dd{u} \\
		&= \int_{\mathbb R^d} e^{-i\inner{u,x}} \widehat{f \ast g_t}(u) \dd{u}
	\end{align*}
\end{proof}
\begin{remark}
	If \( \mu \) is a finite measure, then \( \mu \ast g_t = \mu \ast g_{\frac{t}{2}} \ast g_{\frac{t}{2}} \) with \( \mu \ast g_{\frac{t}{2}} \in L^1 \), so is also a Gaussian convolution.
\end{remark}
\begin{lemma}[Gaussian convolutions are dense in \( L^p \)]
	Let \( f \in L^p \) where \( 1 \leq p < \infty \).
	Then \( \norm{f \ast g_t - f}_p \to 0 \) as \( t \to 0 \).
\end{lemma}
\begin{proof}
	One can easily show that the space \( C_c(\mathbb R^d) \) of continuous functions of compact support is dense in \( L^p \).
	Hence, for all \( \varepsilon > 0 \), there exists \( h \in C_c(\mathbb R^d) \) such that \( \norm{f - h}_p < \frac{\varepsilon}{3} \), and by properties of the convolution, we also obtain
	\[ \norm{f \ast g_t - h \ast g_t}_p = \norm{(f - h) \ast g_t}_p \leq \norm{f - h}_p < \frac{\varepsilon}{3} \]
	So
	\[ \norm{f \ast g_t - f}_p \leq \norm{f \ast g_t - h \ast g_t}_p + \norm{h \ast g_t + h}_p + \norm{h - f}_p < \frac{\varepsilon}{2} + \norm{h \ast g_t - h}_p \]
	so it suffices to prove the result for \( f = h \in C_c(\mathbb R^d) \).
	We define a new map \( e(y) = \int_{\mathbb R^d} \abs{h(x-y) - h(x)}^p \dd{x} \).
	Since \( h \) is bounded on its bounded support, the dominated convergence theorem implies that \( e \) is continuous at \( y = 0 \).
	Note that \( e(y) \leq 2^{p+1} \norm{h}_p^p \).
	Hence, by Jensen's inequality,
	\begin{align*}
		\norm{h \ast g_t - h}_p^p &= \int_{\mathbb R^d} \abs{ \int_{\mathbb R^d} (h(x-y) - h(x)) g_t(y) \dd{y} }^p \dd{x} \\
		&\leq \int_{\mathbb R^d} \int_{\mathbb R^d} \abs{h(x-y) - h(x)}^p \dd{x} g_t(y) \dd{y} \\
		&= \int_{\mathbb R^d} e(y) g_t(y) \dd{y} \\
		&= \int_{\mathbb R^d} \underbrace{e(\sqrt{t} z)}_{\to e(0) = 0 \text{ as } t \to 0} g_1(z) \dd{z} \\
		&\to 0
	\end{align*}
\end{proof}
% TODO: Convert random instances of g into g_t
\begin{theorem}[Fourier inversion]
	Let \( f \in L^1(\mathbb R^d) \) be such that \( \hat f \in L^1(\mathbb R^d) \).
	Then for almost all \( x \in \mathbb R^d \),
	\[ f(x) = \frac{1}{(2\pi)^d} \int_{\mathbb R^d} e^{-i\inner{u,x}} \hat f(u) \dd{u} \]
\end{theorem}
\begin{remark}
	This proves that the Fourier transform is injective; \( \hat f = \hat g \) implies \( \widehat{f - g} = 0 \) so by Fourier inversion, \( f = g \) almost everywhere.
	The identity holds everywhere on \( \mathbb R^d \) for the (unique) continuous representative \( f \) in its equivalence class.
\end{remark}
\begin{proof}
	The Fourier inversion theorem holds for the following Gaussian convolution for all \( t \).
	\[ f \ast g_t(x) = \frac{1}{(2\pi)^d} \int_{\mathbb R^d} e^{-i\inner{u,x}} \hat f(u) e^{\frac{-\abs{u}^2 t}{2}} \dd{u} = f_t(x) \]
	Now, since Gaussian convolutions are dense, \( f \ast g_t \to f \) in \( L^1 \), so \( f \ast g_t \to f \) in measure by Markov's inequality.
	Hence, along a subsequence, \( f \ast g_{t_k} \to f \) almost everywhere.
	On the other hand, by the dominated convergence theorem with dominating function \( \abs{\hat f} \), the right hand side converges to \( \frac{1}{(2\pi)^d} \int_{\mathbb R^d} e^{-i\inner{u,x}} \hat f(u) \dd{u} \).
	So this is equal to \( \lim_{t_k \to 0} f_{t_k} \) almost everywhere by uniqueness of limits.
\end{proof}
\begin{theorem}[Plancherel]
	Let \( f \in L^1(\mathbb R^d) \cap L^2(\mathbb R^d) \).
	Then \( \norm{f}_2 = (2\pi)^{-\frac{d}{2}} \norm{\hat f}_2 \).
\end{theorem}
\begin{remark}
	By the Pythagorean identity, \( \inner{f, g} = (2\pi)^{-d} \inner{\hat f, \hat g} \).
\end{remark}
\begin{proof}
	Initially, we assume \( \hat f \in L^1 \).
	In this case, \( f, \hat f \in L^\infty \), and \( (x,u) \mapsto f(x)\hat f(u) \) is integrable for the product Lebesgue measure \( \dd{x} \otimes \dd{u} \) on \( \mathbb R^d \times \mathbb R^d \), so Fubini's theorem for bounded functions applies.
	\begin{align*}
		(2\pi)^d \norm{f}_2^2 &= (2\pi)^d \int_{\mathbb R^d} f(x) \overline{f(x)} \dd{x} \\
		&= \int_{\mathbb R^d} \qty(\int_{\mathbb R^d} e^{-i\inner{u,x}} \hat f(u) \dd{u}) \overline{f(x)} \dd{x} \\
		&= \int_{\mathbb R^d} \hat f(u) \overline{\int_{\mathbb R^d} e^{i\inner{u,x}} f(x) \dd{x}} \dd{u} \\
		&= \int_{\mathbb R^d} \hat f(u) \overline{\hat f(u)} \dd{u} \\
		&= \norm{\hat f}_2^2
	\end{align*}
	To extend this result to general \( f \), we take the Gaussian convolutions \( f \ast g_t = f_t \) such that \( f_t \to f \) in \( L^2 \).
	By the continuity of the norm, \( \norm{f_t}_2 \to \norm{f}_2 \).
	Since \( \abs{\hat f(u) e^{-\frac{\abs{u}^2 t}{2}}}^2 \) increases to \( \abs{\hat f(u)}^2 \), we have by monotone convergence that \( \norm{\hat f_t}_2^2 \uparrow \norm{\hat f}_2^2 \).
	Therefore, since the Plancherel identity holds for the \( f_t \),
	\[ \norm{f}_2^2 = \lim_{t \to 0} \norm{f_t}_2^2 = \lim_{t \to 0} (2\pi)^{-d} \norm{\hat f_t}_2^2 = (2\pi)^{-d} \norm{\hat f}_2^2 \]
\end{proof}
\begin{remark}
	Since \( L_1 \cap L_2 \) is dense in \( L^2 \), we can extend the linear operator \( F_0(f) = (2\pi)^{-\frac{d}{2}} \hat f \) to \( L^2 \) by continuity to a linear isometry \( F \colon L^2 \to L^2 \) known as the \emph{Fourier--Plancherel transform}.
	One can show that \( F \) is surjective with inverse \( F^{-1} \colon L^2 \to L^2 \).
\end{remark}
\begin{example}
	Consider the Dirac measure \( \delta_0 \) on \( \mathbb R \), so \( \hat \delta_0(u) = \int_{\mathbb R} e^{iux} \dd{\delta_0(x)} = 1 \).
	But the inverse Fourier transform would be \( \frac{1}{2\pi} \int_{\mathbb R} e^{iux} \dd{u} \) which is not a Lebesgue integrable function.
\end{example}
\begin{theorem}
	Let \( X \) be a random vector in \( \mathbb R^d \) with law \( \mu_X \).
	Then the characteristic function \( \varphi_X = \hat \mu_X \) uniquely determines \( \mu_X \).
	In addition, if \( \varphi_X \in L^1 \), then \( \mu_X \) has a probability density function \( f_X \) which can be computed almost everywhere by \( \frac{1}{(2\pi)^d} \int_{\mathbb R^d} e^{-i\inner{u,x}} \varphi_X(u) \dd{u} \).
\end{theorem}
\begin{proof}
	Let \( Z = (Z_1, \dots, Z_d) \) be a vector of independent and identically distributed random variables, independent of \( X \), with \( Z_j \sim N(0,1) \).
	Then \( \sqrt{t} Z \) has probability density function \( g_t \).
	Then \( X + \sqrt{t} Z \) has probability density function \( f_t = \mu_X \ast g_t \).
	This is a Gaussian convolution since \( \mu_X \ast g_t = \mu_X \ast g_{\frac t 2} \ast g_{\frac t 2} \).
	Hence,
	\[ f_t(x) = \frac{1}{(2\pi)^d} \int_{\mathbb R^d} e^{i\inner{u,x}} \varphi_X(u) e^{-\frac{\abs{u}^2 t}{2}} \dd{u} \]
	which is uniquely determined by \( \varphi_X \).
	We show on an example sheet that two Borel probability measures \( \mu, \nu \) on \( \mathbb R^d \) coincide if and only if \( \mu(g) = \nu(g) \) for all \( g \colon \mathbb R^d \to \mathbb R \) that are bounded, continuous, and have compact support.
	Now,
	\[ \int_{\mathbb R^d} g(x) f_t(x) \dd{x} = \expect{\underbrace{g(X + \sqrt{t} Z)}_{\to X \text{ a.s.}}} \]
	Since \( \abs{g(X + \sqrt{t}Z)} \leq \norm{g}_\infty < \infty \), by the bounded convergence theorem, this converges to \( \expect{g(X)} = \int_{\mathbb R^d} g(x) \dd{\mu_X(x)} \).
	So by uniqueness of limits, \( \varphi_X \) determines \( \mu_X \).

	If \( \varphi_X \in L^1 \), by dominated convergence, \( f_t(x) \) converges everywhere to some function \( f_X \).
	In particular, since \( \mu_X \ast g_t \geq 0 \), the limit \( f_X \) is also nonnegative on \( \mathbb R^d \).
	Then, for any bounded continuous function on compact support \( g \in C^b_c(\mathbb R^d) \),
	\[ \int_{\mathbb R^d} g(x) f_X(x) \dd{x} = \int_{\mathbb R^d} g(x) \lim_{t \to 0} \underbrace{f_t(x)}_{\norm{\varphi_X}_1} \dd{x} = \lim_{t \to 0} \int_{\mathbb R^d} g(x) f_t(x) \dd{x} = \int_{\mathbb R^d} g(x) \dd{\mu_X(x)} \]
	by the dominated convergence theorem, since \( g \) has compact support.
\end{proof}
\begin{definition}
	A sequence \( (\mu_n)_{n \in \mathbb N} \) of Borel probability measures on \( \mathbb R^d \) \emph{converges weakly} to a Borel probability measure \( \mu \) if \( \mu_n(g) \to \mu(g) \) for all \( g \colon \mathbb R^d \to \mathbb R \) bounded and continuous.
	If \( (X_n)_{n \in \mathbb N}, X \) are random vectors with laws \( (\mu_{X_n}), \mu_X \) such that \( \mu_{X_n} \) converges weakly to \( \mu_X \), we say \( (X_n) \) converges weakly to \( X \).
\end{definition}
\begin{remark}
	If \( d = 1 \), weak convergence is equivalent to convergence in distribution; this is proven on an example sheet.
	One can also show that convergence of \( \mu_n(g) \) to \( \mu(g) \) for all \( g \in C_c^\infty(\mathbb R^d) \) suffices to show weak convergence, where \( C_c^\infty(\mathbb R^d) \) is the space of smooth functions of compact support.
	This is equivalent to the notion of weak-\( {}^\star \) convergence on the function space \( C_b(\mathbb R^d) \).
\end{remark}
\begin{theorem}[L\'evy's continuity theorem]
	Let \( X_n, X \) be random vectors in \( \mathbb R^d \), such that \( \varphi_{X_n}(u) \to \varphi_X(u) \) for all \( u \), as \( n \to \infty \).
	Then \( \mu_{X_n} \to \mu_X \) weakly.
\end{theorem}
\begin{remark}
	The converse holds by definition of weak convergence, testing against the complex exponentials in the Fourier transform.
\end{remark}
\begin{proof}
	Let \( Z = (Z_1, \dots, Z_d) \) be a vector of standard normal random variables, independent from each other, \( X_n \), and \( X \).
	Let \( g \in C_c^\infty(\mathbb R^d) \).
	Then \( g \in L^1(\mathbb R^d) \), and is Lipschitz by the mean value theorem, as its first derivative is bounded.
	Let \( \abs{g(x) - g(y)} \leq \norm{g}_{\mathrm{Lip}} \abs{x - y} \).
	Let \( \varepsilon > 0 \).
	Let \( t > 0 \) be sufficiently small such that \( \sqrt t \norm{g}_{\mathrm{Lip}} \expect{\abs{Z}} < \frac{\varepsilon}{3} \).
	Then,
	\begin{align*}
		\abs{\mu_{X_n}(g) - \mu_X(g)} &= \abs{\expect{g(X_n)} - \expect{g(X)}} \\
		&\leq \expect{\abs{g(X_n) - g(X_n + \sqrt t Z)}} + \expect{\abs{g(X) - g(X + \sqrt t Z)}} + \abs{\expect{g(X_n + \sqrt t Z) - g(X + \sqrt t Z)}} \\
		&\leq 2\norm{g}_{\mathrm{Lip}}\sqrt t \expect{\abs{Z}} + \abs{\expect{g(X_n + \sqrt t Z) - g(X + \sqrt t Z)}} \\
		&\leq \frac{2\varepsilon}{3} + \abs{\expect{g(X_n + \sqrt t Z) - g(X + \sqrt t Z)}}
	\end{align*}
	We show that the remaining term can be made less than \( \frac{\varepsilon}{3} \) as \( n \to \infty \).
	Let \( f_{t,n}(x) = g_t \ast \mu_{X_n} \).
	Then, by Fourier inversion for Gaussian convolutions,
	\begin{align*}
		\expect{g(X_n + \sqrt t Z)} &= \int_{\mathbb R^d} g(x) f_{t,n}(x) \dd{x} \\
		&= \frac{1}{(2\pi)^d} \int_{\mathbb R^d} g(x) \int_{\mathbb R^d} e^{-i\inner{u,x}} \varphi_{X_n}(u) e^{-\frac{\abs{u}^2 t}{2}} \dd{u} \dd{x}
	\end{align*}
	Since characteristic functions are bounded by 1, we can apply the dominated convergence theorem with dominating function \( \abs{g(x)} e^{-\frac{\abs{u}^2 t}{2}} \) to find
	\begin{align*}
		\expect{g(X_n + \sqrt t Z)} &\to \frac{1}{(2\pi)^d} \int_{\mathbb R^d} g(x) \int_{\mathbb R^d} e^{-i\inner{u,x}} \varphi_X(u) e^{-\frac{\abs{u}^2 t}{2}} \dd{u} \dd{x} \\
		&= \int_{\mathbb R^d} g(x) f_t(x) \dd{x} \\
		&= \expect{g(X + \sqrt t Z)}
	\end{align*}
	where \( f_t = g_t \ast \mu_X \).
	So as \( n \to \infty \), the difference between these two terms can be made less than \( \frac{\varepsilon}{3} \) as required.
\end{proof}
\begin{theorem}[central limit theorem]
	Let \( X_1, \dots, X_n \) be independent and identically distributed random variables with \( \expect{X_i} = 0 \) and \( \Var{X_i} = 1 \).
	Let \( S_n = \sum_{i=1}^n X_n \).
	Then
	\[ \frac{1}{\sqrt{n}} S_n \xrightarrow{\text{weakly}} Z \sim N(0,1) \]
	In particular,
	\[ \prob{\frac{1}{\sqrt{n}} S_n \leq x} \to \prob{Z \leq x} \]
\end{theorem}
\begin{proof}
	Let \( X = X_1 \).
	The characteristic function \( \varphi(u) = \varphi_X(u) = \expect{e^{iuX}} \) satisfies \( \varphi(0) = 1 \), \( \varphi'(u) = i \expect{X e^{iuX}} \), \( \varphi''(u) = i^2 \expect{X^2 e^{iuX}} \).
	We can find \( \varphi'(0) = i\expect{X} = 0 \) and \( \varphi''(0) = -\expect{X^2} = -\Var X = -1 \).
	By Taylor's theorem, \( \varphi(v) = 1 - \frac{v^2}{2} + o(v^2) \) as \( v \to 0 \).
	Now, denoting \( \varphi_n(u) = \varphi_{\frac{1}{\sqrt n} S_n}(u) \), we can write
	\begin{align*}
		\varphi_n(u) &= \expect{e^{i\frac{u}{\sqrt n} (X_1 + \dots + X_n)}} \\
		&= \prod_{j=1}^n \expect{e^{i\frac{u}{\sqrt n} X_j}} \\
		&= \qty[\varphi\qty(\frac{u}{\sqrt n})]^n \\
		&= \qty[1 - \frac{u^2}{2n} + o\qty(\frac{1}{n})]^n
	\end{align*}
	The complex logarithm satisfies \( \log(1 + z) = z + o(z) \), so by taking logarithms, we find
	\[ \log \varphi_n(u) = n \log\qty(1 - \frac{u^2}{2n} + o\qty(\frac{1}{n})) = -\frac{u^2}{2} \]
	Hence, \( \varphi_n(u) \to e^{-\frac{\abs{u}^2}{2}} = \varphi_Z(u) \).
	So by L\'evy's continuity theorem, the result follows.
\end{proof}
\begin{remark}
	This theorem extends to \( \mathbb R^d \) by using the next proposition, using the fact that \( X_n \to X \) weakly in \( \mathbb R^d \) if and only if \( \inner{X_n, v} \to \inner{X, v} \) weakly in \( \mathbb R \) for all \( v \in \mathbb R^d \).
\end{remark}
\begin{definition}
	A random variable \( X \) in \( \mathbb R^d \) is called a \emph{Gaussian vector} if \( \inner{X_n, v} \) are Gaussian for each \( v \in \mathbb R^d \).
\end{definition}
\begin{proposition}
	Let \( X \) be a Gaussian vector in \( \mathbb R^d \).
	Then \( Z = AX + b \) is a Gaussian vector in \( \mathbb R^m \) where \( A \) is an \( m \times d \) matrix and \( b \in \mathbb R^m \).
	Also, \( X \in L^2(\mathbb R^d) \), and \( \mu = \expect{X} \) and \( V = \Cov{X_i, X_j} \) exist and determine \( \mu_X \).
	The characteristic function is
	\[ \varphi_X(u) = e^{i\inner{\mu,u} - \frac{\inner{u,Vu}}{2}} \]
	If \( V \) is invertible, then \( \mu_X \) has a probability density function
	\[ f_X(x) = (2\pi)^{-\frac{d}{2}} (\det V)^{-\frac{1}{2}} \exp{-\inner{x-\mu, V^{-1}(x - \mu)}} \]
	Subvectors \( X_{(1)}, X_{(2)} \) of \( X \) are independent if and only if \( \Cov{X_{(1)}, X_{(2)}} = 0 \).
\end{proposition}
\begin{proposition}
	Let \( X_n \to X \) weakly in \( \mathbb R^d \) as \( n \to \infty \).
	Then,
	\begin{enumerate}
		\item if \( h \colon \mathbb R^d \to \mathbb R^k \) is continuous, then \( h(X_n) \to h(X) \) weakly;
		\item if \( \abs{X_n - Y_n} \to 0 \) in probability, then \( Y_n \to X \) weakly;
		\item if \( Y_n \to c \) in probability where \( c \) is constant on \( \Omega \), then \( (X_n, Y_n) \to (X, c) \) weakly in \( \mathbb R^d \times \mathbb R^d \).
	\end{enumerate}
\end{proposition}
\begin{remark}
	Combining parts (iii) and (i), \( X_n + Y_n \to X + c \) weakly if \( Y_n \to c \) in probability.
	If \( d = 1 \), then in addition \( X_n Y_n \to c X \) weakly.
\end{remark}
\begin{proof}
	\emph{Part (i).}
	This follows from the fact that \( gh \) is continuous for any test function \( g \).

	\emph{Part (ii).}
	Let \( g \colon \mathbb R^d \to \mathbb R \) be bounded and Lipschitz continuous.
	Then
	\[ \abs{\expect{g(Y_n)} - \expect{g(X)}} \leq \underbrace{\abs{\expect{g(X_n)} - \expect{g(X)}}}_{< \frac{\varepsilon}{3}} + \expect{\abs{g(X_n) - g(Y_n)}} \]
	where the bound on \( \expect{g(X_n)} - \expect{g(X)} \) holds for sufficiently large \( n \).
	Then the remaining term is upper bounded by
	\[ \expect{\abs{g(X_n) - g(Y_n)}} \qty(\symbb 1_{\qty{\abs{X_n - Y_n} \leq \frac{\varepsilon}{3\norm{g}_{\mathrm{Lip}}}}} + \symbb 1_{\qty{\abs{X_n - Y_n} > \frac{\varepsilon}{3\norm{g}_{\mathrm{Lip}}}}} ) \]
	\[ \leq \norm{g}_{\mathrm{Lip}} \frac{\varepsilon}{3\norm{g}_{\mathrm{Lip}}} + 2\norm{g}_\infty \prob{\abs{X_n - Y_n} > \frac{\varepsilon}{3\norm{g}_{\mathrm{Lip}}}} < \frac{2\varepsilon}{3} \]
	for sufficiently large \( n \).

	\emph{Part (iii).}
	\( \abs{(X_n, c) - (X_n, Y_n)} = \abs{Y_n - c} \to 0 \) in probability.
	Also, \( \expect{g(X_n, c)} \to \expect{g(X, c)} \) for all bounded continuous maps \( g \colon \mathbb R^d \times \mathbb R^d \to \mathbb R \), so \( (X_n, c) \to (X, c) \) weakly.
	Hence, by (ii), \( (X_n, Y_n) \to (X, c) \) weakly.
\end{proof}
