We aim to show that \( \symsfup{PA} \) is incomplete: there is a sentence \( p \) such that \( \symsfup{PA} \) does not prove \( p \) or \( \neg p \).
Equivalently, there is a sentence \( p \) that is true in \( \mathbb N \) but \( \symsfup{PA} \not\vdash p \).
In this section, by `true' we mean true in \( \mathbb N \), and by `unprovable' we mean \( \symsfup{PA} \) does not prove it, so more concisely we wish to find an unprovable true sentence.
Our aim is to find a sentence \( p \) that asserts that it is not provable in \( \symsfup{PA} \); then \( p \) is true if and only if \( p \) is not provable.
Then the proof is complete, as if \( p \) is false, \( p \) is provable and hence true by soundness.

\subsection{Definability}
Recall that a subset \( S \subseteq \mathbb N \) is \emph{definable} if there is a formula \( p \) with free variable \( x \) such that \( m \in S \) if and only if \( p(m) \) is true.
For example, the set of primes is definable, taking \( p(x) \) to be \( (x \neq 1) \wedge (\forall y)(\forall z)(yz = x \Rightarrow (y = 1) \vee (z = 1)) \).
We might say that `\( m \) is prime' is definable.

A function \( f \colon \mathbb N \to \mathbb N \) is similarly called definable if there is a formula \( p \) with free variables \( x, y \) such that \( f(m) = n \) if and only if \( p(m,n) \) is true.
The function \( f(x) = \floor{\frac{x}{2}} \) is definable, setting \( p(x,y) \) to be \( (x = 2y) \vee (x = 2y + 1) \).
Similarly, \( x^2 \) is definable.
In fact, any function \( f \) given by an algorithm is definable in \( \symsfup{PA} \), but this will not be proven in this section.

\subsection{Coding}
\( L \) has symbols
\[ 0, s, +, \cdot, =, \bot, \Rightarrow, (, ), \forall, x, ' \]
labelling each variable \( x, x', x'' \) and so on.
We code each symbol by assigning it a number, so \( v(0) = 1, \dots, v(') = 12 \).
A formula \( p \) is encoded by
\[ c(p) = 2^{v(\text{first symbol})} 3^{v(\text{second symbol})} \dots \text{\( n \)th prime}^{v(\text{\( n \)th symbol})} \]
For instance, if \( p \) is the assertion \( (\forall x)(x=0) \), then
\[ c(p) = 2^8 3^{10} 5^{11} 7^9 11^8 13^{11} 17^5 19^1 23^9 \]
Clearly, not all numbers encode formulae.
We will write \( S_n \) for the formula encoded by \( n \), with \( S_n = \bot \) if \( n \) does not encode a formula.
Observe that the statement `\( n \) codes a formula' is definable, as there is an algorithm to decide it.

The statement `\( l, m, n \) code formulae and \( S_n \) is obtained from \( S_l, S_m \) by modus ponens' is definable.
The analogous statement for generalisation is also definable in a similar way.
The axioms of \( \symsfup{PA} \) are clearly definable, and `\( n \) codes a logical axiom or axiom of \( \symsfup{PA} \)' is definable.
Given formulae \( p_1, \dots, p_n \), we code the sequence as
\[ s(p_1, \dots, p_n) = 2^{c(p_1)} 3^{c(p_2)}\dots \text{\( n \) prime}^{c(p_n)} \]
Thus, `\( n \) codes a proof' is definable, and `\( n \) codes a proof of \( S_m \)' is definable.
Let \( \theta(m,n) \) be a formula defining `\( n \) codes a proof of \( S_m \)'.
Let \( \phi(m) = \text{`\( S_m \) is provable'} \) is definable, as \( \phi(m) = (\exists n)(\theta(m,n)) \).

\subsection{G\"odel's incompleteness theorem}
Consider \( \chi(m) = \text{`\( m \) codes a formula \( S_m \) with one free variable, such that \( S_m(m) \) is unprovable'} \).
This is definable, so is given by some formula \( p(x) \), so \( \chi(m) \) holds if and only if \( p(m) \) holds.
Let \( N \) be the code for \( p(x) \).
Then, \( p(N) \) is the assertion that \( N \) codes a formula \( S_N \) with one free variable, such that \( S_N(N) \) is unprovable.
Note that \( S_N = p \) and \( S_N(N) = p(N) \), so \( p(N) \) asserts that \( p(N) \) is unprovable.
The sentence \( p(N) \) suffices for the above argument, so we have shown the following theorem.
\begin{theorem}
    \( \symsfup{PA} \) is incomplete.
\end{theorem}
Note that if our proof above could be written in \( \symsfup{PA} \), we would then have that \( p(N) \) is provable in \( \symsfup{PA} \).
One can check that the proof used the fact that a model of \( \symsfup{PA} \) exists (namely, \( \mathbb N \), although this was not particularly important).
We thus used the statement \( \mathrm{Con}(\symsfup{PA}) \), that \( \symsfup{PA} \) is consistent, or equivalently,
\[ (\forall x)(x \text{ does not code a proof of } \bot) \]
Thus, our proof above formalises to the statement
\[ \symsfup{PA} \cup \qty{\mathrm{Con}(\symsfup{PA})} \vdash p(N) \]
The next theorem then follows.
\begin{theorem}
    \( \symsfup{PA} \not\vdash \mathrm{Con}(\symsfup{PA}) \).
\end{theorem}
\( \symsfup{PA} \) is incomplete, but we cannot add any true sentence \( t \) to obtain a complete theory.
Indeed, the proof above can be performed on this new theory \( \symsfup{PA} \cup \qty{t} \) to show that it is incomplete.
However, \( \symsfup{PA} \) can certainly be extended to some complete theory by taking the set of all sentences that hold in \( \mathbb N \).
We cannot use the above proof to show that \( T \) is incomplete, since this would immediately derive a contradiction.
Almost all of the above proof is still valid, so the only invalid part must lead to this contradiction; there must be no algorithm to decide truth of sentences in \( \symsfup{PA} \).
\begin{theorem}
    \( T \) is not definable.
\end{theorem}
Note that \( \symsfup{ZFC} \vdash \mathrm{Con}(\symsfup{PA}) \), where \( \mathrm{Con}(\symsfup{PA}) \) represents the sentence
\[ (\forall x \in \omega)(x \text{ does not code a proof of } \bot) \]
This is because \( \symsfup{ZFC} \) proves that \( \symsfup{PA} \) has a model, namely \( \omega \).
However, as for the above theorems, we obtain the following.
\begin{theorem}
    \( \symsfup{ZFC} \) is incomplete (if \( \symsfup{ZFC} \) is consistent).
\end{theorem}
\begin{theorem}
    \( \symsfup{ZFC} \not\vdash \mathrm{Con}(\symsfup{ZFC}) \) (if \( \symsfup{ZFC} \) is consistent).
\end{theorem}
