In this section, we will attempt to understand the structure of the universe of sets.
In order to do this, we will treat set theory as a first-order theory like any other, and can therefore study it with our usual tools.
In particular, we will study a particular theory called \emph{Zermelo--Fraenkel set theory}, denoted \( \mathsf{ZF} \).
The language has \( \Omega = \varnothing, \Pi = \qty{\in}, \alpha(\in) = 2 \).
A `universe of sets' is simply a model \( (V, \in_V) = (V, \in) \) for the axioms of \( \mathsf{ZF} \).
% This theory has nine axioms: two to begin the theory, four to construct sets, and three that might be unintuitive at first.
We can view this section as a worked example of the concepts of predicate logic, but every model of \( \mathsf{ZF} \) will contain a copy of (most of) mathematics, so they will be very complicated.
