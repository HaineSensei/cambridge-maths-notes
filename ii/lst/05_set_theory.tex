\subsection{Axioms of \texorpdfstring{\(\symsfup{ZF}\)}{ZF}}
In this section, we will attempt to understand the structure of the universe of sets.
In order to do this, we will treat set theory as a first-order theory like any other, and can therefore study it with our usual tools.
In particular, we will study a particular theory called \emph{Zermelo--Fraenkel set theory}, denoted \( \symsfup{ZF} \).
The language has \( \Omega = \varnothing, \Pi = \qty{\in}, \alpha(\in) = 2 \).
A `universe of sets' is simply a model \( (V, \in_V) = (V, \in) \) for the axioms of \( \symsfup{ZF} \).
We can view this section as a worked example of the concepts of predicate logic, but every model of \( \symsfup{ZF} \) will contain a copy of (most of) mathematics, so they will be very complicated.

We now define the axioms of \( \symsfup{ZF} \) set theory.
\begin{enumerate}
    \item \emph{Axiom of extension}.
    \[ (\forall x)(\forall y)((\forall z)(z \in x \Leftrightarrow z \in y) \Rightarrow x = y) \]
    Note that the converse follows from the definition of equality.
    This implies that sets have no duplicate elements, and have no ordering.
    \item \emph{Axiom of separation} or \emph{comprehension}.
    For a set \( x \) and a property \( p \), we can form the set of \( z \in x \) such that \( p(z) \) holds.
    \[ (\forall t_1)\dots(\forall t_n)(\forall x)(\exists y)(\forall z)(z \in y \Leftrightarrow z \in x \wedge p) \]
    where the \( t_i \) are the parameters, and \( p \) is a formula with free variables \( t_1, \dots, t_n, z \).
    Note that we need the parameters as we may wish to form the set \( \qty{z \in x \mid z \in t} \) for some variable \( t \).
    We write \( \qty{z \in x \mid p(z)} \) for the set guaranteed by this axiom; this is an abbreviation and does not change the language.
    \item \emph{Empty-set axiom}.
    \[ (\exists x)(\forall y)(\neg y \in x) \]
    This empty set is unique by extensionality.
    We write \( \varnothing \) for the set guaranteed by this axiom.
    For instance, \( p(\varnothing) \) is the sentence \( (\exists x)((\forall y)(\neg y \in x) \wedge p(x)) \).
    \item \emph{Pair-set axiom}.
    \[ (\forall x)(\forall y)(\exists z)(\forall t)(t \in z \Leftrightarrow t = x \vee t = y) \]
    We write \( \qty{x, y} \) for this set \( z \), which is unique by extensionality.
    Some basic set-theoretic principles can now be defined.
    \begin{itemize}
        \item We write \( \qty{x} = \qty{x, x} \) for the singleton set containing \( x \).
        \item We can now define the ordered pair \( (x, y) = \qty{\qty{x}, \qty{x, y}} \); from the axioms so far we can prove that \( (x, y) = (z, t) \) if and only if \( x = z \) and \( y = t \).
        \item We say that \( x \) is an ordered pair if \( (\exists y)(\exists z)(x = (y,z)) \), and \( f \) is a function if \( (\forall x)(x \in f \Rightarrow x \text{ is an ordered pair}) \) and \( (\forall x)(\forall y)(\forall z)((x,y) \in f \wedge (x,z) \in f \Rightarrow y = z) \).
        \item We call a set \( x \) the domain of \( f \), written \( x = \dom f \), if \( f \) is a function and \( (\forall y)(y \in x \Leftrightarrow (\exists z)((y,z) \in f)) \).
        \item The notation \( f \colon x \to y \) means that \( f \) is a function, \( x = \dom f \), and \( (\forall z)(\forall t)((z, t) \in f \Rightarrow t \in y) \).
    \end{itemize}
    \item \emph{Union axiom}.
    For each family of sets \( x \), we can form its union \( \bigcup_{t \in x} t \).
    \[ (\forall x)(\exists y)(\forall z)(z \in y \Leftrightarrow (\exists t)(z \in t \wedge t \in x)) \]
    The set guaranteed by this axiom can be written \( \bigcup x \), and we can write \( x \cup y \) for \( \bigcup \qty{x, y} \).
    We need no intersection axiom, as such intersections already exist by the axiom of separation.
    This cannot be used to create empty intersections, as the axiom of separation can only create subsets of a set that already exists.
    \item \emph{Power-set axiom}.
    \[ (\forall x)(\exists y)(\forall z)(z \in y \Leftrightarrow z \subseteq x) \]
    where \( z \subseteq x \) means \( (\forall t)(t \in z \Rightarrow t \in x) \).
    We write \( \mathcal P(x) \) for the power set of \( x \).
    We can form the Cartesian product \( x \times y \) as a suitable subset of \( \mathcal P(\mathcal P(x \cup y)) \), as if \( z \in x, t \in y \), we have \( (z, t) = \qty{\qty{z}, \qty{z, t}} \in \mathcal P(\mathcal P(x \cup y)) \).
    The set of all functions \( x \to y \) can be defined as a subset of \( \mathbb P(x \times y) \).
    \item \emph{Axiom of infinity}.
    Using our currently defined axioms, any model \( V \) must be infinite.
    For example, writing \( x^+ \) for the \emph{successor} of \( x \) defined as \( x \cup \qty{x} \), the sets \( \varnothing, \varnothing^+, \varnothing^{++}, \dots \) are distinct.
    \[ \varnothing^+ = \qty{\varnothing};\quad \varnothing^{++} = \qty{\varnothing, \qty{\varnothing}};\quad \varnothing^{+++} = \qty{\varnothing, \qty{\varnothing}, \qty{\varnothing, \qty{\varnothing}}};\quad \dots \]
    We write \( 0 = \varnothing, 1 = \varnothing^+, 2 = \varnothing^{++}, \dots \) for the successors created in this way.
    For instance, \( 3 = \qty{0, 1, 2} \).
    \( V \) may not have an infinite element, even though \( V \) itself is infinite, because no \( x \in V \) has all \( y \in V \) as elements: \( V \) does not think of itself as a set, because Russell's paradox follows from the axioms defined so far.

    We say that \( x \) is a successor set if \( \varnothing \in x \) and \( (\forall y)(y \in x \Rightarrow y^+ \in x) \).
    Note that this is a finite-length formula that characterises an infinite set.
    The axiom of infinity is that there exists a successor set.
    \[ (\exists x)(\varnothing \in x \wedge (\forall y)(y \in x \Rightarrow y^+ \in x)) \]
    Note that this set is not uniquely defined, but any intersection of successor sets is a successor set.
    We can therefore take the intersection of all successor sets by the axiom of separation, giving a least successor set denoted \( \omega \).
    Thus, \( (\forall x)(x \in \omega \Leftrightarrow (\forall y)(y \text{ is a successor set} \Rightarrow x \in y)) \).
    For example, we can prove that \( 3 \in \omega \).

    In particular, if \( x \) is a successor set and a subset of \( \omega \), then \( x = \omega \).
    Hence, \( (\forall x)(x \subseteq \omega \wedge \varnothing \in x \wedge (\forall y)(y \in x \Rightarrow y^+ \in x) \Rightarrow x = \omega) \).
    This is `proper' induction over all subsets of \( \omega \), unlike the weaker first-order induction defined in the Peano axioms.
    It is easy to check that \( (\forall x)(x \in \omega \Rightarrow x^+ \neq \varnothing) \) and \( (\forall x)(\forall y)(x \in \omega \wedge y \in \omega \wedge x^+ = y^+ \Rightarrow x = y) \), so \( \omega \) satisfies (in \( V \)) the usual axioms for the natural numbers.
    We can now define `\( x \) is finite' to mean \( (\exists y)(y \in \omega \wedge x \text{ bijects with } y) \), and define `\( x \) is countable' to mean that \( x \) is finite or bijects with \( \omega \).
    \item \emph{Axiom of foundation} or \emph{regularity}.
    We require that sets are built out of simpler sets.
    For example, we want to disallow a set from being a member of itself, and similarly forbid \( x \in y \) and \( y \in x \).
    In general, we want to forbid sets \( x_i \) such that \( x_{i+1} \in x_i \) for each \( i \in \mathbb N \).

    Note that if \( x \in x \), \( \qty{x} \) has no \( \in \)-minimal element.
    If \( x \in y, y \in x \), \( \qty{x, y} \) has no \( \in \)-minimal element.
    In the last example, \( \qty{x_0, x_1, \dots} \) has no \( \in \)-minimal element.
    We now define the axiom of foundation: every nonempty set has an \( \in \)-minimal element.
    \[ (\forall x)(x \neq \varnothing \Rightarrow (\exists y)(y \in x \wedge (\forall z)(z \in x \Rightarrow z \not\in y))) \]
    Any model of \( \symsfup{ZF} \) without this axiom has a submodel of all of \( \symsfup{ZF} \).
    \item \emph{Axiom of replacement}.
    Often, we are given an index set \( I \) and construct a set \( A_i \) for each \( i \in I \), then take the collection \( \qty{A_i \mid i \in I} \).
    In order to write this down, the mapping \( i \mapsto A_i \) must be a function, or equivalently, there must be a set \( \qty{(i, A_i) \mid i \in I} \).
    This is not clear from the other axioms.
    We would like to say that the image of a set under something that looks like a function (since we do not yet have such a set-theoretic function) is a set.

    Let \( (V, \in) \) be an \( L \)-structure.
    A \emph{class} is a set \( C \subseteq V \) such that for some formula \( p \) with free variables \( x \) and some parameters, we have \( x \in C \) if and only if \( p \) holds in \( V \).
    \( C \) is a set outside of our model; it may not correspond to a set \( x \in V \) inside the model.
    For instance, \( V \) is a class, taking \( p \) to be \( x = x \).
    There is a class of infinite sets, taking \( p \) to be `\( x \) is not finite'.
    For any \( t \in V \), the collection of \( x \) with \( t \in x \) is a class; here, \( t \) is a parameter to the class.
    Every set \( y \in V \) is a class by setting \( p \) to be \( x \in y \).
    A \emph{proper class} is a class that does not correspond to a set \( x \in V \): \( \neg(\exists y)(\forall x)(x \in y \Leftrightarrow p) \).
    When writing about classes inside \( \symsfup{ZF} \), we instead write about their defining formulae, as classes have no direct representation in the language.

    Similarly, a \emph{function-class} is a set \( F \subseteq V \) of ordered pairs from \( V \) such that for some formula \( p \) with free variables \( x, y \) and parameters, we have \( (x, y) \) belongs to \( F \) if and only if \( p \), and if \( (x, y), (x, z) \) belong to \( F \), \( y = z \).
    This is intuitively a function whose domain may not be a set.
    For example, the mapping \( x \mapsto \qty{x} \) is a function-class, taking \( p \) to be \( y = \qty{x} \).
    This is not a function, for example, every \( f \) has a domain which is a set in \( V \), and this function has domain \( V \) which is not a set.

    We can now define the axiom of replacement: the image of a set under a function-class is a set.
    \[ (\forall t_1)\dots(\forall t_n)[(\forall x)(\forall y)(\forall z)(p \wedge p[z/y] \Rightarrow y = z) \Rightarrow \]
    \[ (\forall x)(\exists y)(\forall z)(z \in y \Leftrightarrow (\exists t)(t \in x \wedge p[t/x,z/y]))] \]
    For example, for any set \( x \), we can form the set \( \qty{\qty{t} \mid t \in x} \), which is the image of \( x \) under the function class \( t \mapsto \qty{t} \).
    This set could alternatively have been formed using the power-set and separation axioms; we will later present some examples of sets built with this axiom that cannot be constructed from the other axioms.
\end{enumerate}
This completes the description of the axioms of \( \symsfup{ZF} \).
We write \( \symsfup{ZFC} \) for \( \symsfup{ZF} + \symsfup{AC} \), where \( \symsfup{AC} \) is the axiom
\[ (\forall f)[f\text{ is a function} \wedge (\forall x)(x \in \dom f \Rightarrow f(x) \neq \varnothing) \Rightarrow \]
\[ (\exists g)(g\text{ is a function} \wedge (\dom g = \dom f) \wedge (\forall x)(x \in \dom f \Rightarrow g(x) \in f(x)))] \]

\subsection{Transitive sets}
\begin{definition}
    \( x \) is \emph{transitive} if each member of a member of \( x \) is a member of \( x \).
    \[ (\forall y)((\exists z)(y \in z \wedge z \in x) \Rightarrow y \in x) \]
\end{definition}
Equivalently, \( \bigcup x \subseteq x \).
\begin{example}
    \( \varnothing \) is a transitive set.
    \( \qty{\varnothing} \) is also transitive, and \( \qty{\varnothing, \qty{\varnothing}} \) is transitive.
    In general, elements of \( \omega \) are transitive.
    This can be proven by \( \omega \)-induction (inside a model): \( \varnothing \) is transitive, and if \( y \) is transitive, \( y^+ = y \cup \qty{y} \) is clearly transitive.
\end{example}
\begin{lemma}
    Every set is contained in a transitive set.
\end{lemma}
Here, we define `\( x \) contains \( y \)' to mean \( y \subseteq x \), not \( y \in x \).
\begin{remark}
    This proof takes place inside an arbitrary model of \( \symsfup{ZF} \).
    Technically, the statement of the lemma is `let \( (V, \in) \) be a model of \( \symsfup{ZF} \), then for all sets \( x \in V \), \( x \) is contained in a transitive set \( y \in V \)'.
    By completeness, this will show that there is a proof of this fact from the axioms of \( \symsfup{ZF} \).

    Note also that once this lemma is proven, any \( x \) is contained in a least transitive set by taking intersections, called its \emph{transitive closure}, written \( TC(x) \).
    This holds as any intersection of transitive sets is transitive.
\end{remark}
\begin{proof}
    We want to form \( x \cup (\bigcup x) \cup (\bigcup\bigcup x) \cup \dots \); if this is a set, it is clearly transitive and contains \( x \).
    We can show that this is a set by the union axiom applied to the set \( \qty{x, \bigcup x, \bigcup\bigcup x, \dots} \).
    This is a set by applying the axiom of replacement, it is an image of \( \omega \) under the function-class \( 0 \mapsto x, 1 \mapsto \bigcup x, 2 \mapsto \bigcup \bigcup x \) and so on.
    We want to define the function-class \( p(z,w) \) to be \( (z = 0 \wedge w = x) \vee ((\exists t)(\exists u) z = t^+ \wedge w = \bigcup u \wedge p(t,u)) \), but this is not a first-order formula.

    Define that \( f \) is an \emph{attempt} to mean that
    \[ (f \text{ is a function}) \wedge (\dom f \in \omega) \wedge (\dom f \neq \varnothing) \wedge (f(0) = x)\, \wedge \]
    \[ (\forall n)\qty(n \in \omega \wedge n \in \dom f \wedge n \neq 0 \Rightarrow f(n) = \bigcup f(n-1)) \]
    Then,
    \[ (\forall n)(n \in \omega \Rightarrow (\exists f)(f \text{ is an attempt} \wedge n \in \dom f)) \]
    can be proven by \( \omega \)-induction.
    We can similarly prove
    \[ (\forall n)(n \in \omega \Rightarrow (\forall f)(\forall g)(f, g \text{ are attempts} \wedge n \in \dom f \cap \dom g \Rightarrow f(n) = g(n))) \]
    by \( \omega \)-induction.
    We now define the function-class \( p = p(z,w) \) to be
    \[ (\exists f)(f \text{ is an attempt} \wedge z \in \dom f \wedge f(z) = w) \]
\end{proof}
Intuitively, we needed to use the axiom of replacement because we started with a set \( x \) and needed to go `far away' from it, forming \( \bigcup^n x \) for all \( x \).
We could not have used the other axioms such as the power-set axiom, as the \( \bigcup^n x \) are not contained in an obvious larger set.

Transitive closures allow us to pass from the large universe of sets, which is not a set itself, into a smaller world which is a set closed under \( \in \) that contains the relevant sets in question.

\subsection{\texorpdfstring{\( \in \)}{∈}-induction}
We want the axiom of foundation to capture the idea that sets are built out of simpler sets.
\begin{theorem}[principle of \( \in \)-induction]
    For each formula \( p \) with free variables \( t_1, \dots, t_n, x \),
    \[ (\forall t_1)\dots(\forall t_n)[(\forall x)((\forall y)(y \in x \Rightarrow p(y)) \Rightarrow p(x)) \Rightarrow (\forall x)p(x)] \]
\end{theorem}
\begin{proof}
    Given \( t_1, \dots, t_n \) and the statement \( (\forall x)((\forall y)(y \in x \Rightarrow p(y)) \Rightarrow p(x)) \), we want to show \( (\forall x)p(x) \).
    Suppose this is not the case, so there exists \( x \) such that \( \neg p(x) \).
    We want to look at the set \( \qty{t \mid \neg p(t)} \) and take an \( \in \)-minimal element, but this is not necessarily a set, for instance if \( p(x) \) is the assertion \( x \neq x \).

    Let \( u = \qty{t \in TC(\qty{x}) \mid \neg p(t)} \); this is clearly a set in the model, and \( u \neq \varnothing \) as \( x \in u \).
    Let \( t \) be an \( \in \)-minimal element of \( u \), guaranteed by the axiom of foundation.
    Then \( \neg p(t) \) as \( t \in u \), but \( p(z) \) for all \( z \in t \) by minimality of \( t \), noting that \( z \in t \) implies \( z \in TC(\qty{x}) \).
    This gives a contradiction.
\end{proof}
The name of this theorem should be read `epsilon-induction', even though the membership relation is denoted \( \in \) and not \( \epsilon \) or \( \varepsilon \).

The principle of \( \in \)-induction is equivalent to the axiom of foundation in the presence of the other axioms of \( \symsfup{ZF} \).
We say that \( x \) is \emph{regular} if \( (\forall y)(x \in y \Rightarrow y \text{ has a minimal element}) \).
The axiom of foundation is equivalent to the assertion that every set is regular.
Given \( \in \)-induction, we can prove every set is regular.
Suppose \( (\forall y \in x)(y \text{ is regular}) \); we need to show \( x \) is regular.
For a set \( z \) with \( x \in z \), if \( x \) is minimal in \( z \), \( x \) is clearly regular as required.
If \( x \) is not minimal in \( z \), there exists \( y \in x \) such that \( y \in z \).
So \( z \) has a minimal element as \( y \) is regular.
Hence \( x \) is regular.

\subsection{\texorpdfstring{\( \in \)}{∈}-recursion}
We want to be able to define \( f(x) \) given \( f(y) \) for all \( y \in x \).
\begin{theorem}[\( \in \)-recursion theorem]
    Let \( G \) be a function-class, so \( (x,y) \in G \) if and only if \( p(x,y) \) for some formula \( p \).
    Suppose that \( G \) is defined for all sets.
    Then there is a function-class \( F \) defined for all sets by a formula \( q(x,y) \) such that
    \[ (\forall x)\qty(F(x) = G\qty(\eval{F}_x)) \]
    Moreover, this \( F \) is unique.
\end{theorem}
Note that \( \eval{F}_x = \qty{(y, F(y)) \mid y \in x} \) is a set by the axiom of replacement.
\begin{proof}
    Define that \( f \) is an \emph{attempt} if
    \[ f \text{ is a function} \wedge \dom f \text{ is transitive} \wedge (\forall x)\qty(x \in \dom f \Rightarrow f(x) = G\qty(\eval{f}_x)) \]
    Note that \( \eval{f}_x \) is defined as \( \dom f \) is transitive.
    Then,
    \[ (\forall x)(\forall f)(\forall f')(f, f' \text{ are attempts} \wedge x \in \dom f \cap \dom f' \Rightarrow f(x) = f'(x)) \]
    by \( \in \)-induction: if \( f(y) = f'(y) \) for all \( y \in x \), then \( f(x) = f'(x) \).
    Also,
    \[ (\forall x)(\exists f)(f \text{ is an attempt} \wedge x \in \dom f) \]
    by \( \in \)-induction.
    Indeed, if for all \( y \in x \) there exists an attempt defined at \( y \), then for each \( y \in x \) there is a unique attempt \( f_y \) defined on \( TC(\qty{y}) \).
    Let \( f = \bigcup\qty{f_y \mid y \in x} \), which is an attempt with domain \( TC(x) \).
    We can then define \( f' = f \cup \qty{\qty(x, G\qty(\eval{f}_x))} \).
    This is an attempt defined at \( x \).
    We can then take \( q(x,y) \) to be
    \[ (\exists f)(f \text{ is an attempt} \wedge x \in \dom f \wedge f(x) = y) \]
    This defines the function-class \( F \) as required.
    Uniqueness follows from the fact that if \( F, F' \) are suitable function-classes, we have \( (\forall x)(F(x) = F'(x)) \) by \( \in \)-induction.
\end{proof}

\subsection{Well-founded relations}
Note the similarity between the proofs of \( \in \)-induction and \( \in \)-recursion and the proofs of induction and recursion on ordinals.
These proofs are not specific to the relation \( \in \); we only used some of its properties.
\begin{enumerate}
    \item \( p \) is \emph{well-founded}: every nonempty set has a \( p \)-minimal element.
    \item \( p \) is \emph{local}: \( \qty{x \mid p(x,y)} \) is a set.
    This was required to build the \( p \)-transitive closure.
\end{enumerate}
Therefore, we have \( p \)-induction and \( p \)-recursion for all relation-classes \( p \) that are well-founded and local.
In particular, if \( r \) is a well-founded relation on a set \( a \), it is clearly local and hence we have \( r \)-induction and \( r \)-recursion.
The theorems about induction and recursion on ordinals are therefore special cases of this, as a well-ordering is precisely a well-founded total order.

On the set \( \qty{a, b, c} \), let \( r \) be the relation \( arb, brc \).
Choosing \( a' = \varnothing, b' = \qty{\varnothing}, c' = \qty{\qty{\varnothing}} \), the map \( f \colon \qty{a, b, c} \to \qty{a', b', c'} \) given by \( x \mapsto x' \) is a bijection with a transitive set such that \( xry \) if and only if \( f(x) \in f(y) \).
This models the relation \( r \) by \( \in \).

We say that a relation \( r \) on a set \( a \) is \emph{extensional} if
\[ (\forall x \in a)(\forall y \in a)((\forall z \in a)(zrx \Leftrightarrow zry) \Rightarrow x = y) \]
The relation \( r \) in the above example is extensional.
\begin{theorem}[Mostowski's collapsing theorem]
    Let \( r \) be a relation on a set \( a \) that is well-founded and extensional.
    Then, there exists a transitive set \( b \) and a bijection \( f \colon a \to b \) such that
    \[ (\forall x \in a)(\forall y \in a)(x r y \Leftrightarrow f(x) \in f(y)) \]
    Moreover, \( b \) and \( f \) are unique.
\end{theorem}
This is an analogue of subset collapse from the section on ordinals.
Transitive sets are playing the role of initial segments.
Note that the well-foundedness and extensionality conditions are clearly necessary for the theorem, consider \( (\mathbb Z, <) \) or \( (\qty{a,b,c,},<) \) with \( a<b, a<c \) for counterexamples.
\begin{proof}
    We define the function \( f \) by \( f(x) = \qty{f(y) \mid y r x} \) using \( r \)-recursion.
    Note that \( f \) is a function by the axiom of replacement as it is given by a function-class \( F \) obtained from \( r \)-recursion that is defined on the set \( a \).
    Let \( b = \qty{f(x) \mid x \in a} \); this is a set by the axiom of replacement.
    Clearly \( f \) is surjective by the definition of \( b \), and \( b \) is transitive by definition.

    We claim that \( f \) is injective, and then we have that \( yrx \) if and only if \( f(y) \in f(x) \) by definition of \( f \).
    We show
    \[ (\forall x \in a)(\forall x' \in a)(f(x') = f(x) \Rightarrow x' = x) \]
    by \( r \)-induction on \( x \).
    Suppose that \( (\forall y r x)(\forall z \in a)(f(y) = f(z) \Rightarrow y = z) \), we have \( f(x) = f(x') \), and we want to show that \( x = x' \).
    Note that \( \qty{f(y) \mid y r x} = \qty{f(z) \mid z r x'} \) by the definition of \( f \) as \( f(x) = f(x') \).
    So \( \qty{y \mid y r x} = \qty{z \mid z r x'} \), so \( x = x' \) as \( r \) is extensional.
    Uniqueness holds by \( r \)-induction, as we must have \( f(x) = \qty{f(y) \mid y r x} \) for all \( x \in a \).
\end{proof}
In particular, every well-ordered set is order-isomorphic to a unique transitive set well-ordered by \( \in \).
We can now define that an ordinal is a transitive set well-ordered by \( \in \) (or equivalently, totally-ordered, due to the axiom of foundation).
For example, \( \varnothing \) is an ordinal, \( n \in \omega \) is an ordinal, \( \omega \) is also an ordinal, and so on.
Therefore, each well-ordering is order-isomorphic to a unique ordinal called its order type, by Mostowski collapse.
\begin{remark}
    If \( x, y \) are elements of a well-ordered set \( a \) with \( y < x \), then the order type of \( I_x \), which is precisely the image \( f(x) \) under the Mostowski collapse, has an element \( f(y) \), the order type of \( I_y \).
    In particular, given two ordinals \( \alpha, \beta \), the statement \( \alpha < \beta \) is equivalent to \( \alpha \in \beta \).
    Hence \( \alpha = \qty{\beta \mid \beta < \alpha} \).
    Thus, \( \alpha^+ = \alpha \cup \qty{\alpha} \), and \( \sup \qty{\alpha_i \mid i \in I} = \bigcup \qty{\alpha_i \mid i \in I} \).
\end{remark}

\subsection{The universe of sets}
We would like the universe to be V-shaped, in the sense that we begin with \( \varnothing \) and continue taking power sets to create larger and larger sets.
Define sets \( V_\alpha \) for each ordinal \( \alpha \) by
\begin{itemize}
    \item \( V_0 = \varnothing \);
    \item \( V_{\alpha+1} = \mathcal P(V_\alpha) \);
    \item \( V_\lambda = \bigcup \qty{V_\alpha \mid \alpha < \lambda} \) for a nonzero limit ordinal \( \lambda \).
\end{itemize}
This can be viewed as a well-founded recursion on ordinals, or \( \in \)-recursion on the universe but mapping non-ordinals to \( \varnothing \).
For example, \( V_\omega = V_0 \cup V_1 \cup \dots \), and \( V_{\omega+1} = \mathcal P(V_\omega) \).
We will now show that every set is contained within some \( V_\alpha \).
\begin{lemma}
    Each \( V_\alpha \) is transitive.
\end{lemma}
\begin{proof}
    We show this by induction on \( \alpha \).
    Clearly \( V_0 = \varnothing \) is transitive.
    Suppose \( V_\alpha \) is transitive.
    Then \( V_{\alpha+1} \) is transitive as the power set of a transitive set is transitive.
    Indeed, if \( x \) is transitive and \( z \in y \in \mathcal P(x) \), we have \( z \in x \), so \( z \subseteq x \) as \( x \) is transitive, so \( z \in \mathcal P(x) \).
    Now suppose \( \lambda \) is a limit ordinal, and that the \( V_\alpha \) are transitive for \( \alpha < \lambda \).
    Any union of transitive sets is transitive, so \( V_\lambda \) is transitive.
\end{proof}
\begin{lemma}
    Let \( \alpha \leq \beta \).
    Then \( V_\alpha \subseteq V_\beta \).
\end{lemma}
\begin{proof}
    We show this by induction on \( \beta \) for a fixed \( \alpha \).
    If \( \beta = \alpha \), \( V_\alpha \subseteq V_\beta \) is trivial.
    For successors, note that \( V_\beta \subseteq \mathcal P(V_\beta) \) as \( V_\beta \) is transitive.
    So if \( V_\alpha \subseteq V_\beta \), then \( V_\alpha \subseteq V_{\beta+1} \).
    Limits are trivial.
\end{proof}
\begin{theorem}
    Every set \( x \) belongs to \( V_\alpha \) for some \( \alpha \).
\end{theorem}
If we could construct the set \( V \) defined as the union of the \( V_\alpha \) over all ordinals \( \alpha \), \( V \) would be a model of \( \symsfup{ZF} \).
\begin{remark}
    Note that \( x \subseteq V_\alpha \) if and only if \( x \in V_{\alpha+1} \), so it suffices to show that each set \( x \) is a subset of some \( V_\alpha \).
    Once we have \( x \subseteq V_\alpha \) for some \( \alpha \), there is a least such \( \alpha \), called the \emph{rank} of \( x \).
    For example, the rank of \( \varnothing \) is 0, the rank of 1 is 1, the rank of \( \omega \) is \( \omega \), and in general the rank of any ordinal \( \alpha \) is \( \alpha \).
    Intuitively, the rank of a set is the time at which it was created.
\end{remark}
\begin{proof}
    We proceed by \( \in \)-induction on \( x \); we may assume that for all \( y \in x \), there exists \( \alpha \) such that \( y \subseteq V_\alpha \), so \( y \subseteq V_{\mathrm{rank}(y)} \).
    Thus, for each \( y \in x \), \( y \in V_{\mathrm{rank}(y)+1} \), so define \( \alpha = \sup\qty{\mathrm{rank}(y) + 1 \mid y \in x} \).
    Then for all \( y \in x \), we have \( y \in V_\alpha \).
    So \( x \subseteq V_\alpha \) as required.
\end{proof}
The ordinals can be viewed as the backbone of the universe of sets; each \( V_\alpha \) can be thought of as resting on the ordinal \( \alpha \).
\begin{remark}
    The \( V_\alpha \) are called the \emph{von Neumann hierarchy}.
    The above proof shows that for all \( x \), \( \mathrm{rank}(x) = \sup\qty{\mathrm{rank}(y) + 1 \mid y \in x} \).
    For example, the rank of \( \qty{\qty{2, 3}, 6} \) is
    \[ \sup\qty{\mathrm{rank}\qty{2, 3} + 1, 6 + 1} = \sup\qty{5, 7} = 7 \]
\end{remark}
