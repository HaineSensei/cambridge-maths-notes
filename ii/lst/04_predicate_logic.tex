\subsection{Languages}
Recall that a \emph{group} is a set \( A \) equipped with functions \( m \colon A^2 \to A \) of arity 2, and \( i \colon A^1 \to A \) of arity 1, and a constant \( e \in A \) which can be viewed as a function \( A^0 \to A \) of arity 0, such that a set of axioms hold.
A \emph{poset} is a set \( A \) equipped with a relation \( (\leq) \subseteq A^2 \) of arity 2, such that a set of axioms hold.
Other algebraic structures can be described in the same way.

Let \( \Omega \) and \( \Pi \) be disjoint sets of functions and relations, and \( \alpha \colon \Omega \cup \Pi \to \mathbb N \) be an arity function.
\emph{Variables} are symbols of the form \( x_1, x_2, \dots \).
\emph{Terms} are defined inductively by
\begin{enumerate}
    \item each variable is a term;
    \item if \( f \in \Omega \) with \( \alpha(f) = n \) and terms \( t_1, \dots, t_n \), then \( f\ t_1\dots\ t_n \) is a term.
\end{enumerate}
The \emph{atomic formulae} are defined inductively by
\begin{enumerate}
    \item \( \bot \) is an atomic formula;
    \item for terms \( s, t \), \( (s = t) \) is an atomic formula;
    \item if \( \varphi \in \Pi \) with \( \alpha(\varphi) = n \) and terms \( t_1, \dots, t_n \), then \( \varphi(t_1, \dots, t_n) \) is an atomic formula.
\end{enumerate}
The \emph{formulae} are defined inductively by
\begin{enumerate}
    \item each atomic formula is a formula;
    \item if \( p \) and \( q \) are formulae then \( (p \Rightarrow q) \) is a formula;
    \item if \( p \) is a formula and \( x \) is a variable, then \( (\forall x) p \) is a formula.
\end{enumerate}
The \emph{language} \( L = L(\Omega, \Pi, \alpha) \) is the set of formulae.
\begin{example}
    In the language of groups, \( \Omega = \qty{m, i, e} \) and \( \Pi = \varnothing \) with \( \alpha(m) = 2, \alpha(i) = 1, \alpha(e) = 0 \).
    \( m(x_1, x_2), m(x_1, i(x_2)), e, m(e, e) \) are examples of terms of the language.
    \( e = m(\ell, e), m(x,y) = m(y,x) \) are atomic formulae.
\end{example}
\begin{example}
    In the language of posets, \( \Omega = \varnothing \) and \( \Pi = \qty{\leq} \) with \( \alpha(\leq) = 2 \).
    \( x = y, x \leq y \) are atomic formulae.
    Technically, \( x \leq y \) is written \( \leq(x, y) \).
\end{example}
\begin{example}
    In the language of groups, \( (\forall x) (m(x,x) = e) \) is a formula.
    Another formula is \( m(x,x) = e \Rightarrow (\exists y) (m(y,y) = x) \).
\end{example}
\begin{remark}
    A formula is a certain finite string of symbols; it has no intrinsic semantics.
    We define \( \neg p, p \wedge q, p \vee q \) in the usual way.
    We define \( (\exists x) p \) to mean \( \neg(\forall x) (\neg p) \).
\end{remark}

\subsection{Variables}
A term is \emph{closed} if it contains no variables.
For example, \( e, m(e,i(e)) \) are closed in the language of groups, but \( m(x,i(x)) \) is not closed.

An occurrence of a variable \( x \) in a formula \( p \) is \emph{bound} if it is inside the brackets of a \( (\forall x) \) quantifier.
Otherwise, we say the occurrence is \emph{free}.
In the formula \( (\forall x)(m(x,x) = e) \), each occurrence of \( x \) is bound.
In \( m(x,x) = e \Rightarrow (\exists y)(m(y,y) = x) \), the occurrences of \( x \) are free and the occurrences of \( y \) are bound.
In the formula \( m(x,x) = e \Rightarrow (\forall x)(\forall y)(m(x,y) = m(y,x)) \), the occurrences of \( x \) on the left hand side are free, and the occurrences of \( x \) on the right hand side are bound.

A \emph{sentence} is a formula with no free variables.
\( (\forall x)(m(x,x) = e) \) is a sentence, and \( (\forall x)(m(x,x) \Rightarrow (\exists y)(m(y,y) = x)) \) is a sentence.
In the language of posets, \( (\forall x)(\exists y)(x \geq y \and \neg(x = y)) \) is a sentence.

For a formula \( p \), term \( t \), and variable \( x \), the \emph{substitution} \( p[t/x] \) is obtained from \( p \) by replacing every free occurrence of \( x \) with \( t \).
If \( p = (\exists y)(m(y,y) = x) \), \( p[e/x] = (\exists y)(m(y,y) = e) \).

\subsection{Semantic implication}
\begin{definition}
    Let \( L = L(\Omega, \Pi, \alpha) \) be a language.
    An \emph{\( L \)-structure} is
    \begin{itemize}
        \item a nonempty set \( A \);
        \item for each \( f \in \Omega \), a function \( f_A \colon A^n \to A \) where \( n = \alpha(f) \);
        \item for each \( \varphi \in \Pi \), a subset \( \varphi_A \subseteq A^n \) where \( n = \alpha(\varphi) \).
    \end{itemize}
\end{definition}
\begin{remark}
    We will see later why the restriction that \( A \) is nonempty is given here.
\end{remark}
\begin{example}
    In the language of groups, an \( L \)-structure is a nonempty set \( A \) with functions \( m_A \colon A^2 \to A, i_A \colon A \to A, e_A \in A \).
    Such a structure may not be a group, as we have not placed any axioms on \( A \).
\end{example}
\begin{example}
    In the language of posets, an \( L \)-structure is a nonempty set \( A \) with a relation \( (\leq_A) \subseteq A^2 \).
\end{example}
