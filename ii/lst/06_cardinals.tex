\subsection{Definitions}
We will study the possible sizes of sets in \( \mathsf{ZFC} \).
Write \( x \leftrightarrow y \) if there exists a bijection from \( x \) to \( y \); we wish to define \( \mathrm{card}(x) = \abs{x} \) such that \( x \leftrightarrow y \) if and only if \( \mathrm{card}(x) = \mathrm{card}(y) \).
This cannot be formulated as an equivalence class, due to Russell's paradox.
However, for any \( x \), there exists an ordinal \( \alpha \) such that \( x \leftrightarrow \alpha \) by the well-ordering theorem.
Hence, we can define \( \mathrm{card}(x) \) to be the least ordinal that \( x \) bijects with.
We say that a set \( m \) is a \emph{cardinality} or a \emph{cardinal} if \( m = \mathrm{card}(x) \) for some set \( x \).

If we were studying sets in \( \mathsf{ZF} \) and not \( \mathsf{ZFC} \), there may not be an ordinal that bijects with a given set \( x \).
However, we can apply \emph{Scott's trick}, which is as follows.
We can consider the least \( \alpha \) such that there exists \( y \leftrightarrow x \) with \( \mathrm{rank}(y) = \alpha \).
This is often called the \emph{essential rank} of \( x \).
In this case, we let \( \mathrm{card}(x) \) be the set \( \qty{y \subseteq V_\alpha \mid y \leftrightarrow x} \).

\subsection{The hierarchy of alephs}
An ordinal is \emph{initial} if it does not biject with any smaller ordinal.
Any finite ordinal is initial, and \( \omega, \omega_1 \) are initial.
For any set \( x \), \( \gamma(x) \) is initial.
\( \omega^2 \) is not initial as it bijects with \( \omega \).
We define \( \omega_\alpha \) for each ordinal \( \alpha \) by recursion.
\begin{itemize}
    \item \( \omega_0 = \omega \);
    \item \( \omega_{\alpha + 1} = \gamma(\omega_\alpha) \);
    \item \( \omega_\lambda = \sup\qty{\omega_\alpha \mid \alpha < \lambda} \) for a nonzero limit ordinal \( \lambda \).
\end{itemize}
Each of these ordinals is initial, and every initial ordinal \( \beta \) is of the form \( \omega_\alpha \).
Indeed, the \( \omega_\alpha \) are unbounded, as \( \omega_\alpha \geq \alpha \) for each \( \alpha \) by induction, so there exists a least ordinal \( \delta \) such that \( \beta < \omega_\delta \).
\( \delta \) must be a successor, otherwise \( \omega_\delta = \sup\qty{\omega_\alpha \mid \alpha < \beta} \), contradicting the definition of \( \delta \).
So \( \delta = \alpha + 1 \), so \( \omega_\alpha \leq \beta < \omega_{\alpha+1} \).
Hence \( \beta = \omega_\alpha \), otherwise we contradict \( \omega_{\alpha+1} = \gamma(\omega_\alpha) \).

Since we have potentially different definitions of cardinals, we will write \( \aleph_\alpha \) for \( \mathrm{card}(\omega_\alpha) \) to avoid ambiguity.
The \( \aleph_\alpha \) are precisely the cardinalities of the infinite sets.
In \( \mathsf{ZF} \) without \( \mathsf{AC} \), the \( \aleph_\alpha \) are the cardinalities of the well-orderable sets.

For cardinals \( m, n \), we write \( m \leq n \) if there exists an injection from \( M \) to \( N \) where \( \mathrm{card}(M) = m, \mathrm{card}(N) = n \).
Similarly, we write \( m < n \) if \( m \leq n \) and \( m \neq n \).
For example, \( \mathrm{card}(\omega) < \mathrm{card}(\mathcal P(\omega)) \).
By the Schr\"oder--Bernstein theorem, if \( m \leq n \) and \( n \leq m \), then \( m = n \).
Hence, \( \leq \) is a partial order on cardinals.
This is in fact a total order in \( \mathsf{ZFC} \), since we can well-order the two sets in question, and one injects into the other; alternatively, the \( \aleph \) numbers are clearly totally ordered.

\subsection{Cardinal arithmetic}
Let \( m, n \) be cardinals.
Then,
\begin{enumerate}
    \item \( m + n = \mathrm{card}\qty(M \amalg N) \);
    \item \( m \cdot n = \mathrm{card}(M \times N) \);
    \item \( m^n = \mathrm{card}(M^N) \);
\end{enumerate}
where \( m = \mathrm{card}(M), n = \mathrm{card}(N) \), and \( M^N \) is the set of functions \( N \to M \).
The choice of representatives \( M, N \) do not influence the result.
We can also define \( \sum_{i \in I} m_i = \mathrm{card}\qty(\coprod_{i \in I} M_i) \); this is only well-defined assuming the axiom of choice, as forming the bijection requires infinitely many choices.
\begin{example}
    \( \mathbb R, \mathcal P(\omega), \qty{0,1}^\omega \) biject.
    Hence, \( \mathrm{card}(\mathbb R) = \mathrm{card}(\mathcal P(\omega)) = 2^{\aleph_0} \).
    In particular, cardinal exponentiation and ordinal exponentiation do not coincide, as \( 2^\omega = \omega \).
    
    The cardinality of the set of sequences of reals is
    \[ \mathrm{card}(\mathbb R^\omega) = (2^{\aleph_0})^{\aleph_0} = 2^{\aleph_0 \cdot \aleph_0} = 2^{\aleph_0} \]
    Note that this statement requires that addition and multiplication are commutative, \( \aleph_0 \cdot \aleph_0 = \aleph_0 \) as \( \omega \times \omega \) bijects with \( \omega \), and that \( (m^n)^p = m^{np} \).
    The latter holds as \( (M^N)^P \) is the set of functions \( P \to (N \to M) \), and \( M^{N \times P} \) is the set of functions \( N \times P \to M \).
\end{example}
\begin{theorem}
    \( m^2 = m \) for all infinite cardinals \( m \).
\end{theorem}
\begin{proof}
    We show by induction that \( \aleph_\alpha^2 = \aleph_\alpha \) for all \( \alpha \) by induction.
    Define a well-ordering of \( \omega_\alpha \times \omega_\alpha \) by `going up in squares':
    \[ (x,y) < (z,w) \iff (\max(x,y) < \max(z,w)) \vee (\max(x,y) = \max(z,w) = \beta \wedge (y < \beta, z < \beta \vee x = z = \beta, y < w \vee y = w = \beta, x < z)) \]
    For any \( \delta \in \omega_\alpha \times \omega_\alpha \), \( \delta \in \beta \times \beta \) for some \( \beta < \omega_\alpha \), as \( \omega_\alpha \) is a limit ordinal.
    By induction, we can assume \( \beta \times \beta \) bijects with \( \beta \) (or \( \beta \) is finite).
    Hence, the initial segment \( I_\delta \) is contained in \( \beta \times \beta \) and hence has cardinality at most \( \mathrm{card}(\beta \times \beta) < \mathrm{card}(\omega_\alpha) \).
    
    Therefore, the well-ordering has order type at most \( \omega_\alpha \).
    Thus, \( \omega_\alpha \times \omega_\alpha \) injects into \( \omega_\alpha \), and the converse injection is trivial.
    So \( \omega_\alpha \times \omega_\alpha \) bijects with \( \omega_\alpha \).
\end{proof}
\begin{corollary}
    For any ordinals \( \alpha < \beta \), we have \( \aleph_\alpha + \aleph_\beta = \aleph_\alpha \cdot \aleph_\beta = \aleph_\beta \).
\end{corollary}
\begin{proof}
    \[ \aleph_\beta \leq \aleph_\alpha + \aleph_\beta \leq 2 \cdot \aleph_\beta \leq \aleph_\alpha \aleph_\beta \leq \aleph_\beta^2 = \aleph_\beta \]
\end{proof}
Hence, for example, \( X \amalg X \) bijects with \( X \) for any infinite set \( X \).

Cardinal exponentiation is not as simple as addition and multiplication.
For instance, in \( \mathsf{ZF} \), \( 2^{\aleph_0} \) need not even be an aleph number, for instance if \( \mathbb R \) is not well-orderable.
In \( \mathsf{ZFC} \), the statement \( 2^{\aleph_0} = \aleph_1 \) is independent of the axioms; this is called the \emph{continuum hypothesis}.
\( \mathsf{ZFC} \) does not even decide if \( 2^{\aleph_0} < 2^{\aleph_1} \).
Even today, not all implications about cardinal exponentiation (such as \( \aleph_\alpha^{\aleph_\beta} \)) are known.
