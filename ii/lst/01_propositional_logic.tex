\subsection{The language of propositional logic}
Let \( P \) be a set of \emph{primitive propositions}.
Unless otherwise stated, we let \( P = \qty{p_1, p_2, \dots} \).
The \emph{language} \( L = L(P) \) is defined inductively by
\begin{enumerate}
    \item if \( p \in P \), then \( p \in L \);
    \item \( \bot \in L \), where the symbol \( \bot \) is read `false';
    \item if \( p, q \in L \), then \( (p \implies q) \in L \).
\end{enumerate}
\begin{example}
    \( ((p_1 \implies p_2) \implies (p_1 \implies p_3)) \in L \).
    \( (p_4 \implies \bot) \in L \).
\end{example}
\begin{remark}
    Note that the elements of \( L \), called propositions, are just strings of symbols from the alphabet \( \qty{(, ), \implies, \bot, p_1, p_2, \dots} \).
    Brackets are only given for clarity; we omit those that are unnecessary, and may use other types of brackets such as square brackets.

    Note that the phrase `\( L \) is defined inductively' means more precisely the following.
    Let \( L_1 = P \cup \qty{\bot} \), and define \( L_{n+1} = L_n \cup \qty{(p \implies q) \mid p, q \in L_n} \).
    We set \( L = \bigcup_{n=1}^\infty L_n \).
    Note that the introduction rules for the language are injective and have disjoint ranges, so there is exactly one way in which any element of the language can be constructed using rules (i) to (iii).
\end{remark}
We can now introduce the abbreviations \( \neg, \wedge, \vee \) defined by
\[ \neg p = (p \implies \bot);\quad p \vee q = \neg p \implies q;\quad p \wedge q = \neg (p \implies \neg q) \]

\subsection{Semantic implication}
\begin{definition}
    A \emph{valuation} is a function \( v \colon L \to \qty{0,1} \) such that
    \begin{enumerate}
        \item \( v(\bot) = 0 \);
        \item \( v(p \implies q) = 0 \) if \( v(p) = 1 \) and \( v(q) = 0 \), and 1 otherwise.
    \end{enumerate}
\end{definition}
\begin{remark}
    On \( \qty{0,1} \), we can define the constant \( \bot = 0 \) and the operation \( \implies \) in the obvious way.
    Then, a valuation is precisely a mapping \( L \to \qty{0,1} \) preserving all structure, so it can be considered a homomorphism.
\end{remark}
\begin{proposition}
    Let \( v, v' \colon L \to \qty{0,1} \) be valuations that agree on the primitives \( p_i \).
    Then \( v = v' \).
    Further, any function \( w \colon P \to \qty{0,1} \) extends to a valuation.
\end{proposition}
\begin{remark}
    This is analogous to the definition of a linear map by its action on the basis vectors.
\end{remark}
\begin{proof}
    Clearly, \( v, v' \) agree on \( L_1 \), the set of elements of the language of length 1.
    If \( v, v' \) agree at \( p, q \), then they agree at \( p \implies q \).
    So by induction, \( v, v' \) agree on \( L_n \) for all \( n \), and hence on \( L \).

    Let \( v(p) = w(p) \) for all \( p \in P \), and \( v(\bot) = 0 \) to obtain \( v \) on the set \( L_1 \).
    Assuming \( v \) is defined on \( p, q \) we can define it at \( p \implies q \) in the obvious way.
    This defines \( v \) on all of \( L \).
\end{proof}
\begin{example}
    Let \( v \) be the valuation with \( v(p_1) = v(p_3) = 1 \), and \( v(p_n) = 0 \) for all \( n \neq 1, 3 \).
    Then, \( v((p_1 \implies p_3) \implies p_2) = 0 \).
\end{example}
\begin{definition}
    A \emph{tautology} is \( t \in L \) such that \( v(t) = 1 \) for every valuation \( v \).
    We write \( \models t \).
\end{definition}
\begin{example}
    \( p \implies (q \implies p) \).
    \[ \begin{array}{cccc}
        v(p) & v(q) & v(q \implies p) & v(p \implies (q \implies p)) \\
        0 & 0 & 1 & 1 \\
        0 & 1 & 0 & 1 \\
        1 & 0 & 1 & 1 \\
        1 & 1 & 1 & 1
    \end{array} \]
    Since the right-hand column is always 1, \( \models p \implies (q \implies p) \).
\end{example}
\begin{example}
    \( \neg \neg p \implies p \), which expands to \( ((p \implies \bot) \implies \bot) \implies p \).
    \[ \begin{array}{cccc}
        v(p) & v(\neg p) & v(\neg \neg p) & v(\neg \neg p \implies p) \\
        0 & 1 & 0 & 1 \\
        1 & 0 & 1 & 1
    \end{array} \]
    Hence \( \models \neg \neg p \implies p \).
\end{example}
\begin{example}
    \( (p \implies (q \implies r)) \implies ((p \implies q) \implies (p \implies r)) \).
    Suppose this is not a tautology.
    Then we have a valuation \( v \) such that \( v(p \implies (q \implies r)) = 1 \) and \( v((p \implies q) \implies (p \implies r)) = 0 \).
    Hence, \( v(p \implies q) = 1, v(p \implies r) = 0 \), so \( v(p) = 1, v(r) = 0 \), giving \( v(q) = 1 \), but then \( v(p \implies (q \implies r)) = 0 \) contradicting the assumption.
\end{example}
\begin{definition}
    Let \( S \subseteq L \) and \( t \in L \).
    We say \( S \) \emph{entails} or \emph{semantically implies} \( t \), written \( S \models t \), if \( v(t) = 1 \) whenever \( v(s) = 1 \) for all \( s \in S \).
\end{definition}
\begin{example}
    Let \( S = \qty{p \implies q, q \implies r} \), and let \( t = p \implies r \).
    Suppose \( S \not\models t \), so there is a valuation \( v \) such that \( v(p \implies q) = 1, v(q \implies r) = 1, v(p \implies r) = 0 \).
    Then \( v(p) = 1, v(r) = 0 \), so \( v(q) = 1 \) and \( v(q) = 0 \).
\end{example}
\begin{definition}
    We say that \( v \) is a \emph{model} of \( S \) in \( L \) if \( v(s) = 1 \) for all \( s \in S \).
\end{definition}
Thus, \( S \models t \) is the statement that every model of \( S \) is also a model of \( t \).
\begin{remark}
    The notation \( \models t \) is equivalent to \( \varnothing \models t \).
\end{remark}

\subsection{Syntactic implication}
For a notion of proof, we require a system of axioms and deduction rules.
As axioms, we take (for any \( p, q, r \in L \)),
\begin{enumerate}
    \item \( p \implies (q \implies p) \);
    \item \( (p \implies (q \implies r)) \implies ((p \implies q) \implies (p \implies r)) \);
    \item \( ((p \implies \bot) \implies \bot) \implies p \).
\end{enumerate}
\begin{remark}
    Sometimes, these three axioms are considered axiom \emph{schemes}, since they are really a different axiom for each \( p, q, r \in L \).
    These are all tautologies.
\end{remark}
For deduction rules, we will have only the rule \emph{modus ponens}, that from \( p \) and \( p \implies q \) one can deduce \( q \).
\begin{definition}
    Let \( S \subseteq L \), \( t \in L \).
    We say \( S \) \emph{proves} or \emph{syntactically implies} \( t \), written \( S \vdash t \), if there exists a sequence \( t_1, \dots, t_n = t \) in \( L \) such that every \( t_i \) is either
    \begin{enumerate}
        \item an axiom;
        \item an element of \( S \); or
        \item \( q \), where \( t_j = p \) and \( t_k = p \implies q \) where \( j, k < i \).
    \end{enumerate}
    We say that \( S \) is the set of \emph{premises} or \emph{hypotheses}, and \( t \) is the \emph{conclusion}.
\end{definition}
\begin{example}
    We will show \( \qty{p \implies q, q \implies r} \vdash p \implies r \).
    \begin{enumerate}[1.]
        \item \( q \implies r \) (hypothesis)
        \item \( (q \implies r) \implies (p \implies (q \implies r)) \) (axiom 1)
        \item \( p \implies (q \implies r) \) (modus ponens on lines 1, 2)
        \item \( (p \implies (q \implies r)) \implies ((p \implies q) \implies (p \implies r)) \) (axiom 2)
        \item \( (p \implies q) \implies (p \implies r) \) (modus ponens on lines 3, 4)
        \item \( p \implies q \) (hypothesis)
        \item \( p \implies r \) (modus ponens on lines 5, 6)
    \end{enumerate}
\end{example}
\begin{definition}
    If \( \varnothing \vdash t \), we say \( t \) is a \emph{theorem}, written \( \vdash t \).
\end{definition}
\begin{example}
    \( \vdash p \implies p \).
    \begin{enumerate}[1.]
        \item \( (p \implies ((p \implies p) \implies p)) \implies ((p \implies (p \implies p)) \implies (p \implies p)) \) (axiom 2)
        \item \( p \implies ((p \implies p) \implies p) \) (axiom 1)
        \item \( (p \implies (p \implies p)) \implies (p \implies p) \) (modus ponens on lines 1, 2)
        \item \( p \implies (p \implies p) \) (axiom 1)
        \item \( p \implies p \) (modus ponens on lines 3, 4)
    \end{enumerate}
\end{example}

\subsection{Deduction theorem}
\begin{theorem}
    Let \( S \subseteq L \), and \( p, q \in L \).
    Then \( S \vdash (p \implies q) \) if and only if \( S \cup \qty{p} \vdash q \).
\end{theorem}
Intuitively, provability corresponds to the implication connective in \( L \).
\begin{proof}
    For the forward direction, given a proof of \( p \implies q \) from \( S \), add the line \( p \) by hypothesis and deduce \( q \) from modus ponens, to obtain a proof of \( q \) from \( S \cup \qty{p} \).

    Conversely, suppose we have a proof of \( q \) from \( S \cup \qty{p} \).
    Let \( t_1, \dots, t_n \) be the lines of the proof.
    We will prove that \( S \vdash (p \implies t_i) \) for all \( i \).
    \begin{itemize}
        \item If \( t_i \) is an axiom, we write \( t_i \) (axiom); \( t_i \implies (p \implies t_i) \) (axiom 1); \( p \implies t_i \) (modus ponens).
        \item If \( t_i \in S \), we write \( t_i \) (hypothesis); \( t_i \implies (p \implies t_i) \) (axiom 1); \( p \implies t_i \) (modus ponens).
        \item If \( t_i = p \), we write the proof of \( \vdash p \implies p \) given above.
        \item Suppose \( t_i \) is obtained by modus ponens from \( t_j \) and \( t_k = t_j \implies t_i \).
        We may assume by induction that \( S \vdash p \implies t_k \) and \( S \vdash p \implies (t_j \implies t_i) \).
        We write
        \begin{enumerate}[1.]
            \item \( (p \implies (t_j \implies t_i)) \implies ((p \implies t_j) \implies (p \implies t_i)) \) (axiom 2)
            \item \( (p \implies t_j) \implies (p \implies t_i) \) (modus ponens)
            \item \( p \implies t_i \) (modus ponens)
        \end{enumerate}
        giving \( S \vdash p \implies t_i \).
    \end{itemize}
\end{proof}
\begin{example}
    Consider \( \qty{p \implies q, q \implies r} \vdash p \implies r \).
    By the deduction theorem, it suffices to prove \( \qty{p \implies q, q \implies r, p} \vdash r \), which is obtained easily from modus ponens.
\end{example}

\subsection{Soundness}
We aim to show \( S \models t \) if and only if \( S \vdash t \).
The direction \( S \vdash t \) implies \( S \models t \) is called \emph{soundness}, which is a way of verifying that our axioms and deduction rule make sense.
The direction \( S \models t \) implies \( S \vdash t \) is called \emph{adequacy}, which states that our axioms are powerful enough to deduce everything that is (semantically) true.
\begin{proposition}
    Let \( S \subseteq L \) and \( t \in L \).
    Then \( S \vdash t \) implies \( S \models t \).
\end{proposition}
\begin{proof}
    We have a proof \( t_1, \dots, t_n \) of \( t \) from \( S \).
    We aim to show that any model of \( S \) is also a model of \( t \), so if \( v \) is a valuation that maps every element of \( S \) to 1, then \( v(t) = 1 \).
    We show this by induction on the length of the proof.
    \( v(p) = 1 \) for each axiom \( p \) and for each \( p \in S \).
    Further, \( v(t_i) = 1, v(t_i \implies t_j) = 1 \), then \( v(t_j) = 1 \).
    Therefore, \( v(t_i) = 1 \) for all \( i \).
\end{proof}

\subsection{Adequacy}
Consider the case of adequacy where \( t = \bot \).
If our axioms are adequate, \( S \models \bot \) implies \( S \vdash \bot \), so \( S \not\vdash \bot \).
We say \( S \) is \emph{consistent} if \( S \not\vdash \bot \).
Therefore, in an adequate system, if \( S \) has no models then \( S \) is inconsistent; equivalently, if \( S \) is consistent then it has a model.

In fact, the statement that consistent axiom sets have a model implies adequacy in general.
Indeed, if \( S \models t \), then \( S \cup \qty{\neg t} \) has no models, and so it is inconsistent by assumption.
Then \( S \cup \qty{\neg t} \vdash \bot \), so \( S \vdash \neg t \implies \bot \) by the deduction theorem, giving \( S \vdash t \) by axiom 3.

We aim to construct a model of \( S \) given that \( S \) is consistent.
Intuitively, we want to write
\[ v(t) = \begin{cases}
    1 & t \in S \\
    0 & t \not\in S
\end{cases} \]
but this does not work on the set \( S = \qty{p_1, p_1 \implies p_2} \) as it would evaluate \( p_2 \) to false.

We say a set \( S \subseteq L \) is \emph{deductively closed} if \( p \in S \) whenever \( S \vdash p \).
Any set \( S \) has a \emph{deductive closure}, which is the (deductively closed) set of statements \( \qty{t \in L \mid S \vdash t} \) that \( S \) proves.
If \( S \) is consistent, then the deductive closure is also consistent.
Computing the deductive closure before the valuation solves the problem for \( S = \qty{p_1, p_1 \implies p_2} \).
However, if a primitive proposition \( p \) is not in \( S \), but \( \neg p \) is also not in \( S \), this technique still does not work, as it would assign false to both \( p \) and \( \neg p \).
