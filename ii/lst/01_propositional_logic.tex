\subsection{The language of propositional logic}
Let \( P \) be a set of \emph{primitive propositions}.
Unless otherwise stated, we let \( P = \qty{p_1, p_2, \dots } \).
The \emph{language} \( L = L(P) \) is defined inductively by
\begin{enumerate}
    \item if \( p \in P \), then \( p \in L \);
    \item \( \bot \in L \), where the symbol \( \bot \) is read `false';
    \item if \( p, q \in L \), then \( (p \implies q) \in L \).
\end{enumerate}
\begin{example}
    \( ((p_1 \implies p_2) \implies (p_1 \implies p_3)) \in L \).
    \( (p_4 \implies \bot) \in L \).
\end{example}
\begin{remark}
    Note that the elements of \( L \), called propositions, are just strings of symbols from the alphabet \( \qty{(, ), \implies, \bot, p_1, p_2, \dots} \).
    Brackets are only given for clarity; we omit those that are unnecessary, and may use other types of brackets such as square brackets.

    Note that the phrase `\( L \) is defined inductively' means more precisely the following.
    Let \( L_1 = P \cup \qty{\bot} \), and define \( L_{n+1} = L_n \cup \qty{(p \implies q) \mid p, q \in L_n} \).
    We set \( L = \bigcup_{n=1}^\infty L_n \).
    Note that the introduction rules for the language are injective and have disjoint ranges, so there is exactly one way in which any element of the language can be constructed using rules (i) to (iii).
\end{remark}
We can now introduce the abbreviations \( \neg, \wedge, \vee \) defined by
\[ \neg p = (p \implies \bot);\quad p \vee q = \neg p \implies q;\quad p \wedge q = \neg (p \implies \neg q) \]
