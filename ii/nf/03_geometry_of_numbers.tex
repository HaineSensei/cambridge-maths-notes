\subsection{Imaginary quadratic fields}
Let \( L = \mathbb Q(\sqrt{-d}) \) where \( d \) is square-free and \( d < 0 \).
\( \mathcal O_L = \mathbb Z[\alpha] \) where \( \alpha = \frac{1}{2}(1+\sqrt{d}) \) or \( \alpha = \sqrt{d} \).
Choose a square root of \( d \) in \( \mathbb C \) to construct an embedding of \( \mathcal O_L \) into \( \mathbb C \).

Suppose \( \Lambda = \mathbb Z v_1 + \mathbb Z v_2 \subseteq \mathbb R^2 \) where \( \mathbb R^2 \) is equipped with the Euclidean norm, and \( v_1, v_2 \) are linearly independent over \( \mathbb R \).
Let \( A(\Lambda) \) be the area of the parallelogram generated by \( v_1 \) and \( v_2 \).
If \( v_i = a_i e_1 + b_i e_2 \), we have
\[ A(\Lambda) = \abs{\det \begin{pmatrix}
    a_1 & a_2 \\
    b_1 & b_2
\end{pmatrix}} \]
Minkowski's lemma is that a closed disk \( S \) around zero contains a nonzero point of \( \Lambda \) whenever the area of \( S \) is at least \( 4A(\Lambda) \).
More precisely, there exists \( \alpha \in \Lambda \) such that \( 0 < \abs{\alpha}^2 < \frac{4A(\Lambda)}{\pi} \).
Note that this condition depends only on the area of the parallelogram, not its shape.
This will be proven shortly.

We will apply this to \( \Lambda = \mathfrak a \trianglelefteq \mathcal O_L \) for \( L = \mathbb Q(\sqrt{-d}) \), \( d < 0 \) square-free.
Let \( \sqrt{d} \in \mathbb C \) be chosen with positive imaginary part to embed \( \mathcal O_L \) in \( \mathbb C \).
\begin{lemma}
    \begin{enumerate}
        \item if \( \alpha = a + b \sqrt{d} \in \mathcal O_L \), then \( \abs{\alpha}^2 = (a+b\sqrt{d})(a-b\sqrt{d}) = N(\alpha) \);
        \item \( A(\mathcal O_L) = \frac{1}{2}\sqrt{\abs{D_L}} \);
        \item \( A(\mathfrak a) = N(\mathfrak a) A(\mathcal O_L) \);
        \item \( A(\mathfrak a) = \frac{1}{2} \abs{\Delta(\alpha_1, \alpha_2)}^{\frac{1}{2}} \) where \( \alpha_1, \alpha_2 \) are an integral basis for \( \mathfrak a \).
    \end{enumerate}
\end{lemma}
\begin{proof}
    Part (i) is clear.
    (iv) implies (ii) and (iii).
    We will prove (iv) later in a more general setting, giving the justification for the coefficient \( \frac{1}{2} \).

    We now prove (ii) and (iii) manually, without appealing to (iv).
    For part (ii), \( \mathcal O_L \) has basis \( 1, \alpha \).
    Therefore, \( A(\mathcal O_L) = \frac{1}{2}\sqrt{d} \) or \( \sqrt{d} \), which is exactly \( \frac{1}{2} \sqrt{\abs{D_L}} \).
    Part (iii) is a variant of the fact that \( \Delta(\alpha_1, \dots, \alpha_n) = N(\mathfrak a)^2 D_L \).
\end{proof}
Minkowski's lemma implies that there exists \( \alpha \in \mathfrak a \) with \( N(\alpha) \leq \frac{4A(\mathfrak a)}{\pi} = N(\mathfrak a) c_L \) where \( c_L = \frac{2\sqrt{\abs{D_L}}}{\pi} \) is Minkowski's constant.
Since \( \alpha \in \mathfrak a \), \( (\alpha) \subseteq \mathfrak a \).
Hence \( (\alpha) = \mathfrak a \mathfrak b \) for some \( \mathfrak b \trianglelefteq \mathcal O_L \).
So \( N(\alpha) = N((\alpha)) = N(\mathfrak a) N(\mathfrak b) \), so \( N(\mathfrak b) \leq c_L \).

Recall that the class group of \( L \) is \( \faktor{I_L}{P_L} \), the quotient of fractional ideals over principal ideals.
Then, \( [\mathfrak b] = [\mathfrak a^{-1}] \in \mathrm{Cl}_L \).
Replacing \( \mathfrak a \) with \( \mathfrak a^{-1} \), we have shown that for all \( [\mathfrak a] \in \mathrm{Cl}_L \), there exists a representative \( \mathfrak b \) of \( [\mathfrak a] \) which is an ideal with \( N(\mathfrak b) \leq \frac{2\sqrt{\abs{D_L}}}{\pi} = c_L \).
But for all \( m \in \mathbb Z \), the number of ideals \( \mathfrak a \trianglelefteq \mathcal O_L \) with \( N(\mathfrak a) = m \) is finite; indeed, if \( N(\mathfrak a) = m \), then \( m \in \mathfrak a \) so \( \mathfrak a \mid (m) \), but there are only finitely many ideals dividing \( (m) \), as they biject with ideals in \( \faktor{\mathcal O_L}{m\mathcal O_L} \simeq \qty(\faktor{\mathbb Z}{m\mathbb Z})^n \).

Therefore, we have shown that \( \mathrm{Cl}_L \) is finite, and generated by the class of prime ideals dividing \( (p) \), for \( p \) a prime integer less than \( \frac{2\sqrt{\abs{D_L}}}{\pi} = c_L \).
Indeed, if \( \mathfrak a = \mathfrak p_1^{e_1} \dots \mathfrak p_r^{e_r} \) with \( N(\mathfrak a) < c_L \), then \( N(\mathfrak p_i) < c_L \).
\begin{example}
    Let \( d = -7 \).
    Then \( D_L = -7 \), and \( \frac{2\sqrt{7}}{\pi} < 2 \).
    So there are no primes \( p < c_L \), giving \( \mathrm{Cl}_L = \qty{1} \).
    In particular, \( \mathcal O_L \) is a unique factorisation domain.
    Similarly, \( d = -1, -2, -3 \) give unique factorisation domains.
\end{example}
\begin{example}
    Let \( d = -5 \).
    Here, \( D_L = -20 \), and \( 2 < \frac{4\sqrt{5}}{\pi} < 3 \).
    Hence, \( \mathrm{Cl}_L \) is generated by prime ideals dividing \( (2) \).
    Note that \( (2) = (2, 1 + \sqrt{-5})^2 \) by Dedekind's theorem.

    We now must check if \( (2, 1 + \sqrt{-5}) \) is principal.
    If \( (2, 1 + \sqrt{-5}) = (\beta) \), then \( N(\beta) = 2 \).
    But \( \beta = a + b\sqrt{-5} \), so \( N(\beta) = a^2 + 5 b^2 \), which is not satisfiable by integers.
    So \( (2, 1 + \sqrt{-5}) \) is principal but its square is, so \( \mathrm{Cl}_L = \faktor{\mathbb Z}{2\mathbb Z} \).
\end{example}
\begin{example}
    Let \( d = -17 \), then \( 5 < c_L < 6 \).
    \( \mathrm{Cl}_L \) is generated by prime ideals dividing \( (2), (3), (5) \).
    Modulo 2, \( x^2 + 17 = x^2 + 1 = (x + 1)^2 \), so \( (2) = \mathfrak p^2 \) where \( \mathfrak p = (2, 1 + \sqrt{-17}) \).
    Modulo 3, \( x^2 + 17 = x^2 - 1 = (x + 1)(x - 1) \), giving \( (3) = \mathfrak q \overline{\mathfrak q} \) where \( \mathfrak q = (3, 1 + \sqrt{-17}), \overline{\mathfrak q} = (3, 1 - \sqrt{-17}) \).
    Modulo 5, \( x^2 + 17 = x^2 + 2 \) which is irreducible, so \( (5) \) is inert, so is trivial in the class group.

    Hence \( \mathrm{Cl}_L = (\mathfrak p, \mathfrak q, \overline{\mathfrak q}) = (\mathfrak p, \mathfrak q) \).
    We could compute powers of \( \mathfrak p \) and \( \mathfrak q \) until we obtain all nontrivial relations between them.
    A more efficient way to compute \( \mathrm{Cl}_L \) in this case is to find principal ideals of small norm which are multiples of 2 and 3 to find the relations.
    Consider \( (1 + \sqrt{-17}) \), which has norm \( N(1 + \sqrt{-17}) = 18 = 2 \cdot 3^2 \).
    Note that \( 1 + \sqrt{-17} \in \mathfrak p \cap \mathfrak q \) so \( (1 + \sqrt{-17}) = \mathfrak p \mathfrak q \mathfrak r \) where \( \mathfrak r \in (\mathfrak p, \mathfrak q) \).
    We can show that \( \mathfrak r = \mathfrak q \).
    This shows that \( [\mathfrak p] = [\mathfrak q]^{-2} \) in \( \mathrm{Cl}_L \).
    So \( \mathrm{Cl}_L \) is generated by \( [\mathfrak q] \).
    So it is cyclic, and we can show \( [\mathfrak q]^2 \neq 1 \), as \( \mathfrak p \) is not principal, but \( [\mathfrak q]^4 = [\mathfrak p^2]^{-1} = 1 \).
    So \( \mathrm{Cl}_L = \faktor{\mathbb Z}{4\mathbb Z} \).
\end{example}
\begin{theorem}
    Let \( L = \mathbb Q(\sqrt{-d}) \) with \( d > 0 \).
    \begin{enumerate}
        \item \( \mathcal O_L \) is a unique factorisation domain if \( d \in \qty{1, 2, 3, 7, 11, 19, 43, 67, 163} \);
        \item there are no others.
    \end{enumerate}
\end{theorem}
\begin{definition}
    A subset \( X \subseteq \mathbb R^n \) if for all \( K \subseteq \mathbb R^n \) compact, \( K \cap X \) is finite.
    Equivalently, for all \( x \in X \) there exists \( \varepsilon > 0 \) with \( B_\varepsilon(x) \cap X = \qty{x} \).
\end{definition}
Recall that \( K \subseteq \mathbb R^n \) is compact if and only if it is closed and bounded.
\begin{proposition}
    Let \( \Lambda \subseteq \mathbb R^n \).
    Then the following are equivalent.
    \begin{enumerate}
        \item \( \Lambda \) is a discrete subgroup of \( (\mathbb R^n, +) \);
        \item \( \Lambda = \qty{\sum_{i=1}^m n_i x_i \mid n_i \in \mathbb Z} \) where \( x_1, \dots, x_m \) are linearly independent over \( \mathbb R \).
    \end{enumerate}
\end{proposition}
\begin{example}
    \( \mathbb Z\sqrt{2} + \mathbb Z\sqrt{3} \subseteq \mathbb R \) is not discrete.
    If \( \Lambda = \mathfrak a \subseteq \trianglelefteq O_L \) is an ideal where \( L = \mathbb Q(\sqrt{-d}) \) and \( d > 0 \), \( \Lambda \) is discrete.
\end{example}
\begin{proof}
    \emph{(ii) implies (i).}
    Observe that if \( g \in GL_n(\mathbb R) \), then \( g\Lambda \) is discrete if \( \Lambda \) is.
    \( g\Lambda \) satisfies (ii) if and only if \( \Lambda \) does.
    Suppose property (ii) holds, so \( \Lambda = \qty{\sum_{i=1}^m n_i x_i \mid n_i \in \mathbb Z} \).
    There exists \( g \in GL_n(\mathbb R) \) such that \( gx_i = e_i \) where the \( e_i \) form the standard basis of \( \mathbb R^n \).
    Clearly, \( \bigoplus_{i=1}^m \mathbb Z e_i \) is discrete.

    \emph{(i) implies (ii).}
    Let \( y_1, \dots, y_m \in \Lambda \) which are \( \mathbb R \)-linearly independent such that \( m \) is maximal.
    Note that \( m \leq n \).
    Also,
    \[ \qty{\sum_{i=1}^m \lambda_i y_i \mid \lambda_i \in \mathbb R} = \qty{\sum_{i=1}^N \lambda_\alpha z_\alpha \mid \lambda_\alpha \in \mathbb R, z_\alpha \in \Lambda, N \geq 0} \]
    This is the smallest \( \mathbb R \)-vector subspace of \( \mathbb R^n \) containing \( \Lambda \).
    Let \( X = \qty{\sum_{i=1}^m \lambda_i y_i \mid \lambda_i \in [0,1]} \).
    This is closed and bounded, hence compact.
    \( \Lambda \) is discrete, so \( X \cap \Lambda \) is finite.

    Consider the subgroup \( \mathbb Z^m = \bigoplus_{i=1}^m \mathbb Z y_i \subseteq \Lambda \).
    We can write \( \lambda \in \Lambda \) as \( \lambda = \lambda_0 + \lambda_1 \) where \( \lambda_0 \in X \cap \Lambda \) is the integral part and \( \lambda_1 \in \mathbb Z^m = \bigoplus_{i=1}^m \mathbb Z y_i \) is the fractional part.
    Hence, \( \abs{\faktor{\Lambda}{\mathbb Z^m}} \leq \abs{X \cap \Lambda} \) is finite.
    Let \( d = \abs{\faktor{\Lambda}{\mathbb Z^m}} \), so by Lagrange's theorem, \( d = 0 \) in \( \faktor{\Lambda}{\mathbb Z^m} \), so \( d\Lambda \subseteq \mathbb Z^m \).
    In particular, \( \mathbb Z^m \subseteq \Lambda \subseteq \frac{1}{d} \mathbb Z^m \).
    The structure theorem for finitely generated abelian groups shows that there exist \( x_1, \dots, x_m \in \Lambda \) with \( \Lambda = \bigoplus_{i=1}^m \mathbb Z x_i \).
\end{proof}
\begin{definition}
    If \( \rank \Lambda = n \), so if \( n = m \), we say \( \Lambda \) is a \emph{lattice} in \( \mathbb R^n \).
\end{definition}
\begin{definition}
    Let \( \Lambda \subseteq \mathbb R^n \) be a lattice with basis \( x_1, \dots, x_n \).
    The \emph{fundamental parallelogram} is \( P = \qty{\sum_{i=1}^n \lambda_i x_i \mid \lambda_i \in [0,1]} \).
    The \emph{covolume} of \( \Lambda \) is the volume of \( P \), which is \( \abs{\det A} \) if \( x_i = \sum_{j=1}^n a_{ij} e_j \).
\end{definition}
Note that if \( x_1', \dots, x_n' \) are another basis of \( \Lambda \), the change of basis matrix \( B \) given by \( x_i' = \sum_{j=1}^m b_{ij} x_j \) has integer coefficients, so \( B \in GL_n(\mathbb Z) \), giving \( \det B = \pm 1 \).
Hence, the covolume is well-defined irrespective of the choice of basis.
Observe that \( P \) is a fundamental domain for the action of \( \Lambda \) on \( \mathbb R^n \); \( \mathbb R^n = \bigcup_{\gamma \in \Lambda} (\gamma + P) \) and \( (\gamma + P) \cap (\mu + P) \subseteq \partial P \) if \( \gamma \neq \mu \).
We can think of \( P \) as a set of coset representatives for \( \faktor{\mathbb R^n}{\Lambda} \), ignoring the boundary of \( P \); this can be justified by noting that \( \partial P \) has no volume.

\subsection{Minkowski's lemma}
\begin{theorem}
    Let \( \Lambda \subseteq \mathbb R^n \) be a lattice, and \( P \) be a fundamental parallelogram for it.
    Let \( S \subseteq \mathbb R^n \) be a measurable set.
    \begin{enumerate}
        \item If \( \operatorname{vol}(S) > \operatorname{covol}(\Lambda) \), there exist \( x, y \in S \) with \( x \neq y \) and \( x - y \in \Lambda \).
        \item Suppose \( s \in S \) if and only if \( -s \in S \), so \( S \) is \emph{symmetric around zero}, and that \( S \) is convex.
        Then, if
        \begin{enumerate}
            \item \( \operatorname{vol}(S) > 2^n \operatorname{covol}(\Lambda) \), or
            \item \( \operatorname{vol}(S) \geq 2^n \operatorname{covol}(\Lambda) \) and \( S \) is closed,
        \end{enumerate}
        then there exists \( \gamma \in S \cap \Lambda \) with \( \gamma \neq 0 \).
    \end{enumerate}
\end{theorem}
Note that this implies the result we used when \( n = 2 \).
In the case of the square lattice \( \Lambda = \mathbb Z^n \) and \( S = [-1,1]^n \), we can see that these bounds are sharp.
\begin{proof}
    \emph{Part (i).}
    Observe that \( \operatorname{vol}(S) = \sum_{\gamma \in \Lambda} \operatorname{vol}(S \cap (P + \gamma)) \) as \( P \) is a fundamental domain, volume is additive, and \( \operatorname{vol}(\partial (P + \gamma)) = 0 \).
    Note that \( \operatorname{vol}(S \cap (P + \gamma)) = \operatorname{vol}((S - \gamma) \cap P) \) as volume is translation invariant.
    We claim that the sets \( (S - \gamma) \cap P \) are not pairwise disjoint.
    Indeed, if they were, then \( \operatorname{vol}(P) \geq \sum_{\gamma \in \Lambda} \operatorname{vol}((S - \gamma) \cap P) = \operatorname{vol}(S) \) contradicting the assumption.
    Hence there exists \( \gamma \mu \in \Lambda \) with \( \gamma \neq \mu \) such that \( (S - \gamma) \cap P \) and \( (S - \mu) \cap P \) are not disjoint, so there exist \( x, y \in S \) with \( x - \gamma = y - \mu \), hence \( x - y \in \Lambda \).

    \emph{Part (ii)(a).}
    Let \( S' = \frac{1}{2} S = \qty{\frac{1}{2} s \mid s \in S} \).
    Then \( \operatorname{vol}(S') = 2^{-n} \operatorname{vol}(S) > \operatorname{covol}(\Lambda) \) by assumption.
    By part (i), there exist \( y, z \in S' \) with \( y - z \in \Lambda \setminus \qty{0} \).
    But \( y - z = \frac{1}{2}(2y + -2z) \).
    \( 2z \in S \) so \( -2z \in S \) as \( S \) is symmetric around zero.
    \( 2y \in S \), and \( S \) is convex, so \( y - z \in S \) as required.

    \emph{Part (ii)(b).}
    Apply part (ii)(a) to \( S_m = \qty(1 + \frac{1}{m}) S \) for all \( m \in \mathbb N, m > 0 \).
    We obtain \( \gamma_m \in S_m \cap \Lambda \) with \( \gamma_m \neq 0 \).
    By convexity of \( S \), \( S_m \subseteq S_1 \).
    So \( \gamma_1, \gamma_2, \dots \) are contained in \( S_1 \cap \Lambda \), which is a finite set as \( S_1 \) is closed and bounded (without loss of generality) and \( \Lambda \) is discrete.
    So there exists \( \gamma \in S_m \cap \Lambda \) such that \( \gamma_m = \gamma \) for infinitely many \( m \).
    Hence, \( \gamma \in \bigcap_{m > 0} S_m = S \) as \( S \) is closed.
    Therefore \( \gamma \in S \cap \Lambda \) with \( \gamma \neq 0 \).
\end{proof}
Let \( L \) be a number field and let \( n = [L:\mathbb Q] \).
Let \( \sigma_1, \dots, \sigma_r \colon L \to \mathbb R \) be the real embeddings, and \( \sigma_{r+1}, \dots, \sigma_{r+s}, \overline{\sigma_{r+1}}, \dots, \overline{\sigma_{r+s}} \colon L \to \mathbb C \) be the complex embeddings, where \( r + 2s = n \).
This gives an embedding
\[ (\sigma_1, \dots, \sigma_{r+s}) \colon L \hookrightarrow \mathbb R^r \times \mathbb C^s \xrightarrow{\simeq} \mathbb R^r \times \mathbb R^{2s} = \mathbb R^{r+2s} \]
In other words, we can write
\[ \sigma = (\sigma_1, \dots, \sigma_r, \Re \sigma_{r+1}, \Im \sigma_{r+1}, \dots, \Re\sigma_{r+s}, \Im\sigma_{r+s}) \]
\begin{lemma}
    \( \sigma(\mathcal O_L) \) is a lattice in \( \mathbb R^n \) of covolume \( 2^{-s} \abs{D_L}^{\frac{1}{2}} \).
    If \( \mathfrak a \trianglelefteq \mathcal O_L \) is an ideal, then \( \sigma(\mathfrak a) \) is a lattice, and \( \operatorname{covol}(\sigma(\mathfrak a)) = 2^{-s} \abs{D_L}^{\frac{1}{2}} N(\mathfrak a) \).
\end{lemma}
\begin{proof}
    The first part is a special case of the second part.
    Recall that \( \mathfrak a \) has an integral basis \( \gamma_1, \dots, \gamma_n \), and \( (\det (\sigma_i(\gamma_j)))^2 = \Delta(\gamma_1, \dots, \gamma_n) = N(\mathfrak a)^2 D_L \).
    Hence, \( \abs{\det(\sigma_i(\gamma_j))} = N(\mathfrak a) \abs{D_L}^{\frac{1}{2}} \).
    Note that if \( \sigma_{r+i}(\gamma) \overline{\sigma_{r+i}(\gamma)} = z\overline z \),
    \[ \begin{pmatrix}
        \Re z \\
        \Im z
    \end{pmatrix} = \begin{pmatrix}
        \frac{1}{2}(z + \overline z) \\
        \frac{1}{2i}(z - \overline z)
    \end{pmatrix} = \frac{1}{2}\begin{pmatrix}
        1 & 1 \\
        i & -i
    \end{pmatrix} \begin{pmatrix}
        z \\ \overline z
    \end{pmatrix} \]
    The determinant of the change of basis matrix is \( -\frac{1}{2} \).
\end{proof}
\begin{proposition}[Minkowski bound]
    Let \( \mathfrak a \trianglelefteq \mathcal O_L \).
    Then there exists \( \alpha \in \mathfrak a \) with \( \alpha \neq 0 \) and \( \abs{N(\alpha)} \leq c_L N(\mathfrak a) \) where \( c_L = \qty(\frac{4}{\pi})^s \frac{n!}{n^n} \abs{D_L}^{\frac{1}{2}} \).
\end{proposition}
\begin{proof}
    Let
    \[ B_{r,s}(t) = \qty{(y_1, \dots, y_r, z_1, \dots, z_s) \in \mathbb R^r \times \mathbb C^s \mid \sum \abs{y_i} + 2 \abs{z_i} \leq t} \]
    This set is closed and bounded, hence compact.
    It is also convex, symmetric around zero, and measurable with volume \( 2^r \qty(\frac{\pi}{2})^2 \frac{t^n}{n!} \).
    Choose \( t \) such that the volume of \( B_{r,s}(t) \) is \( 2^n \operatorname{covol}(\mathfrak a) \), so \( t^n = \qty(\frac{4}{\pi})^s n! \abs{D_L}^\frac{1}{2} N(\mathfrak a) \).
    Minkowski's lemma implies that there exists \( \alpha \in \mathfrak a \) and \( \alpha \neq 0 \) such that \( \sigma(\alpha) = (y_1, \dots, y_r, z_1, \dots, z_s) \in B_{r,s}(t) \).

    Note that \( N(\alpha) = y_1 \dots y_r z_1 \overline{z_1} \dots z_s \overline{z_s} = \prod y_i \prod \abs{z_j}^2 \).
    Since the geometric mean is at most the arithmetic mean, taking \( n \)th roots we obtain \( \abs{N(\alpha)}^{\frac{1}{n}} = \frac{1}{n}\qty(\sum y_i + 2 \sum \abs{z_j}) \leq \frac{t}{n} \) as \( \sigma(\alpha) \in B_{r,s}(t) \).
    So \( N(\alpha) \leq \frac{t^n}{n^n} = c_L N(\alpha) \) as required.
\end{proof}
To show that the volume of \( B_{r,s}(t) \) is \( 2^r \qty(\frac{\pi}{2})^2 \frac{t^n}{n!} \), we can use induction with base cases \( B_{1,0}(t) = [-t,t] \) and \( B_{0,1}(t) = \frac{\pi}{4} t^2 \).
Given the result for \( B_{r,s}(t) \), the volume of \( B_{r+1,s}(t) \) is
\[ \int_{-t}^t \operatorname{vol} B_{r,s}(t-\abs{y}) \dd{y} = 2\int_0^t \qty(\frac{\pi}{2})^s 2^r \frac{(ty)^n}{n!} \dd{y} = 2^{r+1} \qty(\frac{\pi}{2})^2 \frac{t^{n+1}}{n!} \]
The other inductive step is on an example sheet.
\begin{corollary}
    Every element of the class group \( [\mathfrak a] \) has a representative \( \mathfrak a \trianglelefteq \mathcal O_L \) with norm at most \( c_L \).
\end{corollary}
\begin{theorem}
    The class group of \( L \) is finite, and generated by prime ideals \( \mathfrak a \trianglelefteq \mathcal O_L \) with \( N(\mathfrak a) \leq c_L \).
\end{theorem}
\begin{proof}
    Follows the argument used for imaginary quadratic fields.
\end{proof}
\begin{theorem}[Hermite, Minkowski]
    Let \( n \geq 2 \).
    Then \( \abs{D_L} \geq \frac{\pi}{3} \qty(\frac{3\pi}{4})^{n-1} > 1 \).
    In particular, \( \abs{D_L} > 1 \), so at least one prime ramifies in \( L \).
\end{theorem}
\begin{proof}
    Apply this to \( [\mathcal O_L] \) and obtain an ideal \( \mathfrak a \trianglelefteq \mathcal O_L \) with \( 1 \leq N(\mathfrak a) \leq c_L \), so \( c_L \geq 1 \).
    So
    \[ \abs{D_L}^{\frac{1}{2}} \geq \qty(\frac{\pi}{4})^s \frac{n^n}{n!} \geq \qty(\frac{\pi}{4})^{\frac{n}{2}} \frac{n^n}{n!} a_n^{\frac{1}{2}} \]
    as \( \frac{\pi}{4} < 1 \) and \( s \leq \frac{n}{2} \).
    So \( a_2 = \frac{\pi^2}{4} \) and \( \frac{a_{n+1}}{a_n} = \frac{\pi}{4} \qty(1 + \frac{1}{n})^{2n} > \frac{\pi}{4}(1 + 2) = \frac{3\pi}{4} \).
    So \( a_n \geq \frac{\pi^2}{4} \qty(\frac{3\pi}{4})^{n-2} = \frac{\pi}{3} \qty(\frac{3\pi}{4})^{n-1} \).
\end{proof}
