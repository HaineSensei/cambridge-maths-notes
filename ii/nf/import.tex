\chapter[Number Fields \\ \textnormal{\emph{Lectured in Lent \oldstylenums{2023} by \textsc{Prof.\ I.\ Grojnowski}}}]{Number Fields}
\emph{\Large Lectured in Lent \oldstylenums{2023} by \textsc{Prof.\ I.\ Grojnowski}}

A number field is a field extension of \( \mathbb Q \), generated by finitely many algebraic numbers.
Common number fields include the rationals \( \mathbb Q \), the Gaussian rationals \( \mathbb Q(i) \), and more general quadratic fields \( \mathbb Q(\sqrt{d}) \).
The field of rationals contain the ring of integers \( \mathbb Z \), and each number field similarly contains its own ring of algebraic integers.
For example, the ring of algebraic integers in \( \mathbb Q(i) \) is \( \mathbb Z[i] \).

While the field structure of a number field is typically relatively simple, the ring structure of the algebraic integers can offer more insight into the field in question.
In general, the ring of algebraic integers is not a unique factorisation domain, but some fields have `better' or `worse' factorisation properties than others.
We quantify the degree to which unique factorisation fails by assigning a group to each number field, called the class group.
If the class group is large, there are many ways in which unique factorisation could fail.
A remarkable fact proven in this course is that the class group is finite.

We also study the units in the ring of algebraic integers.
The roots of unity in a number field are units, but there may be other units not of this form.
Dirichlet's unit theorem describes a geometric interpretation of the set of units.
In particular, modulo the roots of unity, they form a lattice isomorphic to \( \mathbb Z^k \) for some \( k \in \qty{0, 1, \dots} \).
We can use this theorem to find all of the integer solutions of problems like Pell's equation, \( x^2 - d y^2 = \pm 1 \).

\subfile{../../ii/nf/main.tex}
