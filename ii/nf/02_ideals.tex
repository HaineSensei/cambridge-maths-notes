\subsection{?}
\begin{lemma}
    Let \( x \in \mathcal O_L \), where \( L \) is a number field.
    Then \( x \) is a unit in \( \mathcal O_L \) if and only if \( N_{L/\mathbb Q}(x) = \pm 1 \).
    We write \( \mathcal O_L^\star \) for the set of units of \( \mathcal O_L \).
\end{lemma}
\begin{proof}
    If \( x \) is a unit, then as the norm is multiplicative, \( N(x x^{-1}) = 1 \) so \( N(x)N(x^{-1}) = 1 \).
    So \( N(x) = \pm 1 \).
    Conversely, let \( \sigma_1, \dots, \sigma_n \colon L \to \mathbb C \) be the distinct field embeddings.
    Let \( L \subseteq \mathbb C \) be the containment given by \( \sigma_1 \).
    If \( x \in \mathcal O_L \), then \( N(x) = x\sigma_2(x) \dots \sigma_n(x) \).
    So if \( N(x) = \pm 1 \), we have \( \frac{1}{x} = \pm \prod_{i=2}^n \sigma_i(x) \).
    This is a product of algebraic integers, hence an algebraic integer.
    So \( x^{-1} \in \mathcal O_L \).
\end{proof}
Recall that if \( x \in \mathcal O_L \), it is irreducible if it does not factorise as \( ab \) where \( a, b \in \mathcal O_L \) not units.
If \( x = uy \) where \( u \) is a unit, we say \( x \) and \( y \) are associate.
Many number fields have rings of algebraic integers which are not unique factorisation domains.
\begin{example}
    Let \( L = \mathbb Q\qty(\sqrt{-5}) \).
    Here, \( \mathcal O_L = \mathbb Z\qty[\sqrt{-5}] \).
    Note that \( 3 \cdot 7 = \qty(1 + 2\sqrt{-5})\qty(1 - 2\sqrt{-5}) \), and \( N(3) = 9, N(7) = 49, N\qty(1 \pm \sqrt{-5}) = 21 \).
    These are not associates.
    We claim that \( 3, 7, 1 \pm 2 \sqrt{-5} \) are irreducible, so \( \mathcal O_L \) is not a unique factorisation domain.
    If this were not the case, \( 3 = \alpha \overline \alpha \), where \( \alpha = x + y\sqrt{-5} \), but \( N(3) = 9 = N(\alpha) N(\overline \alpha) = N(\alpha)^2 \) so \( N(\alpha) = x^2 + 5y^2 = \pm 3 \), but there are no integer solutions to this equation.
    All of the other factors are similarly irreducible.
\end{example}
\begin{remark}
    In any number field, one can factorise any \( \alpha \in \mathcal O_L \) into a product of irreducibles by induction on \( \abs{N(\alpha)} \), but this factorisation is not in general unique.
    An idea due to Kummer is to measure the failure of unique factorisation by studying ideals \( \mathfrak a \triangleleft \mathcal O_L \).
\end{remark}
If \( x_1, \dots, x_n \in \mathcal O_L \), we write \( \genset{x_1, \dots, x_n} \) for the ideal \( \sum x_i \mathcal O_L \) generated by the \( x_i \).
We will consider products of ideals, rather than products of elements.
\begin{definition}
    If \( \mathfrak a, \mathfrak b \triangleleft \mathcal O_L \), define \( \mathfrak a \mathfrak b = \qty{\sum a_i b_i \mid a_i \in \mathfrak a, b_i \in \mathfrak b} \).
\end{definition}
One can check that this is an ideal, and that products are associative.
\begin{example}
    \( \genset{x_1, \dots, x_n} \genset{y_1, \dots, y_m} = \genset{\qty{x_i y_j \mid 1 \leq i \leq n, 1 \leq j \leq n}} \).
    For instance, \( \genset{x}\genset{y} = \genset{xy} \), so the product of principal ideals is principal.
\end{example}
\begin{example}
    Consider \( \mathbb Z[\sqrt{5}] = \mathcal O_L \), and the ideals \( \mathfrak p_1 = (3, 1 + 2\sqrt{5}), \mathfrak p_2 = (3, 1-2\sqrt{5}) \).
    We obtain \( \mathfrak p_1 \mathfrak p_2 = (9, 3(1-2\sqrt{5}), 3(1+2\sqrt{5}), 21) = (3) \).
    So the ideal \( (3) \) factors as \( \mathfrak p_1 \mathfrak p_2 \) in \( \mathcal O_L \).
    Note that \( 37 = (1 + 2\sqrt{-5})(1 - 2\sqrt{-5}) \), so \( \mathbb Z[\sqrt{5}] \) is not a unique factorisation domain.
\end{example}
Recall that an ideal \( \mathfrak p \triangleleft R \) is \emph{prime} if \( \frak{R}{\mathfrak p} \) is an integral domain, so \( p \neq R \) and for all \( x, y \in R \), \( xy \in \mathfrak p \) implies \( x \in \mathfrak p \) or \( y \in \mathfrak p \).
In this course, we will also define that a prime ideal is nonzero.

\subsection{?}
\begin{lemma}
    If \( \mathfrak a \triangleleft \mathcal O_K \), it contains an integer, and moreover, \( \faktor{\mathcal O_K}{\mathfrak a} \) is a finite set.
\end{lemma}
\begin{proof}
    Let \( \alpha \in \mathfrak a, \alpha \neq 0 \).
    Let \( p_\alpha(x) = x^m + a_{m-1} x^{m-1} + \dots + a_0 \in \mathbb Z[x] \) be its minimal polynomial, and \( a_0 \neq 0 \).
    Then \( a_0 = -\alpha(\alpha^{n-1} + a_{n-1} \alpha^{n-1} + \dots + a_2 \alpha + a_1) \).
    But \( a_0 \in \mathbb Z \), \( \alpha \in \mathfrak a \), and the other factor lies in \( \mathcal O_K \).
    So \( a_0 \in \mathfrak a \) as \( \mathfrak a \) is an ideal.
    Hence \( a_0 \mathcal O_K \subseteq \mathfrak a \), so \( \faktor{\mathcal O_K}{a_0 \mathcal O_K} \) surjects onto \( \faktor{\mathcal O_K}{\mathfrak a} \).
    But for any integer \( d \), \( \faktor{\mathcal O_K}{d\mathcal O_K} = \faktor{\mathbb Z^n}{d\mathbb Z^n} = \qty(\faktor{\mathbb Z}{d\mathbb Z})^n \) is a finite set, so \( \faktor{\mathcal O_K}{\mathfrak a} \) is finite.
\end{proof}
\begin{corollary}
    \( \mathfrak a \simeq \mathbb Z^n \), as \( \mathcal O_K \simeq \mathbb Z^n \) and the quotient is finite.
\end{corollary}
Therefore, nonzero ideals in \( \mathcal O_K \) are isomorphic to \( \mathbb Z^n \) as abelian groups.
\begin{proposition}
    \begin{enumerate}
        \item \( \mathcal O_K \) is an integral domain.
        \item \( \mathcal O_K \) is a Noetherian ring.
        \item \( \mathcal O_K \) is \emph{integrally closed} in \( K \) (which is the field of fractions of \( \mathcal O_K \)): if \( x \in K \) is integral over \( \mathcal O_K \), it lies in \( \mathcal O_K \).
        \item Every (implicitly nonzero) prime ideal is maximal.
        We say that the \emph{Krull dimension} of \( \mathcal O_K \) is 1.
    \end{enumerate}
\end{proposition}
\begin{remark}
    A ring with these four properties is called a \emph{Dedekind domain}.
    Many of the results in this section hold for all Dedekind domains.
\end{remark}
\begin{proof}
    \emph{Part (i).}
    \( \mathcal O_K \subseteq K \), and \( K \) is a field.

    \emph{Part (ii).}
    We have shown that \( \mathcal O_K \simeq \mathbb Z^n \), where \( n = [K : \mathbb Q] \), so \( \mathcal O_K \) is finitely generated as an abelian group, so is certainly finitely generated as a ring.

    \emph{Part (iii).}
    \( \mathcal O_K \) is integral over \( \mathbb Z \) by definition, so if \( x \) is integral over \( \mathcal O_K \), it is integral over \( \mathbb Z \).
    So \( x \) is an algebraic integer, so lies in \( \mathcal O_K \).

    \emph{Part (iv).}
    If \( \mathfrak p \) is a prime ideal, then by the previous lemma \( \faktor{\mathcal O_K}{\mathfrak p} \) is finite and an integral domain, as \( \mathfrak p \) is prime.
    All finite integral domains are fields, hence \( \mathfrak p \) is maximal.
\end{proof}
\begin{example}
    Consider \( R = \mathbb C[X,Y] \).
    Then \( (x) \) is prime but not maximal, since \( (x) \subsetneq (x,y) \).
\end{example}
If \( \mathfrak a, \mathfrak b \) are ideals, we write \( \mathfrak a + \mathfrak b \) for the ideal \( \qty{x + y \mid x \in \mathfrak a, y \in \mathfrak b} \).

\subsection{Unique factorisation of ideals}
We aim to show that every ideal in \( \mathcal O_K \) factors uniquely as a product of prime ideals.
\begin{definition}
    \( \mathfrak b \) divides \( \mathfrak a \) if there exists an ideal \( \mathfrak c \) such that \( \mathfrak a = \mathfrak b \mathfrak c \).
    We write \( \mathfrak b \mid \mathfrak a \).
\end{definition}
\begin{example}
    \( (5, 1 + 2\sqrt{5}) \mid (3) \) in \( \mathcal O_{\mathbb Q(\sqrt{-5})} \).
    \( 3\mathbb Z \mid 6\mathbb Z \) as \( 3\mathbb Z \cdot 2\mathbb Z = 6\mathbb Z \).
\end{example}
Note that \( \mathfrak b \mathfrak c \subseteq \mathfrak b \), as \( \mathfrak b \) is an ideal.
So if \( \mathfrak b \mid \mathfrak a \), then \( \mathfrak a \subseteq \mathfrak b \).
We will show the converse, that \( \mathfrak a \subseteq \mathfrak b \) implies \( \mathfrak b \mid \mathfrak a \).
This allows us to prove results about division by using containment.
Note that prime ideals are maximal, which allows us to use the containment relation.
\begin{lemma}
    Let \( \mathfrak p \) be a prime ideal in a ring \( R \), and let \( \mathfrak a, \mathfrak b \triangleleft R \) be ideals.
    Then if \( \mathfrak ab \subseteq \mathfrak p \), either \( \mathfrak a \subseteq \mathfrak p \) or \( \mathfrak b \subseteq \mathfrak p \).
\end{lemma}
\begin{proof}
    Otherwise, there exists \( a \in \mathfrak a \setminus \mathfrak p \) and \( b \in \mathfrak b \setminus \mathfrak p \), with \( ab \in \mathfrak p \).
    But \( \mathfrak p \) is prime giving a contradiction.
\end{proof}
\begin{lemma}
    Let \( \mathfrak a \trianglelefteq \mathcal O_K \) be a nonzero ideal.
    Then \( \mathfrak a \) contains a product of prime ideals.
\end{lemma}
\begin{proof}
    Otherwise, as \( \mathcal O_K \) is Noetherian, there exists a ideal \( \mathfrak a \) which is maximal with this property.
    As prime ideals are maximal, \( \mathfrak a \) is not prime.
    So there exists \( x, y \in \mathcal O_K \) with \( x, y \not\in \mathfrak a \) but \( xy \in \mathfrak a \).
    So \( \mathfrak a \subsetneq \mathfrak a + (x) \).
    But then, \( \mathfrak a + (x) \) contains a product of prime ideals \( \mathfrak p_1, \dots, \mathfrak p_r \) with \( \mathfrak p_1\dots \mathfrak p_r \subseteq \mathfrak a + (x) \).
    Similarly,  there exist prime ideals \( \mathfrak q_1, \dots \mathfrak q_s \) such that \( \mathfrak q_1 \dots\mathfrak q_s \subseteq \mathfrak a + (y) \).
    Then,
    \[ \mathfrak p_1\dots \mathfrak p_r\mathfrak q_1 \dots\mathfrak q_s \subseteq (\mathfrak a + (x))(\mathfrak a + (y)) = \mathfrak a + (xy) \]
    But \( xy \in \mathfrak a \), giving a contradiction.
\end{proof}
The main proof will use the idea that we can formally introduce the group of fractions of the commutative monoid of ideals.
The object \( \qty{y \in K \mid y\mathfrak a \in \subseteq \mathcal O_K} \) will represent the inverse of \( \mathfrak a \).
\begin{lemma}
    \begin{enumerate}
        \item Let \( 0 \neq \mathfrak a \trianglelefteq \mathcal O_K \) be an ideal.
        If \( x \in K \) has the property that \( x\mathfrak a \subseteq \mathfrak a \), then \( x \in \mathcal O_K \).
        \item Let \( 0 \neq \mathfrak a \triangleleft \mathcal O_K \) be a proper ideal.
        Then, \( \mathcal O_K \subseteq \qty{y \in K \mid y\mathfrak a \in \subseteq \mathcal O_K} \) contains elements which are not in \( \mathcal O_K \).
        Equivalently, \( \faktor{\qty{y \in K \mid y\mathfrak a \in \subseteq \mathcal O_K}}{\mathcal O_K} \neq \qty{1} \) as abelian groups.
    \end{enumerate}
\end{lemma}
\begin{example}
    Let \( \mathcal O_K = \mathbb Z \) and \( \mathfrak a = 3\mathbb Z \).
    Then, part (i) shows that if \( \frac{a}{b} \cdot 3 \in \mathbb 3\mathbb Z \), then \( \frac{a}{b} \in \mathbb Z \).
    Part (ii) shows that if \( \frac{a}{b} \cdot 3 \in \mathbb Z \) then \( \frac{a}{b} \in \frac{1}{3}\mathbb Z \); for instance, if \( \frac{a}{b} = \frac{1}{3} \), we have \( \faktor{\frac{1}{3}\mathbb Z}{\mathbb Z} = \faktor{\mathbb Z}{3\mathbb Z} \neq \qty{1} \).
\end{example}
\begin{proof}
    \emph{Part (i).}
    \( \mathfrak a \subseteq \mathcal O_K \) is finitely generated as an abelian group, as it is isomorphic to \( \mathbb Z^n \).
    Let \( \alpha_1, \dots, \alpha_n \) be a \( \mathbb Z \)-basis of \( \mathfrak a \).
    Consider \( m_x \colon \mathfrak a \to \mathfrak a \) given by multiplication by \( x \in K \).
    We write \( x \alpha_i = \sum a_{ij} \alpha_j \), where by assumption, \( a_{ij} \) are integers.
    Hence,
    \[ (xI - A) \begin{pmatrix}
        \alpha_1 \\
        \vdots \\
        \alpha_n
    \end{pmatrix} = 0 \]
    where \( A = (a_{ij}) \).
    So \( \det(xI - A) = 0 \), so \( x \) is integral over \( \mathbb Z \); that is, \( x \in \mathcal O_K \).

    \emph{Part (ii).}
    If this holds for \( \mathfrak a \), it certainly holds for all prime ideals \( \mathfrak a' \subseteq \mathfrak a \).
    So without loss of generality, let \( \mathfrak a \) be maximal, so \( \mathfrak a = \mathfrak p \) is a prime ideal.
    Let \( \alpha \in \mathfrak p \) be nonzero.
    By the previous lemma, there exist prime ideals \( \mathfrak p_1, \dots, \mathfrak p_r \) such that \( \mathfrak p_1 \dots \mathfrak p_r \subseteq (\alpha) \subseteq \mathfrak p \).
    Suppose that \( r \) is minimal.
    By the first lemma in this subsection, there exists \( i \) such that \( \mathfrak p_i \subseteq \mathfrak p \), and without loss of generality \( i = 1 \).
    So \( \mathfrak p_1 \subseteq \mathfrak p \).
    But \( \mathfrak p_1 \) is maximal, so \( \mathfrak p_1 = \mathfrak p \).

    Since \( r \) is minimal, \( \mathfrak p_2 \dots \mathfrak p_r \subsetneq (\alpha) \).
    Fix \( \beta \in \mathfrak p_2 \dots \mathfrak p_r \not\subseteq (\alpha) \).
    Then \( \beta \mathfrak p \subseteq \mathfrak p (\mathfrak p_2 \dots \mathfrak p_r) \subseteq (\alpha) \), but \( \beta \not\subseteq (\alpha) \).
    So, dividing by \( \alpha \), we obtain \( \frac{\beta}{\alpha} \mathfrak p \subseteq (1) = \mathcal O_K \), but \( \frac{\beta}{\alpha} \not\in \mathcal O_K \).
\end{proof}
