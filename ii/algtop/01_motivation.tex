\subsection{Motivation}
Topological spaces are difficult to study on their own, and so we will assign algebraic invariants to these spaces which allow us to reason more easily about these spaces.
To a topological space \( X \), a `numerical invariant' is a number \( g(X) \in \mathbb R \cup \qty{\infty} \) such that \( X \simeq Y \) (where \( \simeq \) denotes homeomorphism) implies \( g(X) = g(Y) \).
An example of a numerical invariant is the number of path-connected components of \( X \).
An algebraic invariant is a group \( G(X) \) assigned to a topological space \( X \) such that \( X \simeq Y \) implies \( G(X) \simeq G(Y) \), where here \( \simeq \) denotes isomorphism.
We will construct two kinds of such invariants: the fundamental group, and invariants related to homology.
The invariants we construct will behave nicely under maps: if \( f \colon X \to Y \) is a continuous map, we induce a homomorphism \( f_\star \colon G(X) \to G(Y) \).
We will prove the following model theorems.
\begin{itemize}
	\item If \( \mathbb R^n \simeq \mathbb R^m \), then \( n = m \).
	\item If \( f \colon D^n \to D^n \) is continuous, then there exists \( x \in D^n \) with \( f(x) = x \).
\end{itemize}
The above theorems are easy to prove in the case \( n = 1 \) by appealing to path-connectedness and the intermediate value theorem.
Our study allows us to prove similar things about these higher dimensional cases, among other things.

\subsection{Notation}
\begin{itemize}
	\item A \emph{space} is a topological space.
	\item A \emph{map} is a continuous function, unless defined otherwise.
	\item If \( X \) and \( Y \) are spaces, \( X \simeq Y \) means that \( X \) and \( Y \) are homeomorphic.
	\item If \( G \) and \( H \) are groups, \( G \simeq H \) means that \( G \) and \( H \) are isomorphic.
	\item Some common spaces include:
	\begin{itemize}
		\item The one-point space \( \qty{\bullet} \);
		\item \( I = [0,1] \subset \mathbb R \);
		\item \( I^n = \underbrace{I \times \dots \times I}_{n\text{ times}} \), the \( n \)-dimensional closed unit cube;
		\item \( D^n = \qty{v \in \mathbb R^n \mid \norm{v} \leq 1} \), the \( n \)-dimensional closed unit disk (note that \( I^n \simeq D^n \));
		\item \( S^{n-1} = \qty{v \in \mathbb R^n \mid \norm{v} = 1} \), the \( (n-1) \)-dimensional unit sphere.
	\end{itemize}
	\item Common maps include:
	\begin{itemize}
		\item If \( X \) is a space, the identity map \( \mathrm{id}_X \colon X \to X \) is defined by \( x \mapsto x \);
		\item If \( X \) and \( Y \) are spaces with \( p \in Y \), the constant map \( c_{X,p} \colon X \to Y \) is defined by \( x \mapsto p \).
	\end{itemize}
\end{itemize}
% TODO: is the c map lowercase or uppercase?
