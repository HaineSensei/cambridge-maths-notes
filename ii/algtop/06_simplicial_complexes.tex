\subsection{Definitions}
We have shown that \( \pi_1(S_1,x_0) \simeq \mathbb Z \), and \( \pi_1(S^n,x_0) \simeq 1 \) for \( n > 1 \), so \( S_1 \not\sim S^n \).
We would like to show that \( S^n \sim S^m \) only holds if \( n = m \).
One proof of this fact is that any \( f \colon S^n \to S^m \) with \( n < m \) is null-homotopic, but the identity on \( S^m \) is not.
Both of these claims require proof: simplicial complexes will allow us to prove the first, and homology will allow us to prove the second.
\begin{definition}
	The \emph{\( n \)-simplex} is the topological space
	\[ \Delta^n  = \qty{(x_0, \dots, x_n) \in \mathbb R^{n+1} \midd x_i \geq 0, \sum_{i=0}^n x_i = 1 } \]
	with the subspace topology.
\end{definition}
\begin{remark}
	\( \Delta^1 \) is homeomorphic to \( I \).
	\( \Delta^2 \) is an equilateral triangle, and \( \Delta^3 \) is a regular tetrahedron.
	For all \( n \), \( \Delta^n \) is closed and bounded in \( \mathbb R^{n+1} \), and hence compact and Hausdorff.
	The standard basis vectors \( e_0, \dots, e_n \) are the vertices of \( \Delta^n \).
\end{remark}
\begin{definition}
	If \( I \subseteq \qty{0, \dots, n} \), the \emph{\( I \)th face of \( \Delta^n \)} is
	\[ e_I = \qty{x \in \Delta^n \mid x_i = 0 \text{ for } i \not\in I} \]
	We define \( F(\Delta^n) = \qty{e_I \mid I \subseteq \qty{0,\dots, n}} \) to be the set of faces of \( \Delta^n \).
\end{definition}
If \( I = \qty{i_0, \dots, i_k} \) with \( i_0 < \dots < i_k \), we write \( I = i_0i_1\dots i_k \).
\begin{remark}
	Note that \( e_{\qty{i}} = e_i \), and \( \Delta^n = e_{\qty{0,1, \dots, n}} \).
	\( e_I \) is a closed subset of \( \Delta^n \), and is homeomorphic to \( \Delta^{\abs{I} - 1} \).
	\( e_I \subseteq e_J \) if and only if \( I \subseteq J \).
	\( e_I \cap e_J = e_{I \cap J} \).
\end{remark}
\begin{definition}
	A map \( \abs{f} \colon \Delta^n \to \mathbb R^N \) is \emph{affine linear} if it is the restriction of a linear map \( \mathbb R^{n+1} \to \mathbb R^n \).
	Equivalently, \( \abs{f}\qty(\sum_{i=0}^n x_i e_i) = \sum_{i=0}^n x_i \abs{f}(e_i) \).
	We say an affine linear map \( \abs{f} \colon \Delta^n \to \Delta^m \) is \emph{simplicial} if it maps vertices in \( \Delta^n \) to vertices in \( \Delta^m \), so there is a map of sets \( \hat f \colon \qty{0,\dots, n} \to \qty{0, \dots, m} \) where \( \abs{f}(e_i) = e_{\hat f(i)} \).
\end{definition}
\begin{remark}
	Affine linear maps are continuous, and are determined entirely by their action on \( e_i \).
	In particular, simplicial maps \( \abs{f} \) are determined by \( \hat f \).
	For \( I \subseteq \qty{0, \dots, n} \), we have \( \abs{f}(e_I) = e_{\hat f(I)} \).
\end{remark}
\begin{definition}
	Vectors \( v_0, \dots, v_n \in \mathbb R^N \) are \emph{affine linearly independent} if whenever \( \sum t_i v_i = 0 \) and \( \sum t_i = 0 \), we have \( t_i = 0 \) for all \( i \).
	Equivalently,
	\begin{enumerate}
		\item If \( \sum t_i v_i = \sum t_i' v_i' \) and \( \sum t_i = \sum t_i' \), then for each \( i \), \( t_i = t_i' \).
		\item The vectors \( v_1 - v_0, v_2 - v_0, \dots, v_n - v_0 \) are linearly independent.
		\item The unique affine linear map \( \abs{f} \colon \Delta^n \to \mathbb R^N \) given by \( \abs{f}(e_i) = v_i \) is injective.
	\end{enumerate}
	If \( v_0, \dots, v_n \) are affine linearly independent, we write \( [v_0, \dots, v_n] = \Im \abs{f} = \qty{\sum x_i v_i \mid \sum x_i = 1, x_i \geq 0} \), and we say \( [v_0, \dots, v_n] \) is a \emph{Euclidean simplex}.
\end{definition}
\begin{remark}
	\( \Delta^n \) is compact and Hausdorff, so \( \abs{f} \colon \Delta^n \to [v_0, \dots, v_n] \) is a homeomorphism if the \( v_i \) are affine linearly independent.
\end{remark}
\begin{lemma}
	If \( X \subseteq \mathbb R^N \), let \( Z(X) \) be the set of \( x \in X \) such that if \( x = \sum t_i x_i \) for \( t_i > 0, \sum t_i = 1 \) and all \( x_i \in X \), then \( x_i = x \).
	Then \( Z([v_0, \dots, v_n]) = \qty{v_0, \dots, v_n} \).
\end{lemma}
\begin{proof}
	We show that \( v_k \in Z([v_0, \dots, v_n]) \); the converse is clear from the definition of the simplex.
	Suppose \( v_k = \sum t_i x_i \) for \( t_i > 0 \) and \( \sum t_i = 1 \).
	Then \( x_i = \sum_{j=0}^n s_{ij} v_j \), since \( x_i \in [v_0, \dots, v_n] \).
	So \( v_k = \sum_j \qty(\sum_i t_i s_{ij}) v_j \).
	Since the \( v_i \) are affine linearly independent, and \( \sum_j \qty(\sum_i t_i s_{ij}) = 1 \), we must have \( \sum t_i s_{ij} = 0 \) for \( j \neq k \).
	But \( t_i > 0 \) and \( s_{ij} \geq 0 \), so the only case is when all \( s_{ij} \) are exactly zero for \( j \neq k \), so \( x_j = v_k \).
\end{proof}
\begin{corollary}
	If \( [v_0, \dots, v_n] = [v_0', \dots, v_n'] \) as subsets of \( \mathbb R^N \), then \( \qty{v_0, \dots, v_n} = \qty{v_0', \dots, v_n'} \) as sets.
\end{corollary}
Therefore, a simplex determines its set of vertices.
\begin{proof}
	\( \qty{v_0, \dots, v_n} = Z([v_0, \dots, v_n]) = Z([v_0', \dots, v_n']) = \qty{v_0', \dots, v_n'} \).
\end{proof}
\begin{definition}
	\( \mathcal S(\mathbb R^n) \) is the set of Euclidean simplices \( \sigma \subseteq \mathbb R^n \).
	Hence, \( \mathcal S(\mathbb R^n) \) is in bijection with the set \( \qty{\qty{v_0, \dots, v_k} \mid v_i \in \mathbb R^N, k \geq -1, v_i \text{ affine linearly independent}} \).
\end{definition}

\subsection{Abstract simplicial complexes}
\begin{definition}
	An \emph{abstract simplicial complex} in \( \Delta^n \) is a subset \( K \) of the faces \( F(\Delta^n) \) such that \( e_I \in K \) whenever \( e_J \) is in \( K \) and \( I \subseteq J \).
\end{definition}
\begin{remark}
	Abstract simplicial complexes are downward-closed sets of faces.
	They have no intrinsic topology.
\end{remark}
\begin{definition}
	If \( K \) is an abstract simplicial complex, its \emph{polyhedron} is \( \abs{K} = \bigcup_{e_I \in K} e_i \subseteq \Delta^n \).
\end{definition}
\begin{remark}
	Polyhedra are compact and Hausdorff.
\end{remark}
\begin{definition}
	We define \( K_r = \qty{e_I \in K \mid \abs{I} \leq r + 1} \) to be the set of faces of dimension at most \( r \).
	This is called the \emph{\( r \)-skeleton} of \( K \).
\end{definition}
The \( r \)-skeleton is an abstract simplicial complex.
Note that \( \qty{e_\varnothing} = K_{-1} \subset K_0 \subset \dots \subset K_n = K \).
We write \( \dim K = \max\qty{\dim e_I \mid e_I \in K} \).
\begin{definition}
	The \emph{vertex set} \( V(K) \) is the polyhedron \( \abs{K_0} \).
\end{definition}
\begin{example}
	\( \bm \Delta^n = F(\Delta^n) = \qty{e_I \mid I \subseteq \qty{0, \dots, n}} \) is a simplicial complex.
	Its polyhedron is \( \Delta^n \), which is homeomorphic to \( D^n \) by radial projection.
\end{example}
\begin{example}
	\( \mathbb S^{n-1} = \bm \Delta_{n-1}^n = \qty{e_i \mid I \subsetneq \qty{0, \dots, n}} \) is a simplicial complex.
	This has polyhedron \( \partial \Delta^n \) by definition of the boundary.
	This is homeomorphic to \( S^{n-1} \) by radial projection.
\end{example}
\begin{definition}
	Let \( K, L \) be abstract simplicial complexes in \( \Delta^n \) and \( \Delta^m \) respectively.
	A \emph{simplicial map} \( f \colon K \to L \) is a map such that there is a simplicial map \( \abs{f} \colon \Delta^n \to \Delta^m \) with \( f(e_I) = \abs{f}(e_I) \).
	Equivalently, there is a map \( \hat f \colon \qty{0, \dots, n} \to \qty{0, \dots, m} \) such that \( f(e_I) = e_{\hat f(I)} \) and \( e_I \in K \) implies \( e_{\hat f(I)} \in L \).
\end{definition}
\begin{remark}
	The identity map is simplicial.
	The composition of two simplicial maps is simplicial.
	This induces a category of simplices, where the morphisms are simplicial maps.
\end{remark}
\begin{definition}
	We say a simplicial map \( f \colon K \to L \) is a \emph{simplicial isomorphism} if \( f \) is a bijection, or equivalently, \( \abs{f} \) is a bijection or \( \abs{f} \) is a homeomorphism.
\end{definition}

\subsection{Euclidean simplicial complexes}
Recall that \( \mathcal S(\mathbb R^n) \) is the set of elements of the form \( [v_0, \dots, v_n] \) where the \( v_i \) are affine linearly independent.
\begin{definition}
	\( K \subseteq \mathcal S(\mathbb R^n) \) is a \emph{Euclidean simplicial complex} if
	\begin{enumerate}
		\item \( K \) is finite;
		\item if \( \sigma \in K \) and \( \tau \in F(\sigma) \), then \( \tau \in K \);
		\item if \( \sigma_1, \sigma_2 \in K \), then \( \sigma_1 \cap \sigma_2 \in F(\sigma_1) \cap F(\sigma_2) \).
	\end{enumerate}
	If so, we write \( \abs{K} = \bigcup_{\sigma \in K} \sigma \subseteq \mathbb R^n \) with the subspace topology.
	We write \( K_r = \qty{\sigma \in K \mid \dim \sigma \leq r} \) for its \( r \)-skeleton, which is a Euclidean simplicial complex.
\end{definition}
\begin{proposition}
	Let \( \abs{\varphi} \colon \Delta^n \to \mathbb R^n \) be affine linear, and \( K' \) be an abstract simplicial complex in \( \Delta^n \), such that \( \eval{\abs{\varphi}}_{\abs{K'}} \) is injective.
	Then \( \varphi(K') = \qty{\abs{\varphi}(e_I) \mid e_I \in K} \) is a Euclidean simplicial complex.
\end{proposition}
\begin{proof}
	Property (i) is clear since \( F(\Delta^n) \) is finite.
	For property (ii), note that if \( \sigma \in \varphi(K') \), there is \( e_I \in K' \) such that \( \sigma = \abs{\varphi}(e_I) \).
	If \( \tau \in F(\sigma) \), we have \( \tau = \abs{\varphi}(e_J) \) for \( e_J \subseteq e_I \).
	Then \( e_J \in K' \) since \( K' \) is an abstract simplicial complex.
	So \( \tau = \abs{\varphi}(e_J) = \varphi(K') \).

	Suppose \( \sigma_1 = \abs{\varphi}(e_{I_1}) \) and \( \sigma_2 = \abs{\varphi}(e_{I_2}) \) where \( e_{I_1}, e_{I_2} \in K' \).
	Then \( \sigma_1 \cap \sigma_2 = \abs{\varphi}(e_{I_1}) \cap \abs{\varphi}(e_{I_2}) = \abs{\varphi}(e_{I_1} \cap e_{I_2}) \) by injectivity.
	This is equal to \( \abs{\varphi}(e_{I_1 \cap I_2}) \in F(\sigma_1) \cap F(\sigma_2) \).
\end{proof}
\begin{definition}
	We say that the Euclidean simplicial complex \( \varphi(K') \) is a \emph{realisation} of an abstract simplicial complex \( K' \) in \( \Delta^n \), if \( \abs{\varphi} \colon \Delta^n \to \mathbb R^n \) is affine linear and injective on \( \abs{K'} \).
\end{definition}
\begin{remark}
	If \( \varphi(K') \) is a realisation of \( K' \), \( \eval{\abs{\varphi}}_{\abs{K'}} \) is injective, so \( \abs{\varphi} \colon \abs{K'} \to \abs{\varphi(K)} \) is a homeomorphism.
\end{remark}
\begin{proposition}
	Let \( K \subseteq \mathbb R^N \) be a Euclidean simplicial complex.
	Then \( K = \varphi(K') \) for some abstract simplicial complex \( K' \), and \( \abs{\varphi} \colon \abs{K'} \to \abs{K} \).
	Any two \( K' \) are related by a simplicial isomorphism.
\end{proposition}
Informally, every Euclidean simplicial complex is the realisation of some abstract simplicial complex.
\begin{proof}
	Let \( V(K) = \abs{K_0} = \qty{v_0, \dots, v_n} \subset \mathbb R^N \) be the vertex set of the Euclidean simplicial complex.
	Define \( K' = \qty{e_{\qty{i_0, \dots, i_k}} \mid [v_{i_0}, \dots, v_{i_k}] \in K} \).
	Let \( \abs{\varphi} \colon \Delta^n \to \mathbb R^N \) be given by \( \abs{\varphi}(e_i) = v_i \).

	We show that \( \eval{\abs{\varphi}}_{\abs{K'}} \) is injective.
	If \( \sigma = [v_{i_0}, \dots, v_{i_k}] \in K \), we have that \( v_{i_0}, \dots, v_{i_k} \) are affine linearly independent since \( K \) is a Euclidean simplicial complex.
	Then \( \abs{\varphi}_{e_I} \) is injective.

	Suppose \( \abs{\varphi}(p) = \abs{\varphi}(q) = x \in \mathbb R^N \), where \( p \in e_I \in K' \) and \( q \in e_J \in K' \).
	Then \( x \in \abs{\varphi}(e_I) \cap \abs{\varphi}(e_J) \), which is the intersection of simplices in \( K \), so \( x \in \abs{\varphi}(e_{I'}) \) for \( I' \subseteq I \cap J \).
	Since \( \eval{\abs{\varphi}}_{e_I} \) and \( \eval{\abs{\varphi}}_{e_J} \) are injective, we must have \( p, q \in e_{I'} \).
	But \( \eval{\abs{\varphi}}_{e_{I'}} \) is also injective, so \( p = q \).
\end{proof}
\begin{definition}
	A \emph{simplicial map of Euclidean simplicial complexes} is a map \( f \colon K_1 \to K_2 \) if there are realisations \( \varphi_i \colon K_i' \to K_i \) and a simplicial map of abstract simplicial complexes \( f' \colon K_1' \to K_2' \) so that the following diagram commutes.
	% TODO: the obvious diagram
\end{definition}
\begin{remark}
	The composition of simplicial maps of Euclidean simplicial complexes is also a simplicial map.
\end{remark}
