\subsection{Definitions}
We have shown that \( \pi_1(S^1,x_0) \simeq \mathbb Z \), and \( \pi_1(S^n,x_0) \simeq 1 \) for \( n > 1 \), so \( S^1 \not\sim S^n \).
We would like to show that \( S^n \sim S^m \) only holds if \( n = m \).
One proof of this fact is that any \( f \colon S^n \to S^m \) with \( n < m \) is null-homotopic, but the identity on \( S^m \) is not.
Both of these claims require proof: simplicial complexes will allow us to prove the first, and homology will allow us to prove the second.
\begin{definition}
	The \emph{\( n \)-simplex} is the topological space
	\[ \Delta^n  = \qty{(x_0, \dots, x_n) \in \mathbb R^{n+1} \midd x_i \geq 0, \sum_{i=0}^n x_i = 1 } \]
	with the subspace topology.
\end{definition}
\begin{remark}
	\( \Delta^1 \) is homeomorphic to \( I \).
	\( \Delta^2 \) is an equilateral triangle, and \( \Delta^3 \) is a regular tetrahedron.
	For all \( n \), \( \Delta^n \) is closed and bounded in \( \mathbb R^{n+1} \), and hence compact and Hausdorff.
	The standard basis vectors \( e_0, \dots, e_n \) are the vertices of \( \Delta^n \).
\end{remark}
\begin{definition}
	If \( I \subseteq \qty{0, \dots, n} \), the \emph{\( I \)th face of \( \Delta^n \)} is
	\[ e_I = \qty{x \in \Delta^n \mid x_i = 0 \text{ for } i \not\in I} \]
	We define \( F(\Delta^n) = \qty{e_I \mid I \subseteq \qty{0,\dots, n}} \) to be the set of faces of \( \Delta^n \).
\end{definition}
If \( I = \qty{i_0, \dots, i_k} \) with \( i_0 < \dots < i_k \), we write \( I = i_0i_1\dots i_k \).
\begin{remark}
	Note that \( e_{\qty{i}} = e_i \), and \( \Delta^n = e_{\qty{0,1, \dots, n}} \).
	\( e_I \) is a closed subset of \( \Delta^n \), and is homeomorphic to \( \Delta^{\abs{I} - 1} \).
	\( e_I \subseteq e_J \) if and only if \( I \subseteq J \).
	\( e_I \cap e_J = e_{I \cap J} \).
\end{remark}
\begin{definition}
	A map \( \abs{f} \colon \Delta^n \to \mathbb R^N \) is \emph{affine linear} if it is the restriction of a linear map \( \mathbb R^{n+1} \to \mathbb R^n \).
	Equivalently, \( \abs{f}\qty(\sum_{i=0}^n x_i e_i) = \sum_{i=0}^n x_i \abs{f}(e_i) \).
	We say an affine linear map \( \abs{f} \colon \Delta^n \to \Delta^m \) is \emph{simplicial} if it maps vertices in \( \Delta^n \) to vertices in \( \Delta^m \), so there is a map of sets \( \hat f \colon \qty{0,\dots, n} \to \qty{0, \dots, m} \) where \( \abs{f}(e_i) = e_{\hat f(i)} \).
\end{definition}
\begin{remark}
	Affine linear maps are continuous, and are determined entirely by their action on \( e_i \).
	In particular, simplicial maps \( \abs{f} \) are determined by \( \hat f \).
	For \( I \subseteq \qty{0, \dots, n} \), we have \( \abs{f}(e_I) = e_{\hat f(I)} \).
\end{remark}
\begin{definition}
	Vectors \( v_0, \dots, v_n \in \mathbb R^N \) are \emph{affine linearly independent} if whenever \( \sum t_i v_i = 0 \) and \( \sum t_i = 0 \), we have \( t_i = 0 \) for all \( i \).
	Equivalently,
	\begin{enumerate}
		\item If \( \sum t_i v_i = \sum t_i' v_i \) and \( \sum t_i = \sum t_i' \), then for each \( i \), \( t_i = t_i' \).
		\item The vectors \( v_1 - v_0, v_2 - v_0, \dots, v_n - v_0 \) are linearly independent.
		\item The unique affine linear map \( \abs{f} \colon \Delta^n \to \mathbb R^N \) given by \( \abs{f}(e_i) = v_i \) is injective.
	\end{enumerate}
	If \( v_0, \dots, v_n \) are affine linearly independent, we write \( [v_0, \dots, v_n] = \Im \abs{f} = \qty{\sum x_i v_i \mid \sum x_i = 1, x_i \geq 0} \), and we say \( [v_0, \dots, v_n] \) is a \emph{Euclidean simplex}.
\end{definition}
\begin{remark}
	\( \Delta^n \) is compact and \( [v_0, \dots, v_n] \) is Hausdorff, so by the topological inverse function theorem, \( \abs{f} \colon \Delta^n \to [v_0, \dots, v_n] \) is a homeomorphism if the \( v_i \) are affine linearly independent.
\end{remark}
\begin{lemma}
	If \( X \subseteq \mathbb R^N \), let \( Z(X) \) be the set of \( x \in X \) such that if \( x = \sum t_i x_i \) for \( t_i > 0, \sum t_i = 1 \) and all \( x_i \in X \), then \( x_i = x \) for some \( i \).
	Then \( Z([v_0, \dots, v_n]) = \qty{v_0, \dots, v_n} \).
\end{lemma}
\begin{proof}
	We show that \( v_k \in Z([v_0, \dots, v_n]) \); the converse is clear from the definition of the simplex.
	Suppose \( v_k = \sum t_i x_i \) for \( t_i > 0 \) and \( \sum t_i = 1 \).
	Then \( x_i = \sum_{j=0}^n s_{ij} v_j \), since \( x_i \in [v_0, \dots, v_n] \).
	So \( v_k = \sum_j \qty(\sum_i t_i s_{ij}) v_j \).
	Since the \( v_i \) are affine linearly independent, and \( \sum_j \qty(\sum_i t_i s_{ij}) = 1 \), we must have \( \sum t_i s_{ij} = 0 \) for \( j \neq k \).
	But \( t_i > 0 \) and \( s_{ij} \geq 0 \), so the only case is when all \( s_{ij} \) are exactly zero for \( j \neq k \), so \( x_j = v_k \).
\end{proof}
\begin{corollary}
	If \( [v_0, \dots, v_n] = [v_0', \dots, v_n'] \) as subsets of \( \mathbb R^N \), then \( \qty{v_0, \dots, v_n} = \qty{v_0', \dots, v_n'} \) as sets.
\end{corollary}
Therefore, a simplex determines its set of vertices.
\begin{proof}
	\( \qty{v_0, \dots, v_n} = Z([v_0, \dots, v_n]) = Z([v_0', \dots, v_n']) = \qty{v_0', \dots, v_n'} \).
\end{proof}
\begin{definition}
	\( \mathcal S(\mathbb R^n) \) is the set of Euclidean simplices \( \sigma \subseteq \mathbb R^n \).
	Hence, \( \mathcal S(\mathbb R^n) \) is in bijection with the set \( \qty{\qty{v_0, \dots, v_k} \mid v_i \in \mathbb R^N, k \geq -1, v_i \text{ affine linearly independent}} \).
\end{definition}

\subsection{Abstract simplicial complexes}
\begin{definition}
	An \emph{abstract simplicial complex} in \( \Delta^n \) is a subset \( K \) of the faces \( F(\Delta^n) \) such that \( e_I \in K \) whenever \( e_J \) is in \( K \) and \( I \subseteq J \).
\end{definition}
\begin{remark}
	Abstract simplicial complexes are downward-closed sets of faces.
	They have no intrinsic topology.
	The set of faces \( F(\Delta^n) \) of the \( n \)-dimensional simplex \( \Delta^n \) is an abstract simplicial complex.
\end{remark}
\begin{definition}
	If \( K \) is an abstract simplicial complex, its \emph{polyhedron} is \( \abs{K} = \bigcup_{e_I \in K} e_i \subseteq \Delta^n \).
\end{definition}
\begin{remark}
	Polyhedra are compact and Hausdorff.
\end{remark}
\begin{definition}
	We define \( K_r = \qty{e_I \in K \mid \abs{I} \leq r + 1} \) to be the set of faces of dimension at most \( r \).
	This is called the \emph{\( r \)-skeleton} of \( K \).
\end{definition}
The \( r \)-skeleton is an abstract simplicial complex.
Note that \( \qty{e_\varnothing} = K_{-1} \subset K_0 \subset \dots \subset K_n = K \).
We write \( \dim K = \max\qty{\dim e_I \mid e_I \in K} \).
\begin{definition}
	The \emph{vertex set} \( V(K) \) is the polyhedron \( \abs{K_0} \).
\end{definition}
\begin{example}
	\( \bm \Delta^n = F(\Delta^n) = \qty{e_I \mid I \subseteq \qty{0, \dots, n}} \) is a simplicial complex.
	Its polyhedron is \( \Delta^n \), which is homeomorphic to \( D^n \) by radial projection.
\end{example}
\begin{example}
	\( \mathbb S^{n-1} = \bm \Delta_{n-1}^n = \qty{e_i \mid I \subsetneq \qty{0, \dots, n}} \) is a simplicial complex.
	This has polyhedron \( \partial \Delta^n \) by definition of the boundary.
	This is homeomorphic to \( S^{n-1} \) by radial projection.
\end{example}
\begin{definition}
	Let \( K, L \) be abstract simplicial complexes in \( \Delta^n \) and \( \Delta^m \) respectively.
	A \emph{simplicial map} \( f \colon K \to L \) is a map such that there is a simplicial map \( \abs{f} \colon \Delta^n \to \Delta^m \) with \( f(e_I) = \abs{f}(e_I) \).
	Equivalently, there is a map \( \hat f \colon \qty{0, \dots, n} \to \qty{0, \dots, m} \) such that \( f(e_I) = e_{\hat f(I)} \) and \( e_I \in K \) implies \( e_{\hat f(I)} \in L \).
\end{definition}
\begin{remark}
	The identity map is simplicial.
	The composition of two simplicial maps is simplicial.
\end{remark}
\begin{definition}
	We say a simplicial map \( f \colon K \to L \) is a \emph{simplicial isomorphism} if \( f \) is a bijection, or equivalently, \( \abs{f} \) is a bijection or \( \abs{f} \) is a homeomorphism.
\end{definition}

\subsection{Euclidean simplicial complexes}
Recall that \( \mathcal S(\mathbb R^n) \) is the set of Euclidean simplices \( [v_0, \dots, v_n] \) where the \( v_i \) are affine linearly independent.
\begin{definition}
	\( K \subseteq \mathcal S(\mathbb R^n) \) is a \emph{Euclidean simplicial complex} if
	\begin{enumerate}
		\item \( K \) is finite;
		\item if \( \sigma \in K \) and \( \tau \in F(\sigma) \), then \( \tau \in K \);
		\item if \( \sigma_1, \sigma_2 \in K \), then \( \sigma_1 \cap \sigma_2 \in F(\sigma_1) \cap F(\sigma_2) \).
	\end{enumerate}
	If so, we write \( \abs{K} = \bigcup_{\sigma \in K} \sigma \subseteq \mathbb R^n \) with the subspace topology.
	We write \( K_r = \qty{\sigma \in K \mid \dim \sigma \leq r} \) for its \( r \)-skeleton, which is a Euclidean simplicial complex.
\end{definition}
\begin{proposition}
	Let \( \abs{\varphi} \colon \Delta^n \to \mathbb R^n \) be affine linear, and \( K' \) be an abstract simplicial complex in \( \Delta^n \), such that \( \eval{\abs{\varphi}}_{\abs{K'}} \) is injective.
	Then \( \varphi(K') = \qty{\abs{\varphi}(e_I) \mid e_I \in K} \) is a Euclidean simplicial complex.
\end{proposition}
\begin{proof}
	Property (i) is clear since \( F(\Delta^n) \) is finite.
	For property (ii), note that if \( \sigma \in \varphi(K') \), there is \( e_I \in K' \) such that \( \sigma = \abs{\varphi}(e_I) \).
	If \( \tau \in F(\sigma) \), we have \( \tau = \abs{\varphi}(e_J) \) for \( e_J \subseteq e_I \).
	Then \( e_J \in K' \) since \( K' \) is an abstract simplicial complex.
	So \( \tau = \abs{\varphi}(e_J) = \varphi(K') \).

	Suppose \( \sigma_1 = \abs{\varphi}(e_{I_1}) \) and \( \sigma_2 = \abs{\varphi}(e_{I_2}) \) where \( e_{I_1}, e_{I_2} \in K' \).
	Then \( \sigma_1 \cap \sigma_2 = \abs{\varphi}(e_{I_1}) \cap \abs{\varphi}(e_{I_2}) = \abs{\varphi}(e_{I_1} \cap e_{I_2}) \) by injectivity.
	This is equal to \( \abs{\varphi}(e_{I_1 \cap I_2}) \in F(\sigma_1) \cap F(\sigma_2) \).
\end{proof}
\begin{definition}
	We say that the Euclidean simplicial complex \( \varphi(K') \) is a \emph{realisation} of an abstract simplicial complex \( K' \) in \( \Delta^n \), if \( \abs{\varphi} \colon \Delta^n \to \mathbb R^n \) is affine linear and injective on \( \abs{K'} \).
\end{definition}
\begin{remark}
	If \( \varphi(K') \) is a realisation of \( K' \), \( \eval{\abs{\varphi}}_{\abs{K'}} \) is injective, so \( \abs{\varphi} \colon \abs{K'} \to \abs{\varphi(K)} \) is a homeomorphism.
\end{remark}
\begin{proposition}
	Let \( K \subseteq \mathbb R^N \) be a Euclidean simplicial complex.
	Then \( K = \varphi(K') \) for some abstract simplicial complex \( K' \), and \( \abs{\varphi} \colon \abs{K'} \to \abs{K} \).
	Any two \( K' \) are related by a simplicial isomorphism.
\end{proposition}
Informally, every Euclidean simplicial complex is the realisation of some abstract simplicial complex.
\begin{proof}
	Let \( V(K) = \abs{K_0} = \qty{v_0, \dots, v_n} \subset \mathbb R^N \) be the vertex set of the Euclidean simplicial complex.
	Define \( K' = \qty{e_{\qty{i_0, \dots, i_k}} \mid [v_{i_0}, \dots, v_{i_k}] \in K} \).
	Let \( \abs{\varphi} \colon \Delta^n \to \mathbb R^N \) be given by \( \abs{\varphi}(e_i) = v_i \).

	We show that \( \eval{\abs{\varphi}}_{\abs{K'}} \) is injective.
	If \( \sigma = [v_{i_0}, \dots, v_{i_k}] \in K \), we have that \( v_{i_0}, \dots, v_{i_k} \) are affine linearly independent since \( K \) is a Euclidean simplicial complex.
	Then \( \abs{\varphi}_{e_I} \) is injective.

	Suppose \( \abs{\varphi}(p) = \abs{\varphi}(q) = x \in \mathbb R^N \), where \( p \in e_I \in K' \) and \( q \in e_J \in K' \).
	Then \( x \in \abs{\varphi}(e_I) \cap \abs{\varphi}(e_J) \), which is the intersection of simplices in \( K \), so \( x \in \abs{\varphi}(e_{I'}) \) for \( I' \subseteq I \cap J \).
	Since \( \eval{\abs{\varphi}}_{e_I} \) and \( \eval{\abs{\varphi}}_{e_J} \) are injective, we must have \( p, q \in e_{I'} \).
	But \( \eval{\abs{\varphi}}_{e_{I'}} \) is also injective, so \( p = q \).
\end{proof}
\begin{definition}
	A \emph{simplicial map of Euclidean simplicial complexes} is a map \( f \colon K_1 \to K_2 \) if there are realisations \( \varphi_i \colon K_i' \to K_i \) and a simplicial map of abstract simplicial complexes \( f' \colon K_1' \to K_2' \) so that the following diagram commutes.
	\begin{center}
		\begin{tikzcd}
			{K_1'} & {K_2'} \\
			{K_1} & {K_2}
			\arrow["{\varphi_1}"', from=1-1, to=2-1]
			\arrow[from=1-1, to=1-2]
			\arrow["{\varphi_2}", from=1-2, to=2-2]
			\arrow["{f'}", from=1-1, to=1-2]
			\arrow["f"', from=2-1, to=2-2]
		\end{tikzcd}
	\end{center}
\end{definition}
\begin{remark}
	The composition of simplicial maps of Euclidean simplicial complexes is also a simplicial map.
\end{remark}

\subsection{Barycentric subdivision}
\begin{definition}
	Let \( \sigma \) be an \( n \)-dimensional Euclidean simplex.
	Let \( F(\sigma) \) be the set of faces of \( \sigma \), a Euclidean simplicial complex with \( \abs{F(\sigma)} = \sigma \).
	Let \( \bound \sigma = F(\sigma)_{n-1} = F(\sigma) \setminus \sigma \), a Euclidean simplicial complex.
	Let \( \partial \sigma = \abs{\bound \sigma} \subset \mathbb R^N \) be the boundary of \( \sigma \).
	It is homeomorphic to \( S^{n-1} \).
	Let \( \sigma^\circ = \sigma \setminus \partial \sigma \) be the interior of \( \sigma \).
\end{definition}
\begin{definition}
	Let \( X \subseteq \mathbb R^N \) and \( p \in \mathbb R^N \).
	We say \( p \) is \emph{independent} of \( X \) if for each \( x \in X \), the ray \( px \) from \( p \) to \( x \) has \( px \cap X = \qty{x} \).
\end{definition}
\begin{definition}
	If \( p \) is independent of \( X \), the \emph{cone} is defined by \( C_p(X) = \qty{tp + (1-t)x \mid t \in [0,1], x \in X} \)
\end{definition}
\begin{example}
	Let \( X = [v_0,\dots,v_n] \) be an \( n \)-simplex.
	Then \( p \) is independent of \( X \) if and only if \( \qty{v_0, \dots, v_n, p} \) is an affine linearly independent set.
	If so, \( C_p(X) = [v_0, \dots, v_n, p] \).
\end{example}
\begin{definition}
	Let \( K \) be a Euclidean simplicial complex in \( \mathbb R^N \) and \( p \) be independent of \( \abs{K} \).
	Then we define the \emph{cone} \( C_p(K) = K \cup \qty{[v_0, \dots, v_j, p] \mid [v_0, \dots, v_j] \in K} \).
\end{definition}
\begin{lemma}
	If \( p \) is independent of \( \abs{K} \), then \( C_p(K) \) is a Euclidean simplicial complex and \( \abs{C_p(K)} = C_p(\abs{K}) \).
\end{lemma}
\begin{definition}
	If \( \sigma = [v_0, \dots, v_n] \) is an \( n \)-simplex in \( \mathbb R^N \), we define its \emph{barycentre} \( b_\sigma = \frac{n+1}\sum_{i=0}^n v_i \).
\end{definition}
\begin{lemma}
	\( b_\sigma \) is independent of \( \partial\sigma \), and \( C_{b_\sigma}(\partial \sigma) = \sigma \).
\end{lemma}
We will define maps \( \beta \) from \( \mathcal S(\mathbb R^N) \) to the set of Euclidean simplicial complexes in \( \mathbb R^N \), and \( B \) from the set of Euclidean simplicial complexes in \( \mathbb R^N \) to Euclidean simplicial complexes in \( \mathbb R^N \), satisfying \( \abs{\beta(\sigma)} = \sigma \) and \( \abs{B(K)} = \abs{K} \).
The maps \( \beta \) and \( B \) are called \emph{barycentric subdivision}.
In order to do this, we will inductively define \( \beta \) and \( B \) on simplices and Euclidean simplicial complexes of dimension at most \( n \), and prove the following theorems.
\begin{theorem}[first inductive hypothesis]
	Let \( \sigma \in \mathcal S(\mathbb R^N) \) be an \( n \)-simplex.
	Then \( \beta(\sigma) \) is a Euclidean simplicial complex of dimension \( n \), and \( \abs{\beta(\sigma)} = \sigma \).
	If \( \tau \) is a face of \( \sigma \) and \( \sigma_1 \in \beta(\sigma) \) then \( \sigma_1 \cap \tau \in \beta(\tau) \).
\end{theorem}
\begin{theorem}[second inductive hypothesis]
	Let \( K \) be an \( n \)-dimensional Euclidean simplicial complex.
	Then \( B(K) \) is an \( n \)-dimensional Euclidean simplicial complex with polyhedron \( \abs{B(K)} = \abs{K} \).
\end{theorem}
For the base case, let \( n = -1 \).
The only \( -1 \)-dimensional simplex is \( \varnothing \).
We define \( \beta(\varnothing) = \qty{\varnothing} \).
The only \( -1 \)-dimensional simplicial complex is \( \qty{\varnothing} \), and we define \( B(\qty{\varnothing}) = \qty{\varnothing} \).
Both inductive hypotheses hold for this case.

In general, suppose \( \beta \) and \( B \) are defined on \( n-1 \)-dimensional simplices and simplicial complexes and that both inductive hypotheses hold.
We now define \( \beta(\sigma) = C_{b_\sigma}(B(\bound \sigma)) \) and \( B(K) = \bigcup_{\sigma \in K} \beta(\sigma) \).
\begin{example}
	Let \( \sigma \) be the zero-simplex.
	Then \( b_\sigma(\sigma) = \sigma \), and \( \beta(\sigma) = \qty{\varnothing, b_\sigma} \).
\end{example}
\begin{example}
	Let \( \sigma \) be the one-dimensional simplex.
	\( \bound\sigma \) is two points \( p_1, p_2 \) and the empty set.
	Then \( B(\bound\sigma) = \qty{\varnothing,p_1,p_2} \).
	Therefore, \( C_p(B(\bound\sigma)) = \qty{\varnothing, p, p_1, p_2, pp_1, pp_2} \).
\end{example}
\begin{example}
	Let \( \sigma \) be a two-dimensional simplex with vertices \( p_1, p_2, p_3 \).
	Then \( C_p(B(\bound\sigma)) \) has six 2-simplices, twelve 1-simplices, seven 0-simplices and one empty simplex.
\end{example}
\begin{proof}[Proof of first inductive hypothesis]
	\( \bound\sigma \) is a Euclidean simplicial complex of dimension \( n - 1 \), hence \( B(\bound\sigma) \) is a Euclidean simplicial complex by the second inductive hypothesis, and \( \abs{B(\bound\sigma)} = \abs{\bound\sigma} = \partial \sigma \).
	By the lemmas above, \( b_\sigma \) is independent of \( \partial\sigma = \abs{B(\bound\sigma)} \), so \( C_{b_\sigma}(B(\bound\sigma)) \) is a Euclidean simplicial complex with polyhedron \( \abs{C_{b_\sigma}(B(\bound\sigma))} = C_{b_\sigma}(\partial\sigma) = \sigma \).
	The next part follows from the lemma: if \( \sigma \in C_p(K) \), then \( \sigma \cap \abs{K} \in K \).
\end{proof}
\begin{proof}[Proof of second inductive hypothesis]
	We check the properties required for a Euclidean simplicial complex for \( B(K) = \bigcup_{\sigma \in K} \beta(\sigma) \).
	\( \beta(\sigma) \) is finite for each \( \sigma \) and \( K \) is finite, so \( B(K) \) is finite.
	If \( \sigma \in B(K) \) then \( \sigma \in \beta(\sigma') \) for some \( \sigma' \in K \), so if \( \tau \in F(\sigma) \), then \( \tau \in \beta(\sigma') \) since \( \beta(\sigma') \) is a Euclidean simplicial complex, so \( \tau \in B(K) \), so the second property holds.
	Suppose \( \sigma_1, \sigma_2 \in B(K) \) where \( \sigma_i \in \beta(\sigma'_i) \) and \( \sigma_i' \in K \).
	Then \( \sigma_1 \cap \sigma_2 \subseteq \sigma_1' \cap \sigma_2' = \tau \) since \( \abs{\beta(\sigma'_i)} = \sigma'_i \), where \( \tau \in K \) since \( K \) is a Euclidean simplicial complex.
	Then \( \sigma_1 \cap \tau, \sigma_2 \cap \tau \in \beta(\tau) \) by the second part of the first inductive hypothesis.
	In particular, \( \beta(\tau) \) is a Euclidean simplicial complex, so \( \sigma_1 \cap \sigma_2 = \underbrace{(\sigma_1 \cap \tau)}_{\in \beta(\tau)} \cap \underbrace{(\sigma_2 \cap \tau)}_{\in \beta(\tau)} \in \beta(\tau) \subseteq B(K) \), so the third property holds.
	So \( K \) is a Euclidean simplicial complex.
	Now, by the first inductive hypothesis, \( \abs{B(K)} = \bigcup_{\sigma \in K} \beta(\sigma) = \bigcup_{\sigma \in K} \sigma = \abs{K} \).
\end{proof}
\begin{lemma}
	Let \( \sigma \in \mathcal S(\mathbb R^N) \) and \( x, v \in \sigma \).
	Then \( \norm{v - x} \leq \max_{v_i \in V(\sigma)} \norm{v_i - x} \).
\end{lemma}
\begin{proof}
	We can write \( x = \sum x_i v_i \), where \( \sum x_i = 1 \), \( x_i \geq 0 \), and \( v_i \in V(\sigma) \).
	But also, \( v = \sum x_i v \).
	Hence,
	\[ \norm{v - x} = \norm{\sum x_i (v - v_i)} \leq \sum x_i \norm{v - v_i} \leq \sum x_i \max\norm{v - v_i} = \max \norm{v - v_i} \]
	Applying this twice, \( \norm{x - v} \leq \max_{v_i \in V(\sigma)} \norm{v - v_i} \leq \max_{v_i, v_j \in V(\sigma)} \norm{v_i - v_j} \).
\end{proof}
\begin{definition}
	The \emph{mesh} of a simplex \( \sigma \in \mathcal S(\mathbb R^N) \) is \( \mu(\sigma) = \max_{v_i, v_j \in V(\sigma)} \norm{v_i - v_j} = \max_{x, v \in \sigma} \norm{v - x} \).
	If \( K \) is a Euclidean simplicial complex, its mesh is \( \mu(K) = \max_{\sigma \in K} \mu(\sigma) \).
\end{definition}
\begin{lemma}
	Let \( b_\sigma \) be the barycentre of \( \sigma \), so \( b_\sigma = \frac{1}{n+1} \sum_{i=0}^n v_i \) for \( \sigma = [v_0, \dots, v_n] \).
	Then \( \max_{v \in \sigma} \norm{b_\sigma - v} \leq \frac{n}{n+1} \mu(\sigma) \).
\end{lemma}
\begin{proof}
	\( \norm{b_\sigma - v} \leq \max_{v_i \in V(\sigma)} \norm{b_\sigma - v_i} \).
	We have
	\[ \norm{b_\sigma - v_i} = \frac{1}{n+1} \norm{\sum_{j \neq i} v_j - nv_i} \leq \frac{1}{n+1} \sum_{j \neq i} \norm{v_j - v_i} \leq \frac{1}{n+1} \cdot n\mu(\sigma) \]
\end{proof}
\begin{corollary}
	Let \( \sigma \) be a Euclidean simplex of dimension \( n \).
	Then \( \mu(\beta(\sigma)) \leq \frac{n}{n+1}\mu(\sigma) \).
	Let \( K \) be a Euclidean simplicial complex of dimension \( n \).
	Then \( \mu(B(K)) \leq \frac{n}{n+1}\mu(K) \).
\end{corollary}
\begin{proof}
	Let \( \tau \in \beta(\sigma) \).
	Suppose \( \tau \in B(\bound \sigma) \).
	Then, \( \mu(\tau) \leq \frac{n-1}{n}\mu(B(\bound\sigma)) \leq \frac{n}{n+1} \mu(\sigma) \) by induction.
	Otherwise, \( \tau = [v_0, \dots, v_k, b_\sigma] \), where \( [v_0, \dots, v_k] \in B(\bound\sigma) \).
	Then \( \norm{v_i - v_j} \leq \frac{n-1}{n} \mu(\sigma) \) by induction, and \( \norm{v_i - b_\sigma} \leq \frac{n}{n+1} \mu(\sigma) \) by the lemma.
\end{proof}

\subsection{Simplicial approximation}
\begin{lemma}
	\begin{enumerate}
		\item Let \( x \in \Delta^n \).
			Then there exists a unique \( I \subseteq \qty{0, \dots, n} \) such that \( x \in e_I^\circ \).
		\item If \( x \in e_I^\circ \), then \( x \in e_J \) if and only if \( I \subseteq J \), or equivalently, \( e_I \subseteq e_J \).
		\item Let \( K \) be an abstract simplicial complex in \( \Delta^n \), and let \( x \in e_I^\circ \).
			Suppose that \( x \in \abs{K} \).
			Then \( e_I \in K \).
	\end{enumerate}
\end{lemma}
\begin{proof}
	\emph{Part (i).}
	Let \( I = \qty{i \in \qty{0, \dots, n} \mid x_i \neq 0} \).
	\emph{Part (ii).}
	Follows from part (i).

	\emph{Part (iii).}
	\( x \in \abs{K} \) implies \( x \in e_J \) for some \( e_J \in K \).
	By part (ii), we have \( e_I \subseteq e_J \).
	Since \( K \) is an abstract simplicial complex and \( e_J \in K \), we have \( e_I \in K \).
\end{proof}
\begin{corollary}
	Let \( K \) be a Euclidean simplicial complex, and \( x \in \abs{K} \).
	Then there exists a unique \( \sigma \in K \) with \( x \in \sigma^\circ \).
\end{corollary}
\begin{proof}
	Let \( \varphi \colon K' \to K \) be a realisation of \( K \), so \( K' \) is an abstract simplicial complex and \( \varphi \) is a bijection inducing a homeomorphism on the polyhedra.
	Let \( x' = \abs{\varphi^{-1}}(x) \in \abs{K} \).
	Then \( x' \) lies in the interior of a unique \( e_I \) by part (i) of the lemma above.
	Note that \( e_I \in K' \) by part (iii), so \( \varphi(e_I) \) is the unique \( \sigma \in K \) with \( x \in \sigma^\circ \).
\end{proof}
\begin{definition}
	Let \( K \) be a Euclidean simplicial complex, and let \( v \in V(K) \).
	Then the \emph{star} \( \mathsf{St}_K(v) \) is \( \bigcup_{\qty{\sigma \in K \mid v \in \sigma}} \sigma^\circ \).
\end{definition}
\begin{lemma}
	\begin{enumerate}
		\item Let \( x \in \abs{K} \) and \( x \in \sigma^\circ \).
			Then \( x \in \mathrm{St}_K(v) \) if and only if \( v \in V(\sigma) \).
		\item \( \mathrm{St}_K(v) = \abs{K} \setminus \bigcup_{\qty{\sigma \in K \mid v \not\in V(\sigma)}} \sigma^\circ = \abs{K} \setminus \bigcup_{\qty{\sigma \in K \not\in V(\sigma)}} \sigma \).
		\item \( \qty{\mathrm{St}_K(v) \mid v \in V(K)} \) is an open cover of \( \abs{K} \).
	\end{enumerate}
\end{lemma}
\begin{proof}
	\emph{Part (i).}
	Follows from the fact that if \( x \in \abs{K} \), \( x \) lies in a unique interior of \( \sigma \) for \( \sigma \in K \).

	\emph{Part (ii).}
	The first equality follows from part (i).
	The second follows from the fact that if \( \tau \in F(\sigma) \) and \( v \not\in V(\sigma) \), then \( v \not\in V(\tau) \).

	\emph{Part (iii).}
	Part (ii) exhibits \( \mathrm{St}_K(v) \) as the complement of a finite union of closed sets in \( \abs{K} \), so it is open.
	If \( x \in \abs{K} \), then \( x \in \sigma^\circ \) for some \( \sigma \), and if \( v \in V(\sigma) \), then \( x \in \mathrm{St}_K(v) \), so it is a cover.
\end{proof}
\begin{definition}
	Let \( K, L \) be Euclidean simplicial complexes.
	Let \( f \colon \abs{K} \to \abs{L} \) be a continuous map, and let \( \hat g \colon V(K) \to V(L) \).
	We say that \( \hat g \) is a \emph{simplicial approximation} of \( f \) if \( f(\mathrm{St}_K(v)) \subseteq \mathrm{St}_L(\hat g(v)) \) for all \( v \in V(K) \).
\end{definition}
\begin{theorem}
	Let \( \varphi \colon K' \to K \) be a realisation of a Euclidean simplicial complex \( K \), and let \( L \) be a Euclidean simplicial complex in \( \mathbb R^M \).
	We define \( g' \colon \abs{K'} \to \mathbb R^M \) to be the affine linear map with \( \abs{g'}(v) = \hat g(\varphi(v)) \) if \( v \in V(K') \).
	Let \( \abs{g} = \abs{g'} \circ \abs{\varphi}^{-1} \).
	Then \( \abs{g} \) defines a simplicial map \( g \colon K \to L \), and \( \abs{g} \sim f \).
\end{theorem}
\begin{proof}
	Let \( \sigma \in K \).
	We must show that \( \abs{g}(\sigma) \in L \).
	Let \( x \in \sigma^\circ \) be an arbitrary point in the interior.
	Then \( f(x) \in \abs{L} \), so \( f(x) \in \tau^\circ \) with \( \tau \in L \).
	Then \( x \in \bigcap_{v \in V(\sigma)} \mathrm{St}_K(v) \), so \( f(x) \in \bigcap_{v \in V(\sigma)} f(\mathrm{St}_K(v)) \subseteq \bigcap_{v \in V(\sigma)} \mathrm{St}_L(g(v)) \) since \( g \) is a simplicial approximation of \( f \).
	Now, if \( v \in V(\sigma) \), \( f(x) \in \tau^\circ \) and \( f(x) \in \mathrm{St}_L(g(v)) \), so \( g(v) \in \tau \) by part (i) of the lemma above.
	Hence, every vertex of \( \abs{g}(\sigma) \) is a vertex of \( \tau \), so \( \abs{g}(\sigma) \) is a face of \( \tau \in L \), so \( \abs{g}(\sigma) \in L \) as required.
	So \( g \colon K \to L \) is simplicial.

	For the second part, we define \( H \colon \abs{K} \times I \to \mathbb R^M \) by \( H(x,t) = t \abs{g}(x) + (1-t)f(x) \).
	This is clearly a homotopy in \( \mathbb R^M \), but we need to show it is a homotopy in \( \abs{L} \).
	Suppose \( x \in \sigma^\circ \) and \( f(x) \in \tau^\circ \) as before.
	Then \( x = \sum_{v_i \in V(\sigma)} x_i v_i \), so \( \abs{g}(x) = \sum_{v_i \in V(\sigma)} x_i \abs{g}(v_i) \in \tau \) since \( \abs{g}(v_i) \in \tau \).
	Since \( \tau \) is convex, and \( \abs{g}(x), f(x) \in \tau \), we must have \( H(x,t) \in \tau \) for \( t \in [0,1] \).
	So \( H \colon \abs{K} \times I \to \abs{L} \), which is the desired homotopy.
\end{proof}
\begin{theorem}[simplicial approximation theorem]
	Let \( K, L \) be Euclidean simplicial complexes.
	Let \( f \colon \abs{K} \to \abs{L} \) be a continuous map.
	Then there exists \( r > 0 \) and a simplicial map \( g \colon B^r(K) \to L \) such that \( \abs{g} \sim f \).
\end{theorem}
Note that \( \abs{B^r(K)} = \abs{K} \), so \( \abs{g} \colon \abs{B^r(K)} \to \abs{L} \) can be thought of as a map \( \abs{K} \to \abs{L} \).
\begin{proof}
	We have the open cover \( \qty{\mathrm{St}_L(v) \mid v \in V(L)} \) of \( \abs{L} \).
	\( f \colon \abs{K} \to \abs{L} \) is continuous, so \( \qty{f^{-1}(\mathrm{St}_L(v)) \mid v \in V(L)} \) is an open cover of \( \abs{K} \).
	Now, \( \abs{K} \) is a compact metric space, so we can apply the Lebesgue covering lemma to find \( \delta > 0 \) and a function \( \abs{K} \to V(L) \) mapping \( x \) to some vertex \( v_x \) such that \( B_\delta(x) \subseteq f^{-1}(\mathrm{St}_L(v_x)) \).
	Let \( r \) be a natural number such that \( \mu(B^r(K)) < \delta \), and let \( K' = B^r(K) \).
	If \( \sigma \in K' \) and \( x \in V(\sigma) \), then \( \sigma \subseteq B_\delta(x) \), since \( \mu(K') < \delta \).
	If \( x \in V(K') \), then
	\[ \mathrm{St}_{K'}(x) = \bigcup_{\qty{\sigma \mid x \in V(\sigma)}} \sigma^\circ \subseteq \bigcup_{\qty{\sigma \mid x \in V(\sigma)}} \sigma \subseteq B_\delta(x) \]
	Hence, \( f(\mathrm{St}_{K'}(x)) \subseteq f(B_\delta(x)) \subseteq \mathrm{St}_L(v_x) \), so the function \( \hat g \colon V(K') \to V(L) \) given by \( \hat g(x) = v_x \) is a simplicial approximation of \( f \).
	So by the previous theorem, \( \hat g \) determines a simplicial map \( g \colon K' \to L \) with \( \abs{g} \sim f \).
\end{proof}
\begin{corollary}
	Let \( K, L \) be Euclidean simplicial complexes, where \( \dim K < \dim L \).
	Let \( f \colon \abs{K} \to \abs{L} \) be continuous.
	Then \( f \sim \abs{g} \) where \( \abs{g} \) is not surjective.
\end{corollary}
\begin{proof}
	Let \( g \colon B^r(K) \to L \) be a simplicial map such that \( f \sim \abs{g} \).
	Let \( k = \dim B^r(K) = \dim K \).
	Then \( \abs{g} \colon \abs{K} \to \abs{L_k} \subsetneq \abs{L} \) since \( \dim L > k \).
	So \( \abs{g} \) is not surjective.
\end{proof}
\begin{remark}
	It is a general fact that simplicial functions map an \( i \)-skeleton into an \( i \)-skeleton for each \( i \).
\end{remark}
\begin{theorem}
	If \( k < n \), any continuous map \( S^k \to S^n \) is null-homotopic.
\end{theorem}
\begin{proof}
	\( S^k = \abs{\mathbb S^k} \) and \( S^n = \abs{\mathbb S^n} \).
	By the above corollary, \( f \sim \abs{g} \) where \( \abs{g} \colon S^k \to S^n \) is not surjective.
	Let \( \abs{g} \colon S^k \to S^n \setminus \qty{p} \).
	\begin{center}
		\begin{tikzcd}
			{S^k} & {S^n\setminus\qty{p}} \\
			& {S^n}
			\arrow["{g'}", from=1-1, to=1-2]
			\arrow["\iota", from=1-2, to=2-2]
			\arrow["{\abs{g}}"', from=1-1, to=2-2]
		\end{tikzcd}
	\end{center}
	But \( S^n \setminus \qty{p} \simeq \mathbb R^n \) is contractible.
	So \( g' \) is null-homotopic, so \( \abs{g} \sim \iota \circ g' \) is null-homotopic.
\end{proof}
