\subsection{Chain complexes}
\begin{definition}
	A \emph{(finitely generated) chain complex} \( (C_\bullet, d) \) is
	\begin{enumerate}
		\item a collection of free (finitely generated) abelian groups \( C_i \) for \( i \in \mathbb Z \) (and if finitely generated, \( C_i = 0 \) for all but finitely many \( i \));
		\item a collection of homomorphisms \( d_i \colon C_i \to C_{i-1} \);
		\item \( d_{i-1} \circ d_i = 0 \) for all \( i \).
	\end{enumerate}
	\begin{center}
		\begin{tikzcd}
			\cdots & {C_{-2}} & {C_{-1}} & {C_0} & {C_1} & {C_2} & \cdots
			\arrow["{d_3}"', from=1-7, to=1-6]
			\arrow["{d_2}"', from=1-6, to=1-5]
			\arrow["{d_1}"', from=1-5, to=1-4]
			\arrow["{d_0}"', from=1-4, to=1-3]
			\arrow["{d_{-1}}"', from=1-3, to=1-2]
			\arrow["{d_{-2}}"', from=1-2, to=1-1]
		\end{tikzcd}
	\end{center}
\end{definition}
Usually, we write \( C_\bullet = \bigoplus_i C_i \), and \( d = \bigoplus_i d_i \colon C_\bullet \to C_\bullet \).
We can check that \( d_{i-1} \circ d_i = 0 \) for all \( i \) is equivalent to the statement that \( d \circ d = d^2 = 0 \).
\begin{remark}
	Free finitely generated abelian groups are isomorphic to \( \mathbb Z^n \) for some \( n \).
	A chain complex defined over \( \mathbb Q \), \( \mathbb R \), or \( \mathbb F_p \) is similar, except that \( C_i \) is a vector space over the \( \mathbb Q, \mathbb R, \mathbb F_p \) and the \( d_i \) are linear maps.
	Every chain complex determines another chain complex over \( \mathbb Q, \mathbb R, \mathbb F_p \) by replacing \( \mathbb Z^{n_i} \) with \( \mathbb Q^{n_i} \), for example, and the \( d_i \) are given by the same matrices.
\end{remark}
\begin{remark}
	There is a unique group homomorphism to and from the trivial abelian group \( 0 \).
	Arrows to and from this group can therefore be unlabelled.
\end{remark}
\begin{example}[reduced chain complex of the simplex]
	Consider the reduced chain complex of \( \Delta^n \).
	We define \( \widetilde C_k(\Delta^n) = \genset{e_I \mid \abs{I} = k + 1, I \subseteq \qty{0, \dots, n}} \), the free abelian group on a basis given by the \( e_I \).
	We also define \( d(e_I) = \sum_{j=0}^{\abs{I}} (-1)^j e_{I_{\hat\jmath}} \) where if \( I = i_0 i_1 \dots i_k \) and \( i_0 < \dots < i_k \), we define \( I_{\hat\jmath} = I \setminus \qty{i_j} \).
	For example, consider \( C_\bullet(\Delta^2) \).
	\[ \widetilde C_2(\Delta^2) = \genset{e_{012}};\quad \widetilde C_1(\Delta^2) = \genset{e_{01}, e_{02}, e_{12}};\quad \widetilde C_0(\Delta^2) = \qty{e_0, e_1, e_2};\quad \widetilde C_{-1}(\Delta^2) = \qty{e_\varnothing} \]
	and, for example,
	\[ d(e_{012}) = (-1)^0 e_{12} + (-1)^1 e_{02} + (-1)^2 e_{01} = e_{12} - e_{02} + e_{01} \]
	\[ d(e_{01}) = e_1 - e_0;\quad d(e_{02}) = e_2 - e_0;\quad d(e_{12}) = e_2 - e_1;\quad d(e_0) = d(e_1) = d(e_2) = e_\varnothing \]
	Note that \( \widetilde C_i(\Delta^2) = 0 \) if \( i < -1 \) or \( i > 2 \).
	We have \( d^2(e_{012}) = d(e_{12} - e_{02} + e_{01}) = e_2 - e_1 - e_2 + e_0 + e_1 - e_0 = 0 \), as required.
	\begin{center}
		\begin{tikzcd}
			0 & {\widetilde C_{-1}} & {\widetilde C_0} & {\widetilde C_1} & {\widetilde C_2} & 0
			\arrow[from=1-6, to=1-5]
			\arrow["{d_2}"', from=1-5, to=1-4]
			\arrow["{d_1}"', from=1-4, to=1-3]
			\arrow["{d_0}"', from=1-3, to=1-2]
			\arrow[from=1-2, to=1-1]
		\end{tikzcd}
	\end{center}
\end{example}
\begin{proposition}
	For \( \widetilde C_\bullet(\Delta^n) \), \( d^2 = 0 \).
\end{proposition}
\begin{proof}
	The \( e_I \) are a basis for \( \widetilde C_\bullet(\Delta^n) \), so it suffices to check that \( d^2(e_I) = 0 \) for each \( I \).
	For some \( c_{jj'} \), we have \( d^2(e_I) = \sum_{j<j'} c_{jj'} e_{I_{\hat\jmath,\hat\jmath'}} \) where \( I_{\hat\jmath,\hat\jmath'} = I \setminus \qty{i_j, i_{j'}} \).
	We can compute that \( c_{jj'} \) has a contribution of \( (-1)^j (-1)^{j' - 1} \) by first considering \( j \) then \( j' \), since \( i_{j'} \) is the \( (j'-1) \)th element of \( I_{\hat\jmath} \).
	Note also that by computing the term in the sum with \( j, j' \) in the other order, we have a contribution of \( (-1)^{j'} (-1)^i \).
	Hence \( c_{jj'} = (-1)^j (-1)^{j'-1} + (-1)^{j'} (-1)^i = 0 \).
\end{proof}
\begin{example}[chain complex of the simplex]
	The chain complex of \( \Delta^n \) is defined by \( C_i(\Delta^n) = \widetilde C_i(\Delta^n) \) if \( i \geq 0 \), but \( C_{-1}(\Delta^n) = 0 \).
	This removes the empty face \( e_\varnothing \).
	The \( d_i \) are unchanged.
	\begin{center}
		\begin{tikzcd}
			0 & {C_0} & {C_1} & {C_2} & 0
			\arrow[from=1-5, to=1-4]
			\arrow["{d_2}"', from=1-4, to=1-3]
			\arrow["{d_1}"', from=1-3, to=1-2]
			\arrow[from=1-2, to=1-1]
		\end{tikzcd}
	\end{center}
\end{example}
\begin{definition}
	Let \( K \) be an abstract simplicial complex in \( \Delta^n \).
	Let \( \widetilde C_k(K) = \genset{e_I \mid \abs{I} = k + 1, e_I \in K} \leq \widetilde C_k(\Delta^n) \).
	Since \( e_I \in K \) implies \( e_{I_{\hat\jmath}} \in K \), \( d_k \colon \widetilde C_k(K) \to \widetilde C_{k-1}(K) \).
	So \( (\widetilde C_\bullet(K), d) \) is a chain complex.
\end{definition}
\begin{definition}
	Let \( (C_\bullet, d) \) be a chain complex, and let \( x \in C_k \).
	We say that \( x \) is a \emph{cycle} or \emph{closed} if \( d x = 0 \), so \( x \in \ker d_k \).
	We say that \( x \) is a \emph{boundary} or \emph{exact} if \( x = d y \) for some \( y \), so \( x \in \Im d_{k+1} \).
\end{definition}
\begin{remark}
	The statement \( d^2 = 0 \) is equivalent to the statement \( \Im d_{k+1} \subseteq \ker d_k \) for each \( k \), so boundaries are always cycles.
\end{remark}

\subsection{Homology groups}
\begin{definition}
	Let \( (C_\bullet, d) \) be a chain complex.
	Its \emph{\( k \)th homology group} is \( H_k(C) = \faktor{\ker d_k}{\Im d_{k+1}} \).
\end{definition}
\begin{remark}
	Homology groups are abelian.
\end{remark}
\begin{example}
	Consider \( \widetilde C_\bullet(\Delta^2) \).
	Recall \( \widetilde C_2 = \genset{e_{012}} \) and \( d(e_{012}) = e_{12} - e_{02} + e_{01} \).
	Hence \( \ker d_2 = 0 \) and \( \Im d_3 = 0 \), so \( H_2(\widetilde C_\bullet(\Delta^2)) = 0 \).

	We have \( \widetilde C_1 = \genset{e_{12}, e_{02}, e_{01}} \), and \( d(ae_{01} + be_{12} + ce_{02}) = a(e_1 - e_0) + b(e_2 - e_1) + c(e_2 - e_0) = -(a+c)e_0 + (a-b)e_1 + (b+c)e_2 \).
	Hence \( ae_{01} + be_{12} + ce_{02} \in \ker d \) if and only if \( a = b = -c \).
	So \( x \in \genset{e_{12} - e_{02} + e_{01}} = \Im d_2 \), giving \( H_1(\widetilde C_\bullet(\Delta^2)) = 0 \).

	We have \( \widetilde C_0 = \genset{e_0, e_1, e_2} \) and \( d(e_i) = e_\varnothing \), so \( \ker d_0 = \qty{a e_0 + b e_1 + c e_2 \mid a + b + c = 0} \).
	Conversely, \( \Im d_1 = \vecspan\qty{e_1 - e_0, e_2 - e_0, e_2 - e_1} = \ker d_0 \).
	So in fact \( H_0(\widetilde C_\bullet(\Delta^2)) = 0 \).

	Now \( \widetilde C_{-1} = \genset{e_\varnothing} = \ker d_{-1} = \genset{e_\varnothing} = \Im d_0 \) so \( H_{-1}(\widetilde C_\bullet(\Delta^2)) = 0 \).
	So all of the homology groups of \( \widetilde C_\bullet(\Delta^2) \) are trivial.
	Note that
	\[ H_i(C_\bullet(\Delta^2)) = \begin{cases}
		H_i(\widetilde C_\bullet(\Delta^2)) & i > 0 \\
		\faktor{\genset{e_0, e_1, e_2}}{\vecspan\qty{e_1 - e_0, e_2 - e_0, e_2 - e_1}} \simeq \mathbb Z & i = 0
	\end{cases} \]
\end{example}
\begin{definition}
	Let \( K \) be an abstract simplicial complex in \( \Delta^n \).
	Then we define \( \widetilde H_i(K) = H_i(\widetilde C_\bullet(K)) \) to be the \( i \)th reduced homology group of \( K \).
	Then \( C\bullet(K) = \faktor{\widetilde C_\bullet(K)}{\genset{e_\varnothing}} \) is a chain complex, and \( H_i(K) = H_i(C_\bullet(K)) \) is the \( i \)th homology group of \( K \).
\end{definition}
\begin{example}
	Let \( K = \qty{e_0, e_1, \dots, e_r, e_\varnothing} \), so \( \abs{K} \) is a collection of \( r + 1 \) disjoint points.
	In this case, \( \widetilde C_i(K) = 0 \) for \( i > 0 \).
	\( \widetilde C_0(K) = \genset{e_0, \dots, e_r} \) and \( d(e_i) = \varnothing \).
	\( \widetilde C_{-1}(K) = \genset{e_\varnothing} \).
	Hence \( \ker d_0 = \genset{e_1-e_0, \dots, e_r-e_0} \) and \( \Im d_1 = 0 \), so \( H_0(\widetilde C_\bullet(K)) = \mathbb Z^r \), and \( H_{-1}(\widetilde C_\bullet(K)) = 0 \).
	Note that \( H_0(C_\bullet(K)) = \mathbb Z^{r+1} = \genset{e_0, \dots, e_r} \).
\end{example}
\begin{example}
	Recall that any Euclidean simplicial complex is realised by an abstract simplicial complex, but we have choice in the labelling of the vertices.
	Let \( T_n \) be the boundary of a convex \( n \)-gon in \( \mathbb R^2 \).
	Then the abstract simplicial complex \( K' = \qty{e_\varnothing, e_0, \dots, e_{n-1}, e_{01}, e_{12}, \dots, e_{(n-2)(n-1)}, e_{(n-1)0}} \) in \( \Delta^{n-1} \) realises \( T_n \).

	Then \( C_1(K') = \genset{e_{01}, e_{12}, \dots, e_{(n-2)(n-1)}, e_{(n-1)0}} \) and \( C_0(K') = \genset{e_0, \dots, e_{n-1}} \).
	We have \( d(e_{i(i+1)}) = e_{i+1} - e_i \), so \( \ker d_1 = \genset{x} \) where \( x = e_{01} + e_{12} + \dots + e_{(n-2)(n-1)} - e_{0(n-1)} \).
	Note that \( \Im d_1 = \vecspan\qty{e_{i+1} - e_i} \).

	Hence \( H_1(K') = \faktor{\ker d_1}{\Im d_2} = \faktor{\genset{x}}{0} \simeq \mathbb Z \), and \( H_0(K') = \faktor{\ker d_0}{\Im d_1} = \faktor{\genset{e_0, \dots, e_{n-1}}}{\vecspan\qty{e_1 - e_0, \dots, e_{n-1} - e_{n-2}}} \simeq \mathbb Z \).
	Note that this result does not depend on the choice of \( n \), and \( \abs{T_n} \simeq S^1 \) also does not depend on \( n \).
	In fact, \( H_\bullet(K) \) depends only on \( \abs{K} \).
\end{example}

\subsection{Chain maps}
\begin{definition}
	Let \( (C_\bullet, d) \) and \( (C_\bullet', d') \) be chain complexes.
	A \emph{chain map} \( f \colon C_\bullet \to C_\bullet' \) is
	\begin{enumerate}
		\item for each \( i \), a function \( f_i \colon C_i \to C_i' \);
		\item \( f_{i-1} \circ d_i = d_i' \circ f_i \);
	\end{enumerate}
\end{definition}
\begin{remark}
	We can interpret \( f \) as \( \bigoplus_i f_i \colon C_\bullet \to C_\bullet' \), given by a block matrix
	\[ \begin{pmatrix}
		f_n \\
		& f_{n-1} \\
		& & \ddots
	\end{pmatrix} \]
	Then part (ii) of the definition is equivalent to the statement \( d f = f d \).
\end{remark}
If \( x \in \ker d \), we write \( [x] \in H_\bullet(C) \) for its image under the map \( \ker d \to \faktor{\ker d}{\Im d} \).
\begin{remark}
	\( f(\ker d) \subseteq \ker d \) because if \( dx = 0 \), we have \( d'(f(x)) = f(d(x)) = f(0) = 0 \).
	\( f(\Im d) \subseteq \Im d' \), because if \( x = dy \), we have \( f(x) = f(d(y)) = d'(f(y)) \).
	So \( f \) descends to a well-defined homomorphism \( f_\star \colon \faktor{\ker d}{\Im d'} \to \faktor{\ker d'}{\Im d'} \) such that \( f_\star([x]) = [f(x)] \).
	So \( f_\star \colon H_k(C) \to H_k(C') \).
	This is called the map \emph{induced by} \( f \).
\end{remark}
\begin{remark}
	The composition of two chain maps is a chain map, and \( (f \circ g)_\star = f_\star \circ g_\star \).
\end{remark}
Let \( K \) be an abstract simplicial complex in \( \Delta^n \), and \( L \) be an abstract simplicial complex in \( \Delta^m \).
Let \( f \colon K \to L \) be a simplicial map, so it is determined by \( \hat f \colon \qty{0, \dots, n} \to \qty{0, \dots, m} \).
We wish to define a chain map \( f_\sharp \colon C_\bullet(K) \to C_\bullet(L) \), which will induce \( f_\star \colon H_\bullet(K) \to H_\bullet(L) \).
Perhaps the most obvious guess would be to define \( f_\sharp(e_I) = f(e_I) = e_{\hat f(I)} \).
This is not the correct definition.

First, consider \( f \colon \bm \Delta^1 \to \bm \Delta^1 \) given by \( e_0 \mapsto e_0 \), \( e_1 \mapsto e_0 \).
Then \( f(e_{01}) = e_0 \), but \( e_{01} \in C_1(\bm\Delta^1) \) and \( e_0 \in C_0(\bm\Delta^1) \).
So \( f \) does not preserve grading, and hence cannot be a chain map.

Consider also \( f \colon \bm\Delta^1 \to \bm\Delta^1 \) given by \( e_0 \mapsto e_1 \) and \( e_1 \mapsto e_0 \).
Now, \( f(e_{01}) = e_{01}, f(e_0) = e_1, f(e_1) = e_0 \), so \( df(e_{01}) = d(e_{01}) = e_1 - e_0 \) but \( fd(e_{01}) = f(e_1 - e_0) = e_0 - e_1 \).

The solution to both problems is to change our perspective on the indices \( I \).
Until now, we have defined \( I \subseteq \qty{0, \dots, n} \) and written \( I = i_0 i_1 \dots i_k \) where \( i_0 < \dots < i_k \).
Instead, we will allow \( I \in \qty{0, \dots, n}^{k+1} \), so \( I = (i_0, i_1, \dots, i_k) = i_0 i_1 \dots i_k \) with no restriction on order.
For instance, \( e_{00}, e_{10} \) are permitted.

We impose relations on the set of all such \( I \) to form an abelian group generated by equivalence classes of the \( \qty{0, \dots, n}^{k+1} \).
We will define that \( e_I = -e_{I'} \) when \( I, I' \) are related by switching two indices; so \( e_{102} = -e_{012} = e_{210} \).
If \( e_I \) contains a repetition, we require \( e_I = 0 \).

More concretely, if \( I \in \qty{0, \dots, n}^{k+1} \), let \( I' \) be the unique ordered permutation of \( I \) if \( I \) has no repetitions.
Then \( e_I = (-1)^{S(I)} e_{I'} \) if \( I \) has no repetitions, and \( e_I = 0 \) if \( I \) has a repetition, where \( (-1)^{S(I)} \) is the sign of the permutation \( \sigma \in S^{k+1} \) mapping \( I \) to \( I' \).
If we draw \( I \) and \( I' \) in order as a bipartite planar graph, connected by matching labels, \( S(I) \) is the number of crossings.
\begin{lemma}
	Let \( i_j \in I \), and suppose \( i_j \) is in position \( i_{j'} \) in \( I' \).
	Then \( S(I) - S(I_{\hat\jmath}) \equiv j - j' \mod 2 \).
\end{lemma}
\begin{proposition}
	Let \( I \in \qty{0, \dots, n}^{k+1} \).
	Then \( d(e_I) = \sum_{j=0}^k (-1)^j e_{I_{\hat\jmath}} \), where \( I_{\hat\jmath} \) is obtained from \( I \) by omitting the \( j \)th entry in the tuple \( I \).
\end{proposition}
We have already defined \( d \) for ordered sequences of indices; this proposition states that this formula holds for all sequences of indices.
\begin{proof}
	\[ \sum_{j=0}^k (-1)^k e_{I_{\hat\jmath}} = \sum_{j=0}^k (-1)^j (-1)^{S(I_{\hat\jmath})} e_{I_{\hat\jmath}'} = \sum_{j=0}^k (-1)^{j'} (-1)^{S(I)} e_{(I')_{\hat\jmath}} = (-1)^{S(I)} d(e_{I'}) = d(e_I) \]
\end{proof}
\begin{example}
	\( d(e_{210}) = (-1)^0 e_{10} + (-1)^1 e_{20} + (-1)^2 e_{21} = -e_{01} + e_{02} - e_{12} = d(-e_{012}) \), where by definition, \( e_{210} = -e_{012} \) so \( d(e_{210}) = -d(e_{012}) \).
\end{example}
\begin{definition}
	Let \( f \colon K \to L \) be a simplcial map.
	Then \( f_\sharp \colon C_k(K) \to C_k(L) \) is defined by \( f_\sharp(e_I) = e_{\hat f(I)} \) where if \( I = (i_0, \dots, i_k) \) we define \( \hat f(I) = (\hat f(i_0), \dots, \hat f(i_k)) \).
\end{definition}
This definition of \( f_\sharp \) preserves grading.
\begin{proposition}
	\( f_\sharp \) is a chain map.
\end{proposition}
\begin{proof}
	\[ d(f_\sharp(e_I)) = d(e_{\hat f(I)}) = \sum_{j=0}^k (-1)^j e_{(\hat f(I))_{\hat \jmath}} = f_\sharp\qty(\sum_{j=0}^k (-1)^j e_{I_{\hat\jmath}}) = f_\sharp(d(e_I)) \]
\end{proof}
\begin{example}
	Let \( f \colon \bm\Delta^1 \to \bm\Delta^1 \) be the simplicial map deifned by \( f(e_0) = e_0 \) and \( f(e_1) = e_0 \).
	Then \( f_\sharp(e_{01}) = e_{00} = 0 \).

	Now let \( f(e_0) = e_1 \) and \( f(e_1) = e_0 \).
	Then \( f_\sharp(e_{01}) = e_{10} = -e_{01}, f_\sharp(e_0) = e_1, f_\sharp(e_1) = e_0 \).
	So \( d(f_\sharp(e_{01})) = -d(e_{01}) = e_0 - e_1 = f(d(e_{01})) \).
\end{example}
