\subsection{Free groups and presentations}
Consider \( \pi_1(S^1 \vee S^1, x_0) \) where \( x_0 \) is the wedge point.
The universal cover is the infinite 4-valent tree \( T_\infty(4) \), so \( \pi_1(S^1 \vee S^1) \) is in bijection with \( q^{-1}(x_0) \), the vertices of \( T_\infty(4) \).
Let \( \widetilde x_0 \) be one such vertex.
If \( \widetilde x \) is a vertex, there is a unique shortest path from \( \widetilde x_0 \) to \( \widetilde x \).
This gives an `address' for \( \widetilde x \) in \( T_\infty(4) \) given by recording the type and direction of each edge used in the path.
The set of such `addresses' is in bijection with the set of \emph{reduced words} \( w = \ell_1 \dots \ell_r \) where \( r \in \mathbb N \), and each \( l_i \) is one of \( a, a^{-1}, b, b^{-1} \), such that \( w \) does not contain any substring of the form \( aa^{-1}, a^{-1}a, bb^{-1} b^{-1}b \).
Then each word \( w \) corresponds to an element \( w \in \pi_1(S^1 \vee S^1, x_0) \), the image of the shortest path under \( q \).
Note that the multiplication \( ww' \) in \( \pi_1(S^1 \vee S^1, x_0) \) corresponds to concatenation of words \( ww' \) and then the reduction of substrings such as \( aa^{-1} \).
\begin{definition}
    A \emph{free group} with generating set \( S \) is a group \( F_S \) and a subset \( S \subseteq F_s \) such that if \( G \) is a group and \( \varphi \colon S \to G \) is a map of sets, there is a unique homomorphism \( \Phi \colon F_s \to G \) with \( \eval{\Phi}_S = \varphi \).
    \begin{center}
        \begin{tikzcd}
            & F_S \arrow[d, "\Phi", dotted] \\
            S \arrow[r, "\varphi"'] \arrow[ru] & G
        \end{tikzcd}
    \end{center}
\end{definition}
\begin{remark}
    The action of taking the free group of a set is a functor from \( \mathbf{Set} \) to \( \mathbf{Grp} \), and it is left adjoint to the forgetful functor from \( \mathbf{Grp} \) to \( \mathbf{Set} \).
    This property is known as the universal property of the free group.
\end{remark}
\begin{example}
    \( \pi_1(S^1 \vee S^1) \simeq F_{\qty{a,b}} \).
    Indeed, given \( \varphi \colon \qty{a,b} \to G \), we define \( \Phi(\ell_1 \dots \ell_r) = \varphi(\ell_1) \dots \varphi(\ell_r) \), where we extend \( \varphi \) to \( \qty{a, a^{-1}, b, b^{-1}} \) by defining \( \varphi(a^{-1}) = \varphi(a)^{-1} \) and \( \varphi(b^{-1}) = \varphi(b)^{-1} \).
    This is a homomorphism: indeed, \( \Phi(ww') = \varphi(\ell_1) \dots \varphi(\ell_k) \varphi(\ell_1') \dots \varphi(\ell_k') = \Phi(w)\Phi(w') \) cancelling substrings of the form \( aa^{-1} \) as required.
\end{example}
\begin{lemma}
	Let \( F_S, F_T \) be free groups on set \( S \subseteq F_S, T \subseteq F_T \).
	Let \( \varphi \colon S \to T \) be a bijection.
	Then \( \Phi \colon F_S \to F_T \) is an isomorphism.
\end{lemma}
\begin{proof}
	Let \( \psi = \varphi^{-1} \).
	Since \( F_T \) is free, there exists a homomorphism \( \Psi \colon F_T \to F_S \) such that \( \eval{\Psi}_T = \psi \).
	Then \( \Psi \circ \Phi \colon F_S \to F_S \) has the property that for all \( s \in S \), we have \( \psi \circ \varphi(s) = s \).
	\( F_S \) is free, so there is a unique homomorphism \( \alpha \colon F_S \to F_S \) mapping \( s \in S \) to \( s \).
	So \( \alpha = \mathrm{id}_{F_S} \).
	Hence \( \Psi \circ \Phi = \mathrm{id}_{F_S} \), so by symmetry, they are inverse functions.
\end{proof}
\begin{corollary}
	If \( F_S, F_S' \) are free groups generated by \( S \), \( F_S \cong F_S' \).
	So the isomorphism type of \( F_S \) depends only on \( \abs{S} \), the cardinality of \( S \).
\end{corollary}
We therefore can write \( F_n \) for \emph{the} free group (up to isomorphism) generated by \( n \) elements \( a_1, \dots, a_n \).
Let \( X = \bigvee_{i=1}^n S^1 \) where \( x_0 \) is the wedge point, with inclusion maps \( j_n \colon S^1 \to X \).
Let \( a_i = j_{i\star}(1) \) for \( 1 \in \pi_1(S^1,1) \) be a generator.
Then \( X \) has universal cover \( \widetilde X = T_\infty(2n) \), the infinite regular \( 2n \)-valent tree.
In particular, \( \pi_1(X,x_0) \) is the set of reduced words in \( \qty{a_1^{\pm 1}, \dots, a_n^{\pm 1}} \), which is isomorphic to \( F_{2n} \).

\subsection{Presentations}
\begin{definition}
	Let \( G \) be a group and \( S \subseteq G \) be a subset.
	Let \( \mathcal S_S = \qty{H \leq G \mid S \subseteq H} \), then let \( \genset S = \bigcap_{H \in \mathcal S_S} H \) be the smallest subgroup of \( G \) containing \( S \), known as the \emph{subgroup generated by \( S \)}.
	Similarly, let \( \mathcal N_S = \qty{N \trianglelefteq G \mid S \subseteq H} \), and let \( \ngenset S = \bigcap_{H \in \mathcal N_S} H \) be the smallest normal subgroup of \( G \) containing \( S \), called the \emph{subgroup normally generated by \( S \)}.
\end{definition}
Note that \( \genset S \) is nonempty since \( 1 \in H \) for all \( H \in \mathcal S_S \).

If \( \genset S = G \), we say that \( S \) \emph{generates} \( G \).
If so, there is a unique homomorphism \( \Phi_S \colon F_S \to G \) that maps \( s \) to \( s \).
\( \Im \Phi_S \leq G \), and it contains \( S \), so \( \Phi_S \) is surjective.
\begin{definition}
	Given a set \( S \) and \( R \subseteq F_S \), we define \( \genset{S\mid R} = \faktor{F_S}{\ngenset R} \).
	If in addition \( \ngenset R = \ker \Phi_S \), then \( G \simeq \faktor{F_S}{\ker \Phi_S} = \faktor{F_S}{\ngenset R} \).
	We say \( \genset{S\mid R} \) is a \emph{presentation} for \( G \).
\end{definition}
\begin{proposition}
	Any group \( G \) admits a presentation.
\end{proposition}
\begin{proof}
	Clearly \( \genset G = G \), so let \( S = G \).
	Let \( R = \ker \Phi_G \), where \( \Phi_G \colon F_G \to G \).
	Then by construction, \( \faktor{F_S}{\ngenset R} = \faktor{F_S}{\ker \Phi_G} \cong G \).
\end{proof}
\begin{remark}
	These presentations are very large.
	It is often more useful to consider \emph{finite} presentations of \( G \), where both \( S \) and \( R \) are finite.
\end{remark}
\begin{example}
	\( \genset{a,b \mid} \simeq F_2 \).
	\( \genset{a \mid} \simeq F_1 = \pi_1(S^1,1) \simeq \mathbb Z \).
	\( \genset{a \mid a^3} \simeq \faktor{\mathbb Z}{3\mathbb Z} \).
	\( \genset{a, b \mid ab^{-3}} \simeq \mathbb Z \).
\end{example}
\begin{proposition}
	Let \( \genset{S\mid R} \) be a presentation, and let \( w \in F_S \).
	Then \( \genset{S \mid R} \simeq \genset{S \cup \qty{\alpha} \mid R \cup \qty{aw^{-1}}} \).
\end{proposition}
\begin{proof}
	We have homomorphisms \( \varphi \colon \genset{S \mid R} \to \genset{S \cup \qty{\alpha} \mid R \cup \qty{aw^{-1}}} \) mapping \( s \in S \) to \( s \), and \( \psi \colon \genset{S \cup \qty{\alpha} \mid R \cup \qty{aw^{-1}}} \to \genset{S \mid R} \) mapping \( s \in S \) to \( s \) and \( \alpha \) to \( w \).
	These are inverses.
\end{proof}
There are other operations we can apply to presentations.
If \( w \in R \), we can replace \( w \) with a conjugate \( sws^{-1} \) for \( s \in S \), and it leaves the group unchanged.
For example, \( \genset{ab \mid abb} = \genset{ab \mid bab} \).
Also, if \( w_1, w_2 \in R \), we can replace \( w_1 \) with \( w_1w_2 \), so for example, \( \genset{ab \mid babb, abb} = \genset{ab \mid b, abb} \simeq \genset{a \mid a} \simeq 1 \).
\begin{theorem}
	Given a finite set \( S \) and a finite set of relations \( R \subseteq F_S \), there is no algorithm to determine if \( \genset{S \mid R} \simeq 1 \).
\end{theorem}

\subsection{Topology}
\begin{theorem}
	Let \( U_1, U_2 \subseteq X \) be open, and \( U_1 \cap U_2 \) be path-connected with \( x_0 \in U_1 \cap U_2 \) and \( U_1 \cup U_2 = X \).
	Then \( \iota_{1\star}(\pi_1(U_1,x_0)) \cup \iota_{2\star}(\pi_1(U_2,x_0)) \) generates \( \pi_1(X,x_0) \), where \( \iota_j \colon U_j \to X \) is the inclusion.
\end{theorem}
\begin{corollary}
	Let \( U_1, U_2 \subseteq X \) be open, and \( U_1 \cap U_2 \) be path-connected and simply connected with \( x_0 \in U_1 \cap U_2 \) and \( U_1 \cup U_2 = X \), where \( U_1 \cap U_2 \) is path connected.
	Then \( X \) is simply connected.
\end{corollary}
\begin{proof}
	\( \pi_1(X,x_0) \) is generated by \( \iota_{1\star}(\pi_1(U_1,x_0)) \cup \iota_{2\star}(\pi_1(U_2,x_0)) = \qty{1} \).
\end{proof}
\begin{example}
	\( S^n = U^+ \cup U^- \), where \( U^+ = S^n = \qty{\qty(1, 0, \dots, 0)} \) and \( U^- = S^n - \qty{\qty(-1,0,\dots,0)} \).
	Then \( U^+ \simeq U^- \simeq \mathbb R^n \) by stereographic projection.
	\( U^+ \cap U^- \simeq \mathbb R^n - \qty{0} \).
	Hence \( \pi_1(U^\pm, x_0) = 1 \) since \( \mathbb R^n \) is contractible.
	\( U^+ \cap U^- \) is path connected if \( n > 1 \), so \( \pi_1(S^n, x_0) = 1 \) for \( n > 1 \).
\end{example}
\begin{example}[attaching a disk]
	If \( f \colon S^1 \to X \), let \( X \cup_f D^2 = \faktor{X \amalg D^2}{\sim} \), where \( \sim \) is the smallest equivalence relation such that \( z \sim f(z) \) for \( z \in S^1 \).
	Let \( \pi \) be the quotient map from \( X \amalg D^2 \) to \( X \cup_f D^2 \).
	Then let \( U_1 = \pi(X \cup D^2 \setminus \qty{0}) \) and \( U_2 = \pi(D^2) \).
	Then \( U_1 \cup U_2 = X \cup_f D^2 \), and \( U_1 \cap U_2 = (D^2)^\circ \setminus \qty{0} \) is path connected.
	\( \pi_1(U_2) = 1 \), so \( \pi_1(X \cup_f D^2) \) is generated by \( \pi_1(X) \).
	Note that \( f_\star \colon \pi_1(S^1, 1) \to \pi_1(X,x_0) \), so \( f_\star(1) \) lies in the kernel of the inclusion \( \pi_1(X,x_0) \to \pi_1(X \cup_f D^2, x_0) \), since \( f_\star(1) \) is null homotopic in \( X \cup_f D^2 \).
	So \( \pi_1(X \cup_f D^2) \simeq \faktor{\pi_1(X)}{\ngenset{f_\star(1)}} \).
\end{example}
