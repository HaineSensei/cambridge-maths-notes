\subsection{Free groups and presentations}
Consider \( \pi_1(S^1 \vee S^1, x_0) \) where \( x_0 \) is the wedge point.
The universal cover is the infinite 4-valent tree \( T_\infty(4) \), so \( \pi_1(S^1 \vee S^1) \) is in bijection with \( q^{-1}(x_0) \), the vertices of \( T_\infty(4) \).
Let \( \widetilde x_0 \) be one such vertex.
If \( \widetilde x \) is a vertex, there is a unique shortest path from \( \widetilde x_0 \) to \( \widetilde x \).
This gives an `address' for \( \widetilde x \) in \( T_\infty(4) \) given by recording the type and direction of each edge used in the path.
The set of such `addresses' is in bijection with the set of \emph{reduced words} \( w = \ell_1 \dots \ell_r \) where \( r \in \mathbb N \), and each \( l_i \) is one of \( a, a^{-1}, b, b^{-1} \), such that \( w \) does not contain any substring of the form \( aa^{-1}, a^{-1}a, bb^{-1} b^{-1}b \).
Then each word \( w \) corresponds to an element \( w \in \pi_1(S^1 \vee S^1, x_0) \), the image of the shortest path under \( q \).
Note that the multiplication \( ww' \) in \( \pi_1(S^1 \vee S^1, x_0) \) corresponds to concatenation of words \( ww' \) and then the reduction of substrings such as \( aa^{-1} \).
\begin{definition}
    A \emph{free group} with generating set \( S \) is a group \( F_S \) and a subset \( S \subseteq F_s \) such that if \( G \) is a group and \( \varphi \colon S \to G \) is a map of sets, there is a unique homomorphism \( \Phi \colon F_s \to G \) with \( \eval{\Phi}_S = \varphi \).
    \begin{center}
        \begin{tikzcd}
            & F_S \arrow[d, "\Phi", dotted] \\
            S \arrow[r, "\varphi"'] \arrow[ru] & G
        \end{tikzcd}
    \end{center}
\end{definition}
\begin{remark}
    The action of taking the free group of a set is a functor from \( \mathbf{Set} \) to \( \mathbf{Grp} \), and it is left adjoint to the forgetful functor from \( \mathbf{Grp} \) to \( \mathbf{Set} \).
\end{remark}
\begin{example}
    \( \pi_1(S^1 \vee S^1) \simeq F_{\qty{a,b}} \).
    Indeed, given \( \varphi \colon \qty{a,b} \to G \), we define \( \Phi(\ell_1 \dots \ell_r) = \varphi(\ell_1) \dots \varphi(\ell_r) \), where we extend \( \varphi \) to \( \qty{a, a^{-1}, b, b^{-1}} \) by defining \( \varphi(a^{-1}) = \varphi(a)^{-1} \) and \( \varphi(b^{-1}) = \varphi(b)^{-1} \).
    This is a homomorphism: indeed, \( \Phi(ww') = \varphi(\ell_1) \dots \varphi(\ell_k) \varphi(\ell_1') \dots \varphi(\ell_k') = \Phi(w)\Phi(w') \) cancelling substrings of the form \( aa^{-1} \) as required.
\end{example}
