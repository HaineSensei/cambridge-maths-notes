\subsection{Definitions}
\begin{definition}
	Let \( p : \hat X \to X \) be a continuous function.
	We say \( U \subset X \) is \emph{evenly covered} by \( p \) if \( p^{-1}(U) \simeq \coprod_{\alpha \in A} U_\alpha \) and \( \eval{p}_{U_\alpha} \colon U_\alpha \to U \) is a homeomorphism for all \( \alpha \).
\end{definition}
The topology on the coproduct \( \coprod_{\alpha \in A} U_\alpha \) is such that \( V \) is open if and only if each projection \( V \cap U_\alpha \) is open.
The topology on \( p^{-1}(U) \) is the subspace topology.
In particular, the inclusions \( \iota_\alpha \colon U_\alpha \to \coprod_{\alpha \in A} U_\alpha \to \hat X \) are continuous, as is the composition \( \iota_\alpha \qty(\eval{p}_{U_\alpha})^{-1} \colon U \to X \) since \( \eval{p}_{U_\alpha} \) is a homeomorphism.
\begin{definition}
	\( p \colon \hat X \to X \) is a \emph{covering map} if every \( x \in X \) has an open neighbourhood \( U_x \) which is evenly covered by \( p \).
	If so, we say \( \hat X \) is a \emph{covering space} of \( X \).
\end{definition}
\begin{example}
	If \( A \) is a space with the discrete topology, then \( p \colon A \times X \to X \) is a covering map, because \( p^{-1}(X) = \coprod_{\alpha \in A} \qty{\alpha} \times X \).
\end{example}
\begin{example}
	\( p \colon \mathbb R \to S^1 \) given by \( p(t) = e^{2\pi i t} \) is a covering map.
	Indeed, if \( V \subseteq \mathbb R \) is an open interval of at most unit length, let \( U = p(V) \) and then \( p^{-1}(U) = \coprod_{n \in \mathbb Z} V_n \) for \( V_n = \qty{n + v \mid v \in V} \).
\end{example}
\begin{example}
	Consider \( p_n \colon S^1 \to S^1 \) defined by \( z \mapsto z^n \).
	If \( V \subseteq S^1 \) is an open interval of length \( <\frac{2\pi}{n} \), let \( U = p_n(V) \).
	Then \( p_n^{-1}(V) = \coprod_{i \in \faktor{\mathbb Z}{n\mathbb Z}} \omega^i V \) for \( \omega = e^{\frac{2\pi i}{n}} \).
	Hence \( U \) is evenly covered.
\end{example}
\begin{definition}
	We define the \( n \)-dimensional real projective space as \( \mathbb R\mathbb P^n = \faktor{S^n}{\sim} \) where \( \sim \) is the equivalence relation generated by \( x \sim -x \) for all \( x \in S^n \).
\end{definition}
\begin{example}
	The quotient map \( p \colon S^n \to \mathbb R\mathbb P^n \) is a covering map.
	Indeed, for \( x \in S^n \), let \( V_x \) be the open hemisphere centred at \( x \).
	Then letting \( U_x = p(V_x) \), we have \( p^{-1}(U(x)) = U_x \amalg -U_x \), giving that \( U_x \) is evenly covered.
\end{example}

\subsection{Lifting paths and homotopies}
\begin{definition}
	Let \( p \colon \hat X \to X \) be a covering map, and \( f \colon Z \to X \) be continuous.
	A continuous function \( \hat f \colon Z \to \hat X \) is a \emph{lift} if \( p \circ \hat f = f \).
	Hence, the following commutative diagram holds.
	\begin{center}
		% https://tikzcd.yichuanshen.de/#N4Igdg9gJgpgziAXAbVABwnAlgFyxMJZARgBoAGAXVJADcBDAGwFcYkQAdDgC3pwAIAGiAC+pdJlz5CKMsWp0mrdsLETseAkXKl5NBizaIQALVEKYUAObwioAGYAnCAFskAJho4ISHYsPsXLwC9iA0jFhgRiBQ9HDclqLiIE6uHl4+iGT+ysah4fQARjCMAAqSmjIgjlhW3DhJDs5uiH7eSNkGuSBo5iJAA
		\begin{tikzcd}
				& \hat X \arrow[d, "p"] \\
			Z \arrow[ru, "\hat f", dashed] \arrow[r, "f"'] & X
		\end{tikzcd}
	\end{center}
\end{definition}
\begin{theorem}[Path lifting]
	Let \( p \colon (\hat X, \hat x_0) \to (X, x_0) \) be a covering map, and \( \gamma \colon [a,b] \to X \) be a path.
	Let \( \gamma(a) = x_0 \) and \( p(\hat x_0) = \hat x_0 \).
	Then there exists a unique lift \( \hat\gamma \colon [a,b] \to \hat X \) with \( \hat \gamma(a) = \hat x_0 \).
\end{theorem}
The proof will be given after some lemmas.
We say \( f \colon Z \to X \) has the \emph{(unique) lifting property at \( z \in Z \)} if for any \( \hat x \in \hat X \) such that \( p(\hat x) = f(z) \), there exists a (unique) lift \( \hat f \colon Z \to \hat X \) such that \( \hat f(z) = \hat x \).
\begin{lemma}[Lebesgue covering lemma]
	Let \( X \) be a compact metric space, and \( \qty{U_\alpha \mid \alpha \in A} \) is an open cover of \( X \).
	Then there exists \( \delta > 0 \) such that for every \( x \in X \), the open ball \( B_\delta(x) \) is contained in \( U_\alpha \) for some \( \alpha \in A \).
\end{lemma}
\begin{proof}
	We have an open cover \( \qty{U_\alpha \mid \alpha \in A} \) of \( X \), so given \( x \in X \), we can find \( \alpha_x \in A \) such that \( x \in U_{\alpha_x} \) and \( U_{\alpha_x} \) is open.
	Hence there exists \( \delta_x > 0 \) such that \( B_{2\delta_x}(x) \subset U_{\alpha_x} \).
	Then \( \qty{B_{\delta_x}(x) \mid x \in X} \) is an open cover of \( X \).
	By compactness there is a finite subcover \( \qty{B_{\delta_{x_i}}(x_i) \mid i \in \qty{1, \dots, k}} \).
	Let \( \delta = \min_{i \in \qty{1, \dots, k}} \delta_{x_i} > 0 \).
	Then for \( y \in X \), we have \( y \in B_{\delta_{x_i}}(x_i) \) for some \( i \), and \( B_\delta(y) \subset B_{\delta_{x_i} + \delta}(x_i) \subset B_{2\delta_{x_i}}(x_i) \subset U_{\alpha_x} \).
\end{proof}
\begin{lemma}
	Let \( p \colon (\hat X, \hat x_0) \to (X, x_0) \) be a covering map, and \( \gamma \colon [a,b] \to X \) be a path such that \( \gamma(a) = x_0 \).
	Let \( \Im \gamma \subset U \) where \( U \subset X \) is evenly covered.
	Then \( \gamma \) has the unique lifting property.
\end{lemma}
Note that this is simply the above path lifting theorem with an additional hypothesis.
\begin{proof}
	Since \( U \) is evenly covered, \( p^{-1}(U) = \coprod_{\alpha \in A} U_\alpha \), and \( \eval{p}_{U_\alpha} \colon U_\alpha \to U \) is a homeomorphism onto its image.
	So \( \hat x_0 \in U_{\alpha_0} \) for some \( \alpha_0 \in A \).
	Then the map \( (p_\alpha)^{-1} = \iota_\alpha \circ \qty(\eval{p}_{U_\alpha})^{-1} \colon U \to \hat X \) is continuous.
	Then \( \qty(\eval{p}_{U_{\alpha_0}})^{-1}(x_0) = \hat x_0 \), so \( \hat \gamma = \qty(p_{\alpha_0})^{-1} \circ \gamma \) is a lift of \( \gamma \) with \( \hat\gamma(a) = \hat x_0 \).

	Now we will prove uniqueness of the lift.
	Observe that \( p^{-1}(U) = U_{\alpha_0} \amalg \coprod_{\alpha \neq \alpha_0} U_\alpha \) disconnects \( p^{-1}(U) \).
	Note that \( [a,b] \) is connected.
	We have that if \( \hat\gamma \colon[a,b] \to \hat X \) with \( \hat \gamma(a) = \hat x_0 \) and \( p \circ \hat\gamma = \gamma \), then \( \Im \hat\gamma \subset p^{-1}(U) \) implies \( \Im \hat\gamma \subset U_{\alpha_0} \).
	But \( \eval{p}_{U_{\alpha_0}} \) is a homeomorphism, so we must have \( \hat \gamma = \qty(p_{\alpha_0})^{-1} \circ \gamma \).
\end{proof}
\begin{lemma}
	Let \( \gamma \colon [a,b] \to X \) and \( a' \in [a,b] \).
	If \( \eval{\gamma}_{[a,a']} \) has the unique lifting property at \( a \) and \( \eval{\gamma}_{[a',b]} \) has the unique lifting property at \( a' \), then \( \gamma \) has the unique lifting property at \( a \).
\end{lemma}
\begin{proof}
	If \( p(\hat x) = \gamma(a) \), since \( \eval{\gamma}_{[a,a']} \) has the unique lifting property at \( a \), there exists a unique lift \( \hat\gamma_1 : [a,a'] \to \hat X \) such that \( \hat\gamma_1(a) = \hat x \).
	Then \( \eval{\gamma}_{[a',b]} \) has the unique lifting property at \( a' \), so there exists a unique lift \( \hat\gamma_2 \colon [a',b] \to \hat X \) with \( \hat\gamma_2(a') = \hat\gamma_1(a') \).
	Then the composition \( \hat\gamma=\hat\gamma_1\hat\gamma_2 \) is a lift of \( \gamma \), with \( \hat\gamma(a) = \hat x \).

	For uniqueness, suppose \( \hat\gamma \) is a lift of \( \gamma \) with \( \hat\gamma(a) = \hat x \).
	Then \( \eval{\hat\gamma}_{[a,a']} \) is a lift of \( \eval{\gamma}_{[a,a']} \), so by the unique lifting property, \( \eval{\hat\gamma}_{[a,a']} \) is uniquely determined such that \( \hat\gamma(a) = \hat x \).
	Then by the unique lifting property again, \( \eval{\hat\gamma}_{[a',b]} \) is also uniquely determined such that \( \eval{\hat\gamma}_{[a',b]}(a') = \eval{\hat\gamma}_{[a,a']}(a') \).
\end{proof}
We can now prove the path lifting theorem: any \( \gamma \colon I \to X \) has the unique lifting property.
\begin{proof}
	Let \( p \colon \hat X \to X \) be a covering map.
	Hence, for all \( x \in X \), there exists an open neighbourhood \( U_x \) which is evenly covered.
	\( \qty{U_x \mid x \in X} \) is therefore an open cover of \( X \), and so \( \qty{\gamma^{-1}(U_x) \mid x \in X} \) is an open cover of \( I \).
	Since \( I \) is compact, by the Lebesgue covering lemma, there exists \( \delta > 0 \) such that for all \( t \), \( B_\delta(t) \subseteq \gamma^{-1}(U_{x(t)}) \) for some \( x(t) \).
	In other words, \( \gamma(B_\delta(t)) \subseteq U_{x(t)} \).

	Let \( n \in \mathbb N \) such that \( \frac 1n < \delta \), and \( a_i = \frac{i}{n} \in I \).
	Then \( [a_i, a_{i+1}] \subset B_\delta(a_i) \) for all \( i \).
	Hence \( \gamma[a_i,a_{i+1}] \subseteq U_{x(a_i)} \).
	Then \( [a_i,a_{i+1}] \) is connected, hence \( \gamma[a_i, a_{i+1}] \) is connected.
	Since \( U_{x(a_i)} \) is evenly covered, \( \eval{\gamma}_{[a_i,a_{i+1}]} \) has the unique lifting property.
	Then by induction on \( i \), we can see that \( \eval{\gamma}_{[0,a_i]} \) has the unique lifting property, and hence so does \( \gamma \) in its entirety.
\end{proof}
\begin{theorem}[Homotopy lifting]
	Let \( p \colon (\hat X, \hat x_0) \to (X, x_0) \) be a covering map, and \( H \colon I \times I \to X \) be a homotopy.
	Then \( H \) has the lifting property at \( (0,0) \).
\end{theorem}
It also has the unique lifting property, but this will be more easily proven later.
\begin{proof}
	\( I \) is compact and connected, so by Tychonoff's theorem, \( I \times I \) is compact and connected.
	Suppose \( \qty{U_x \mid x \in X} \) is an open cover of \( X \) consisting of evenly covered neighbourhoods of points as before.
	Then, since \( I \times I \) is compact, by the Lebesgue covering lemma there exists \( \delta > 0 \) such that for all \( v \in I \times I \), \( B_\delta(v) \subseteq H^{-1}(U_{x(v)}) \).
	In particular, \( H(B_\delta(v)) \subseteq U_{x(v)} \).

	Let \( n \in \mathbb N \) such that \( \frac{\sqrt 2}{n} < \delta \), dividing \( I \times I \) into squares of size \( \frac 1n \), ordered from left-to-right and then bottom-to-top.
	Label each square with an index \( i \in \qty{1,\dots, n^2} \).
	Let each square \( A_i \) have lower left-hand corner \( v_i \), for \( i \in \qty{1,\dots,n^2} \).
	Note that \( H(A_i) \subseteq H(B_\delta(v_i)) \subseteq U_{x(v_i)} = U_i \) is evenly covered.

	Let \( B_k = \bigcup_{i=1}^k A_i \).
	Then \( A_i \simeq I \times I \) is connected, so \( \eval{H}_{A_i} \) has the lifting property at \( v_i \).

	We show by induction that \( \eval{H}_{B_k} \) has the lifting property at \( (0,0) \).
	For \( k = 1 \), \( B_1 = A_1 \) and \( (0,0) = v_1 \), so the result follows.

	For other \( k \), suppose that \( \eval{H}_{B_k} \) has the lifting property at \( (0,0) \), so \( \hat H_k \colon B_k \to \hat X \) with \( \hat H_k(0,0) = \hat x \).
	Then \( \eval{H}_{A_{k+1}} \) has the lifting property at \( v_i \), so choose a lift \( \hat h_k \colon A_{k+1} \to \hat X \) such that \( \hat h_k(v_{k+1}) = \hat H_k(v_{k+1}) \).
	Note that \( p(\hat H_k(v_{k+1})) = H(v_{k+1}) \), so this exists by the lifting property.
	Observe that \( A_{k+1} \cap B_k = I_k \cup I_k' \) is the union of (at most) two intervals with intersection at their endpoints, so is homeomorphic to \( I \).
	Hence by uniqueness of path lifting, \( \eval{\hat H_k}_{I_k} = \eval{\hat h_k}_{I_k} \) since both are lifts of \( \eval{H}_{I_k} \) with \( v_{k+1} \mapsto \hat H_k(v_{k+1}) \).
	Similarly, \( \eval{\hat H_k}_{I_k'} = \eval{\hat h_k}_{I_k'} \).
	In other words, \( \eval{\hat H_k}_{A_{k+1} \cap B_k} = \eval{\hat h_k}_{A_{k+1} \cap B_k} \).
	By the gluing lemma, we can construct the well-defined and continuous map \( \hat H_{k+1} \colon B_{k+1} \to X \) given by \( \hat H_k \) and \( \hat h_k \) on their domains.
	Then \( \hat H_{k+1} \) is a lift of \( \eval{H}_{B_{k+1}} \).
\end{proof}
\begin{proposition}
	Let \( p \colon (\hat X, \hat x_0) \to (X, x_0) \) be a covering map.
	Let \( \gamma_0, \gamma_1 \in \Omega(X,x_0,x_1) \), and \( \gamma_0 \sim_e \gamma_1 \).
	Let \( \hat\gamma_i \) be the lift of \( \gamma_i \) to \( \hat X \) with \( \hat\gamma_i(0) = \hat x_0 \), which exists by the path lifting property.
	Then \( \hat\gamma_0 \sim_e \hat\gamma_1 \).
\end{proposition}
\begin{proof}
	Let \( H \colon I \times I \to X \) be a homotopy between \( \gamma_0 \) and \( \gamma_1 \).
	By the homotopy lifting property, there exists a lifted homotopy \( \hat H \colon I \times I \to \hat X \) such that \( \hat H(0,0) = \hat x_0 \).
	Let \( \alpha_i(t) = \hat H(t,i) \) for \( i = 0, 1 \), and \( \beta_i(t) = \hat H(i,t) \) for \( i = 0, 1 \).
	Applying the uniqueness of path lifting to the \( \alpha_i \) and the \( \beta_i \),
	\begin{enumerate}
		\item \( \alpha_0 \) is a lift of \( \gamma_0 \) with \( \alpha_0(0) = \hat x_0 \), so \( \alpha_0 = \hat\gamma_0 \);
		\item \( \beta_0 \) is a lift of \( c_{I,x_0} \) with \( \beta_0(0) = \hat x_0 \), so \( \beta_0 = \hat c_{I,x_0} = c_{I,\hat x_0} \) by uniqueness, and in particular, \( \alpha_1(0) = \beta_0(1) = \hat x_0 \);
		\item \( \alpha_1 \) is a lift of \( \gamma_1 \) with \( \alpha_1(0) = \hat x_0 \), so \( \alpha_1 = \hat\gamma_1 \);
		\item let \( \hat x_1 = \hat \gamma_0(1) \), and then \( \beta_1 \) is a lift of \( c_{I,x_1} \), so \( \beta_1(0) = \hat x_1 \), so \( \beta_1 = c_{I,\hat x_1} \).
	\end{enumerate}
	Hence \( \hat \gamma_0 \sim_e \hat \gamma_1 \) via \( \hat H \).
\end{proof}
\begin{corollary}
	Let \( p \colon (\hat X, \hat x_0) \to (X, x_0) \) be a covering map.
	Let \( \gamma_0, \gamma_1 \in \Omega(X,x_0,x_1) \), and \( \gamma_0 \sim_e \gamma_1 \).
	Then \( \hat \gamma_0(1) = \hat \gamma_1(1) \).
\end{corollary}

\subsection{Simply connected lifting}
Let \( p \colon (\hat X, \hat x_0) \to (X, x_0) \) be a covering map.
If \( \gamma \colon I \to X \) has \( \gamma(0) = x_0 \), let \( \hat \gamma \colon I \to \hat X \) be its unique lift such that \( \hat \gamma(0) = \hat x_0 \).

Consider \( \widehat{\gamma\gamma'} = \hat\gamma\widetilde\gamma' \), where \( \widetilde\gamma' \) is a lift of \( \gamma' \) such that \( \widetilde\gamma'(0) = \hat\gamma(1) \).
Note that we needed to change the start point of \( \widetilde\gamma' \) in the covering space.
\begin{definition}
	A space \( X \) is \emph{locally path-connected} if for every open set \( U \subseteq X \) and \( x \in U \), there exists an open \( V \subseteq U \) with \( x \in V \) and \( V \) path-connected.
\end{definition}
\begin{example}
	Consider
	\[ X = \qty{(x,0) \in \mathbb R^2} \cup \qty{\qty(\frac{1}{n}, y) \in \mathbb R^2, n \in \mathbb Z} \cup \qty{(0,y) \in \mathbb R^2} \]
	Then, an open set containing a point \( (0,y) \) but not \( (0,0) \) admits no smaller path-connected open neighbourhood.
\end{example}
\begin{proposition}[simply connected lifting property]
	Let \( Z \) be a simply connected (and hence path-connected) space that is also locally path-connected.
	If \( f \colon (Z,z_0) \to (X,x_0) \), then \( f \) has a unique lift \( \hat f \colon (Z,z_0) \to (\hat X, \hat x_0) \).
\end{proposition}
\begin{remark}
	This proposition then implies the path lifting and homotopy lifting properties.
\end{remark}
\begin{proof}
	Suppose \( \hat f \colon (Z,z_0) \to (\hat X, \hat x_0) \) is a lift of \( f \).
	Given \( z \in Z \), consider a path \( \gamma \in \Omega(Z,z_0,z) \), which exists since \( Z \) is path-connected.
	Then \( \hat f \circ \gamma \) is a lift of \( f \circ \gamma \), since \( p(\hat f \circ \gamma) = (p \circ \hat f)\circ \gamma = f \circ \gamma \).
	Then, \( (\hat f \circ \gamma)(0) = \hat f(z_0) = \hat x_0 \), so \( \hat f \circ \gamma = \widehat{f \circ \gamma} \) is the unique lift of \( f \circ \gamma \) given by the unique path lifting property.
	Then \( \hat f(z) = \hat f(\gamma(1)) = (\hat f \circ \gamma)(1) = \widehat{f \circ \gamma}(1) \) is uniquely determined by the unique path lifting property.
	So any such lift is unique.

	If \( \gamma_0, \gamma_1 \in \Omega(Z,z_0,z) \), \( \gamma_0 \sim_e \gamma_1 \) by simply-connectedness.
	In particular, \( f \circ \gamma_0 \sim_e f \circ \gamma_1 \), and by the homotopy lifting property, \( \widehat{f \circ \gamma_0}(1) = \widehat{f \circ \gamma_1}(1) \).
	So the choice of path \( \gamma \) used above is not relevant.
	Now, let us define \( \hat f \colon (Z,z_0) \to (\hat X, \hat x_0) \) by \( \hat f(z) = \widehat{f \circ \gamma}(1) \) where \( \gamma \in \Omega(Z,z_0,z) \) is any path from \( z_0 \) to \( z \).
	Then \( p(\hat f(z)) = p \circ \widehat{f \circ \gamma}(1) = f \circ \gamma(1) = f(z) \) since \( \widehat{f \circ \gamma} \) is a lift of \( f \circ \gamma \).
	Hence \( \hat f \) as defined is a lift.
	If \( z = z_0 \), we can take \( \gamma = c_{I,z_0} \), so \( f \circ \gamma = c_{I,x_0} \).
	In particular, \( \widehat{f \circ \gamma} = c_{I,\hat x_0} \), so \( \hat f(z) = \widehat{f \circ \gamma}(1) = \hat x_0 \) as required.

	Now, it suffices to check that \( \hat f \) is a continuous function.
	Let \( U \subseteq \hat X \) be an open neighbourhood of \( \hat f(z) \).
	We need to find an open neighbourhood \( V \subseteq Z \) of \( z \) such that \( \hat f(V) \subseteq U \).

	First, we find a subset \( U' \subset U \) with \( \hat f(z) \in U' \) such that \( p(U') \) is open and evenly covered.
	Since \( p \) is a covering map, there exists an open \( W \subseteq X \) with \( f(z) \in W \) and which is evenly covered.
	Hence \( p^{-1}(W) = \coprod_{\alpha \in A} W_\alpha \), and \( p(\hat f(z)) = f(z) \), so \( \hat f(z) \in W_{\alpha_0} \) for some \( \alpha_0 \in A \).
	Then, \( W_{\alpha_0} \subseteq \hat X \) is an open set.
	Let \( U' = U \cap W_{\alpha_0} \).
	Then \( \hat f(z) \in U' \), and \( \eval{p}_{W_{\alpha_0}} \colon W_{\alpha_0} \to W \) is a homeomorphism, so \( p(U') = p_{\alpha_0}(U') \) is open and evenly covered.

	Next, \( f \colon Z \to X \) is continuous, so we need to find an open set \( V' \subseteq Z \) with \( z \in V' \) and \( f(V') \subseteq p(U') \).
	Since \( Z \) is locally path-connected, there exists \( V \subseteq V' \) which is an open path-connected set with \( z \in V \).

	Now we need to show \( V \) satisfies the continuity requirement, that \( \hat f(V) \subseteq U \).
	Given \( z' \in V \), let \( \gamma' \in \Omega(V,z,z') \), which exists because \( V \) is path-connected.
	Then \( \Im f \circ \gamma' \subseteq f(V) \subseteq p(U') \).
	Note that \( \Im f \circ \gamma' \) is evenly covered.
	Hence \( \widetilde \gamma' = p_{\alpha_0}^{-1} \circ f \circ \gamma' \) is a lift of \( f \circ \gamma' \) with \( \widetilde \gamma'(0) = p_{\alpha_0}^{-1}(f(z)) = \hat f(z) \).
	Then \( \gamma\gamma' \in \Omega(Z,z_0,z') \), and \( \widehat{f \circ (\gamma\gamma')} = \widehat{f \circ \gamma} \widetilde \gamma' \) by the discussion at the beginning of the subsection.
	Hence \( \hat f(z') = \widehat{f \circ (\gamma\gamma')}(1) = \widetilde\gamma'(1) = p_{\alpha_0}^{-1} \circ f \circ \gamma'(1) \in U' \).
	So \( \hat f(V) \subseteq U \) as required.
\end{proof}

\subsection{Universal covers}
Let \( p \colon (\hat X, \hat x_0) \to (X, x_0) \) be a covering map.
If \( \gamma \in \Omega(X,x_0) \), let \( \hat \gamma \colon I \to \hat X \) be its unique lift such that \( \hat \gamma(0) = \hat x_0 \), which exists by the path lifting property.
Then there is a map \( \varepsilon_p \colon \Omega(X,x_0) \to p^{-1}(x_0) \) by \( \gamma \mapsto \hat\gamma(1) \), since \( p(\hat\gamma(1)) = \gamma(1) = x_0 \).
By the corollary above, if \( [\gamma_0] = [\gamma_1] \) in \( \pi_1 \), we have \( \varepsilon_p(\gamma_0) = \varepsilon_p(\gamma_1) \).
In particular, \( \varepsilon_p \) descends to a well-defined map from \( \pi_1(X,x_0) \) to \( p^{-1}(x_0) \).
\begin{definition}
	A covering map \( p \colon \hat X \to X \) is a \emph{universal cover} if \( \hat X \) is simply connected.
\end{definition}
\begin{example}
	\( p \colon \mathbb R \to S^1 \) defined by \( x \mapsto e^{2 \pi i x} \) is a universal cover of \( S^1 \), since \( \mathbb R \) is contractible.
	\( p_2 \colon \mathbb R^2 \to S^1 \times S^1 = T^2 \) defined by \( p_2(x,y) = (p(x),p(y)) \) is a universal cover.
\end{example}
\begin{proposition}
	If \( p \colon (\hat X, \hat x_0) \to (X, x_0) \) is a universal cover, then \( \varepsilon_p \colon \pi_1(X,x_0) \to p^{-1}(x_0) \) is a bijection of sets.
\end{proposition}
\begin{proof}
	Suppose \( \varepsilon_p[\gamma_0] = \hat x_1 = \varepsilon_p[\gamma_1] \).
	Then \( \hat \gamma_0 \) and \( \hat \gamma_1 \) are paths in \( \Omega(\hat X, \hat x_0, \hat x_1) \).
	Since \( \hat X \) is simply connected, \( \hat \gamma_0 \sim_e \hat \gamma_1 \).
	In particular, \( \gamma_0 = p \circ \hat \gamma_0 \sim_e p \circ \hat \gamma_1 = \gamma_1 \).
	Hence \( [\gamma_0] = [\gamma_1] \), so \( \varepsilon_p \) is injective.

	Given \( \hat x \in p^{-1}(x_0) \), \( \hat X \) is path-connected as it is simply connected, so there exists a path \( \eta \in \Omega(\hat X, \hat x_0, \hat x) \).
	Since \( p(\hat x) = x_0 \), we find \( \gamma = p \circ \eta \in \Omega(X,x_0) \).
	Then \( \eta = \hat\gamma \) is the unique lift of \( \gamma \).
	In particular, \( \varepsilon_p(\gamma) = \eta(1) = \hat x \), so \( \varepsilon_p \) is surjective.
\end{proof}
\begin{example}
	Let \( p \colon (\mathbb R, 0) \to (S^1, 1) \) be defined by \( x \mapsto e^{2 \pi i x} \).
	We have \( p^{-1}(1) = \mathbb Z \).
	Then, \( \varepsilon \colon \pi_1(S^1, 1) \to \mathbb Z \) is a bijection.
\end{example}
\begin{theorem}
	\( \varepsilon_p \colon \pi_1(S^1, 1) \to \mathbb Z \) is an isomorphism of groups.
\end{theorem}
\begin{proof}
	It is a bijection, so it suffices to check that it is a homomorphism.
	Given \( n \in \mathbb Z \), we can define \( \varphi_n \colon \mathbb R \to \mathbb R \) by \( \varphi_n(x) = x + n \).
	Then, \( p \circ \varphi_n = p \).
	If \( \gamma \in \Omega(S^1, 1) \), we can find a lift \( \hat \gamma \) of \( \gamma \) with \( \hat \gamma(0) = 0 \).
	Then \( p \circ \varphi_n \circ \hat \gamma = p \circ \hat \gamma = \gamma \), so \( \varphi_n \circ \hat \gamma \) is a lift of \( \gamma \) with \( \varphi_n \circ \hat \gamma(0) = n \).

	Suppose \( \varepsilon_p[\gamma] = n \), and \( \varepsilon_p[\gamma'] = n' \).
	Then \( \hat\gamma(1) = n \), \( \hat\gamma'(1) = n' \), so \( \varphi_n \circ \hat \gamma' \) is a lift of \( \gamma' \) that starts at \( n \).
	Hence, \( \widehat{\gamma\gamma'} = \hat \gamma (\varphi_n \circ \hat \gamma') \) is a lift of the composition of paths.
	Thus, \( \varepsilon[\gamma\gamma'] = \widehat{\gamma\gamma'}(1) = \varphi_n(\hat\gamma'(1)) = n + n' \).
	So \( \varepsilon_p \) is a homomorphism.
\end{proof}
\begin{corollary}
	\( S^1 \) is not contractible.
\end{corollary}
\begin{example}
	Let \( f \colon S^1 \to S^1 \) be the identity map.
	Let \( p \colon (\mathbb R, 0) \to (S^1, 1) \) be a covering map.
	Then there is no lift of \( f \) to \( \mathbb R \).
	Otherwise, the identity map on \( \mathbb Z \) would factor through the trivial group.
	This shows that the simply connected lifting property does not extend to all path-connected spaces.
\end{example}

\subsection{Degree of maps on the circle}
\begin{lemma}
	Let \( z \in S^1 \), and \( u, v \in \Omega(S^1, z, 1) \).
	Then, the isomorphisms \( u_\sharp, v_\sharp \colon \pi_1(S^1, z) \to \pi_1(S^1, 1) \) are equal.
\end{lemma}
\begin{proof}
	Consider \( v_\sharp^{-1} \circ u_\sharp = (v^{-1})_\sharp \circ u_\sharp \).
	Note, \( (v_\sharp^{-1} \circ u_\sharp)[\gamma] = [vu^{-1}\gamma uv^{-1}] \).
	Since \( vu^{-1} \in \Omega(S^1, 1) \), we can write \( [vu^{-1}\gamma uv^{-1}] = [\eta][\gamma][\eta^{-1}] \) where \( \eta = vu^{-1} \).
	But this is exactly \( [\gamma] \), since \( \pi_1(S^1,1) \simeq \mathbb Z \) is abelian.
	Hence \( v_\sharp^{-1} \circ u_\sharp = \mathrm{id} \), and by symmetry, \( u_\sharp^{-1} \circ v_\sharp = \mathrm{id} \).
\end{proof}
\begin{definition}
	Let \( f \colon S^1 \to S^1 \), \( f(1) = z \).
	Then choose \( u \in \Omega(S^1, z, 1) \), then \( f_\star \colon \pi_1(S^1,1) \to \pi_1(S^1,z) \), giving \( u_\sharp \circ f_\star \colon \pi_1(S^1, 1) \to \pi_1(S^1, 1) \).
	This is a homomorphism \( \mathbb Z \to \mathbb Z \), so is uniquely determined by its action on 1.
	We define the \emph{degree} of \( f \), written \( \deg f \), to be \( (u_\sharp \circ f_\star)(1) \).
\end{definition}
By the above lemma, this definition does not depend on the choice of path \( u \).
\begin{example}
	Let \( \gamma_n \in \Omega(S^1,1) \) be given by \( \gamma_n(t) = e^{2\pi i n t} \) for \( n \in \mathbb Z \).
	Then \( \hat \gamma_n(t) = n t \), so \( \varepsilon_p[\gamma_n] = n \).
	The integers \( n \) correspond to the classes \( [\gamma_n] \) in \( \pi_1(S^1,1) \).

	Let \( f_n = \overline \gamma_n \colon S^1 \to S^1 \), so \( f_n(z) = z^n \).
	Then \( f_n \circ \gamma_1 = \gamma_n \), so \( f_{n\star}[\gamma_1] = [\gamma_n] \).
	Hence the degree of \( f_n \) is \( n \).
\end{example}
\begin{proposition}
	The degree of \( f_n : S^1 \to S^1 \), defined by \( z \mapsto z^n \), is \( n \).
	If \( g_0, g_1 \colon S^1 \to S^1 \), then \( g_0 \sim g_1 \) if and only if \( \deg g_0 = \deg g_1 \).
	\( g \colon S^1 \to S^1 \) extends to \( G \colon D^2 \to S^1 \) if and only if \( \deg g = 0 \).
\end{proposition}
\begin{proof}
	Suppose \( g_0 \sim g_1 \) via \( H \colon S^1 \times I \to S^1 \).
	Let \( u(t) = H(1,t) \), so \( g_{1\star} = u_\sharp \circ g_{0\star} \), where \( u \in \Omega(S^1,g_0(1),g_1(1)) \).
	Let \( v \in \Omega(S^1,g_1(1),1) \).
	Then \( uv \in \Omega(S^1,g_0(1),1) \), and so \( \deg g_1 = v_\sharp \circ g_{1\star}(1) = v_\sharp(u_\sharp\circ g_0(1)) = (uv)_\sharp \circ g_{0\star}(1) = \deg g_0 \), since \( u_\sharp[\gamma] = [u^{-1}\gamma u] \) so \( (u \circ v)_\sharp = v_\sharp \circ u_\sharp \).

	Conversely, it suffices to show that \( g \sim f_{\deg g} \) by transitivity.
	Suppose \( g(1) = 1 \).
	Then \( g = \overline \gamma \) where \( \gamma = g \circ \gamma_1 \).
	Then \( \deg g = g_\star(1) = [g \circ \gamma_1] = [\gamma] \in \pi_1(S^1,1) \).
	In particular, if \( \deg g = n \), we have \( \gamma \sim \gamma_n \), so \( g = \overline \gamma \sim \overline \gamma_n = f_n \).

	In general, if \( g(1) = e^{2\pi i x} \), then \( g \sim g_0 \) where \( g_0(z) = e^{-2\pi i x}g(z) \) via \( g_t(z) = e^{-2\pi i t x}g(z) \).
	Then \( g \sim g_0 \) so \( \deg g = \deg g_0 \), so in particular \( g \sim g_0 \sim \gamma_{\deg g} \).

	\( g \) extends to \( D^2 \) if and only if \( g \sim c_{S^1,z_0} \) for some \( z_0 \in S^1 \).
	Equivalently, \( g \sim c_{S^1,1} = f_0 \), so \( \deg g = 0 \) by above.
\end{proof}

\subsection{Fundamental theorem of algebra}
Let \( p \colon \mathbb C \to \mathbb C \) be a polynomial, so \( p(w) = w^n + a_{n-1} w^{n-1} + \dots + a_0 = w^n + q(w) \).
\begin{lemma}
	Let \( R_0 = \max \qty{1, \sum_{i=0}^{n-1} \abs{a_i}} \).
	Then if \( \abs{w} > R_0 \), \( \abs{w^n} > \abs{q(w)} \).
\end{lemma}
\begin{proof}
	Consider
	\[ \frac{\abs{q(w)}}{\abs{w^{n-1}}} \leq \sum_{i=0}^{n-1} \abs{a_i} \abs{w}^{i-n+1} \]
	Hence, if \( \abs{w} > 1 \), each term \( \abs{w}^{i-n+1} \) is at most one.
	\[ \sum_{i=0}^{n-1} \abs{a_i} \abs{w}^{i-n+1} \leq \sum_{i=0}^{n-1} \abs{a_i} \leq R_0 \]
	Hence \( \frac{\abs{q(w)}}{\abs{w}^n} < \frac{R_0}{\abs{w}} < 1 \).
\end{proof}
Consider \( g_0, g_1 \colon S^1 \to \mathbb C \setminus \qty{0} \) given by \( g_0(z) = (Rz)^n \) for some fixed \( R > R_0 \), and \( g_1(z) = p(Rz) \).
Then \( g_0 \sim g_1 \) via \( g_t(z) = p_t(Rz) \) where \( p_t(w) = w^n + tq(w) \).
This map has codomain \( \mathbb C \setminus \qty{0} \) by the above lemma.
Let \( \pi \colon \mathbb C \setminus \qty{0} \to S^1 \) be the radial projection \( w \mapsto \frac{w}{\abs{w}} \).
Then \( \pi \circ g_0, \pi \circ g_1 \colon S^1 \to S^1 \) are homotopic maps.
Therefore, \( n = \deg (\pi \circ g_0) = \deg (\pi \circ g_1) \).
\begin{theorem}
	If \( n > 0 \), \( p \) has a root \( w_0 \in \mathbb C \).
\end{theorem}
\begin{proof}
	If \( p(w) \neq 0 \) for all \( w \), \( p \colon \mathbb C \to \mathbb C \setminus \qty{0} \), so \( g_1 \) extends to \( G_1 \colon D^2 \to \mathbb C \setminus \qty{0} \) given by \( G_1(z) = p(Rz) \).
	Then \( \pi \circ G_1 \) is an extension of \( \pi \circ g_1 \).
	So \( n = \deg \pi \circ g_1 = 0 \), so we have a constant polynomial.
\end{proof}

\subsection{Wedge product}
\begin{definition}
	Let \( (X_i, x_i) \) be pointed spaces.
	The \emph{wedge product} \( \bigvee_{i=1}^n (X_i, x_i) = \faktor{\coprod_{i=1}^n (X_i, x_i)}{\sim} \) for the equivalence relation \( \sim \) generated by \( x_i \sim x_j \).
	For \( n = 2 \), we also write \( (X_1, x_1) \vee (X_2, x_2) \) for \( \bigvee_{i=1}^2 (X_i, x_i) \).
\end{definition}
If each \( X_i \) has the property that for any \( x_i, x_i' \in X_i \), there exists a homeomorphism \( \varphi \colon X_i \to X_i \) such that \( \varphi(x_i) = \varphi(x_i') \), then the particular choice of base point used in the wedge product does not matter, and the expression \( \bigvee_{i=1}^n X_i = \bigvee_{i=1}^n (X_i, x_i) \) is well-defined up to homeomorphism independent of the choice of the \( x_i \).
\begin{example}
	Consider the figure-eight \( S^1 \vee S^1 \).
	There are inclusion maps \( \iota_1, \iota_2 \colon (S^1,1) \to (S^1 \vee S^1, x_0) \) where \( x_0 \) is the point at which the two circles are joined.
	Let \( a = \iota_{1\star}(1) \in \pi_1(S^1 \vee S^1,x_0) \), and similarly let \( b = \iota_{2\star}(1) \in \pi_1(S^1 \vee S^1,x_0) \).
	The universal cover of \( S^1 \vee S^1 \) is the infinite regular 4-valent tree, \( T_\infty(4) \).
	If \( T_n(4) \) is the regular 4-valent tree of depth \( n \), \( T_\infty(4) = \bigcup_{n=1}^\infty T_n(4) \), so \( U \subseteq T_\infty(4) \) is open if and only if \( U \cap T_n(4) \) is open for all \( n \).
	There is a covering map from \( T_\infty(4) \) to \( S^1 \vee S^1 \) by mapping each edge to one of the circles.
	\( T_\infty(4) \) is simply connected, because the interval \( I \) is compact, so if \( \gamma \colon I \to T_\infty(4) \), \( \Im \gamma \subseteq T_n(4) \) for some \( n \), and each of the finite trees is contractible and therefore simply connected.

	In particular, there is a bijection \( \pi_1(S^1 \vee S^1, x_0) \to p^{-1}(\qty{x_0}) \) given by \( [\gamma] \to \varepsilon_p(\gamma) \).
	Here, \( \varepsilon_p(ab) = \widehat{ab}(1) \), but \( \varepsilon_p(ba) = \widehat{ba}(1) \neq \widehat{ab}(1) \).
	In \( \pi_1(S^1 \vee S^1, x_0) \), \( ab \neq ba \), so \( \pi_1(S^1 \vee S^1,x_0) \) is not abelian.
\end{example}

\subsection{Covering transformations}
\begin{definition}
	Let \( p_i \colon \hat X_i \to X \) be covering maps for \( i = 1, 2 \).
	A \emph{covering transformation} \( p \colon (p_1,\hat X_1) \to (p_2,\hat X_2) \) is a map \( p \colon \hat X_1 \to \hat X_2 \) such that \( p_2 \circ p = p_1 \).
	\begin{center}
		\begin{tikzcd}
			\hat X_1 \arrow[rd, "p_1"'] \arrow[rr, "p", dashed] &   & \hat X_2 \arrow[ld, "p_2"] \\
															    & X &
		\end{tikzcd}
	\end{center}
\end{definition}
\begin{remark}
	We can think of \( p \) as a lift of \( p_1 \) to \( \hat X_2 \).
	\begin{center}
		\begin{tikzcd}
			& \hat X_2 \arrow[d, "p_2"] \\
			\hat X_1 \arrow[r, "p_1"'] \arrow[ru, "p", dashed] & X
		\end{tikzcd}
	\end{center}
\end{remark}
\begin{example}
	Let \( p_1 \colon S^1 \to S^1 \) be defined by \( z \mapsto z^6 \), and \( p_2 \colon S^1 \to S^1 \) be defined by \( z \mapsto z^2 \).
	Then \( p \colon (p_1, S^1) \to (p_2,S^1) \) defined by \( z \mapsto z^3 \) is a covering transformation.
	\begin{center}
		\begin{tikzcd}
			S^1 \arrow[rd, "z \mapsto z^6"'] \arrow[rr, "z \mapsto z^3", dashed] &     & S^1 \arrow[ld, "z \mapsto z^2"] \\
																				 & S^1 &
		\end{tikzcd}
	\end{center}
\end{example}
\begin{lemma}
	Let \( X \) be locally path connected.
	If \( p \colon (p_1,\hat X_1) \to (p_2,\hat X_2) \) is a covering transformation, \( p \colon \hat X_1 \to \hat X_2 \) is a covering map.
	\begin{center}
		\begin{tikzcd}
			\hat X_1 \arrow[dd, "p_1"', bend right] \arrow[d, "p", dashed] \\
			\hat X_2 \arrow[d, "p_2"]                                      \\
			X
		\end{tikzcd}
	\end{center}
\end{lemma}
\begin{proof}
	Given \( x_2 \in \hat X_2 \), we find an open evenly covered neighbourhood \( U_{x_2} \).
	Let \( x = p_2(x_2) \in X \).
	Then \( p_1, p_2 \) are covering maps of \( X \), so there exist open neighbourhoods \( U_1, U_2 \) of \( x \) such that \( U_i \) is evenly covered by \( p_i \).
	Then \( U = U_1 \cap U_2 \) is open and evenly covered by \( p_1 \) and \( p_2 \).
	Since \( X \) is locally path connected, let \( V \subseteq U \) be an open neighbourhood of \( x \) that is path connected.
	Then \( p_1^{-1}(V) = \coprod_{\alpha \in A} V_\alpha \) and \( p_2^{-1}(V) = \coprod_{\beta \in B} V_\beta \), where \( V_\alpha \simeq V \simeq V_\beta \) are all path connected.
	Let \( x_\alpha = p_{1,\alpha}^{-1}(x) \), and \( x_\beta = p_{2,\beta}^{-1}(x) \).
	Then \( p_2(p(x_\alpha)) = p_1(x_\alpha) = x \), so \( p(x_\alpha) = x_\beta \) for some \( \beta \in B \).
	Now, \( V_\alpha, V_\beta \) are path connected, so \( p(V_\alpha) \subseteq V_\beta \) since each \( V_\beta \) is a (maximal) path-connected component of \( p_2^{-1}(V) \).
	Therefore, \( \eval{p}_{V_\alpha} \colon V_\alpha \to V_\beta \) satisfies \( p_{2,\beta} \circ \eval{p}_{V_\alpha} = p_{1,\alpha} \), so \( \eval{p}_{V_\alpha} = p_{2,\beta}^{-1} \circ p_{1,\alpha} \) is a homeomorphism.
	In particular, \( p^{-1}(V_\beta) = \coprod_{\alpha \in V, p(x_\alpha) = x_\beta} V_\alpha \), and \( \eval{p}_{V_\alpha} \colon V_\alpha \to V_\beta \) is a homeomorphism.
	So \( V_\beta \) is evenly covered, so \( p \) is indeed a covering map.
\end{proof}
% TODO path-connected or path connected?

\subsection{Uniqueness of universal covers}
Let \( X \) be a locally path connected space, and \( q \colon (\widetilde X, \widetilde x_0) \to (X,x_0) \) be a universal cover.
Let \( p \colon (\hat X, \hat x_0) \to (X, x_0) \).
\begin{lemma}
	If \( p \colon \hat Y \to Y \) is a bijective covering map, then \( p \) is a homeomorphism.
\end{lemma}
\begin{proof}
	\( p \) is continuous and bijective, therefore \( p^{-1} \colon Y \to \hat Y \) exists as a map of sets.
	We must show that this map is continuous.
	Since \( p \) is a covering map, \( Y \) has an open cover \( \qty{U_y \mid y \in Y} \) such that \( U_y \) is evenly covered.
	In particular, \( \eval{p^{-1}}_{U_y} \colon U_y \to p^{-1}(U_y) \) is a homeomorphism.
	Hence \( p^{-1} \) is continuous.
\end{proof}
Recall that if \( p_i \colon \hat X_i \to X \) are covering maps, a covering transformation from \( (p_1, \hat X_1) \) to \( (p_2, \hat X_2) \) is a lift \( \hat p_1 \) of \( p_1 \) to \( X_2 \).
\( \hat p_1 \) is a covering isomorphism if it is bijective.
Then, by the lemma, it is a homeomorphism.
\begin{proposition}
	Let \( X \) be a locally path connected space, and \( q \colon (\widetilde X, \widetilde x_0) \to (X,x_0) \) be a universal cover.
	Let \( p \colon (\hat X, \hat x_0) \to (X, x_0) \).
	Then there is a unique covering transformation \( \hat q \colon (p, \hat X) \to (q, \widetilde X) \)
	\begin{center}
		\begin{tikzcd}
			& (\hat X, \hat x_0) \arrow[d, "p"] \\
			(\widetilde X, \widetilde x_0) \arrow[r, "q"'] \arrow[ru, "\hat q", dashed] & (X, x_0)
		\end{tikzcd}
	\end{center}
\end{proposition}
\begin{proof}
	Note that \( \widetilde X \) is simply connected, and since \( X \) is locally path connected, so is \( \widetilde X \).
	So existence and uniqueness of \( \hat q \) is exactly the simply connected lifting property.
\end{proof}
\begin{corollary}
	If \( p \) is also a universal cover, \( \hat q \) is a covering isomorphism, and in particular, \( \hat X \simeq \widetilde X \).
\end{corollary}
\begin{proof}
	\( \widetilde X \) is simply connected, so \( \hat q \colon \widetilde X \to \hat X \) is a universal cover.
	Hence, there is a bijection between points \( \hat q^{-1}(\hat x) \) and elements \( \pi_1(\hat X, \hat x) \).
	But this is the one-element set, since \( \hat X \) is simply connected.
	So \( \hat q^{-1}(\hat x) \) has a single element, and so \( \hat q \) is a bijection.
\end{proof}
Equivalently, if \( q \colon (\widetilde X, \widetilde x_0) \to (X, x_0) \) and \( q' \colon (\widetilde X', \widetilde x_0') \to (X, x_0) \) are universal covers, there is a unique covering isomorphism \( \hat q \colon (\widetilde X, \widetilde x_0) \to (\widetilde X', \widetilde x_0') \).

\subsection{Deck groups}
\begin{definition}
	The \emph{deck group}, written \( G_D(p) \), is the set of covering automorphisms \( g \colon (p, \hat X) \to (p, \hat X) \), which forms a group under composition \( gf = g \circ f \).
	This has a left action on \( \hat X \) by \( g \cdot \hat x = g(\hat x) \).
\end{definition}
\begin{example}
	Let \( p \colon (\mathbb R, 0) \to (S^1, 1) \).
	The deck group \( G_D(p) \) is exactly \( \qty{g_n \colon \mathbb R \to \mathbb R \mid g_n(t) = t + n} \simeq \mathbb Z \).
	In this case, \( G_D(p) \simeq \pi_1(S^1, 1) \).
\end{example}
\begin{example}
	There is a bijection between \( G_D(q) \) and \( q^{-1}(x_0) \), by \( g \mapsto g(\widetilde x_0) \), by the above proposition with \( \hat X = \widetilde X \).
\end{example}
\begin{theorem}
	Let \( q \colon (\widetilde X, \widetilde x_0) \to (X, x_0) \) be a universal cover.
	Then \( G_D(q) \simeq \pi_1(X,x_0) \).
\end{theorem}
\begin{proof}
	There is a bijection between \( \pi_1(X, x_0) \) and \( q^{-1}(x_0) \) since \( q \) is a universal cover.
	By the above example, \( q^{-1}(x_0) \) is in bijection with \( G_D(q) \).
	In particular, we can map \( [\gamma] \in \pi_1(X,x_0) \) to \( \widetilde \gamma(1) \in q^{-1}(x_0) \), where \( \widetilde \gamma \) is the unique lift of \( \gamma \) starting at \( \widetilde x_0 \), and \( g(\widetilde x_0) \in q^{-1}(x_0) \) is mapped to \( g \in G_D(q) \).
	We need to check that this composed map is a homomorphism: it is already a bijection of sets.

	\( [\gamma\gamma'] \) is mapped to \( \widetilde{\gamma\gamma'}(1) = \widetilde \gamma (g_{\widetilde \gamma(1)} \circ \widetilde \gamma) \) where \( g_{\widetilde\gamma(1)} \) is the unique element of \( G_D(q) \) with \( g_{\widetilde\gamma(1)}(x_0) = \widetilde \gamma(1) \).
	Since \( g_{\widetilde\gamma(1)} \circ \widetilde\gamma' \) is a lift of \( \widetilde \gamma' \) starting at \( \widetilde \gamma(1) \), we have \( \widetilde{\gamma\gamma'}(1) = (g_{\widetilde\gamma(1)} \circ \widetilde\gamma')(1) = g_{\widetilde\gamma(1)}(\widetilde\gamma'(1)) = g_{\widetilde\gamma(1)}(g_{\widetilde\gamma'(1)}(\widetilde x_0)) \).
	So \( \widetilde{\gamma\gamma'}(1) \) is the image of \( \widetilde x_0 \) under \( g_{\widetilde\gamma(1)} \circ g_{\widetilde\gamma'(1)} \), so this is indeed a homomorphism.
\end{proof}

\subsection{Correspondence of subgroups and covers}
\begin{proposition}
	Let \( G = G_D(q) \simeq \pi_1(X, x_0) \).
	If \( H \leq G \) is a subgroup, we have a tower of covering maps
	\begin{center}
		\begin{tikzcd}
			\widetilde X \arrow[d, "\pi_H", dashed] \\
			X_H \arrow[d, "p_H"]                                      \\
			X
		\end{tikzcd}
	\end{center}
	where \( X_H = H \setminus \widetilde X \) is the quotient given by \( h \cdot x \sim x \) for all \( h \in H \).
	In particular, \( \pi_H \colon \widetilde X \to H \setminus \widetilde X \) is the quotient map, and \( p_H \colon X_H \to X \) is given by \( p_H(H \cdot x) = q(x) \).
	This is well-defined because \( q \circ h = q \) as \( h \) is a deck transformation.
	In particular, if \( H = G \), \( p_G \) is a covering isomorphism, so \( X \simeq G \setminus \widetilde X \).
\end{proposition}
A universal covering map is a quotient by the action of \( G_D(q) \simeq \pi_1(X,x_0) \).
\begin{proof}
	Let \( x \in X \).
	Then choose \( U_x \) to be evenly covered by \( q \).
	Then \( q^{-1}(U_x) = \coprod_{\alpha \in A} U_\alpha = \coprod_{g \in G_D(q)} g \cdot U_{\alpha_0} \) for \( \widetilde x_0 \in U_{\alpha_0} \).
	Then \( p_H^{-1}(U_x) = \coprod_{\beta = gH \in \text{cosets of } H} U_\beta \).
	Then \( \pi_H^{-1}(U_\beta) = \coprod_{gh \in gH} gh \cdot U_{\alpha_0} \), and \( p_H^{-1}(U_x) = \coprod U_\beta \).
	So each is evenly covered.
\end{proof}
\begin{definition}
	\( p \colon \widetilde X \to X \) is a \emph{normal cover} if \( G_D(p) \) acts transitively on \( p^{-1}(x_0) \).
\end{definition}
\begin{example}
	The universal cover \( q \) is always a normal cover.
\end{example}
\begin{proposition}
	Let \( p \colon (\hat X, \hat x_0) \to (X, x_0) \) be a covering map.
	Then \( p_\star \colon \pi_1(\hat X, \hat x_0) \to \pi_1(X, x_0) \) is injective.
	In particular, \( \Im p_\star \simeq \pi_1(\hat X, \hat x_0) \) is a subgroup of \( \pi_1(X, x_0) \).
\end{proposition}
\begin{proof}
	If \( p_\star[\gamma_0] = p_\star[\gamma_1] \), we have \( p \circ \gamma_0 \sim_e p \circ \gamma_1 \), so \( p \circ \hat\gamma_0 \sim_e p \circ \hat\gamma_1 \), so \( \gamma_0 \sim_e \gamma_1 \).
	In particular, \( [\gamma_0] = [\gamma_1] \).
\end{proof}

Let \( q \colon (\widetilde X, \widetilde x_0) \to (X, x_0) \) be a universal cover, so \( \widetilde X \) and hence \( X \) are path connected.
Suppose further that \( X \) is locally path connected, so \( \widetilde X \) is also locally path connected.
Consider
\begin{align*}
	S(X,x_0) &= \qty{H \leq \pi_1(X,x_0)} \\
	C(X,x_0) &= \faktor{\qty{(p,\hat X, \hat x_0) \mid p \colon (\hat X, \hat x_0) \to (X,x_0) \text{ is a covering map, } \hat X \text{ is path connected}}}{\sim}
\end{align*}
where \( (p, \hat X, \hat x_0) \sim (p', \hat X', \hat x_0') \) if there is a covering isomorphism \( q \colon (p, \hat X) \to (p', \hat X') \) mapping \( \hat x_0 \mapsto \hat x_0' \).
Let \( \alpha \colon S(X,x_0) \to C(X,x_0) \) be given by \( \alpha(H) = (p_H, X_H, x_{0,H}) \), where \( X_H = H \setminus \widetilde X \), so \( \widetilde X \xrightarrow{\pi_H} X_H \xrightarrow{p_H} X \) mapping \( \widetilde x_0 \) to \( x_{0,H} \).
Let \( \beta \colon C(X,x_0) \to S(X,x_0) \) be defined by \( (p, \hat X, \hat x_0) \mapsto p_\star(\pi_1(\hat X, \hat x_0)) \).
\begin{theorem}
	\( \alpha, \beta \) are inverses, and hence bijections.
\end{theorem}
\begin{remark}
	The entire group \( G = \pi_1(X,x_0) \) is mapped to \( (\mathrm{id}, X, x_0) \).
	The trivial group \( 1 \subseteq G \) is mapped to the universal cover \( (q, \widetilde X, \widetilde x_0) \).
	The index \( [G : H] \) is exactly \( \abs{p_H^{-1}(x_0)} \).
	A conjugation \( g^{-1}Hg \) corresponds to a change of base point \( (p_H, X_H, \hat\gamma(1)) \), where \( g = [\gamma] \) and \( \hat\gamma \colon I \to X_H \) is a lift of \( \gamma \) with \( \hat\gamma(0) = x_{0,H} \).
	If \( H \trianglelefteq G \) is a normal subgroup, \( p_H \) is a normal covering.
	The quotient \( \faktor{G}{H} \) corresponds to the deck group \( G_D(p_H) \).
\end{remark}
% \begin{example}
% 	Consider \( X = S^1 \vee S^1 \).
% 	Consider the subgroup \( \genset a \leq \pi_1(X,x_0) = \iota_{1\star}(\pi_1(S^1,1)) \) generated by the loop \( a \).
% \end{example}
\begin{proof}
	Consider \( \beta(\alpha(H)) = p_{H\star}(\pi_1(X_H,x_{0,H})) \).
	There are isomorphisms \( H \to \pi_1(X_H,x_{0,H}) \to p_{H\star}(\pi_1(X,x_0)) \) mapping \( [\gamma] \mapsto [\pi_H \circ \widetilde \gamma] \mapsto [p_H \circ \pi_H \circ \widetilde \gamma] = [\pi_G \circ \widetilde \gamma] = [\gamma] \), where \( \widetilde \gamma \) is a lift of \( \gamma \) such that \( \widetilde\gamma(0) = \widetilde x_0 \).
	Hence \( \beta(\alpha(H)) = H \).

	Conversely, consider \( \alpha(\beta((p, \hat X, \hat x_0))) = (p_H, X_H, x_{0,H}) \) where \( H = p_\star(\pi_1(X,x_0)) \).
	Consider
	\begin{center}
		\begin{tikzcd}
			{(X_H, x_{0,H})} \arrow[r, "p'"]                                                                     & {(\hat X,\hat x_0)} \arrow[d, "p"] \\
			{(\widetilde X, \widetilde x_0)} \arrow[r, "q"'] \arrow[ru, "\hat q" description] \arrow[u, "\pi_H"] & {(X,x_0)}
		\end{tikzcd}
	\end{center}
	We claim that \( \hat q = p' \circ \pi_H \), where \( p' \) is a covering isomorphism.
	If we can show this, we have \( (p_H, X_H, x_{0,H}) \sim (p, \hat X, \hat x_0) \), so \( \alpha \circ \beta \) is the identity on \( C(X,x_0) \).
	If \( h \in H = p_\star(\pi_1(\hat X, \hat x_0)) \), \( h = [p \circ \gamma] \) for some \( \gamma \in \Omega(\hat X, \hat x_0) \).
	Then \( \hat q(\widetilde x) = \widehat{q \circ \eta_{\widetilde x}}(1) \) where \( \eta_{\widetilde x} \in \Omega(\widetilde X, \widetilde x_0, \widetilde x) \).
	Then \( \eta_{h \cdot \widetilde x} = \eta_{h \circ \widetilde x_0} (h \circ \eta_{\widetilde x}) \), so \( q \circ \eta_{h \cdot \widetilde x} = (q \circ \eta_{h \cdot \widetilde x_0})(q \circ \eta_{\widetilde x}) = (p \circ \gamma)(q \circ \eta_{\widetilde x}) \), so in particular, \( \widehat{q \circ \eta_{h \cdot \widetilde x}} = (\gamma)(\widehat{q \circ \eta_{\widetilde x}}) \).
	Hence \( \hat q(h \cdot \widetilde x) = (q \circ \eta_{h \cdot \widetilde x})(1) = \widehat{q \circ \eta_{\widetilde x}}(1) = \hat q(\widetilde x) \), so \( \hat q \) factors as shown.
	\( \hat X \) is connected, so \( p' \) is surjective, so it is bijective and hence a covering isomorphism.
\end{proof}
