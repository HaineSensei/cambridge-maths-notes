\subsection{Definitions}
\begin{definition}
	Let \( p : \hat X \to X \) be a continuous function.
	We say \( U \subset X \) is \emph{evenly covered} by \( p \) if \( p^{-1}(U) \simeq \coprod_{\alpha \in A} U_\alpha \) and \( \eval{p}_{U_\alpha} \colon U_\alpha \to U \) is a homeomorphism for all \( \alpha \).
\end{definition}
The topology on the coproduct \( \coprod_{\alpha \in A} U_\alpha \) is such that \( V \) is open if and only if each projection \( V \cap U_\alpha \) is open.
The topology on \( p^{-1}(U) \) is the subspace topology.
In particular, the inclusions \( \iota_\alpha \colon U_\alpha \to \coprod_{\alpha \in A} U_\alpha \to \hat X \) are continuous, as is the composition \( \iota_\alpha \qty(\eval{p}_{U_\alpha})^{-1} \colon U \to X \) since \( \eval p_{U_\alpha} \) is a homeomorphism.
\begin{definition}
	\( p \colon \hat X \to X \) is a \emph{covering map} if every \( x \in X \) has an open neighbourhood \( U_x \) which is evenly covered by \( p \).
	If so, we say \( \hat X \) is a \emph{covering space} of \( X \).
\end{definition}
\begin{example}
	If \( A \) is a space with the discrete topology, then \( p \colon A \times X \to X \) is a covering map, because \( p^{-1}(X) = \coprod_{\alpha \in A} \qty{\alpha} \times X \).
\end{example}
\begin{example}
	\( p \colon \mathbb R \to S^1 \) given by \( p(t) = e^{2\pi i t} \) is a covering map.
	Indeed, if \( V \subseteq \mathbb R \) is an open interval of at most unit length, let \( U = p(V) \) and then \( p^{-1}(U) = \coprod_{n \in \mathbb Z} V_n \) for \( V_n = \qty{n + v \mid v \in V} \).
\end{example}
\begin{example}
	Consider \( p_n \colon S^1 \to S^1 \) defined by \( z \mapsto z^n \).
	If \( V \subseteq S^1 \) is an open interval of length \( <\frac{2\pi}{n} \), let \( U = p_n(V) \).
	Then \( p_n^{-1}(V) = \coprod_{i \in \faktor{\mathbb Z}{n\mathbb Z}} \omega^i V \) for \( \omega = e^{\frac{2\pi i}{n}} \).
	Hence \( U \) is evenly covered.
\end{example}
\begin{definition}
	We define the \( n \)-dimensional real projective space as \( \mathbb R\mathbb P^n = \faktor{S^n}{\sim} \) where \( \sim \) is the equivalence relation generated by \( x \sim -x \) for all \( x \in S^n \).
\end{definition}
\begin{example}
	The quotient map \( p \colon S^n \to \mathbb R\mathbb P^n \) is a covering map.
	Indeed, for \( x \in S^n \), let \( V_x \) be the open hemisphere centred at \( x \).
	Then letting \( U_x = p(V_x) \), we have \( p^{-1}(U(x)) = U_x \sqcup -U_x \), giving that \( U_x \) is evenly covered.
\end{example}

\subsection{Lifting paths and homotopies}
\begin{definition}
	Let \( p \colon \hat X \to X \) be a covering map, and \( f \colon Z \to X \) be continuous.
	A continuous function \( \hat f \colon Z \to \hat X \) is a \emph{lift} if \( p \circ \hat f = f \).
	Hence, the following commutative diagram holds.
	\begin{center}
		% https://tikzcd.yichuanshen.de/#N4Igdg9gJgpgziAXAbVABwnAlgFyxMJZARgBoAGAXVJADcBDAGwFcYkQAdDgC3pwAIAGiAC+pdJlz5CKMsWp0mrdsLETseAkXKl5NBizaIQALVEKYUAObwioAGYAnCAFskAJho4ISHYsPsXLwC9iA0jFhgRiBQ9HDclqLiIE6uHl4+iGT+ysah4fQARjCMAAqSmjIgjlhW3DhJDs5uiH7eSNkGuSBo5iJAA
		\begin{tikzcd}
				& \hat X \arrow[d, "p"] \\
			Z \arrow[ru, "\hat f", dashed] \arrow[r, "f"'] & X
		\end{tikzcd}
	\end{center}
\end{definition}
\begin{theorem}[Path lifting]
	Let \( p \colon (\hat X, \hat x_0) \to (X, x_0) \) be a covering map, and \( \gamma \colon [a,b] \to X \) be a path.
	Let \( \gamma(a) = x_0 \) and \( p(\hat x_0) = \hat x_0 \).
	Then there exists a unique lift \( \hat\gamma \colon [a,b] \to \hat X \) with \( \hat \gamma(a) = \hat x_0 \).
\end{theorem}
The proof will be given after some lemmas.
\begin{lemma}[Lebesgue covering lemma]
	Let \( X \) be a compact metric space, and \( \qty{U_\alpha \mid \alpha \in A} \) is an open cover of \( X \).
	Then there exists \( \delta > 0 \) such that for every \( x \in X \), the open ball \( B_\delta(x) \) is contained in \( U_\alpha \) for some \( \alpha \in A \).
\end{lemma}
\begin{proof}
	We have an open cover \( \qty{U_\alpha \mid \alpha \in A} \) of \( X \), so given \( x \in X \), we can find \( \alpha_x \in A \) such that \( x \in U_{\alpha_x} \) and \( U_{\alpha_x} \) is open.
	Hence there exists \( \delta_x > 0 \) such that \( B_{2\delta_x}(x) \subset U_{\alpha_x} \).
	Then \( \qty{B_{\delta_x}(x) \mid x \in X} \) is an open cover of \( X \).
	By compactness there is a finite subcover \( \qty{B_{\delta_{x_i}}(x_i) \mid i \in \qty{1, \dots, k}} \).
	Let \( \delta = \min_{i \in \qty{1, \dots, k}} \delta_{x_i} > 0 \).
	Then for \( y \in X \), we have \( y \in B_{\delta_{x_i}}(x_i) \) for some \( i \), and \( B_\delta(y) \subset B_{\delta_{x_i} + \delta}(x_i) \subset B_{2\delta_{x_i}}(x_i) \subset U_{\alpha_x} \).
\end{proof}
\begin{lemma}
	Let \( p \colon (\hat X, \hat x_0) \to (X, x_0) \) be a covering map, and \( \gamma \colon [a,b] \to X \) be a path.
	Let \( \gamma(a) = x_0 \) and \( p(\hat x_0) = \hat x_0 \).
	Let \( \Im \gamma \subset U \) where \( U \subset X \) is evenly covered.
	Then there exists a unique lift \( \hat\gamma \colon [a,b] \to \hat X \) with \( \hat \gamma(a) = \hat x_0 \).
\end{lemma}
Note that this is simply the above path lifting theorem with an additional hypothesis.
\begin{proof}
	Since \( U \) is evenly covered, \( p^{-1}(U) = \coprod_{\alpha \in A} U_\alpha \), and \( \eval p_{U_\alpha} \colon U_\alpha \to U \) is a homeomorphism onto its image.
	So \( \hat x_0 \in U_{\alpha_0} \) for some \( \alpha_0 \in A \).
	Then the map \( (p_\alpha)^{-1} = \iota_\alpha \circ \qty(\eval p_{U_\alpha})^{-1} \colon U \to \hat X \) is continuous.
	Then \( \qty(\eval p_{U_0})^{-1}(x_0) = \hat x_0 \), so \( \hat \gamma = \qty(p_\alpha)^{-1} \circ \gamma \) is a lift of \( \gamma \) with \( \gamma(a) = \hat x_0 \).

	Now we will prove uniqueness of the lift.
	Observe that \( p^{-1}(U) = U_{\alpha_0} \sqcup \coprod_{\alpha \neq \alpha_0} U_\alpha \) disconnects \( p^{-1}(U) \).
	Note that \( [a,b] \) is connected.
	We have that if \( \hat\gamma \colon[a,b] \to \hat X \) with \( \hat \gamma(a) = \hat x_0 \) and \( p \circ \hat\gamma = \gamma \), then \( \Im \hat\gamma \subset p^{-1}(U) \) implies \( \Im \hat\gamma \subset U_{\alpha_0} \).
	But \( \eval{p}_{U_{\alpha_0}} \) is a homeomorphism, so we must have \( \hat \gamma = \qty(p_\alpha)^{-1} \circ \gamma \).
\end{proof}
