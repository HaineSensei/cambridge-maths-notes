\subsection{Definitions}
\begin{definition}
	Let \( p : \hat X \to X \) be a continuous function.
	We say \( U \subset X \) is \emph{evenly covered} by \( p \) if \( p^{-1}(U) \simeq \coprod_{\alpha \in A} U_\alpha \) and \( \eval{p}_{U_\alpha} \colon U_\alpha \to U \) is a homeomorphism for all \( \alpha \).
\end{definition}
The topology on the coproduct \( \coprod_{\alpha \in A} U_\alpha \) is such that \( V \) is open if and only if each projection \( V \cap U_\alpha \) is open.
The topology on \( p^{-1}(U) \) is the subspace topology.
In particular, the inclusions \( \iota_\alpha \colon U_\alpha \to \coprod_{\alpha \in A} U_\alpha \to \hat X \) are continuous, as is the composition \( \iota_\alpha \qty(\eval{p}_{U_\alpha})^{-1} \colon U \to X \) since \( \eval p_{U_\alpha} \) is a homeomorphism.
\begin{definition}
	\( p \colon \hat X \to X \) is a \emph{covering map} if every \( x \in X \) has an open neighbourhood \( U_x \) which is evenly covered by \( p \).
	If so, we say \( \hat X \) is a \emph{covering space} of \( X \).
\end{definition}
\begin{example}
	If \( A \) is a space with the discrete topology, then \( p \colon A \times X \to X \) is a covering map, because \( p^{-1}(X) = \coprod_{\alpha \in A} \qty{\alpha} \times X \).
\end{example}
\begin{example}
	\( p \colon \mathbb R \to S^1 \) given by \( p(t) = e^{2\pi i t} \) is a covering map.
	Indeed, if \( V \subseteq \mathbb R \) is an open interval of at most unit length, let \( U = p(V) \) and then \( p^{-1}(U) = \coprod_{n \in \mathbb Z} V_n \) for \( V_n = \qty{n + v \mid v \in V} \).
\end{example}
\begin{example}
	Consider \( p_n \colon S^1 \to S^1 \) defined by \( z \mapsto z^n \).
	If \( V \subseteq S^1 \) is an open interval of length \( <\frac{2\pi}{n} \), let \( U = p_n(V) \).
	Then \( p_n^{-1}(V) = \coprod_{i \in \faktor{\mathbb Z}{n\mathbb Z}} \omega^i V \) for \( \omega = e^{\frac{2\pi i}{n}} \).
	Hence \( U \) is evenly covered.
\end{example}
\begin{definition}
	We define the \( n \)-dimensional real projective space as \( \mathbb R\mathbb P^n = \faktor{S^n}{\sim} \) where \( \sim \) is the equivalence relation generated by \( x \sim -x \) for all \( x \in S^n \).
\end{definition}
\begin{example}
	The quotient map \( p \colon S^n \to \mathbb R\mathbb P^n \) is a covering map.
	Indeed, for \( x \in S^n \), let \( V_x \) be the open hemisphere centred at \( x \).
	Then letting \( U_x = p(V_x) \), we have \( p^{-1}(U(x)) = U_x \sqcup -U_x \), giving that \( U_x \) is evenly covered.
\end{example}

\subsection{Lifting paths and homotopies}
\begin{definition}
	Let \( p \colon \hat X \to X \) be a covering map, and \( f \colon Z \to X \) be continuous.
	A continuous function \( \hat f \colon Z \to \hat X \) is a \emph{lift} if \( p \circ \hat f = f \).
	Hence, the following commutative diagram holds.
	\begin{center}
		% https://tikzcd.yichuanshen.de/#N4Igdg9gJgpgziAXAbVABwnAlgFyxMJZARgBoAGAXVJADcBDAGwFcYkQAdDgC3pwAIAGiAC+pdJlz5CKMsWp0mrdsLETseAkXKl5NBizaIQALVEKYUAObwioAGYAnCAFskAJho4ISHYsPsXLwC9iA0jFhgRiBQ9HDclqLiIE6uHl4+iGT+ysah4fQARjCMAAqSmjIgjlhW3DhJDs5uiH7eSNkGuSBo5iJAA
		\begin{tikzcd}
				& \hat X \arrow[d, "p"] \\
			Z \arrow[ru, "\hat f", dashed] \arrow[r, "f"'] & X
		\end{tikzcd}
	\end{center}
\end{definition}
\begin{theorem}[Path lifting]
	Let \( p \colon (\hat X, \hat x_0) \to (X, x_0) \) be a covering map, and \( \gamma \colon [a,b] \to X \) be a path.
	Let \( \gamma(a) = x_0 \) and \( p(\hat x_0) = \hat x_0 \).
	Then there exists a unique lift \( \hat\gamma \colon [a,b] \to \hat X \) with \( \hat \gamma(a) = \hat x_0 \).
\end{theorem}
The proof will be given after some lemmas.
We say \( f \colon Z \to X \) has the \emph{(unique) lifting property at \( z \in Z \)} if for any \( \hat x \in \hat X \) such that \( p(\hat x) = f(z) \), there exists a (unique) lift \( \hat f \colon Z \to \hat Z \) such that \( \hat f(z) = \hat x \).
\begin{lemma}[Lebesgue covering lemma]
	Let \( X \) be a compact metric space, and \( \qty{U_\alpha \mid \alpha \in A} \) is an open cover of \( X \).
	Then there exists \( \delta > 0 \) such that for every \( x \in X \), the open ball \( B_\delta(x) \) is contained in \( U_\alpha \) for some \( \alpha \in A \).
\end{lemma}
\begin{proof}
	We have an open cover \( \qty{U_\alpha \mid \alpha \in A} \) of \( X \), so given \( x \in X \), we can find \( \alpha_x \in A \) such that \( x \in U_{\alpha_x} \) and \( U_{\alpha_x} \) is open.
	Hence there exists \( \delta_x > 0 \) such that \( B_{2\delta_x}(x) \subset U_{\alpha_x} \).
	Then \( \qty{B_{\delta_x}(x) \mid x \in X} \) is an open cover of \( X \).
	By compactness there is a finite subcover \( \qty{B_{\delta_{x_i}}(x_i) \mid i \in \qty{1, \dots, k}} \).
	Let \( \delta = \min_{i \in \qty{1, \dots, k}} \delta_{x_i} > 0 \).
	Then for \( y \in X \), we have \( y \in B_{\delta_{x_i}}(x_i) \) for some \( i \), and \( B_\delta(y) \subset B_{\delta_{x_i} + \delta}(x_i) \subset B_{2\delta_{x_i}}(x_i) \subset U_{\alpha_x} \).
\end{proof}
\begin{lemma}
	Let \( p \colon (\hat X, \hat x_0) \to (X, x_0) \) be a covering map, and \( \gamma \colon [a,b] \to X \) be a path.
	Let \( \gamma(a) = x_0 \) and \( p(\hat x_0) = \hat x_0 \).
	Let \( \Im \gamma \subset U \) where \( U \subset X \) is evenly covered.
	Then \( \gamma \) has the unique lifting property.
\end{lemma}
Note that this is simply the above path lifting theorem with an additional hypothesis.
\begin{proof}
	Since \( U \) is evenly covered, \( p^{-1}(U) = \coprod_{\alpha \in A} U_\alpha \), and \( \eval p_{U_\alpha} \colon U_\alpha \to U \) is a homeomorphism onto its image.
	So \( \hat x_0 \in U_{\alpha_0} \) for some \( \alpha_0 \in A \).
	Then the map \( (p_\alpha)^{-1} = \iota_\alpha \circ \qty(\eval p_{U_\alpha})^{-1} \colon U \to \hat X \) is continuous.
	Then \( \qty(\eval p_{U_0})^{-1}(x_0) = \hat x_0 \), so \( \hat \gamma = \qty(p_\alpha)^{-1} \circ \gamma \) is a lift of \( \gamma \) with \( \gamma(a) = \hat x_0 \).

	Now we will prove uniqueness of the lift.
	Observe that \( p^{-1}(U) = U_{\alpha_0} \sqcup \coprod_{\alpha \neq \alpha_0} U_\alpha \) disconnects \( p^{-1}(U) \).
	Note that \( [a,b] \) is connected.
	We have that if \( \hat\gamma \colon[a,b] \to \hat X \) with \( \hat \gamma(a) = \hat x_0 \) and \( p \circ \hat\gamma = \gamma \), then \( \Im \hat\gamma \subset p^{-1}(U) \) implies \( \Im \hat\gamma \subset U_{\alpha_0} \).
	But \( \eval{p}_{U_{\alpha_0}} \) is a homeomorphism, so we must have \( \hat \gamma = \qty(p_\alpha)^{-1} \circ \gamma \).
\end{proof}
\begin{lemma}
	Let \( \gamma \colon [a,b] \to X \) and \( a' \in [a,b] \).
	If \( \eval{\gamma}_{[a,a']} \) has the unique lifting property at \( a \) and \( \eval{\gamma}_{[a',b]} \) has the unique lifting property at \( a' \), then \( \gamma \) has the unique lifting property at \( a \).
\end{lemma}
\begin{proof}
	If \( p(\hat x) = \gamma(a) \), since \( \eval{\gamma}_{[a,a']} \) has the unique lifting property at \( a \), there exists a unique lift \( \hat\gamma_1 : [a,a'] \to \hat X \) such that \( \hat\gamma_1(a) = \hat x \).
	Then \( \eval{\gamma}_{[a',b]} \) has the unique lifting property at \( a' \), so there exists a unique lift \( \hat\gamma_2 \colon [a',b] \to \hat X \) with \( \hat\gamma_2(a') = \hat\gamma_1(a') \).
	Then the composition \( \hat\gamma=\hat\gamma_1\hat\gamma_2 \) is a lift of \( \gamma \), with \( \hat\gamma(a) = \hat x \).

	For uniqueness, suppose \( \hat\gamma \) is a lift of \( \gamma \) with \( \hat\gamma(a) = \hat x \).
	Then \( \eval{\hat\gamma}_{[a,a']} \) is a lift of \( \eval{\gamma}_{[a,a']} \), so by the unique lifting property, \( \eval{\hat\gamma}_{[a,a']} \) is uniquely determined such that \( \hat\gamma(a) = \hat x \).
	Then by the unique lifting property again, \( \eval{\hat\gamma}_{[a',b]} \) is also uniquely determined such that \( \eval{\hat\gamma}_{[a',b]}(a') = \eval{\hat\gamma}_{[a,a']}(a') \).
\end{proof}
We can now prove the path lifting theorem: any \( \gamma \colon I \to X \) has the unique lifting property.
\begin{proof}
	Let \( p \colon \hat X \to X \) be a covering map.
	Hence, for all \( x \in X \), there exists an open neighbourhood \( U_x \) which is evenly covered.
	\( \qty{U_x \mid x \in X} \) is therefore an open cover of \( X \), and so \( \qty{\gamma^{-1}(U_x) \mid x \in X} \) is an open cover of \( I \).
	Since \( I \) is compact, by the Lebesgue covering lemma, there exists \( \delta > 0 \) such that for all \( t \), \( B_\delta(t) \subseteq \gamma^{-1}(U_{x(t)}) \) for some \( x(t) \).
	In other words, \( \gamma(B_\delta(t)) \subseteq U_{x(t)} \).

	Let \( n \in \mathbb N \) such that \( \frac 1n < \delta \), and \( a_i = \frac{i}{n} \in I \).
	Then \( [a_i, a_{i+1}] \subset B_\delta(a_i) \) for all \( i \).
	Hence \( \gamma[a_i,a_{i+1}] \subseteq U_{x(a_i)} \).
	Then \( [a_i,a_{i+1}] \) is connected, hence \( \gamma[a_i, a_{i+1}] \) is connected.
	Since \( U_{x(a_i)} \) is evenly covered, \( \eval{\gamma}_{[a_i,a_{i+1}]} \) has the unique lifting property.
	Then by induction on \( i \), we can see that \( \eval{\gamma}_{[0,a_i]} \) has the unique lifting property, and hence so does \( \gamma \) in its entirety.
\end{proof}
\begin{theorem}[Homotopy lifting]
	Let \( p \colon (\hat X, \hat x_0) \to (X, x_0) \) be a covering map, and \( H \colon I \times I \to X \) be a homotopy.
	Then \( H \) has the lifting property at \( (0,0) \).
\end{theorem}
It also has the unique lifting property, but this will be more easily proven later.
\begin{proof}
	\( I \) is compact and connected, so by Tychonoff's theorem, \( I \times I \) is compact and connected.
	Suppose \( \qty{U_x \mid x \in X} \) is an open cover of \( X \) consisting of evenly covered neighbourhoods of points as before.
	Then, since \( I \times I \) is compact, by the Lebesgue covering lemma there exists \( \delta > 0 \) such that for all \( v \in I \times I \), \( B_\delta(v) \subseteq H^{-1}(U_{x(v)}) \).
	In particular, \( H(B_\delta(v)) \subseteq U_{x(v)} \).

	Let \( n \in \mathbb N \) such that \( \frac{\sqrt 2}{n} < \delta \), dividing \( I \times I \) into squares of size \( \frac 1n \), ordered from left-to-right and then bottom-to-top.
	Label each square with an index \( i \in \qty{1,\dots, n^2} \).
	Let each square \( A_i \) have lower left-hand corner \( v_i \), for \( i \in \qty{1,\dots,n^2} \).
	Note that \( H(A_i) \subseteq H(B_\delta(v_i)) \subseteq U_{x(v_i)} = U_i \) is evenly covered.

	Let \( B_k = \bigcup_{i=1}^k A_i \).
	Then \( A_i \simeq I \times I \) is connected, so \( \eval{H}_{A_i} \) has the lifting property at \( v_i \).

	We show by induction that \( \eval{H}_{B_k} \) has the lifting property at \( (0,0) \).
	For \( k = 1 \), \( B_1 = A_1 \) and \( (0,0) = v_1 \), so the result follows.

	For other \( k \), suppose that \( \eval{H}_{B_k} \) has the lifting property at \( (0,0) \), so \( \hat H_k \colon B_k \to \hat X \) with \( \hat H_k(0,0) = \hat x \).
	Then \( \eval{H}_{A_{k+1}} \) has the lifting property at \( v_i \), so choose a lift \( \hat h_k \colon A_{k+1} \to \hat X \) such that \( \hat h_k(v_{k+1}) = \hat H_k(v_{k+1}) \).
	Note that \( p(\hat H_k(v_{k+1})) = H(v_{k+1}) \), so this exists by the lifting property.
	Observe that \( A_{k+1} \cap B_k = I_k \cup I_k' \) is the union of (at most) two intervals with intersection at their endpoints, so is homeomorphic to \( I \).
	Hence by uniqueness of path lifting, \( \eval{\hat H_k}_{I_k} = \eval{\hat h_k}_{I_k} \) since both are lifts of \( \eval{H}_{I_k} \) with \( v_{k+1} \mapsto \hat H_k(v_{k+1}) \).
	Similarly, \( \eval{\hat H_k}_{I_k'} = \eval{\hat h_k}_{I_k'} \).
	In other words, \( \eval{\hat H_k}_{A_{k+1} \cap B_k} = \eval{\hat h_k}_{A_{k+1} \cap B_k} \).
	By the gluing lemma, we can construct the well-defined and continuous map \( \hat H_{k+1} \colon B_{k+1} \to X \) given by \( \hat H_k \) and \( \hat h_k \) on their domains.
	Then \( \hat H_{k+1} \) is a lift of \( \eval{H}_{B_{k+1}} \).
\end{proof}
