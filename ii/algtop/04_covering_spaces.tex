\subsection{Definitions}
\begin{definition}
	Let \( p : \hat X \to X \) be a continuous function.
	We say \( U \subset X \) is \emph{evenly covered} by \( p \) if \( p^{-1}(U) \simeq \coprod_{\alpha \in A} U_\alpha \) and \( \eval{p}_{U_\alpha} \colon U_\alpha \to U \) is a homeomorphism for all \( \alpha \).
\end{definition}
The topology on the coproduct \( \coprod_{\alpha \in A} U_\alpha \) is such that \( V \) is open if and only if each projection \( V \cap U_\alpha \) is open.
The topology on \( p^{-1}(U) \) is the subspace topology.
In particular, the inclusions \( \iota_\alpha \colon U_\alpha \to \coprod_{\alpha \in A} U_\alpha \to \hat X \) are continuous, as is the composition \( \iota_\alpha \qty(\eval{p}_{U_\alpha})^{-1} \colon U \to X \) since \( \eval p_{U_\alpha} \) is a homeomorphism.
\begin{definition}
	\( p \colon \hat X \to X \) is a \emph{covering map} if every \( x \in X \) has an open neighbourhood \( U_x \) which is evenly covered by \( p \).
	If so, we say \( \hat X \) is a \emph{covering space} of \( X \).
\end{definition}
\begin{example}
	If \( A \) is a space with the discrete topology, then \( p \colon A \times X \to X \) is a covering map, because \( p^{-1}(X) = \coprod_{\alpha \in A} \qty{\alpha} \times X \).
\end{example}
\begin{example}
	\( p \colon \mathbb R \to S^1 \) given by \( p(t) = e^{2\pi i t} \) is a covering map.
	Indeed, if \( V \subseteq \mathbb R \) is an open interval of at most unit length, let \( U = p(V) \) and then \( p^{-1}(U) = \coprod_{n \in \mathbb Z} V_n \) for \( V_n = \qty{n + v \mid v \in V} \).
\end{example}
\begin{example}
	Consider \( p_n \colon S^1 \to S^1 \) defined by \( z \mapsto z^n \).
	If \( V \subseteq S^1 \) is an open interval of length \( <\frac{2\pi}{n} \), let \( U = p_n(V) \).
	Then \( p_n^{-1}(V) = \coprod_{i \in \faktor{\mathbb Z}{n\mathbb Z}} \omega^i V \) for \( \omega = e^{\frac{2\pi i}{n}} \).
	Hence \( U \) is evenly covered.
\end{example}
\begin{definition}
	We define the \( n \)-dimensional real projective space as \( \mathbb R\mathbb P^n = \faktor{S^n}{\sim} \) where \( \sim \) is the equivalence relation generated by \( x \sim -x \) for all \( x \in S^n \).
\end{definition}
\begin{example}
	The quotient map \( p \colon S^n \to \mathbb R\mathbb P^n \) is a covering map.
	Indeed, for \( x \in S^n \), let \( V_x \) be the open hemisphere centred at \( x \).
	Then letting \( U_x = p(V_x) \), we have \( p^{-1}(U(x)) = U_x \sqcup -U_x \), giving that \( U_x \) is evenly covered.
\end{example}

\subsection{Lifting paths and homotopies}
\begin{definition}
	Let \( p \colon \hat X \to X \) be a covering map, and \( f \colon Z \to X \) be continuous.
	A continuous function \( \hat f \colon Z \to \hat X \) is a \emph{lift} if \( p \circ \hat f = f \).
	Hence, the following commutative diagram holds.
	\begin{center}
		% https://tikzcd.yichuanshen.de/#N4Igdg9gJgpgziAXAbVABwnAlgFyxMJZARgBoAGAXVJADcBDAGwFcYkQAdDgC3pwAIAGiAC+pdJlz5CKMsWp0mrdsLETseAkXKl5NBizaIQALVEKYUAObwioAGYAnCAFskAJho4ISHYsPsXLwC9iA0jFhgRiBQ9HDclqLiIE6uHl4+iGT+ysah4fQARjCMAAqSmjIgjlhW3DhJDs5uiH7eSNkGuSBo5iJAA
		\begin{tikzcd}
				& \hat X \arrow[d, "p"] \\
			Z \arrow[ru, "\hat f", dashed] \arrow[r, "f"'] & X
		\end{tikzcd}
	\end{center}
\end{definition}
\begin{theorem}[Path lifting]
	Let \( p \colon (\hat X, \hat x_0) \to (X, x_0) \) be a covering map, and \( \gamma \colon [a,b] \to X \) be a path.
	Let \( \gamma(a) = x_0 \) and \( p(\hat x_0) = \hat x_0 \).
	Then there exists a unique lift \( \hat\gamma \colon [a,b] \to \hat X \) with \( \hat \gamma(a) = \hat x_0 \).
\end{theorem}
The proof will be given after some lemmas.
We say \( f \colon Z \to X \) has the \emph{(unique) lifting property at \( z \in Z \)} if for any \( \hat x \in \hat X \) such that \( p(\hat x) = f(z) \), there exists a (unique) lift \( \hat f \colon Z \to \hat Z \) such that \( \hat f(z) = \hat x \).
\begin{lemma}[Lebesgue covering lemma]
	Let \( X \) be a compact metric space, and \( \qty{U_\alpha \mid \alpha \in A} \) is an open cover of \( X \).
	Then there exists \( \delta > 0 \) such that for every \( x \in X \), the open ball \( B_\delta(x) \) is contained in \( U_\alpha \) for some \( \alpha \in A \).
\end{lemma}
\begin{proof}
	We have an open cover \( \qty{U_\alpha \mid \alpha \in A} \) of \( X \), so given \( x \in X \), we can find \( \alpha_x \in A \) such that \( x \in U_{\alpha_x} \) and \( U_{\alpha_x} \) is open.
	Hence there exists \( \delta_x > 0 \) such that \( B_{2\delta_x}(x) \subset U_{\alpha_x} \).
	Then \( \qty{B_{\delta_x}(x) \mid x \in X} \) is an open cover of \( X \).
	By compactness there is a finite subcover \( \qty{B_{\delta_{x_i}}(x_i) \mid i \in \qty{1, \dots, k}} \).
	Let \( \delta = \min_{i \in \qty{1, \dots, k}} \delta_{x_i} > 0 \).
	Then for \( y \in X \), we have \( y \in B_{\delta_{x_i}}(x_i) \) for some \( i \), and \( B_\delta(y) \subset B_{\delta_{x_i} + \delta}(x_i) \subset B_{2\delta_{x_i}}(x_i) \subset U_{\alpha_x} \).
\end{proof}
\begin{lemma}
	Let \( p \colon (\hat X, \hat x_0) \to (X, x_0) \) be a covering map, and \( \gamma \colon [a,b] \to X \) be a path.
	Let \( \gamma(a) = x_0 \) and \( p(\hat x_0) = \hat x_0 \).
	Let \( \Im \gamma \subset U \) where \( U \subset X \) is evenly covered.
	Then \( \gamma \) has the unique lifting property.
\end{lemma}
Note that this is simply the above path lifting theorem with an additional hypothesis.
\begin{proof}
	Since \( U \) is evenly covered, \( p^{-1}(U) = \coprod_{\alpha \in A} U_\alpha \), and \( \eval p_{U_\alpha} \colon U_\alpha \to U \) is a homeomorphism onto its image.
	So \( \hat x_0 \in U_{\alpha_0} \) for some \( \alpha_0 \in A \).
	Then the map \( (p_\alpha)^{-1} = \iota_\alpha \circ \qty(\eval p_{U_\alpha})^{-1} \colon U \to \hat X \) is continuous.
	Then \( \qty(\eval p_{U_0})^{-1}(x_0) = \hat x_0 \), so \( \hat \gamma = \qty(p_\alpha)^{-1} \circ \gamma \) is a lift of \( \gamma \) with \( \gamma(a) = \hat x_0 \).

	Now we will prove uniqueness of the lift.
	Observe that \( p^{-1}(U) = U_{\alpha_0} \sqcup \coprod_{\alpha \neq \alpha_0} U_\alpha \) disconnects \( p^{-1}(U) \).
	Note that \( [a,b] \) is connected.
	We have that if \( \hat\gamma \colon[a,b] \to \hat X \) with \( \hat \gamma(a) = \hat x_0 \) and \( p \circ \hat\gamma = \gamma \), then \( \Im \hat\gamma \subset p^{-1}(U) \) implies \( \Im \hat\gamma \subset U_{\alpha_0} \).
	But \( \eval{p}_{U_{\alpha_0}} \) is a homeomorphism, so we must have \( \hat \gamma = \qty(p_\alpha)^{-1} \circ \gamma \).
\end{proof}
\begin{lemma}
	Let \( \gamma \colon [a,b] \to X \) and \( a' \in [a,b] \).
	If \( \eval{\gamma}_{[a,a']} \) has the unique lifting property at \( a \) and \( \eval{\gamma}_{[a',b]} \) has the unique lifting property at \( a' \), then \( \gamma \) has the unique lifting property at \( a \).
\end{lemma}
\begin{proof}
	If \( p(\hat x) = \gamma(a) \), since \( \eval{\gamma}_{[a,a']} \) has the unique lifting property at \( a \), there exists a unique lift \( \hat\gamma_1 : [a,a'] \to \hat X \) such that \( \hat\gamma_1(a) = \hat x \).
	Then \( \eval{\gamma}_{[a',b]} \) has the unique lifting property at \( a' \), so there exists a unique lift \( \hat\gamma_2 \colon [a',b] \to \hat X \) with \( \hat\gamma_2(a') = \hat\gamma_1(a') \).
	Then the composition \( \hat\gamma=\hat\gamma_1\hat\gamma_2 \) is a lift of \( \gamma \), with \( \hat\gamma(a) = \hat x \).

	For uniqueness, suppose \( \hat\gamma \) is a lift of \( \gamma \) with \( \hat\gamma(a) = \hat x \).
	Then \( \eval{\hat\gamma}_{[a,a']} \) is a lift of \( \eval{\gamma}_{[a,a']} \), so by the unique lifting property, \( \eval{\hat\gamma}_{[a,a']} \) is uniquely determined such that \( \hat\gamma(a) = \hat x \).
	Then by the unique lifting property again, \( \eval{\hat\gamma}_{[a',b]} \) is also uniquely determined such that \( \eval{\hat\gamma}_{[a',b]}(a') = \eval{\hat\gamma}_{[a,a']}(a') \).
\end{proof}
We can now prove the path lifting theorem: any \( \gamma \colon I \to X \) has the unique lifting property.
\begin{proof}
	Let \( p \colon \hat X \to X \) be a covering map.
	Hence, for all \( x \in X \), there exists an open neighbourhood \( U_x \) which is evenly covered.
	\( \qty{U_x \mid x \in X} \) is therefore an open cover of \( X \), and so \( \qty{\gamma^{-1}(U_x) \mid x \in X} \) is an open cover of \( I \).
	Since \( I \) is compact, by the Lebesgue covering lemma, there exists \( \delta > 0 \) such that for all \( t \), \( B_\delta(t) \subseteq \gamma^{-1}(U_{x(t)}) \) for some \( x(t) \).
	In other words, \( \gamma(B_\delta(t)) \subseteq U_{x(t)} \).

	Let \( n \in \mathbb N \) such that \( \frac 1n < \delta \), and \( a_i = \frac{i}{n} \in I \).
	Then \( [a_i, a_{i+1}] \subset B_\delta(a_i) \) for all \( i \).
	Hence \( \gamma[a_i,a_{i+1}] \subseteq U_{x(a_i)} \).
	Then \( [a_i,a_{i+1}] \) is connected, hence \( \gamma[a_i, a_{i+1}] \) is connected.
	Since \( U_{x(a_i)} \) is evenly covered, \( \eval{\gamma}_{[a_i,a_{i+1}]} \) has the unique lifting property.
	Then by induction on \( i \), we can see that \( \eval{\gamma}_{[0,a_i]} \) has the unique lifting property, and hence so does \( \gamma \) in its entirety.
\end{proof}
\begin{theorem}[Homotopy lifting]
	Let \( p \colon (\hat X, \hat x_0) \to (X, x_0) \) be a covering map, and \( H \colon I \times I \to X \) be a homotopy.
	Then \( H \) has the lifting property at \( (0,0) \).
\end{theorem}
It also has the unique lifting property, but this will be more easily proven later.
\begin{proof}
	\( I \) is compact and connected, so by Tychonoff's theorem, \( I \times I \) is compact and connected.
	Suppose \( \qty{U_x \mid x \in X} \) is an open cover of \( X \) consisting of evenly covered neighbourhoods of points as before.
	Then, since \( I \times I \) is compact, by the Lebesgue covering lemma there exists \( \delta > 0 \) such that for all \( v \in I \times I \), \( B_\delta(v) \subseteq H^{-1}(U_{x(v)}) \).
	In particular, \( H(B_\delta(v)) \subseteq U_{x(v)} \).

	Let \( n \in \mathbb N \) such that \( \frac{\sqrt 2}{n} < \delta \), dividing \( I \times I \) into squares of size \( \frac 1n \), ordered from left-to-right and then bottom-to-top.
	Label each square with an index \( i \in \qty{1,\dots, n^2} \).
	Let each square \( A_i \) have lower left-hand corner \( v_i \), for \( i \in \qty{1,\dots,n^2} \).
	Note that \( H(A_i) \subseteq H(B_\delta(v_i)) \subseteq U_{x(v_i)} = U_i \) is evenly covered.

	Let \( B_k = \bigcup_{i=1}^k A_i \).
	Then \( A_i \simeq I \times I \) is connected, so \( \eval{H}_{A_i} \) has the lifting property at \( v_i \).

	We show by induction that \( \eval{H}_{B_k} \) has the lifting property at \( (0,0) \).
	For \( k = 1 \), \( B_1 = A_1 \) and \( (0,0) = v_1 \), so the result follows.

	For other \( k \), suppose that \( \eval{H}_{B_k} \) has the lifting property at \( (0,0) \), so \( \hat H_k \colon B_k \to \hat X \) with \( \hat H_k(0,0) = \hat x \).
	Then \( \eval{H}_{A_{k+1}} \) has the lifting property at \( v_i \), so choose a lift \( \hat h_k \colon A_{k+1} \to \hat X \) such that \( \hat h_k(v_{k+1}) = \hat H_k(v_{k+1}) \).
	Note that \( p(\hat H_k(v_{k+1})) = H(v_{k+1}) \), so this exists by the lifting property.
	Observe that \( A_{k+1} \cap B_k = I_k \cup I_k' \) is the union of (at most) two intervals with intersection at their endpoints, so is homeomorphic to \( I \).
	Hence by uniqueness of path lifting, \( \eval{\hat H_k}_{I_k} = \eval{\hat h_k}_{I_k} \) since both are lifts of \( \eval{H}_{I_k} \) with \( v_{k+1} \mapsto \hat H_k(v_{k+1}) \).
	Similarly, \( \eval{\hat H_k}_{I_k'} = \eval{\hat h_k}_{I_k'} \).
	In other words, \( \eval{\hat H_k}_{A_{k+1} \cap B_k} = \eval{\hat h_k}_{A_{k+1} \cap B_k} \).
	By the gluing lemma, we can construct the well-defined and continuous map \( \hat H_{k+1} \colon B_{k+1} \to X \) given by \( \hat H_k \) and \( \hat h_k \) on their domains.
	Then \( \hat H_{k+1} \) is a lift of \( \eval{H}_{B_{k+1}} \).
\end{proof}
\begin{proposition}
	Let \( p \colon (\hat X, \hat x_0) \to (X, x_0) \) be a covering map.
	Let \( \gamma_0, \gamma_1 \in \Omega(X,x_0,x_1) \), and \( \gamma_0 \sim_e \gamma_1 \).
	Let \( \hat\gamma_i \) be the lift of \( \gamma_i \) to \( \hat X \) with \( \hat\gamma_i(0) = \hat x_0 \), which exists by the path lifting property.
	Then \( \hat\gamma_0 \sim_e \hat\gamma_1 \).
\end{proposition}
\begin{proof}
	Let \( H \colon I \times I \to X \) be a homotopy between \( \gamma_0 \) and \( \gamma_1 \).
	By the homotopy lifting property, there exists a lifted homotopy \( \hat H \colon I \times I \to \hat X \) such that \( \hat H(0,0) = \hat x_0 \).
	Let \( \alpha_i(t) = \hat H(t,i) \) for \( i = 0, 1 \), and \( \beta_i(t) = \hat H(i,t) \) for \( i = 0, 1 \).
	Applying the uniqueness of path lifting to the \( \alpha_i \) and the \( \beta_i \),
	\begin{enumerate}
		\item \( \alpha_0 \) is a lift of \( \gamma_0 \) with \( \alpha_0(0) = \hat x_0 \), so \( \alpha_0 = \hat\gamma_0 \);
		\item \( \beta_0 \) is a lift of \( c_{I,x_0} \) with \( \beta_0(0) = \hat x_0 \), so \( \beta_0 = \hat c_{I,x_0} = c_{I,\hat x_0} \) by uniqueness, and in particular, \( \alpha_1(0) = \beta_0(1) = \hat x_0 \);
		\item \( \alpha_1 \) is a lift of \( \gamma_1 \) with \( \alpha_1(0) = \hat x_0 \), so \( \alpha_1 = \hat\gamma_1 \);
		\item let \( \hat x_1 = \hat \gamma_0(1) \), and then \( \beta_1 \) is a lift of \( c_{I,x_1} \), so \( \beta_1(0) = \hat x_1 \), so \( \beta_1 = c_{I,\hat x_1} \).
	\end{enumerate}
	Hence \( \hat \gamma_0 \sim_e \hat \gamma_1 \) via \( \hat H \).
\end{proof}
\begin{corollary}
	Let \( p \colon (\hat X, \hat x_0) \to (X, x_0) \) be a covering map.
	Let \( \gamma_0, \gamma_1 \in \Omega(X,x_0,x_1) \), and \( \gamma_0 \sim_e \gamma_1 \).
	Then \( \hat \gamma_0(1) = \hat \gamma_1(1) \).
\end{corollary}

\subsection{Universal covers}
Let \( p \colon (\hat X, \hat x_0) \to (X, x_0) \) be a covering map.
If \( \gamma \in \Omega(X,x_0) \), let \( \hat \gamma \colon I \to \hat X \) be its unique lift such that \( \hat \gamma(0) = \hat x_0 \), which exists by the path lifting property.
Then there is a map \( \varepsilon_p \colon \Omega(X,x_0) \to p^{-1}(x_0) \) by \( \gamma \mapsto \hat\gamma(1) \), since \( p(\hat\gamma(1)) = \gamma(1) = x_0 \).
By the corollary above, if \( [\gamma_0] = [\gamma_1] \) in \( \pi_1 \), we have \( \varepsilon_p(\gamma_0) = \varepsilon_p(\gamma_1) \).
In particular, \( \varepsilon_p \) descends to a well-defined map from \( \pi_1(X,x_0) \) to \( p^{-1}(x_0) \).
\begin{definition}
	A covering map \( p \colon \hat X \to X \) is a \emph{universal cover} if \( \hat X \) is simply connected.
\end{definition}
\begin{example}
	\( p \colon \mathbb R \to S^1 \) defined by \( x \mapsto e^{2 \pi i x} \) is a universal cover of \( S^1 \), since \( \mathbb R \) is contractible.
	\( p_2 \colon \mathbb R^2 \to S^1 \times S^1 = T^2 \) defined by \( p_2(x,y) = (p(x),p(y)) \) is a universal cover.
\end{example}
\begin{proposition}
	If \( p \colon (\hat X, \hat x_0) \to (X, x_0) \) is a universal cover, then \( \varepsilon_p \colon \pi_1(X,x_0) \to p^{-1}(x_0) \) is a bijection of sets.
\end{proposition}
\begin{proof}
	Suppose \( \varepsilon_p[\gamma_0] = \hat x_1 = \varepsilon_p[\gamma_1] \).
	Then \( \hat \gamma_0 \) and \( \hat \gamma_1 \) are paths in \( \Omega(\hat X, \hat x_0, \hat x_1 \).
	Since \( \hat X \) is simply connected, \( \hat \gamma_0 \sim_e \hat \gamma_1 \).
	In particular, \( \gamma_0 = p \circ \hat \gamma_0 \sim_e p \circ \hat \gamma_1 = \gamma_1 \).
	Hence \( [\gamma_0] = [\gamma_1] \), so \( \varepsilon_p \) is injective.

	Given \( \hat x \in p^{-1}(x_0) \), \( \hat X \) is path-connected as it is simply connected, so there exists a path \( \eta \in \Omega(\hat X, \hat x_0, \hat x) \).
	Since \( p(\hat x) = x_0 \), we find \( \gamma = p \circ \eta \in \Omega(X,x_0) \).
	Then \( \eta = \hat\gamma \) is the unique lift of \( \gamma \).
	In particular, \( \varepsilon_p(\gamma) = \eta(1) = \hat x \), so \( \varepsilon_p \) is surjective.
\end{proof}
\begin{example}
	Let \( p \colon (\mathbb R, 0) \to (S^1, 1) \) be defined by \( x \mapsto e^{2 \pi i x} \).
	We have \( p^{-1}(1) = \mathbb Z \).
	Then, \( \varepsilon \colon \pi_1(S^1, 1) \to \mathbb Z \) is a bijection.
\end{example}
\begin{theorem}
	\( \varepsilon_p \colon \pi_1(S^1, 1) \to \mathbb Z \) is an isomorphism of groups.
\end{theorem}
\begin{proof}
	It is a bijection, so it suffices to check that it is a homomorphism.
	Given \( n \in \mathbb Z \), we can define \( \varphi_n \colon \mathbb R \to \mathbb R \) by \( \varphi_n(x) = x + n \).
	Then, \( p \circ \varphi_n = p \).
	If \( \gamma \in \Omega(S^1, 1) \), we can find a lift \( \hat \gamma \) of \( \gamma \) with \( \hat \gamma(0) = 0 \).
	Then \( p \circ \varphi_n \circ \hat \gamma = p \circ \hat \gamma = \gamma \), so \( \varphi_n \circ \hat \gamma \) is a lift of \( \gamma \) with \( \varphi_n \circ \hat \gamma(0) = n \).

	Suppose \( \varepsilon_p[\gamma] = n \), and \( \varepsilon_p[\gamma'] = n' \).
	Then \( \hat\gamma(1) = n \), \( \hat\gamma'(1) = n' \), so \( \varphi_n \circ \hat \gamma' \) is a lift of \( \gamma' \) that starts at \( n \).
	Hence, \( \widehat{\gamma\gamma'} = \hat \gamma (\varphi_n \circ \hat \gamma') \) is a lift of the composition of paths.
	Thus, \( \varepsilon[\gamma\gamma'] = \widehat{\gamma\gamma'}(1) = \varphi_n(\hat\gamma'(1)) = n + n' \).
	So \( \varepsilon_p \) is a homomorphism.
\end{proof}
\begin{corollary}
	\( S^1 \) is not contractible.
\end{corollary}

\subsection{Degree of maps on the circle}
\begin{lemma}
	Let \( z \in S^1 \), and \( u, v \in \Omega(S^1, z, 1) \).
	Then we have isomorphisms \( u_\sharp, v_\sharp \colon \pi_1(S^1, z) \to \pi_1(S^1, 1) \).
	Then \( u_\sharp = v_\sharp \).
\end{lemma}
\begin{proof}
	Consider \( v_\sharp^{-1} \circ u_\sharp = (v^{-1})_\sharp \circ u_\sharp \).
	Note, \( (v_\sharp^{-1} \circ u_\sharp)[\gamma] = [vu^{-1}\gamma uv^{-1}] \).
	Since \( vu^{-1} \in \Omega(S^1, 1) \), we can write \( [vu^{-1}\gamma uv^{-1}] = [\eta][\gamma][\eta^{-1}] \) where \( \eta = vu^{-1} \).
	But this is exactly \( [\gamma] \), since \( \pi_1(S^1,1) \cong \mathbb Z \) is abelian.
	Hence \( v_\sharp^{-1} \circ u_\sharp = \mathrm{id} \), and symmetrically.
\end{proof}
\begin{definition}
	Let \( f \colon S^1 \to S^1 \), \( f(1) = z \).
	Then choose \( u \in \Omega(S^1, z, 1) \), then \( f_\star \colon \pi_1(S^1,1) \to \pi_1(S^1,z) \), giving \( u_\sharp \circ f_\star \colon \pi_1(S^1, 1) \to \pi_1(S^1, 1) \).
	This is	a homomorphism \( \mathbb Z \to \mathbb Z \), so is uniquely determined by its action on 1.
	We define the \emph{degree} of \( f \), written \( \deg f \), to be \( (u_\sharp \circ f_\star)(1) \).
\end{definition}
By the above lemma, this definition does not depend on the choice of path \( u \).
\begin{example}
	Let \( \gamma_n \in \Omega(S^1,1) \) be given by \( \gamma_n(t) = e^{2\pi i n t} \) for \( n \in \mathbb Z \).
	Then \( \hat \gamma_n(t) = n t \), so \( \varepsilon_p[\gamma_n] = n \).
	The integers \( n \) correspond to the classes \( [\gamma_n] \) in \( \pi_1(S^1,1) \).

	Let \( f_n = \overline \gamma_n \colon S^1 \to S^1 \), so \( f_n(z) = z^n \).
	Then \( f_n \circ \gamma_1 = \gamma_n \), so \( f_{n\star}[\gamma_1] = [\gamma_n] \).
	Hence the degree of \( f_n \) is \( n \).
\end{example}
\begin{proposition}
	The degree of \( f_n : S^1 \to S^1 \), defined by \( z \mapsto z^n \), is \( n \).
	If \( g_0, g_1 \colon S^1 \to S^1 \), then \( g_0 \sim g_1 \) if and only if \( \deg g_0 = \deg g_1 \).
	\( g \colon S^1 \to S^1 \) extends to \( G \colon D^2 \to S^1 \) if and only if \( \deg g = 0 \).
\end{proposition}
\begin{proof}
	Suppose \( g_0 \sim g_1 \) via \( H \colon S^1 \times I \to S^1 \).
	Let \( u(t) = H(1,t) \), so \( g_{1\star} = u_\sharp \circ g_{0\star} \), where \( u \in \Omega(S^1,g_0(1),g_1(1)) \).
	Let \( v \in \Omega(S^1,g_1(1),1) \).
	Then \( uv = \in \Omega(S^1,g_0(1),1) \), and so \( \deg g_1 = v_\sharp \circ g_{1\star}(1) = v_\sharp(u_\sharp\circ g_0(1)) = (uv)_\sharp = g_{0\star}(1) = \deg g_0 \), since \( u_\sharp[\gamma] = [u^{-1}\gamma u] \) so \( (u \circ v)_\sharp = v_\sharp \circ u_\sharp \).

	Conversely, it suffices to show that \( g \sim f_{\deg g} \) by transitivity.
	Suppose \( g(1) = 1 \).
	Then \( g = \overline \gamma \) where \( \gamma = g \circ \gamma_1 \).
	Then \( \deg g = g_\star(1) = [g \circ \gamma_1] = [\gamma] \in \pi_1(S^1,1) \).
	In particular, if \( \deg g = n \), we have \( \gamma \sim \gamma_n \), so \( g = \overline \gamma \sim \overline \gamma_n = f_n \).

	In general, if \( g(1) = e^{2\pi i x} \), then \( g \sim g_0 \) where \( g_0(z) = e^{-2\pi i x}g(z) \) via \( g_t(z) = e^{-2\pi i t x}g(z) \).
	Then \( g \sim g_0 \) so \( \deg g = \deg g_0 \), so in particular \( g \sim g_0 \sim \gamma_{\deg g} \).

	\( g_0 \) extends to \( D^2 \) if and only if \( g_0 \sim c_{S^1,z_0} \) for some \( z_0 \in S^1 \).
	Equivalently, \( g_0 \sim c_{S^1,1} = f_0 \), so \( \deg g = 0 \) by above.
\end{proof}

\subsection{Fundamental theorem of algebra}
Let \( p \colon \mathbb C \to \mathbb C \) be a polynomial, so \( p(w) = w^n + a_{n-1} w^{n-1} + \dots + a_0 = w^n + q(w) \).
\begin{lemma}
	Let \( R_0 = \max \qty{1, \sum_{i=0}^{n-1} \abs{a_i}} \).
	Then if \( \abs{w} > R_0 \), \( \abs{w_n} > \abs{q(w)} \).
\end{lemma}
\begin{proof}
	Consider
	\[ \frac{\abs{q(w)}}{\abs{w^{n-1}}} \leq \sum_{i=0}^{n-1} \abs{a_i} \abs{w}^{i-n+1} \]
	Hence, if \( \abs{w} > 1 \), each term \( \abs{w}^{i-n+1} \) is at most one.
	\[ \sum_{i=0}^{n-1} \abs{a_i} \abs{w}^{i-n+1} \leq \sum_{i=0}^{n-1} \abs{a_i} \leq R_0 \]
	Hence \( \frac{\abs{q(w)}}{\abs{w}^n} < \frac{R_0}{\abs{w}} < 1 \).
\end{proof}
Consider \( g_0, g_1 \colon S^1 \to \mathbb C \setminus \qty{0} \) given by \( g_0(z) = (Rz)^n \) for some fixed \( R > R_0 \), and \( g_1(z) = p(Rz) \).
Then \( g_0 \sim g_1 \) via \( g_t(z) = p_t(Rz) \) where \( p_t(w) = w^n + tq(w) \).
This map has codomain \( \mathbb C \setminus \qty{0} \) by the above lemma.
Let \( \pi \colon \mathbb C \setminus \qty{0} \to S^1 \) be the radial projection \( w \mapsto \frac{w}{\abs{w}} \).
Then \( \pi \circ g_0, \pi \circ g_1 \colon S^1 \to S^1 \) are homotopic maps.
Therefore, \( n = \deg (\pi \circ g_0) = \deg (\pi \circ g_1) \).
\begin{theorem}
	If \( n > 0 \), \( p \) has a root \( w_0 \in \mathbb C \).
\end{theorem}
\begin{proof}
	If \( p(w) \neq 0 \) for all \( w \), \( p \colon \mathbb C \to \mathbb C \setminus \qty{0} \), so \( g_1 \) extends to \( G_1 \colon D^2 \to \mathbb C \setminus \qty{0} \) given by \( G_1(z) = p(Rz) \).
	Then \( \pi \circ G_1 \) is an extension of \( \pi \circ g_1 \).
	So \( n = \deg \pi \circ g_1 = 0 \), so we have a constant polynomial.
\end{proof}

\subsection{Wedge product}
\begin{definition}
	Let \( (X_i, x_i) \) be pointed spaces.
	The \emph{wedge product} \( \bigvee_{i=1}^n (X_i, x_i) = \faktor{\coprod_{i=1}^n (X_i, x_i)}{\sim} \) for the equivalence relation \( \sim \) generated by \( x_i \sim x_j \).
	For \( n = 2 \), we also write \( (X_1, x_1) \vee (X_2, x_2) \) for \( \bigvee_{i=1}^2 (X_i, x_i) \).
\end{definition}
If each \( X_i \) has the property that for any \( x_i, x_i' \in X_i \), there exists a homeomorphism \( \varphi \colon X_i \to X_1 \) such that \( \varphi(x_i) = \varphi(x_i') \), then the particular choice of base point used in the wedge product does not matter, and the expression \( \bigvee_{i=1}^n X_i = \bigvee_{i=1}^n (X_i, x_i) \) is well-defined up to homeomorphism independent of the choice of the \( x_i \).
\begin{example}
	Consider the figure-eight \( S^1 \vee S^1 \).
	There are inclusion maps \( \iota_1, \iota_2 \colon (S^1,1) \to (S^1 \vee S^1, x_0) \) where \( x_0 \) is the point at which the two circles are joined.
	Let \( a = \iota_{1\star} \in \pi_1(S^1 \vee S^1,x_0) \), and similarly let \( b = \iota_{2\star} \in \pi_1(S^1 \vee S^1,x_0) \).
	The universal cover of \( S^1 \vee S^1 \) is the infinite regular 4-valent tree, \( T_\infty(4) \).
	If \( T_n(4) \) is the regular 4-valent tree of depth \( n \), \( T_\infty(4) = \bigcup_{n=1}^\infty T_n(4) \), so \( U \subseteq T_\infty(4) \) is open if and only if \( U \cap T_n(4) \) is open for all \( n \).
	There is a covering map from \( T_\infty(4) \) to \( S^1 \vee S^1 \) by mapping each edge to one of the circles.
	\( T_\infty(4) \) is simply connected, because the interval \( I \) is compact, so if \( \gamma \colon I \to T_\infty(4) \), \( \Im \gamma \subseteq T_n(4) \) for some \( n \), and each of the finite trees is contractible and therefore simply connected.

	In particular, there is a bijection \( \pi_1(S^1 \vee S^1, x_0) \to p^{-1}(\qty{x_0}) \) given by \( [\gamma] \to \varepsilon_p(\gamma) \).
	Here, \( \varepsilon_p(ab) = \widehat{ab}(1) \), but \( \varepsilon_p(ba) = \widehat{ba}(1) \neq \widehat{ab}(1) \).
	In \( \pi_1(S^1 \vee S^1, x_0) \), \( ab \neq ba \), so \( \pi_1(S^1 \vee S^1,x_0) \) is not abelian.
\end{example}
