\subsection{Zariski and Euclidean topologies}
\begin{definition}
    The \emph{Zariski topology} on \( \mathbb A^n \) is the topology where the closed sets are precisely the affine varieties.
    If \( V \subseteq \mathbb A^n \) is a (sub)variety, the Zariski topology on \( V \) is the subspace topology for the Zariski topology on \( \mathbb A^n \).
\end{definition}
\begin{remark}
    This is in fact a topology, as all of the relevant axioms ahve been proven.
\end{remark}
\begin{definition}
    The \emph{Euclidean topology} or \emph{analytic topology} on \( \mathbb A^n \) is the topology induced by the metric space structure on \( \mathbb C^n \).
    If \( V \subseteq \mathbb A^n \), the Euclidean topology on \( V \) is the subspace topology of the Euclidean topology on \( \mathbb A^n \).
\end{definition}
\begin{proposition}
    The Zariski topology on \( \mathbb A^1 \) coincides with the cofinite topology; the closed sets are exactly the finite sets.
    This topology is not Hausdorff but it is compact.
    The Euclidean topology on \( \mathbb A^1 \) is Hausdorff but not compact.
\end{proposition}
\begin{remark}
    \( \mathbb A^2 \) with the Zariski topology is not homeomorphic to \( \mathbb A^1 \times \mathbb A^1 \) with the product of the Zariski topologies.
\end{remark}
% TODO: see notes for more observations on these topologies

% TODO: new section?
\subsection{Ideals from zero sets}
\begin{theorem}[weak form of Hilbert's Nullstellensatz]
    Let \( I \triangleleft \mathbb C[\vb X] \) be a proper ideal.
    Then, \( \mathbb V(I) \neq \varnothing \).
\end{theorem}
We will prove this theorem later.
\begin{definition}
    Let \( V \subseteq \mathbb A^n \) be an affine variety.
    The \emph{ideal of functions vanishing on \( V \)} is \( I(V) = \qty{f \in \mathbb C[\vb X] \mid \forall P \in V,\, f(P) = 0} \).
\end{definition}
\begin{proposition}
    Let \( V \subseteq \mathbb A^n \) be an affine variety.
    Then,
    \begin{enumerate}
        \item If \( V = \mathbb V(S) \) where \( S \subseteq \mathbb V[\vb X] \), then \( S \subseteq I(V) \).
        In particular, \( I(V) \) is the largest ideal vanishing on \( V \).
        \item \( V = \mathbb V(I(V)) \).
        \item Varieties \( V, W \subseteq \mathbb A^n \) are equal if and only if \( I(V) = I(W) \).
    \end{enumerate}
\end{proposition}
