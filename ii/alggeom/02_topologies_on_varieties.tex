\subsection{Zariski and Euclidean topologies}
\begin{definition}
    The \emph{Zariski topology} on \( \mathbb A^n \) is the topology where the closed sets are precisely the affine varieties.
    If \( V \subseteq \mathbb A^n \) is a (sub)variety, the Zariski topology on \( V \) is the subspace topology for the Zariski topology on \( \mathbb A^n \).
\end{definition}
\begin{remark}
    This is in fact a topology, as all of the relevant axioms have been proven.
\end{remark}
\begin{definition}
    The \emph{Euclidean topology} or \emph{analytic topology} on \( \mathbb A^n \) is the topology induced by the metric space structure on \( \mathbb C^n \).
    If \( V \subseteq \mathbb A^n \), the Euclidean topology on \( V \) is the subspace topology of the Euclidean topology on \( \mathbb A^n \).
\end{definition}
\begin{proposition}
    The Zariski topology on \( \mathbb A^1 \) coincides with the cofinite topology; the closed sets are exactly the finite sets.
    This topology is not Hausdorff but it is compact.
    The Euclidean topology on \( \mathbb A^1 \) is Hausdorff but not compact.
\end{proposition}
\begin{remark}
    \( \mathbb A^2 \) with the Zariski topology is not homeomorphic to \( \mathbb A^1 \times \mathbb A^1 \) with the product of the Zariski topologies.
\end{remark}
% TODO: see notes for more observations on these topologies

% TODO: new section?
\subsection{Ideals from zero sets}
\begin{theorem}[weak form of Hilbert's Nullstellensatz]
    Let \( I \triangleleft \mathbb C[\vb X] \) be a proper ideal.
    Then, \( \mathbb V(I) \neq \varnothing \).
\end{theorem}
We will prove this theorem later.
\begin{definition}
    Let \( V \subseteq \mathbb A^n \) be an affine variety.
    The \emph{ideal of functions vanishing on \( V \)} is \( I(V) = \qty{f \in \mathbb C[\vb X] \mid \forall P \in V,\, f(P) = 0} \).
\end{definition}
\begin{proposition}
    Let \( V \subseteq \mathbb A^n \) be an affine variety.
    Then,
    \begin{enumerate}
        \item If \( V = \mathbb V(S) \) where \( S \subseteq \mathbb V[\vb X] \), then \( S \subseteq I(V) \).
        In particular, \( I(V) \) is the largest ideal vanishing on \( V \).
        \item \( V = \mathbb V(I(V)) \).
        \item Varieties \( V, W \subseteq \mathbb A^n \) are equal if and only if \( I(V) = I(W) \).
    \end{enumerate}
\end{proposition}
\begin{proof}
    Follows from the definitions.
\end{proof}
Therefore, we have an injective map \( I \) from the space of affine varieties in \( \mathbb A^n \) to the space of ideals in \( \mathbb C[\vb X] \), and \( \mathbb V \) gives a left inverse.
\begin{proposition}
    If \( V, W \) are affine varieties, \( V \subseteq W \) if and only if \( I(W) \subseteq I(V) \).
\end{proposition}
\begin{proof}
    The forward implication follows from set theory.
    For the reverse, if \( V \not\subseteq W \), we can choose \( P \in V \setminus W \).
    Since \( P \not\in \mathbb V(I(W)) \), there exists a function \( f \in I(W) \) such that \( f(P) \neq 0 \), so \( f \not\in I(V) \).
\end{proof}
\begin{proposition}
    Let \( V \) be a variety.
    Then \( V \) is irreducible if and only if \( I(V) \) is a prime ideal.
\end{proposition}
Recall that \( I(V) \) is prime when \( f_1 f_2 \in I(V) \) implies \( f_1 \in I(V) \) or \( f_2 \in I(V) \).
Geometrically, the ideal is not prime when we can find two functions where the product is zero on \( V \) but are individually not zero on all of \( V \).
\begin{proof}
    Recall that \( I(V_1 \cup V_2) = I(V_1) \cap I(V_2) \).
    Suppose \( V \) were reducible, so \( V = V_1 \cup V_2 \) where \( V_1, V_2 \neq V \).
    In particular, \( V_1 \not\subseteq V_2 \not\subseteq V_1 \).
    Now, let \( I_j = I(V_j) \), giving \( I_1 \not\supseteq I_2 \not\supseteq I_1 \), and \( I(V) = I_1 \cap I_2 \).
    Therefore, there exists \( f_1 \in I_1 \setminus I_2 \) and \( f_2 \in I_2 \setminus I_1 \).
    Each \( f_i \) is not an element of \( I(V) \), but \( f_1 f_2 \in I(V) \).
    So \( I(V) \) cannot be prime.

    Conversely, suppose \( I(V) \) is not prime, so \( f_1 f_2 \in I(V) \) but \( f_1, f_2 \not\in I(V) \).
    Define \( V_1 = V \cap \mathbb V(f_1) \) and \( V_2 = V \cap \mathbb V(f_2) \).
    Since neither \( f_i \) is contained in \( I(V) \), \( V_i \neq V \).
    Also, if \( P \in V \), we have \( f_1(P) f_2(P) = 0 \), so \( P \in V_1 \cup V_2 \).
    So \( V \) is reducible.
\end{proof}
\begin{example}
    Let \( V = \mathbb V(XY) \subset \mathbb A^2 \).
    Then \( V = \mathbb V(X) \cup \mathbb V(Y) \) is a decomposition of \( V \) into irreducible components.
    Indeed, \( \mathbb V(X) \) is irreducible, as \( I(\mathbb V(X)) = (X) \) is a prime ideal in \( \mathbb C[X, Y] \), and similarly for \( Y \).
\end{example}

\subsection{Coordinate rings}
Consider a polynomial \( f \in \mathbb C[\vb X] \).
We obtain a function \( f \colon \mathbb A^n \to \mathbb A^1 \),
If \( V \subseteq \mathbb A^n \) and \( f, g \in \mathbb C[\vb X] \), we are interested in when \( f, g \) induce the same set-theoretic function on \( V \).
We intend to show that \( f, g \) induce the same function if and only if \( f - g \in I(V) \).
Therefore, we can study polynomials modulo this relation by taking the quotient with respect to this ideal.
\begin{definition}
    Let \( V \subseteq \mathbb A^n \) be a variety.
    The \emph{coordinate ring} of \( V \), or the \emph{ring of regular functions} of \( V \), is defined as \( \faktor{\mathbb C[\vb X]}{I(V)} \), denoted \( \mathbb C[V] \) or \( \mathcal O(V) \).
\end{definition}
\begin{corollary}
    Let \( V \) be a variety.
    Then \( V \) is irreducible if and only if \( \mathbb C[V] \) is an integral domain.
\end{corollary}
\begin{remark}
    \( \mathbb C[V] \) does not precisely determine \( V \) or \( I(V) \).
    For instance, consider a surjective homomorphism \( \theta \colon \mathbb C[\vb X] \to \mathbb C[V] \), then \( \ker \theta = I \) is an ideal, and \( \mathbb V(I) \) is a variety with coordinate ring \( \mathbb C[V] \).
    However, there is not a unique such homomorphism in general.
    For instance, \( \mathbb C[X] \simeq \faktor{\mathbb C[X,Y]}{(Y)} \).
\end{remark}
\begin{definition}
    Let \( I \triangleleft \mathbb C[\vb X] \).
    We define the \emph{radical ideal} of \( I \) to be
    \[ \sqrt{I} = \qty{f \in \mathbb C[\vb X] \mid \exists m > 0,\, f^m \in I} \]
\end{definition}
This is an ideal.
\( \sqrt{\sqrt{I}} = \sqrt{I} \).
Note that \( \mathbb V(I) = \mathbb V\qty(\sqrt{I}) \).
\begin{theorem}[strong form of Hilbert's Nullstellensatz]
    Let \( I \triangleleft \mathbb C[\vb X] \) be an ideal, and \( V = \mathbb V(I) \).
    Then \( I(V) = \sqrt{I} \).
\end{theorem}
Therefore, the map \( V \mapsto I(V) \) maps precisely onto the space of radical ideals, ideals which are equal to their radicals.
\begin{example}
    Let \( V = \qty{0} \in \mathbb A^1 \).
    We can write \( V = \mathbb V(X^2) \), so its coordinate ring is
    \[ \faktor{\mathbb C[X]}{I(\mathbb V(X^2))} = \faktor{\mathbb C[X]}{\sqrt{(X^2)}} = \faktor{\mathbb C[X]}{(X)} \simeq \mathbb C \]
    In building the coordinate ring, we forget the structure of \( X^2 \).
    If we had instead considered \( \faktor{\mathbb C[X]}{(X^2)} \), we would have certain nonzero elements whose squares are zero.
\end{example}
