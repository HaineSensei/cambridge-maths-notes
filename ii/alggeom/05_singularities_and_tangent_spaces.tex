\subsection{Definitions}
Let \( V \subseteq \mathbb A^n \) be an affine hypersurface, so \( V = \mathbb V(f) \).
We assume that \( f \) is irreducible, so \( V \) is also irreducible.
Let \( P = (a_1, \dots, a_n) = (\vb a) \in V \).
An affine line through \( P \) has the form \( L = \qty{(a_1 + b_1 t, \dots, a_n + b_n t) \mid t \in \mathbb C} \) for \( (\vb b) \in \mathbb C^n \setminus \qty{\vb 0} \) is fixed.

The intersection \( V \cap L \) is the set of points on \( L \) where \( f \) vanishes.
This gives \( 0 = f(a_1 + b_1 t, \dots, a_n + b_n t) = g(t) = \sum_r c_r t^r \), a polynomial in \( t \).
Since \( P \in V \cap L \), \( c_0 = 0 \).
The linear term \( c_1 \) is given by \( c_1 = \sum_i b_i \pdv{f}{X_i}(\vb a) \).
Geometric tangency of \( L \) to \( V \) is equivalent to the statement that \( c_1 = 0 \).
\begin{definition}
    The line \( L \) through \( P \) is \emph{tangent} to \( V \) at \( P \) if it is contained in the \emph{tangent space} of \( V \) at \( P \), defined by \( T_{V,P}^{\mathrm{aff}} = \mathbb V(g) \subseteq \mathbb A^n \) where \( g = \sum_{i=1}^n \qty(\pdv{f}{X_i}(P))(X_i - a_i) \).
\end{definition}
