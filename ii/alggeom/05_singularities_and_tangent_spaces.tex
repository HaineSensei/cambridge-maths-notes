\subsection{Definitions}
Let \( V \subseteq \mathbb A^n \) be an affine hypersurface, so \( V = \mathbb V(f) \).
We assume that \( f \) is irreducible, so \( V \) is also irreducible.
Let \( P = (a_1, \dots, a_n) = (\vb a) \in V \).
An affine line through \( P \) has the form \( L = \qty{(a_1 + b_1 t, \dots, a_n + b_n t) \mid t \in \mathbb C} \) for \( (\vb b) \in \mathbb C^n \setminus \qty{\vb 0} \) is fixed.

The intersection \( V \cap L \) is the set of points on \( L \) where \( f \) vanishes.
This gives \( 0 = f(a_1 + b_1 t, \dots, a_n + b_n t) = g(t) = \sum_r c_r t^r \), a polynomial in \( t \).
Since \( P \in V \cap L \), \( c_0 = 0 \).
The linear term \( c_1 \) is given by \( c_1 = \sum_i b_i \pdv{f}{X_i}(\vb a) \).
Geometric tangency of \( L \) to \( V \) is equivalent to the statement that \( c_1 = 0 \).
\begin{definition}
    The line \( L \) through \( P \) is \emph{tangent} to \( V = \mathbb V(f) \) at \( P \) if it is contained in the \emph{tangent space} of \( V \) at \( P \), defined by \( T_{V,P}^{\mathrm{aff}} = \mathbb V(g) \subseteq \mathbb A^n \) where
    \[ g = \sum_{i=1}^n \qty(\pdv{f}{X_i}\qty(P))(X_i - a_i) \]
\end{definition}
Note that \( g \) is linear.
\( T_{V,P}^{\mathrm{aff}} \) is \( n \)-dimensional if \( g = 0 \) and \( (n-1) \)-dimensional otherwise, taking the dimensions as an affine space.
If \( \dim T_{V,P}^{\mathrm{aff}} = n \), we say that \( P \) is a \emph{singular point} of \( V \).
Otherwise, it is a \emph{smooth point}.
\begin{example}[nodal cubic]
    Consider the affine hypersurface \( C = \mathbb V(Y^2 - X^2(X+1)) \).
    One can show by direct calculation that the only singular point is \( (0, 0) \).
\end{example}
\begin{example}[cusp]
    Consider \( C = \mathbb V(Y^2 - X^3) \).
    Again, the point \( (0, 0) \) is the only singular point.
\end{example}
Let \( V \subseteq \mathbb V(F) \subseteq \mathbb P^n \) for \( F \) an irreducible homogeneous polynomial.
\begin{definition}
    The \emph{projective tangent space} to \( V \) at \( P \) is \( T_{V,P}^{\mathrm{proj}} = \mathbb V(G) \) where
    \[ G = \sum_{i=0}^n \qty(\pdv{F}{X_i}\qty(P)) X_i \]
\end{definition}
To see that \( P \in \mathbb V(G) \), note that \( F(X_0, \dots, X_n) = \frac{1}{\deg F} \sum_{i=0}^n X_i \pdv{F}{X_i} \); this is sometimes known as \emph{Euler's formula}.
Smooth points and singular points are defined as in the affine case.
From the inverse function theorem, if all points are smooth, the tangent space is a manifold.

The affine and projective tangent spaces are compatible in a particular sense.
Let \( V = \mathbb V(F) \not\subseteq \qty{X_0 = 0} \), and consider \( V_0 = V \cap U_0 \).
If \( P \in V_0 \subseteq V \), we can compute \( T_{V,P}^{\mathrm{proj}} \cap U_0 \) and \( T_{V_0,P}^{\mathrm{aff}} \), which are both subsets of \( \mathbb A^n \).
Let \( V_0 = \mathbb V(f) \), then \( F(X_0, \dots, X_n) = X_0^{\deg F} f\qty(\frac{X_1}{X_0}, \dots, \frac{X_n}{X_0}) \).
By computing \( \pdv{F}{X_i} \), we find that if \( P \in V_0 \), \( T_{V,P}^{\mathrm{proj}} \cap U_0 = T_{V_0,P}^{\mathrm{aff}} \).
\begin{proposition}
    The set of singular points on a nonempty irreducible projective hypersurface is a proper Zariski closed subset.
    In particular, the set of smooth points is dense.
\end{proposition}
\begin{proof}
    The set of singular points is the intersection of \( V \) with \( \bigcap_i \mathbb V\qty(\pdv{F}{X_i}) \), so is a closed subvariety of \( V \).
    If \( V \cap \bigcap_i \mathbb V\qty(\pdv{F}{X_i}) = V \), then by the Nullstellensatz, \( \pdv{F}{X_i} \in I^h(V) \).
    However, \( I^h(V) \) is principal and generated by \( F \).
    Since \( \pdv{F}{X_i} \) is homogeneous and of smaller degree, \( \pdv{F}{X_i} \mid F \) gives that \( \pdv{F}{X_i} = 0 \).
    So \( F \) is a constant polynomial, giving \( V = \mathbb P^n \) as it is nonempty, which has no singular points.
\end{proof}
We can extend the definition of a tangent space to general varieties not generated by a single polynomial.
\begin{definition}
    Let \( V \subseteq \mathbb A^n \) be an affine variety, and let \( P \in V \).
    Then the \emph{tangent space} to \( V \) at \( P \) is
    \[ T_{V,P} = \qty{\vb v \in \mathbb C^n \mid \sum_{i=1}^n \vb v_i \pdv{f}{x_i}\qty(P) = 0 \text{ for all } f \in I(V)} \subseteq \mathbb C^n \]
\end{definition}
This is a vector subspace of \( \mathbb C^n \).
\begin{definition}
    Let \( V \subseteq \mathbb P^n \) be a projective variety, and let \( P \in V \).
    Suppose \( V_j = V \cap \qty{X_j \neq 0} \) is an affine piece containing \( P \).
    Then the \emph{tangent space} to \( V \) at \( P \) is \( T_{V,P} = T_{V_j,P} \).
\end{definition}
Note that this definition requires a choice of \( j \); it is not clear that this is well-defined.
This will be addressed by the following propositions.

Recall that \( \mathbb P^n \) is covered by \( U_0, \dots, U_n \), and \( U_i \simeq \mathbb A^n \).
Each point \( P \in \mathbb P^n \) is contained in at least one of these \( U_i \).
If the index is unimportant, we will write \( \mathbb A_n \subseteq \mathbb P^n \) rather than \( U_i \subseteq \mathbb P^n \).

Let \( V \subseteq \mathbb P^n, W \subseteq \mathbb P^m \) be irreducible varieties and \( \varphi \colon V \dashrightarrow W \) be a rational map.
Given \( P \in \dom \varphi \subseteq V \) and \( Q = \varphi(P) \subseteq W \cap \mathbb A^m \), we will now define \( \dd{\varphi}_P \colon T_{V,P} \dashrightarrow T_{W,P} \).
Suppose \( \varphi \) is determined by \( (F_0, \dots, F_m) \), where the \( F_i \) are homogeneous and of the same degree.
By restricting to \( \mathbb A^n \), we can write \( \frac{F_j}{F_0}(1, X_1, \dots, X_n) = f_j \in \mathbb C(X_1, \dots, X_n) \).
This gives rational functions \( f_1, \dots, f_m \) on \( V \cap \mathbb A^n \).
The \emph{derivative} of \( \varphi \) at \( P \) or \emph{linearisation} of \( \varphi \) at \( P \) is defined by
\[ \dd{\varphi}_P(\vb v) = \qty(\sum_{i=1}^n v_i \pdv{f_j}{X_i}\qty(P))_j \]
which is initially a function \( T_{V,P} \to \mathbb C^m \).
This can be thought of as an application of the matrix of derivatives of \( f \) at \( P \) to the vector \( \vb v \).
