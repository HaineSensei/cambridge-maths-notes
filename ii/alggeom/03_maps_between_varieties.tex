\subsection{?}
Let \( V \subseteq \mathbb A^n \) and \( W \subseteq \mathbb A^m \) be affine varieties.
\begin{definition}
    A \emph{regular map} or \emph{morphism} from \( V \) to \( W \) is a function \( \varphi \colon V \to W \) such that there exist elements \( f_1, \dots, f_m \in \mathbb C[V] \) such that
    \[ \varphi(P) = (f_1(P), \dots, f_m(P)) \]
    for all \( P \in V \).
\end{definition}
The set of all morphisms from \( V \) to \( W \) is denoted \( \mathrm{Mor}(V,W) \).
\begin{example}
    The morphisms \( V \) to \( \mathbb A^1 \) are precisely the functions in the coordinate ring \( \mathbb C[V] \).
\end{example}
\begin{example}
    Linear projections \( \mathbb A^n \to \mathbb A^m \) are morphisms.
    More generally, linear transformations and affine translations are also morphisms.
\end{example}
\begin{example}
    If \( V \subseteq W \subseteq \mathbb A^n \) where \( V, W \) are varieties, then the inclusion map \( V \hookrightarrow W \) is a morphism.
\end{example}
\begin{proposition}
    Let \( \varphi \colon V \to W, \psi \colon W \to Z \) be morphisms.
    Then the composite map \( \psi \circ \varphi \) is a morphism \( V \to Z \).
\end{proposition}
\begin{proof}
    The composition of polynomials is a polynomial.
\end{proof}
\begin{remark}
    Affine varieties form a category.
\end{remark}

\subsection{Pullbacks}
\begin{definition}
    Let \( \varphi \colon V \to W \) be a morphism, and let \( g \in \mathbb C[W] \).
    Then, the \emph{pullback} is \( \varphi^\star(g) = g \circ \varphi \colon V \to \mathbb C \).
    Note that \( \varphi^\star(g) \in \mathbb C[V] \), so \( \varphi^\star \) gives a map \( \mathbb C[W] \to \mathbb C[V] \).
\end{definition}
\begin{remark}
    This map \( \varphi^\star \) is a ring homomorphism, and restricts to the identity on \( \mathbb C \).
\end{remark}
\begin{definition}
    A ring homomorphism \( \mathbb C[X] \to \mathbb C[Y] \) that restricts to the identity on \( \mathbb C \) is called a \emph{\( \mathbb C \)-algebra homomorphism}.
\end{definition}
\begin{theorem}
    Let \( V \subseteq \mathbb A^n, W \subseteq \mathbb A^m \) be affine varieties.
    The map \( \alpha \colon \varphi \mapsto \varphi^\star \) defines a bijection from \( \mathrm{Mor}(V, W) \) to the space of \( \mathbb C \)-algebra homomorphisms \( \mathbb C[W] \to \mathbb C[V] \).
\end{theorem}
\begin{proof}
    Let \( y_1, \dots, y_n \in \mathbb C[W] \) be the coordinate projections on \( W \), which are the restrictions of the standard linear coordinate projections on \( \mathbb A^n \).

    First, we show injectivity of \( \alpha \).
    Let \( \varphi \colon V \to W \) be a morphism.
    For any point \( P \in V \),
    \[ \varphi(P) = (y_1(\varphi(P)), \dots, y_m(\varphi(P))) = (\varphi^\star(y_1)(P), \dots, \varphi^\star(y_n)(P)) \]
    So \( \varphi \) is determined by the values of \( \varphi^\star(y_1), \dots, \varphi^\star(y_n) \).

    Now we show its surjectivity.
    Let \( \lambda \colon \mathbb C[W] \to \mathbb C[V] \) be a \( \mathbb C \)-algebra homomorphism, and let \( f_i = \lambda(y_i) \in \mathbb C[V] \).
    We can now define the map \( \varphi = (f_1, \dots, f_m) \colon V \to \mathbb A^m \).
    We will show that \( \varphi \) has image contained in \( W \), so that we have \( \varphi \colon V \to W \), which then shows that \( \varphi \) is a morphism \( V \to W \).
    For \( P \in V \), we must show \( g(\varphi(P)) = 0 \) for all \( g \in I(W) \).
    We obtain \( g(f_1(P), \dots, f_m(P)) = \lambda(g)(P) \).
    But \( g = 0 \) in \( \mathbb C[W] \), so \( g(\varphi(P)) = 0 \) as required.
    Hence \( \varphi \colon V \to W \) is a morphism, and \( \lambda = \varphi^\star \) since \( \varphi^\star(y_i) = f_i = \lambda(y_i) \).
\end{proof}
\begin{definition}
    Two affine varieties \( V, W \) are \emph{isomorphic} if we have \( \varphi \colon V \to W, \psi \colon W \to V \) where \( \varphi \circ \psi = \mathrm{id}_W \) and \( \psi \circ \varphi = \mathrm{id}_V \).
\end{definition}
\begin{remark}
    \( V \) is isomorphic to \( W \) if and only if \( \mathbb C[V] \) is isomorphic to \( \mathbb C[W] \) as \( \mathbb C \)-algebras.
\end{remark}
\begin{example}
    The affine line \( \mathbb A^1 \) is isomorphic to the twisted cubic \( \qty{(t, t^2, t^3) \mid t \in \mathbb C} \).
    This can be easily shown by calculating the coordinate rings explicitly.
\end{example}
\begin{example}
    Let \( V \subseteq \mathbb A^2 \) be given by \( X_1 X_2 (X_1 - X_2) = 0 \).
    This is the union of three lines, intersecting at the origin.
    Let \( W \subseteq \mathbb A^3 \) be given by \( X_1 X_2 = X_2 X_3 = X_3 X_1 = 0 \), which is also a union of three lines, which in this case are the coordinate axes.
    These are not isomorphic as varieties, because their coordinate rings are not isomorphic, which can be easily shown using tangent spaces, defined in later sections.
    Note, however, that \( V \) and \( W \) are homeomorphic in the Euclidean topology.
\end{example}

\subsection{Rational functions}
\begin{definition}
    Let \( V \subseteq \mathbb A^n \) be an irreducible variety.
    Its \emph{function field}, \emph{field of rational functions}, or \emph{field of meromorphic functions} is the field of fractions \( \mathbb C(V) = FF(\mathbb C[V]) \) of \( \mathbb C[V] \).
\end{definition}
\begin{remark}
    Since \( V \) is irreducible, \( I(V) \) is prime, so \( \mathbb C[V] \) is an integral domain.
    This allows us to construct the field of fractions.
\end{remark}
\begin{definition}
    Let \( \varphi \) be a rational function.
    A point \( P \in V \) is called \emph{regular} if \( \varphi \) can be expressed as a ratio \( \frac{f}{g} \) with \( g(P) \neq 0 \).
\end{definition}
\begin{remark}
    If \( \varphi = \frac{f}{g} \), we obtain a well-defined function \( \varphi \colon V \setminus \mathbb V(g) \to \mathbb C \).
    The domain is an open set in \( V \), since \( \mathbb V(g) \) is Zariski closed.
\end{remark}
\begin{example}
    Consider the rational function \( X_1^2 / X_2 \in \mathbb C(\mathbb A^2) \).
    This defines a map on the complement of the \( X_2 \)-axis.
    Note that \( X^3 / X_1 X_2 \) defines the same function, but only on points other than \( \mathbb V(X_1 X_2) \).
\end{example}
\begin{remark}
    A rational function on \( V \) can be thought of as a pair \( (U, f) \) with \( U \subseteq V \) Zariski open, such that \( f \) is a function \( U \to \mathbb C \).
    We define the equivalence relation \( (U, f) \sim (U', f') \) if \( f, f' \) agree on some non-empty Zariski open set contained in \( U \) and \( U' \).
    Note that if \( V \) is irreducible, every nonempty open subset is dense in the Zariski topology.
\end{remark}
