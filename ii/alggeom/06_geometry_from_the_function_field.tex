If \( K \subseteq L \) are fields and \( \alpha \in L \), we say that \( \alpha \) is \emph{transcendental} over \( K \) if it is not a solution to a nontrivial polynomial \( f \in K[t] \).
More generally, if \( S \subseteq L \) is any set of elements, we say they are \emph{algebraically independent} if they do not satisfy a nontrivial polynomial relation over \( K \).
A field extension \( K / \mathbb C \) is a \emph{pure transcendental extension} if \( K \) is generated by transcendental algebraically independent elements \( x_1, \dots, x_n \in K \).

If \( V \) is an irreducible affine variety, recall that \( \mathbb C(V) = FF\qty(\faktor{\mathbb C[\vb X]}{I(V)}) \).
If \( V = \mathbb P^1 \), \( \mathbb C(V) \simeq \mathbb C(X) \).

\begin{proposition}
    Let \( K / \mathbb C \) be a finitely generated field extension.
    Then, there exists a pure transcendental subfield \( K_0 = \mathbb C(x_1, \dots, x_m) \subseteq K \) such that \( K / K_0 \) is finite (and hence algebraic).
    Moreover, \( K = K_0(y) \) for some \( y \in K \).
\end{proposition}
\begin{proof}
    The final statement follows from the primitive element theorem from Part II Galois Theory.
    We now prove the first part.
    \( K \) is finitely generated, so let \( x_1, \dots x_n \) generate \( K \).
    There is a maximal algebraically independent subset which after relabelling is given by \( \qty{x_1, \dots, x_m} \) for \( m \leq n \).
    Then \( x_{m+1}, \dots, x_n \) are algebraic over \( K_0 = \mathbb C(x_1, \dots, x_m) \).
\end{proof}
\begin{proposition}
    Let \( K = \mathbb C(x_1, \dots, x_n) \), where \( x_1, \dots, x_n \) are algebraically independent.
    Let \( x_{n+1} \) be algebraic over \( K \).
    Define
    \[ I = \qty{g \in \mathbb C[X_1, \dots, X_{n+1}] \mid g(x_1, \dots, x_n, x_{n+1}) = 0} \]
    Then \( I \) is a principal ideal generated by an irreducible element \( f \in \mathbb C[\vb X] \).
    Moreover, if \( f \) contains the variable \( X_i \), then \( \qty{x_1, \dots, x_{i-1}, x_{i+1}, \dots, x_n, x_{n+1}} \) is algebraically independent.
\end{proposition}
\begin{proof}
    As \( x_1, \dots, x_n \) are algebraically independent, the subring \( R = \mathbb C[x_1, \dots, x_n] \subseteq K \) is isomorphic to the polynomial ring \( \mathbb C[X_1, \dots, X_n] \).
    \( \mathbb C[X_1, \dots, X_n] \) is a unique factorisation domain.
    There exist polynomials \( g \in K[T] \) where \( x_{n+1} \) is a root, as it is algebraic.
    Since \( K[T] \) is a principal ideal domain, the ideal of such polynomials is principal, and generated by a unique monic polynomial \( h(t) \), called the minimal polynomial of \( x_{n+1} \).
    The minimal polynomial is irreducible.

    Let \( b \) be the least common multiple of the denominators in \( h(t) \), so \( b \in R \).
    By Gauss' lemma, \( f = bh \) is irreducible in \( R[T] \).
    By the isomorphism \( R \simeq \mathbb C[X_1, \dots, X_n] \), we can think of \( f \) as an element of \( \mathbb C[X_1, \dots, X_{n+1}] \).

    We show that \( f \) generates \( I \).
    Suppose \( g \in \mathbb C[\vb X] \) such that \( g(x_1, \dots, x_{n+1}) = 0 \).
    In \( K[T] \), \( g(x_1, \dots, x_n, T) \) is divisible by \( f(x_1, \dots, x_n) \).
    By Gauss' lemma, \( f \mid g \) in \( \mathbb C[\vb X] \).
    Hence \( f \) generates \( I \) as required.
    The last part is left as an exercise.
    % in online notes
\end{proof}
\begin{corollary}
    Let \( V \) be any irreducible variety.
    Then \( V \) is birational to a hypersurface.
\end{corollary}
\begin{proof}
    Let \( K \) be the function field of \( V \).
    By the above discussion, we can find elements that generate \( K \) that are given by \( x_1, \dots, x_{n+1} \) where \( x_1, \dots, x_n \) are algebraically independent and \( x_{n+1} \) is algebraic over \( \mathbb C(x_1, \dots, x_n) \).
    By the previous proposition, \( K \supseteq \mathbb C[x_1, \dots, x_{n+1}] = \faktor{\mathbb C[X_1, \dots, X_{n+1}]}{f} \).
    We take the hypersurface \( \mathbb V(f) \subseteq \mathbb A^{n+1} \).
\end{proof}
We know that birational varieties have the same dimension, so % finish from notes

% new section
\subsection{Proof of Hilbert's Nullstellensatz}
\begin{theorem}
    Every maximal ideal in \( \mathbb C[\vb X] \) has the form \( (X_1 - a_1, \dots, X_n - a_n) \) for \( a_i \in \mathbb C \).
    Moreover, if \( I \) is any non-unit ideal, \( \mathbb V(I) \neq \varnothing \subseteq \mathbb A^n \).
\end{theorem}
This is the weak form of the Nullstellensatz.
We prove this over the complex numbers; the given proof only works for this case, but the statement holds for all algebraically closed fields.
\begin{proof}
    Every ideal of this form has quotient \( \mathbb C \), so they are all maximal.
    Let \( \mathfrak m \triangleleft \mathbb C[\vb X] \) be a maximal ideal, and let \( K = \faktor{\mathbb C[\vb X]}{\mathfrak m} \).
    \( K \) is a field as \( \mathfrak m \) is maximal, and it is a field extension of \( \mathbb C \).
    Define \( a_i \) to be the coset \( X_i + \mathfrak m \).
    If \( a_i \in \mathbb C \) for all \( i \), this gives the result as required because the ideal is generated by \( (X_1 - a_1, \dots, X_n - a_n) \).
    
    Otherwise, \( K \supsetneq \mathbb C \).
    But \( \mathbb C \) is algebraically closed, so there exists \( t \in K \setminus \mathbb C \) which is transcendental over \( \mathbb C \).
    Let \( U_m \) be the \( \mathbb C \)-span inside \( K \) of products of the form \( a_1^{r_1} \dots a_n^{r_n} \) where the \( r_i \) are nonnegative, and \( \sum_{i=1}^n r_i \leq m \).
    Observe that \( U_m \) is finite-dimensional, and \( K = \bigcup_{m \geq 0} U_m \) is countable-dimensional.
    One can show that the elements \( \qty{\frac{1}{t - c} \mid c \in \mathbb C} \) are linearly independent over \( \mathbb C \).
    There are uncountably many such elements, giving a contradiction.

    For the last part, let \( I \) be a nonzero ideal.
    There exists a maximal ideal \( \mathfrak m \supseteq I \), so \( \mathbb V(I) \supseteq \mathbb V(\mathfrak m) \), but \( \mathbb V(\mathfrak m) \) is nonempty as it contains the point \( (a_1, \dots, a_m) \).
\end{proof}
% cor 12.2 is nonexaminable; 12.4 (strong nullstellensatz) has proof in notes, proof nonexaminable
