We will construct the projective space \( \mathbb P^n \), which will be an upgrade to \( \mathbb A^n \); it is not immediately obvious why \( \mathbb P^n \) is considered `better'.
Projective space has some interesting properties, such as:
\begin{itemize}
    \item every pair of lines in \( \mathbb P^2 \) that are distinct meet at a unique point;
    \item if \( V \) is a projective variety (defined shortly) in \( \mathbb P^2 \) defined by a degree \( d \) polynomial, if \( V \) is a manifold then \( V \) is homeomorphic in the Euclidean topology to a closed orientable topological surface of genus \( \binom{d-1}{2} \).
    \item \( \mathbb P^n \) is compact in the Euclidean topology, but \( \mathbb A^n \) is not.
\end{itemize}

\subsection{Definition}
\begin{definition}
    Let \( U \) be a finite-dimensional complex vector space.
    The \emph{projectivisation} of \( U \), written \( \mathbb P(U) \), is the set of lines in \( U \) through the origin \( \vb 0 \in U \).
    Define \( \mathbb P^n = \mathbb P(\mathbb C^{n+1}) \).
\end{definition}
We usually index the coordinates on \( \mathbb C^{n+1} \) with indices \( 0, \dots, n \).
A line in \( \mathbb C^{n+1} \) is therefore given by \( \qty{(a_0 t, \dots, a_n t) \mid t \in \mathbb C} \), and is written \( L_{(a_0, \dots, a_n)} \), where not all \( a_i \) are zero.
We write \( (a_0 : a_1 : \dots : a_n) \) for the corresponding element of \( \mathbb P^n \).
Therefore,
\[ \mathbb P^n = \faktor{\qty{(a_0, \dots, a_n) \mid a_i \in \mathbb C, \text{not all } a_i = 0}}{\text{scaling by } \mathbb C^\star} \]
For example, \( (2 : 1 : -2) = (4 : 2 : -4) \in \mathbb P^2 \).

We can decompose \( \mathbb P^1 \) as
\[ \qty{(a_0 : a_1) \mid a_0 \neq 0} \cup \qty{(a_0 : a_1) \mid a_0 = 0} = \qty{\qty(1 : z) \mid z \in \mathbb C} \cup \qty{(0 : 1)} = \mathbb A^1 \cup \text{a point at infinity} \]
More generally,
\[ \mathbb P^n = \qty{(a_0 : \dots : a_n) \mid a_0 \neq 0} \cup \qty{(0 : a_1 : \dots : a_n)} = \mathbb A^n \amalg \mathbb P^{n-1} \]
By induction, \( \mathbb P^n = \mathbb A^n \cup \mathbb A^{n-1} \cup \dots \cup \mathbb A^1 \cup \) a point, where the terms other than \( \mathbb A^n \) are considered `at infinity'.
\begin{definition}
    The \emph{Zariski} (respectively \emph{Euclidean}) topology on projective space is the quotient topology for the subspace topology for the Zariski (respectively Euclidean) topology on \( \mathbb C^{n+1} \setminus \qty{\vb 0} \), where \( \mathbb P^n = \faktor{\mathbb C^{n+1} \setminus \qty{\vb 0}}{\sim} \) and \( \mathbb C^{n+1} \setminus \qty{0} \subseteq \mathbb C^{n+1} \).
\end{definition}
There is a copy of \( \mathbb S^{2n+1} \) inside \( \mathbb C^{n+1} \setminus \qty{\vb 0} \), which therefore surjects onto \( \mathbb P^n \).
\begin{corollary}
    \( \mathbb P^n \) is compact.
\end{corollary}
\begin{proof}
    It is the continuous image of the compact set \( \mathbb S^{2n+1} \).
\end{proof}
\begin{definition}
    For \( 0 \leq j \leq n \), we define the \emph{\( j \)th coordinate hyperplane} is the set \( H_j = \qty{(\vb a_i) \mid a_j = 0} \subseteq \mathbb P^n \).
\end{definition}
We can naturally identify \( H_j \) with \( \mathbb P^{n-1} \).
\begin{definition}
    The \emph{\( j \)th standard affine patch} \( U_j \) is the complement of \( H_j \). 
\end{definition}
There is a natural bijection \( U_j \to \mathbb A^n \) by mapping \( (a_0 : \dots : a_n) \) to \( \qty(\frac{a_0}{a_j}, \dots, \widehat{\frac{a_j}{a_j}}, \dots, \frac{a_n}{a_j}) \) where the hat denotes `forgetting' that element of the tuple.
The inverse function maps \( (b_1, \dots, b_n) \) to \( (b_1 : \dots : b_{j-1} : 1 : b_j : \dots : b_n) \).
We observe that \( \qty{U_j}_{j=0}^n \) is an open cover of \( \mathbb P^n \) in the Zariski topology.

\subsection{Projective varieties}
\begin{example}
    Consider the polynomial \( X_0 + 1 \in \mathbb C[X_0, X_1] \).
    Note that \( X_0 + 1 \) does not define a function on \( \mathbb P^1 \).
    For example, \( (-1 : 0) = (1 : 0) \), but \( X_0 + 1 \) vanishes on the first representative and not the second.
    The vanishing locus of \( X_0 + 1 \) on \( \mathbb P^1 \) is therefore undefined.
    Therefore, we need a slightly more subtle definition of a variety in projective space.
\end{example}
\begin{definition}
    A \emph{monomial} in \( \mathbb C[\vb X] = \mathbb C[X_0, \dots, X_n] \) is an element of the form \( X_0^{d_0} X_1^{d_1} \dots X_n^{d_n} \) where \( d_i \geq 0 \).
    A \emph{term} is a nonzero multiple of a monomial.
    The \emph{degree} of a term \( cX_0^{d_0} \dots X_n^{d_n} \) is \( \sum_{i=0}^n d_i \).
    A \emph{homogeneous polynomial} of degree \( d \) is a finite sum of terms of degree \( d \).
\end{definition}
Any polynomial can be uniquely decomposed as a sum of homogeneous polynomials of different degree; we write \( f = \sum_{i=0}^\infty f_{[i]} \) where the \( f_{[i]} \) are homogeneous of degree \( i \).
Note that this sum is always finite.
\begin{lemma}
    Let \( f \in \mathbb C[\vb X] \) be homogeneous, and let \( (a_0, \dots, a_n) \in \mathbb C^{n+1} \setminus \qty{\vb 0} \).
    Then, if \( f(\vb a) = 0 \), we have \( f(\lambda \vb a) = 0 \) for all \( \lambda \in \mathbb C^\star \).
\end{lemma}
\begin{proof}
    Trivial by checking the definitions.
\end{proof}
\begin{corollary}
    Let \( f \in \mathbb C[\vb X] \) be homogeneous.
    Then \( \mathbb V(f) = \qty{P \in \mathbb P^n \mid f(\vb a) = 0 \text{ for any (or every) representative of } P} \) is well-defined.
\end{corollary}
\begin{definition}
    An ideal \( I \trianglelefteq \mathbb C[\vb X] \) is called \emph{homogeneous} if it can be generated by homogeneous polynomials (of potentially different degrees).
\end{definition}
\begin{lemma}
    Let \( I \trianglelefteq \mathbb C[\vb X] \) be an ideal.
    Then \( I \) is homogeneous if and only if whenever \( f \in I \), all of the homogeneous parts \( f_{[r]} \) are also contained in \( I \).
\end{lemma}
\begin{proof}
    Suppose \( I \) is homogeneous.
    Then let \( g_j \) be homogeneous generators of \( I \) of degree \( d_j \).
    Writing \( f = \sum h_j g_j \) for arbitrary \( h_j \in \mathbb C[\vb X] \), we can split each \( h_j \) into its pieces \( h_{j[r]} \).
    Now, \( h_{j[r]} g_j \in I \) is homogeneous, and its degree is \( rd_j \).
    Hence, \( f_{[r]} = \sum_j h_{j[r-dj]} g_j \in I \) as required.
    The converse is trivial by decomposing the generators of the ideal.
\end{proof}
\begin{definition}
    Let \( I \trianglelefteq \mathbb C[\vb X] \) be a homogeneous ideal.
    Then, the \emph{vanishing locus} is \( \mathbb V(I) = \qty{P = (\vb a_i) \in \mathbb P^n \mid \forall f \in I,\, f((\vb a_i)) = 0} \).
    A \emph{projective variety} in \( \mathbb P^n \) is any set of this form.
\end{definition}
Note that we could have defined the vanishing locus using the quantifier `for all \emph{homogeneous} \( f \in I \)'.
\begin{example}
    Let \( U \subseteq \mathbb C^{n+1} \) be any vector subspace.
    Let the projectivisation of \( U \) is a subset of \( \mathbb P^n \), and is a projective variety.
    More concretely, \( U = \qty{\vb v \in \mathbb C^{n+1} \mid \forall j,\, \sum_{i=0}^n a_i^{(j)} v_i = 0} \) for a subset \( \vb a^{(j)} = (a_0^{(j)}, \dots, a_n^{(j)}) \), as a vector subspace is the kernel of some linear map.
    Therefore, \( \mathbb P(U) = \mathbb V(I) \) where \( I \) is the ideal generated by \( F_j = \sum_i a^{(j)}_i X_i \in \mathbb C[\vb X] \).
    More generally, a projective linear space is the projectivisation of a linear subspace.
    Hence, projective linear spaces in \( \mathbb P^n \) are in bijection with linear subspaces in \( \mathbb C^{n+1} \).
\end{example}
\( GL_{n+1}(\mathbb C) \) acts on \( \mathbb P^n \) coordinatewise.
The normal subgroup of scalar matrices \( \mathbb C^\star \subseteq GL_{n+1}(\mathbb C) \) acts trivially on \( \mathbb P^n \).
The quotient is written \( PGL_n(\mathbb C) = \faktor{GL_{n+1}(\mathbb C)}{\mathbb C^\star} \), and acts transitively on \( \mathbb P^n \).
\begin{example}
    The \emph{Segre surface} is the hypersurface \( S_{11} = \mathbb V(X_0X_3 - X_1X_2) \subseteq \mathbb P^3 \).
    Consider the map \( \sigma_{11} \colon \mathbb P^1 \times \mathbb P_1 \to \mathbb P^3 \) given by \( \sigma_{11}((a_0 : a_1), (b_0 : b_1)) = (a_0 b_0 : a_0 b_1 : a_1 b_0 : a_1 b_1) \).
    One can show that this map is well-defined, and in fact, \( \Im \sigma_{11} = S_{11} \).

    First, consider the map \( \mathbb C^2 \times \mathbb C^2 \to \mathbb C^4 \) where we identify \( \mathbb C^4 \) with the space of \( 2 \times 2 \) matrices on \( \mathbb C \), given by the outer product.
    More precisely, \( (v,w) \mapsto vw^\transpose \).
    The image of this map is precisely the set of matrices of rank at most 1.
    Hence, the image is the vanishing locus of \( X_0 X_3 - X_1 X_2 \), the determinant of such a matrix.
\end{example}
