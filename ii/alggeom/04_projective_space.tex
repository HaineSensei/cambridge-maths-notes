We will construct the projective space \( \mathbb P^n \), which will be an upgrade to \( \mathbb A^n \); it is not immediately obvious why \( \mathbb P^n \) is considered `better'.
Projective space has some interesting properties, such as:
\begin{itemize}
    \item every pair of lines in \( \mathbb P^2 \) that are distinct meet at a unique point;
    \item if \( V \) is a projective variety (defined shortly) in \( \mathbb P^2 \) defined by a degree \( d \) polynomial, if \( V \) is a manifold then \( V \) is homeomorphic in the Euclidean topology to a closed orientable topological surface of genus \( \binom{d-1}{2} \).
    \item \( \mathbb P^n \) is compact in the Euclidean topology, but \( \mathbb A^n \) is not.
\end{itemize}

\subsection{Definition}
\begin{definition}
    Let \( U \) be a finite-dimensional complex vector space.
    The \emph{projectivisation} of \( U \), written \( \mathbb P(U) \), is the set of lines in \( U \) through the origin \( \vb 0 \in U \).
    Define \( \mathbb P^n = \mathbb P(\mathbb C^{n+1}) \).
\end{definition}
We usually index the coordinates on \( \mathbb C^{n+1} \) with indices \( 0, \dots, n \).
A line in \( \mathbb C^{n+1} \) is therefore given by \( \qty{(a_0 t, \dots, a_n t) \mid t \in \mathbb C} \), and is written \( L_{(a_0, \dots, a_n)} \), where not all \( a_i \) are zero.
We write \( (a_0 : a_1 : \dots : a_n) \) for the corresponding element of \( \mathbb P^n \).
Therefore,
\[ \mathbb P^n = \faktor{\qty{(a_0, \dots, a_n) \mid a_i \in \mathbb C, \text{not all } a_i = 0}}{\text{scaling by } \mathbb C^\star} \]
For example, \( (2 : 1 : -2) = (4 : 2 : -4) \in \mathbb P^2 \).

We can decompose \( \mathbb P^1 \) as
\[ \qty{(a_0 : a_1) \mid a_0 \neq 0} \cup \qty{(a_0 : a_1) \mid a_0 = 0} = \qty{\qty(1 : z) \mid z \in \mathbb C} \cup \qty{(0 : 1)} = \mathbb A^1 \cup \text{a point at infinity} \]
More generally,
\[ \mathbb P^n = \qty{(a_0 : \dots : a_n) \mid a_0 \neq 0} \cup \qty{(0 : a_1 : \dots : a_n)} = \mathbb A^n \amalg \mathbb P^{n-1} \]
By induction, \( \mathbb P^n = \mathbb A^n \cup \mathbb A^{n-1} \cup \dots \cup \mathbb A^1 \cup \) a point, where the terms other than \( \mathbb A^n \) are considered `at infinity'.
\begin{definition}
    The \emph{Zariski} (respectively \emph{Euclidean}) topology on projective space is the quotient topology for the subspace topology for the Zariski (respectively Euclidean) topology on \( \mathbb C^{n+1} \setminus \qty{\vb 0} \), where \( \mathbb P^n = \faktor{\mathbb C^{n+1} \setminus \qty{\vb 0}}{\sim} \) and \( \mathbb C^{n+1} \setminus \qty{0} \subseteq \mathbb C^{n+1} \).
\end{definition}
