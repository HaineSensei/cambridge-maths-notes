\subsection{Decoding rules}
\begin{definition}
    A \emph{binary \( [n,m] \)-code} is a subset \( C \) of \( \qty{0,1}^n \) of size \( m = \abs{C} \).
    We say \( n \) is the \emph{length} of the code, and elements of \( C \) are called \emph{codewords}.
\end{definition}
We use an \( [n,m] \)-code to send one of \( m \) messages through a channel using \( n \) bits.
For instance, if the channel is a binary symmetric channel, we use the channel \( n \) times.
Note that \( 1 \leq m \leq 2^n \), so the information rate \( \rho(C) = \frac{1}{n} \log m \) satisfies \( 0 \leq \rho(C) \leq 1 \).
If \( m = 1 \), \( \rho(C) = 0 \), and if \( C = \qty{0,1}^n \), \( \rho(C) = 1 \).
\begin{definition}
    Let \( x, y \in \qty{0,1}^n \).
    The \emph{Hamming distance} between \( x \) and \( y \) is
    \[ d(x,y) = \abs{\qty{i \mid x_i \neq y_i}} \]
\end{definition}
In this section, we consider only the binary symmetric channel with probability \( p \).
\begin{definition}
    Let \( C \) be a binary \( [n,m] \)-code.
    \begin{itemize}
        \item The \emph{ideal observer} decoding rule decodes \( x \in \qty{0,1}^n \) as the \( c \in C \) maximising the probability that \( c \) was sent given that \( x \) was received;
        \item The \emph{maximum likelihood} decoding rule decodes \( x \in \qty{0,1}^n \) as the \( c \in C \) maximising the probability that \( x \) was received given that \( c \) was sent;
        \item The \emph{minimum distance} decoding rule decodes \( x \in \qty{0,1}^n \) as the \( c \in C \) minimising the Hamming distance \( d(x,c) \).
    \end{itemize}
\end{definition}
\begin{lemma}
    Let \( C \) be a binary \( [n,m] \)-code.
    \begin{enumerate}
        \item If all messages are equally likely, the ideal observer and maximum likelihood decoding rules agree.
        \item If \( p < \frac{1}{2} \), then the maximum likelihood and minimum distance decoding rules agree.
    \end{enumerate}
\end{lemma}
Note that the hypothesis in part (i) is reasonable if we first encode a message using noiseless coding.
The hypothesis in part (ii) is reasonable, since a channel with \( p = \frac{1}{2} \) can carry no information, and a channel with \( p > \frac{1}{2} \) can be used as a channel with probability \( 1 - p \) by inverting its outputs.
Channels with \( p = 0 \) are called \emph{lossless channels}, and channels with \( p = \frac{1}{2} \) are called \emph{useless channels}.
\begin{proof}
    \emph{Part (i).}
    By Bayes' rule,
    \[ \prob{c \text{ sent} \mid x \text{ received}} = \frac{\prob{c \text{ sent, } x \text{ received}}}{x \text{ received}} = \frac{\prob{c \text{ sent}}}{\prob{x \text{ received}}} \prob{x \text{ received} \mid c \text{ sent}} \]
    By hypothesis, \( \prob{c \text{ sent}} \) is independent of \( c \).
    Hence, for some fixed received message \( x \), maximising \( \prob{c \text{ sent} \mid x \text{ received}} \) is the same as maximising \( \prob{x \text{ received} \mid c \text{ sent}} \).

    \emph{Part (ii).}
    Let \( r = d(x,c) \).
    Then,
    \[ \prob{x \text{ received} \mid c \text{ sent}} = p^r (1-p)^{n-r} = (1-p)^n \qty(\frac{p}{1-p})^r \]
    As \( p < \frac{1}{2} \), \( \frac{p}{1-p} < 1 \).
    Hence, maximising \( \prob{x \text{ received} \mid c \text{ sent}} \) is equivalent to minimising \( r = d(x,c) \).
\end{proof}
We can therefore choose to use minimum distance decoding from this point.
\begin{example}
    Suppose codewords \( 000, 111 \) are sent with probabilities \( \alpha = \frac{9}{10} \) and \( 1 - \alpha = \frac{1}{10} \), through a binary symmetric channel with error probability \( p = \frac{1}{4} \).
    Suppose that we receive \( 110 \).
    Clearly, an error has been introduced.
    \begin{align*}
        \prob{000 \text{ sent} \mid 110 \text{ received}} &= \frac{\alpha p^2 (1-p)}{\alpha p^2 (1-p) + (1-\alpha)p(1-p)^2} = \frac{3}{4} \\
        \prob{111 \text{ sent} \mid 110 \text{ received}} &= \frac{1}{4}
    \end{align*}
    The ideal observer therefore decodes \( 110 \) as \( 000 \).
    The maximum likelihood or minimum distance decoding rules decode \( 110 \) as \( 111 \).
\end{example}
\begin{remark}
    Minimum distance decoding may be expensive in terms of time and storage if \( \abs{C} \) is large, since the distance to all codewords must be calculated \emph{a priori}.
    One must also specify a convention in case of a tie between the probabilities or distances, for instance, using a random choice, or requesting a retransmission.
\end{remark}

\subsection{Error detection and correction}
The aim when constructing codes for noisy channels is to detect errors, and if possible, to correct them.
\begin{definition}
    A binary \( [n,m] \)-code \( C \) is
    \begin{itemize}
        \item \emph{\( d \)-error detecting} if, when changing up to \( d \) digits in each codeword, we can never produce another codeword;
        \item \emph{\( e \)-error correcting} if, knowing that \( x \in \qty{0,1}^n \) differs from a codeword in at most \( e \) positions, we can deduce the codeword.
    \end{itemize}
\end{definition}
\begin{example}
    A \emph{repetition code} of length \( n \) has codewords \( 0^n, 1^n \).
    This is an \( [n,2] \)-code.
    It is \( (n-1) \)-error detecting, and \( \floor{\frac{n-1}{2}} \)-error correcting.
    Its information rate is \( \frac{1}{n} \).
\end{example}
\begin{example}
    A \emph{simple parity check code} or \emph{paper tape code} of length \( n \) identifies the set \( \qty{0,1} \) with the field \( \mathbb F_2 \) of two elements, and defines \( C = \qty{(x_1, \dots, x_n) \in \mathbb F_2^n \mid \sum x_i = 0} \).
    This is an \( [n,2^{n-1}] \)-code.
    This is 1-error detecting and 0-error correcting, but has information rate \( \frac{n-1}{n} \).
\end{example}
\begin{example}
    Hamming's original code is a 1-error correcting binary \( [7,16] \)-code, defined on a subset of \( \mathbb F_2^7 \) by
    \[ C = \qty{c \in \mathbb F_2^7 \mid c_1 + c_3 + c_5 + c_7 = 0; c_2 + c_3 + c_6 + c_7 = 0; c_4 + c_5 + c_6 + c_7 = 0} \]
    The bits \( c_3, c_5, c_6, c_7 \) are chosen arbitrarily, and \( c_1, c_2, c_4 \) are check digits, giving a size of \( 2^4 = 16 \).
    Suppose that we receive \( x \in \mathbb F_2^7 \).
    We form the \emph{syndrome} \( z = z_x = (z_1, z_2, z_4) \in \mathbb F_2^3 \) where
    \[ z_1 = x_1 + x_3 + x_5 + x_7;\quad z_2 = x_2 + x_3 + x_6 + x_7;\quad z_4 = x_4 + x_5 + x_6 + x_7 \]
    By definition of \( C \), if \( x \in C \) then \( z = (0, 0, 0) \).
    If \( d(x,c) = 1 \) for some \( c \in C \), then the place where \( x \) and \( c \) differ is given by \( z_1 + 2z_2 + 4z_4 \) (not modulo 2).
    Indeed, if \( x = c + e_i \) where \( e_i \) is the zero vector with a one in the \( i \)th position, \( z_x = z_{e_i} \), and one can check that this holds for each \( 1 \leq i \leq 7 \).
    Therefore, Hamming's original code is 1-error correcting.
\end{example}
\begin{lemma}
    The Hamming distance is a metric on \( \mathbb F_2^n \).
\end{lemma}
\begin{proof}
    Clearly, \( d(x,y) \geq 0 \) and equality holds if and only if \( x = y \), and \( d(x,y) = d(y,x) \).
    Let \( x, y, z \in \mathbb F_2^n \).
    Then,
    \[ \qty{i \mid x_i \neq z_i} \subseteq \qty{i \mid x_i \neq y_i} \cup \qty{i \mid y_i \neq z_i} \]
    Hence \( d(x,z) \leq d(x,y) + d(y,z) \).
\end{proof}
\begin{remark}
    We could write \( d(x,y) \) as \( \sum d_1(x_i,y_i) \) where \( d_1 \) is the discrete metric on \( \mathbb F_2 \).
\end{remark}

\subsection{Minimum distance}
\begin{definition}
    The \emph{minimum distance} of a code is the minimum value of \( d(c_1, c_2) \) for codewords \( c_1 \neq c_2 \).
\end{definition}
\begin{lemma}
    Let \( C \) be a code with minimum distance \( d > 0 \).
    Then,
    \begin{enumerate}
        \item \( C \) is \( (d-1) \)-error detecting, but cannot detect all sets of \( d \) errors;
        \item \( C \) is \( \floor{\frac{d-1}{2}} \)-error correcting, but cannot correct all sets of \( \floor{\frac{d-1}{2}} + 1 \) errors.
    \end{enumerate}
\end{lemma}
\begin{proof}
    \emph{Part (i).}
    If \( x \in \mathbb F_2^n \) and \( c \) is a codeword with \( 1 \leq d(x,c) \leq d - 1 \).
    Then \( x \not\in C \), so \( d - 1 \) errors are detected.
    Suppose \( c_1, c_2 \) are codewords with \( d(c_1, c_2) = d \).
    Then \( c_1 \) can be corrupted into \( c_2 \) with only \( d \) errors, and this is undetectable.

    \emph{Part (ii).}
    Let \( e = \floor{\frac{d-1}{2}} \).
    By definition, \( e \leq \frac{d-1}{2} < e + 1 \), so \( 2e < d \leq 2(e+1) \).
    Let \( x \in \mathbb F_2^n \).
    If \( c_1 \in C \) with \( d(x,c_1) \leq e \), we want to show that \( d(x,c_2) > e \) for all \( c_2 \neq c_1 \).
    By the triangle inequality, \( d(x,c_2) \geq d(c_1,c_2) - d(x,c_1) \geq d - e > e \) as required.
    Hence, \( C \) is \( e \)-error correcting.

    Let \( c_1, c_2 \in C \) with \( d(c_1, c_2) = d \).
    Let \( x \in \mathbb F_2^n \) differ from \( c_1 \) in precisely \( e + 1 \) places that \( c_1 \) and \( c_2 \) differ.
    Then \( d(x,c_1) = e + 1 \), and \( d(x,c_2) = d - (e+1) \leq e + 1 \).
    Hence, \( C \) cannot correct all sets of \( e + 1 \) errors.
\end{proof}
\begin{definition}
    An \( [n,m] \)-code with minimum distance \( d \) is called an \( [n,m,d] \)-code.
\end{definition}
Note that \( m \leq 2^n \) with equality if and only if \( C = \mathbb F_2^n \).
Similarly, \( d \leq n \), with equality in the case of the repetition code.
\begin{example}
    The repetition code of length \( n \) is an \( [n,2,n] \)-code.
    The simple parity check code of length \( n \) is an \( [n,2^{n-1},2] \)-code.
    The trivial code on \( n \) bits is an \( [n,2^n,1] \)-code.
    Hamming's original code is 1-error correcting, so has minimum distance at least 3.
    The minimum distance can easily be shown to be exactly 3 as \( 0000000, 1110000 \) are codewords, so it is a \( [7,16,3] \)-code.
\end{example}

\subsection{Covering estimates}
\begin{definition}
    Let \( x \in \mathbb F_2^n \) and \( r \geq 0 \).
    Then, we denote the \emph{closed Hamming ball} with centre \( x \) and radius \( r \) by \( B(x,r) \).
    We write \( V(n,r) = \abs{B(x,r)} = \sum_{i=0}^r \binom{n}{i} \) for the \emph{volume} of this ball.
\end{definition}
\begin{lemma}[Hamming's bound; sphere packing bound]
    An \( e \)-error correcting code \( C \) of length \( n \) has
    \[ \abs{C} \leq \frac{2^n}{V(n,e)} \]
\end{lemma}
\begin{proof}
    \( C \) is \( e \)-error correcting, so \( B(c_1, e) \cap B(c_2, e) \) is empty for all codewords \( c_1 \neq c_2 \).
    Hence,
    \[ \sum_{c \in C} \abs{B(c,e)}  \leq \abs{\mathbb F_2^n} \implies \abs{C} V(n,e) \leq 2^n \]
    as required.
\end{proof}
\begin{definition}
    An \( e \)-error correcting code \( C \) of length \( n \) such that \( \abs{C} = \frac{2^n}{V(n,e)} \) is called \emph{perfect}.
\end{definition}
\begin{remark}
    Equivalently, a code is perfect if for all \( x \in \mathbb F_2^n \), there exists a unique \( c \in C \) such that \( d(x,c) \leq e \).
    Alternatively, \( \mathbb F_2^n \) is a union of disjoint balls \( B(c,e) \) for all \( c \in C \), or that any collection of \( e + 1 \) will cause the message to be decoded incorrectly.
\end{remark}
\begin{example}
    Consider Hamming's \( [7,16,3] \)-code.
    This is 1-error correcting, and
    \[ \frac{2^n}{V(n,e)} = \frac{2^7}{V(7,1)} = \frac{2^7}{1+7} = 2^4 = \abs{C} \]
    So Hamming's original code is perfect.
\end{example}
\begin{example}
    The binary repetition code of length \( n \) is perfect if and only if \( n \) is odd.
\end{example}
\begin{remark}
    If \( \frac{2^n}{V(n,e)} \) is not an integer, there does not exist a perfect \( e \)-error correcting code of length \( n \).
    The converse is false; the case \( n = 90, e = 2 \) is discussed on the second example sheet.
\end{remark}
\begin{definition}
    \( A(n,d) \) is the largest possible size \( m \) of an \( [n,m,d] \)-code.
\end{definition}
The values of the \( A(n,d) \) are unknown in general.
\begin{example}
    \( A(n,1) = 2^n \), considering the trivial code.
    \( A(n,2) = 2^{n-1} \), maximised at the simple parity check code.
    \( A(n,n) = 2 \), maximised at the repetition code.
\end{example}
\begin{lemma}
    \( A(n,d+1) \leq A(n,d) \).
\end{lemma}
\begin{proof}
    Let \( m = A(n,d+1) \), and let \( C \) be an \( [n,m,d+1] \)-code.
    Let \( c_1, c_2 \in C \) be distinct codewords such that \( d(c_1,c_2) = d+1 \).
    Let \( c_1' \) differ from \( c_1 \) in exactly one of the places where \( c_1 \) and \( c_2 \) differ.
    Then \( d(c_1', c_2) = d \).
    If \( c \in C \) is any codeword not equal to \( c_1 \), then \( d(c,c_1) \leq d(c,c_1') + d(c_1',c_1) \) hence \( d + 1 \leq d(c,c_1') + 1 \), so the code given by \( C \cup \qty{c_1'} \setminus \qty{c_1} \) has minimum distance \( d \), but has length \( n \) and size \( m \).
    This is therefore an \( [n,m,d] \)-code as required.
\end{proof}
\begin{corollary}
    Equivalently, \( A(n,d) = \max \qty{m \mid \exists [n,m,d'] \text{-code, for some } d' \geq d} \).
\end{corollary}
\begin{theorem}
    \[ \frac{2^n}{V(n,d-1)} \leq A(n,d) \leq \frac{2^n}{V\qty(n,\floor{\frac{d-1}{2}})} \]
\end{theorem}
The upper bound is Hamming's bound; the lower bound is known as the GSV (Gilbert--Shannon--Varshamov) bound.
The upper bound can be thought of as a sphere packing bound, and the lower bound is a sphere covering bound.
\begin{proof}
    We prove the lower bound.
    Let \( m = A(n,d) \), and let \( C \) be an \( [n,m,d] \)-code.
    Then, there exists no \( x \in \mathbb F_2^n \) with \( d(x,c) \geq d \) for all codewords.
    Indeed, if such an \( x \) exists, we could consider the code \( C \cup \qty{x} \), which would be an \( [n,m+1,d] \)-code, contradicting maximality of \( m \).
    Then,
    \[ \mathbb F_2^n \subseteq \bigcup_{c \in C} B(c,d-1) \implies 2^n \leq \sum_{c \in C} \abs{B(c,d-1)} = mV(n,d-1) \]
    as required.
\end{proof}
\begin{example}
    Let \( n = 10, d = 3 \).
    Then \( V(n,1) = 11 \) and \( V(n,2) = 56 \), so the GSV bound is \( \frac{2^{10}}{56} \leq A(10,3) \leq \frac{2^{10}}{11} \).
    Hence, \( 19 \leq A(10,3) \leq 93 \).
    It was known that the lower bound could be improved to 72.
    We now know that the true value of \( A(10,3) \) is exactly 72.
    In this case, the GSV bound was not a sharp inequality.
\end{example}

\subsection{Asymptotics}
We study the information rate \( \frac{\log A(n,\floor{n\delta})}{n} \) as \( n \to \infty \) to see how large the information rate can be for a fixed error rate.
\begin{proposition}
    Let \( 0 < \delta < \frac{1}{2} \).
    Then,
    \begin{enumerate}
        \item \( \log V(n,\floor{n\delta}) \leq nH(\delta) \);
        \item \( \frac{1}{n} \log A(n,\floor{n\delta}) \geq 1 - H(\delta) \);
    \end{enumerate}
    where \( H(\delta) = -\delta \log \delta - (1-\delta)\log (1-\delta) \).
\end{proposition}
\begin{proof}
    \emph{(i) implies (ii).}
    By the GSV bound, we find
    \[ A(n,\floor{n\delta}) \geq \frac{2^n}{V(n,\floor{n\delta} - 1)} \geq \frac{2^n}{V(n,\floor{n\delta})} \]
    Taking logarithms,
    \[ \frac{1}{n}\log A(n,\floor{n\delta}) \geq 1 - \frac{\log V(n,\floor{n\delta})}{n} \geq 1 - H(\delta) \]
    \emph{Part (i).}
    \( H(\delta) \) is increasing for \( \delta < \frac{1}{2} \).
    Therefore, without loss of generality, we may assume \( n\delta \) is an integer.
    Now, as \( \frac{\delta}{1-\delta} < 1 \),
    \begin{align*}
        1 &= (\delta + (1-\delta))^n \\
        &= \sum_{i=0}^n \binom{n}{i} \delta^i (1-\delta)^{n-i} \\
        &\geq \sum_{i=0}^{n\delta} \binom{n}{i} \delta^i (1-\delta)^{n-i} \\
        &= (1-\delta)^n \sum_{i=0}^{n\delta} \binom{n}{i} \qty(\frac{\delta}{1-\delta})^i \\
        &\geq (1-\delta)^n \sum_{i=0}^{n\delta} \binom{n}{i} \qty(\frac{\delta}{1-\delta})^{n\delta} \\
        &= \delta^{n\delta} (1-\delta)^{n(1-\delta)} V(n,n\delta)
    \end{align*}
    Taking logarithms,
    \[ 0 \geq n\delta \log \delta + n(1-\delta) \log(1-\delta) + \log V(n,n\delta) \]
    as required.
\end{proof}
The constant \( H(\delta) \) in the proposition is optimal.
\begin{lemma}
    \( \lim_{n \to \infty} \frac{\log V(n,\floor{n\delta})}{n} = H(\delta) \).
\end{lemma}
\begin{proof}
    Exercise.
    Follows from Stirling's approximation to factorials.
\end{proof}

\subsection{Constructing new codes from old}
Let \( C \) be an \( [n,m,d] \)-code.
\begin{example}
    The \emph{parity check extension} is an \( [n+1,m,d'] \)-code given by
    \[ C^+ = \qty{\qty(c_1, \dots, c_n, \sum_{i=1}^n c_i) \midd (c_1, \dots, c_n) \in C} \]
    where \( d' \) is either \( d \) or \( d + 1 \), depending on whether \( d \) is odd or even.
\end{example}
\begin{example}
    Let \( 1 \leq i \leq n \).
    Then, deleting the \( i \)th digit from each codeword gives the \emph{punctured code}
    \[ C^- = \qty{(c_1, \dots, c_{i-1}, c_{i+1}, \dots, c_n) \midd (c_1, \dots, c_n) \in C} \]
    If \( d \geq 2 \), this is an \( [n-1, m, d'] \)-code where \( d' \) is either \( d \) or \( d - 1 \).
\end{example}
\begin{example}
    Let \( 1 \leq i \leq n \) and let \( \alpha \in \mathbb F_2 \).
    The \emph{shortened code} is
    \[ C' = \qty{(c_1, \dots, c_{i-1}, c_{i+1}, \dots, c_n) \midd (c_1, \dots, c_{i-1}, \alpha, c_{i+1}, \dots, c_n) \in C} \]
    This is an \( [n-1,m',d'] \) with \( d' \geq d \) and \( m' \geq \frac{m}{2} \) for a suitable choice of \( \alpha \).
\end{example}
