\subsection{Linear codes}
\begin{definition}
    A binary code \( C \subseteq \mathbb F_2^n \) is \emph{linear} if \( 0 \in C \), and whenever \( x, y \in C \), we have \( x + y \in C \).
\end{definition}
Equivalently, \( C \) is a vector subspace of \( \mathbb F_2^n \).
\begin{definition}
    The \emph{rank} of a linear code \( C \), denoted \( \operatorname{rk} C \), is its dimension as an \( \mathbb F_2 \)-vector space.
    A linear code of length \( n \) and rank \( k \) is called an \( (n,k) \)-code.
    If it has minimum distance \( d \), it is called an \( (n,k,d) \)-code.
\end{definition}
Let \( v_1, \dots, v_k \) be a basis for \( C \).
Then \( C = \qty{\sum_{i=1}^k \lambda_i v_i \mid \lambda_i \in \mathbb F_2} \).
The size of the code is therefore \( 2^k \), so an \( (n,k) \)-code is an \( [n,2^k] \)-code, and an \( (n,k,d) \)-code is an \( [n,2^k,d] \)-code.
The information rate is \( \frac{k}{n} \).
\begin{definition}
    The \emph{weight} of \( x \in \mathbb F_2^n \) is \( w(x) = d(x,0) \).
\end{definition}
\begin{lemma}
    The minimum distance of a linear code is the minimum weight of a nonzero codeword.
\end{lemma}
\begin{proof}
    Let \( x, y \in C \).
    Then, \( d(x,y) = d(x+y,0) = w(x+y) \).
    Observe that \( x \neq y \) if and only if \( x + y \neq 0 \), so \( d(C) \) is the minimum \( w(x+y) \) for \( x + y \neq 0 \). 
\end{proof}
\begin{definition}
    Let \( x, y \in \mathbb F_2^n \).
    Define \( x \cdot y = \sum_{i=1}^n x_i y_i \in \mathbb F_2 \).
    This is symmetric and bilinear.
\end{definition}
There are nonzero \( x \) such that \( x \cdot x = 0 \).
\begin{definition}
    Let \( P \subseteq \mathbb F_2^n \).
    The \emph{parity check code} defined by \( P \) is
    \[ C = \qty{x \in \mathbb F_2^n \mid \forall p \in P,\,p \cdot x = 0} \]
\end{definition}
\begin{example}
    \begin{enumerate}
        \item \( P = \qty{11\dots 1} \) gives the simple parity check code.
        \item \( P = \qty{1010101, 0110011, 0001111} \) gives Hamming's original \( [7,16,3] \)-code.
        \item \( C^+ \) and \( C^- \) are linear if \( C \) is linear.
    \end{enumerate}
\end{example}
\begin{lemma}
    Every parity check code is linear.
\end{lemma}
\begin{proof}
    \( 0 \in C \) as \( p \cdot 0 = 0 \).
    If \( p \cdot x = 0 \) and \( p \cdot y = 0 \) then \( p \cdot (x + y) = 0 \), so \( x, y \in C \) implies \( x + y \in C \).
\end{proof}
\begin{definition}
    Let \( C \subseteq \mathbb F^2 \) be a linear code.
    The \emph{dual code} \( C^\perp \) is defined by
    \[ C^\perp = \qty{x \in \mathbb F_2^n \mid \forall y \in C,\, x \cdot y = 0} \]
\end{definition}
By definition, \( C^\perp \) is a parity check code, and hence is linear.
Note that \( C \cap c^\perp \) may contain elements other than 0.
\begin{lemma}
    \( \operatorname{rk} C + \operatorname{rk} C^\perp = n \).
\end{lemma}
\begin{proof}
    One can prove this by defining \( C^\perp \) as an annihilator from linear algebra.
    A proof using coding theory is shown later.
\end{proof}
\begin{corollary}
    Let \( C \) be a linear code.
    Then \( (C^\perp)^\perp = C \).
    In particular, all linear codes are parity check codes, defined by \( C^\perp \).
\end{corollary}
\begin{proof}
    If \( x \in C \), then \( x \cdot y = 0 \) for all \( y \in C^\perp \) by definition, so \( x \in (C^\perp)^\perp \).
    Then \( \operatorname{rk} C = n - \operatorname{rk} C^\perp = n - (n - \operatorname{rk} (C^\perp)^\perp) = \operatorname{rk} (C^\perp)^\perp \), so \( C = (C^\perp)^\perp \).
\end{proof}
\begin{definition}
    Let \( C \) be an \( (n,k) \)-code.
    A \emph{generator matrix} \( G \) for \( C \) is a \( k \times n \) matrix where the rows form a basis for \( C \).
    A \emph{parity check matrix} \( H \) for \( C \) is a generator matrix for the dual code \( C^\perp \), so it is an \( (n-k) \times n \) matrix.
\end{definition}
The codewords of a linear code can be viewed either as linear combinations of rows of \( G \), or linear dependence relations between the columns of \( H \), so \( C = \qty{x \in \mathbb F_2^n \mid H x = 0} \).
