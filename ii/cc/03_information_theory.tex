\subsection{Sources and information rate}
\begin{definition}
    A \emph{source} is a sequence of random variables \( X_1, X_2, \dots \) taking values in \( \mathcal A \).
\end{definition}
\begin{example}
    The \emph{Bernoulli} (or \emph{memoryless}) source is a source where the \( X_i \) are independent and identically distributed according to a Bernoulli distribution.
\end{example}
\begin{definition}
    A source \( X_1, X_2, \dots \) is \emph{reliably encodable} at rate \( r \) if there exist subsets \( A_n \subseteq \mathcal A^n \) such that
    \begin{enumerate}
        \item \( \lim \frac{\log \abs{A_n}}{n} = r \);
        \item \( \lim \prob{(X_1, \dots, X_n) \in A_n} = 1 \).
    \end{enumerate}
\end{definition}
\begin{definition}
    The \emph{information rate} \( H \) of a source is the infimum of all reliable encoding rates.
\end{definition}
\begin{example}
    \( 0 \leq H \leq \log\abs{\mathcal A} \), with both bounds attainable.
    The proof is left as an exercise.
\end{example}
Shannon's first coding theorem computes the information rate of certain sources, including Bernoulli sources.

Recall from IA Probability that a probability space is a tuple \( (\Omega, \mathcal F, \mathbb P) \), and a discrete random variable is a function \( X \colon \Omega \to \mathcal A \).
The probability mass function is the function \( p_X \colon \mathcal A \to [0,1] \) given by \( p_X(x) = \prob{X = x} \).
We can consider the function \( p(X) \colon \Omega \to [0,1] \) defined by the composition \( p_X \circ X \), which assigns \( p(X)(\omega) = \prob{X = X(\omega)} \); hence, \( p(X) \) is also a random variable.

Similarly, given a source \( X_1, X_2, \dots \) of random variables with values in \( \mathcal A \), the probability mass function of any tuple \( X^{(n)} = (X_1, \dots, X_n) \) is \( p_{X^{(n)}}(x_1, \dots, x_n) = \prob{X_1 = x_1, \dots, X_n = x_n} \).
As \( p_{X^{(n)}} \colon \mathcal A^n \to [0,1] \), and \( X^{(n)} \colon \Omega \to \mathcal A^n \), we can consider \( p(X^{(n)}) = p_{X^{(n)}} \circ X^{(n)} \) defined by \( \omega \mapsto p_{X^{(n)}}(X^{(n)}(\omega)) \).
\begin{example}
    Let \( \mathcal A = \qty{A, B, C} \).
    Suppose
    \[ X^{(2)} = \begin{cases}
        AB & \text{with probability } 0.3 \\
        AC & \text{with probability } 0.1 \\
        BC & \text{with probability } 0.1 \\
        BA & \text{with probability } 0.2 \\
        CA & \text{with probability } 0.25 \\
        CB & \text{with probability } 0.05
    \end{cases} \]
    Then, \( p_{X^{(2)}}(AB) = 0.3 \), and so on.
    Hence,
    \[ p(X^{(2)}) = \begin{cases}
        0.3 & \text{with probability } 0.3 \\
        0.1 & \text{with probability } 0.2 \\
        0.2 & \text{with probability } 0.2 \\
        0.25 & \text{with probability } 0.25 \\
        0.05 & \text{with probability } 0.05
    \end{cases} \]
\end{example}
We say that a source \( X_1,X_2, \dots \) converges in probability to a random variable \( L \) if for all \( \varepsilon > 0 \), \( \lim_{n \to \infty} \prob{\abs{X_n - L} > \varepsilon} = 0 \).
We write \( X_n \xrightarrow{\mathbb P} L \).
The weak law of large numbers states that if \( X_1, X_2, \dots \) is a sequence of independent identically distributed real-valued random variables with finite expectation \( \expect{X_1} \), then \( \frac{1}{n} \sum_{i=1}^n X_i \xrightarrow{\mathbb P} \expect{X} \).
\begin{example}
    Let \( X_1, X_2, \dots \) be a Bernoulli source.
    Then \( p(X_1), p(X_2), \dots \) are independent and identically distributed random variables, and \( p(X_1, \dots, X_n) = p(X_1) \dots p(X_n) \).
    Note that by the weak law of large numbers,
    \[ -\frac{1}{n} \log p(X_1, \dots, X_n) = -\frac{1}{n} \sum_{i=1}^n \log p(X_i) \xrightarrow{\mathbb P} \expect{-\log p(X_1)} = H(X_1) \]
\end{example}
\begin{lemma}
    The information rate of a Bernoulli source \( X_1, X_2, \dots \) is at most the expected word length of an optimal code \( c \colon \mathcal A \to \qty{0,1}^\star \) for \( X_1 \).
\end{lemma}
\begin{proof}
    Let \( \ell_1, \ell_2, \dots \) be the codeword lengths when we encode \( X_1, X_2, \dots \) using \( c \).
    Let \( \varepsilon > 0 \).
    Let
    \[ A_n = \qty{x \in \mathcal A^n \mid c^\star(x) \text{ has length less than } n \qty(\expect{\ell_1} + \varepsilon)} \]
    Then,
    \[ \prob{(X_1, \dots, X_n) \in A_n} = \prob{\sum_{i=1}^n \ell_i \leq n \qty(\expect{\ell_1} + \varepsilon)} = \prob{\abs{\frac{1}{n} \sum_{i=1}^n \ell_i - \expect{\ell_i}} < \varepsilon} \to 1 \]
    Now, \( c \) is decipherable so \( c^\star \) is injective.
    Hence, \( \abs{A_n} \leq 2^{n\qty(\expect{\ell_1} + \varepsilon)} \).
    Making \( A_n \) larger if necessary, we can assume \( \abs{A_n} = \floor{2^{n\qty(\expect{\ell_1} + \varepsilon)}} \).
    Taking logarithms, \( \frac{\log \abs{A_n}}{n} \to \expect{\ell_1} + \varepsilon \).
    So \( X_1, X_2, \dots \) is reliably encodable at rate \( r = \expect{\ell_1} + \varepsilon \) for all \( \varepsilon > 0 \).
    Hence the information rate is at most \( \expect{\ell_1} \).
\end{proof}
\begin{corollary}
    A Bernoulli source has information rate less than \( H(X_1) + 1 \).
\end{corollary}
\begin{proof}
    Combine the previous lemma with the noiseless coding theorem.
\end{proof}
Suppose we encode \( X_1, X_2, \dots \) in blocks of size \( N \).
Let \( Y_1 = (X_1, \dots, X_N), Y_2 = (X_{N+1}, \dots, X_{2N}) \) and so on, such that \( Y_1, Y_2, \dots \) take values in \( \mathcal A^N \).
One can show that if the source \( X_1, X_2, \dots \) has information rate \( H \), then \( Y_1, Y_2, \dots \) has information rate \( NH \).
\begin{proposition}
    The information rate \( H \) of a Bernoulli source is at most \( H(X_1) \).
\end{proposition}
\begin{proof}
    Apply the previous corollary to the \( Y_i \) to obtain
    \[ NH < H(Y_1) + 1 = H(X_1, \dots, X_N) + 1 = NH(X_1) + 1 \implies H < H(X_1) + \frac{1}{N} \]
    as required.
\end{proof}

\subsection{Asymptotic equipartition property}
\begin{definition}
    A source \( X_1, X_2, \dots \) satisfies the \emph{asymptotic equipartition property} if there exists a constant \( H \geq 0 \) such that
    \[ -\frac{1}{n} \log p(X_1, \dots, X_n) \xrightarrow{\mathbb P} H \]
\end{definition}
\begin{example}
    Suppose we toss a biased coin with probability \( p \) of obtaining a head.
    Let \( X_1, X_2, \dots \) be the results of independent coin tosses.
    If we toss the coin \( N \) times, we expect \( pN \) heads and \( (1-p)N \) tails.
    The probability of any particular sequence of \( pN \) heads and \( (1-p)N \) tails is
    \[ p^{pN} (1-p)^{(1-p)N} = 2^{N (p \log p + (1-p) \log(1-p))} = 2^{-NH(X)} \]
    Not every sequence of tosses is of this form, but there is only a small probability of `atypical sequences'.
    With high probability, it is a `typical sequence' which has a probability close to \( 2^{-NH(X)} \).
\end{example}
\begin{lemma}
    The asymptotic equipartition property for a source \( X_1, X_2, \dots \) is equivalent to the property that for all \( \varepsilon > 0 \), there exists \( n \in \mathbb N \) such that for all \( n \geq n_0 \), there exists a `typical set' \( T_n \subseteq \mathcal A^n \) such that
    \begin{enumerate}
        \item \( \prob{(X_1, \dots, X_n) \in T_n} > 1 - \varepsilon \);
        \item \( 2^{-n(H+\varepsilon)} \leq p(x_1, \dots, x_n) \leq 2^{-n(H-\varepsilon)} \) for all \( (x_1, \dots, x_n) \in T_n \).
    \end{enumerate}
\end{lemma}
\begin{proof}[Proof sketch]
    First, we show that the asymptotic equipartition property implies the alternative definition.
    We define
    \[ T_n = \qty{(x_1, \dots, x_n) \midd \abs{-\frac{1}{n} \log p(x_1, \dots, x_n) - H} \leq \varepsilon} = \qty{(x_1, \dots, x_n) \mid \text{condition (ii) holds}} \]
    For the converse,
    \[ \prob{\abs{\frac{1}{n}\log p(x_1, \dots, x_n) - H} < \varepsilon} \geq \prob{T_n} \to 1 \]
\end{proof}

\subsection{Shannon's first coding theorem}
\begin{theorem}
    Let \( X_1, X_2, \dots \) be a source satisfying the asymptotic equipartition property with constant \( H \).
    Then this source has information rate \( H \).
\end{theorem}
\begin{proof}
    Let \( \varepsilon > 0 \), and let \( T_n \subseteq \mathcal A^n \) be typical sets.
    Then, for all \( n \geq n_0(\varepsilon) \), for all \( (x_1, \dots, x_n) \in T_n \) we have \( p(x_1, \dots, x_n) \geq 2^{-n(H + \varepsilon)} \).
    Therefore, \( 1 \geq \prob{T_n} \geq 2^{-n(H + \varepsilon)} \cdot \abs{T_n} \), giving \( \frac{1}{n} \log \abs{T_n} \leq H + \varepsilon \).
    Taking \( A_n = T_n \) in the definition of reliable encoding shows that the source is reliably encodable at rate \( H + \varepsilon \).

    Conversely, if \( H = 0 \) the proof concludes, so we may assume \( H > 0 \).
    Let \( 0 < \varepsilon < \frac{H}{2} \), and suppose that the source is reliably encodable at rate \( H - 2\varepsilon \) with sets \( A_n \subseteq \mathcal A^n \).
    Let \( T_n \subseteq \mathcal A^n \) be typical sets.
    Then, for all \( (x_1, \dots, x_n) \in T_n \), \( p(x_1, \dots, x_n) \leq 2^{-n(H - \varepsilon)} \), so \( \prob{A_n \cap T_n} \leq 2^{-n(H - \varepsilon)} \abs{A_n} \), giving
    \[ \frac{1}{n} \log \prob{A_n \cap T_n} \leq -(H - \varepsilon) + \frac{1}{n} \log \abs{A_n} \to -(H - \varepsilon) + (H - 2 \varepsilon) = -\varepsilon \]
    Then, \( \log \prob{A_n \cap T_n} \to -\infty \), so \( \prob{A_n \cap T_n} \to 0 \).
    But \( \prob{T_n} \leq \prob{A_n \cap T_n} + \prob{\mathcal A^n \setminus A_n} \to 0 + 0 \), contradicting typicality.
    So we cannot reliably encode at rate \( H - \varepsilon \), so the information rate is at least \( H \).
\end{proof}
\begin{corollary}
    A Bernoulli source \( X_1, X_2, \dots \) has information rate \( H(X_1) \).
\end{corollary}
\begin{proof}
    In a previous example we showed that for a Bernoulli source, \( -\frac{1}{n} \log(X_1, \dots, X_n) \xrightarrow{\mathbb P} H(X_1) \).
    So the asymptotic equipartition property holds with \( H = H(X_1) \), giving the result by Shannon's first coding theorem.
\end{proof}
\begin{remark}
    The asymptotic equipartition property is useful for noiseless coding.
    We can encode the typical sequences using a block code, and encode the atypical sequences arbitrarily.

    Many sources, which are not necessarily Bernoulli, also satisfy the property.
    Under suitable hypotheses, the sequence \( \frac{1}{n} H(X_1, \dots, X_n) \) is decreasing, and the asymptotic equipartition property is satisfied with constant \( H = \lim_{n \to \infty} \frac{1}{n} H(X_1, \dots, X_n) \).
\end{remark}

\subsection{Capacity}
Consider a communication channel with input alphabet \( \mathcal A \) and output alphabet \( \mathcal B \).
Recall the following definitions.
A \emph{code} of length \( n \) is a subset \( C \subseteq \mathcal A^n \).
The \emph{error rate} is
\[ \hat e(C) = \max_{c \in C} \prob{\text{error} \mid c \text{ sent}} \]
The \emph{information rate} is \( \rho(C) = \frac{\log \abs{C}}{n} \).
A channel can \emph{transmit reliably at rate \( R \)} if there exist codes \( C_1, C_2, \dots \) where \( C_n \) has length \( n \) such that \( \lim_{n \to \infty} \rho(C_n) = R \) and \( \lim_{n \to \infty} \hat e(C_n) = 0 \).
The \emph{(operational) capacity} of a channel is the supremum of all rates at which it can transmit reliably.

Suppose we are given a source with information rate \( r \) bits per second that emits symbols at a rate of \( s \) symbols per second.
Suppose we also have a channel with capacity \( R \) bits per transmission that transmits symbols at a rate of \( S \) transmissions per second.
Usually, information theorists take \( S = s = 1 \).
We will show that reliable encoding and transmission is possible if and only if \( rs \leq RS \).

We will now compute the capacity of the binary symmetric channel with error probability \( p \).
\begin{proposition}
    A binary symmetric channel with error probability \( p < \frac{1}{4} \) has nonzero capacity.
\end{proposition}
\begin{proof}
    Let \( \delta \) be such that \( 2p < \delta < \frac{1}{2} \).
    We claim that we can reliably transmit at rate \( R = 1 - H(\delta) > 0 \).
    Let \( C_n \) be a code of length \( n \), and suppose it has minimum distance \( \floor{n\delta} \) of maximal size.
    Then, by the GSV bound,
    \[ \abs{C_n} = A(n, \floor{n\delta}) \geq 2^{-n(1-H(\delta))} = 2^{nR} \]
    Replacing \( C_n \) with a subcode if necessary, we can assume \( \abs{C_n} = \floor{2^{nR}} \), with minimum distance at least \( \floor{n\delta} \).
    Using minimum distance decoding,
    \begin{align*}
        \hat e(C_n) &\leq \prob{\text{in \( n \) uses, the channel makes at least } \floor*{\frac{\floor{n\delta}-1}{2}} \text{ errors}} \\
        &\leq \prob{\text{in \( n \) uses, the channel makes at least } \floor*{\frac{n\delta - 1}{2}} \text{ errors}}
    \end{align*}
    Let \( \varepsilon > 0 \) be such that \( p + \varepsilon < \frac{\delta}{2} \).
    Then, for \( n \) sufficiently large, \( \frac{n\delta - 1}{2} = n\qty(\frac{\delta}{2} - \frac{1}{2n}) > n(p + \varepsilon) \).
    Hence, \( \hat e(C_n) \leq \prob{\text{in \( n \) uses, the channel makes at least } n(p+\varepsilon) \text{ errors}} \).
    We show that this value converges to zero as \( n \to \infty \) using the next lemma.
\end{proof}
\begin{lemma}
    Let \( \varepsilon > 0 \).
    A binary symmetric channel with error probability \( p \) is used to transmit \( n \) digits.
    Then,
    \[ \lim_{n \to \infty} \prob{\text{in \( n \) uses, the channel makes at least \( n(p + \varepsilon) \) errors}} = 0 \]
\end{lemma}
\begin{proof}
    Consider random variables \( U_i = \mathbbm 1\qty[\text{the \( i \)th digit is mistransmitted}] \).
    The \( U_i \) are independent and identically distributed with \( \prob{U_i = 1} = p \).
    In particular, \( \expect{U_i} = p \).
    Therefore, the probability that the channel makes at least \( n(p + \varepsilon) \) errors is
    \[ \prob{\sum_{i=1}^n U_i \geq n(p + \varepsilon)} \leq \prob{\abs{\frac{1}{n} \sum_{i=1}^n U_i - p} \geq \varepsilon} \]
    so the result holds by the weak law of large numbers.
\end{proof}

\subsection{Conditional entropy}
\begin{definition}
    Let \( X, Y \) be random variables taking values in alphabets \( \mathcal A, \mathcal B \) respectively.
    Then, the \emph{conditional entropy} is defined by
    \[ H(X \mid Y = y) = - \sum_{x \in \mathcal A} \prob{X = x \mid Y = y} \log \prob{X = x \mid Y = y} \]
    and
    \[ H(X \mid Y) = \sum_{y \in \mathcal B} \prob{Y = y} H(X \mid Y = y) \]
    Note that \( H(X \mid Y) \geq 0 \).
\end{definition}
\begin{lemma}
    \( H(X,Y) = H(X \mid Y) + H(Y) \).
\end{lemma}
\begin{proof}
    \begin{align*}
        H(X \mid Y) &= -\sum_{y \in \mathcal B} \sum_{x \in \mathcal A} \prob{X = x \mid Y = y} \prob{Y = y} \log \qty(\prob{X = x \mid Y = y}) \\
        &= -\sum_{y \in \mathcal B} \sum_{x \in \mathcal A} \prob{X = x \mid Y = y} \prob{Y = y} \log \qty(\frac{\prob{X = x, Y = y}}{\prob{Y = y}}) \\
        &= -\sum_{y \in \mathcal B} \sum_{x \in \mathcal A} \prob{X = x, Y = y} \qty(\log \prob{X = x, Y = y} - \log \prob{Y = y}) \\
        &= -\sum_{y \in \mathcal B} \sum_{x \in \mathcal A} \prob{X = x, Y = y} \log \prob{X = x, Y = y} + \sum_{y \in \mathcal B} \sum_{x \in \mathcal A} \prob{X = x, Y = y} \log \prob{Y = y} \\
        &= -\sum_{y \in \mathcal B} \sum_{x \in \mathcal A} \prob{X = x, Y = y} \log \prob{X = x, Y = y} + \sum_{y \in \mathcal B} \prob{Y = y} \log \prob{Y = y} \\
        &= H(X,Y) - H(Y)
    \end{align*}
\end{proof}
\begin{example}
    Let \( X \) be a uniform random variable on \( \qty{1, \dots, 6} \) modelling a dice roll, and \( Y \) is defined to be zero if \( X \) is even, and one if \( X \) is odd.
    Then, \( H(X,Y) = H(X) = \log 6 \) and \( H(Y) = \log 2 \).
    Therefore, \( H(X \mid Y) = \log 3 \) and \( H(Y \mid X) = 0 \).
\end{example}
\begin{corollary}
    \( H(X\mid Y) \leq H(X) \), with equality if and only if \( X \) and \( Y \) are independent.
\end{corollary}
\begin{proof}
    Combine this result with the fact that \( H(X,Y) \leq H(X) + H(Y) \) where equality holds if and only if \( H(X), H(Y) \) are independent.
\end{proof}
Now, replace random variables \( X \) and \( Y \) with random vectors \( X^{(r)} = (X_1, \dots, X_r) \) and \( Y^{(s)} = (Y_1, \dots, Y_s) \).
Similarly, we can define \( H(X_1, \dots, X_r \mid Y_1, \dots, Y_s) = H(X^{(r)} \mid Y^{(s)}) \).
Note that \( H(X,Y\mid Z) \) is the entropy of \( X \) and \( Y \) combined, given the value of \( Z \), and is not the entropy of \( X \), together with \( Y \) given \( Z \).
\begin{lemma}
    Let \( X, Y, Z \) be random variables.
    Then, \( H(X \mid Y) \leq H(X \mid Y, Z) + H(Z) \).
\end{lemma}
\begin{proof}
    Expand \( H(X,Y,Z) \) in two ways.
    \[ H(Z \mid X,Y) + \underbrace{H(X\mid Y) + H(Y)}_{H(X,Y)} = H(X,Y,Z) = H(X \mid Y,Z) + \underbrace{H(Z \mid Y) + H(Y)}_{H(Y,Z)} \]
    Since \( H(Z \mid X,Y) \geq 0 \), we have
    \[ H(X \mid Y) \leq H(X \mid Y,Z) + H(Z \mid Y) \leq H(X \mid Y,Z) + H(Z) \]
\end{proof}
\begin{proposition}[Fano's inequality]
    Let \( X, Y \) be random variables taking values in \( \mathcal A \).
    Let \( \abs{\mathcal A} = m \), and let \( p = \prob{X \neq Y} \).
    Then \( H(X \mid Y) \leq H(p) + p \log(m-1) \).
\end{proposition}
\begin{proof}
    Define \( Z \) to be zero if \( X = Y \) and one if \( X \neq Y \).
    Then, \( \prob{Z = 0} = \prob{X = Y} = 1 - p \), and \( \prob{Z = 1} = \prob{X \neq Y} = p \).
    Hence, \( H(Z) = H(p) \).
    Applying the previous lemma, \( H(X \mid Y) \leq H(X \mid Y, Z) + H(p) \), so it suffices to show \( H(X \mid Y, Z) \leq p\log(m-1) \).

    Since \( Z = 0 \) implies \( X = Y \), \( H(X \mid Y = y, Z = 0) = 0 \).
    There are \( m - 1 \) remaining possibilities for \( X \).
    Hence, \( H(X \mid Y = y, Z = 1) \leq \log(m-1) \).
    \begin{align*}
        H(X \mid Y, Z) &= \sum_{y \in \mathcal A} \sum_{z \in \qty{0,1}} \prob{Y = y, Z = z} H(X \mid Y = y, Z = z) \\
        &\leq \sum_{y \in \mathcal A} \prob{Y = y, Z = 1} \log(m-1) \\
        &= \prob{Z = 1} \log (m-1) \\
        &= p\log(m-1)
    \end{align*}
    as required.
\end{proof}
Let \( X \) be a random variable describing the input to a channel and \( Y \) be a random variable describing the output of the channel.
\( H(p) \) provides the information required to decide whether an error has occurred, and \( p\log(m-1) \) gives the information needed to resolve that error in the worst possible case.

\subsection{Shannon's second coding theorem}
\begin{definition}
    Let \( X, Y \) be random variables taking values in \( \mathcal A \).
    The \emph{mutual information} is \( I(X;Y) = H(X) - H(X \mid Y) \).
\end{definition}
This is nonnegative, as \( I(X;Y) = H(X) + H(Y) - H(X,Y) \geq 0 \).
Equality holds if and only if \( X, Y \) are independent.
Clearly, \( I(X;Y) = I(Y;X) \).
\begin{definition}
    Consider a discrete memoryless channel with input alphabet \( \mathcal A \) of size \( m \) and output alphabet \( \mathcal B \).
    Let \( X \) be a random variable taking values in \( \mathcal A \), used as the input to this channel.
    Let \( Y \) be the random variable output by the channel, depending on \( X \) and the channel matrix.
    The \emph{information capacity} of the channel is \( \max_{X} I(X;Y) \).
\end{definition}
The maximum is taken over all discrete random variables \( X \) taking values in \( \mathcal A \), or equivalently.
This maximum is attained since \( I \) is continuous and the space
\[ \qty{(p_1, \dots, p_m) \in \mathbb R^m \midd p_i \geq 0, \sum_{i=1}^m p_i = 1} \]
is compact.
The information capacity depends only on the channel matrix.
\begin{theorem}
    For a discrete memoryless channel, the (operational) capacity is equal to the information capacity.
\end{theorem}
We prove that the operational capacity is at most the information capacity in general, and we will prove the other inequality for the special case of the binary symmetric channel.
\begin{example}
    Assuming this result holds, we compute the capacity of certain specific channels.
    \begin{enumerate}
        \item Consider the binary symmetric channel with error probability \( p \), input \( X \), and output \( Y \).
        Let \( \prob{X = 0} = \alpha, \prob{X = 1} = 1 - \alpha \), so \( \prob{Y = 0} = (1-p)\alpha p(1-\alpha), \prob{Y = 1} = (1-p)(1-\alpha) + p\alpha \).
        Then, as \( H(Y \mid X) = \prob{X=0} H(p) + \prob{X = 1} H(p) \),
        \begin{align*}
            C &= \max_\alpha I(X;Y) = \max_\alpha [H(Y) - H(Y\mid X)] \\
            &= \max_\alpha \qty[H(\alpha(1-p) + (1-\alpha)p) - H(p)] = 1 - H(p)
        \end{align*}
        with the maximum attained at \( \alpha = \frac{1}{2} \).
        Hence, the capacity of the binary symmetric channel is \( C = 1 + p \log p + (1-p) \log (1-p) \).
        If \( p = 0 \) or \( p = 1 \), \( C = 1 \).
        If \( p = \frac{1}{2} \), \( C = 0 \).
        Note that \( I(X;Y) = I(Y;X) \); we can choose which to calculate for convenience.
        \item Consider the binary erasure channel with erasure probability \( p \), input \( X \), and output \( Y \).
        Let \( \prob{X = 0} = \alpha, \prob{X = 1} = 1 - \alpha \), so \( \prob{Y = 0} = (1-p)\alpha, \prob{Y = 1} = (1-p)(1-\alpha), \prob{Y = \star} = p \).
        We obtain
        \[ H(X \mid Y = 0) = 0;\quad H(X \mid Y = 1) = 0;\quad H(X \mid Y = \star) = H(\alpha) \]
        Therefore, \( H(X \mid Y) = pH(\alpha) \), giving
        \begin{align*}
            C &= \max_\alpha I(X;Y) = \max_\alpha \qty[H(X) - H(X \mid Y)] \\
            &= \max_\alpha \qty[H(\alpha) - pH(\alpha)] = (1-p) \max_\alpha H(\alpha) = 1-p
        \end{align*}
        with maximum attained at \( \alpha = \frac{1}{2} \).
    \end{enumerate}
\end{example}
We will now model using a channel \( n \) times as the \emph{\( n \)th extension}, replacing \( \mathcal A \) with \( \mathcal A^n \) and \( \mathcal B \) with \( \mathcal B^n \), and use the channel matrix defined by
\[ \prob{y_1 \dots y_n \text{ received} \mid x_1 \dots x_n \text{ sent}} = \prod_{i=1}^n \prob{y_i \mid x_i} \]
\begin{lemma}
    Consider a discrete memoryless channel with information capacity \( C \).
    Then, its \( n \)th extension has information capacity \( nC \).
\end{lemma}
\begin{proof}
    Let \( X_1, \dots, X_n \) be the input producing an output \( Y_1, \dots, Y_n \).
    Since the channel is memoryless,
    \[ H(Y_1, \dots, Y_n \mid X_1, \dots, X_n) = \sum_{i=1}^n H(Y_i \mid X_1, \dots, X_n) = \sum_{i=1}^n H(Y_i \mid X_i) \]
    Therefore,
    \begin{align*}
        I(X_1, \dots, X_n; Y_1, \dots, Y_n) &= H(Y_1, \dots, Y_n) - H(Y_1, \dots, Y_n \mid X_1, \dots, X_n) \\
        &\leq \sum_{i=1}^n H(Y_i) - \sum_{i=1}^n H(Y_i \mid X_i) \\
        &= \sum_{i=1}^n \qty[H(Y_i) - H(Y_i \mid X_i)] \\
        &= \sum_{i=1}^n I(X_i;Y_i) \leq nC
    \end{align*}
    Equality is attained by taking \( X_1, \dots, X_n \) independent and identically distributed such that \( I(X_i;Y_i) = C \).
    Indeed, if \( X_1, \dots, X_n \) are independent, then so are \( Y_1, \dots, Y_n \), so \( H(Y_1, \dots, Y_n) = \sum_{i=1}^n H(Y_i) \).
    Therefore,
    \[ \max_{X_1, \dots, X_n} I(X_1, \dots, X_n; Y_1, \dots, Y_n) = nC \]
    as required.
\end{proof}
We now prove part of Shannon's second coding theorem, that the operational capacity is at most the information capacity for a discrete memoryless channel.
\begin{proof}
    Let \( C \) be the information capacity.
    Suppose reliable transmission is possible at a rate \( R > C \).
    Then, there is a sequence of codes \( (C_n)_{n \geq 1} \) where \( C_n \) has length \( n \) and size \( \floor{2^{nR}} \), such that \( \lim_{n \to \infty} \rho(C_n) = R \) and \( \lim_{n \to \infty} \hat e(C_n) = 0 \).

    Recall that \( \hat e(C_n) = \max_{c \in C_n} \prob{\text{error} \mid c \text{ sent}} \).
    Define the \emph{average error rate} \( e(C) \) by \( e(C) = \frac{1}{\abs{C_n}} \sum_{c \in C} \prob{\text{error} \mid c \text{ sent}} \).
    Note that \( e(C_n) \leq \hat e(C_n) \).
    As \( \hat e(C_n) \to 0 \), we also have \( e(C_n) \to 0 \).

    Consider an input random variable \( X \) distributed uniformly over \( C_n \).
    Let \( Y \) be the output given by \( X \) and the channel matrix.
    Then \( e(C_n) = \prob{X \neq Y} = p \).
    Hence, \( H(X) = \log \abs{C_n} = \log \floor{2^{nR}} \geq nR - 1 \) for sufficiently large \( n \).
    Also, by Fano's inequality, \( H(X \mid Y) \leq H(p) + p \log(\abs{C_n} - 1) \leq 1 + pnR \).

    Recall that \( I(X;Y) = H(X) - H(X \mid Y) \).
    By the previous lemma, \( nC \geq I(X;Y) \), so
    \[ nC \geq nR - 1 - 1 - pnR \implies pnR \geq n(R - c) - 2 \implies p \geq \frac{n(R - C) - 2}{nR} \]
    As \( n \to \infty \), the right hand side converges to \( \frac{R - C}{R} > 0 \).
    This contradicts the fact that \( p = e(C_n) \to 0 \).
    Hence, we cannot transmit reliably at any rate which exceeds \( C \), hence the capacity is at most \( C \).
\end{proof}
To complete the proof of Shannon's second coding theorem for the binary symmetric channel with error probability \( p \), we prove that the operational capacity is at least \( 1 - H(p) \).
\begin{proposition}
    Consider a binary symmetric channel with error probability \( p \), and let \( R < 1 - H(p) \).
    Then there exists a sequence of codes \( (C_n)_{n \geq 1} \) with \( C_n \) of length \( n \) and size \( \floor{2^{nR}} \) such that \( \lim_{n \to \infty} \rho(C_n) = R \) and \( \lim_{n \to \infty} e(C_n) = 0 \).
\end{proposition}
\begin{remark}
    This proposition deals with the average error rate, instead of the error rate \( \hat e \).
\end{remark}
\begin{proof}
    We use the method of random coding.
    Without loss of generality let \( p < \frac{1}{2} \).
    Let \( \varepsilon > 0 \) such that \( p + \varepsilon < \frac{1}{2} \) and \( R < 1 - H(p + \varepsilon) \).
    We use minimum distance decoding, and in the case of a tie, we make an arbitrary choice.
    Let \( m = \floor{2^{nR}} \), and let \( C = \qty{c_1, \dots, c_m} \) be a code chosen uniformly at random from \( \mathcal C = \qty{[n,m]\text{-codes}} \), a set of size \( \binom{2^n}{m} \).

    Choose \( 1 \leq i \leq m \) uniformly at random, and send \( c_i \) through the channel, and obtain an output \( Y \).
    Then, \( \prob{Y \text{ not decoded as } c_i} \) is the average value of \( e(C) \) for \( C \) ranging over \( \mathcal C \), giving \( \frac{1}{\abs{\mathcal C}} \sum_{C \in \mathcal C} e(C) \).
    We can choose a code \( C_n \in \mathcal C \) such that \( e(C_n) \leq \frac{1}{\abs{\mathcal C}}\sum_{C \in \mathcal C} e(C) \).
    So it suffices to show \( \prob{Y \text{ not decoded as } c_i} \to 0 \).

    Let \( r = \floor{n(p + \varepsilon)} \).
    Then if \( B(Y,r) \cap C = \qty{c_i} \), \( Y \) is correctly decoded as \( c_i \).
    Therefore,
    \[ \prob{Y \text{ not decoded as } c_i} \leq \prob{c_i \not\in B(Y,r)} + \prob{B(Y,r) \cap C \supsetneq \qty{c_i}} \]
    We consider the two cases separately.

    In the first case with \( d(c_i,Y) > r \), \( \prob{d(c_i,Y) > r} \) is the probability that the channel makes more than \( r \) errors, and hence more than \( n(p + \varepsilon) \) errors.
    We have already shown that this converges to zero as \( n \to \infty \).

    In the second case with \( d(c_i,Y) \leq r \), if \( j \neq i \),
    \[ \prob{c_j \in B(Y,r) \mid c_i \in B(Y,r)} = \frac{V(n,r) - 1}{2^n - 1} \leq \frac{V(n,r)}{2^n} \]
    Therefore,
    \begin{align*}
        \prob{B(Y,r) \cap C \supsetneq \qty{c_i}} &\leq \sum_{j \neq i} \prob{c_j \in B(Y,r), c_i \in B(Y,r)} \\
        &\leq \sum_{j \neq i} \prob{c_j \in B(Y,r) \mid c_i \in B(Y,r)} \\
        &\leq (m-1) \frac{V(n,r)}{2^n} \\
        &\leq \frac{mV(n,r)}{2^n} \\
        &\leq 2^{nR} 2^{nH(p+\varepsilon)} 2^{-n} \\
        &= 2^{n(R - (1 - H(p+\varepsilon)))} \to 0
    \end{align*}
    as required.
\end{proof}
\begin{proposition}
    We can replace \( e \) with \( \hat e \) in the previous result.
\end{proposition}
\begin{proof}
    Let \( R' \) be such that \( R < R' < 1 - H(p) \).
    Then, apply the previous result to \( R' \) to construct a sequence of codes \( (C_n')_{n \geq 1} \) of length \( n \) and size \( \floor{2^{nR'}} \), where \( e(C_n') \to 0 \).
    Order the codewords of \( C_n' \) by the probability of error given that the codeword was sent, and delete the worst half.
    This gives a code \( C_n \) with \( \hat e(C_n) \leq 2 e(C_n') \).
    Hence \( \hat e(C_n) \to 0 \) as \( n \to \infty \).
    Since \( C_n \) has length \( n \), and size \( \frac{1}{2} \floor{2^{nR'}} = \floor{2^{nR' - 1}} \).
    But \( 2^{nR' - 1} = 2^{n(R' - \frac{1}{n})} \geq 2^{nR} \) for sufficiently large \( n \).
    So we can replace \( C_n' \) with a code of smaller size \( \floor{2^{nR}} \) and still have \( \hat e(C_n) \to 0 \) and \( \rho(C_n) \to R \) as \( n \to \infty \).
\end{proof}
Therefore, a binary symmetric channel with error probability \( p \) has operational capacity \( 1 - H(p) \), as we can transmit reliably at any rate \( R < 1 - H(p) \), and the capacity is at most \( 1 - H(p) \).
The result shows that codes with certain properties exist, but does not give a way to construct them.

\subsection{The Kelly criterion}
Let \( 0 < p < 1 \), \( u > 0 \), \( 0 \leq w < 1 \).
Suppose that a coin is tossed \( n \) times in succession with probability \( p \) of obtaining a head.
If a stake of \( k \) is paid ahead of a particular throw, the return is \( ku \) if the result is a head, and the return is zero if the result is a tail.

Suppose the initial bankroll is \( X_0 = 1 \).
After \( n \) throws, the bankroll is \( X_n \).
We bet \( w X_n \) on the \( (n + 1) \)th coin toss, retaining \( (1-w)X_n \).
The bankroll after the toss is
\[ X_{n+1} = \begin{cases}
    X_n(wu + (1-w)) & (n + 1)\text{th toss is a head} \\
    X_n(1-w) & (n + 1)\text{th toss is a tail}
\end{cases} \]
Define \( Y_{n+1} = \frac{X_{n+1}}{X_n} \), then the \( Y_i \) are independent and identically distributed.
Then \( \log Y_i \) is a sequence of independent and identically distributed random variables.
Note that \( \log X_n = \sum_{i=1}^n \log Y_i \).
\begin{lemma}
    Let \( \mu = \expect{\log Y_1}, \sigma^2 = \Var{\log Y_1} \).
    Then, if \( a > 0 \),
    \begin{enumerate}
        \item \( \prob{\abs{\frac{1}{n} \sum_{i=1}^n \log Y_i - \mu} \geq a} \leq \frac{\sigma^2}{na^2} \) by Chebyshev's inequality;
        \item \( \prob{\abs{\frac{\log X_n}{n} - \mu} \geq a} \leq \frac{\sigma^2}{na^2} \);
        \item given \( \varepsilon > 0 \) and \( \delta > 0 \), there exists \( N \) such that \( \prob{\abs{\frac{\log X_n}{n} - \mu} \geq \delta} \leq \varepsilon \) for all \( n \geq N \).
    \end{enumerate}
\end{lemma}
Consider a single coin toss, with probability \( p < 1 \) of a head.
Suppose that a bet of \( k \) on a head gives a payout of \( ku \) for some payout ratio \( u > 0 \).
Suppose further that we have an initial bankroll of 1, and we bet \( w \) on heads, retaining \( 1 - w \), for some \( 0 \leq w < 1 \).
Then, if \( Y \) is the expected fortune after the throw, \( \expect{\log Y} = p \log(1 + (u-1)w) + (1-p) \log(1-w) \).
One can show that the value of \( \expect{\log Y} \) is maximised by taking \( w = 0 \) if \( up \leq 1 \), and setting \( w = \frac{up-1}{u-1} \) if \( up > 1 \).

Let \( q = 1-p \).
If \( up > 1 \), at the optimum value of \( w \), we find
\[ \expect{\log Y} = p \log p + q \log q + \log u - q \log(u-1) = -H(p) + \log u - q \log(u-1) \]
Kelly's criterion is that in order to maximise profit, \( \expect{\log Y} \) should be optimised, given that we can bet arbitrarily many times.

One can show that if \( w \) is set below the optimum, the bankroll will still increase, but does so more slowly.
If \( w \) is set sufficiently high, the bankroll will tend to decrease.
