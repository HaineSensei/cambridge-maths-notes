To reason about communication, we use the following model.
We have a \emph{source} which knows a message, that uses an \emph{encoder} to produce some \emph{code words}.
The code words are sent through a \emph{channel}, but errors and noise may be introduced in this channel.
The code words are received by a \emph{decoder}, which performs some form of error detection and correction.
The message is finally received by a \emph{receiver}.

The source is often named \emph{Alice}, and the receiver is named \emph{Bob}.
There may be an agent watching the channel called \emph{Eve}, short for eavesdropper.

Examples of these ideas include the optical and electrical telegraph, SMS, postcodes, CDs and their error correction, compression algorithms such as \texttt{gzip}, and PINs.

Given a source and a channel, modelled probabilistically, the basic problem is to design an encoder and decoder to transmit messages \emph{economically} (noiseless coding, compression) and \emph{reliably} (noisy coding).

An example of noiseless coding is Morse code, where every letter is assigned a unique sequence of dots and dashes, where more common letters are assigned shorter strings.
Noiseless coding is adapted to the source.

Here is an example of noisy coding.
Each book has an ISBN \( a_1 a_2 \dots a_9 a_{10} \) where the \( a_1, \dots, a_9 \) are digits in \( \qty{0, \dots, 9} \), and \( a_{10} \in \qty{0, \dots, 9, X} \) such that \( 11 \mid \sum_{j=1}^{10} j a_j \).
This coding system detects the common human errors of writing an incorrect digit and transposing two adjacent digits.
Noisy coding is adapted to the channel, which in this case is the human reading the number and typing it into a computer.

\begin{definition}
    A \emph{communication channel} accepts a string of symbols from a finite alphabet \( \mathcal A = \qty{a_1, \dots, a_r} \) and outputs a string of symbols from another finite alphabet \( \mathcal B = \qty{b_1, \dots, b_s} \).
    It is modelled by the probabilities \( \prob{y_1 \dots y_n \text{ received} \mid x_1 \dots x_n \text{ sent}} \).
\end{definition}
\begin{definition}
    A \emph{discrete memoryless channel} is a channel where \( p_{ij} = \prob{b_i \text{ received} \mid a_i \text{ sent}} \) are the same for each channel use, and independent of all past and future uses of the channel.
    Its \emph{channel matrix} is the \( r \times s \) stochastic matrix \( P = (p_{ij}) \).
\end{definition}
\begin{example}
    The \emph{binary symmetric channel} with error probability \( p \in [0,1] \) is a discrete memoryless channel with input and output alphabets \( \qty{0, 1} \), where the channel matrix is
    \[ \begin{pmatrix}
            1-p & p \\
            p & 1-p
    \end{pmatrix} \]
    Here, a symbol is transmitted correctly with probability \( 1 - p \).
    Usually, we assume \( p < \frac{1}{2} \).
\end{example}
\begin{example}
    The \emph{binary erasure channel} has \( \mathcal A = \qty{0, 1} \) and \( \mathcal B = \qty{0, 1, \star} \).
    The channel matrix is
    \[ \begin{pmatrix}
        1-p & 0 & p \\
        0 & 1-p & p
    \end{pmatrix} \]
    \( p \) can be interpreted as the probability that the symbol received is unreadable.
    If \( \star \) is received, we say that we have received a \emph{splurge error}.
\end{example}
\begin{definition}
    We model \( n \) uses of a channel by the \emph{\( n \)th extension}, with input alphabet \( \mathcal A^n \) and output alphabet \( \mathcal B^n \).
    A \emph{code} \( C \) of length \( n \) is a function \( \mathcal M \to \mathcal A^n \), where \( \mathcal M \) is a set of messages.
    Implicitly, we also have a decoding rule \( \mathcal B^n \to \mathcal M \).
    \begin{itemize}
        \item The \emph{size} of this code is \( m = \abs{\mathcal M} \).
        \item The \emph{information rate} of the code is \( \rho(C) = \frac{1}{n} \log_2 m \).
        \item The \emph{error rate} of the code is \( \hat e(C) = \max_{x \in \mathcal M} \qty{\text{error} \mid x \text{ sent}} \).
    \end{itemize}
\end{definition}
\begin{definition}
    A channel can \emph{transmit reliably at rate \( R \)} if there is a sequence of codes \( (C_n)_{n=1}^\infty \) with each \( C_n \) a code of length \( n \) such that \( \lim_{n \to \infty} \rho(C_n) = R \) and \( \lim_{n \to \infty} \hat e(C_n) = 0 \).
    The \emph{capacity} of a channel is the supremum of all reliable transmission rates.
\end{definition}
It is a nontrivial fact that the capacity of the binary symmetric channel with \( p < \frac{1}{2} \) is nonzero.
This is one of Shannon's theorems, proven later.
