\subsection{Cryptosystems}
We want to modify a message such that it becomes unintelligible to an eavesdropper.
Certain secret information is shared between two participants Alice and Bob, called the \emph{key}, chosen from a set of possible keys \( \mathcal K \).
The unencrypted message is called the \emph{plaintext}, which lies in a set \( \mathcal M \), and the encrypted message is called the \emph{ciphertext}, and lies in a set \( \mathcal C \).
A \emph{cryptosystem} consists of \( (\mathcal K, \mathcal M, \mathcal C) \) together with the \emph{encryption} function \( e \colon \mathcal M \times \mathcal K \to \mathcal C \) and \emph{decryption} function \( d \colon \mathcal C \times \mathcal K \to \mathcal M \).
These maps have the property that \( d(e(m, k), k) = m \) for all \( m \in \mathcal M, k \in \mathcal K \).
\begin{example}
    Suppose \( \mathcal M = \mathcal C = \qty{A, B, \dots, Z}^\star = \Sigma^\star \).
    The \emph{simple substitution cipher} defines \( \mathcal K \) to be the set of permutations of \( \Sigma \).
    To encrypt a message, each letter of plaintext is replaced with its image under a chosen permutation \( \pi \in \mathcal K \).

    The \emph{Vigen\`ere} cipher has \( \mathcal K = \Sigma^d \) for some \( d \).
    We identify \( \Sigma \) and \( \faktor{\mathbb Z}{26\mathbb Z} \).
    Write out the key repeatedly below the plaintext, and add each plaintext letter with the corresponding key letter to produce a letter of ciphertext.
    For instance, encrypting the plaintext ATTACKATDAWN with the key LEMON gives ciphertext LXFOPVEFRNHR.
    Note, for instance, that each occurrence of the letter A in the plaintext corresponds to a letter of the key in the ciphertext.
    If \( d = 1 \), this is the \emph{Caesar cipher}.
\end{example}
