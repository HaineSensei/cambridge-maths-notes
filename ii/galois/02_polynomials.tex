\subsection{Definitions}
In this course, \emph{ring} means a commutative nonzero ring.
If \( R \) is a ring, \( R[X] \) denotes the ring of polynomials with elements \( \sum_{i=0}^n a_i X^i \), and the usual operations of addition and multiplication.
A polynomial \( f \in R[X] \) can be interpreted as a function \( f \colon R \to R \), given by \( x \mapsto \sum_{i=0}^n a_i x^i \).
It is, however, important to distinguish the polynomial and its associated function; the polynomial is not in general uniquely determined by the function.
For example, let \( R = \faktor{\mathbb Z}{p\mathbb Z} \), so for all \( a \in R \), we have \( a^p = a \), and hence \( X^p \) and \( X \) are different polynomials yet represent the same function.

Recall from Groups, Rings and Modules that if \( R = K \) is a field, \( K[X] \) is a Euclidean domain (and hence is a unique factorisation domain, a Noetherian ring, a principal ideal domain, and an integral domain).
Hence, there is a division algorithm: for polynomials \( f, g \in K[X] \), there exists a unique \( q, r \in K[X] \) such that \( f = gq + r \) and \( \deg r < \deg g \), where we denote \( \deg 0 = -\infty \).
If \( g = X - a \) is linear, \( f = (X-a)q + r \) where \( r = f(a) \in K \); this is the familiar remainder theorem.
Note that every polynomial \( f \in K[X] \) is a product of irreducible polynomials since \( K[X] \) is a unique factorisation domain, and there are greatest common divisors which can be computed using Euclid's algorithm in the usual way.

\begin{proposition}
	Let \( K \) be a field, and \( 0 \neq f \in K[X] \).
	Then, \( f \) has at most \( \deg f \) roots in \( K \).
\end{proposition}
\begin{proof}
	If \( f \) has no roots, the proof is complete.
	If \( f \) has a root \( a \), consider \( f = (X - a)q + f(0) = (X - a)q \).
	For a root \( b \) of \( f \), either \( b = a \) or \( q(b) = 0 \).
	By induction, \( q \) has at most \( \deg q \) roots, since \( \deg q < \deg f \).
	Then \( \deg q + 1 \leq \deg f \) as required.
\end{proof}

\subsection{Symmetric polynomials}
\begin{definition}
	Let \( R \) be a ring, and let \( n \geq 1 \).
	A polynomial \( f \in R[X_1, \dots, X_n] \) is \emph{symmetric} if, for every permutation \( \sigma \in S_n \), we have \( f(X_{\sigma(1)}, \dots, X_{\sigma(n)}) = f(X_1, \dots, X_n) \), where \( S_n \) is the symmetric group of degree \( n \).
\end{definition}
Note that constant polynomials are symmetric, and the property of symmetry is closed under addition and multiplication.
Hence, the set of symmetric polynomials is a subring of \( R[X_1, \dots, X_n] \).
\begin{example}
	\( X_1 + \dots + X_n \) is symmetric.
	More generally, \( p_k = X_1^k + \dots + X_n^k \) is symmetric.
\end{example}
\begin{proposition}
	Let \( f \sigma(X) = f(X_{\sigma(1)}, \dots, X_{\sigma(n)}) \).
	This gives an action (on the right) of the group \( S_n \) on \( R[X_1, \dots, X_n] \).
	A polynomial \( f \in R[X_1, \dots, X_n] \) is symmetric if and only if \( f \) is fixed under the action of \( S_n \); in other words, \( f\sigma = f \) for all \( \sigma \in S_n \).
\end{proposition}
\begin{definition}
	The \emph{elementary symmetric functions} or \emph{elementary symmetric polynomials} are
	\[ s_r(X_1, \dots, X_n) = \sum_{i_1 < \dots < i_r} X_{i_1} X_{i_2} \cdots X_{i_r} \]
	For instance,
	\[ s_2(X_1, X_2, X_3) = X_1 X_2 + X_1 X_3 + X_2 X_3 \]
	It is clear that these are symmetric polynomials.
\end{definition}
\begin{definition}
	A \emph{monomial} is an expression of the form \( X_I = X_1^{I_1} \cdots X_n^{I_n} \) for \( I \in \mathbb N^n \).
	The \emph{(total) degree} of a monomial is \( \sum_{i=1}^n I_i \).
	A \emph{term} is a scalar multiple of a monomial.
	A polynomial is uniquely characterised by a sum of terms.
	The total degree of a polynomial is the maximum total degree of its terms.

	Monomials are equipped with a \emph{lexicographic ordering}, where we say monomials \( X_I > X_J \) if either \( I_1 > J_1 \) or \( I_1 = J_1 \) and for some \( r \in \qty{1, \dots, n-1} \) we have \( I_1 = J_1, \dots, I_r = J_r, I_{r+1} > J_{r+1} \).
	This is a total order.
\end{definition}
\begin{theorem}
	Every symmetric polynomial in \( n \) variables over a ring \( R \) can be expressed as a polynomial in the \( s_r \) for \( 1 \leq r \leq n \), with coefficients in \( R \).
	Further, there are no non-trivial relations between the \( s_r \).
\end{theorem}
\begin{remark}
	Consider the ring homomorphism \( \theta \colon R[Y_1, \dots, Y_n] \to R[X_1, \dots, X_n] \) given by \( \theta(Y_r) = s_r \) and \( \theta(r) = r \) for \( r \in R \).
	The first part of the above theorem stipulates that \( \Im \theta \) is the set of symmetric polynomials.
	The second part implies that \( \theta \) is injective, since any element of \( \ker \theta \) is a polynomial between the \( s_r \) that evaluates to zero.

	Note that we can equivalently define the \( s_r \) as
	\[ \prod_{i=1}^n (T + X_i) = T^n + s_1 T^{n-1} + \dots + s_{n-1} T + s_n \]
	If we need to specify the number of variables, we use \( s_{r,n} \) instead of \( s_r \).
\end{remark}
\begin{proof}
	Let \( d \) be the total degree of a symmetric polynomial \( f \).
	Let \( X_I \) be the largest (in lexicographic order) monomial which occurs in \( f \) with coefficient \( c \).
	Since \( f \) is symmetric, any permutation of the \( X_i \) yields another monomial that occurs in \( f \).
	Hence, \( I_1 \geq I_2 \geq \dots \geq I_n \), because otherwise the rearranged monomial that satisfies this will be a strictly larger monomial in \( f \).
	We can therefore write
	\[ X_I = X_1^{I_1 - I_2}(X_1 X_2)^{I_2 - I_3} \cdots (X_1 \dots X_n)^{I_n} \]
	Consider
	\[ g = s_1^{I_1 - I_2} s_2^{I_2 - I_3} \cdots s_n^{I_n} \]
	By construction, the largest monomial in \( g \) is \( X_I \).
	Since \( g \) is symmetric, \( cg \) is symmetric.
	By induction, we may assume \( f - cg \) is expressible as a sum of symmetric polynomials as it has total degree no larger than \( d \), its leading monomial term is smaller than \( X_I \), and there are only finitely many monomials of degree at most \( d \).
	Hence \( f \) is also expressible as a sum of polynomials as required.

	Now we prove uniqueness by induction on \( n \).
	Let \( G \in R[Y_1, \dots, Y_n] \) such that \( G(s_{1,n}, \dots, s_{n,n}) = 0 \).
	We want to show that \( G \) is the zero polynomial.
	If \( n = 1 \), the result is trivial as \( s_{1,1} = X_1 \).
	If \( G = Y^k_n H \) with \( Y_n \) not dividing \( H \), then \( s_{n,n}^k H(s_{1,n}, \dots, s_{n,n}) = 0 \).
	Since \( s_{n,n} = X_1 \dots X_n \), it is not a zero divisor in \( R[X_1, \dots, X_n] \).
	Hence \( H(s_{1,n}, \dots, s_{n,n}) = 0 \).
	Without loss of generality, we can assume that \( G \) is not divisible by \( Y_n \).
	Now, replacing \( X_n \) with zero, \( s_{k,n} \) is mapped to \( s_{k,n-1} \) for \( k \neq n \), and \( s_{n,n} \) is mapped to zero.
	Hence, \( G(s_{1,n-1}, \dots, s_{n-1,n-1}, 0) = 0 \).
	By induction, \( G(Y_1, \dots, Y_{n-1}, 0) = 0 \).
	Hence \( Y_n \mid G \), contradicting our assumption.
\end{proof}
\begin{example}
	Consider, for \( n \geq 3 \),
	\[ f = \sum_{i\neq j} X_i^2 X_j \]
	The leading term is \( X_1^2 X_2 = X_1 (X_1 X_2) \), so we consider
	\begin{align*}
		f - s_1 s_2 &= \qty(\sum_{i \neq j} X_i^2 X_j) - \sum_i \sum_{j < k} X_i X_j X_k \\
		&= \qty(\sum_{i \neq j} X_i^2 X_j) - \qty(\sum_{i \neq j} X_i^2 X_j + 3 \sum_{i < j < k} X_i X_j X_k) \\
		&= -3 \sum_{i < j < k} X_i X_j X_k \\
		&= -3s_3
	\end{align*}
	Hence \( f = s_1 s_2 - 3 s_3 \).
\end{example}
Consider \( f = p_5 = \sum_i X_i^5 \).
Computing this in terms of elementary symmetric polynomials by hand is somewhat tedious, but there are various formulae, known as Newton's formulae, which can help in simplifying such expressions.
\begin{theorem}[Newton's formulae]
	Let \( n \geq 1 \).
	Then for all \( k \geq 1 \),
	\[ p_k - s_1 p_{k-1} + \dots + (-1)^{k-1} s_{k-1} p_1 + (-1)^k k s_k = 0 \]
	By convention, let \( s_0 = 1 \) and \( s_r = 0 \) if \( r > n \).
\end{theorem}
\begin{proof}
	It suffices to consider \( R = \mathbb Z \) (or, for example, \( R = \mathbb R \)).
	Consider the generating function
	\[ F(T) = \prod_{i=1}^n (1 - X_i T) = \sum_{r=0}^n (-1)^r s_r T^r \]
	Note that for polynomials \( f(x), g(x) \), their formal derivatives satisfy
	\[ \frac{\dv{T} (fg)}{fg} = \frac{f'g + fg'}{fg} = \frac{f'}{f} + \frac{g'}{g} \]
	Then, taking the logarithmic derivative with respect to \( T \),
	\begin{align*}
		\frac{F'(T)}{F(T)} &= \frac{\dv{T} \prod_{i=1}^n (1 - X_i T)}{\prod_{i=1}^n (1 - X_i T)} \\
		&= \sum_{i=1}^n \frac{\dv{T} (1 - X_i T)}{1 - X_i T} \\
		&= -\sum_{i=1}^n \frac{X_i}{1 - X_i T} \\
		&= \frac{-1}{T} \sum_{i=1}^n \sum_{r=1}^\infty X_i^r T^r \\
		&= \frac{-1}{T} \sum_{r=1}^\infty p_r T^r
	\end{align*}
	Hence,
	\[ -TF'(T) = s_1 T - 2s_2 T^2 + \dots + (-1)^{n-1} n s_n T^n \]
	but also
	\[ -TF'(T) = F(T) \sum_{r=1}^\infty p_r T_r = \qty(s_0 - s_1 T + \dots + (-1)^n s_n T^n) \qty(p_1 T + p_2 T^2 + \dots) \]
	Equating the coefficients of powers of \( T \), we find the identity as required by the theorem.
\end{proof}
\begin{example}
	The \emph{discriminant polynomial} is
	\[ D(X_1, \dots, X_n) = \Delta(X_1, \dots, X_n)^2 \]
	where
	\[ \Delta(X_1, \dots, X_n) = \prod_{i < j} (X_i - X_j) \]
	This is used in defining the sign of a permutation: applying a permutation \( \sigma \) to \( \Delta \) multiplies \( \Delta \) by the sign of \( \sigma \).
	Hence \( D \) is symmetric.
	Therefore, \( D \) can be written in terms of the symmetric polynomials.
	\[ D(X_1, \dots, X_n) = d(s_1, \dots, s_n) \]
	where \( d \) has integer coefficients.
	For example, \( n = 2 \) gives \( D = (X_1 - X_2)^2 = s_1^2 - 4s_2 \).
\end{example}
\begin{definition}
	Let \( f = T^n + \sum_{i=0}^{n-1} a_{n-1} T^i \) be a monic polynomial in \( R[T] \).
	Its discriminant is
	\[ \mathrm{Disc}(f) = d(-a_1, a_2, -a_3, \dots, (-1)^n a_n) \in R \]
\end{definition}
Observe that if \( f \) is a product of linear polynomials \( f = \prod_{i=1}^n (T - x_i) \), then \( a_r = (-1)^r s_r(x_1, \dots, x_n) \), giving \( \mathrm{Disc}(f) = \prod_{i < j} (x_i - x_j)^2 = D(x_1, \dots, x_n) \).
In particular, if \( R = K \) is a field, \( \mathrm{Disc}(f) = 0 \) if and only if \( f \) has a repeated root.
For example, \( \mathrm{Disc}(T^2 + bT + c) = b^2 - 4c \).
