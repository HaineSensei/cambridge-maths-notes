\subsection{Fundamental theorem of algebra}
This subsection is non-examinable.
We show that \( \mathbb C \) is algebraically closed over \( \mathbb Q \), without using complex analysis.
We will only use the following facts:
\begin{enumerate}
	\item every polynomial of odd degree over \( \mathbb R \) has a root, due to the intermediate value theorem;
	\item every quadratic over \( \mathbb C \) splits into linear factors, so we can take square roots;
	\item every finite group \( G \) has a subgroup \( H \) such that \( (G:H) \) is odd and \( \abs{H} \) is a power of 2, by Sylow's theorem for \( p = 2 \);
	\item if \( G \) is a \( p \)-group, so \( \abs{G} = p^k \) and \( k > 0 \), then \( G \) has a subgroup of index \( p \), since \( G \) has a non-trivial centre.
\end{enumerate}
Let \( K / \mathbb C \) be a finite extension.
Let \( L / K \) be a normal closure of \( K \) over \( \mathbb R \), so \( L \) is a Galois extension of \( \mathbb R \) containing \( \mathbb C \).
Let \( G = \Gal(L/\mathbb R) \).
We will show that \( L = \mathbb C \).

Let \( H \leq G \) be a Sylow 2-subgroup, and consider \( L^H \).
% TODO: diagram
We have \( [L^H:\mathbb R] = (G:H) \), which is odd.
So if \( x \in L^H \), by (i), its minimal polynomial is linear over \( \mathbb R \), so \( x \in \mathbb R \).
Hence \( L^H = \mathbb R \), so \( H = G \).
So \( G \) is a 2-group.

Let \( G \supset G_1 = \Gal(L/\mathbb C) \), and \( G_2 \leq G_1 \) be a subgroup of index 2, which exists by (iv).
Then \( [L^{G_2}:\mathbb C] = (G_1 : G_2) \), contradicting the fact (ii) that quadratics split in \( \mathbb C \).
So there cannot exist a subgroup of index 2, so \( G_1 = \qty{e} \), and \( L = \mathbb C \).

% TODO: reorganise
\subsection{Artin's theorem on invariants}
\begin{theorem}[Artin]
	Let \( L \) be a field and \( G \leq \Aut(L) \) be a finite subgroup of automorphisms of \( L \).
	Define \( L^G = \qty{x \in L \mid \forall \sigma \in G,\, \sigma(x) = x} \).
	Then \( L / L^G \) is finite, and satisfies \( [L:L^G] = \abs{G} \).
\end{theorem}
\begin{remark}
	Unlike in the Galois correspondence, this theorem does not rely on a field extension, just a single field and a finite group of automorphisms.
	In particular, we find that \( L / L^G \) is finite and Galois, with Galois group \( G \).
\end{remark}
\begin{proof}
	It suffices to show \( L/L^G \) is finite, because then we can apply the Galois correspondence to show \( [L:L^G] = \abs{G} \).
	Let \( K = L^G \), and let \( x \in L \).
	Then if \( \qty{\sigma_1(x), \dots, \sigma_r(x)} \) is the orbit of \( G \) on \( x \), then \( x \) is a root of \( f = \prod_{i=1}^r (T - \sigma_i(x)) \).
	But \( f \in L^G[T] = K[T] \).
	By construction, \( f \) is separable.
	Hence \( x \) is algebraic and separable over \( K \), and \( \deg_K x \leq \abs{G} \).

	Let \( y \in L \) have maximal degree.
	We claim that \( K(y) = L \).
	If not, there exists \( x \in L \setminus K(y) \).
	By above, \( x, y \) are algebraic and separable over \( K \).
	By the primitive element theorem, there exists \( z \in L \) such that \( K(x,y) = K(z) \supsetneq K(y) \), so \( \deg_K z > \deg_K y \).
	But \( y \) was chosen to have maximal degree, so this is a contradiction.
\end{proof}
\begin{remark}
	One can prove this theorem directly without appealing to the Galois correspondence or the primitive element theorem.
	This can then be used as a starting point for Galois theory, which then allows the more complicated theorems to be proven.

	There are two common ways to construct finite Galois extensions.
	The first, studied earlier in the course, involves taking the splitting field of a separable polynomial; this method constructs a larger field from a given base field.
	Artin's theorem provides another way to construct such extensions, by fixing a large field \( L \) and constructing the subfield \( L^G \).
\end{remark}
\begin{example}
	Let \( \symbb k \) be a field, and let \( L = \symbb k(X_1, \dots, X_n) \) be the field of rational functions, defined as the fractions of the polynomial ring \( \symbb k[X_1, \dots, X_n] \).
	Let \( G = S_n \) be the symmetric group permuting the \( X_i \).
	Then \( G \leq \Aut(L) \).
\end{example}
\begin{theorem}
	Let \( \symbb k \) be a field and let \( L = \symbb k(X_1, \dots, X_n) \).
	Then \( L^G = \symbb k(s_1, \dots, s_n) \).
\end{theorem}
\begin{proof}
	Recall that \( \symbb k[X_1, \dots, X_n]^G = \symbb k[s_1, \dots, s_n] \) where the \( s_i \) are the elementary symmetric polynomials in the \( X_i \), and there are no nontrivial relations between the \( s_i \).
	In particular, \( \symbb k(s_1, \dots, s_n) \subseteq L^G \).

	Conversely, let \( \frac{f}{g} \in L^G \) for \( f, g \in \symbb k[X_1, \dots, X_n] = R \).
	Without loss of generality let \( f, g \) be coprime.
	Then for all \( \sigma \in G \), \( \frac{f}{g} = \frac{\sigma f}{\sigma g} \).
	By Gauss' lemma, \( R \) is a unique factorisation domain, and the units in \( R \) are the nonzero constants \( \symbb k^\times \).
	Hence \( \sigma f = c_\sigma f \) and \( \sigma g = c_\sigma g \) where \( c_\sigma \in \symbb k^\times \).

	Since \( G \) is finite and has order \( N = n! \), \( f = \sigma^N f = c_\sigma^N f \).
	So \( c_\sigma \) is an \( N \)th root of unity.
	Then \( fg^{N-1}, g^N \) are invariant under \( \sigma \), so \( fg^{N-1}, g^N \in R^G = \symbb k[s_1, \dots, s_n] \).
	So \( \frac{f}{g} = \frac{fg^{N-1}}{g^N} \in \symbb k(s_1, \dots, s_n) \).
\end{proof}
\begin{example}
	Let \( L = \symbb k(X_1, \dots, X_n) \), and let \( K = \symbb k(s_1, \dots, s_n) = L^G \) where \( G = S^n \).
	Then by Artin's theorem, \( L/K \) is a finite Galois extension with Galois group \( G \).
	Let \( f = T^n - s_1 T^{n-1} + \dots + (-1)^n s_n \in K[T] \).
	Then in \( L \), \( f = \prod_{i=1}^n (T - X_i) \).
	Since the \( X_i \) are different, \( f \) is separable, and \( L / K \) is a splitting field for \( f \).
	Hence \( \Gal(f/K) = S^n \).
	Informally, the general polynomial of degree \( n \) has Galois group \( S^n \).
	It is not difficult to show that for any finite group \( G \), there exists a Galois extension with Galois group isomorphic to \( G \).
\end{example}

\subsection{Other areas of study}
This is one of a number of theories in \emph{invariant theory}, in which one considers a ring \( R \) and a group \( G \leq \Aut(R) \), and study \( R^G \).
If \( R \) is a polynomial ring \( \symbb k[X_1, \dots, X_n] \) and \( G \leq S_n \), then knowing \( R^G \) can help with the computation of Galois groups algorithmically.
For example, if \( G = A_n \), then \( \symbb k[X_1, \dots, X_n]^{A_n} = \symbb k[s_1, \dots, s_n, \Delta] \) where \( \Delta = \prod_{i < j} (X_i - X_j) \), for \( \fchar \symbb k \neq 2 \).

Now consider \( R = \symbb k[X_1, X_2] \) and \( G = \qty{1, \sigma} \) where \( \sigma(X_i) = -X_i \).
Let \( \fchar \symbb k \neq 2 \).
Then one can show \( R^G = \symbb k[X_1^2, X_2^2, X_1 X_2] = \faktor{\symbb k[Y_1, Y_2, Y_3]}{(Y_1 Y_2 - Y_3^2)} \).
Geometrically, \( \qty{Y_1 Y_2 = Y_3^2} \subset \mathbb R^3 \) is a double cone.
The point at which the cones meet is known as a singularity; such singularities occur in the study of algebraic geometry.

If \( K \) and \( G \) are fixed, it is not always the case that there exists a Galois extension \( L / K \) such that \( \Gal(L/K) = G \).
For instance, if \( K \) is algebraically closed, it has no nontrivial Galois extensions.
If \( K = \mathbb F_p \), then \( \Gal(L/K) \) must be cyclic.

The \emph{inverse Galois problem} asks whether every finite group \( G \) is the Galois group of some Galois extension \( L / \mathbb Q \).
This is unsolved in the general case.
On the extra example sheet, one shows that every abelian group is in fact the Galois group of some Galois extension \( L / \mathbb Q \).
There is a famous theorem by Shafarevich that every finite soluble group is such a Galois group over \( \mathbb Q \).
This is also known to hold for most finite simple groups; in particular, due to a theorem of John Thompson, the monster group is known to be a Galois group over \( \mathbb Q \).

Perhaps to solve this problem, it would be better to instead understand \( \Gal(\overline{\mathbb Q}/\mathbb Q) \).
The inverse Galois problem is equivalent to asking whether every finite group is a quotient of \( \Gal(\overline{\mathbb Q}/\mathbb Q) \).
We may also be interested in finding the representations of this group.
This leads to the Langlands programme.
