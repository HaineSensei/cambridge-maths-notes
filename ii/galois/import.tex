\chapter[Galois Theory \\ \textnormal{\emph{Lectured in Michaelmas \oldstylenums{2022} by \textsc{Prof.\ A.\ J.\ Scholl}}}]{Galois Theory}
\emph{\Large Lectured in Michaelmas \oldstylenums{2022} by \textsc{Prof.\ A.\ J.\ Scholl}}

Suppose \( K \) and \( L \) are fields, and \( K \subseteq L \).
We can view \( L \) as a vector space over \( K \), and therefore analyse things like its dimension.
We study how these extensions of fields interact, and how they can embed inside each other.

To analyse these field extensions, we will understand the Galois group associated to a particular field extension \( K \subseteq L \).
This group describes the different ways that \( L \) can embed into itself, while preserving the structure of \( K \).
This turns out to provide a measure of the complexity of \( L \) with respect to \( K \).

The central result of the course is the Galois correspondence.
If \( K \subseteq L \), there may be other fields lying between \( K \) and \( L \).
We prove that the subgroups of the Galois group correspond exactly to these intermediate fields.

We apply Galois theory to some problems that had been unsolved for many centuries or millenia.
Classic examples include doubling the cube and trisecting the angle.
We also prove that there is no formula for finding a root of the general quintic.

\subfile{../../ii/galois/main.tex}
