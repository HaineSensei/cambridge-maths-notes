\subsection{Construction of finite fields}
Every finite field has characteristic \( p > 0 \), and so it can be regarded as a field extension of \( \mathbb F_p \).
We will classify every finite field and study their Galois theory.
Recall that, for a finite field \( F \) of characteristic \( p \),
\begin{enumerate}
	\item \( \abs{F} = p^n \), where \( [F : \mathbb F_p] = n \);
	\item \( F^\times \) is cyclic, of order \( p^n - 1 \);
	\item The Frobenius automorphism \( \varphi_p \colon F \to F \) given by \( x \mapsto x^p \) is an automorphism of \( F \).
\end{enumerate}
\begin{theorem}
	Let \( p \) be a prime, and \( n \geq 1 \).
	Then there is a finite field with \( q = p^n \) elements.
	Any such field is a splitting field of the polynomial \( f = T^q - T \) over \( \mathbb F_p \).
	Since splitting fields are unique up to \( \mathbb F_p \)-isomorphism, any two finite fields of the same order are isomorphic.
\end{theorem}
\begin{proof}
	Let \( F \) be a field with \( q = p^n \) elements.
	Then if \( x \in F^\times \), \( x^{q-1} = 1 \).
	Hence, for all \( x \in F \), \( x^q = x \).
	In particular, \( F \) has \( q \) distinct roots in \( F \), which are all of the elements of \( F \).
	So \( f \) splits into linear factors in \( F \), and not in any proper subfield, so \( F \) is indeed a splitting field for \( f \) as required.

	Now, we wish to explicitly construct such a field.
	Let \( L \) be a splitting field for \( f = T^q - T \) over \( \mathbb F_p \).
	Let \( F \subseteq L \) be the fixed field of \( \varphi_p^n \), the map \( x \mapsto x^q \).
	So \( F \) is the set of roots of \( f \) in \( L \).
	So \( \abs{F} = q \).
	Therefore, \( L = F \) because \( F \) has \( q \) elements, using the above argument.
\end{proof}
Now that we have shown isomorphism, we simply write \( \mathbb F_q \) for any finite field of \( q \) elements.
There is no canonical finite field of a given order in general.

\subsection{Galois theory of finite fields}
\begin{theorem}
	The extension \( \mathbb F_{p^n} / \mathbb F_p \) is Galois, and the Galois group is cyclic of order \( n \), generated by the Frobenius automorphism \( \varphi_p \).
\end{theorem}
\begin{proof}
	Since \( \mathbb F_{p^n} \) is the splitting field of the separable polynomial \( T^{p^n} - T \), the extension is Galois.
	Let \( G \leq \Gal(\mathbb F_{p^n} / \mathbb F_p) \) be the subgroup generated by \( \varphi_p \).
	Then \( \mathbb F_{p^n}^G = \qty{x \mid x^p = x} = \mathbb F^p \), so by the Galois correspondence, \( G \) must be the entire group \( \Gal(\mathbb F_{p^n} / \mathbb F_p) \).
\end{proof}
\begin{theorem}
	\( \mathbb F_{p^n} \) has a unique subfield of order \( p^m \) for all \( m \mid n \), and no others.
	If \( m \mid n \), then \( \mathbb F_{p^m} \subseteq \mathbb F^{p^n} \) is the fixed field of \( \varphi_p^m \).
\end{theorem}
\begin{proof}
	By the Galois correspondence, it suffices to check the subgroups of \( \faktor{\mathbb Z}{n\mathbb Z} \).
	The subgroups of \( \faktor{\mathbb Z}{n\mathbb Z} \) are \( \faktor{m\mathbb Z}{n\mathbb Z} \) for \( m \mid n \).
	Hence, the subfields of \( \mathbb F_{p^m} \) are the fixed fields of the subgroups \( \genset{\varphi_p^m} \), which have degree equal to the indices \( \qty(\faktor{\mathbb Z}{n\mathbb Z} : \faktor{m\mathbb Z}{n\mathbb Z}) = m \).
\end{proof}
\begin{remark}
	If \( m \mid n \), \( \Gal(\mathbb F_{p^n} / \mathbb F_{p^m}) = \genset{\varphi_p^n} \), which has order \( \frac{n}{m} \).
\end{remark}
\begin{theorem}
	Let \( f \in \mathbb F_p[T] \) be separable, and let \( n = \deg f \).
	Suppose the irreducible factors of \( f \) have degrees \( n_1, \dots, n_r \), so \( \sum_{i=1}^r n_i = n \).
	Then \( \Gal(f/\mathbb F_p) \subseteq S_n \) is cyclic and generated by an element of cycle type \( (n_1, \dots, n_r) \).
	In particular, \( \abs{\Gal(f/\mathbb F_p)} \) is the least common multiple of the \( n_i \).
\end{theorem}
Recall that \( \pi \in S_n \) has cycle type \( (n_1, \dots, n_r) \) if it is a product of \( r \) disjoint cycles \( \pi_i \), each with length \( n_i \).
\begin{proof}
	Let \( L \) be a splitting field for \( f \) over \( \mathbb F_p \).
	Consider \( x_1, \dots, x_n \in L \).
	Then \( \Gal(L/\mathbb F_p) \) is cyclic and generated by \( \varphi_p \).
	As the irreducible factors \( g_i \) of \( f \) are the minimal polynomial of the \( x_i \), and the roots of the minimal polynomial of \( x_i \) is the orbit of \( \varphi_p \) on \( x_i \), the cycle type must be as required.
	The order of any such permutation is the lowest common multiple of the lengths of the cycles.
\end{proof}
