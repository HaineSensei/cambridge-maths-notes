\subsection{Field automorphisms}
\begin{definition}
	A bijective homomorphism from a field to itself is called an \emph{automorphism}.
	The set of automorphisms of a field \( L \) forms a group \( \Aut(L) \) under composition: \( (\sigma\tau)(x) = \sigma(\tau(x)) \).
	This is called the \emph{automorphism group of \( L \)}.
	Let \( S \subseteq \Aut(L) \).
	Then, we define
	\[ L^S = \qty{x \in L \mid \forall \sigma \in S, \sigma(x) = x} \]
	This is a subfield of \( L \), known as the \emph{fixed field of \( S \)}, since each \( \sigma \) is a homomorphism.
\end{definition}
\begin{example}
	Let \( L = \mathbb C \) and \( \sigma \) be the complex conjugation automorphism.
	Then the fixed field of \( \qty{\sigma} \) is \( \mathbb C^{\qty{\sigma}} = \mathbb R \).
\end{example}
\begin{definition}
	Let \( L / K \) be a field extension.
	We define \( \Aut(L/K) \) to be the set of \( K \)-automorphisms of \( L \), so \( \Aut(L/K) = \qty{\sigma \in \Aut(L) \mid \forall x \in K, \sigma(x) = x} \).
	Equivalently, \( \sigma \in \Aut(L) \) is an element of \( \Aut(L/K) \) if \( K \subseteq L^{\qty{\sigma}} \).
	\( \Aut(L/K) \) is a subgroup of \( \Aut(L) \).
\end{definition}
\begin{theorem}
	Let \( L / K \) be a finite extension.
	Then \( \abs{\Aut(L/K)} \leq [L:K] \).
\end{theorem}
\begin{proof}
	Let \( M = L \), then \( \Hom_K(L,M) = \Aut(L/K) \), which has at most \( [L:K] \) elements.
\end{proof}
\begin{proposition}
	If \( K = \mathbb Q \) or \( K = \mathbb F_q \), \( \Aut(K) = \qty{1} \).
\end{proposition}
\begin{proof}
	\( \sigma(1_K) = 1_K \) hence \( \sigma(n_K) = n_K \).
\end{proof}
In particular, \( \Aut(L) = \Aut(L/K) \) where \( K \) is the prime subfield of \( L \).

\subsection{Galois extensions}
We need to define a notion of when an extension \( L / K \) has `many symmetries'.
\begin{definition}
	An extension \( L / K \) is a \emph{Galois extension} if it is algebraic, and \( L^{\Aut(L/K)} = K \).
\end{definition}
\begin{remark}
	If \( x \in L \setminus K \), there is a \( K \)-automorphism \( \sigma : L \to L \) such that \( x \neq \sigma(x) \).
\end{remark}
\begin{example}
	\( \mathbb C / \mathbb R \) is a Galois extension, since the fixed field of complex conjugation is \( \mathbb R \).
	Similarly, \( \mathbb Q(i) / \mathbb Q \) is a Galois extension.
\end{example}
\begin{example}
	Let \( K / \mathbb F_p \) be a finite extension, so \( K \) is a finite field.
	The Frobenius automorphism of \( K \), given by \( \varphi_p(x) = x^p \), has fixed field
	\[ K^{\qty{\varphi_p}} = \qty{x \in K \mid x \text{ a root of } T^p - T} \]
	But since this has at most \( p \) roots, and each element of \( \mathbb F_p \) is a root, the fixed field is exactly \( \mathbb F_p \).
	So \( K^{\Aut(K/\mathbb F_p)} = \mathbb F_p \), so this is a Galois extension.
\end{example}
\begin{definition}
	Let \( L / K \) be a Galois extension.
	We say \( \Gal(L/K) \) for the automorphism group \( \Aut(L/K) \), called the \emph{Galois group of \( L / K \)}.
\end{definition}
\begin{theorem}[classification of finite Galois extensions]
	Let \( L / K \) be a finite extension, and let \( G = \Aut(L/K) \), then the following are equivalent.
	\begin{enumerate}[(i)]
		\item \( L / K \) is a Galois extension, so \( K = L^G \).
		\item \( L / K \) is normal and separable.
		\item \( L \) is a splitting field of a separable polynomial in \( K \).
		\item \( \abs{\Aut(L/K)} = [L : K] \).
	\end{enumerate}
	If this holds, the minimal polynomial of any \( x \in L \) over \( K \) is \( m_{x,K} = \prod_{i=1}^r (T - x_i) \), where \( \qty{x_1, \dots, x_r} \) is the orbit of \( G \) on \( x \).
\end{theorem}
\begin{proof}
	\emph{(i) implies (ii) and the minimal polynomial result.}
	Let \( x \in L \), and \( \qty{x_1, \dots, x_r} \) be the orbit of \( G \) on \( x \).
	Let \( f = \prod_{i=1}^r (T - x_i) \).
	Then \( f(x) = 0 \).
	Since \( G \) permutes the \( x_i \), the coefficients of \( f \) are fixed by \( G \).
	By assumption, the coefficients of \( f \) lie in \( K \), so the minimal polynomial of \( x \) must divide \( f \).
	Since \( m_{x,K}(\sigma(x)) = \sigma(m_{x,K}(x)) = 0 \), so every \( x_i \) is a root of the minimal polynomial of \( m_{x,K} \).
	So \( f \) is exactly the minimal polynomial as required.
	\( m_{x,K} \) is a separable polynomial and splits in \( L \).
	So \( L / K \) is normal and separable.

	\emph{(ii) implies (iii).}
	Since splitting fields are normal extensions, \( L \) is a splitting field for some polynomial \( f \in K[T] \).
	Write \( f = \prod_{i=1}^r q_i^{e_i} \) where the \( q_i \) are distinct irreducible polynomials, and \( e_i \geq 1 \).
	Since \( L \) and \( K \) are separable, the \( q_i \) are separable as they are irreducible, so \( g = \prod_{i=1}^r q_i \) is separable and \( L \) is also a splitting field for \( g \).

	\emph{(iii) implies (iv).}
	Let \( L = K(x_1, \dots, x_k) \) be the splitting field of a separable polynomial \( f \in K[T] \) with roots \( x_i \).
	By the theorem on counting embeddings with \( M = L \), since \( m_{x_i,K} \mid f \), conditions (i) and (ii) in the theorem are satisfied, and we find \( \abs{\Aut(L/K)} = \abs{\Hom_K(L,M)} = [L:K] \).

	\emph{(iv) implies (i).}
	Suppose \( \abs{\Aut(L/K)} = \abs{G} = [L : K] \).
	Note that \( G \subseteq \Aut(L/L^G) \subseteq \Aut(L/K) \), so these inclusions are both equalities.
	So \( G = \Aut(L/L^G) \), so \( [L : K] = \abs{G} \leq [L : L^G] \).
	But since \( L^G \supseteq K \), we must have equality by the tower law.
\end{proof}
\begin{corollary}
	Let \( L / K \) be a finite Galois extension.
	Then \( L = K(x) \) for some \( x \in L \) which is separable over \( K \), and has degree \( [L : K] \).
\end{corollary}
\begin{proof}
	By (ii) above, \( L / K \) is separable.
	Then the primitive element theorem implies that \( L = K(x) \) for some \( x \).
\end{proof}

\subsection{Galois correspondence}
\begin{theorem}[Galois correspondence: part (a)]
	Let \( L / K \) be a finite Galois extension with \( G = \Gal(L / K) \).
	Suppose \( F \) is another field, and \( K \subseteq F \subseteq L \).
	Then \( L / F \) is also a Galois extension where \( \Gal(L / F) \leq \Gal(L / K) \).
	The map \( F \mapsto \Gal(L / F) \) is a bijection between the set of intermediate fields \( F \) and the set of subgroups of \( H \leq \Gal(L/K) \).
	The inverse of this map is \( H \mapsto L^H \).
	This bijection reverses inclusions, and if \( F = L^H \), we have \( [F : K] = (G : H) \).
\end{theorem}
\begin{proof}
	Let \( x \in L \).
	Then \( m_{x,F} \mid m_{x,K} \) in \( F[T] \).
	As \( m_{x,K} \) splits into distinct linear factors in \( L \) so does \( m_{x,F} \).
	So \( L / F \) is normal and separable, and hence a Galois extension as required.
	By definition, \( \Gal(L / F) \leq \Gal(L / K) \).

	To check the map \( F \mapsto \Gal(L / F) \) is a bijection with the given inverse, we first consider a field \( F \), and its image \( L^{\Gal(L / F)} \) under both maps.
	We have \( L^{\Gal(L / F)} = F \), since \( L / F \) is Galois as required.
	Conversely, suppose \( H \leq \Gal(L / F) \), and consider its image \( \Gal(L / L^H) \).
	To show \( \Gal(L / L^H) = H \), it suffices to show that \( [L : L^H] \leq \abs{H} \), because certainly \( H \leq \Gal(L / L^H) \) and \( \abs{\Gal(L / L^H)} \leq [L : L^H] \).
	By the previous corollary, \( L = L^H(x) \) for some \( x \), and \( f = \prod_{\sigma \in H} (T - \sigma(x)) \in L^H[T] \) is a polynomial with \( x \) as a root.
	In particular, \( [L : L^H] = \deg_{L^H}(x) \leq \deg f = \abs{H} \).
	So we have a bijection as claimed.

	Suppose \( F \subseteq F' \) are fields between \( K \) and \( L \).
	Then \( \Gal(L/F') \subseteq \Gal(L/F) \), so the bijection reverses inclusions.
	Finally, if \( F = L^H \), we have \( [F : K] = \frac{[L : K]}{[L : F]} = \frac{\abs{\Gal(L / K)}}{\abs{\Gal(L / F)}} = \frac{\abs{G}}{\abs{H}} = (G : H) \).
\end{proof}
\begin{theorem}[Galois correspondence: part (b)]
	Let \( H \leq G \) be a subgroup of a Galois group \( G = \Gal(L / K) \).
	Then \( \sigma H \sigma^{-1} \) corresponds to the field \( \sigma L^H \).
\end{theorem}
\begin{proof}
	Under the Galois correspondence, \( \sigma H \sigma^{-1} \) corresponds to its fixed field
	\[ L^{\sigma H \sigma^{-1}} = \qty{x \in L \mid \sigma \tau \sigma^{-1}(x) = x \text{ for all } \tau \in H} \]
	Note that \( \sigma \tau \sigma^{-1}(x) = x \) if and only if \( \tau \sigma^{-1}(x) = \tau \sigma^{-1}(x) \), so \( \tau(y) = y \) for \( x = \sigma(y) \).
	Hence \( x \in L^{\sigma H \sigma^{-1}} \) if and only if there exists \( y \in L^H \), \( x = \sigma(y) \).
	Therefore \( L^{\sigma H \sigma^{-1}} = \sigma L^H \) as required.
\end{proof}
\begin{theorem}[Galois correspondence: part (c)]
	Let \( H \leq G = \Gal(L/K) \).
	Then the following are equivalent.
	\begin{enumerate}
		\item \( L^H / K \) is Galois;
		\item \( L^H / K \) is normal;
		\item for all \( \sigma \in G \), \( \sigma(L^H) = L^H \);
		\item \( H \) is a normal subgroup of \( G \).
	\end{enumerate}
	If so, \( \Gal(L^H / K) = \faktor{\Gal(L / K)}{H} = \faktor{G}{H} \).
\end{theorem}
\begin{proof}
	\emph{(i) and (ii) are equivalent.}
	\( L / K \) is separable since it is Galois.
	So \( L^H / K \) is also separable.

	\emph{(iii) and (iv) are equivalent.}
	Let \( F = L^H \), and let \( x \in F \).
	Then the set of roots of \( m_{x,K} \) is the orbit of \( x \) under \( G \), so the minimal polynomial splits in \( F \) if and only if for all \( \sigma \in G \), \( \sigma(x) \in F \).
	As this holds for all \( x \in F \), \( F \) is normal if and only if \( \sigma F \subseteq F \).
	Since \( [\sigma F : K] = [F : K] \), as \( F \) and \( \sigma F \) are \( K \)-isomorphic, this holds if and only if \( \sigma F = F \).
	By part (b) of the Galois correspondence, this is equivalent to the statement that \( \sigma H \sigma^{-1} = H \) for all \( \sigma \), so \( H \) is normal.

	If any of the above hold, for all \( \sigma \in G \), we have \( \sigma F = F \), so we have homomorphisms \( G \to \Gal(F/K) \) given by the restriction of \( \sigma \in G \) to \( F \).
	Its kernel is \( H \).
	Then from the isomorphism theorem, \( \faktor{G}{H} \) is isomorphic to a subgroup of \( \Gal(F/K) \).
	This must be an isomorphism because \( [F : K] = (G : H) \).
\end{proof}
\begin{example}
	Let \( K = \mathbb Q \) and \( L = \mathbb Q(\sqrt[3]{2}, \omega) \) where \( \omega = e^{\frac{2\pi i}{3}} \).
	\( L \) is a splitting field for \( T^3 - 2 \) with \( [L : \mathbb Q] = 6 \).
	Since \( T^3 - 2 \) is a separable polynomial, \( L \) is the splitting field of a separable polynomial and hence Galois.
	Therefore \( G = \Gal(L/\mathbb Q) \) has order 6.

	We have the subfields \( F_1 = \mathbb Q(\omega) \), \( F_2 = \mathbb Q(\sqrt[3]{2}) \), where \( [F_1 : \mathbb Q] = 2 \) and \( [F_2 : \mathbb Q] = 3 \).
	In the following diagram, the arrows on the left hand side are annotated with the degrees of an extensions, and the arrows on the right hand side are labelled with the index of the relevant subgroup.
	\begin{center}
		\begin{tikzcd}
			& L                                                              &                      &  &                      & \qty{1} \arrow[ld, "3"'] \arrow[rd, "2"] \arrow[dd, "6" description] &                     \\
		F_1 \arrow[ru, "3"] &                                                                & F_2 \arrow[lu, "2"'] &  & H_1 \arrow[rd, "2"'] &                                                                      & H_2 \arrow[ld, "3"] \\
					& K \arrow[lu, "2"] \arrow[ru, "3"'] \arrow[uu, "6" description] &                      &  &                      & G                                                                    &
		\end{tikzcd}
	\end{center}
	By the classification of finite groups of order 6, \( G \) is isomorphic either to \( C_6 \) or \( S_3 \).
	\( F_2 = \mathbb Q(\sqrt[3]{2}) \) is not a normal extension of \( \mathbb Q \), because \( \omega\sqrt[3]{2} \not\in F_2 \).
	So \( H_2 \) is not a normal subgroup of \( G \).
	Since all subgroups of abelian groups are normal, \( G \) is not abelian.
	So \( G \cong S_3 \).
	Hence \( H_1 \cong A_3 \), and \( H_2 \) is a transposition, but since all subgroups generated by transpositions are conjugate, we can set \( H_2 = \genset{(1\ 2)} \).

	The other two subgroups are conjugate to \( H_2 \), corresponding to the subfields \( \sigma F_2 \) where \( \sigma \in G \).
	Hence, these subfields are exactly \( \mathbb Q(\omega\sqrt[3]{2}) \) and \( \mathbb Q(\omega^2\sqrt[3]{2}) \), since the conjugates of \( \sqrt[3]{2} \) are exactly the roots of the minimal polynomial.
	Note that since these are the only subgroups, we have found all intermediate fields between \( \mathbb Q \) and \( \mathbb Q(\sqrt[3]{2},\omega) \).
\end{example}
There is an easier way to prove \( G \cong S_3 \).
Consider a separable polynomial \( f \in K[T] \), and its roots \( x_1, \dots, x_n \) in a splitting field \( L \).
Then \( G = \Gal(L / K) \) permutes the \( \qty{x_i} \), because \( f(\sigma x_i) = \sigma f(x_i) = 0 \).
If \( \sigma(x_i) = x_i \) for all \( i \), since \( L = K(x_1, \dots, x_n) \), \( \sigma \) must be the identity map.
This gives an injective homomorphism from \( G \) into \( S_n \).
So \( G \) is isomorphic to a subgroup of \( S_n \).
In our example above, \( \abs{G} = 6 \) and \( G \) is isomorphic to a subgroup of \( S_n \), so \( G \cong S_n \).
\begin{definition}
	The subgroup \( \Gal(f/K) \subseteq S_n \), given by the image of \( \Gal(L/K) \), is the \emph{Galois group of \( f \) over \( K \)}.
\end{definition}
\begin{remark}
	Note that \( [L : K] = \abs{\Gal(L/K)} = \abs{\Gal(f/K)} \mid n! \).
\end{remark}

\subsection{Finding the Galois group}
There exist several methods for finding the Galois group for a particular field extension.
\begin{proposition}
	\( f \in K[T] \) is irreducible if and only if \( \Gal(f/K) \) is \emph{transitive}, so for all \( i, j \in \qty{1, \dots, n} \), there exists \( \sigma \in \Gal(f/K) \) such that \( \sigma(i) = j \).
\end{proposition}
\begin{remark}
	A subgroup of \( S_n \) is transitive if and only if there is exactly one orbit.
\end{remark}
\begin{proof}
	Let \( x \) be a root of \( f \) in a splitting field \( L \).
	Then its orbit under \( G = \Gal(f/K) \) is exactly the set of roots of \( m_{x,K} \).
	Since \( m_{x,K} \) is an irreducible factor of \( f \), \( m_{x,K} = f \) if and only if \( f \) is irreducible.
	Conversely, \( m_{x,K} = f \) if and only if each root of \( f \) is in the orbit of \( x \), which is exactly the statement that \( G \) acts transitively on the roots of \( f \).
\end{proof}
\begin{remark}
	If \( G \subseteq S_n \) is transitive, by the orbit-stabiliser theorem, \( n \mid \abs{G} \).
\end{remark}
Recall that for a monic polynomial \( f = \prod_{i=1}^n (T - x_i) \), the \emph{discriminant} of \( f \) is \( \mathrm{Disc}(f) = \Delta^2 \in K \), where \( \Delta = \prod_{i < j} (x_i - x_j) \).
The discriminant is nonzero if and only if \( f \) is separable.
\begin{proposition}
	Let \( \fchar K \neq 2 \), and let \( f \in K[T] \) be a monic polynomial with splitting field \( L \).
	Let \( G = \Gal(f/K) \).
	Then the fixed field of \( G \cap A_n \) is \( K(\Delta) \), where \( \Delta^2 \) is the discriminant.
	In particular, \( \Gal(f/K) \subseteq A_n \) if and only if the discriminant \( \mathrm{Disc}(f) \) is a square.
\end{proposition}
\begin{proof}
	Let \( \pi \in S_n \).
	The sign of the permutation is given by \( \prod_{i < j} (T_{\pi(i)} - T_{\pi(j)}) = \sgn \pi \prod_{i < j} (T_i - T_j) \).
	Hence, if \( \sigma \in G \), we have \( \sigma(\Delta) = \sgn \sigma \cdot \Delta \).
	Because the characteristic is not 2, \( -1 \neq 1 \).
	Since \( \Delta \neq 0 \), this implies \( \Delta \in K \) if and only if \( G \subseteq A_n \), and \( \Delta \) lies in the fixed field \( F \) of \( G \cap A_n \).
	Because \( [F : K] = (G : G \cap A_n) \in \qty{1, 2} \), \( F = K(\Delta) \) exactly.
\end{proof}
\begin{example}
	Let \( n = 3, f = T^3 + aT + b = \prod_{i=1}^3 (T - x_i) \) where \( x_i \) lie in a splitting field for \( f \).
	Since there is no \( T^2 \) term, \( x_3 = -x_1 - x_2 \).
	Hence, \( a = x_1 x_2 - (x_1 + x_2)^2 \), and \( b = x_1x_2(x_1 + x_2) \).
	Therefore,
	\[ \mathrm{Disc}(f) = \qty[(x_1-x_2)(2x_1+x_2)(x_1+2x_2)]^2 = -4a^3 - 27b^2 \]
	In particular, \( \Gal(f/K) \subseteq A_3 \) if and only if \( -4a^3 - 27b^2 \) is a square in \( K \).

	For example, consider \( f = T^3 - 21T - 7 \in \mathbb Q[T] \).
	This is irreducible by Eisenstein's criterion.
	Its discriminant is \( 4 \cdot 21^3 - 27 \cdot 7^2 = (27 \cdot 7)^2 \), which is a square.
	So \( \Gal(f/K) \subseteq A_3 \).
	Since \( f \) is irreducible, \( \Gal(f/K) \) is transitive, so its order is divisible by 3.
	So \( \Gal(f/K) \) must be exactly \( A_3 \).
\end{example}
\begin{remark}
	This technique can be used to calculate the Galois group of any cubic polynomial for characteristic not \( 2, 3 \), for example.
\end{remark}
