\subsection{Cubics}
Let \( f \in K[T] \) be a monic separable cubic.
Then \( G = \Gal(f/K) \leq S_3 \) acting on the roots \( x_1, x_2, x_3 \) in a splitting field \( L \) of \( K \).
If \( f \) is reducible, \( f \) is either a product of three linear factors, in which case \( G \) is trivial, or \( f \) is a linear factor multiplied by a quadratic, in which case \( G \) is isomorphic to \( S_2 \).

Now suppose \( f \) is irreducible.
We will assume that \( \fchar K \neq 2, 3 \).
We have \( G = S_3 \) or \( G = A_3 \).
We know that \( G = A_3 \) if and only if the discriminant \( \mathrm{Disc}(f) \) is a square in \( K^\times \).
In general, the Galois correspondence yields
\begin{center}
    \begin{tikzcd}
        {L = K(x_1, x_2, x_3)} && {\qty{1}} \\
        {K_1 = K(\Delta)=L^{G \cap A_3}} && {G \cap A_3} \\
        K && G
        \arrow["{3 \text{ if } f \text{ irreducible, else } 1}", from=1-1, to=2-1]
        \arrow["{2 \text{ or } 1}", from=2-1, to=3-1]
        \arrow[from=3-3, to=2-3]
        \arrow[from=2-3, to=1-3]
    \end{tikzcd}
\end{center}
Then \( K_1 = K(\sqrt{\mathrm{Disc}(f)}) \), and \( K_1 = L \) if \( f \) is reducible.

In the irreducible case, \( L / K_1 \) is Galois with \( \Gal(L/K_1) \simeq \faktor{\mathbb Z}{3\mathbb Z} \).
Recall that if \( \omega \in K_1 \) is a primitive third root of unity, then \( L = K_1(y) \) where \( y^3 \in K_1 \), by Kummer theory.

We can compute this \( y \) explicitly.
Suppose \( f = T^3 + bT + c \) without loss of generality.
Then \( \Delta^2 = -4b^3 - 27c^2 \).
If \( b = 0 \), the roots of \( f \) are \( w^i \sqrt[3]{-c} \), so let \( y \) be any of them.
In the other case \( b \neq 0 \), let \( y \) be a Lagrange resolvent.
If the roots of \( f \) in \( L \) are \( x_1, x_2, x_3 \), take \( y = x_1 + \omega^2 x_2 + \omega x_3 = (1-\omega)(x_1 - \omega x_2) \) as \( x_1 + x_2 + x_3 = 0 \).
Then \( L(\omega) = K(\Delta, \omega, y) \) if and only if \( y \neq 0 \), by the proof of the structure of Kummer extensions.
Let \( y' = x_1 + \omega x_2 + \omega^2 x_3 \), then \( yy' = -3b \neq 0 \) since we are not in characteristic 3.
Note that \( y + y' = y + y' + x_1 + x_2 + x_3 = 3x_1 \).
One can calculate \( y^3 = \frac{1}{2} \qty(-3\sqrt{-3} \Delta + 27c) \), so \( x_1 = y - \frac{3b}{y} \).

If not, let \( L(\omega) \) be the splitting field of \( f \cdot (T^3 - 1) \) over \( K \).
Then \( L(\omega) / K_1(\omega) \) is Galois with Galois group \( \faktor{\mathbb Z}{3\mathbb Z} \) as before.
So \( L(\omega) = K_1(\omega, y) \) where \( y^3 \in K_1(\omega) \).

Therefore, in every case, \( x_i \) lie in the field obtained by adjoining successive square roots and cube roots to \( K \), since \( \omega = \frac{-1 + \sqrt{-3}}{2} \).
This is a theoretical description of Cardano's solution to the cubic.

\subsection{Quartics}
Let \( f \in K[T] \) be a monic separable quartic, with \( \fchar K \neq 2, 3 \).
Then \( \Gal(f/K) \leq S_4 \).
Note that \( S_4 \) acts on the partitions \( (12\mid 34), (13\mid 24), (14\mid 23) \) of \( \qty{1, 2, 3, 4} \).
Then we have a homomorphism \( S_4 \to S_3 \).
The kernel of this homomorphism is the Klein four-group \( V = \qty{e, (12)(34), (13)(24), (14)(23)} \vartriangleleft S_4 \).
Hence the homomorphism is surjective, as \( \abs{V} \cdot \abs{S_3} = \abs{S_4} \).

Let \( f \) have splitting field \( L \) with (distinct) roots \( x_1, \dots, x_4 \).
Suppose that \( x_1 + \dots + x_4 = 0 \) without loss of generality as the characteristic is not 2, so \( f = T^4 + aT^2 + bT + c \).
% TODO: diagram
Since \( V \) is a normal subgroup of \( S_4 \), \( G \cap V \) is a normal subgroup of \( G \) and contains \( V \).
In particular, we have a homomorphism \( \faktor{G}{G \cap V} \hookrightarrow \faktor{S_4}{V} \simeq S_3 \).
But \( \faktor{G}{G \cap V} = \Gal(M/K) \).
So we should be able to write \( M \) as the splitting field of a cubic \( g \in K[T] \).

Let \( y_{12} = x_1 + x_2 = -(x_3 + x_4) = -y_{34} \), and let \( y_{13}, y_{24}, y_{14}, y_{23} \) be defined similarly.
Note that \( G \cap V \) maps \( y_{12} \) to \( y_{12} \) or \( y_{34} = -y_{12} \), and so on.
So \( y_{12}^2, y_{13}^2, y_{14}^2 \) are fixed under \( G \cap V \).
Hence they lie in \( M = L^{G \cap V} \).

Suppose \( y_{12}^2 = y_{13}^2 \).
Then either \( y_{12} = y_{13} \), so \( x_2 = x_3 \), contradicting separability, or \( y_{12} = -y_{13} \), so \( 2x_1 + x_2 + x_3 = 0 \), giving \( x_1 = x_4 \), also contradicting separability.
So these are distinct elements of \( M \), and hence are indeed the roots of a separable cubic \( g \in K[T] \).
This is called the \emph{resolvent cubic}.

\( M = L^{G \cap V} \) is a splitting field of \( g \).
Note that \( x_1 = \frac{1}{2}(y_{12} + y_{13} + y_{14}) \) and similar results hold for \( x_2, x_3, x_4 \).
Hence \( L = M(y_{12}, y_{13}, y_{14}) \).
We can compute \( g = (T - y_{12}^2)(T - y_{13}^2)(T - y_{14}^2) = T^3 + 2aT^2 + (a^2 - 4c)T - b^2 \).
In particular, \( y_{12} y_{13} y_{14} = b \), hence we can simplify to \( L = M(y_{12}, y_{13}) \) where \( y_{12}^2, y_{13}^2 \in M \).

In conclusion, we have found a way to solve \( f = 0 \).
First, we solve the resolvent equation \( g = 0 \), and then we take at most two square roots to obtain the relevant field generators.

\subsection{Solubility by radicals}
Let \( f \in K[T] \) be a monic polynomial in a field \( K \) of characteristic zero.
To prove that there is no quintic formula, we must first establish a definition of `formula'.
The relevant notion is solubility by radicals.
\begin{definition}
	An irreducible polynomial \( f \in K[T] \) is \emph{soluble by radicals} over \( K \) if there exists a sequence of fields \( K = K_0 \subseteq K_1 \subseteq \dots \subseteq K_m \), with \( x \in K_m \) a root of \( f \), and each \( K_i \) is obtained from \( K_{i-1} \) by adjoining a root, so \( K_i = K_{i-1}(y_i) \) where \( y_i^{d_i} \in K_{i-1} \).
\end{definition}
\begin{remark}
	This is a generalisation of ruler and compass constructions to permit roots of arbitrary degree.
\end{remark}
Note that we can adjoin extra roots if desired.
In particular, adjoining roots of unity, \( f \) is soluble by radicals over \( K \) if there exists \( d \geq 1 \) and \( K = K_0 \subseteq \dots \subseteq K_m \), such that \( x \in K_m \) is a root of \( f \), and \( K_1 = K_0(\zeta_d) \) where \( \zeta_d \) is a primitive \( d \)th root of unity.
We can also assume that the other extensions satisfy \( K_i = K_{i-1}(y_i) \) for \( y_i^d = a_i \in K_{i-1} \).
This condition can be easily satisfied by letting \( d \) be the least common multiple of the \( d_i \) that occurs in the tower of fields.

Note that \( K_1 / K_0 \) is Galois with abelian Galois group.
\( K_i / K_{i-1} \) for \( i > 1 \) is Galois, where the Galois group is a subgroup of \( \faktor{\mathbb Z}{d\mathbb Z} \) as it is a Kummer extension.

To obtain all roots of \( f \), we consider a normal closure \( M \) of \( K_m \); this will contain a splitting field for \( f \), since it contains one root and \( f \) is irreducible.
To determine \( M \), let \( K_i' \subseteq M \) be a normal closure of \( K_i \) for each \( i \).
As we are in characteristic zero, an extension is Galois if and only if it is normal.
Note that \( K_1 \) is Galois, so \( K_1 = K_1' = K(\zeta_d) \).
\begin{proposition}
	\( K_i' = K_{i-1}'\qty(\qty{\sqrt[d]{\sigma(a_i)} \midd \sigma \in \Gal(K_{i-1}'/K)}) \).
\end{proposition}
\begin{proof}
	Suppose \( \sigma \in \Gal(K_{i-1}'/K) \).
	Then we can lift \( \sigma \) to an element \( \overline \sigma \in \Gal(K_i'/K) \) such that \( \eval{\overline\sigma}_{K_{i-1}'} = \sigma \).
	Since \( K_i' / K \) is normal, it contains \( \overline \sigma(y_i) \), and \( \overline \sigma(y)^d = \sigma(y^d) = \sigma(a_i) \).
	So the right hand side is contained in \( K_i' \).

	It suffices to show the right hand side is a normal extension.
	It is the splitting field over \( K_{i-1}' \) of the polynomial \( g_i = \prod_{\sigma \in \Gal(K_{i-1}'/K)} (T^d - \sigma(a_i)) \).
	This has coefficients in \( K \).
	If \( K_{i-1}' \) is the splitting field of some polynomial \( h_{i-1} \) over \( K \), then the right hand side is the splitting field of the product \( g_i h_{i-1} \) over \( K \).
	So it is normal.
\end{proof}
\begin{proposition}
	\( \Gal(K_i'/K_{i-1}') \) is abelian.
\end{proposition}
\begin{proof}
	This proof is a variant on the proof of a previous theorem.
	Consider the case \( i > 1 \).
	Let \( A = \Gal(K_i'/K_{i-1}') \).
	Let \( \tau \in A \) and \( \sigma \in \Gal(K_i'/K) \).
	Then \( \tau(\sqrt[d]{\sigma(a_i)}) = \zeta_d^{m_\sigma} \sqrt[d]{\sigma(a_i)} \) where \( m_\sigma \in \faktor{\mathbb Z}{d\mathbb Z} \).
	Hence \( \tau \mapsto (m_\sigma) \in \qty(\faktor{\mathbb Z}{d\mathbb Z})^r \) is an injective homomorphism, where \( r = \abs{\Gal(K_{i-1}'/K)} \).

	If \( i = 1 \), then \( K_1 = K(\zeta_d) \).
	So the Galois group is a subgroup of \( \qty(\faktor{\mathbb Z}{d\mathbb Z})^\times \), so is abelian.
\end{proof}
% TODO: diagram
Since all of the fields \( K_i' \) are normal closures, the \( N_i \) are normal subgroups of \( G \).
\begin{definition}
	A finite group \( G \) is \emph{soluble} if there exists a chain of normal subgroups \( N_i \trianglelefteq G \) with \( G = N_0 \supseteq N_1 \supseteq \dots \supseteq N_m = \qty{1} \) such that \( \faktor{N_i}{N_{i+1}} \) is abelian for all \( i \).
\end{definition}
\begin{example}
	Any abelian group is soluble.
	\( S_3 \) is soluble, by considering the chain \( S_3 \supset A_3 \supset \qty{1} \), as \( \faktor{S_3}{A_3} \simeq \faktor{\mathbb Z}{2\mathbb Z} \) and \( A_3 \simeq \faktor{\mathbb Z}{3\mathbb Z} \).
	\( S_4 \) is also soluble; the chain \( S_4 \supset A_4 \supset V \supset \qty{1} \) suffices.
	Note that \( \faktor{S_4}{A_4} \simeq \faktor{\mathbb Z}{2\mathbb Z} \), \( \faktor{A_4}{V} \simeq \faktor{\mathbb Z}{3\mathbb Z} \), \( V \simeq \qty(\faktor{\mathbb Z}{2\mathbb Z})^2 \).
\end{example}
We have shown that \( \faktor{N_i}{N_{i+1}} = \Gal(K_i'/K_{i-1}') \) is abelian.
Hence \( \Gal(M/K) \) is soluble.
\begin{lemma}
	Every subgroup and quotient of a soluble group is soluble.
\end{lemma}
\begin{proof}
	Let \( G = N_0 \supset N_1 \supset \dots N_m = \qty{1} \), where the quotients \( \faktor{N_i}{N_{i+1}} \) are abelian.
	Let \( H \leq G \).
	Then \( H \cap N_i \trianglelefteq H \), and there is an injective homomorphism from \( \faktor{H \cap N_i}{H \cap N_{i+1}} \) to \( \faktor{N_i}{N_{i+1}} \).
	Hence the \( \faktor{H \cap N_i}{H \cap N_{i+1}} \) are abelian, so \( H \) is soluble.

	Now let \( \pi \colon G \to \overline G = \faktor{G}{H} \) for \( H \trianglelefteq G \).
	Then \( \pi(N_i) \trianglelefteq \overline G \), and \( \faktor{N_i}{N_{i+1}} \) surjects onto \( \faktor{\pi(N_i)}{\pi(N_{i+1})} \).
\end{proof}
\begin{theorem}[Abel--Ruffini]
	Let \( f \in K[T] \) be soluble by radicals over \( K \).
	Then \( \Gal(f/K) \) is soluble.
\end{theorem}
\begin{proof}
	\( \Gal(f/K) = \Gal(L/K) \simeq \faktor{\Gal(M/K)}{\Gal(M/L)} \).
	We know that \( \Gal(M/K) \) is soluble, so the result follows from the fact that quotients of soluble groups are soluble.
\end{proof}
\begin{remark}
	One can easily show the converse to this theorem.
\end{remark}
\begin{proposition}
	If \( n \geq 5 \), then \( S_n \) and \( A_n \) are insoluble.
\end{proposition}
\begin{proof}
	\( S_n \) and \( A_n \) contain \( A_5 \) as a subgroup, so it suffices to show that \( A_5 \) is insoluble.
	\( A_5 \) is not abelian, and it is simple, so it is insoluble.
\end{proof}
\begin{corollary}
	Let \( n = \deg f \geq 5 \), and \( A_n \leq \Gal(f/K) \).
	Then \( f \) is not soluble by radicals over \( K \).
\end{corollary}
