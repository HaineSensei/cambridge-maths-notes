\subsection{Algebraic elements and minimal polynomials}
\begin{definition}
	Let \( L / K \) be an extension and \( x \in L \).
	\( x \) is \emph{algebraic over \( K \)} if there exists a nonzero polynomial \( f \in K[T] \) such that \( f(x) = 0 \).
	Otherwise, we say \( x \) is \emph{transcendental over \( K \)}.
\end{definition}
For \( f \in K[T] \), we have \( f(x) \in L \).
Varying \( f \), this gives a map \( \mathrm{ev}_x \colon K[T] \to L \) defined by \( f \mapsto f(x) \).
This is a ring homomorphism.

The kernel \( I = \ker(\mathrm{ev}_x) \subset K[T] \) is an ideal, the set of polynomials which vanish at \( x \).
As \( \Im(\mathrm{ev}_x) \) is a subring of \( L \) which is a field, it is an integral domain.
In particular, \( I \) is a prime ideal, so either \( I = 0 \), in which case \( x \) is transcendental over \( K \), or there exists a unique monic irreducible polynomial \( 0 \neq g \in K[T] \) such that \( I = (g) \), in which case \( x \) is algebraic and we say \( g \) is the \emph{minimal polynomial} of \( x \) over \( K \).
In this case, \( f(x) = 0 \) if and only if \( g \mid f \).
We write \( m_{x,K} \) for the minimal polynomial of \( x \) over \( K \).
Note that \( m_{x,K} \) is the monic polynomial in \( K \) of least degree with \( x \) as a root.

\begin{example}
	If \( x \in K \), \( m_{x,K} = T - x \).
	If \( p \) is prime and \( d \geq 1 \), \( T^d - p \in \mathbb Q[T] \) is irreducible by Eisenstein's criterion, so it is the minimal polynomial of \( \sqrt[d]{p} \in \mathbb R \) over \( \mathbb Q \).
	If \( p \) is prime, \( z = e^{\frac{2\pi i}{p}} \) is a root of \( T^p - 1 = (T-1)(T^{p-1} + T^{p-2} + \dots + 1) = (T-1)g(T) \), which is a product of irreducibles as
	\[ g(T+1) = \binom p p T^{p-1} + \binom p {p-1} T^{p-2} + \dots + \binom p 2 T + \binom p 1 \]
	This is irreducible by Eisenstein's criterion, so \( g \) is minimal for \( z \) over \( \mathbb Q \).
\end{example}

We say the degree of an algebraic element \( x \) over \( K \) is the degree of its minimal polynomial, written \( \deg_K x = \deg(x/K) \).

\begin{proposition}
	Let \( L / K \) and \( x \in L \).
	Then, the following are equivalent.
	\begin{enumerate}
		\item \( x \) is algebraic over \( K \).
		\item \( [K(x):K] \) is finite.
		\item \( K[x] \) is finite-dimensional as a \( K \)-vector space.
		\item \( K[x] = K(x) \).
		\item \( K[x] \) is a field.
	\end{enumerate}
	If these hold, \( \deg x = [K(x):K] \).
\end{proposition}
\begin{proof}
	\emph{(ii) implies (iii).} This follows since \( K[x] \subseteq K(x) \).

	\emph{(iv) is equivalent to (v)} is trivial.

	\emph{(iii) implies (v) and (ii).}
	Let \( 0 \neq y = g(x) \in K[x] \).
	Consider the map \( K[x] \to K[x] \) given by \( z \mapsto yz \).
	This is a \( K \)-linear transformation, and since \( y \neq 0 \) this is injective.
	Because \( \dim K[x] \) is finite, this injective map must be a bijection.
	Therefore there exists \( z \) such that \( yz = 1 \), so \( y \) is invertible.
	Hence (v) holds.
	Since (v) implies (iv), \( [K(x):K] = \dim_K K[x] < \infty \) as required for (ii).

	\emph{(v) implies (i).}
	If \( x = 0 \), the proof is complete, so assume \( x \neq 0 \).
	Then \( x^{-1} = a_0 + a_1 x + \dots + a_n x^n \in K[x] \).
	Therefore, \( a_n x^{n+1} + \dots + a_0 x - 1 = 0 \), so \( x \) is algebraic over \( K \).

	\emph{(i) implies (iii) and the degree formula.}
	The image of \( \mathrm{ev}_x \colon K[T] \to L \) is the subring \( K[x] \subset L \).
	If \( x \) is algebraic over \( K \), \( \ker(\mathrm{ev}_x) = (m_{x,K}) \) is a maximal ideal by irreducibility of \( m_{x,K} \).
	By the first isomorphism theorem, \( \faktor{K[T]}{(m_{x,K})} \cong K[x] \).
	But quotients by maximal ideals are fields, so \( K[x] \) is a field, proving (v).
	This polynomial is monic of degree \( d = \deg_K x \).
	Hence \( \faktor{K[T]}{(m_{x,K})} \) has a \( K \)-basis \( 1, T, \dots, T^{d-1} \).
	Thus, \( \dim_K K[x] = d = [K(x):K] < \infty \), proving (iii) and the degree formula.
\end{proof}
\begin{corollary}
	\( x_1, \dots, x_n \) are algebraic over \( K \) if and only if \( L = K(x_1, \dots, x_n) \) is finite over \( K \).
	If so, every element of \( K(x_1, \dots, x_n) \) is algebraic over \( K \).

	If \( x, y \) are algebraic over \( K \), then so are \( x \pm y \), \( xy \), and \( x^{-1} \) if \( x \) is nonzero.
	If \( L / K \) is a field extension, the set of algebraic elements of \( L \) forms a subfield of \( L \).
\end{corollary}
\begin{proof}
	If \( x_n \) is algebraic over \( K \), then it is also algebraic over \( K(x_1, \dots, x_{n-1}) \).
	Hence \( L / K(x_1, \dots, x_{n-1}) \) is finite.
	By induction on \( n \), the tower law gives the required result.
	Conversely, if \( L \) is finite over \( K \), the subfield \( K(y) \) is finite over \( K \) for all \( y \in L \), so \( y \) is algebraic over \( K \).

	Suppose \( x, y \) are algebraic over \( K \).
	Then \( x \pm y, xy, x^{-1} \in K(x,y) \), which is finite over \( K \) as required.
\end{proof}
\begin{example}
	Consider \( z = e^{2\pi i/p} \in \mathbb C \) where \( p \) is an odd prime.
	This has degree \( p - 1 \) as discussed above.
	Now consider \( x = 2\cos \frac{2\pi}{p} \), so \( x = z + \frac 1z \in \mathbb Q(z) \).
	This is algebraic over \( \mathbb Q \) because it belongs to this finite extension.
	Note that \( \mathbb Q(z) \supset \mathbb Q(x) \supset \mathbb Q \), and \( z^2 - xz + 1 = 0 \).
	Hence the degree of \( z \) over \( \mathbb Q(x) \) is at most 2.
	But \( [\mathbb Q(z):\mathbb Q(x)] \neq 1 \) because \( z \in \mathbb C \setminus \mathbb R \).
	By the tower law, we must have \( [\mathbb Q(z):\mathbb Q] = \frac{p-1}{2} \).

	We can now derive the minimal polynomial by considering \( z^{\frac{p-1}{2}} + z^{\frac{p-3}{2}} + \dots + z^{\frac{-p-1}{2}} = 0 \).
	Since \( z + z^{-1} = x \), we can express this as a polynomial in \( x \) of degree \( \frac{p-1}{2} \).
\end{example}
\begin{example}
	Let \( x = \sqrt m + \sqrt n \) where \( m, n \) are integers, and \( m, n, mn \) are not squares.
	We know that \( n = (x-\sqrt m)^2 = x^2 - 2x\sqrt m + m \), so \( [\mathbb Q(x):\mathbb Q(\sqrt m)] \leq 2 \).
	By symmetry, \( [\mathbb Q(x):\mathbb Q(\sqrt n)] \leq 2 \).
	Note that \( \sqrt m \in \mathbb Q(x) \) because \( \frac{x^2 + m - n}{2x} = \sqrt m \).

	\( m, n \) are not squares, so \( [\mathbb Q(\sqrt m):\mathbb Q] = 2 \).
	By the tower law we have \( [\mathbb Q(x):\mathbb Q] \in \qty{2,4} \).
	If \( [\mathbb Q(x):\mathbb Q] = 2 \), we have \( \mathbb Q(x) = \mathbb Q(\sqrt m) = \mathbb Q(\sqrt n) \).
	In this case, \( \sqrt m = a + b \sqrt n \implies m = a^2 + b^2 n + 2ab \sqrt n \), but \( n \) is not a square, so by rationality, \( ab = 0 \).
	But if \( b = 0 \), \( m \) is a square, and if \( a = 0 \), \( mn = b^2 n^2 \) is a square.
	Hence the degree of the field extension is 4.
\end{example}
\begin{definition}
	An extension \( L / K \) is \emph{algebraic} if all elements of \( L \) are algebraic over \( K \).
\end{definition}
\begin{lemma}
	Let \( M / L / K \), where \( L / K \) is algebraic.
	Suppose \( x \) is algebraic over \( L \).
	Then \( x \) is algebraic over \( K \).
\end{lemma}
\begin{proof}
	There exists \( f = T^n + a_{n-1} T^{n-1} + \dots + a_0 \in L[T] \) where \( f \neq 0 \) and \( f(x) = 0 \).
	Let \( L_0 = K(a_0, \dots, a_{n-1}) \).
	As each \( a_i \in L \) is algebraic over \( K \), we must have that \( [L_0 : K] \) is finite.
	As \( f \in L_0[T] \), \( x \) is algebraic over \( L_0 \).
	So \( [L_0(x):L_0] < \infty \implies [L_0(x):K] < \infty \).
	Hence \( [K(x):K] < \infty \), so \( x \) is algebraic over \( K \).
\end{proof}
\begin{proposition}
	\begin{enumerate}
		\item Finite extensions are algebraic.
		\item \( K(x) \) is algebraic over \( K \) if and only if \( x \) is algebraic over \( K \).
		\item If \( M / L / K \), we have \( M / K \) is algebraic if and only if \( M / L \) and \( L / K \) are algebraic.
	\end{enumerate}
\end{proposition}
\begin{proof}
	\begin{enumerate}
		\item \( [L : K] < \infty \), so for all \( x \in L \), \( [K(x) : K] < \infty \), so \( x \) is algebraic.
		\item Certainly if \( K(x) \) is algebraic over \( K \), we have that \( x \) is algebraic over \( K \).
			Conversely, if \( x \) is algebraic over \( K \), \( [K(x) : K] \) is finite, so it is algebraic by part (i).
		\item Suppose \( M / K \) is algebraic.
			Then for all \( x \in M \), we have that \( x \) is algebraic over \( K \), so it satisfies a polynomial \( f \in K[T] \).
			Hence \( f \in L[T] \) is another polynomial that \( x \) satisfies, so \( M / L \) is algebraic.
			\( L / K \) is clearly algebraic because it is contained within \( M \).

			Conversely, suppose \( M / L \) and \( L / K \) are algebraic.
			Let \( x \in M \).
			Then by the previous lemma, \( x \) is algebraic over \( K \) as required.
	\end{enumerate}
\end{proof}
\begin{example}
	Let \( K = \mathbb Q \) and \( L = \qty{x \in \mathbb C \mid x \text{ is algebraic over } \mathbb Q} = \overline{\mathbb Q} \).
	This extension \( \overline{\mathbb Q}/\mathbb Q \) is algebraic, but not finite.
	Indeed, for every \( n \geq 1 \), \( \sqrt[n]{2} \in L \), and \( [\mathbb Q(\sqrt[n]{2}):\mathbb Q] = n \) by irreducibility of \( T^n - 2 \).
	In particular, \( L \) contains subfields of arbitrarily large degree, so cannot be a finite extension.
\end{example}

\subsection{Algebraic numbers in the real line and complex plane}
Traditionally, we call \( x \in \mathbb C \) algebraic if it is algebraic over \( \mathbb Q \), otherwise it is transcendental.
\( \overline{\mathbb Q} = \qty{x \mid x \text{ algebraic}} \) is a proper subfield of \( \mathbb C \).
Indeed, \( \mathbb Q[T] \) is a countable set, and \( \mathbb C \) is uncountable.
However, it is difficult to explicitly find an element of \( \mathbb C \setminus \overline{\mathbb Q} \), or to show that a given number is transcendental.
\begin{example}
	Liouville's constant \( c = \sum_{n \geq 1} 10^{-n!} \) is transcendental, as proven in IA Numbers and Sets.
	This can be proven by showing that algebraic numbers cannot be `well approximated' by rationals.
\end{example}
\begin{example}
	Hermite and Lindemann showed that \( e \) and \( \pi \) are transcendental.
\end{example}
\begin{example}
	Let \( x, y \) be algebraic, and \( x \neq 0,1 \).
	Gelfond and Schneider showed that \( x^y \) is algebraic if and only if \( y \) is rational.
	In particular, \( e^\pi = (-1)^{-i} \) is transcendental.
\end{example}
