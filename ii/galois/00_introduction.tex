\subsection{Introduction}
Galois theory concerns itself with solving polynomial equations of higher degree, and discussing how the symmetries of these polynomials relate to their solubility.
The modern interpretation of Galois theory is more interested in the fields that particular polynomials generate, rather than their particular solutions; this naturally extends to studying symmetries of fields.

\subsection{Solving quadratics, cubics and quartics}
Methods for solving quadratic equations have been known since the time of the Babylonians.
Consider \( aX^2 + bX + c \), and complete the square into \( (X + \frac{1}{2} b)^2 + c - \frac{b^2}{4} \).
This leads directly into the usual formula.

Alternatively, consider \( (X-x_1)(X-x_2) \) and expand, giving \( X^2 - (x_1 + x_2) X + x_1 x_2 \).
Thus, \( x_1 + x_2 = -b \) and \( x_1 x_2 = c \).
We can write \( x_1 = \frac{1}{2} \qty[(x_1 + x_2) + (x_1 - x_2)] \), where \( x_1 + x_2 = b \) and \( (x_1 - x_2)^2 = b^2 - 4c \).

Cubics were solved much later, in the early 16th century, by the Italian mathematician del Ferro.
Consider the cubic \( X^3 + aX^2 + bX + c \), written as \( (X-x_1)(X-x_2)(X-x_3) \).
Multiplying, we find
\[ x_1 + x_2 + x_3 = -a;\quad x_1 x_2 + x_2 x_3 + x_3 x_1 = b;\quad x_1 x_2 x_3 = -c \]
Without loss of generality we can set \( a = 0 \) by replacing \( X \mapsto X - \frac{a}{3} \).
Now,
\[ x_1 = \frac{1}{3} \qty[(x_1 + x_2 + x_3) + \underbrace{(x_1 + \omega x_2 + \omega^2 x_3)}_{u} + \underbrace{(x_1 + \omega^2 x_2 + \omega x_3)}_{v}] \]
where \( \omega = e^{\frac{2\pi i}{3}} \).
The \( u, v \) are known as Lagrange resolvents.
Applying a cyclic permutation to \( x_1, x_2, x_3 \) in \( u \) or \( v \), we find \( u \mapsto \omega u \) and \( v \mapsto \omega v \).
Hence, the cubes of \( u \) and \( v \) are invariant under cyclic permutations of \( x_1, x_2, x_3 \).
Under a permutation \( x_2 \mapsto x_3, x_3 \mapsto x_2 \), \( u \) and \( v \) swap.
Hence, \( u^3 + v^3 \) and \( u^3 v^3 \) are invariant under all permutations of roots.
A general fact that we will prove later is that such invariant expressions can be written in terms of the coefficients of the polynomial.
In this case, we have
\[ u^3 + v^3 = -27c;\quad u^3 v^3 = -27 b^2 \]
Now, \( u^3 \) and \( v^3 \) are the roots of the quadratic \( Y^2 + 27cY - 27b^2 \).
This then provides a formula for the root \( x_1 \).
This process is known as Cardano's formula.

Similarly, the quartic \( X^4 + aX^3 + bX^2 + cX + d \) can be solved by producing an auxiliary cubic equation, in a similar way to the auxiliary quadratic equation found for the cubic case above.
However, the same process does not work for the quintic; the auxiliary equation has a degree which is too large.
The underlying reason behind this is to do with group theory, and in particular, the group structure of \( S_5 \) and \( A_5 \).
This will be explored later in the course.
