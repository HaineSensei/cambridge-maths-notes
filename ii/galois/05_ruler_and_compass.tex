\subsection{Definitions}
\begin{definition}
	A \emph{ruler and compass construction} in plane geometry is a drawing constructed with the following methods.
	\begin{enumerate}
		\item Given \( P_1, P_2, Q_1, Q_2 \) in the plane and \( P_i \neq Q_i \), we can construct the point of intersection of the lines \( P_1Q_1 \) and \( P_2Q_2 \), if indeed they do intersect.
		\item Given \( P_1, P_2, Q_1, Q_2 \) in the plane and \( P_i \neq Q_i \), we can construct the points of intersection of the circles with centres \( P_i \) that pass through the \( Q_i \), if they intersect.
		\item Similarly we can construct the points of intersection of a line and a circle.
	\end{enumerate}
	A point \( (x,y) \in \mathbb R^2 \) is \emph{constructible} from a set \( \qty{(x_1, y_1), \dots, (x_n, y_n)} \) if it can be obtained by finitely many expansions of the set under applications of the above operations.
	A real number \( x \in \mathbb R \) is \emph{constructible} if \( (x,0) \) is constructible from \( \qty{(0,0), (1,0)} \).
\end{definition}
\begin{remark}
	Every rational is constructible.
	Square roots of constructible numbers are constructible.
\end{remark}
\begin{definition}
	Let \( K \subseteq \mathbb R \) be a subfield of the reals.
	We say \( K \) is \emph{constructible} if there exists \( n \in \mathbb N \) and fields \( Q = F_0 \subset F_1 \subset \dots \subset F_n \subseteq \mathbb R \) and \( a_i \in F_i \) for \( 1 \leq i \leq n \) such that
	\begin{enumerate}
		\item \( K \subseteq F_n \);
		\item \( F_i = F_{i-1}(a_i) \);
		\item \( a_i^2 \in F_{i-1} \).
	\end{enumerate}
\end{definition}
\begin{remark}
	By conditions (ii) and (iii), \( F_i / F_{i-1} \) is at most a quadratic extension.
	Then, by the tower law, \( F_n / \mathbb Q \) has degree a power of two, so \( K / \mathbb Q \) is a finite extension with degree a power of two.
\end{remark}
\begin{theorem}
	If \( x \) is constructible, \( \mathbb Q(x) \) is constructible.
\end{theorem}
\begin{proof}
	Let \( K = \mathbb Q(x) \).
	We show that if \( (x,y) \) can be constructed with \( k \) steps, \( \mathbb Q(x,y) \) is a constructible extension of \( \mathbb Q \).
	By induction, suppose \( \mathbb Q = F_0 \subset \dots \subset F_n \) satisfy conditions (ii) and (iii) such that the coordinates of the points obtained after \( k-1 \) constructions lie in \( F_n \).

	The intersection point of two lines has coordinates given by rational functions of the coordinates of the points \( P_i, Q_i \) with rational coefficients.
	In particular, if the \( k \)th construction is of this type, the intersection point has coordinates in \( F_n \).
	We can similarly see that the intersection points of two circles and the intersection points of a line and a circle have coordinates given by quadratic equations \( a \pm b \sqrt e, c \pm d \sqrt e \), where \( a, b, c, d, e \) are rational functions of the coordinates \( P_i, Q_i \).
	Thus the new points have coordinates which lie in \( F_n(\sqrt e) \), a constructible extension of \( \mathbb Q \) as required.
\end{proof}
\begin{corollary}
	If \( x \) is constructible, \( x \) is algebraic over \( \mathbb Q \) and the degree of the minimal polynomial is a power of two.
\end{corollary}
\begin{remark}
	One can show that if \( \mathbb Q(x) \) is constructible, we also have \( x \) is constructible, so the above theorem is a bi-implication.
	However, this will not be required for our purposes in this course.
\end{remark}

\subsection{Classical problems}
\begin{theorem}
	It is impossible to square the circle.
\end{theorem}
\begin{proof}
	The statement is to construct a square with area equal to that of a given circle.
	In particular, we must construct \( \sqrt \pi \).
	Suppose such a construction can occur.
	Then \( \pi \) is also constructible.
	But \( \pi \) is transcendental and hence inconstructible.
\end{proof}
\begin{theorem}
	It is impossible to duplicate the cube.
\end{theorem}
\begin{proof}
	To duplicate the cube, one must be able to construct \( \sqrt[3]{2} \).
	The minimal polynomial of \( \sqrt[3]{2} \) is \( X^3 - 2 \).
	This can be easily checked with Eisenstein's criterion.
	Since the minimal polynomial is of degree not a power of two, \( \sqrt[3]{2} \) is inconstructible.
\end{proof}
\begin{theorem}
	It is impossible to trisect a given angle.
\end{theorem}
\begin{proof}
	If we can trisect any constructible angle, we can in particular trisect the (constructible) angle \( \frac{2\pi}{3} \), for example to construct a regular nonagon.
	Then the angle \( \frac{2\pi}{9} \) would be constructible, so its sine and cosine would be constructible.
	By the triple angle formula for cosine,
	\[ \cos 3\theta = 4\cos^3 \theta - 3 \cos\theta \implies 4\cos\qty(\frac{2\pi}{9})^3 - 3\cos\qty(\frac{2\pi}{9}) = \cos\qty(\frac{2\pi}{3}) \]
	Hence \( \cos\qty(\frac{2\pi}{9}) \) is a root of \( 8X^3 - 6X + 1 \).
	In particular, \( 2\cos\qty(\frac{2\pi}{9}) - 2 \) is a root of \( X^3 + 6X^2 + 9X + 3 \) is irreducible by Eisenstein's criterion.
	But this has degree 3, so \( \deg_{\mathbb Q} \cos \qty(\frac{2\pi}{9}) = 3 \), so this is inconstructible.
	In particular, the regular nonagon is inconstructible.
\end{proof}
We will later prove the following theorem.
\begin{theorem}[Gauss]
	A regular \( n \)-gon is is constructible if and only if \( n \) is the product of a power of two and distinct \emph{Fermat primes}, which are the primes of the form \( 2^{2^k} + 1 \).
\end{theorem}
