\subsection{?}
Let \( L / K \) be an extension of degree \( n \), so \( L \) is a \( K \)-vector space of dimension \( n \).
Let \( x \in L \).
Then the map \( U_x \colon L \to L \) defined by \( U_x(y) = xy \) is \( K \)-linear, as it is \( L \)-linear.
Since it is a linear map, it has a characteristic polynomial, a determinant, and a trace.
\begin{definition}
	The \emph{trace} and \emph{norm} of \( x \in L \) (relative to the extension \( L / K \)) are \( \Tr_{L/K}(x) = \tr U_x \in K \) and \( N_{L/K}(x) = \det U_x \in K \) respectively.
	The \emph{characteristic polynomial} of \( x \in L \) is \( f_{x,L/K} = \det (TI - U_x) \in K[T] \) where \( I \) is the identity linear transformation.
\end{definition}
We sometimes write \( \tr_K \), \( \det_K \).
Let \( e_1, \dots, e_n \) be a basis for \( L / K \).
Then \( U_x \) can be written as a unique \( K \)-valued matrix \( A = (a_{ij}) \), so \( xe_i = \sum_j a_{ji} e_j \).
Then \( \Tr_{L/K}(x) = \tr(A) \), and so on.
\begin{example}
	Consider the quadratic extension \( \mathbb Q(\sqrt d)/\mathbb Q \) with the basis \( 1, \sqrt d \).
	Let \( x = a + b\sqrt d \).
	Since \( x \cdot 1 = a + b \sqrt d \) and \( x \cdot \sqrt d = bd + a\sqrt d \),
	\[ A = \begin{pmatrix}
		a & bd \\
		b & a
	\end{pmatrix} \]
	Hence \( \Tr_{L/K}(x) = 2a \) and \( N_{L/K}(x) = a^2 - b^2 d \).
\end{example}
\begin{example}
	Consider \( \mathbb C / \mathbb R \) with the basis \( 1, i \).
	Then the matrix of \( U_{x+iy} \) is
	\[ \begin{pmatrix}
		x & -y \\
		y & x
	\end{pmatrix} \]
	which is the usual encoding of complex numbers as \( 2 \times 2 \) real matrices.
	Note the similarity between this matrix and the Cauchy-Riemann equations
	% TODO: fill them in i can't remember them now
\end{example}
\begin{lemma}
	Let \( x, y \in L \) and \( a \in K \), where \( n = [L : K] \).
	Then,
	\begin{enumerate}
		\item \( \Tr_{L/K}(x+y) = \Tr_{L/K}(x) + \Tr_{L/K}(y) \);
		\item \( N_{L/K}(xy) = N_{L/K}(x)N_{L/K}(y) \);
		\item \( N_{L/K}(x) = 0 \) if and only if \( x = 0 \);
		\item \( \Tr_{L/K}(1) = n \) and \( N_{L/K}(1) = 1 \);
		\item \( \Tr_{L/K}(ax) = a\Tr_{L/K}(x) \) and \( N_{L/K}(ax) = a^n N_{L/K}(x) \).
	\end{enumerate}
	In particular, \( \Tr_{L/K} \) is \( K \)-linear and \( N_{L/K} \colon L^\times \to K^\times \) is a homomorphism.
\end{lemma}
\begin{proof}
	For part (iii), \( N_{L/K}(x) = \det(U_x) \neq 0 \) if and only if \( U_x \) is invertible.
	But this holds if and only if \( x \) is nonzero because \( L \) is a field.
	The other results follow from the laws of linear transformations.
\end{proof}
