\subsection{Trace and norm}
Let \( L / K \) be an extension of degree \( n \), so \( L \) is a \( K \)-vector space of dimension \( n \).
Let \( x \in L \).
Then the map \( U_x \colon L \to L \) defined by \( U_x(y) = xy \) is \( K \)-linear, as it is \( L \)-linear.
Since it is a linear map, it has a characteristic polynomial, a determinant, and a trace.
\begin{definition}
	The \emph{trace} and \emph{norm} of \( x \in L \) (relative to the extension \( L / K \)) are \( \Tr_{L/K}(x) = \tr U_x \in K \) and \( N_{L/K}(x) = \det U_x \in K \) respectively.
	The \emph{characteristic polynomial} of \( x \in L \) is \( f_{x,L/K} = \det (TI - U_x) \in K[T] \) where \( I \) is the identity linear transformation.
\end{definition}
We sometimes write \( \tr_K \), \( \det_K \).
Let \( e_1, \dots, e_n \) be a basis for \( L / K \).
Then \( U_x \) can be written as a unique \( K \)-valued matrix \( A = (a_{ij}) \), so \( xe_i = \sum_j a_{ji} e_j \).
Then \( \Tr_{L/K}(x) = \tr(A) \), and so on.
\begin{example}
	Consider the quadratic extension \( \mathbb Q(\sqrt d)/\mathbb Q \) with the basis \( 1, \sqrt d \).
	Let \( x = a + b\sqrt d \).
	Since \( x \cdot 1 = a + b \sqrt d \) and \( x \cdot \sqrt d = bd + a\sqrt d \),
	\[ A = \begin{pmatrix}
		a & bd \\
		b & a
	\end{pmatrix} \]
	Hence \( \Tr_{L/K}(x) = 2a \) and \( N_{L/K}(x) = a^2 - b^2 d \).
\end{example}
\begin{example}
	Consider \( \mathbb C / \mathbb R \) with the basis \( 1, i \).
	Then the matrix of \( U_{x+iy} \) is
	\[ \begin{pmatrix}
		x & -y \\
		y & x
	\end{pmatrix} \]
	which is the usual encoding of complex numbers as \( 2 \times 2 \) real matrices.
	Note the similarity between this matrix and the Cauchy--Riemann equations
	\[ \pdv{u}{x} = \pdv{v}{y};\quad \pdv{u}{y} = -\pdv{v}{x} \]
\end{example}
\begin{lemma}
	Let \( x, y \in L \) and \( a \in K \), where \( n = [L : K] \).
	Then,
	\begin{enumerate}
		\item \( \Tr_{L/K}(x+y) = \Tr_{L/K}(x) + \Tr_{L/K}(y) \);
		\item \( N_{L/K}(xy) = N_{L/K}(x)N_{L/K}(y) \);
		\item \( N_{L/K}(x) = 0 \) if and only if \( x = 0 \);
		\item \( \Tr_{L/K}(1) = n \) and \( N_{L/K}(1) = 1 \);
		\item \( \Tr_{L/K}(ax) = a\Tr_{L/K}(x) \) and \( N_{L/K}(ax) = a^n N_{L/K}(x) \).
	\end{enumerate}
	In particular, \( \Tr_{L/K} \) is \( K \)-linear and \( N_{L/K} \colon L^\times \to K^\times \) is a homomorphism.
\end{lemma}
\begin{proof}
	For part (iii), \( N_{L/K}(x) = \det(U_x) \neq 0 \) if and only if \( U_x \) is invertible.
	But this holds if and only if \( x \) is nonzero because \( L \) is a field.
	The other results follow from the laws of linear transformations.
\end{proof}

\subsection{Formulae and applications}
\begin{theorem}
	Let \( M / L / K \) be a tower of finite extensions.
	Then, for all \( x \in M \),
	\[ \Tr_{L/K}(\Tr_{M/L}(x)) = \Tr_{M/K}(x);\quad N_{L/K}(N_{M/L}(x)) = N_{M/K}(x) \]
\end{theorem}
\begin{proof}
	We prove the theorem for the trace; we will not need the result for the norm.
	% TODO: Fill in from online
	Let \( x \in M \).
	Let \( u_1, \dots, u_m \) be a basis for \( M / L \), and let \( v_1, \dots, v_n \) be a basis for \( L / K \).
	Let \( (a_{ij}) \) be the matrix of \( U_{x,M/L} \), so \( (a_{ij}) \in \mathrm{Mat}_{m,m}(L) \).
	Then \( \Tr_{M/L}(x) = \sum_{i=1}^m a_{ii} \).
	For each \( (i, j) \), let the matrix of \( U_{a_{ij}} \) be \( A_{ij} \in \mathrm{Mat}_{n,n}(K) \).
	Then, \( \Tr_{L/K}(\Tr_{M/L}(x)) = \sum_{i=1}^m \Tr_{L/K}(a_{ii}) = \sum_{i=1}^m \tr(A_{ii}) \).

	Consider the basis \( u_1v_1, \dots, u_1v_m, u_2v_1, \dots, u_nv_m \) for \( M \) over \( K \).
	Then the matrix of \( U_{x,M/K} \) is the block matrix
	\[ \begin{pmatrix}
		A_{11} \\
		& A_{22} \\
		& & \ddots \\
		& & & A_{nn}
	\end{pmatrix} \]
	which has trace \( \sum_{i=1}^m \tr(A_{ii}) \) as required.
\end{proof}
\begin{proposition}
	Let \( L = K(x) \), and \( f = T^n + c_{n-1} T^{n-1} + \dots + c_0 \in K[T] \) be the minimal polynomial for \( x \) over \( K \).
	Then \( f_{x,L/K} = f \).
	Further, \( \Tr_{L/K}(x) = -c_{n-1} \) and \( N_{L/K}(x) = (-1)^n c_0 \).
\end{proposition}
\begin{proof}
	It suffices to prove the first statement, since the second follows from the fact that the determinant and trace are the given coefficients of the characteristic polynomial for any linear transformation.
	Consider the basis \( 1, x, \dots, x^{n-1} \) for \( L / K \).
	Then, the matrix of \( U_x \) is
	\[ \begin{pmatrix}
		0 & \cdots & & & & -c_0 \\
		1 & 0 & \cdots & & & -c_1 \\
		0 & 1 & 0 & \cdots \\
		\vdots & 0 & 1 & 0 & \cdots \\
		& \vdots & 0 & 1 & \cdots \\
		& & \vdots & \vdots & \ddots \\
		& & & & & -c_{n-1}
	\end{pmatrix} \]
	which has characteristic polynomial \( f \) since it is in rational canonical form.
\end{proof}
\begin{corollary}
	Let \( \fchar K = p > 0 \), and \( L = K(x) \) where \( x \not\in K \) but \( x^p \in K \).
	Then for all \( y \in L \), we have \( \Tr_{L/K}(y) = 0 \) and \( N_{L/K}(y) = y^p \).
\end{corollary}
\begin{proof}
	Recall that the minimal polynomial of \( x \) is \( T^p - x^p \), so \( [L : K] = p \).
	Suppose that \( y \in K \).
	By a previous lemma, \( \Tr_{L/K}(y) = py = 0 \) and \( N_{L/K}(y) = y^p \).
	Otherwise, since \( [L:K] \) is prime, \( K(y) = L \), and in particular, if \( y = \sum a_i x^i \) then \( y^p = \qty(\sum a_i x^i)^p = \sum a_i (x^p)^i \in K \).
	So the minimal polynomial of \( y \) is \( T^p - y^p \).
	Applying the previous proposition, the result follows.
\end{proof}
\begin{proposition}
	Let \( L / K \) be a finite separable extension of degree \( n \).
	Let \( \sigma_1, \dots, \sigma_n \colon L \to M \) be the distinct \( K \)-homomorphisms of \( L \) into a normal closure \( M \) for \( L / K \).
	Then
	\[ \Tr_{L/K}(x) = \sum_{i=1}^n \sigma_i(x); \quad N_{L/K}(x) = \prod_{i=1}^n \sigma_i(x);\quad f_{x,L/K} = \prod_{i=1}^n (T - \sigma_i(x)) \]
\end{proposition}
\begin{remark}
	If \( L / K \) is finite and Galois, then \( \Tr_{L/K}(x) = \sum_{\sigma \in \Gal(L/K)} \sigma(x) \), and the other results are similar.
\end{remark}
\begin{proof}
	It suffices to show the result for the characteristic polynomial.
	Let \( e_1, \dots, e_n \) be a basis for \( L / K \).
	Let \( P = (\sigma_i(e_j)) \in \mathrm{Mat}_{n,n}(M) \).
	Recall that the \( \sigma_i \) are linearly independent, so there exist no \( y_i \in M \) such that for all \( j \), \( \sigma_i(e_j) = 0 \).
	Hence \( P \) is nonsingular.
	Let \( A = (a_{ij}) \) be the matrix of \( U_x \), so \( xe_j = \sum_r a_{rj} e_r \).
	Applying \( \sigma_i \), we have
	\[ \sigma_i(x) \sigma_i(e_j) = \sum_r \sigma_i(e_r) a_{rj} \]
	So if \( S \) is the diagonal matrix with \( (i,i) \)th entry \( \sigma_i(x) \), then the given equation can be rewritten as \( SP = PA \).
	Therefore \( S = PAP^{-1} \).
	So \( S \) and \( A \) are conjugate matrices and hence have the same characteristic polynomial.
	We explicitly find that the characteristic polynomial of \( S \) is \( \prod(T - \sigma_i(x)) \) and the characteristic polynomial of \( A \) is \( f_{x,L/K} \).
	So they are equal as required.
\end{proof}
Note that since the trace \( \Tr_{L/K} \colon L \to K \) is \( K \)-linear, it is either the zero map or surjective.
\begin{theorem}
	Let \( L / K \) be a finite extension.
	Then, \( L / K \) is separable if and only if \( \Tr_{L/K} \) is surjective.
\end{theorem}
\begin{remark}
	If \( \fchar K = 0 \), \( \Tr_{L/K}(1) = n \neq 0 \), so the result holds easily.
\end{remark}
\begin{proof}
	Suppose \( L / K \) is separable, and \( \sigma_i, \dots, \sigma_n \) are the \( K \)-homomorphisms of \( L \) into a normal closure \( M \) of \( L / K \).
	Then \( \Tr_{L/K}(x) = \sum_{i=1}^n \sigma_i(x) \).
	As the \( \sigma_i \) are linearly independent, there exists \( x \) such that \( \sum_{i=1}^n \sigma_i(x) \neq 0 \).
	So \( \Tr_{L/K}(x) \neq 0 \), and in particular, it must be surjective as it is \( K \)-linear.

	Now suppose \( L / K \) is inseparable.
	Then there exists \( x \in L \) such that \( K(x) \supsetneq K(x^p) \) from example 7 on example sheet 2.
	As we have shown, \( \Tr_{K(x)/K(x^p)} = 0 \), so
	\[ \Tr_{L/K} = \Tr_{L/K(x)} \circ \Tr_{K(x)/K(x^p)} \circ \Tr_{K(x^p)/K} = 0 \]
\end{proof}
\begin{example}
	Consider the extension of finite fields \( \mathbb F_{q^n} / \mathbb F_q \) for \( q = p^r \).
	This is separable, so there exists \( x \in \mathbb F_{q^n} \) such that \( \Tr(x) = 1 \).
	It is also possible to prove this directly by using the fact that the multiplicative group is cyclic.
\end{example}
\begin{remark}
	This criterion can be used to give another proof that if \( M / L \) and \( L / K \) are separable, \( M / K \) is also separable.
\end{remark}
