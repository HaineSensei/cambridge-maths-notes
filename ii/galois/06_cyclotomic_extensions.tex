\subsection{Primitive roots of unity}
\begin{lemma}
	Let \( C \) be a cyclic group of order \( n > 1 \).
	Let \( a \in \mathbb Z \) be coprime with \( n \).
	Then the map \( [a] \colon C \to C \) given by \( [a](g) = g^a \) is an automorphism of \( C \), and the map \( \qty(\faktor{\mathbb Z}{n\mathbb Z})^\times \to \Aut(C) \) defined by \( a \mapsto [a] \) is an isomorphism.
\end{lemma}
\begin{proof}
	\( [a] \) is clearly a homomorphism, and since \( a \) is coprime to \( n \), it is an automorphism since there exists \( b \) such that \( ab \) is congruent to 1 modulo \( n \).
	Hence, there is an injection \( \qty(\faktor{\mathbb Z}{n\mathbb Z})^\times \to \Aut(C) \) given by \( a \mapsto [a] \), and it is a homomorphism.
	If \( \varphi \in \Aut(C) \) and \( g \) is a generator for \( C \), \( \varphi(g) = g^a \) for some \( a \in \qty(\faktor{\mathbb Z}{n\mathbb Z})^\times \).
	So \( \varphi = [a] \), and in particular, the map is an isomorphism.
\end{proof}
Let \( K \) be a field and \( n \geq 1 \).
We define \( \bm \mu_n(K) = \qty{x \in K \mid x^n = 1} \) for the group (under multiplication) of \( n \)th roots of unity in \( K \).
This is a finite subgroup of \( K^\times \), hence it is cyclic.
The order of any element divides \( n \), so it has order dividing \( n \).

We say that \( \zeta \in \bm \mu_n(K) \) is a \emph{primitive} \( n \)th root of unity if its order is exactly \( n \).
Such a \( \zeta \) exists if and only if \( \bm \mu_n(K) \) has \( n \) elements, and then \( \zeta \) is a generator for the group.
In particular, \( f = T^n - 1 \) has \( n \) distinct roots, \( \zeta_i \) for \( i \in \qty{0, \dots, n-1} \), and hence it is separable.
In general, \( f = T^n - 1 \) is separable if and only if \( f \) is coprime with \( f' = nT^{n-1} \), which holds if and only if \( n \neq 0 \).
In this section, we assume that the characteristic of \( K \) is zero or is a positive number \( p \) that does not divide \( n \), so \( f \) is separable.

Let \( L / K \) be a splitting field for \( T^n - 1 \).
This is Galois since \( f \) is separable, so we can define \( G = \Gal(L/K) \).
Then \( \abs{\bm \mu_n(L)} = n \), and so there exists a primitive \( n \)th root of unity \( \zeta = \zeta_n \in L \).
Such an \( L \) is called a \emph{cyclotomic extension}.
\begin{proposition}
	\( L = K(\zeta) \).
	There exists an injective homomorphism \( \chi = \chi_n \colon \Gal(L/K) \to \qty(\faktor{\mathbb Z}{n\mathbb Z})^\times \) such that \( \chi(\sigma) = a \) implies \( \sigma(\zeta) = \zeta^n \).
	In particular, \( G \) is abelian.
	\( \chi \) is an isomorphism if and only if \( G \) acts transitively on the set of primitive roots of unity in \( L \).
\end{proposition}
The homomorphism \( \chi \) is called the \emph{cyclotomic character}.
\begin{proof}
	\( \bm \mu_n(L) \) is cyclic and generated by \( \zeta \), so the roots of \( T^n - 1 \) are the powers of \( \zeta \), so \( L = K(1,\zeta,\zeta^2,\dots,\zeta^{n-1}) = K(\zeta) \).
	Consider the action of \( G \) on \( L \).
	This action permutes \( \bm \mu_n(L) \), and if \( \zeta, \zeta' \in \bm \mu_n(L) \) and \( \sigma \in G \), then \( \sigma(\zeta\zeta) = \sigma(\zeta)\sigma(\zeta') \), so \( \sigma \) acts as an automorphism of \( \bm \mu_n(L) \).
	\( \sigma(\zeta) = \zeta \) if and only if \( \sigma \) is the identity because \( L = K(\zeta) \).
	This gives an injective homomorphism \( G \hookrightarrow \Aut \bm \mu_n(L) \simeq \qty(\faktor{\mathbb Z}{n\mathbb Z})^\times \).

	\( \zeta_n^a \) is primitive if and only if \( a \) is coprime to \( n \).
	Therefore the set of primitive \( n \)th roots of unity is \( \qty{\zeta^a \mid a \in \qty(\faktor{\mathbb Z}{n\mathbb Z})^\times} \), which by the previous part, is the orbit of \( \zeta \) under \( G \).
	The map is surjective if and only if there is one orbit, so the result follows.
\end{proof}

\subsection{Cyclotomic polynomials}
\begin{definition}
	Let \( K \) have characteristic zero or a prime \( p \) that does not divide \( n \).
	The \emph{\( n \)th cyclotomic polynomial} is
	\[ \Phi_n(t) = \prod_{a \in \faktor{\mathbb Z}{n\mathbb Z}} (T - \zeta_n^a) \]
	in a splitting field \( L \) of \( T^n - 1 \).
\end{definition}
This is the polynomial where the roots are the primitive \( n \)th roots of unity.
As \( G \) permutes the primitive \( n \)th roots of unity in \( L \), \( \Phi_n \) has coefficients in \( L^G[T] = K[T] \).
The last part of the above proposition shows that \( \chi \) is surjective if and only if \( \Phi_n \in K[T] \) is irreducible.

\( x \in L \) satisfies \( x^n - 1 = 0 \) if and only if \( x \) is a primitive \( d \)th root of unity for some unique \( d \mid n \).
Hence \( T^n - 1 = \prod_{d \mid n} \Phi_d \), since the sets of roots are equal.
In particular, we could have inductively defined the cyclotomic polynomials by \( \Phi_n = \frac{T^n - 1}{\prod_{d \mid n, d \neq n} \Phi_d} \).
This shows that the \( \Phi_n \) do not depend on the choice of field \( K \), since \( \Phi_n \) is the image in \( K[T] \) of a polynomial in \( \mathbb Z[T] \).

For example, \( \Phi_p = \frac{T^p - 1}{T - 1} = T^{p-1} + T^{p-2} + \dots + \dots + T + 1 \).
We also have \( \Phi_1 = T - 1 \) and \( \Phi_{p^n}(T) = \frac{T^{p^n} - 1}{T^{p^{n-1}} - 1} = \Phi_p(T^{p^{n-1}}) \).
We have \( \deg \Phi_n = \abs{\qty(\faktor{\mathbb Z}{n\mathbb Z})^\times} = \varphi(n) \) where \( \varphi \) is the Euler totient function.
\begin{theorem}[irreducibility of cyclotomic polynomials]
	Let \( K = \mathbb Q \).
	Then \( \chi_n \) is an isomorphism for all \( n > 1 \).
	In particular, \( [\mathbb Q(\zeta_n) : \mathbb Q] = \varphi(n)] \), and \( \Phi_n \) is irreducible over \( \mathbb Q \).
\end{theorem}
\begin{proof}
	The statements in the theorem are all equivalent by the previous results, so it suffices to prove that \( \Phi_n \) is irreducible over \( \mathbb Q \).
	If \( n \) is prime, we have already proven its irreducibility by Eisenstein's criterion and Gauss' lemma.
	We can easily extend this to the case where \( n \) is a prime power.
	
	Note that \( \chi_n \) is an isomorphism if for all primes \( p \nmid n \), the residue class of \( p \in \qty(\faktor{\mathbb Z}{n\mathbb Z})^\times \) is in the image of \( \chi \), by factorising \( a \) as a product of primes if \( a \) is coprime to \( n \).
	Let \( f \) be the minimal polynomial of \( \zeta \) over \( \mathbb Q \), and let \( g \) be the minimal polynomial of \( \zeta^p \) over \( \mathbb Q \).
	If \( f = g \), then \( \zeta^p \) lies in the orbit of \( \Gal(L/K) \) on \( \zeta \), so \( p \) lies in the image of \( \chi \) as required.
	Otherwise, \( f \) and \( g \) are coprime, and they divide \( T^n - 1 \) so \( fg \mid T^n - 1 \).
	As \( \zeta \) is a root of \( g(T^p) \), we have \( f \mid g(T^p) \).
	Reducing modulo \( p \), \( \overline f \in \mathbb F_p[T] \) divides \( \overline{g(T^p)} \in \mathbb F_p[T] \).
	But since we are in a field of characteristic \( p \), \( \overline{g(T^p)} = \overline g(T)^p \).
	Now, \( \overline f \) and \( \overline g \) divide \( T^n - 1 \) in \( \mathbb F_p[T] \), which is separable because \( p \nmid n \).
	So \( \overline f \mid \overline g^p \), so \( \overline f \mid \overline g \).
	But then \( \overline f^2 \mid \overline f \overline g \mid T^n - 1 \), contradicting separability of \( \overline f \).
\end{proof}
Therefore, the minimal polynomial of \( e^{\frac{2\pi i}{n}} \) over \( \mathbb Q \) is \( \Phi_n \).
