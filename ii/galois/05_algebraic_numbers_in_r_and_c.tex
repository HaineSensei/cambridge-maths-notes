\subsection{Examples and non-examples}
Traditionally, we call \( x \in \mathbb C \) algebraic if it is algebraic over \( \mathbb Q \), otherwise it is transcendental.
\( \overline{\mathbb Q} = \qty{x \mid x \text{ algebraic}} \) is a proper subfield of \( \mathbb C \).
Indeed, \( \mathbb Q[T] \) is a countable set, and \( \mathbb C \) is uncountable.
However, it is difficult to explicitly find an element of \( \mathbb C \setminus \overline{\mathbb Q} \), or to show that a given number is transcendental.
\begin{example}
	Liouville's constant \( c = \sum{n \geq 1} 10^{-n!} \) is transcendental (refer to IA Numbers and Sets).
	This can be proven by showing that algebraic numbers cannot be `well approximated' by rationals.
\end{example}
\begin{example}
	Hermite and Lindemann showed that \( e \) and \( \pi \) are transcendental.
\end{example}
\begin{example}
	Let \( x, y \) be algebraic, and \( x \neq 0,1 \).
	Gelfond and Schneider showed that \( x^y \) is algebraic if and only if \( y \) is rational.
	In particular, \( e^\pi = (-1)^{\frac{-i}{2}} \) is transcendental.
\end{example}
