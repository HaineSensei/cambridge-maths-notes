\subsection{Splitting fields}
Suppose \( K \) is a field and \( f \in K[T] \).
We wish to find an extension \( L / K \) of degree as small as possible such that \( f \) is expressible as a product of linear factors in \( L[T] \).
\begin{example}
	Let \( K = \mathbb Q \).
	Then by the fundamental theorem of algebra, a monic polynomial \( f \in \mathbb Q[T] \) is expressible as a product of \( n \) linear factors \( (T - x_i) \) in \( \mathbb C[T] \).
	One example of such a field extension is \( L = \mathbb Q(x_1, \dots, x_n) \), which is a finite extension of \( \mathbb Q \).
\end{example}
We will later give another proof of the fundamental theorem of algebra using techniques from Galois theory.
\begin{example}
	Let \( K = \mathbb F_p \), and \( f \) is irreducible and has degree \( d > 1 \).
	Since there is no ambient field structure, explicitly finding \( L \) is more challenging.
	We will first find an extension in which \( f \) has at least one root, and then use induction.
\end{example}
\begin{theorem}
	Let \( f \) be a monic irreducible polynomial.
	Let \( L_f = \faktor{K[T]}{(f)} \).
	Since \( f \) is irreducible, \( (f) \) is maximal, hence \( L_f \) is a field.
	Let \( t \in L_f \) be the residue class \( T \) modulo \( (f) \).
	Then \( L_f/K \) is a finite field extension of degree \( \deg f \), and \( f \) is the minimal polynomial for \( t \).
\end{theorem}
We have thus constructed a field extension of \( K \) for which \( f \) has at least a single root.
Recall that if \( x \) is algebraic over \( K \), then \( K(x) \cong \faktor{K[T]}{(f)} \) where \( f \) is minimal for \( x \).
\begin{definition}
	Let \( K \) be a field, and \( L / K, M / K \) are field extensions.
	A \emph{\( K \)-homomorphism} or \emph{\( K \)-embedding} from \( L \) to \( M \) is a field homomorphism \( \sigma \colon L \to M \) such that \( \eval{\sigma}_K = \mathrm{id}_K \).
\end{definition}
The naming `\( K \)-embedding' is justified because any field homomorphism is injective.
\begin{theorem}
	Let \( f \in K[T] \) be irreducible, and \( L / K \) a field extension.
	Then:
	\begin{enumerate}
		\item If \( x \in L \) is a root of \( f \), there exists a unique \( K \)-homomorphism \( \sigma \colon L_f = \faktor{K[T]}{(f)} \to L \) such that \( t = T + (f) \mapsto x \).
		\item Every \( K \)-homomorphism \( \sigma \colon L_f \to L \) arises in this way.
	\end{enumerate}
	Hence, we have a bijection between \( K \)-homomorphisms \( \sigma \colon L_f \to L \) and the set of roots of \( f \) in \( L \).
	In particular, there are at most \( \deg f \)-many \( K \)-homomorphisms.
\end{theorem}
\begin{proof}
	Let \( x \in L \) be a root of \( f \).
	Then \( f(x) = 0 \), so \( \mathrm{ev}_x(f) = 0 \) where \( \mathrm{ev}_x \colon K[T] \to L \) is the evaluation homomorphism \( g \mapsto g(x) \).
	Equivalently, \( \ker \mathrm{ev}_x = (f) \).
	Hence, by the isomorphism theorem, \( \mathrm{ev}_x \) comes from a homomorphism \( \sigma \colon \faktor{K[T]}{(f)} \to L \).
	Since the evaluation map is an identity on \( K \), this is a \( K \)-homomorphism as required.
\end{proof}
\begin{corollary}
	Let \( L = K(x) \) for some \( x \) algebraic over \( K \).
	Then there exists a unique isomorphism \( \sigma \colon L_f \to K(x) \) such that \( \sigma(t) = x \), where \( f \) is minimal for \( x \) over \( K \).
\end{corollary}
\begin{definition}
	Let \( x, y \) be algebraic over \( K \).
	We say \( x, y \) are \emph{\( K \)-conjugate} if they have the same minimal polynomial over \( K \).
\end{definition}
By the corollary above, \( K(x) \) and \( K(y) \) are isomorphic to \( L_f \) where \( f \) is minimal for \( x \) and \( y \) over \( K \).
\begin{corollary}
	Algebraic elements \( x, y \) are \( K \)-conjugate if and only if there exists a \( K \)-isomorphism \( \sigma \colon K(x) \to K(y) \) such that \( \sigma(x) = y \).
\end{corollary}
\begin{proof}
	The above corollary shows the forward direction.
	Conversely, for all \( g \in K[T] \), we have \( \sigma(g(x)) = g(\sigma(x)) \) so they have the same minimal polynomial.
\end{proof}
Informally, the roots of an irreducible polynomial are algebraically indistinguishable.

It can be useful for inductive arguments to have a generalisation of the above theorem.
\begin{definition}
	Let \( L / K, L' / K' \) be field extensions, and let \( \sigma \colon K \to K' \) be a field homomorphism.
	Let \( \tau \colon L \to L' \) be a field homomorphism such that \( \tau(x) = \sigma(x) \) for all \( x \in K \).
	Then we say \( \tau \) is a \emph{\( \sigma \)-homomorphism} from \( L \) to \( L' \).
	We also say \( \tau \) \emph{extends} \( \sigma \), or that \( \sigma \) is the \emph{restriction} of \( \tau \) to \( K \).
\end{definition}
We can now define the following variant of the previous theorem.
\begin{theorem}
	Let \( f \in K[T] \) be irreducible, and \( \sigma \colon K \to L \) be a field homomorphism.
	Let \( \sigma f \) be the polynomial obtained by applying \( \sigma \) to the coefficients of \( f \).
	\begin{enumerate}
		\item If \( x \in L \) is a root of \( f \), there exists a unique \( \sigma \)-homomorphism \( \tau \colon L_f \to L \) such that \( \tau(t) = x \).
		\item Every \( \sigma \)-homomorphism \( L_f \to L \) is of this form.
	\end{enumerate}
	Therefore there is a bijection between the \( \sigma \)-homomorphisms \( L_f \to L \) and the roots of \( f \) in \( L \).
\end{theorem}
\begin{example}
	Let \( K = \mathbb Q(\sqrt 2) \subset \mathbb R \), and \( L = \mathbb C \).
	Let \( \sigma \colon K \to L \) be the homomorphism such that \( \sigma(x+y\sqrt 2) = x-y\sqrt 2 \).
	Then let \( f = T^2 - (1 + \sqrt 2) \).
	Then the map \( \tau \colon L_f \to \mathbb C \) must satisfy \( \tau(t) = \pm \sqrt{1 - \sqrt 2} = \pm i \sqrt{\sqrt 2 - 1} \in \mathbb C \).
	If instead we let \( \sigma(x+y\sqrt 2) = x+y\sqrt 2 \), we have \( \tau(t) = \pm\sqrt{\sqrt 2 + 1} \), which are both real.
\end{example}

\subsection{Splitting completely}
\begin{definition}
	Let \( f \in K[T] \) be a nonzero polynomial that is not necessarily irreducible.
	We say that an extension \( L / K \) is a \emph{splitting field} for \( f \) over \( K \) if
	\begin{enumerate}
		\item \( f \) splits into linear factors in \( L[T] \);
		\item \( L = K(x_1, \dots, x_n) \), where the \( x_i \) are the roots of \( f \) in \( L \).
	\end{enumerate}
\end{definition}
\begin{remark}
	The second criterion ensures that \( f \) does not split into linear factors in any proper subfield of \( L \).
	Note that any splitting field is finite, because the adjoined elements are algebraic.
\end{remark}
\begin{theorem}
	Every nonzero polynomial has a splitting field.
\end{theorem}
\begin{proof}
	Let \( f \in K[T] \).
	We prove this by induction on the degree of \( f \), but allow \( K \) to vary.
	If \( f \) is constant, there is nothing to prove, since \( K \) is already a splitting field.
	Suppose that for all fields \( K' \) and all polynomials in \( K'[T] \) of degree less than \( f \), there is a splitting field.
	Consider an irreducible factor \( g \) of \( f \), and consider \( K' = L_g = \faktor{K[T]}{(g)} \).
	Let \( x_1 = T + (g) \).
	Then \( g(x_1) = 0 \), so \( f(x_1) = 0 \), hence \( f = (T - x_1)f_1 \), where \( f_1 \in K'[T] \).
	By induction, there exists a splitting field \( L \) for \( f_1 \) over \( K' \) since \( \deg f_1 < \deg f \).
	Let \( x_2, \dots, x_n \in L \) be the roots of \( f_1 \) in \( L \).
	Then \( f \) splits into linear factors in \( L \) with roots \( \qty{x_1, x_2, \dots, x_n} \).
	Because \( L \) is a splitting field for \( f_1 \) over \( K' \), we have \( L = K'(x_2, \dots, x_n) = K(x_1)(x_2, \dots, x_n) = K(x_1, \dots, x_n) \), so \( L \) is a splitting field for \( f \).
\end{proof}
\begin{remark}
	If \( K \subseteq \mathbb C \), we already know by the fundamental theorem of algebra that any polynomial over \( K \) has a subfield of \( \mathbb C \) as its splitting field.
\end{remark}

\subsection{Uniqueness of splitting fields}
\begin{theorem}
	Let \( f \in K[T] \) be a polynomial and \( L / K \) be a splitting field for \( f \).
	Then let \( \sigma \colon K \to M \) be a field homomorphism such that \( \sigma f \) splits in \( M[T] \).
	Then
	\begin{enumerate}
		\item \( \sigma \) can be extended to a homomorphism \( \tau \colon L \to M \);
		\item if \( M \) is a splitting field for \( \sigma f \) over \( \sigma K \), then any \( \tau \colon L \to M \) is an isomorphism.
	\end{enumerate}
	In particular, any two splitting fields are \( K \)-isomorphic.
\end{theorem}
\begin{remark}
	When constructing the splitting field for a polynomial, we had choice in which irreducible factors to consider first.
	It is not clear, without this theorem, that two splitting fields have the same degree.

	Note that we can have different \( \tau_1, \tau_2 \colon L \to M \) for splitting fields \( L, M \) of \( f \) over \( K \).
\end{remark}
\begin{proof}
	We will prove (i) by induction on \( [L : K] \).
	If \( n = 1 \), we have \( L = K \) and there is nothing to prove.
	Suppose \( x \in L \setminus K \) is a root of an irreducible factor \( g \) of \( f \) in \( K \), so \( \deg g > 1 \).
	Let \( y \in M \) be a root of \( \sigma g \in M[T] \), which exists because \( \sigma f \) splits in \( M \).
	Then, there exists \( \sigma_1 \colon K(x) \to M \) such that \( \sigma_1(x) = y \), and \( \sigma_1 \) extends \( \sigma \).
	Then, \( [L : K(x)] < [L : K] \), so by induction, \( \sigma_1 \colon K(x) \to M \) can be extended to \( \tau \colon L \to M \), because \( L \) is a splitting field for \( f \) over \( K(x) \).
	This \( \tau \) therefore extends \( \sigma \) as required.

	To prove (ii), suppose \( M \) is a splitting field for \( \sigma f \) over \( \sigma K \).
	Let \( \tau \) be as in (i), and \( \qty{x_i} \) be the roots of \( f \) in \( L \).
	Then the roots of \( \sigma f \) in \( M \) are \( \qty{\tau(x_i)} \).
	Since \( M \) is a splitting field, \( M = \sigma K(\qty{\tau(x_i)}) = \tau L \) as \( L = K(\qty{x_i}) \).
	So \( \tau \) is an isomorphism.

	If \( K \subseteq M \) and \( \sigma \) is the inclusion homomorphism, \( \tau \) is a \( K \)-isomorphism.
\end{proof}
\begin{example}
	Let \( f = T^3 - 2 \in \mathbb Q[T] \).
	This has splitting field \( L = \mathbb Q(\sqrt[3]{2}, \omega) \subseteq \mathbb C \) where \( \omega = e^{\frac{2\pi i}{3}} \).
	We know \( [\mathbb Q(\sqrt[3]{2}) : \mathbb Q] = 3 \), but \( \omega \not\in \mathbb R \) and \( \omega^2 + \omega + 1 = 0 \), so \( [L : \mathbb Q(\sqrt[2]{3})] = 2 \) giving \( [L : \mathbb Q] = 6 \) by the tower law.
	In particular, adjoining a single root to \( \mathbb Q \) is not enough to generate \( L \).
\end{example}
\begin{example}
	Let \( f = \frac{T^5 - 1}{T - 1} = T^4 + \dots + T + 1 \in \mathbb Q[T] \).
	Let \( z = e^{\frac{2\pi i}{5}} \), then this is the minimal polynomial of \( z \).
	We find \( f = \prod_{1 \leq a \leq 4} (T - z^a) \), so \( \mathbb Q(z) \) is already a splitting field for \( f \) over \( \mathbb Q \), and \( [\mathbb Q(z) : \mathbb Q] = 4 \).
\end{example}
\begin{example}
	Let \( f = T^3 - 2 \in \mathbb F_7[T] \).
	This is irreducible because 2 is not a cube in \( \mathbb F_7 \).
	Consider \( L = \faktor{\mathbb F_7[X]}{X^3 - 2} = \mathbb F_7(x) \), so \( x^3 = 2 \).
	Since \( 2^3 = 4^3 = 1 \) in \( \mathbb F_7 \), we have \( (2x)^3 = (4x)^3 = 2 \), so \( x, 2x, 4x \) are roots of \( f \) in \( L \).
	In particular, \( L \) is a splitting field for \( f \), since \( f = (T - x)(T - 2x)(T - 4x) \); here, adjoining one root is enough to make \( f \) split.
\end{example}

\subsection{Normal extensions}
\begin{definition}
	An extension \( L / K \) is a \emph{normal extension} if it is algebraic and for all \( x \in L \), the minimal polynomial splits in \( L \).
\end{definition}
\begin{remark}
	This condition is equivalent to the statement that for every \( x \in L \), \( L \) contains a splitting field for \( x \).
	In other words, if an irreducible polynomial \( f \in K[T] \) has a single root in \( L \), it splits and has all roots in \( L \).
\end{remark}
\begin{theorem}
	Let \( L / K \) be a finite extension.
	Then \( L \) is normal over \( K \) if and only if \( L \) is a splitting field for some (not necessarily irreducible) polynomial \( f \in K[T] \).
\end{theorem}
\begin{proof}
	Suppose \( L \) is normal.
	Then \( L = K(x_1, \dots, x_n) \) since \( L \) is algebraic.
	Then the minimal polynomial \( m_{x_i,K} \) of each \( x_i \) over \( K \) splits in \( L \).
	\( L \) is generated by the roots of \( \prod_i m_{x_i,K} \), so \( L \) is a splitting field for \( f \).

	For the converse, suppose \( L \) is a splitting field for \( f \in K[T] \).
	Let \( x \in L \), and let \( g = m_{x,K} \) be its minimal polynomial.
	We want to show that \( g \) splits in \( L \).
	Let \( M \) be a splitting field for \( g \) over \( L \), and let \( y \in M \) be a root of \( g \).
	We want to show \( y \in L \).

	Since \( L \) is a splitting field for \( f \) over \( K \), \( L \) is a splitting field for \( f \) over \( K(x) \), and \( L(y) \) is a splitting field for \( f \) over \( K(y) \).
	Now, there exists a \( K \)-isomorphism between \( K(x) \) and \( K(y) \), because \( x, y \) are roots of the same irreducible polynomial \( g \).
	By the uniqueness of splitting fields, \( [L:K(x)] = [L(y):K(y)] \).
	Multiplying by \( [K(x):K] \), we find \( [L:K] = [L(y):K] \) because \( [K(y):K] = [K(x):K] \) as they are roots of the same irreducible polynomial.
	Hence \( [L(y):L] = 1 \), so \( y \in L \) as required.
\end{proof}
\begin{corollary}[normal closure]
	Let \( L / K \) be a finite extension.
	Then there exists a finite extension \( M / L \) such that \( M / K \) is normal, and if \( L \subseteq M' \subseteq M \) and \( M' / K \) is normal, \( M = M' \).
	Moreover, any two such extensions \( M \) are \( L \)-isomorphic.
\end{corollary}
Such an \( M \) is said to be a \emph{normal closure} of \( L / K \).
\begin{proof}
	Let \( L = K(x_1, \dots, x_k) \), and \( f = \prod_{i=1}^k m_{x_i,K} \in K[T] \).
	Then let \( M \) be a splitting field of \( f \) over \( L \).
	Then, since the \( x_i \) are roots of \( f \), \( M \) is also a splitting field for \( f \) over \( K \).
	So \( M / K \) is normal.

	Let \( M' \) be such that \( L \subseteq M' \subseteq M \) and \( M' / K \) be normal.
	Then as \( x_i \in M' \), the minimal polynomial \( m_{x_i,K} \) splits in \( M' \).
	So \( M' = M \).

	Any normal extension \( M / K \) must contain a splitting field for \( f \), and by the minimality condition, \( M \) must be a splitting field.
	By uniqueness of splitting fields, any two such extensions are \( L \)-isomorphic as required.
\end{proof}

\subsection{Separable polynomials}
Recall that over \( \mathbb C \), a polynomial has a multiple zero by considering its derivative.
Over arbitrary fields, the same is true, but the analytic concept of derivative must be replaced with an algebraic process.
\begin{definition}
	The \emph{formal derivative} of a polynomial \( f(T) = \sum_{i=0}^d a_i X^i \) is
	\[ f'(T) = \sum_{i=1}^d i a_i X^{i-1} \]
\end{definition}
\begin{remark}
	One can check from the definition that the familiar rules \( (f + g)' = f' + g' \), \( (fg)' = f'g + fg' \), and \( (f^n)' = nf'f^{n-1} \) hold.
\end{remark}
\begin{example}
	Consider a field \( K \) of characteristic \( p > 0 \), and let \( f = T^p + a_0 \).
	Then \( f' = 0 \), so a non-constant polynomial can have a zero derivative.
\end{example}
\begin{proposition}
	Let \( f \in K[T] \), \( L / K \) be a field extension, and \( x \in L \) a root of \( f \).
	Then \( x \) is a simple root if and only if \( f'(x) \neq 0 \).
\end{proposition}
\begin{proof}
	We can write \( f = (T-x)g \in L[T] \).
	Then \( f' = g + (T-x)g' \), so \( f'(x) = g(x) \).
	In particular, \( g(x) \neq 0 \) if and only if \( (T-x) \) does not divide \( g \), which is the criterion that \( x \) is a simple root of \( f \).
\end{proof}
\begin{definition}
	A polynomial \( f \in K[T] \) is \emph{separable} if it splits into distinct linear factors in a splitting field.
	Equivalently, it has \( \deg f \) distinct roots.
\end{definition}
\begin{corollary}
	\( f \) is separable if and only if the greatest common divisor of \( f \) and \( f' \) is \( 1 \).
\end{corollary}
For convenience, we will take \( \gcd(f, g) \) to be the unique monic polynomial \( h \) such that \( (h) = (f, g) \).
Then since \( K[T] \) is a Euclidean domain, we can compute a representation \( h = af + bg \) for polynomials \( a, b \).
Note that \( \gcd(f, g) \) is invariant under a field extension, because Euclid's algorithm does not depend on the ambient field structure.
\begin{proof}
	We can replace \( K \) by a splitting field of \( f \), so we can factorise \( f \) into a product of linear factors in \( K \).
	The two are separable if \( f, f' \) have no common root, which is true if and only if \( \gcd(f, f') = 1 \).
\end{proof}
\begin{example}
	Let \( K \) have characteristic \( p > 0 \), and let \( f = T^p - b \) for \( b \in K \).
	Then \( f' = 0 \), so \( \gcd(f, f') = f \neq 1 \).
	Hence \( f \) is inseparable.
	Let \( L \) be an extension of \( K \) containing a \( p \)th root \( a \in L \) of \( b \), so \( a^p = b \).
	Then \( f = (T - a)^p = T^p + (-a)^p = T^p - b \).
	In particular, \( f \) has only one root in a splitting field.

	If \( b \) is not a \( p \)th power in \( K \), then \( f \) is irreducible.
	This is seen on the example sheets.
\end{example}
\begin{theorem}
	Let \( f \in K[T] \) be an irreducible polynomial.
	Then \( f \) is separable if and only if \( f' \neq 0 \).

	In addition, if \( K \) has characteristic zero, every irreducible polynomial \( f \in K[T] \) is separable.
	If \( K \) has positive characteristic \( p > 0 \), an irreducible polynomial \( f \in K[T] \) is inseparable if and only if \( f(T) = g(T^p) \) for some \( g \in K[T] \).
\end{theorem}
\begin{proof}
	Without loss of generality, we can assume \( f \) is monic.
	Then, since \( f \) is irreducible, the greatest common divisor \( \gcd(f,f') \) is either \( f \) or \( 1 \).
	If \( \gcd(f,f') = f \), then \( f' = 0 \) by considering the degree.

	For a polynomial \( f \), we can write \( f = \sum_{i=0}^d a_i T^i \) and \( f' = \sum_{i=1}^d i a_i T^{i-1} \), so \( f' = 0 \) if and only if \( i a_i = 0 \) for all \( 1 \leq i \leq d \).
	In particular, if \( K \) has characteristic zero, this is true if and only if \( a_i = 0 \) for all \( 1 \leq i \leq d \), so \( f = a_0 \) is a constant so not irreducible.
	If \( K \) has characteristic \( p > 0 \), the requirement is that \( a_i = 0 \) for all \( i \) not divisible by \( p \), or equivalently, \( f(T) = g(T^p) \).
\end{proof}

\subsection{Separable extensions}
\begin{definition}
	Let \( L / K \) be a field extension.
	We say \( x \in L \) is \emph{separable} over \( K \) if \( x \) is algebraic and its minimal polynomial \( f \) is separable over \( K \).
	\( L \) is \emph{separable} over \( K \) if all elements \( x \) are separable over \( K \).
\end{definition}
\begin{theorem}
	Let \( x \) be algebraic over \( K \), and \( L / K \) be an extension in which the minimal polynomial \( m_{x,K} \) splits.
	Then \( x \) is separable over \( K \) if and only if there are exactly \( \deg x \) \( K \)-homomorphisms from \( K(x) \) to \( L \).
\end{theorem}
\begin{proof}
	The number of \( K \)-homomorphisms from \( K(x) \) to \( L \) is the number of roots of \( m_{x,K} \) in \( L \).
	This is equal to the degree of \( x \) if and only if \( x \) is separable.
\end{proof}
Let \( \mathrm{Hom}_K(L,M) \) be the set of \( K \)-homomorphisms from \( L \) to \( M \).
Note that not all \( K \)-linear maps from \( L \) to \( M \) are \( K \)-homomorphisms.
\begin{theorem}[counting embeddings]
	Let \( L = K(x_1, \dots, x_k) \) be a finite extension of \( K \), so the \( x_i \) are algebraic.
	Let \( M / K \) be any field extension.
	Then \( \abs{\mathrm{Hom}_K(L,M)} \leq [L : K] \), with equality if and only if
	\begin{enumerate}
		\item for all \( i \), the minimal polynomial \( m_{x_i,K} \) splits into linear factors in \( M \);
		\item all the \( x_i \) are separable over \( K \).
	\end{enumerate}
\end{theorem}
\begin{remark}
	The conditions (i) and (ii) are equivalent to the statement that \( m_{x_i,K} \) split into distinct linear factors over \( M \).
	There is a variant of this theorem: let \( \sigma : K \to M \) be a field homomorphism, then \( \abs{\mathrm{Hom}_\sigma(L,M)} \leq \abs{L:K} \), and equality holds if and only if the \( \sigma m_{x_i,K} \) split into distinct linear factors over \( M \).
\end{remark}
\begin{proof}
	We prove this by induction on \( k \).
	The case \( k = 0 \) is trivial.
	Let \( K_1 = K(x_1) \) and write \( d = \deg_K x_1 = [K_1 : K] \).
	Then the number of \( K \)-homomorphisms from \( K_1 \) to \( M \), denoted \( e = \abs{\mathrm{Hom}_K(K_1,M)} \), is the number of roots of \( m_{x_1,K} \) in \( M \).
	Let \( \sigma : K_1 \to M \) be a \( K \)-homomorphism.
	By the inductive hypothesis, there exist at most \( [L : K_1] \) extensions of \( \sigma \) to a \( K \)-homomorphism \( L \to M \).
	Hence the number of \( K \)-homomorphisms from \( L \) to \( M \) is at most \( e[L : K_1] \leq d[L : K_1] = [L : K] \).

	If equality holds, then \( e = d \), and so \( m_{x_1,K} \) splits into \( d \) distinct linear factors in \( M \), so (i) and (ii) hold for \( x_1 \).
	Replacing \( x_1 \) with an arbitrary \( x_i \), one implication follows.
	Conversely, suppose conditions (i) and (ii) hold.
	Then, by the previous theorem, there are \( d \) distinct homomorphisms from \( K_1 \) to \( M \).
	Conditions (i) and (ii) still hold over \( K_1 \), then by induction, each \( \sigma \colon K_1 \to M \) has \( [L : K_1] \) extensions to a homomorphism \( L \to M \).
	Hence \( \abs{\mathrm{Hom}_K(L,M)} = [L : K] \) as required.
\end{proof}
\begin{theorem}[separably generated implies separable]
	Let \( L = K(x_1, \dots, x_k) \) be a finite extension of \( K \).
	Then \( L / K \) is a separable extension if and only if each \( x_i \) is separable over \( K \).
\end{theorem}
\begin{proof}
	If \( L / K \) is separable, the \( x_i \) are separable by definition.
	Suppose the \( x_i \) are separable.
	Let \( M \) be a normal closure of \( L / K \), so the splitting field of the product of the \( m_{x_i,K} \) over \( L \).
	By the counting embeddings theorem, conditions (i) and (ii) are satisfied so \( \abs{\mathrm{Hom}_K(L,M)} = [L : K] \).
	But if \( x \in L \), \( L = K(x, x_1, \dots, x_k) \), so \( x \) is separable.
\end{proof}
\begin{corollary}
	Let \( x, y \in L \), and \( L / K \) a field extension.
	If \( x, y \) are separable over \( K \), so are \( x + y, xy, x^{-1} \) for \( x \neq 0 \).
\end{corollary}
\begin{proof}
	Consider the fields \( K(x,y) \) and \( K(x) \).
	These are separable extensions of \( K \).
	In particular, \( \qty{x \in L \mid x \text{ separable over } K} \) is a subfield of \( L \).
\end{proof}
\begin{theorem}[primitive element theorem for separable extensions]
	Let \( K \) be an infinite field and \( L = K(x_1, \dots, x_k) \) be a finite separable extension.
	Then there exists \( x \in L \) such that \( L = K(x) \).
	In particular, \( x \) is separable over \( K \).
\end{theorem}
\begin{proof}
	It suffices to consider the case when \( k = 2 \), because if we can turn \( K(x,y) \) into \( K(z) \) for \( z \in K(x,y) \), we can perform this inductively.
	Let \( L = K(x,y) \) with \( x, y \) separable over \( K \).
	Let \( n = [L : K] \), and let \( M \) be a normal closure for \( L / K \).
	Then there exist \( n \) distinct \( K \)-homomorphisms \( \sigma_i \colon L \to M \).
	Let \( a \in K \), and consider \( z = x + ay \).
	We will choose \( a \) such that \( L = K(z) \).

	Since \( L = K(x,y) \), we have \( \sigma_i(x) = \sigma_j(x) \) and \( \sigma_i(y) = \sigma_j(y) \) implies \( i = j \).
	Consider \( \sigma_i(z) = \sigma_i(x) + a \sigma_i(y) \).
	If \( \sigma_i(z) = \sigma_j(z) \), we must have \( \qty(\sigma_i(x) - \sigma_j(x)) + a \qty(\sigma_i(y) - \sigma_j(y)) = 0 \).
	If \( i \neq j \), at least one of the parenthesised terms is nonzero.
	Therefore there is at most one \( a \in K \) such that \( \sigma_i(z) = \sigma_j(z) \).
	Since \( K \) is infinite, there exists \( a \in K \) such that all of the \( \sigma_i(z) \) are distinct.
	But then \( \deg_K z = n \), so \( L = K(z) \).
\end{proof}
\begin{theorem}
	Let \( L / K \) be an extension of finite fields.
	Then \( L = K(x) \) for some \( x \in L \).
\end{theorem}
\begin{proof}
	The multiplicative group \( L^\times \) is cyclic.
	Let \( x \) be a generator of this group.
	Then \( L = K(x) \), since every nonzero element is a power of \( x \).
\end{proof}
