\subsection{Motivation}
In classical computation, the elementary unit of information is the \emph{bit}, which takes a value in \( \qty{0,1} \).
This gives the result of a single binary decision problem, where the zero and one correspond to different answers to the problem.
Binary strings of length greater than one are used to provide more than 2 answers to a problem; if we have \( n \) bits, we can encode \( 2^n \) different messages.

Classical computation is understood to be the processing of information: taking an initial bit string and and updating it by a prescribed sequence of steps.
The steps are taken to be the action of local Boolean logic gates, such as conjunction, disjunction, or negation.
At each step, a small number of bits in prescribed locations are edited.

Information in the real world must be tied to a physical representation.
For example, bits in a processor are often represented by different voltages of specific components.
Importantly, there is no information \emph{without} representation.
Performing a computation classically must therefore involve the evolution of a physical system over time, which is coverned by the laws of classical physics.

However, nature does not abide by classical physics at subatomic levels, and we must use quantum mechanics to accurately model such behaviours.
One such behaviour modelled by quantum mechanics is the superposition principle, that the corresponding quantum analog of the bit need not be in precisely one state.
Quantum entanglement is the phenomenon where particles can be linked in such a way that their states can be manipulated even at a distance.
Quantum measurement is probabilistic and alters the underlying system.

Quantum information and computation therefore exploits these features of quantum mechanics to address issues of information storage, communication, computation, and cryptography.
The features of quantum mechanics seem to allow us benefits which are beyond the limits of classical information and computation, even in principle.
Note that a quantum computer cannot perform any task that cannot in principle be performed classically.
We only hope that quantum techniques allow a reduction in the complexity of certain algorithms.

\subsection{Benefits of quantum information and computation}
In complexity theory, we study the \emph{hardness} of a certain computational task.
One must consider the resources required for the task; which in classical computation are normally limited to time (measured in number of computational steps) and space (amount of memory required).

If an algorithm takes time bounded by a polynomial function in the input size \( n \), we say the algorithm is \emph{polynomial-time}.
Otherwise, we say it is an \emph{exponential-time} algorithm.
Polynomial-time algorithms are typically taken to be computable in practice, but exponential-time algorithms are usually considered only computable in principle.
Quantum mechanical techniques can provide polynomial-time algorithms that have only exponential-time classical versions.
One example is Shor's integer factorisation algorithm.

Quantum states of physical sytems can be used to encode information, such as spin states of electrons.
There are certain tasks possible with such quantum states which are impossible in classical physics; one example is quantum teleportation.

There are also some technological issues with classical physics.
Components of processors have become minified to atomic scale, and therefore they cannot be shrunk much further without dealing with the effects of quantum mechanics.
Conversely, there are technological challenges with quantum physics.
Quantum systems are very fragile, and modern quantum computers typically require temperatures close to absolute zero to reduce noise.

\emph{Quantum supremacy} refers to the hypothetical moment at which a programmable quantum computer can first solve a problem in practice that a classical computer cannot.
At the time of writing, there is no concensus that quantum supremacy has been achieved.

\subsection{Hilbert spaces}
Every quantum mechanical system is associated with a Hilbert space \( \mathcal V \), a complex inner product space that is a complete metric space with respect to the distance function induced by the inner product.
We use Dirac's \emph{bra-ket} notation: a vector is represented by \( \ket{v} \in \mathcal V \), and its conjugate transpose is denoted \( \bra{v} \in \mathcal V^\star \).
If \( \mathcal V = \mathbb C^n \), we write
\[ \ket{\psi} = \begin{pmatrix}
    a_1 \\
    \vdots \\
    a_n
\end{pmatrix};\quad \bra{\psi} = \begin{pmatrix}
    a_1^\star & \cdots & a_n^\star
\end{pmatrix} \]
The inner product of \( \psi \) and \( \phi \) is written \( \bra{\psi}\ket{\phi} \).
Recall that an inner product satisfies
\begin{itemize}
    \item \( \bra{\psi}\ket{\psi} \geq 0 \), and equal to zero if and only if \( \ket{\psi} = 0 \);
    \item linearity in the second argument, so \( \bra{\psi}\ket{a\phi_1 + b\phi_2} = a \bra{\psi}\ket{\phi_1} + b \bra{\psi}\ket{\phi_2} \);
    \item antilinearity in the first argument, so \( \bra{a\psi_1 + b\psi_2}\ket{\phi} = a^\star \bra{\psi_1}\ket{\phi} + b^\star \bra{\psi_2}\ket{\phi} \);
    \item skew-symmetry, so \( \bra{\psi}\ket{\phi}^\star = \bra{\phi}\ket{\psi} \);
\end{itemize}
and induces a norm \( \norm{\psi} = \norm{\ket{\psi}} = \sqrt{\bra{\psi}\ket{\psi}} \).
In this course, we will often consider \( \mathcal V = \mathbb C^2 \) and define
\[ \ket{0} = \begin{pmatrix}
    1 \\ 0
\end{pmatrix};\quad \ket{1} = \begin{pmatrix}
    0 \\ 1
\end{pmatrix} \]
For an arbitrary \( \ket{v} \in \mathbb C^2 \), we can write \( \ket{v} = a \ket{0} + b \ket{1} \), giving
\[ \ket{v} = \begin{pmatrix}
    a \\ b
\end{pmatrix};\quad \bra{v} = \begin{pmatrix}
    a^\star & b^\star
\end{pmatrix} \]
If \( \ket{w} = c \ket{0} + d \ket{1} \), then \( \ip{v}{w} = a^\star c + b^\star d \).

We can also compute the \emph{outer product} of two vectors, defined to be \( \op{\psi}{\phi} \).
If \( \mathcal V = \mathbb C^n \), the outer product is an \( n \times n \) matrix.
An orthonormal basis \( (\ket{i})_{i=1}^n \) for \( \mathcal V \) is called \emph{complete} if \( \sum_{i=1}^n \op{i}{i} \) is the identity matrix.

If \( \mathcal V \) has a complete orthonormal basis, we can write \( \ket{\psi} = \sum_{i=1}^n c_i \ket{i} \) for some \( c_i \).
If \( \ip{\psi}{\psi} = 1 \), we say \( \ket{\psi} \) is \emph{normalised}.
In this case, \( \sum \abs{c_i}^2 = 1 \), and the \( \abs{c_i}^2 \) form a discrete probability distribution.
We call the \( c_i \) the \emph{probability amplitudes}.

Let \( \mathcal V, \mathcal W \) be vector spaces, where \( \dim \mathcal V = n, \dim \mathcal W = m \).
Let \( \ket{v} \in \mathcal V, \ket{w} \in \mathcal W \).
Suppose \( \ket{v} = \begin{pmatrix}
    a_1 & \cdots & a_n
\end{pmatrix}^\transpose \), and \( \ket{w} = \begin{pmatrix}
    b_1 & \cdots & b_m
\end{pmatrix}^\transpose \).
Then, \( \ket{v} \otimes \ket{w} \) is the \emph{tensor product} of \( \ket{v} \) and \( \ket{w} \), defined by
\[ \ket{v} \otimes \ket{w} = \begin{pmatrix}
    a_1 b_1 \\
    \vdots \\
    a_1 b_m \\
    a_2 b_1 \\
    \vdots \\
    a_n b_m
\end{pmatrix} \in \mathcal V \otimes \mathcal W \]
If \( \qty(\ket{e_i})_{i=1}^n \) is a complete orthonormal basis for \( \mathcal V \) and \( \qty(\ket{f_j})_{j=1}^m \) is a complete orthonormal basis for \( \mathcal W \), then \( \qty(\ket{e_i} \otimes \ket{f_j})_{i,j=1}^{n,m} \) is a complete orthonormal basis for \( \mathcal V \otimes \mathcal W \).
We sometimes write \( \ket{v} \otimes \ket{w} \) as \( \ket{v} \ket{w} \) or \( \ket{vw} \).

If \( \ket{\alpha} \in \mathcal V \), we can write \( \ket{\alpha} = \sum a_i \ket{e_i} \), and similarly if \( \ket{\beta} \in \mathcal W \), we can write \( \ket{\beta} = \sum b_j \ket{f_j} \).
Then, \( \ket{\alpha\beta} = \sum a_i c_j \ket{e_i f_j} \).

We say \( \ket{\Psi} \in \mathcal V \otimes \mathcal W \) is a \emph{product vector} if \( \ket{\Psi} = \ket{\psi}\otimes\ket{\phi} \) for some \( \psi, \phi \).
Vectors that are not product vectors are called \emph{entangled vectors}.

Let \( \mathcal V = \mathbb C^2 = \mathcal W \).
Define \( \ket{\phi^+} = \frac{1}{\sqrt{2}} \qty(\ket{00} + \ket{11}) \).
Suppose \( \ket{\phi^+} = \ket{\psi}\otimes\ket{\phi} = (a\ket{0} + b\ket{1}) \otimes (c\ket{0} + d\ket{1}) \).
Then, \( \ket{\phi^+} = ac \ket{00} + ad \ket{01} + bc \ket{10} + bd \ket{11} \).
So one of \( a \) and \( d \), and one of \( b \) and \( c \) is equal to zero, contradicting the assumption, so \( \ket{\phi^+} \) is entangled.

We define the inner product on the product space by defining
\[ \ip{\phi_1}{\psi_2} = \qty(\bra{\alpha_1} \bra{\beta_1})\qty(\ket{\beta_2} \ket{\alpha_2}) = \ip{\alpha_1}{\alpha_2} \ip{\beta_1}{\beta_2} \]
where \( \ket{\psi_i} = \ket{\alpha_i}\ket{\beta_i} \).
In the general case, \( \ket{A} = \sum a_{ij} \ket{e_i}\ket{f_j}, \ket{B} = \sum b_{ij} \ket{e_i}\ket{f_j} \), and we define
\[ \ip{A}{B} = \qty(\sum a_{ij}^\star \bra{e_i} \bra{f_j}) \qty(\sum b_{ij} \ket{e_i} \ket{f_j}) = \sum a_{ij}^\star b_{ij} \delta_{ii'}\delta_{jj'} = \sum a_{ij}^\star b_{ij} \]
where \( \delta \) is the Kronecker \( \delta \) symbol.

We define the \( k \)-fold \emph{tensor power} of a vector space \( \mathcal V \) by
\[ \mathcal V^{\otimes n} = \underbrace{\mathcal V \otimes \dots \otimes \mathcal V}_{n \text{ times}} \]
If \( \mathcal V = \mathbb C^2 \), this has dimension \( 2^k \), and complete orthonormal basis \( \ket{i_1 \dots i_k} \) for \( i_j \in \qty{0,1} \).
Note that \( \ket{v}\ket{w} \neq \ket{w}\ket{v} \).

\subsection{First postulate: quantum states}
In this course, we will restrict our attention to finite-dimensional vector spaces, and finite time evolution.
We describe the \emph{postulates} for quantum mechanics that we will work under.

The first postulate is that, given an isolated quantum mechanical system \( S \), we can associate a finite-dimensional vector space \( \mathcal V \).
The physical state of the system is given by a unit vector \( \ket{\psi} \) in \( \mathcal V \).
More precisely, the state is given by a \emph{ray}, an equivalence class of vectors \( e^{i\theta} \ket{\psi} \) for \( \theta \in \mathbb R \).
No measurements can distinguish states in a given equivalence class.
Note that states \( a\ket{\psi_1} + b\ket{\psi_2} \) and \( a\ket{\psi_1} + be^{i\theta}\ket{\psi_2} \) can be distinguished by measurement, since the phase difference is relative, not global.
\begin{example}
    Let \( \mathcal V = \mathbb C^2 \) with (complete orthonormal) basis \( \ket{0}, \ket{1} \).
    The elementary unit of quantum information is known as the \emph{qubit}, which is any quantum system with \( \mathcal V = \mathbb C^2 \).
    The spin of an electron, which is some superposition of spin-up and spin-down, can be modelled by \( \mathbb C^2 \).
    A property of the polarisation of a photon, such as vertical or horizontal, or right-circular or left-circular, can also be modelled in this way.

    Define \( \ket{+} = \frac{1}{\sqrt{2}}\qty(\ket{0} + \ket{1}) \) and \( \ket{-} = \frac{1}{\sqrt{2}}\qty(\ket{0} - \ket{1}) \).
    This is another complete orthonormal basis for \( \mathcal V \), sometimes called the \emph{conjugate basis}.
\end{example}

\subsection{Second postulate: composite systems}
The second postulate of quantum mechanics is that two quantum systems \( S_1, S_2 \) with associated vector spaces \( \mathcal V_1, \mathcal V_2 \) can be composed into the \emph{composite system} with vector space \( \mathcal V_1 \otimes \mathcal V_2 \).
\begin{example}
    Consider \( \mathcal V^{\otimes n} \), the space of \( n \) qubits.
    An orthonormal basis is \( \ket{i_1 \dots i_n} \) where \( i_j \in \qty{0,1} \).
    A vector in \( \mathcal V^{\otimes n} \) can be written \( \sum a_{i_1 \dots i_n} \ket{i_1 \dots i_n} \).
    There are \( 2^n \) different amplitudes \( a_{i_1 \dots i_n} \), providing exponential growth in information.
    However, in a product state, we obtain only linear growth in information.
\end{example}

\subsection{Observables}
An \emph{observable} is a property of a physical system which can, in theory, be measured.
Mathematically, these are modelled by linear self-adjoint (or Hermitian) operators.

The action of a linear operator \( A \) on a state space \( \mathcal V \) is a written \( A \ket{\psi} \).
Note that \( A \qty(a \ket{\psi} + b \ket{\phi}) = a A \ket{\psi} + b A \ket{\phi} \) for \( a, b \in \mathbb C \).
For any operator \( A \) acting on \( \mathcal V \), there is a unique linear operator \( A^\dagger \) such that \( \ip{v}{Aw} = \ip{A^\dagger v}{w} \), called the \emph{adjoint} of \( A \); operators equal to their adjoints are called \emph{self-adjoint}.

We can easily show that \( (AB)^\dagger = B^\dagger A^\dagger \).
By convention, we define \( \ket{\psi}^\dagger = \bra{\psi} \), so for a self-adjoint operator \( A \), we have \( \qty(A \ket{\psi})^\dagger = \bra{\psi} A \).
There are four important operators which act on the single-qubit space \( \mathbb C^2 \).
\[ \sigma_0 = \begin{pmatrix}
    1 & 0 \\
    0 & 1
\end{pmatrix};\quad \sigma_x = \begin{pmatrix}
    0 & 1 \\
    1 & 0
\end{pmatrix};\quad \sigma_y = \begin{pmatrix}
    0 & -i \\
    i & 0
\end{pmatrix};\quad \sigma_z = \begin{pmatrix}
    1 & 0 \\
    0 & -1
\end{pmatrix} \]
\( \sigma_0 \) is the identity matrix, and \( \sigma_x, \sigma_y, \sigma_z \) are called the \emph{Pauli matrices}.
The actions of these matrices on the basis vectors \( \ket{0} \) and \( \ket{1} \) are
\[ \sigma_0\ket{0} = \ket{0};\quad \sigma_0\ket{1} = \ket{1};\quad \sigma_x\ket{0} = \ket{1};\quad \sigma_x\ket{1} = \ket{0}; \]
\[ \sigma_y\ket{0} = i\ket{1};\quad \sigma_y\ket{1} = -i\ket{0};\quad \sigma_z\ket{0} = \ket{0};\quad \sigma_z\ket{1} = -\ket{1} \]
Note that
\[ \sigma_x \sigma_y = i \sigma_z;\quad \sigma_y \sigma_z = i \sigma_x;\quad \sigma_z \sigma_x = i\sigma_y \]
Intuitively, \( \sigma_x \) is a bit flip, \( \sigma_y \) is a phase flip, and \( \sigma_z \) is a combined bit and phase flip.

\subsection{Dirac notation for linear operators}
Let \( \ket{v} = a \ket{0} + b \ket{1} \), and \( \ket{w} = c \ket{0} + d \ket{1} \).
The outer product is
\[ M = \op{v}{w} = \begin{pmatrix}
    a \\ b
\end{pmatrix} \begin{pmatrix}
    c^\star & d^\star
\end{pmatrix} = \begin{pmatrix}
    ac^\star & ad^\star \\
    bc^\star & bd^\star
\end{pmatrix} \]
which is a linear map on \( \mathcal V = \mathbb C^2 \).
One can show that \( M \ket{x} = \qty(\op{v}{w})\ket{x} = \ket{v} \ip{w}{x} \), which is the scalar product of the vector \( \ket{v} \) with the inner product \( \ip{w}{x} \).
Such outer products yield the linear maps from \( \mathbb C^2 \) to \( \mathbb C^2 \) that have rank 1, and the kernel of \( M \) is the subspace of vectors orthogonal to \( \ket{w} \).
Note that
\[ \op{0}{0} = \begin{pmatrix}
    1 & 0 \\
    0 & 0
\end{pmatrix};\quad \op{0}{1} = \begin{pmatrix}
    0 & 1 \\
    0 & 0
\end{pmatrix};\quad \op{1}{0} = \begin{pmatrix}
    0 & 0 \\
    1 & 0
\end{pmatrix};\quad \op{1}{1} = \begin{pmatrix}
    0 & 0 \\
    0 & 1
\end{pmatrix} \]
Hence, we can write
\[ A = \begin{pmatrix}
    a & b \\
    c & d
\end{pmatrix} \implies A = a \op{0}{0} + b \op{0}{1} + c \op{1}{0} + d \op{1}{1} \]
In particular, \( \op{0}{0}, \op{0}{1}, \op{1}{0}, \op{1}{1} \) forms a basis for the vector space \( \mathcal V \otimes \mathcal V^\star \) of linear maps on \( \mathcal V \).
Note also that \( \ip{w}{v} = \tr \op{v}{w} \).

\subsection{Projection operators}
Suppose that \( \ket{v} \) is a normalised vector, so \( \ip{v}{v} = 1 \).
Then, \( \Pi_v = \op{v}{v} \) is the \emph{projection operator} onto \( v \), satisfying \( \Pi_v \Pi_v = \Pi_v \) and \( \Pi_v^\dagger = \Pi_v \).
In Dirac notation, one can see that
\[ \Pi_v \Pi_v = \op{v}{v} \op{v}{v} = \ket{v} \ip{v}{v} \bra{v} = \op{v}{v} = \Pi_v \]
If \( \ket{a} \) is orthogonal to \( \ket{v} \), then \( \Pi_v \ket{a} = \ket{v} \ip{v}{a} = 0 \).
Therefore, \( \Pi_v \ket{x} \) is the vector obtained by projection of \( \ket{x} \) onto the one-dimensional subspace of \( \mathcal V \) spanned by \( \ket{v} \).

Now suppose \( \mathcal E \) is any linear subspace of some vector space \( \mathcal V \), and \( \ket{e_1}, \dots, \ket{e_d} \) is any orthonormal basis of \( \mathcal E \).
Then,
\[ \Pi_{\mathcal E} = \op{e_1}{e_1} + \dots + \op{e_d}{e_d} \]
is the projection operator into \( \mathcal E \).
This property can be checked by extending \( \ket{e_1}, \dots, \ket{e_d} \) into an orthonormal basis of \( \mathcal V \).

Note that if \( \ket{x} = A \ket{v} \), then \( \bra{x} = (A \ket{v})^\dagger = \ket{v}^\dagger A^\dagger = \bra{v} A^\dagger \).
Therefore, when constructing inner products, we can write \( \mel{a}{M}{b} \) as \( \ip{a}{x} \) or \( \ip{y}{b} \) where \( \ket{x} = M \ket{b} \) or \( \ket{y} = M^\dagger \ket{a} \) (so that we have \( \bra{y} = \bra{a} M \)).

\subsection{Tensor products of linear maps}
Suppose \( A, B \) are linear maps \( \mathbb C^2 \to \mathbb C^2 \).
Then, we define \( A \otimes B \colon \mathbb C^2 \otimes \mathbb C^2 \to \mathbb C^2 \otimes \mathbb C^2 \) by its action on the basis \( (A \otimes B)\ket{i} \ket{j} = A \ket{i} B \ket{j} \).
In particular, for product vectors we obtain \( (A \otimes B) (\ket{v} \ket{w}) = A \ket{v} \otimes B \ket{w} \).

The \( 4 \times 4 \) matrix of components of \( A \otimes B \) has a simple block form, which can be seen by writing down its action on basis states.
\[ A = \begin{pmatrix}
    a & b \\
    c & d
\end{pmatrix};\quad B = \begin{pmatrix}
    p & q \\
    r & s
\end{pmatrix} \implies A \otimes B = \begin{pmatrix}
    aB & bB \\
    cB & dB
\end{pmatrix} = \begin{pmatrix}
    ap & aq & bp & bq \\
    ar & as & br & bs \\
    cp & cq & dp & dq \\
    cr & cs & dr & ds
\end{pmatrix} \]
Note that \( A \otimes I \) and \( I \otimes A \) can be thought of as acting only on one of the subspaces.
Consider \( \ket{\Phi} = \frac{1}{\sqrt{2}} \qty(\ket{00} + \ket{11}) \), and define \( A \) as above.
Then,
\begin{align*}
    (A \otimes I) \ket{\Phi} &= \frac{1}{\sqrt{2}} \qty[(A \ket{0}) \ket{0} + (A \ket{1}) \ket{1}] \\
    &= \frac{1}{\sqrt{2}} \qty[(a \ket{0} + c \ket{1}) \ket{0} + (b \ket{0} + d \ket{1}) \ket{1}] \\
    &= \frac{1}{\sqrt{2}} \qty[a \ket{00} + b \ket{01} + c \ket{10} + d \ket{11}] \\
    (I \otimes A) \ket{\Phi} &= \frac{1}{\sqrt{2}} \qty[\ket{0} (A \ket{0}) + \ket{1} (A \ket{1})] \\
    &= \frac{1}{\sqrt{2}} \qty[\ket{0} (a \ket{0} + c \ket{1}) + \ket{1} (b \ket{0} + d \ket{1})] \\
    &= \frac{1}{\sqrt{2}} \qty[a \ket{00} + c \ket{01} + b \ket{10} + d \ket{11}]
\end{align*}

\subsection{Third postulate: physical evolution of quantum systems}
The third postulate of quantum mechanics is that any physical finite-time evolution of a closed quantum system is represented by a unitary operation on the corresponding vector space of states.
Recall that the following are equivalent for a linear operator \( U \):
\begin{itemize}
    \item \( U \) is unitary, so \( U^{-1} = U^\dagger \);
    \item \( U \) maps an orthonormal basis to an orthonormal set of vectors;
    \item the columns (or rows) of \( U \) form an orthonormal set of vectors.
\end{itemize}
If a system is in a state \( \ket{\psi(t_1)} \) at a time \( t_1 \) and later in a state \( \ket{\psi(t_2)} \) at a time \( t_2 \), then \( \ket{\psi(t_2)} = U(t_1, t_2) \ket{\psi(t_1)} \) for some unitary map \( U(t_1, t_2) \) which depends only on \( t_1, t_2 \).
This operator is derived from the \emph{Schr\"odinger equation}, which is
\[ i\hbar \pdv{t} \ket{\psi(t)} = H \ket{\psi(t)} \]
where \( H \) is a self-adjoint operator known as the \emph{Hamiltonian}.
In particular, if \( H \) is time-independent, we have
\[ U(t_1, t_2) = e^{-\frac{i}{\hbar} H(t_2 - t_1)} \]
In the more general case,
\[ U(t_1, t_2) = e^{-\frac{i}{\hbar} \int_{t_1}^{t_2} H(t) \dd{t}} \]
The unitary evolution of a closed system is deterministic.

\subsection{Partial inner products}
A vector \( \ket{v} \in \mathcal V \) defines a linear map \( \mathcal V \otimes \mathcal W \to \mathcal W \) called the \emph{partial inner product} with \( \ket{v} \), defined on the basis \( \ket{e_i} \ket{f_j} \) of \( \mathcal V \otimes \mathcal W \) by \( \ket{e_i} \ket{f_j} \mapsto \ip{v}{e_i} \ket{f_j} \).
Similarly, for any \( \ket{w} \in \mathcal W \), we obtain a partial inner product \( \mathcal V \otimes \mathcal W \to \mathcal V \).
If \( \mathcal V, \mathcal W \) are isomorphic, we must specify which partial inner product is intended.

\subsection{Fourth postulate: quantum measurement}
Consider a system \( S \) with state space \( \mathcal V \), and let \( A \) be an observable.
\( A \) can be written as its \emph{spectral projection} \( A = \sum_k a_k P_k \) where \( A \ket{\varphi_k} = a_k \ket{\varphi_k} \).
If \( a_k \) is nondegenerate, \( P_k = \op{\varphi_k}{\varphi_k} \).
If \( a_k \) is degenerate of multiplicity \( m \), then \( P_k = \sum_{i=1}^m \op{\varphi_k^i}{\varphi_k^i} \).

The fourth postulate is that when an observable is measured, the resulting measurement will be an eigenvalue \( a_j \), with probability \( p(a_j) = \ev{P_j}{\psi} \).
Then, \( \ket{\psi} \) is replaced with the post-measurement state \( \frac{P_j \ket{\psi}}{\sqrt{p(a_j)}} \).
This is known as \emph{Born's rule}.
Such a measurement is called a \emph{projective measurement} (or sometimes a \emph{von Neumann measurement}), since the post-measurement state is given by a projection operator.

Suppose \( A, B \) are operators that do not commute, so \( [A,B] = AB - BA \neq 0 \).
Then, the measurement of \( A \) will influence the outcome probabilities of a subsequent measurement of \( B \).
For instance, suppose \( \ket{\psi} = \ket{+}, A = \sigma_z, B = \sigma_x \).

\subsection{Complete and incomplete projective measurements}
Let \( \ket{\psi} \in \mathcal V \) be a state in a state space of dimension \( n \).
Let \( \mathcal B = \qty{\ket{e_i}} \) be a set of \( n \) orthogonal basis vectors for \( \mathcal V \).
Then \( \ket{\psi} = \sum a_j \ket{e_j} \) where \( a_k = \ip{e_k}{\psi} \).
If the outcomes of a measurement are the indices of the basis vectors \( j = 1, \dots, n \), we have \( p(j) = \ev{P_j}{\psi} \) where \( P_j = \op{e_j}{e_j} \).
Therefore, \( p(j) = \abs{\ip{\psi}{e_j}}^2 = \abs{a_j}^2 \).
If the outcome is \( j \), the post-measurement state is \( \frac{P_j \ket{\psi}}{\sqrt{p(j)}} = \frac{\ket{e_j} \ip{e_j}{\psi}}{\sqrt{p(j)}} = \ket{e_j} \), so the state collapses to a basis vector.
Taking another measurement immediately in the same basis, we obtain the result \( j \) with probability 1.
Such a measurement is called a \emph{complete} projective measurement; it is called complete as all \( P_j \) are of rank 1.
When we measure a state \( \ket{\psi} \) in a basis, it is often helpful to consider an orthogonal decomposition of \( \mathcal V \) using the basis vectors.

Conversely, an \emph{incomplete} projective measurement corresponds to an arbitrary orthogonal decomposition of \( \mathcal V \).
Consider a decomposition of \( \mathcal V \) into \( d \) mutually orthogonal subspaces \( \mathcal E_1, \dots, \mathcal E_d \), so \( \mathcal V = \mathcal E_1 \oplus \dots \oplus \mathcal E_d \), and \( \dim \mathcal V = \sum \dim \mathcal E_j \).
Let \( \Pi_i \) be a projection operator onto \( \mathcal E_i \).
Since the spaces are mutually orthogonal, \( \Pi_i \Pi_j = \delta_{ij} \Pi_i \).
Consider a measurement with outcomes \( 1, \dots, d \) representing a particular subspace.
The probability of observing outcome \( i \) is \( \ev{\Pi_i}{\psi} \).
If the outcome is \( i \), \( \ket{\psi} \) is replaced with \( \frac{\Pi_i \ket{\psi}}{\sqrt{p(i)}} \).
In this case, the \( \Pi_i \) are no longer rank 1 projection operators.
If \( \mathcal E_i \) has basis \( \qty{\ket{f_j}} \), we can write \( \Pi_i = \sum \op{f_j}{f_j} \).

Incomplete projective measurement is a generalisation of complete projective measurement.
One can refine an incomplete measurement into a complete measurement by first considering a complete measurement, and then summing the relevant outcome probabiilities to obtain a description of the incomplete measurement probabilities.
Let \( \qty{\ket{e_k^{(j)}}}_{k=1}^{d_j} \) be a basis for \( \mathcal E_j \) for each \( j \).
Then \( \mathcal V = \bigoplus_{i=1}^d \mathcal E_j \) has orthonormal basis \( \qty{\ket{e_k^{(j)}}}_{j,k} \).
Then, \( \ip{e_i^{(k_1)}}{e_j^{(k_2)}} = \delta_{ij} \delta_{k_1 k_2} \).

Consider a two-bit string \( b_1 b_2 \).
The \emph{parity} of this string is \( b_1 \oplus b_2 \), where \( \oplus \) represents addition modulo 2.
Consider the orthogonal decomposition of \( \mathcal V \) into \( \mathcal E_0 \oplus \mathcal E_1 \), where \( \mathcal E_0 = \vecspan \qty{\ket{00}, \ket{11}} \) is the even parity subspace, and \( \mathcal E_1 = \vecspan \qty{\ket{01}, \ket{10}} \) is the odd parity subspace.
The outcomes of an incomplete measurement are then the labels 0 and 1 of the subspaces \( \mathcal E_0 \) and \( \mathcal E_1 \).
Note that \( \qty{\ket{00}, \ket{01}, \ket{10}, \ket{11}} \) is a complete orthonormal basis for \( \mathcal V \), so we can consider the complete projective measurement.
\( \ev{P_{00}}{\psi} \) is the probability of outcome \( 00 \) for the complete measurement, where \( P_{00} = \op{00}{00} \).
For the incomplete measurement, \( p(0) = \ev{\Pi_0}{\psi} \) is the probability of outcome 0, where \( \Pi_0 = P_{00} + P_{11} \).
So \( p(0) = \ev{P_{00}}{\psi} + \ev{P_{11}}{\psi} \).

\subsection{Extended Born rule}
Let \( S_1, S_2 \) be quantum systems with state spaces \( \mathcal V, \mathcal W \) with dimensions \( m, n \), and we consider the composite system \( S_1 S_2 \).
Let \( \qty{\ket{e_i}} \) be a complete orthonormal basis of \( \mathcal V \), and let \( \qty{\ket{f_j}} \) be a complete orthonormal basis of \( \mathcal W \).
Suppose the composite system is in an initial state \( \ket{\psi} = \sum a_{ij} \ket{e_i} \ket{f_j} \).
Suppose now that we want to measure \( \ket{\psi} \) in the basis \( \qty{\ket{e_i}} \); this amounts to an incomplete measurement with subspaces \( \mathcal E_i = \vecspan \qty{\ket{e_i} \otimes \ket{\varphi} \mid \ket{\varphi} \in \mathcal W} \) for \( 1 \leq i \leq m \).
The outcomes of such a measurement are \( \qty{1, \dots, m} \), and the \( \mathcal E_i \) are mutually orthogonal.
The probability of a given outcome is \( p(k) = \ev{P_k \otimes I}{\psi} \), where \( P_k = \op{e_k}{e_k} \).
Hence,
\[ p(k) = \qty(\sum a_{i'j'}^\star \bra{e_i'} \bra{f_j'}) \qty(\op{e_k}{e_k} \otimes I) \qty(\sum a_{ij} \ket{e_i} \ket{f_j}) = \sum_{j=1}^n a_{kj}^\star a_{kj} \]
If the outcome is \( k \), then the post-measurement state is given by \( \ket{\psi_{\text{after}}} = \frac{(P_k \otimes I)\ket{\psi}}{p(k)} = \frac{\sum_j a_{kj} \ket{e_k} \ket{f_j}}{\sqrt{\sum_j \abs{a_{kj}}^2}} \).
Using partial inner products, one can show that \( \ket{\psi_{\text{after}}} \) is normalised.
These rules are referred to as the \emph{extended Born rule}.

Consider a quantum system \( S \) with state space \( \mathcal V \).
A measurement relative to any basis \( \mathcal C \) can be performed by first performing a unitary operator, then performing a measurement in a fixed basis \( \mathcal B \).
Let \( \mathcal B = \qty{\ket{e_i}} \), and \( \mathcal C = \qty{\ket{e_i'}} \).
Let \( U \) be a unitary operator such that \( \ket{e_i'} = U \ket{e_i} \).
Then, \( U^\dagger = U^{-1} \) has the property that \( U^{-1} \ket{e_i'} = \ket{e_i} \).
Suppose we have a state \( \ket{\psi} \in \mathcal V \).
Let \( \ket{\psi} = \sum c_i \ket{e_i'} \).
Applying \( U^{-1} \) to \( \ket{\psi} \), we obtain \( U^{-1} \ket{\psi} = \sum c_i \ket{e_i} \) by linearity.
We can then measure \( \ket{\psi'} = U^{-1} \ket{\psi} \) in the basis \( \mathcal B \).
By the Born rule, \( p(i) = \ev{P_i}{\psi'} = \bra{\psi} U P_i U^\dagger \ket{\psi} \) where \( P_i = \op{e_i}{e_i} \), as we are performing a complete projective measurement.
If the outcome is \( i \), then the post-measurement state is \( \ket{\psi'_{\text{after}}} = \frac{P_i \ket{\psi'}}{p(i)} \).

\subsection{Standard measurement on multi-qubit systems}
Consider a system of \( n \) qubits.
The state space is \( (\mathbb C^2)^{\otimes n} \).
The \emph{computational basis} or \emph{standard basis} is \( \mathcal B = \qty{\ket{i_1 \dots i_n} \mid i_j \in \qty{0,1}} \).
The labels of the elements of the standard basis are labelled by bit strings of length \( n \).

Suppose we are measuring a subset of \( k \) qubits of the \( n \)-qubit system.
Let \( n = 3 \), and let
\[ \ket{\psi} = \frac{i}{2} \ket{000} + \frac{1+i}{2\sqrt{2}} \ket{001} - \frac{1}{2} \ket{101} + \frac{3}{10} \ket{110} - \frac{2i}{5} \ket{111} \]
The standard measurement of any of the three qubits will always have the outcome zero or one.
Suppose we perform a standard measurement on the first qubit.
By the extended Born rule, we obtain
\[ p^{(1)}(1) = \bra{\psi} P_1 \otimes I \otimes I \ket{\psi} = \bra{\psi} \qty(\op{1}{1} \otimes I \otimes I) \ket{\psi} = \frac{1}{4} + \frac{9}{100} + \frac{4}{25} = \frac{1}{2} \]
If we measure the outcome 1, the post-measurement state is \( \ket{\psi_{\text{after}}} = \frac{(P_1 \otimes I \otimes I)\ket{\psi}}{\sqrt{p^{(1)}(1)}} \).

\subsection{Reliably distinguishing states}
Note that the measurement postulate implies that states with guaranteed (with probability 1) different measurement outcomes always lie in mutually orthogonal subspaces.
We say that two states are \emph{reliably distinguishable} if there exists a measurement which outputs two distinct outcomes with probability 1 when applied to the two states.
Therefore, two states \( \ket{\psi}, \ket{\varphi} \) are reliably distinguishable if and only if they are orthogonal, so \( \ip{\psi}{\varphi} = 0 \).

Let \( \ket{\psi} \) and \( \ket{\varphi} \) be orthogonal.
Let \( \mathcal B = \qty{\ket{\psi}, \ket{f_1}, \dots, \ket{f_{m-1}}} \) be a complete orthonormal basis for \( \mathcal V \).
Then \( \ip{\psi}{f_j} = 0 \) and \( \ip{f_j}{f_k} = \delta_{jk} \).
Measuring \( \ket{\psi} \) in this basis, \( p(1) = \bra{\psi} P_1 \ket{\psi} \) where \( P_1 = \op{\psi}{\psi} \), so the probability is 1.
Measuring \( \ket{\varphi} \) in this basis, \( p(1) = \ip{\psi}{\varphi} \ip{\varphi}{\psi} = 0 \).
This is an example of a measurement which can reliably distinguish \( \ket{\psi} \) and \( \ket{\varphi} \).

Vectors \( \ket{v} = \ket{\psi} \) and \( \ket{v'} = e^{i\theta} \ket{\psi} \) are not distinguishable.
For any measurement, the probability of obtaining a particular outcome when measuring \( \ket{v} \) is always the same as the probability when measuring \( \ket{v'} \).
