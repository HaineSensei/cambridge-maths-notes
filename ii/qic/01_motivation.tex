\subsection{Motivation}
In classical computation, the elementary unit of information is the \emph{bit}, which takes a value in \( \qty{0,1} \).
This gives the result of a single binary decision problem, where the zero and one correspond to different answers to the problem.
Binary strings of length greater than one are used to provide more than 2 answers to a problem; if we have \( n \) bits, we can encode \( 2^n \) different messages.

Classical computation is understood to be the processing of information: taking an initial bit string and and updating it by a prescribed sequence of steps.
The steps are taken to be the action of local Boolean logic gates, such as conjunction, disjunction, or negation.
At each step, a small number of bits in prescribed locations are edited.

Information in the real world must be tied to a physical representation.
For example, bits in a processor are often represented by different voltages of specific components.
Importantly, there is no information \emph{without} representation.
Performing a computation classically must therefore involve the evolution of a physical system over time, which is coverned by the laws of classical physics.

However, nature does not abide by classical physics at subatomic levels, and we must use quantum mechanics to accurately model such behaviours.
One such behaviour modelled by quantum mechanics is the superposition principle, that the corresponding quantum analog of the bit need not be in precisely one state.
Quantum entanglement is the phenomenon where particles can be linked in such a way that their states can be manipulated even at a distance.
Quantum measurement is probabilistic and alters the underlying system.

Quantum information and computation therefore exploits these features of quantum mechanics to address issues of information storage, communication, computation, and cryptography.
The features of quantum mechanics seem to allow us benefits which are beyond the limits of classical information and computation, even in principle.
Note that a quantum computer cannot perform any task that cannot in principle be performed classically.
We only hope that quantum techniques allow a reduction in the complexity of certain algorithms.

\subsection{Benefits of quantum information and computation}
In complexity theory, we study the \emph{hardness} of a certain computational task.
One must consider the resources required for the task; which in classical computation are normally limited to time (measured in number of computational steps) and space (amount of memory required).

If an algorithm takes time bounded by a polynomial function in the input size \( n \), we say the algorithm is \emph{polyomial-time}.
Otherwise, we say it is an \emph{exponential-time} algorithm.
Polynomial-time algorithms are typically taken to be computable in practice, but exponential-time algorithms are usually considered only computable in principle.
Quantum mechanical techniques can provide polynomial-time algorithms that have only exponential-time classical versions.
One example is Shor's integer factorisation algorithm.

Quantum states of physical sytems can be used to encode information, such as spin states of electrons.
There are certain tasks possible with such quantum states which are impossible in classical physics; one example is quantum teleportation.

There are also some technological issues with classical physics.
Components of processors have become minified to atomic scale, and therefore they cannot be shrunk much further without dealing with the effects of quantum mechanics.
Conversely, there are technological challenges with quantum physics.
Quantum systems are very fragile, and modern quantum computers typically require temperatures close to absolute zero to reduce noise.

\emph{Quantum supremacy} refers to the hypothetical moment at which a programmable quantum computer can first solve a problem in practice that a classical computer cannot.
At the time of writing, there is no concensus that quantum supremacy has been achieved.

\subsection{Mathematical preliminaries}
Every quantum mechanical system is associated with a Hilbert space \( \mathcal V \), a complex inner product space that is a complete metric space with respect to the distance function induced by the inner product.
