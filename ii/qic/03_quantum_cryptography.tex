\subsection{One-time pads}
We can use quantum information theory to securely transmit messages between agents Alice and Bob, who may be in distant locations, without the possibility that an eavesdropper Eve can recover the message that was sent.

We will assume that Alice and Bob have an authenticated classical channel through which they can send classical information; Alice and Bob can verify that any particular message on the channel came from a particular sender.
We also assume that Eve cannot block the channel or modify any messages transmitted, but she can monitor the channel freely.
Hence, Alice and Bob can receive messages from each other without error.

In the classical setting, there exists a provably secure classical scheme for private communications, called the \emph{one-time pad}.
This requires that Alice and Bob share a private key \( K \), which is a binary string.
\( K \) must have been created beforehand, and must be chosen uniformly at random from the set of binary strings of the same length as the message \( M \).
Suppose \( M, K \in \qty{0,1}^n \).

The protocol is as follows.
First, Alice computes the encrypted message \( C = M \oplus K \).
She then sends \( C \) to Bob through the classical channel.
Bob can then compute \( C \oplus K = C \oplus K \oplus K = M \) to obtain the message that was sent by Alice.
Eve cannot learn any information about the message (apart from its length), as she has no knowledge of \( K \).
In general, the probability that a particular \( K \) was chosen is \( 2^{-n} \).
This scheme cannot be broken.

Suppose that Alice and Bob use the same key \( K \) to send two messages \( M_1, M_2 \).
Eve can obtain \( M_1 \oplus K \) and \( M_2 \oplus K \), and can therefore compute \( (M_1 \oplus K) \oplus (M_2 \oplus K) = M_1 \oplus M_2 \), which gives some information about the messages that were sent.
Any key must only be used once, so the one-time pad protocol is inefficient.
To solve this problem, we will construct methods for distributing keys, using techniques from quantum information theory.

\subsection{The BB84 protocol}
Quantum key distribution allows Alice and Bob to generate a private key without needing to physically meet.
This key can then be used to send messages over the one-time pad protocol.
In addition to a classical channel, we assume that Alice and Bob also have access to a quantum channel through which they can send qubits.
We will show that Eve cannot gain information about the key that Alice and Bob generate without being detected.

Consider the bases \( \mathcal B_0 = \qty{\ket{0}, \ket{1}}, \mathcal B_1 = \qty{\ket{+}, \ket{-}} \).
These are examples of \emph{mutually unbiased bases}; a pair of bases such that if any basis vector is measured relative to the other basis, all outcomes are equally likely.
For example, measuring \( \ket{+} \) relative to \( \mathcal B_0 \) gives probability \( \frac{1}{2} \) for outcomes 0 and 1.

First, Alice generates two \( m \)-bit strings \( x = x_1 \dots x_m \in \qty{0,1}^m, y = y_1 \dots y_m \in \qty{0,1}^m \) uniformly at random.
She then prepares the \( m \)-qubit state \( \ket{\psi_xy} = \ket{\psi_{x_1 y_1}} \otimes \dots \otimes \ket{\psi_{x_m y_m}} \) where
\[ \ket{\psi_{x_i y_i}} = \begin{cases}
    \ket{0} & x_i = 0; y_i = 0 \\
    \ket{1} & x_i = 1; y_i = 0 \\
    \ket{+} & x_i = 0; y_i = 1 \\
    \ket{-} & x_i = 1; y_i = 1
\end{cases} \]
