\subsection{Using higher Hilbert spaces}
Quantum information is encoded in the states of a quantum system.
Classical information is encoded in states chosen from an orthonormal set, since all distinct classical messages can be distinguished.
Given a quantum system \( S \) and a quantum state \( \ket{\psi} \), we can perform this sequence of operations.
\begin{itemize}
    \item (ancilla) Consider an auxiliary system \( A \) in a fixed state \( \ket{A} \in \mathcal V_A \).
    The composite system \( SA \) has vector space \( \mathcal V_S \otimes \mathcal V_A \).
    The initial joint state is \( \ket{\psi}\ket{A} \).
    This results in an embedding of quantum information in a higher dimensional space.
    \item (unitary) Consider the action of a unitary operator \( U \) on \( SA \) (or on \( S \)), modelling the time evolution of the quantum system.
    \item (measure) We can perform measurements on \( SA \) (or on \( S \)).
    The post-measurement state of \( S \) is retained, and the auxiliary system \( A \) is discarded.
\end{itemize}
This process is sometimes known as `going to the church of the higher Hilbert space'.
The presence of the ancilla allows for entanglement with other quantum systems.

\subsection{No-cloning theorem}
Classically, information can be easily copied by measuring all relevant information and reproducing it.
% We can model this by considering a state \( A \) containing information and a `blank' state \( B \), and performing an operation to yield the original state \( A \) together with a new copy of \( A \) in the place of \( B \).
Quantum copying involves three systems:
\begin{itemize}
    \item a system \( A \) containing some quantum information to be copied;
    \item a system \( B \) with \( \mathcal V_B \simeq \mathcal V_A \) initially in some fixed state \( \ket{0} \) where the information is to be copied;
    \item a system \( M \) which represents any physical machinery in some `ready' state \( \ket{M_0} \) required for performing the copy.
\end{itemize}
The initial state of this composite system \( ABM \) is \( \ket{\psi} \ket{0} \ket{M_0} \).
Note that the \( \ket{\psi} \) and \( \ket{0}\ket{M_0} \) are \emph{uncorrelated} in this state, as we are using the tensor product to combine them.
Suppose that the cloning process is performed using some unitary operator \( U \), so \( U\ket{\psi_A}\ket{0}\ket{M_0} = \ket{\psi_A}\ket{\psi_B}\ket{M_\psi} \).
This cloning process may be required to work either for all states of \( A \), or for some subset of \( A \).
\begin{theorem}
    Let \( \mathcal S \) be any set of states of the system \( A \) that contains at least one pair of distinct non-orthogonal states.
    Then there does not exist any unitary operator \( U \) that clones all states in \( \mathcal S \).
\end{theorem}
\begin{proof}
    Let \( \ket{\xi}, \ket{\eta} \) be distinct non-orthogonal states in \( \mathcal S \), so \( \ip{\xi}{\eta} \neq 0 \).
    Suppose such a unitary operator \( U \) exists.
    Then, we must have
    \[ U \ket{\xi_A}\ket{0_B}\ket{M_0} = \ket{\xi_A}\ket{\xi_B}\ket{M_\xi};\quad U \ket{\eta_A}\ket{0_B}\ket{M_0} = \ket{\eta_A}\ket{\eta_B}\ket{M_\eta} \]
    Unitary operators preserve inner products.
    Hence,
    \[ \ip{\xi_A}{\eta_A} \ip{0_B}{0_B} \ip{M_0}{M_0} = \ip{\xi_A}{\eta_A} \ip{\xi_B}{\eta_B} \ip{M_\xi}{M_\eta} \]
    Hence, \( \ip{\xi}{\eta} = \qty(\ip{\xi}{\eta})^2 \ip{M_\xi}{M_\eta} \).
    By taking the absolute value, \( \abs{\ip{\xi}{\eta}} = \abs{\ip{\xi}{\eta}}^2 \abs{\ip{M_\xi}{M_\eta}} \).
    Since \( \xi \neq \eta \), we must have \( 0 < \abs{\ip{\xi}{\eta}} < 1 \), and \( 0 \leq \abs{\ip{M_\xi}{M_\eta}} \leq 1 \).
    Therefore, \( 1 = \abs{\ip{\xi}{\eta}} \abs{\ip{M_\xi}{M_\eta}} < 1 \), which is a contradiction.
\end{proof}
If quantum cloning were possible, superluminal (indeed, instantaneous) communication would also be possible.
Suppose we have a state \( \ket{\psi^+_{AB}} = \frac{1}{\sqrt{2}} \qty(\ket{00} + \ket{11}) \in \mathbb C^2 \otimes \mathbb C^2 \).
Let \( A, B \) be the entangled parts of this quantum state, and suppose that we send qubit \( A \) to Alice and \( B \) to Bob, far apart from each other.

If we want to send the bit `yes' from Alice to Bob, we measure the qubit \( A \) in the basis \( \qty{\ket{0}, \ket{1}} \), which gives outcomes \( 0, 1 \) with probability \( \frac{1}{2} \).
If the outcome is 0, the final state of \( B \) is \( \ket{0} \), and if the outcome is 1, the final state of \( B \) is \( \ket{1} \).
If we want to send `no', we instead measure \( A \) in the basis \( \qty{\ket{+}, \ket{-}} \), which gives the outcomes \( +, - \) with probability \( \frac{1}{2} \).
Similarly, the final state of \( B \) is \( \ket{+} \) or \( \ket{-} \).

We claim that these `yes' \( (\ket{0}, \ket{1}) \) and `no' \( (\ket{+}, \ket{-}) \) \emph{preparations} of qubit \( B \) are indistinguishable by Bob with any local action on the qubit.
That is, they each give exactly the same probability distribution of outcomes of any measurement.
In fact, the distribution matches the prior distribution before qubit \( A \) was measured.

Let \( \Pi_i \) be the projection operator for outcome \( i \) on qubit \( B \).
Suppose that `yes' was sent.
Then,
\[ p_{\text{yes}}(i) = \frac{1}{2}\ev{\Pi_i}{0} + \frac{1}{2}\ev{\Pi_i}{1} = \frac{1}{2} \Tr \qty[\Pi_i \op{0}{0}] + \frac{1}{2} \Tr \qty[\Pi_i \op{1}{1}] = \frac{1}{2} \Tr \qty[\Pi_i \qty(\op{0}{0} + \op{1}{1})] = \frac{1}{2} \Tr \Pi_i \]
In the `no' case,
\[ p_{\text{no}}(i) = \frac{1}{2}\ev{\Pi_i}{+} + \frac{1}{2}\ev{\Pi_i}{-} = \frac{1}{2} \Tr \qty[Pi_i \qty(\op{+}{+} + \op{-}{-})] = \frac{1}{2} \Tr \Pi_i \]
These probability distributions match.

Suppose that cloning were possible.
We clone the qubit \( B \) multiple times after the message was sent, to produce one of the states \( \ket{0} \dots \ket{0}, \ket{1} \dots \ket{1}, \ket{+} \dots \ket{+}, \ket{-} \dots \ket{-} \).
We now measure each qubit in the basis \( \ket{0}, \ket{1} \) separately.
If the `yes' message was sent, all measurements will result in 0 or 1.
If `no' was sent, it is possible that two measurements would differ.
In expectation, half of the measurements would result in the outcome 0 and half would result in the outcome 1.
Therefore, the `yes' and `no' errors can be distinguished with probability of error \( 2^{-N+1} \) if we make \( N \) copies of \( B \).

\subsection{Distinguishing non-orthogonal states}
Suppose you know a state \( \ket{\psi} \) has state \( \ket{\alpha_0} \) or \( \ket{\alpha_1} \) with probability \( \frac{1}{2} \), where \( \ip{\alpha_0}{\alpha_1} \neq 0 \).
Since the states are non-orthogonal, we cannot perfectly distinguish the states, but must allow some error rate.
The simplest possibility is to not make a measurement and guess randomly; in which case, the guess is correct with probability \( \frac{1}{2} \).

Suppose we append an auxiliary system \( \ket{A} \) to \( \ket{\alpha_i} \).
Note that \( \bra{A}\bra{\alpha_i}\ket{\alpha_i}\ket{A} = \ip{\alpha_i}{\alpha_i} \) as \( \ket{A} \) is normalised.
Applying a unitary operator \( U \) to \( \ket{\alpha_i} \) then performing a projection \( \Pi_0 \) or \( \Pi_1 \), this corresponds to simply performing a measurement \( \Pi_0' = U^\dagger \Pi_0 U \) or \( \Pi_1' = U^\dagger \Pi_1 U \), which leads to the same probabilities of outcomes.
Indeed,
\[ p(i) = \ev{\Pi_i}{U\xi} = \ev{U^\dagger \Pi_i U}{\xi} = \ev{\Pi_i'}{\xi} \]
Therefore, in this particular problem, we gain no benefit from moving to a larger Hilbert space or applying unitary operators.

We now describe the \emph{state estimation} or \emph{state discrimination} process.
We will consider a two-outcome measurement \( \qty{\Pi_0, \Pi_1} \), where \( \Pi_0 + \Pi_1 = I \).
The average success probability is
\begin{align*}
    p_S(\Pi_0, \Pi_1) &= \frac{1}{2} \prob{0 \mid \ket{\psi} = \ket{\alpha_0}} + \frac{1}{2} \prob{1 \mid \ket{\psi} = \ket{\alpha_1}} \\
    &= \frac{1}{2}\ev{\Pi_0}{\alpha_0} + \frac{1}{2}\ev{\Pi_1}{\alpha_1} \\
    &= \frac{1}{2} + \frac{1}{2} \Tr \qty[\Pi_0 \qty(\op{\alpha_0}{\alpha_0} - \op{\alpha_1}{\alpha_1})]
\end{align*}
as \( \Tr(A\op{\psi}{\psi}) = \ev{A}{\alpha} \).
The optimal choice of measurement maximises the average success probability \( p_S \).
Note that \( \Delta = \op{\alpha_0}{\alpha_0} - \op{\alpha_1}{\alpha_1} \) is self-adjoint, and we can write \( p_S = \frac{1}{2} + \frac{1}{2} \Tr(\Pi_0 \Delta) \).
Therefore, the eigenvalues of \( \Delta \) are real, and the eigenvalues form an orthonormal basis.
For a state \( \ket{\beta} \) orthogonal to both \( \ket{\alpha_0} \) and \( \ket{\alpha_1} \), we have \( \Delta \ket{\beta} = 0 \).
Therefore, \( \Delta \) acts nontrivially only in the vector space spanned by \( \ket{\alpha_0} \) and \( \ket{\alpha_1} \), and hence has at most two nonzero eigenvalues, and its eigenvectors lie in \( \vecspan\qty{\ket{\alpha_0}, \ket{\alpha_1}} \).

Now, \( \Tr\Delta = 0 \) so the eigenvalues are \( \delta \) and \( -\delta \) for some \( \delta \in \mathbb R \).
Let \( \ket{p} \) be the eigenvector for \( \delta \), and \( \ket{m} \) be the eigenvector for \( -\delta \), so \( \ip{p}{m} = 0 \).
We can write \( \Delta \) in its spectral decomposition, giving \( \Delta = \delta \op{p}{p} - \delta \op{m}{m} \).

Let \( \ket{\alpha_0^\perp} \in \vecspan\qty{\ket{\alpha_0}, \ket{\alpha_1}} \) be a normalised vector such that \( \ip{\alpha_0^\perp}{\alpha_0} = 0 \).
Then, \( \vecspan\qty{\ket{\alpha_0}, \ket{\alpha_0^\perp}} \) forms an orthonormal basis.
Hence, we can write \( \ket{\alpha_1} = c_0 \ket{\alpha_0} + c_1 \ket{\alpha_0^\perp} \).
In this basis,
\[ \Delta = \begin{pmatrix}
    1 & 0 \\
    0 & 0
\end{pmatrix} + \begin{pmatrix}
    -\abs{c_0}^2 & -c_0 c_1^\star \\
    -c_0^\star c_1 & -\abs{c_1}^2
\end{pmatrix} = \begin{pmatrix}
    1 - \abs{c_0}^2 & -c_0 c_1^\star \\
    -c_0^\star c_1 & -\abs{c_1}^2
\end{pmatrix} = \begin{pmatrix}
    \abs{c_1}^2 & -c_0 c_1^\star \\
    -c_0^\star c_1 & -\abs{c_1}^2
\end{pmatrix} \]
which has eigenvalues \( \delta = \abs{c_1}, -\delta = -\abs{c_1} \).
Since \( \abs{c_0} = \abs{\ip{\alpha_0}{\alpha_1}} = \cos \theta \) where \( \theta \geq 0 \), we have \( \delta = \sin \theta \).
Then,
\begin{align*}
    p_S(\Pi_0, \Pi_1) &= \frac{1}{2} + \frac{1}{2} \Tr(\Pi_0 \Delta) \\
    &= \frac{1}{2} + \frac{1}{2} \Tr(\Pi_0 \qty[\sin\theta \op{p}{p} - \sin\theta \op{m}{m}]) \\
    &= \frac{1}{2} + \frac{\sin\theta}{2} \qty[\ev{\Pi_0}{p} - \ev{\Pi_0}{m}]
\end{align*}
Note that for any \( \ket{\varphi} \), we have \( 0 \leq \ev{\Pi}{\varphi} \leq 1 \), so the measurement is maximised when \( \ev{\Pi_0}{p} = 1 \) and \( \ev{\Pi_0}{m} = 0 \).
We therefore define \( \Pi_0 = \op{p}{p} \).
Then, the optimal average success probability is
\[ p_S^\star = \frac{1}{2} + \frac{\sin\theta}{2} \]
\begin{theorem}[Holevo--Helstrom; pure states]
    Let \( \ket{\alpha_0}, \ket{\alpha_1} \) be equally likely states, with \( \abs{\ip{\alpha_0}{\alpha_1}} = \cos\theta \), \( \theta \geq 0 \).
    Then, the probability \( p_S \) of correctly identifying the state by any quantum measurement satisfies
    \[ p_S \leq \frac{1}{2} + \frac{\sin\theta}{2} \]
    and this bound can be attained.
\end{theorem}
In the case of orthogonal states, the theorem implies that \( p_S \leq 1 \) and the bound can be attained, which was shown before.

\subsection{No-signalling principle}
Suppose we have a possibly entangled state \( \ket{\phi_{AB}} \in \mathcal V_A \otimes \mathcal V_B \) shared between two agents Alice (\( A \)) and Bob (\( B \)).
Suppose we perform a complete projective measurement on \( \ket{\phi_A} \).
By the extended Born rule, each measurement outcome will lead to an instantaneous change of \( \ket{\phi_B} \).
If this change in state could be detected by measuring \( \ket{\phi_B} \), instantaneous communication between \( A \) and \( B \) would be possible.

Consider \( \ket{\phi_{AB}^+} = \frac{1}{\sqrt{2}}\qty(\ket{00} + \ket{11}) \).
Suppose qubit \( A \) is measured in the standard basis \( \qty{\ket{0}, \ket{1}} \).
\begin{center}
    \begin{tabular}{c c c c}
        outcome & probability & post-measurement state & final state of \( B \) \\
        0 & \( \frac{1}{2} \) & \( \ket{00} \) & \( \ket{0} \) \\
        1 & \( \frac{1}{2} \) & \( \ket{11} \) & \( \ket{1} \)
    \end{tabular}
\end{center}
Suppose qubit \( B \) is subsequently measured in \( \qty{\ket{b_0}, \ket{b_1}} \).
If \( B \) is in the state \( \ket{0} \), we can write \( \ket{0} = c_0\ket{b_0} + c_1\ket{b_1} \), and \( p_{\ket{0}}(i) = \abs{c_i}^2 = \abs{\ip{b_i}{0}}^2 \).
If \( B \) is in the state \( \ket{1} \), we write \( \ket{1} = d_0\ket{b_0} + d_1\ket{b_1} \), and \( p_{\ket{1}}(i) = \abs{d_i}^2 = \abs{\ip{b_i}{1}}^2 \).
Therefore, \( p(i) = \frac{1}{2} \abs{\ip{b_i}{0}}^2 + \frac{1}{2} \abs{\ip{b_i}{1}}^2 = \frac{1}{2} \).
The two outcomes for this measurement are equally likely, regardless of the choice of complete orthonormal basis \( \qty{\ket{b_0}, \ket{b_1}} \).

Suppose instead \( A \) is not measured, but we perform the same measurement on \( B \).
The initial state is \( \ket{\phi_{AB}^+} \), so by the extended Born rule, \( p(i) = \ev{\qty(I_A \otimes \op{b_i}{b_i})}{\phi_{AB}^+} = \frac{1}{2} \).
We can therefore not detect through measuring \( B \) whether a measurement was performed at \( A \).
This is the no-signalling principle.

We now prove the more general case.
Let \( \ket{\phi_{AB}} \in \mathcal V_A \otimes \mathcal V_B \) be an arbitrary possibly entangled state.

Suppose we measure \( B \) in a complete orthonormal basis \( \qty{\ket{b}}_{b=1}^{\dim \mathcal V_B} \), which is a complete projective measurement on \( B \).
Let \( \qty{\ket{a}}_{a=1}^{\dim \mathcal V_A} \) be a complete orthonormal basis for \( \mathcal V_A \).
Then, expanding \( \ket{\phi_{AB}} \), in this basis, we can write \( \ket{\phi_{AB}} = \sum_{a,b} c_{ab} \ket{a} \ket{b} \).
We obtain outcome \( b \) with probability \( p(b) = \ev{\qty(I_A \otimes P_b)}{\phi_{AB}} = \sum_{a=1}^{\dim \mathcal V_A} \abs{c_{ab}}^2 \).
The post-measurement state is \( \ket{\phi_{AB}'} \).

Suppse that we first measure \( A \) in a complete orthonormal basis \( \qty{\ket{a}}_{a=1}^{\dim \mathcal V_A} \), and then perform the measurement \( \qty{\ket{b}}_{b=1}^{\dim \mathcal V_B} \) on \( B \).
The outcome of the first measurement is \( a \) with probability \( p(a) = \ev{\qty(P_a \otimes I_B)}{\phi_{AB}} = \sum_{b=1}^{\dim \mathcal V_B} \abs{c_{ab}}^2 \).
We denote the post-measurement state of the joint system by \( \ket{\phi_{AB}''}= \frac{(P_a \otimes I_B) \ket{\phi_{AB}}}{\sqrt{p(a)}} \).
Then, the outcome of the second measurement is \( b \) with probability
\begin{align*}
    p(a \mid b) &= \ev{\qty(I_A \otimes P_b)}{\phi_{AB}''} \\
    &= \frac{1}{p(a)} \ev{(P_a \otimes I_B) (I_A \otimes P_b) (P_a \otimes I_B)}{\phi_{AB}} \\
    &= \frac{1}{p(a)} \ev{(P_a \otimes P_b)}{\phi_{AB}} \\
    p(a,b) = p(a) p(a \mid b) &= \ev{(P_a \otimes P_b)}{\phi_{AB}} = \abs{c_{ab}}^2
\end{align*}
Hence \( p(b) = \sum_{a=1}^{\dim \mathcal V_A} \abs{c_{ab}}^2 \), which is exactly the distribution we obtained when no measurement on \( A \) was performed.
This proves the no-signalling principle.

\subsection{The Bell basis}
Let \( \mathcal C^2 \otimes \mathcal C^2 \) model a quantum system representing the spins of two electrons.
Consider \( \ket{\phi_{AB}^+} = \frac{1}{2}\qty(\ket{00} + \ket{11}) \in \mathcal C^2 \otimes \mathcal C^2 \).
This is a \emph{maximally entangled state}; we have information about the whole system, but no information about the individual states.
\[ \ket{\phi_{AB}^\pm} = \frac{1}{\sqrt{2}} \qty(\ket{00} \pm \ket{11});\quad \ket{\psi_{AB}^\pm} = \frac{1}{\sqrt{2}} \qty(\ket{01} \pm \ket{10}) \]
\( \qty{\ket{\phi_{AB}^\pm}, \ket{\psi_{AB}^\pm}} \) forms a complete orthonormal basis of \( \mathcal C^2 \otimes \mathcal C^2 \).
This is called the \emph{Bell basis}.
The basis vectors are sometimes known as \emph{EPR states}, after Einstein, Podolsky, and Rosen.

One bit of classical information can be encoded in a single qubit, and two bits can be encoded in a pair of qubits in the Bell basis.
The Bell states have a \emph{parity} 0 or 1, representing parallel \( \qty{\ket{\phi^\pm}} \) or antiparallel \( \qty{\ket{\psi^\pm}} \) spins.
The states also have a \emph{phase}, which can be positive \( \qty{\ket{\phi^+}, \ket{\psi^+}} \) or negative \( \qty{\ket{\phi^-}, \ket{\psi^-}} \).
For example, we can encode the classical message 01 using the state \( \ket{\phi^-} \).

We can perform a complete projective measurement on both qubits in the Bell basis to recover the encoded information with certainty.
For instance, \( P_{00} = \op{\phi^+}{\phi^+} \).
If we prepare a pair of electrons \( \ket{\phi} \) in the state \( \ket{\phi^-} \) for example, we obtain \( p(00) = p(10) = p(11) = 0 \) and \( p(01) = 1 \).

\subsection{Superdense coding}
Suppose Alice wants to send a classical message to Bob.
Two bits of classical information can be sent reliably via a single qubit, provided that Alice and Bob share an entangled state, using \emph{superdense coding} or \emph{quantum dense coding}.
Let \( X = \sigma_x, Z = \sigma_z, Y = i\sigma_y = \begin{pmatrix}
    0 & 1 \\
    -1 & 0
\end{pmatrix} \).
One can check that the Bell basis vectors satisfy
\begin{align*}
    \ket{\phi^+} &= (I \otimes I) \ket{\phi^+} = (I \otimes I) \ket{\phi^+}\\
    \ket{\phi^-} &= (Z \otimes I) \ket{\phi^+} = (I \otimes Z) \ket{\phi^+}\\
    \ket{\psi^+} &= (X \otimes I) \ket{\phi^+} = (I \otimes X) \ket{\phi^+}\\
    \ket{\psi^-} &= (Y \otimes I) \ket{\phi^+} = -(I \otimes Y) \ket{\phi^+}
\end{align*}
Suppose we have shared the entangled Bell state \( \ket{\phi^+_{AB}} \) between Alice and Bob.
The superdense coding protocol is
\begin{center}
    \begin{tabular}{c c c}
        Alice's message & local action on \( A \) & final state of \( AB \) \\
        00 & \( I \) & \( \ket{\phi^+} \) \\
        01 & \( Z \) & \( \ket{\phi^-} \) \\
        10 & \( X \) & \( \ket{\psi^+} \) \\
        11 & \( Y \) & \( \ket{\psi^-} \)
    \end{tabular}
\end{center}
Then, Alice sends qubit \( A \) to Bob, so Bob has the entire state \( AB \).
Bob performs a Bell measurement, which distinguishes between the four Bell states, thus recovering Alice's message.
Since the state is maximally entangled, an eavesdropper who may intercept Alice's transmission cannot recover any part of the message.

\subsection{Quantum gates}
A quantum gate is given by a unitary operator acting on some qubits.
Such gates have matrix representations in the computational basis.
\begin{enumerate}
    \item The \emph{Hadamard gate} is
    \[ H = \frac{1}{\sqrt{2}} \begin{pmatrix}
        1 & 1 \\
        1 & -1
    \end{pmatrix} \]
    One can show that
    \[ H \ket{0} = \ket{+};\quad H\ket{1} = \ket{-};\quad H\ket{+} = \ket{0};\quad H\ket{-} = \ket{1} \]
    Note that \( H^\transpose = H^\dagger = H \) and \( H^2 = I \).
    As an orthogonal transformation in \( \mathbb R^2 \), it acts as a reflection by an angle of \( \frac{\pi}{8} \) to the positive \( x \) axis.
    This gate is drawn
    \begin{center}
        \leavevmode
        \Qcircuit { & \gate{H} & \qw }
    \end{center}
    In general, by linearity we obtain
    \begin{center}
        \leavevmode
        \Qcircuit {
            \lstick{a \ket{0} + b \ket{1}} & \gate{H} & \rstick{a \ket{+} + b \ket{-}} \qw
        }
    \end{center}
    \item The \( X, Z \) gates are given by
    \[ X \ket{k} = \ket{k \oplus 1};\quad Z \ket{k} = (-1)^k \ket{k} \]
    where \( \oplus \) denotes addition modulo 2.
    They \( X, Z, Y \) gates are drawn
    \begin{center}
        \leavevmode
        \Qcircuit { & \gate{X} & \qw };\quad
        \Qcircuit { & \gate{Z} & \qw };\quad
        \Qcircuit { & \gate{X} & \gate{Z} & \qw }
    \end{center}
    \item The \emph{phase gate} is
    \[ P_\theta = \begin{pmatrix}
        1 & 0 \\
        0 & e^{i\theta}
    \end{pmatrix} \]
    Note that \( Z = P_\pi \).
    \item The \emph{controlled-X} gate, also called a \emph{CNOT} gate, is
    \[ CX = \begin{pmatrix}
        1 & 0 & 0 & 0 \\
        0 & 1 & 0 & 0 \\
        0 & 0 & 0 & 1 \\
        0 & 0 & 1 & 0
    \end{pmatrix} = \begin{pmatrix}
        I & 0 \\
        0 & X
    \end{pmatrix} \]
    Note that \( CX \ket{i} \ket{j} = \ket{i}\ket{i \oplus j} \).
    The first qubit is called the \emph{control} qubit, and the second is called the \emph{target} qubit.
    If \( i = 0 \), there is no action on the second qubit.
    If \( i = 1 \), \( X \) is performed on the second qubit.
    In general, \( CX \ket{0}\ket{\psi} = \ket{0}\ket{\psi} \), and \( CX \ket{1}\ket{\psi} = \ket{1}(X \ket{\psi}) \).
    The circuit diagram is as follows.
    \begin{center}
        \leavevmode
        \Qcircuit {
            \lstick{\ket{i}} & \ctrl{1} & \rstick{\ket{i}} \qw \\
            \lstick{\ket{j}} & \targ{} & \rstick{\ket{i \oplus j}} \qw
        }
    \end{center}
    One can show that
    \begin{center}
        \leavevmode
        \Qcircuit {
            & \targ{} & \qw \\
            & \ctrl{-1} & \qw
        }
        \quad = \quad
        \Qcircuit {
            & \gate{H} & \ctrl{1} & \gate{H} & \qw \\
            & \gate{H} & \targ{} & \gate{H} & \qw
        }
    \end{center}
    \item The \emph{controlled-Z} gate, also called a \emph{CZ} gate, is
    \[ CZ = \begin{pmatrix}
        I & 0 \\
        0 & Z
    \end{pmatrix} \]
    So \( CZ \ket{0}\ket{\psi} = \ket{0}\ket{\psi} \) and \( CZ \ket{1}\ket{\psi} = \ket{1}(Z\ket{\psi}) \).
    \( CZ \) is symmetric in its action on the two qubits; for example, \( CZ_{12} \ket{0} \ket{1} = CZ_{21} \ket{0} \ket{1} \).
    This gate is drawn
    \begin{center}
        \leavevmode
        \Qcircuit {
            & \ctrl{1} & \qw \\
            & \gate{Z} & \qw
        }
        \quad or \quad
        \Qcircuit {
            & \ctrl{1} & \qw \\
            & \control{} \qw & \qw
        }
    \end{center}
\end{enumerate}

\subsection{Quantum teleportation}
Suppose Alice and Bob share the Bell state \( \ket{\phi^+}_{AB} \), and that Alice wants to send the state of qubit \( \ket{\psi}_C \) to Bob, but only classical communication between them is possible.
It is possible to transfer the information about the state of \( \ket{\psi}_C \) without physically transferring qubit \( C \) to Bob.
This state transfer can be accomplished in such a way that is unaffected by any physical process in the space between Alice and Bob, since it relies only on classical communication.

The initial state of \( C A B \) is \( \ket{\Psi} = \ket{\psi}_C \otimes \ket{\phi^+}_{AB} \), assuming \( \ket{\psi}_C \) is uncorrelated with \( \ket{\phi^+}_{AB} \).
Let \( \ket{\psi}_C = a \ket{0}_C + b \ket{1}_C \), so
\[ \ket{\Psi} = \ket{\psi}_C \otimes \ket{\phi^+}_{AB} = \frac{1}{\sqrt{2}} \qty[a \ket{000} + a \ket{011} + b \ket{100} + b \ket{111}] \]
Alice sends \( C \) and \( A \) through a \( CX \) gate.
Now,
\[ \ket{\Psi} = \ket{\varphi_1} = \frac{1}{\sqrt{2}} \qty[a \ket{000} + a \ket{011} + b \ket{110} + b \ket{101}] \]
She now sends \( C \) through a Hadamard gate.
\begin{align*}
    \ket{\Psi} = \ket{\varphi_2} &= \frac{1}{\sqrt{2}} \qty[a \ket{+00} + a \ket{+11} + b \ket{-10} + b \ket{-01}] \\
    &= \frac{1}{2} \big[\ket{00} \ket{\psi} + \ket{01} (X \ket{\psi}) + \ket{10} (Z \ket{\psi}) + \ket{11} (-Y\ket{\psi})\big]
\end{align*}
Alice now measures \( CA \) in the computational basis of \( \mathbb C^2 \otimes \mathcal C^2 \).
The probability of each outcome is \( \frac{1}{4} \), irrespective of the values of \( a \) and \( b \) and hence of \( \ket{\psi} \).
She then sends the result of her measurement to Bob.
If Alice measures outcome \( ij \), \( B \) is in state \( X^j Z^i \ket{\psi} \).
Then, Bob can act on \( B \) using \( Z^i X^j \), as \( X \) and \( Z \) are involutive, giving \( \ket{\psi} \) as desired.
This process can be represented with the following diagram, where double-struck wires are classical, and the meter symbol denotes a measurement of the quantum state.
\begin{center}
    \leavevmode
    \Qcircuit {
        \lstick{\ket{\psi}_C} & \ctrl{1} & \gate{H} & \meter{} & \cw & \control \cw \cwx[2] \\
        \lstick{\ket{\phi^+}_A} & \targ{} & \qw & \meter{} & \control \cw \cwx[1] & \\
        \lstick{\ket{\phi^+}_B} & \qw & \qw & \qw & \gate{X} & \gate{Z} & \rstick{\ket{\psi}} \qw
    }
\end{center}
Note that after the measurement of \( CA \), the entanglement between \( CA \) and \( B \) is broken.
No-cloning is not violated, as the original state \( \ket{\psi}_C \) is destroyed.

Note that the first steps of this process including Alice's measurement correspond to performing a Bell measurement on \( CA \).
This is because the action of \( CX_{CA} \) then \( H_C \) corresponds to a rotation of the Bell basis to the standard basis.
