\subsection{Matching in bipartite graphs}
\begin{definition}
	Let \( G = (X \sqcup Y, E) \) be a bipartite graph.
	A \emph{matching from \( X \) to \( Y \)} is a set of edges \( E' \subseteq \qty{xy_x \mid x \in X, y_x \in Y} = E \) such that the map \( x \mapsto y_x \) is injective.
\end{definition}
\begin{definition}
	Let \( G \) be a graph, \( A \subseteq V(G) \).
	We define \( N_G(A) = \qty{\bigcup_{x \in A} N(x)} \).
\end{definition}
\begin{theorem}[Hall]
	Let \( G = (X \sqcup Y, E) \) be a bipartite graph.
	There exists a matching from \( X \) to \( Y \) if and only if \emph{Hall's criterion} holds: that \( \abs{A} \leq \abs{N(A)} \) for all \( A \subseteq X \).
\end{theorem}
\begin{proof}
	The forward direction is simple, by considering the image of the injective map \( x \mapsto y_x : A \to N(A) \) for each subset \( A \subseteq X \).
	Conversely, suppose Hall's criterion is satisfied.
	We apply induction on \( \abs{X} \).
	If \( \abs{X} = 1 \), \( N(X) \) is nonempty and so the proof is complete.

	If there does not exist \( \varnothing \neq A \subsetneq X \) such that \( \abs{N(A)} = \abs{A} \), we have \( \abs{A} < \abs{N(A)} \) for all \( \varnothing \neq A \neq X \).
	Let \( xy \in E \), and let \( G' = G[X \setminus \qty{x} \sqcup Y \setminus \qty{y}] \).
	By induction, it suffices to show Hall's criterion holds for \( G' \).
	If \( B \subseteq X \setminus \qty{x} \), we have
	\[ \abs{N_{G'}(B)} \geq \abs{N_G(B)} - 1 \geq \abs{B} \]
	as required.

	However, suppose there exists such a set \( \varnothing A \subsetneq X \) with \( \abs{A} = \abs{N(A)} \).
	Let \( G_1 = G[A \sqcup N(A)] \) and \( G_2 = G[X \setminus A \sqcup Y \setminus N(A)] \).
	\( G_1 \) satisfies Hall's criterion.
	Indeed, for \( B \subseteq A \), \( N_{G_1}(B) = N_G(B) \) as required.
	\( G_2 \) also satisfies Hall's criterion.
	Suppose \( B \subseteq X \setminus A \), and consider \( N_G(A \cup B) \).
	We have
	\[ \abs{A} + \abs{B} \leq \abs{N_G(A \cup B)} = \abs{N_G(A)} + \abs{N_{G_2}(B)} \implies \abs{B} \leq \abs{N_{G_2}(B)} \]
	Hence Hall's criterion is satisfied.

	Then by induction on \( G_1 \) and \( G_2 \), the proof is complete.
\end{proof}
\begin{definition}
	A \emph{matching of deficiency \( d \) from \( X \) to \( Y \)} is a matching from \( X' \subseteq X \) to \( Y \) where \( \abs{X'} + d = \abs{X} \).
\end{definition}
\begin{theorem}[defect Hall]
	Let \( G = (X \sqcup Y, E) \) be a bipartite graph.
	\( G \) contains a matching of deficiency \( d \leq \abs{X} \) if and only if \( \abs{A} \leq \abs{N(A)} + d \) for all \( A \subseteq X \).
\end{theorem}
\begin{proof}
	The forward direction is again a simple proof.
	Let \( G = (X \sqcup Y, E) \) be a graph such that \( \abs{A} \leq \abs{N(A)} + d \) for all \( A \subseteq X \).
	Let \( G' = (X \sqcup (Y \cup \qty{z_1, \dots, z_d}), E \cup E') \) where \( E' = \qty{xz_i \mid x \in X, i \in \qty{1, \dots, d}} \).
	Hall's criterion on \( G' \) is satisfied, so there exists a matching.
	Deleting these new vertices \( \qty{z_1, \dots, z_d} \) and the edge set \( E' \), we construct a matching from \( X \) to \( Y \) of deficiency at most \( d \).
	To construct a matching of deficiency precisely \( d \), we can delete extra edges as required.
\end{proof}
\begin{definition}
	The \emph{maximum degree} \( \Delta(G) \) (resp.\ \emph{minimum degree} \( \delta(G) \)) of a graph \( G \) is the maximum (resp.\ minimum) degree of a vertex in \( G \).
\end{definition}
\begin{definition}
	A graph is \emph{regular} if all vertices have the same degree, or equivalently, \( \delta(G) = \Delta(G) \).
	A graph is \emph{\( k \)-regular} if \( \delta(G) = \Delta(G) = k \).
\end{definition}
\begin{corollary}
	Let \( G = (X \sqcup Y, E) \) be a \( k \)-regular bipartite graph and \( k \geq 1 \).
	Then there exists a matching from \( X \) to \( Y \).
\end{corollary}
\begin{proof}
	It suffices to show Hall's criterion holds.
	Let \( A \subseteq X \).
	Then
	\[ e(G[A \cup N(A)]) = \sum_{x \in A} \deg x = k \abs{A};\quad e(G[A \cup N(A)]) = \sum_{x \in N(A)} \deg v \leq k \abs{N(A)} \]
	Hence \( \abs{A} \leq \abs{N(A)} \).
\end{proof}
\begin{example}
	Let \( \Gamma \) be a finite group, and let \( H \leq \Gamma \).
	Let \( L_1, \dots, L_n \) be the left cosets, and \( R_1, \dots, R_n \) be the right cosets.
	We want to find \( g_1, \dots, g_n \) such that \( g_1 H, \dots, g_n H \) are the left cosets and \( H g_1, \dots, H g_n \) are the right cosets.

	Consider the graph \( G = (\qty{L_1, \dots, L_n} \sqcup \qty{R_1, \dots, R_n}, E) \) where an edge lies between \( L_i \) and \( R_j \) if \( L_i \cap R_j \neq \varnothing \).
	It suffices to find a matching in this graph, because then each edge in the matching implies the existence of a representative for both cosets.
	Let \( A \subseteq \qty{L_1, \dots, L_n} \), so \( A = \qty{L_{i_1}, \dots, L_{i,k}} \).
	Consider \( \abs{\bigcup_{j=1}^k L_{ij}} = k \abs{H} \), but since \( R_1, \dots, R_n \) partition \( \Gamma \) and have size \( \abs{H} \), at least \( k \) right cosets of \( H \) must intersect \( \bigcup R_{ij} \).
	Hence Hall's criterion is satisfied.
\end{example}

\subsection{Connectivity}
Let \( S \subseteq V(G) \).
Then we define \( G - S = G[V(G) \setminus S] \).
\begin{definition}
	Let \( G \) be a graph, and \( \abs{G} \geq 1 \).
	Then we define the \emph{connectivity parameter} \( \kappa \) of \( G \) is
	\[ \kappa = \min{\abs{S} \mid S \subseteq V(G), G - S \text{ is disconnected or a single vertex}} \]
	We say that \( G \) is \( k \)-connected if \( k \leq \kappa \).
	Hence \( G \) is \( k \)-connected if and only if for all sets \( S \) of at most \( k-1 \) vertices, \( G - S \) is connected and not a single vertex.
\end{definition}
\begin{example}
	\( \kappa(\text{Petersen graph}) = 3 \), because deleting any two vertices leaves the graph connected, but deleting the neighbourhood of any vertex disconnects the graph.
	\( \kappa(G) = 1 \) if \( G \) is a tree.
	\( \kappa(C_n) = 2 \) for \( n \geq 3 \).
	\( \kappa(K_n) = n - 1 \).
\end{example}
\begin{definition}
	Let \( G \) be a graph, and \( a, b \in V(G) \).
	We say that the \( a \)--\( b \) paths \( P_1, \dots, P_k \) are \emph{disjoint} if \( P_i \cap P_j = \qty{a, b} \) for \( i \neq j \).
\end{definition}
Note that \( \delta(G) \geq \kappa(G) \).
This follows because removing the neighbours of the vertex of minimum degree disconnects the graph or leaves it a single vertex.
Also, we can easily see that \( \kappa(G - x) \geq \kappa(G) - 1 \).
Note that we can have \( \kappa(G-x) > \kappa(G) \) by considering a \( 2 \)-connected graph with an additional leaf.
\begin{definition}
	Let \( G \) be a graph and \( a \neq b \in V(G) \), where \( a \not\sim b \).
	We say that \( S \subseteq V(G) \setminus \qty{a,b} \) is a \emph{\( a \)--\( b \) separator} if \( G - S \) disconnects \( a \) and \( b \).
\end{definition}
\begin{theorem}[Menger, form 1]
	Let \( G \) be a graph and \( a \neq b \in V(G) \), where \( a \not\sim b \).
	The minimum size of an \( a \)--\( b \) separator is the maximum number of disjoint paths from \( a \) to \( b \).
	Equivalently, if all \( a \)--\( b \) separators have size at least \( k \), then there exists \( P_1, \dots, P_k \) disjoint \( a \)--\( b \) paths.
\end{theorem}
